% !TEX TS-program = pdflatex
\documentclass[notoc,notitlepage]{tufte-book}
% \nonstopmode % uncomment to enable nonstopmode

\usepackage{classnotetitle}

\title{PMATH451 --- Measure and Integration}
\author{Johnson Ng}
\subtitle{Class notes for Fall 2019}
\credentials{BMath (Hons), Pure Mathematics major, Actuarial Science Minor}
\institution{University of Waterloo}

\setcounter{secnumdepth}{3}
\setcounter{tocdepth}{3}

\renewcommand{\baselinestretch}{1.2}
\usepackage{geometry}
\geometry{letterpaper}
\usepackage[parfill]{parskip}
\usepackage{graphicx}

% Essential Packages
\usepackage{makeidx}
\makeindex
\usepackage{enumitem}
\usepackage[T1]{fontenc}
\usepackage{natbib}
\bibliographystyle{apalike}
\usepackage{ragged2e}
\usepackage{etoolbox}
\usepackage{amssymb}
\usepackage{eso-pic}
\usepackage[fixed]{fontawesome5}
\usepackage{todonotes}
\usepackage{apptools, chngcntr}
\usepackage{amsmath}
\usepackage{mathrsfs}
\usepackage{stmaryrd}
\usepackage{mathtools}
\usepackage{tocloft}
\usepackage{tocbibind}
\usepackage{xparse}
\usepackage{tkz-euclide}
\usetkzobj{all}
\usepackage[utf8]{inputenc}
\usepackage{csquotes}
\usepackage[english]{babel}
\usepackage{marvosym}
\usepackage{pgf,tikz}
\usepackage{tikz-cd}
\usepackage{ifthen}
\usepackage{pgfplots}
\usepackage{fancyhdr}
\usepackage{array}
\usepackage{float}
\usepackage{xcolor}
\usepackage{soul}
\usepackage{centernot}
\usepackage{silence}
  \WarningFilter*{latex}{Marginpar on page \thepage\space moved}
\usepackage{tcolorbox}
\tcbuselibrary{skins,breakable}
\usepackage{longtable,booktabs}
\usepackage[amsmath,hyperref,thmmarks]{ntheorem}
\usepackage{thmtools}
\usepackage{hyperref}
\usepackage[noabbrev,capitalize,nameinlink]{cleveref}

\newcommand{\personalcolor}{false}
\ifthenelse{\equal{\personalcolor}{true}}{
  \usepackage{colorscheme-chaos}
}{
  \usepackage{colorscheme-student}
}

% hyperref Package Settings
\hypersetup{
    unicode=true,          % non-Latin characters in Acrobat’s bookmarks
    pdftoolbar=false,        % show Acrobat’s toolbar?
    pdfmenubar=false,        % show Acrobat’s menu?
    pdffitwindow=true,     % window fit to page when opened
    colorlinks=true,
    allcolors=magenta,
}

% tikz
\usepgfplotslibrary{polar}
\usepgflibrary{shapes.geometric}
\usetikzlibrary{angles,patterns,calc,decorations.markings,arrows.meta,tikzmark,bending}
\tikzset{midarrow/.style 2 args={
        decoration={markings,
            mark= at position #2 with {\arrow{#1}} ,
        },
        postaction={decorate}
    },
    midarrow/.default={latex}{0.5}
}
\def\centerarc[#1](#2)(#3:#4:#5)% Syntax: [draw options] (center) (initial angle:final angle:radius)
    { \draw[#1] ($(#2)+({#5*cos(#3)},{#5*sin(#3)})$) arc (#3:#4:#5); }
% from https://tex.stackexchange.com/questions/67573/tikz-shift-and-rotate-in-3d
\newcommand{\rotateRPY}[4][0/0/0]% point to be saved to \savedxyz, roll, pitch, yaw
{   \pgfmathsetmacro{\rollangle}{#2}
    \pgfmathsetmacro{\pitchangle}{#3}
    \pgfmathsetmacro{\yawangle}{#4}

    % to what vector is the x unit vector transformed, and which 2D vector is this?
    \pgfmathsetmacro{\newxx}{cos(\yawangle)*cos(\pitchangle)}% a
    \pgfmathsetmacro{\newxy}{sin(\yawangle)*cos(\pitchangle)}% d
    \pgfmathsetmacro{\newxz}{-sin(\pitchangle)}% g
    \path (\newxx,\newxy,\newxz);
    \pgfgetlastxy{\nxx}{\nxy};

    % to what vector is the y unit vector transformed, and which 2D vector is this?
    \pgfmathsetmacro{\newyx}{cos(\yawangle)*sin(\pitchangle)*sin(\rollangle)-sin(\yawangle)*cos(\rollangle)}% b
    \pgfmathsetmacro{\newyy}{sin(\yawangle)*sin(\pitchangle)*sin(\rollangle)+ cos(\yawangle)*cos(\rollangle)}% e
    \pgfmathsetmacro{\newyz}{cos(\pitchangle)*sin(\rollangle)}% h
    \path (\newyx,\newyy,\newyz);
    \pgfgetlastxy{\nyx}{\nyy};

    % to what vector is the z unit vector transformed, and which 2D vector is this?
    \pgfmathsetmacro{\newzx}{cos(\yawangle)*sin(\pitchangle)*cos(\rollangle)+ sin(\yawangle)*sin(\rollangle)}
    \pgfmathsetmacro{\newzy}{sin(\yawangle)*sin(\pitchangle)*cos(\rollangle)-cos(\yawangle)*sin(\rollangle)}
    \pgfmathsetmacro{\newzz}{cos(\pitchangle)*cos(\rollangle)}
    \path (\newzx,\newzy,\newzz);
    \pgfgetlastxy{\nzx}{\nzy};

    % transform the point given by #1
    \foreach \x/\y/\z in {#1}
    {   \pgfmathsetmacro{\transformedx}{\x*\newxx+\y*\newyx+\z*\newzx}
        \pgfmathsetmacro{\transformedy}{\x*\newxy+\y*\newyy+\z*\newzy}
        \pgfmathsetmacro{\transformedz}{\x*\newxz+\y*\newyz+\z*\newzz}
        \xdef\savedx{\transformedx}
        \xdef\savedy{\transformedy}
        \xdef\savedz{\transformedz}     
    }
}
\tikzset{RPY/.style={x={(\nxx,\nxy)},y={(\nyx,\nyy)},z={(\nzx,\nzy)}}}
\newcommand{\AxisRotator}[1][rotate=0]{%
    \tikz [x=0.25cm,y=0.60cm,line width=.2ex,-stealth,#1] \draw (0,0) arc (-150:150:1 and 1);%
  }

% enumitems
\newlist{inlinelist}{enumerate*}{1}
\setlist*[inlinelist,1]{%
  label=(\roman*),
}

% Theorem Style Customization
\setlength\theorempreskipamount{2ex}
\setlength\theorempostskipamount{3ex}

\makeatletter
\let\nobreakitem\item
\let\@nobreakitem\@item
\patchcmd{\nobreakitem}{\@item}{\@nobreakitem}{}{}
\patchcmd{\nobreakitem}{\@item}{\@nobreakitem}{}{}
\patchcmd{\@nobreakitem}{\@itempenalty}{\@M}{}{}
\patchcmd{\@xthm}{\ignorespaces}{\nobreak\ignorespaces}{}{}
\patchcmd{\@ythm}{\ignorespaces}{\nobreak\ignorespaces}{}{}

\renewtheoremstyle{break}%
  {\item{\theorem@headerfont
          ##1\ ##2\theorem@separator}\hskip\labelsep\relax\nobreakitem}%
  {\item{\theorem@headerfont
          ##1\ ##2\ (##3)\theorem@separator}\hskip\labelsep\relax\nobreakitem}
\makeatother

% ntheorem Declarations
\theorempreskip{10pt}
\theorempostskip{5pt}
\theoremstyle{break}

\theoremsymbol{\faComment}
\newtheorem{remark}{Remark}[section]
\theoremsymbol{}
\newtheorem*{strategy}{\faPaperPlane Strategy}
\newtheorem*{procedure}{\faCodeBranch\ }
\newtheorem{ex}{Exercise}[section]
\theorembodyfont{\normalfont}
\newtheorem*{solution}{\faPencil* Solution}
\theoremsymbol{\faGavel}
\newtheorem{eg}{Example}[section]
\theoremsymbol{}
\theorembodyfont{\it}

    % definition env
\theoremprework{\textcolor{blue}{\hrule height 2pt width \textwidth}}
\theoremheaderfont{\color{blue}\normalfont\bfseries}
\theorempostwork{\textcolor{blue}{\hrule height 2pt width \textwidth}}
\theoremindent10pt
\newtheorem{defn}{\faBook Definition}

    % definition env no num
\theoremprework{\textcolor{blue}{\hrule height 2pt width \textwidth}}
\theoremheaderfont{\color{blue}\normalfont\bfseries}
\theorempostwork{\textcolor{blue}{\hrule height 2pt width \textwidth}}
\theoremindent10pt
\newtheorem*{defnnonum}{\faBook Definition}

\theoremprework{\textcolor{blue}{\hrule height 2pt width \marginparwidth}}
\theoremheaderfont{\color{blue}\normalfont\bfseries}
\theorempostwork{\textcolor{blue}{\hrule height 2pt width \marginparwidth}}
\theoremindent10pt
\newtheorem{margindefn}[defn]{\faBook Definition}

\theoremprework{\textcolor{blue}{\hrule height 2pt width \textwidth}}
\theoremheaderfont{\color{blue}\normalfont\bfseries}
\theorempostwork{\textcolor{blue}{\hrule height 2pt width \textwidth}}
\theoremindent10pt
\newtheorem*{margindefnnonum}{\faBook Definition}

    % theorem envs
\theoremprework{\textcolor{magenta}{\hrule height 2pt width \textwidth}}
\theoremheaderfont{\color{magenta}\normalfont\bfseries}
\theorempostwork{\textcolor{magenta}{\hrule height 2pt width \textwidth}}
\theoremindent10pt
\newtheorem{thm}{\faCoffee Theorem}

\theoremprework{\textcolor{magenta}{\hrule height 2pt width \textwidth}}
\theorempostwork{\textcolor{magenta}{\hrule height 2pt width \textwidth}}
\theoremindent10pt
\newtheorem{propo}[thm]{\faTint Proposition}

\theoremprework{\textcolor{magenta}{\hrule height 2pt width \textwidth}}
\theorempostwork{\textcolor{magenta}{\hrule height 2pt width \textwidth}}
\theoremindent10pt
\newtheorem{crly}[thm]{\faSpaceShuttle Corollary}

\theoremprework{\textcolor{magenta}{\hrule height 2pt width \textwidth}}
\theorempostwork{\textcolor{magenta}{\hrule height 2pt width \textwidth}}
\theoremindent10pt
\newtheorem{lemma}[thm]{\faTree Lemma}

\theoremprework{\textcolor{magenta}{\hrule height 2pt width \textwidth}}
\theorempostwork{\textcolor{magenta}{\hrule height 2pt width \textwidth}}
\theoremindent10pt
\newtheorem{axiom}[thm]{\faShield Axiom}

    % theorem envs without counter
\theoremprework{\textcolor{magenta}{\hrule height 2pt width \textwidth}}
\theoremheaderfont{\color{magenta}\normalfont\bfseries}
\theorempostwork{\textcolor{magenta}{\hrule height 2pt width \textwidth}}
\theoremindent10pt
\newtheorem*{thmnonum}{\faCoffee Theorem}

\theoremprework{\textcolor{magenta}{\hrule height 2pt width \textwidth}}
\theorempostwork{\textcolor{magenta}{\hrule height 2pt width \textwidth}}
\theoremindent10pt
\newtheorem*{propononum}{\faTint Proposition}

\theoremprework{\textcolor{magenta}{\hrule height 2pt width \textwidth}}
\theorempostwork{\textcolor{magenta}{\hrule height 2pt width \textwidth}}
\theoremindent10pt
\newtheorem*{crlynonum}{\faSpaceShuttle Corollary}

\theoremprework{\textcolor{magenta}{\hrule height 2pt width \textwidth}}
\theorempostwork{\textcolor{magenta}{\hrule height 2pt width \textwidth}}
\theoremindent10pt
\newtheorem*{lemmanonum}{\faTree Lemma}

\theoremprework{\textcolor{magenta}{\hrule height 2pt width \textwidth}}
\theorempostwork{\textcolor{magenta}{\hrule height 2pt width \textwidth}}
\theoremindent10pt
\newtheorem*{axiomnonum}{\faShield Axiom}

    % envs on margins
\theoremprework{\textcolor{magenta}{\hrule height 2pt width \marginparwidth}}
\theoremheaderfont{\color{magenta}\normalfont\bfseries}
\theorempostwork{\textcolor{magenta}{\hrule height 2pt width \marginparwidth}}
\theoremindent10pt
\newtheorem{marginthm}[thm]{\faCoffee Theorem}

\theoremprework{\textcolor{magenta}{\hrule height 2pt width \marginparwidth}}
\theorempostwork{\textcolor{magenta}{\hrule height 2pt width \marginparwidth}}
\theoremindent10pt
\newtheorem{marginpropo}[thm]{\faTint Proposition}

\theoremprework{\textcolor{magenta}{\hrule height 2pt width \marginparwidth}}
\theorempostwork{\textcolor{magenta}{\hrule height 2pt width \marginparwidth}}
\theoremindent10pt
\newtheorem{margincrly}[thm]{\faSpaceShuttle Corollary}

\theoremprework{\textcolor{magenta}{\hrule height 2pt width \marginparwidth}}
\theorempostwork{\textcolor{magenta}{\hrule height 2pt width \marginparwidth}}
\theoremindent10pt
\newtheorem{marginlemma}[thm]{\faTree Lemma}

\theoremprework{\textcolor{magenta}{\hrule height 2pt width \marginparwidth}}
\theorempostwork{\textcolor{magenta}{\hrule height 2pt width \marginparwidth}}
\theoremindent10pt
\newtheorem{marginaxiom}[thm]{\faShield Axiom}

    % envs on margins without counter
\theoremprework{\textcolor{magenta}{\hrule height 2pt width \marginparwidth}}
\theoremheaderfont{\color{magenta}\normalfont\bfseries}
\theorempostwork{\textcolor{magenta}{\hrule height 2pt width \marginparwidth}}
\theoremindent10pt
\newtheorem*{marginthmnonum}{\faCoffee Theorem}

\theoremprework{\textcolor{magenta}{\hrule height 2pt width \marginparwidth}}
\theorempostwork{\textcolor{magenta}{\hrule height 2pt width \marginparwidth}}
\theoremindent10pt
\newtheorem*{marginpropononum}{\faTint Proposition}

\theoremprework{\textcolor{magenta}{\hrule height 2pt width \marginparwidth}}
\theorempostwork{\textcolor{magenta}{\hrule height 2pt width \marginparwidth}}
\theoremindent10pt
\newtheorem*{margincrlynonum}{\faSpaceShuttle Corollary}

\theoremprework{\textcolor{magenta}{\hrule height 2pt width \marginparwidth}}
\theorempostwork{\textcolor{magenta}{\hrule height 2pt width \marginparwidth}}
\theoremindent10pt
\newtheorem*{marginlemmanonum}{\faTree Lemma}

\theoremprework{\textcolor{magenta}{\hrule height 2pt width \marginparwidth}}
\theorempostwork{\textcolor{magenta}{\hrule height 2pt width \marginparwidth}}
\theoremindent10pt
\newtheorem*{marginaxiomnonum}{\faShield Axiom}

    % proof env
\theoremprework{\textcolor{green}{\hrule height 2pt width \textwidth}}
\theorembodyfont{\normalfont}
\theoremheaderfont{\color{green}\normalfont\bfseries}
\theorempostwork{\textcolor{green}{\hrule height 2pt width \textwidth}}
\theoremsymbol{\ensuremath{_\square}}
\newtheorem*{proof}{\faPencil* Proof}
\theoremsymbol{}

\theoremprework{\textcolor{green}{\hrule height 2pt width \marginparwidth}}
\theorembodyfont{\normalfont}
\theoremheaderfont{\color{green}\normalfont\bfseries}
\theorempostwork{\textcolor{green}{\hrule height 2pt width \marginparwidth}}
\theoremsymbol{\ensuremath{_\square}}
\newtheorem*{mproof}{\faPencil* Proof}
\theoremsymbol{}

    % note and notation env
\theorembodyfont{\it}

\theoremprework{\textcolor{yellow}{\hrule height 2pt width \textwidth}}
\theoremheaderfont{\color{yellow}\normalfont\bfseries}
\theorempostwork{\textcolor{yellow}{\hrule height 2pt width \textwidth}}
\newtheorem{note}{\faQuoteLeft Note}[section]

\theoremprework{\textcolor{yellow}{\hrule height 2pt width \marginparwidth}}
\theoremheaderfont{\color{yellow}\normalfont\bfseries}
\theorempostwork{\textcolor{yellow}{\hrule height 2pt width \marginparwidth}}
\newtheorem{mnote}[note]{\faQuoteLeft Note}

\theoremprework{\textcolor{yellow}{\hrule height 2pt width \textwidth}}
\theorempostwork{\textcolor{yellow}{\hrule height 2pt width \textwidth}}
\newtheorem*{notation}{\faPaw Notation}

    % warning env
\theoremprework{\textcolor{red}{\hrule height 2pt width \textwidth}}
\theoremheaderfont{\color{red}\normalfont\bfseries}
\theorempostwork{\textcolor{red}{\hrule height 2pt width \textwidth}}
\theoremindent10pt
\newtheorem*{warning}{\faBug Warning}

\theoremprework{\textcolor{red}{\hrule height 2pt width \marginparwidth}}
\theoremheaderfont{\color{red}\normalfont\bfseries}
\theorempostwork{\textcolor{red}{\hrule height 2pt width \marginparwidth}}
\theoremindent10pt
\newtheorem*{marginwarning}{\faBug Warning}

% rule for appendices
\AtAppendix{\counterwithin{defn}{chapter}}
\AtAppendix{\counterwithin{thm}{chapter}}
\AtAppendix{\counterwithin{propo}{chapter}}
\AtAppendix{\counterwithin{lemma}{chapter}}
\AtAppendix{\counterwithin{crly}{chapter}}
\AtAppendix{\counterwithin{axiom}{chapter}}

% more environments
\newtcolorbox{quotebox}[2]{
  blanker,enhanced,breakable,standard jigsaw,
  opacityback=0,
  coltext=\ifblank{#2}{black}{#2},
  left=5mm,right=5mm,top=2mm,bottom=2mm,
  colframe=\ifblank{#1}{bblack}{#1},
  boxrule=0pt,leftrule=3pt,
  fontupper=\itshape
}

\providecommand{\parthook}{}
\patchcmd{\part}{\thispagestyle}{\parthook\thispagestyle}{}{}
\newcommand{\partimage}[2][]{% \parthook[<options>]{<image>}
  \renewcommand{\parthook}{% Update \parthook
    \AddToShipoutPictureBG*{% Add picture to background of THIS page only
      \AtPageLowerLeft{\includegraphics[width=\paperwidth,height=\paperheight,#1]{#2}}}% Insert image
    \renewcommand{\parthook}{}}}% Restore \parthook

\AtBeginDocument{\renewcommand\contentsname{\slshape Table of Contents\normalfont}}
\cftpagenumbersoff{part}

\newcommand{\tuftepart}[1]{\newgeometry{}\part{#1}\restoregeometry}

% Heading formattings
% chapter format
\titleformat{\chapter}%
  {\huge\rmfamily\itshape\color{magenta}}% format applied to label+text
  {\llap{\colorbox{magenta}{\parbox[c][1cm]{3cm}{\hfill\itshape\Huge\textcolor{background}{\thechapter}}}}}% label
  {5pt}% horizontal separation between label and title body
  {\faLeaf}% before the title body
  []% after the title body

% section format
\titleformat{\section}%
  {\normalfont\Large\rmfamily\itshape\color{blue}}% format applied to label+text
  {\llap{\colorbox{blue}{\parbox{3cm}{\hfill\itshape\textcolor{background}{\thesection}}}}}% label
  {5pt}% horizontal separation between label and title body
  {}% before the title body
  []% after the title body

% subsection format
\titleformat{\subsection}%
  {\normalfont\large\itshape\color{green}}% format applied to label+text
  {\llap{\colorbox{green}{\parbox{3cm}{\hfill\textcolor{background}{\thesubsection}}}}}% label
  {1em}% horizontal separation between label and title body
  {}% before the title body
  []% after the title body

% subsubsection format
\titleclass{\subsubsection}{straight}
\titleformat{\subsubsection}%
  {\normalfont\large\itshape\color{yellow}}% format applied to label+text
  {\llap{\colorbox{yellow}{\parbox{3cm}{\hfill\textcolor{background}{\thesubsubsection}}}}}% label
  {1em}% horizontal separation between label and title body
  {}% before the title body
  []% after the title body

% Sidenote enhancements
\def\mathmarginnote#1{%
  \tag*{\rlap{\hspace\marginparsep\smash{\parbox[t]{\marginparwidth}{%
  \footnotesize#1}}}}
}

% Custom table columning
\newcolumntype{L}[1]{>{\raggedright\let\newline\\\arraybackslash\hspace{0pt}}m{#1}}
\newcolumntype{C}[1]{>{\centering\let\newline\\\arraybackslash\hspace{0pt}}m{#1}}
\newcolumntype{R}[1]{>{\raggedleft\let\newline\\\arraybackslash\hspace{0pt}}m{#1}}

% Graph styles
\pgfplotsset{compat=1.15}
\usepgfplotslibrary{fillbetween}
\pgfplotsset{four quads/.append style={axis x line=middle, axis y line=
middle, xlabel={$x$}, ylabel={$y$}, axis equal }}
\pgfplotsset{four quad complex/.append style={axis x line=middle, axis y line=
middle, xlabel={$\re$}, ylabel={$\im$}, axis equal }}
\def\axisdefaultwidth{360pt}
\pgfplotsset{
  tufteaxis/.append style = {thick},tick style = {thick,black},
  %
  % #1 = x, y, or z
  % #2 = the shift value
  /tikz/normal shift/.code 2 args = {%
    \pgftransformshift{%
        \pgfpointscale{#2}{\pgfplotspointouternormalvectorofticklabelaxis{#1}}%
    }%
  },%
  %
  range3frame/.style = {
    tick align        = outside,
    scaled ticks      = false,
    enlargelimits     = false,
    ticklabel shift   = {10pt},
    axis lines*       = left,
    line cap          = round,
    clip              = false,
    xtick style       = {normal shift={x}{10pt}},
    ytick style       = {normal shift={y}{10pt}},
    ztick style       = {normal shift={z}{10pt}},
    x axis line style = {normal shift={x}{10pt}},
    y axis line style = {normal shift={y}{10pt}},
    z axis line style = {normal shift={z}{10pt}},
  }
}

% Shortcuts
\DeclareMathOperator{\id}{id}
\DeclareMathOperator{\Img}{Img}
\DeclareMathOperator{\Res}{Res}
\DeclareMathOperator*{\argmax}{arg\,max}
\DeclareMathOperator*{\argmin}{arg\,min}
\DeclareMathOperator{\re}{Re}
\DeclareMathOperator{\im}{Im}
\DeclareMathOperator{\caparg}{Arg}
\DeclareMathOperator{\Char}{Char}
\DeclareMathOperator{\sgn}{sgn}
\DeclareMathOperator{\Range}{range}

\newcommand{\floor}[1]{\lfloor #1 \rfloor}      % simplifying the writing of a floor function
\newcommand{\ceiling}[1]{\lceil #1 \rceil}      % simplifying the writing of a ceiling function
\newcommand{\dotp}{\, \cdotp}			        % dot product to distinguish from \cdot
\newcommand{\abs}[1]{\left|#1\right|}						% absolute value
\newcommand{\lra}[1]{\left\langle \; #1 \; \right\rangle}
\newcommand{\at}[2]{\Big|_{#1}^{#2}}
\newcommand{\Arg}[1]{\caparg #1}
\renewcommand{\bar}[1]{\mkern 1.5mu \overline{\mkern -1.5mu #1 \mkern -1.5mu} \mkern 1.5mu}
\newcommand{\faktor}[2]{{\raisebox{.2em}{$#1$}\left/\raisebox{-.2em}{$#2$}\right.}}
\newcommand{\quotient}[2]{\faktor{#1}{#2}}
\newcommand{\cyclic}[1]{\left\langle #1 \right\rangle}
\newcommand{\ind}[2]{\Ind_{#2}\left( #1 \right)}
\newcommand{\notimply}{\centernot\implies}
\newcommand{\res}[2]{\underset{#2}{\Res} #1 }
\newcommand{\tworow}[3]{\begin{tabular}{@{}#1@{}} #2 \\ #3 \end{tabular}}
\renewcommand{\epsilon}{\varepsilon}
\renewcommand{\phi}{\varphi}
\newcommand{\lrarrow}{\leftrightarrow}
\newcommand{\larrow}{\leftarrow}
\newcommand{\rarrow}{\rightarrow}
\renewcommand{\atop}[2]{\genfrac{}{}{0pt}{}{#1}{#2}}
\newcommand*\dif{\mathop{}\!d}
\newcommand{\mmid}{\; \middle| \;}
\newcommand{\coprime}{\; \bot \;}
\newcommand{\norm}[1]{\left\| #1 \right\|}
\newenvironment{spmatrix}
  {\left(\begin{smallmatrix}}
  {\end{smallmatrix}\right)}

  % inspiration from: https://tex.stackexchange.com/questions/8720/overbrace-underbrace-but-with-an-arrow-instead#37758
\newcommand{\overarrow}[2]{
  \overset{\makebox[0pt]{\begin{tabular}{@{}c@{}}#2\\[0pt]\ensuremath{\uparrow}\end{tabular}}}{\ensuremath{#1}}
}
\newcommand{\underarrow}[2]{
  \underset{\makebox[0pt]{\begin{tabular}{@{}c@{}}\downarrow\\[0pt]\ensuremath{#2}\end{tabular}}}{\ensuremath{#1}}
}


	% highlighting shortcuts
\newcommand{\hlimpo}[1]{\textcolor{red}{\textbf{#1}}}
\newcommand{\hlwarn}[1]{\textcolor{yellow}{\textbf{#1}}}
\newcommand{\hldefn}[1]{\textcolor{blue}{\index{#1}\textbf{#1}}}
\newcommand{\hlnotea}[1]{\textcolor{green}{\textbf{#1}}}
\newcommand{\hlnoteb}[1]{\textcolor{cyan}{\textbf{#1}}}
\newcommand{\hlb}[2]{\colorbox{#1!30!background}{#2}}
\newcommand{\hlbnotea}[1]{\hlb{green}{#1}}
\newcommand{\hlbnoteb}[1]{\hlb{cyan}{#1}}
\newcommand{\hlbnotec}[1]{\hlb{yellow}{#1}}
\newcommand{\hlbnoted}[1]{\hlb{magenta}{#1}}
\newcommand{\hlbnotee}[1]{\hlb{red}{#1}}
\newcommand{\WTP}{\textcolor{bwhite}{WTP} }
\newcommand{\WTS}{\textcolor{bwhite}{WTS} }

  % stars on important stuff
\newcommand{\imponote}{\faStar}
\newcommand{\vimponote}{\faStar\faStar}
\newcommand{\vvimponote}{\faStar\faStar\faStar}

% Document header formatting
\makeatletter
\pagestyle{fancy}
\fancyhead{}
\fancyhead[RO]{\textsl{\@title} \enspace \thepage}
\fancyhead[LE]{\thepage \enspace \textsl{\leftmark \enspace \rightmark}}
\makeatother
\renewcommand{\chaptermark}[1]{\markboth{#1}{}}
\renewcommand{\sectionmark}[1]{\markright{#1}}

% Comment the two lines below if you want to print the document
\pagecolor{background}
\color{foreground}


\begin{document}
\hypersetup{pageanchor=false}
\maketitle
\hypersetup{pageanchor=true}
\begin{fullwidth}
\tableofcontents
\end{fullwidth}

\newpage
\begin{fullwidth}
  \renewcommand{\listtheoremname}{\faBook\ \slshape List of Definitions}
  \listoftheorems[ignoreall,show={defn}]
\end{fullwidth}

\newpage 
\begin{fullwidth}
  \renewcommand{\listtheoremname}{\faCoffee\ \slshape List of Theorems}
  \listoftheorems[ignoreall,
    show={axiom,lemma,thm,crly,propo,marginthm,marginpropo,marginlemma,marginaxiom,margincrly}
  ]
\end{fullwidth}

\newpage
\begin{fullwidth}
  \renewcommand{\listtheoremname}{\faCodeBranch\ \slshape List of Procedures}
  \listoftheorems[ignoreall, show={procedure}]
\end{fullwidth}


\chapter*{Preface}%
\label{chp:preface}
\addcontentsline{toc}{chapter}{Preface}
% chapter preface



% chapter preface (end)

\chapter{Lecture 1 Sep 04th, 2019}%
\label{chp:lecture_1_sep_04th_2019}
% chapter lecture_1_sep_04th_2019

\section{Motivation for the Study of Measures}%
\label{sec:motivation_for_the_study_of_measures}
% section motivation_for_the_study_of_measures

Recall Riemann integration.

\begin{defnnonum}[Riemann Integration]\index{Riemann Integration}\label{defn:riemann_integration}
  Let $f : [a, b] \to \mathbb{R}$ be a \hlnotec{bounded} function.
  We call
  \begin{equation*}
    P = \{ a = x_0 < x_1 < \hdots < x_n = b \} \subseteq [a, b]
  \end{equation*}
  a \hldefn{partition} of $[a, b]$,
  and
  \begin{equation*}
    \Delta x_i = x_i - x_{i-1}
  \end{equation*}
  as the \hlnoteb{length of the $i$\textsuperscript{th} interval}
  for $i = 1, \ldots, n$.
  \begin{marginfigure}
    \centering
    \begin{tikzpicture}
      \draw[->] (-0.5, 0) -- (3, 0);
      \draw[->] (0, -0.5) -- (0, 3);
      \draw[domain=0.5:2.2,range=0:3] plot ({\x}, {-(\x-1)*(\x-1) + 2});
      \node[label={270:{$x_{i-1}$}}] at (0.5, 0) {|};
      \node[label={270:{$x_{i}$}}] at (2.2, 0) {|};
      \draw[dashed] (1, 2) -- (-0.5, 2) node[left] {$M_i$};
      \draw[dashed] (2.2, 0.56) -- (-0.5, 0.56) node[left] {$m_i$};
    \end{tikzpicture}
    \caption{Idea of Riemann integration}\label{fig:idea_of_riemann_integration}
  \end{marginfigure}
  Let
  \begin{equation*}
    M_i = \sup \{ f(x) : x \in [x_{i-1}, x_i] \}
  \end{equation*}
  be the \hlnoteb{supremum of $f$ on the $i$\textsuperscript{th} interval},
  and 
  \begin{equation*}
    m_i = \inf \{ f(x) : x \in [x_{i-1}, x_i] \}
  \end{equation*}
  be the \hlnoteb{infimum of $f$ on the $i$\textsuperscript{th} interval}.
  We define the \hldefn{Riemann upper sum} as
  \begin{equation*}
    U(f, P) = \sum_{i} M_i \Delta x_i,
  \end{equation*}
  and the \hldefn{Riemann lower sum} as
  \begin{equation*}
    L(f, P) = \sum_{i} m_i \Delta x_i. 
  \end{equation*}
  We define the \hldefn{Riemann upper integral} as
  \begin{equation*}
    \overline{\int_{a}^{b}} f \dif{x} = \inf_{P} U(f, P)
  \end{equation*}
  and the \hldefn{Riemann lower integral} as
  \begin{equation*}
    \underline{\int_{a}^{b}} f \dif{x} = \sup L(f, P).
  \end{equation*}
  We say that $f$ is \hldefn{Riemann integrable} if
  \begin{equation*}
    \overline{\int_{a}^{b}} f \dif{x}
    = \underline{\int_{a}^{b}} f \dif{x},
  \end{equation*}
  and we write the integral of $f$ as
  \begin{equation*}
    \int_{a}^{b} f \dif{x}
    = \overline{\int_{a}^{b}} f \dif{x}
    = \underline{\int_{a}^{b}} f \dif{x}.
  \end{equation*}
\end{defnnonum}

As hyped up as one does earlier in university about Riemann integration,
there are functions that are not Riemann integrable!

\begin{eg}
  Consider a function $f : [0, 1] \to \mathbb{R}$ given by
  \begin{equation*}
    f(x) = \begin{cases}
      1 & x \in \mathbb{Q} \\
      0 & x \notin \mathbb{Q}
    \end{cases}.
  \end{equation*}
  Then
  \begin{equation*}
    \overline{\int_{a}^{b}} f \dif{x} = 1 \text{ and }
    \underline{\int_{a}^{b}} f \dif{x} = 0.
  \end{equation*}
  Thus $f$ is not Riemann integrable.
\end{eg}

\begin{note}[Shortcomings of the Riemann integral]
  \begin{enumerate}
    \item We cannot characterize functions that are Riemann integrable,
      i.e. we do not have a list of characteristics that we can check
      against to see if a function is Riemann integrable.

      This remained an open problem in the earlier 1920s.
    \item The Riemann integral behaves badly when it comes to
      pointwise limits of functions. The next example shall illustrate
      this.
    \item The Riemann integral is awkward when $f$ is unbounded.
      In particular, we used to hack our way around by looking at
      whether the Riemann integral converges to some value the function
      approaches the unbounded point, and then ``conclude''
      that the integral is the limit of that convergence.
    \item Recall that the \hlnotea{Fundamental Theorem of Calculus} states
      that
      \begin{equation*}
        \frac{\dif{}}{\dif{x}} \int_{a}^{x} f(t) \dif{t} = f(x).
      \end{equation*}
      We know that this works for Riemann integrals.
      By the first shortcoming, the problem here is that we do not
      fully know what are the functions that the Fundamental Theorem
      is true for.
    \item In PMATH450, we saw that Fourier developed the Fourier series,
      which is an extremely useful tool
      in solving \hlnotea{Differential Equations} using sines and cosines.
      However, the convergence of the Fourier series remains largely
      unexplained by Fourier, and we have but developed
      some roundabout ways of showing some convergence.
    \item Consider the set $R$ if Riemann integrable functions
      on the interval $[a, b]$.
      The set $R$ has a natural metric:
      \begin{equation*}
        d(f, g) = \int_{a}^{b} \abs{f - g} \dif{x}.
      \end{equation*}
      However, the metric space $(R, d)$ is \hlnotec{not complete}.
      This means many of our favorite results in PMATH351 are not usable!
    \item There are many functions that seem like they should have
      an integral, but turned out that they did not under Riemann integration.
  \end{enumerate}
\end{note}

\begin{eg}[Pointwise Limits of Riemann Integrable Functions is not necessarily Riemann Integrable]
  Let $\mathbb{Q} = \{ x_n \}_{n \in \mathbb{N}}$. Then
  consider a sequence of functions
  \begin{equation*}
    f_n(x) = \begin{cases}
      1 & x \in \{ x_1, \ldots, x_n \} \\
      0 & x \notin \{ x_1, \ldots, x_n \}
    \end{cases}.
  \end{equation*}
  It is rather clear that
  \begin{equation*}
    \overline{\int_{a}^{b}} f \dif{x} = \underline{\int_{a}^{b}} f \dif{x} = 0.
  \end{equation*}
  However, the pointwise limit of the $f_n$'s, and that is
  \begin{equation*}
    \lim_{n \to \infty} f_n(x) = f(x) = \begin{cases}
      1 & x \in \mathbb{Q} \\
      0 & x \notin \mathbb{Q}
    \end{cases},
  \end{equation*}
  is, as mentioned in the last example, not Riemann integrable.
\end{eg}

To address the shortcomings of the Riemann integral,
Henri Lebesgue developed the \hlnotea{Lebesgue integral},
of which we have seen in PMATH450.

Instead of dividing the $x$-axis,
Lebesgue decided to divide the $y$-axis first.

If the range of a function $f$ is $[c, d]$, where $c, d$ can be infinite,
then we partition the interval such that
\begin{equation*}
  P = \{ c = y_0 < y_1 < \hdots < y_n = d \},
\end{equation*}
and we define
\begin{equation*}
  E_i = \{ x : f(x) \in [y_{i-1}, y_i] \}.
\end{equation*}
Then if $A_i$ is the area of the ``rectangle'' for
the $i$\textsuperscript{th} interval of $[c, d]$, we have
\begin{equation*}
  y_{i-1} \cdot \ell(E_i) \leq A_i \leq y_i \cdot \ell(E_i),
\end{equation*}
where $\ell(E_i)$ is the \hlnotea{Lebesgue measure} of the set $E_i$.
Then if we let $\int_{a}^{b} f$ denote the Lebesgue integral of $f$,
we would expect
\begin{equation*}
  \sum_{i=1}^{n} y_{i-1} \cdot \ell(E_i)
  \leq \int_{a}^{b} f
  \leq \sum_{i=1}^{n} y_i \cdot \ell(E_i).
\end{equation*}
However, to truly understand what this means,
we need to understand what the Lebesgue measure is.

Furthermore, recall that in PMATH450, we saw that not all
sets, in $\mathbb{R}$ for example, are measurable,
and for `good' reasons, there always exists non-measurable sets.

% section motivation_for_the_study_of_measures (end)

\section{Algebras and \texorpdfstring{$\sigma$}{sigma}-Algebra of Sets}%
\label{sec:algebras_and_sigma_algebra_of_sets}
% section algebras_and_sigma_algebra_of_sets

\marginnote{For this course, we shall use the convention that
\begin{itemize}
  \item the `ambient' space $X$ is always non-empty;
  \item $\mathcal{P}(X)$, the power set of $X$, has non-trivial elements; and
  \item we denote $A^C = \{ x \in X : x \notin A \}$ for $A \subseteq X$.
\end{itemize}
}

\begin{defn}[Algebra of Sets]\index{Algebra of Sets}\label{defn:algebra_of_sets}
  Given $X$, a non-empty collection of subsets of $X$,
  i.e. $\emptyset \neq \mathcal{A} \subseteq \mathcal{P}(X)$,
  is called an \hlnoteb{algebra of sets} of $X$ provided that:
  \begin{enumerate}
    \item $A_1, \ldots, A_n \in \mathcal{A} \implies
      \bigcup_{i=1}^{n} A_i \in \mathcal{A}$; and
    \item $A \in \mathcal{A} => A^C \in \mathcal{A}$.
  \end{enumerate}
\end{defn}

\begin{propo}[Properties of Algebra of Sets]\index{Properties of Algebra of Sets}\label{propo:properties_of_algebra_of_sets}
  If $\mathcal{A}$ is an algebra of sets of $X$, then
  \begin{enumerate}
    \setcounter{enumi}{2}
    \item $\emptyset, X \in \mathcal{A}$;
    \item $A, B \in \mathcal{A} \implies
      A \setminus B = \{ x \in X \mid x \in A \land x \notin B \} \in
      \mathcal{A}$ ; and
    \item $A_1, \ldots, A_n \in \mathcal{A} \implies
      \bigcap_{i=1}^{n} A_i \in \mathcal{A}$.
  \end{enumerate}
\end{propo}

\begin{proof}
  \begin{enumerate}
    \setcounter{enumi}{2}
    \item $\mathcal{A} \neq \emptyset \implies \exists A \in \mathcal{A}
      \implies A^C \in \mathcal{A} \implies A \cup A^C = X \in \mathcal{A}
      \implies \emptyset = X^C \in \mathcal{A}$.
    \item $A, B \in \mathcal{A} \implies A^C \in \mathcal{A}
      \implies A^C \cup B \in \mathcal{A}
      \implies A \setminus B = (A^C \cup B)^C \in \mathcal{A}$.
    \item (\hlnotea{De Morgan's Law})
      Notice that
      $(A_1 \cap A_2 \cap \hdots \cap A_n)^C
      = A_1^C \cup A_2^C \cup \hdots A_n^C \in \mathcal{A}$
      since $A_i^C \in \mathcal{A}$.
      Thus the complement
      \begin{equation*}
        A_1 \cap A_2 \cap \hdots \cap A_n \in \mathcal{A}.
      \end{equation*}
  \end{enumerate}
\end{proof}

\begin{defn}[$\sigma$-Algebra of Sets]\index{$\sigma$-Algebra of Sets}\label{defn:_sigma_algebra_of_sets}
  Given $X$ and $\emptyset \neq \mathcal{A} \subseteq \mathcal{P}(X)$,
  we say that $\mathcal{A}$ is a \hlnoteb{$\sigma$-algebra of sets}
  of $X$ if it is an algebra of sets and
  \begin{equation*}
    \forall A_n \in \mathcal{A}, n \in \mathbb{N}, \quad
    \bigcup_{n \in \mathbb{N}} A_n \in \mathcal{A}.
  \end{equation*}
\end{defn}

\begin{eg}
  \begin{enumerate}
    \item $\mathcal{P}(X)$ is a $\sigma$-algebra.
    \item Consider $X$ as an infinite set.
      We say that a set $A$ is \hldefn{cofinite} if $A^C$ is finite.
      Let
      \begin{equation*}
        \mathcal{A} \coloneqq \{ A \in \mathcal{P}(X)
          \mid A \text{ is finite or cofinite } \}.
      \end{equation*}
      Then $\mathcal{A}$ is an algebra of sets:
      \begin{itemize}
        \item finite union of finite sets remains finite;
        \item finite union of finite and cofinite sets remains cofinite; and
        \item complement of finite sets are the cofinite sets and vice versa.
      \end{itemize}
      However, $\mathcal{A}$ is \hlnotec{not} a $\sigma$-algebra:
      consider $A_n = \{ 2^n \} \subseteq X = \mathbb{N}$,
      which we then realize that
      \begin{equation*}
        \bigcup_{n \in \mathbb{N}} A_n = \text{ set of all even numbers },
      \end{equation*}
      but the set of all even numbers is clearly not finite, and
      its complement, which is the set of all odd numbers, is not finite.
    \item Consider $X$ as an uncountable set.
      We say that a set $A$ is \hldefn{co-countable} if $A^C$ is countable.
      \sidenote{Recall that a set $A$ is said to be countable if
      there is a one-to-one correspondence between elements of $A$
      and the natural numbers.}
      The set
      \begin{equation*}
        \mathcal{A} \coloneqq \{ A \subseteq X
          \mid A \text{ is countable or co-countable } \}
      \end{equation*}
      is a $\sigma$-algebra:
      \begin{itemize}
        \item countable union of countable sets is countable;
        \item countable union of countable
          and co-countable sets is co-countable; and
        \item complement of countable sets are co-countable and vice versa.
      \end{itemize}
  \end{enumerate}
\end{eg}

% section algebras_and_sigma_algebra_of_sets (end)

% chapter lecture_1_sep_04th_2019 (end)

\appendix

\backmatter

\fancyhead[LE]{\thepage \enspace \textsl{\leftmark}}

% \nobibliography*
\bibliography{references}

\printindex

\end{document}
% vim:tw=80:fdm=syntax

