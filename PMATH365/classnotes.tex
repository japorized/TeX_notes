% !TEX TS-program = pdflatex
\documentclass[notoc,notitlepage]{tufte-book}
% \nonstopmode % uncomment to enable nonstopmode

\usepackage{classnotetitle}

\title{PMATH365 --- Differential Geometry}
\author{Johnson Ng}
\subtitle{Classnotes for Winter 2019}
\credentials{BMath (Hons), Pure Mathematics major, Actuarial Science Minor}
\institution{University of Waterloo}

\setcounter{secnumdepth}{3}
\setcounter{tocdepth}{3}

\renewcommand{\baselinestretch}{1.2}
\usepackage{geometry}
\geometry{letterpaper}
\usepackage[parfill]{parskip}
\usepackage{graphicx}

% Essential Packages
\usepackage{makeidx}
\makeindex
\usepackage{enumitem}
\usepackage[T1]{fontenc}
\usepackage{natbib}
\bibliographystyle{apalike}
\usepackage{ragged2e}
\usepackage{etoolbox}
\usepackage{amssymb}
\usepackage{eso-pic}
\usepackage[fixed]{fontawesome5}
\usepackage{todonotes}
\usepackage{apptools, chngcntr}
\usepackage{amsmath}
\usepackage{mathrsfs}
\usepackage{stmaryrd}
\usepackage{mathtools}
\usepackage{tocloft}
\usepackage{tocbibind}
\usepackage{xparse}
\usepackage{tkz-euclide}
\usetkzobj{all}
\usepackage[utf8]{inputenc}
\usepackage{csquotes}
\usepackage[english]{babel}
\usepackage{marvosym}
\usepackage{pgf,tikz}
\usepackage{tikz-cd}
\usepackage{ifthen}
\usepackage{pgfplots}
\usepackage{fancyhdr}
\usepackage{array}
\usepackage{float}
\usepackage{xcolor}
\usepackage{soul}
\usepackage{centernot}
\usepackage{silence}
  \WarningFilter*{latex}{Marginpar on page \thepage\space moved}
\usepackage{tcolorbox}
\tcbuselibrary{skins,breakable}
\usepackage{longtable,booktabs}
\usepackage[amsmath,hyperref,thmmarks]{ntheorem}
\usepackage{thmtools}
\usepackage{hyperref}
\usepackage[noabbrev,capitalize,nameinlink]{cleveref}

\newcommand{\personalcolor}{false}
\ifthenelse{\equal{\personalcolor}{true}}{
  \usepackage{colorscheme-chaos}
}{
  \usepackage{colorscheme-student}
}

% hyperref Package Settings
\hypersetup{
    unicode=true,          % non-Latin characters in Acrobat’s bookmarks
    pdftoolbar=false,        % show Acrobat’s toolbar?
    pdfmenubar=false,        % show Acrobat’s menu?
    pdffitwindow=true,     % window fit to page when opened
    colorlinks=true,
    allcolors=magenta,
}

% tikz
\usepgfplotslibrary{polar}
\usepgflibrary{shapes.geometric}
\usetikzlibrary{angles,patterns,calc,decorations.markings,arrows.meta,tikzmark,bending}
\tikzset{midarrow/.style 2 args={
        decoration={markings,
            mark= at position #2 with {\arrow{#1}} ,
        },
        postaction={decorate}
    },
    midarrow/.default={latex}{0.5}
}
\def\centerarc[#1](#2)(#3:#4:#5)% Syntax: [draw options] (center) (initial angle:final angle:radius)
    { \draw[#1] ($(#2)+({#5*cos(#3)},{#5*sin(#3)})$) arc (#3:#4:#5); }
% from https://tex.stackexchange.com/questions/67573/tikz-shift-and-rotate-in-3d
\newcommand{\rotateRPY}[4][0/0/0]% point to be saved to \savedxyz, roll, pitch, yaw
{   \pgfmathsetmacro{\rollangle}{#2}
    \pgfmathsetmacro{\pitchangle}{#3}
    \pgfmathsetmacro{\yawangle}{#4}

    % to what vector is the x unit vector transformed, and which 2D vector is this?
    \pgfmathsetmacro{\newxx}{cos(\yawangle)*cos(\pitchangle)}% a
    \pgfmathsetmacro{\newxy}{sin(\yawangle)*cos(\pitchangle)}% d
    \pgfmathsetmacro{\newxz}{-sin(\pitchangle)}% g
    \path (\newxx,\newxy,\newxz);
    \pgfgetlastxy{\nxx}{\nxy};

    % to what vector is the y unit vector transformed, and which 2D vector is this?
    \pgfmathsetmacro{\newyx}{cos(\yawangle)*sin(\pitchangle)*sin(\rollangle)-sin(\yawangle)*cos(\rollangle)}% b
    \pgfmathsetmacro{\newyy}{sin(\yawangle)*sin(\pitchangle)*sin(\rollangle)+ cos(\yawangle)*cos(\rollangle)}% e
    \pgfmathsetmacro{\newyz}{cos(\pitchangle)*sin(\rollangle)}% h
    \path (\newyx,\newyy,\newyz);
    \pgfgetlastxy{\nyx}{\nyy};

    % to what vector is the z unit vector transformed, and which 2D vector is this?
    \pgfmathsetmacro{\newzx}{cos(\yawangle)*sin(\pitchangle)*cos(\rollangle)+ sin(\yawangle)*sin(\rollangle)}
    \pgfmathsetmacro{\newzy}{sin(\yawangle)*sin(\pitchangle)*cos(\rollangle)-cos(\yawangle)*sin(\rollangle)}
    \pgfmathsetmacro{\newzz}{cos(\pitchangle)*cos(\rollangle)}
    \path (\newzx,\newzy,\newzz);
    \pgfgetlastxy{\nzx}{\nzy};

    % transform the point given by #1
    \foreach \x/\y/\z in {#1}
    {   \pgfmathsetmacro{\transformedx}{\x*\newxx+\y*\newyx+\z*\newzx}
        \pgfmathsetmacro{\transformedy}{\x*\newxy+\y*\newyy+\z*\newzy}
        \pgfmathsetmacro{\transformedz}{\x*\newxz+\y*\newyz+\z*\newzz}
        \xdef\savedx{\transformedx}
        \xdef\savedy{\transformedy}
        \xdef\savedz{\transformedz}     
    }
}
\tikzset{RPY/.style={x={(\nxx,\nxy)},y={(\nyx,\nyy)},z={(\nzx,\nzy)}}}
\newcommand{\AxisRotator}[1][rotate=0]{%
    \tikz [x=0.25cm,y=0.60cm,line width=.2ex,-stealth,#1] \draw (0,0) arc (-150:150:1 and 1);%
  }

% enumitems
\newlist{inlinelist}{enumerate*}{1}
\setlist*[inlinelist,1]{%
  label=(\roman*),
}

% Theorem Style Customization
\setlength\theorempreskipamount{2ex}
\setlength\theorempostskipamount{3ex}

\makeatletter
\let\nobreakitem\item
\let\@nobreakitem\@item
\patchcmd{\nobreakitem}{\@item}{\@nobreakitem}{}{}
\patchcmd{\nobreakitem}{\@item}{\@nobreakitem}{}{}
\patchcmd{\@nobreakitem}{\@itempenalty}{\@M}{}{}
\patchcmd{\@xthm}{\ignorespaces}{\nobreak\ignorespaces}{}{}
\patchcmd{\@ythm}{\ignorespaces}{\nobreak\ignorespaces}{}{}

\renewtheoremstyle{break}%
  {\item{\theorem@headerfont
          ##1\ ##2\theorem@separator}\hskip\labelsep\relax\nobreakitem}%
  {\item{\theorem@headerfont
          ##1\ ##2\ (##3)\theorem@separator}\hskip\labelsep\relax\nobreakitem}
\makeatother

% ntheorem Declarations
\theorempreskip{10pt}
\theorempostskip{5pt}
\theoremstyle{break}

\theoremsymbol{\faComment}
\newtheorem{remark}{Remark}[section]
\theoremsymbol{}
\newtheorem*{strategy}{\faPaperPlane Strategy}
\newtheorem*{procedure}{\faCodeBranch\ }
\newtheorem{ex}{Exercise}[section]
\theorembodyfont{\normalfont}
\newtheorem*{solution}{\faPencil* Solution}
\theoremsymbol{\faGavel}
\newtheorem{eg}{Example}[section]
\theoremsymbol{}
\theorembodyfont{\it}

    % definition env
\theoremprework{\textcolor{blue}{\hrule height 2pt width \textwidth}}
\theoremheaderfont{\color{blue}\normalfont\bfseries}
\theorempostwork{\textcolor{blue}{\hrule height 2pt width \textwidth}}
\theoremindent10pt
\newtheorem{defn}{\faBook Definition}

    % definition env no num
\theoremprework{\textcolor{blue}{\hrule height 2pt width \textwidth}}
\theoremheaderfont{\color{blue}\normalfont\bfseries}
\theorempostwork{\textcolor{blue}{\hrule height 2pt width \textwidth}}
\theoremindent10pt
\newtheorem*{defnnonum}{\faBook Definition}

\theoremprework{\textcolor{blue}{\hrule height 2pt width \marginparwidth}}
\theoremheaderfont{\color{blue}\normalfont\bfseries}
\theorempostwork{\textcolor{blue}{\hrule height 2pt width \marginparwidth}}
\theoremindent10pt
\newtheorem{margindefn}[defn]{\faBook Definition}

\theoremprework{\textcolor{blue}{\hrule height 2pt width \textwidth}}
\theoremheaderfont{\color{blue}\normalfont\bfseries}
\theorempostwork{\textcolor{blue}{\hrule height 2pt width \textwidth}}
\theoremindent10pt
\newtheorem*{margindefnnonum}{\faBook Definition}

    % theorem envs
\theoremprework{\textcolor{magenta}{\hrule height 2pt width \textwidth}}
\theoremheaderfont{\color{magenta}\normalfont\bfseries}
\theorempostwork{\textcolor{magenta}{\hrule height 2pt width \textwidth}}
\theoremindent10pt
\newtheorem{thm}{\faCoffee Theorem}

\theoremprework{\textcolor{magenta}{\hrule height 2pt width \textwidth}}
\theorempostwork{\textcolor{magenta}{\hrule height 2pt width \textwidth}}
\theoremindent10pt
\newtheorem{propo}[thm]{\faTint Proposition}

\theoremprework{\textcolor{magenta}{\hrule height 2pt width \textwidth}}
\theorempostwork{\textcolor{magenta}{\hrule height 2pt width \textwidth}}
\theoremindent10pt
\newtheorem{crly}[thm]{\faSpaceShuttle Corollary}

\theoremprework{\textcolor{magenta}{\hrule height 2pt width \textwidth}}
\theorempostwork{\textcolor{magenta}{\hrule height 2pt width \textwidth}}
\theoremindent10pt
\newtheorem{lemma}[thm]{\faTree Lemma}

\theoremprework{\textcolor{magenta}{\hrule height 2pt width \textwidth}}
\theorempostwork{\textcolor{magenta}{\hrule height 2pt width \textwidth}}
\theoremindent10pt
\newtheorem{axiom}[thm]{\faShield Axiom}

    % theorem envs without counter
\theoremprework{\textcolor{magenta}{\hrule height 2pt width \textwidth}}
\theoremheaderfont{\color{magenta}\normalfont\bfseries}
\theorempostwork{\textcolor{magenta}{\hrule height 2pt width \textwidth}}
\theoremindent10pt
\newtheorem*{thmnonum}{\faCoffee Theorem}

\theoremprework{\textcolor{magenta}{\hrule height 2pt width \textwidth}}
\theorempostwork{\textcolor{magenta}{\hrule height 2pt width \textwidth}}
\theoremindent10pt
\newtheorem*{propononum}{\faTint Proposition}

\theoremprework{\textcolor{magenta}{\hrule height 2pt width \textwidth}}
\theorempostwork{\textcolor{magenta}{\hrule height 2pt width \textwidth}}
\theoremindent10pt
\newtheorem*{crlynonum}{\faSpaceShuttle Corollary}

\theoremprework{\textcolor{magenta}{\hrule height 2pt width \textwidth}}
\theorempostwork{\textcolor{magenta}{\hrule height 2pt width \textwidth}}
\theoremindent10pt
\newtheorem*{lemmanonum}{\faTree Lemma}

\theoremprework{\textcolor{magenta}{\hrule height 2pt width \textwidth}}
\theorempostwork{\textcolor{magenta}{\hrule height 2pt width \textwidth}}
\theoremindent10pt
\newtheorem*{axiomnonum}{\faShield Axiom}

    % envs on margins
\theoremprework{\textcolor{magenta}{\hrule height 2pt width \marginparwidth}}
\theoremheaderfont{\color{magenta}\normalfont\bfseries}
\theorempostwork{\textcolor{magenta}{\hrule height 2pt width \marginparwidth}}
\theoremindent10pt
\newtheorem{marginthm}[thm]{\faCoffee Theorem}

\theoremprework{\textcolor{magenta}{\hrule height 2pt width \marginparwidth}}
\theorempostwork{\textcolor{magenta}{\hrule height 2pt width \marginparwidth}}
\theoremindent10pt
\newtheorem{marginpropo}[thm]{\faTint Proposition}

\theoremprework{\textcolor{magenta}{\hrule height 2pt width \marginparwidth}}
\theorempostwork{\textcolor{magenta}{\hrule height 2pt width \marginparwidth}}
\theoremindent10pt
\newtheorem{margincrly}[thm]{\faSpaceShuttle Corollary}

\theoremprework{\textcolor{magenta}{\hrule height 2pt width \marginparwidth}}
\theorempostwork{\textcolor{magenta}{\hrule height 2pt width \marginparwidth}}
\theoremindent10pt
\newtheorem{marginlemma}[thm]{\faTree Lemma}

\theoremprework{\textcolor{magenta}{\hrule height 2pt width \marginparwidth}}
\theorempostwork{\textcolor{magenta}{\hrule height 2pt width \marginparwidth}}
\theoremindent10pt
\newtheorem{marginaxiom}[thm]{\faShield Axiom}

    % envs on margins without counter
\theoremprework{\textcolor{magenta}{\hrule height 2pt width \marginparwidth}}
\theoremheaderfont{\color{magenta}\normalfont\bfseries}
\theorempostwork{\textcolor{magenta}{\hrule height 2pt width \marginparwidth}}
\theoremindent10pt
\newtheorem*{marginthmnonum}{\faCoffee Theorem}

\theoremprework{\textcolor{magenta}{\hrule height 2pt width \marginparwidth}}
\theorempostwork{\textcolor{magenta}{\hrule height 2pt width \marginparwidth}}
\theoremindent10pt
\newtheorem*{marginpropononum}{\faTint Proposition}

\theoremprework{\textcolor{magenta}{\hrule height 2pt width \marginparwidth}}
\theorempostwork{\textcolor{magenta}{\hrule height 2pt width \marginparwidth}}
\theoremindent10pt
\newtheorem*{margincrlynonum}{\faSpaceShuttle Corollary}

\theoremprework{\textcolor{magenta}{\hrule height 2pt width \marginparwidth}}
\theorempostwork{\textcolor{magenta}{\hrule height 2pt width \marginparwidth}}
\theoremindent10pt
\newtheorem*{marginlemmanonum}{\faTree Lemma}

\theoremprework{\textcolor{magenta}{\hrule height 2pt width \marginparwidth}}
\theorempostwork{\textcolor{magenta}{\hrule height 2pt width \marginparwidth}}
\theoremindent10pt
\newtheorem*{marginaxiomnonum}{\faShield Axiom}

    % proof env
\theoremprework{\textcolor{green}{\hrule height 2pt width \textwidth}}
\theorembodyfont{\normalfont}
\theoremheaderfont{\color{green}\normalfont\bfseries}
\theorempostwork{\textcolor{green}{\hrule height 2pt width \textwidth}}
\theoremsymbol{\ensuremath{_\square}}
\newtheorem*{proof}{\faPencil* Proof}
\theoremsymbol{}

\theoremprework{\textcolor{green}{\hrule height 2pt width \marginparwidth}}
\theorembodyfont{\normalfont}
\theoremheaderfont{\color{green}\normalfont\bfseries}
\theorempostwork{\textcolor{green}{\hrule height 2pt width \marginparwidth}}
\theoremsymbol{\ensuremath{_\square}}
\newtheorem*{mproof}{\faPencil* Proof}
\theoremsymbol{}

    % note and notation env
\theorembodyfont{\it}

\theoremprework{\textcolor{yellow}{\hrule height 2pt width \textwidth}}
\theoremheaderfont{\color{yellow}\normalfont\bfseries}
\theorempostwork{\textcolor{yellow}{\hrule height 2pt width \textwidth}}
\newtheorem{note}{\faQuoteLeft Note}[section]

\theoremprework{\textcolor{yellow}{\hrule height 2pt width \marginparwidth}}
\theoremheaderfont{\color{yellow}\normalfont\bfseries}
\theorempostwork{\textcolor{yellow}{\hrule height 2pt width \marginparwidth}}
\newtheorem{mnote}[note]{\faQuoteLeft Note}

\theoremprework{\textcolor{yellow}{\hrule height 2pt width \textwidth}}
\theorempostwork{\textcolor{yellow}{\hrule height 2pt width \textwidth}}
\newtheorem*{notation}{\faPaw Notation}

    % warning env
\theoremprework{\textcolor{red}{\hrule height 2pt width \textwidth}}
\theoremheaderfont{\color{red}\normalfont\bfseries}
\theorempostwork{\textcolor{red}{\hrule height 2pt width \textwidth}}
\theoremindent10pt
\newtheorem*{warning}{\faBug Warning}

\theoremprework{\textcolor{red}{\hrule height 2pt width \marginparwidth}}
\theoremheaderfont{\color{red}\normalfont\bfseries}
\theorempostwork{\textcolor{red}{\hrule height 2pt width \marginparwidth}}
\theoremindent10pt
\newtheorem*{marginwarning}{\faBug Warning}

% rule for appendices
\AtAppendix{\counterwithin{defn}{chapter}}
\AtAppendix{\counterwithin{thm}{chapter}}
\AtAppendix{\counterwithin{propo}{chapter}}
\AtAppendix{\counterwithin{lemma}{chapter}}
\AtAppendix{\counterwithin{crly}{chapter}}
\AtAppendix{\counterwithin{axiom}{chapter}}

% more environments
\newtcolorbox{quotebox}[2]{
  blanker,enhanced,breakable,standard jigsaw,
  opacityback=0,
  coltext=\ifblank{#2}{black}{#2},
  left=5mm,right=5mm,top=2mm,bottom=2mm,
  colframe=\ifblank{#1}{bblack}{#1},
  boxrule=0pt,leftrule=3pt,
  fontupper=\itshape
}

\providecommand{\parthook}{}
\patchcmd{\part}{\thispagestyle}{\parthook\thispagestyle}{}{}
\newcommand{\partimage}[2][]{% \parthook[<options>]{<image>}
  \renewcommand{\parthook}{% Update \parthook
    \AddToShipoutPictureBG*{% Add picture to background of THIS page only
      \AtPageLowerLeft{\includegraphics[width=\paperwidth,height=\paperheight,#1]{#2}}}% Insert image
    \renewcommand{\parthook}{}}}% Restore \parthook

\AtBeginDocument{\renewcommand\contentsname{\slshape Table of Contents\normalfont}}
\cftpagenumbersoff{part}

\newcommand{\tuftepart}[1]{\newgeometry{}\part{#1}\restoregeometry}

% Heading formattings
% chapter format
\titleformat{\chapter}%
  {\huge\rmfamily\itshape\color{magenta}}% format applied to label+text
  {\llap{\colorbox{magenta}{\parbox[c][1cm]{3cm}{\hfill\itshape\Huge\textcolor{background}{\thechapter}}}}}% label
  {5pt}% horizontal separation between label and title body
  {\faLeaf}% before the title body
  []% after the title body

% section format
\titleformat{\section}%
  {\normalfont\Large\rmfamily\itshape\color{blue}}% format applied to label+text
  {\llap{\colorbox{blue}{\parbox{3cm}{\hfill\itshape\textcolor{background}{\thesection}}}}}% label
  {5pt}% horizontal separation between label and title body
  {}% before the title body
  []% after the title body

% subsection format
\titleformat{\subsection}%
  {\normalfont\large\itshape\color{green}}% format applied to label+text
  {\llap{\colorbox{green}{\parbox{3cm}{\hfill\textcolor{background}{\thesubsection}}}}}% label
  {1em}% horizontal separation between label and title body
  {}% before the title body
  []% after the title body

% subsubsection format
\titleclass{\subsubsection}{straight}
\titleformat{\subsubsection}%
  {\normalfont\large\itshape\color{yellow}}% format applied to label+text
  {\llap{\colorbox{yellow}{\parbox{3cm}{\hfill\textcolor{background}{\thesubsubsection}}}}}% label
  {1em}% horizontal separation between label and title body
  {}% before the title body
  []% after the title body

% Sidenote enhancements
\def\mathmarginnote#1{%
  \tag*{\rlap{\hspace\marginparsep\smash{\parbox[t]{\marginparwidth}{%
  \footnotesize#1}}}}
}

% Custom table columning
\newcolumntype{L}[1]{>{\raggedright\let\newline\\\arraybackslash\hspace{0pt}}m{#1}}
\newcolumntype{C}[1]{>{\centering\let\newline\\\arraybackslash\hspace{0pt}}m{#1}}
\newcolumntype{R}[1]{>{\raggedleft\let\newline\\\arraybackslash\hspace{0pt}}m{#1}}

% Graph styles
\pgfplotsset{compat=1.15}
\usepgfplotslibrary{fillbetween}
\pgfplotsset{four quads/.append style={axis x line=middle, axis y line=
middle, xlabel={$x$}, ylabel={$y$}, axis equal }}
\pgfplotsset{four quad complex/.append style={axis x line=middle, axis y line=
middle, xlabel={$\re$}, ylabel={$\im$}, axis equal }}
\def\axisdefaultwidth{360pt}
\pgfplotsset{
  tufteaxis/.append style = {thick},tick style = {thick,black},
  %
  % #1 = x, y, or z
  % #2 = the shift value
  /tikz/normal shift/.code 2 args = {%
    \pgftransformshift{%
        \pgfpointscale{#2}{\pgfplotspointouternormalvectorofticklabelaxis{#1}}%
    }%
  },%
  %
  range3frame/.style = {
    tick align        = outside,
    scaled ticks      = false,
    enlargelimits     = false,
    ticklabel shift   = {10pt},
    axis lines*       = left,
    line cap          = round,
    clip              = false,
    xtick style       = {normal shift={x}{10pt}},
    ytick style       = {normal shift={y}{10pt}},
    ztick style       = {normal shift={z}{10pt}},
    x axis line style = {normal shift={x}{10pt}},
    y axis line style = {normal shift={y}{10pt}},
    z axis line style = {normal shift={z}{10pt}},
  }
}

% Shortcuts
\DeclareMathOperator{\id}{id}
\DeclareMathOperator{\Img}{Img}
\DeclareMathOperator{\Res}{Res}
\DeclareMathOperator*{\argmax}{arg\,max}
\DeclareMathOperator*{\argmin}{arg\,min}
\DeclareMathOperator{\re}{Re}
\DeclareMathOperator{\im}{Im}
\DeclareMathOperator{\caparg}{Arg}
\DeclareMathOperator{\Char}{Char}
\DeclareMathOperator{\sgn}{sgn}
\DeclareMathOperator{\Range}{range}

\newcommand{\floor}[1]{\lfloor #1 \rfloor}      % simplifying the writing of a floor function
\newcommand{\ceiling}[1]{\lceil #1 \rceil}      % simplifying the writing of a ceiling function
\newcommand{\dotp}{\, \cdotp}			        % dot product to distinguish from \cdot
\newcommand{\abs}[1]{\left|#1\right|}						% absolute value
\newcommand{\lra}[1]{\left\langle \; #1 \; \right\rangle}
\newcommand{\at}[2]{\Big|_{#1}^{#2}}
\newcommand{\Arg}[1]{\caparg #1}
\renewcommand{\bar}[1]{\mkern 1.5mu \overline{\mkern -1.5mu #1 \mkern -1.5mu} \mkern 1.5mu}
\newcommand{\faktor}[2]{{\raisebox{.2em}{$#1$}\left/\raisebox{-.2em}{$#2$}\right.}}
\newcommand{\quotient}[2]{\faktor{#1}{#2}}
\newcommand{\cyclic}[1]{\left\langle #1 \right\rangle}
\newcommand{\ind}[2]{\Ind_{#2}\left( #1 \right)}
\newcommand{\notimply}{\centernot\implies}
\newcommand{\res}[2]{\underset{#2}{\Res} #1 }
\newcommand{\tworow}[3]{\begin{tabular}{@{}#1@{}} #2 \\ #3 \end{tabular}}
\renewcommand{\epsilon}{\varepsilon}
\renewcommand{\phi}{\varphi}
\newcommand{\lrarrow}{\leftrightarrow}
\newcommand{\larrow}{\leftarrow}
\newcommand{\rarrow}{\rightarrow}
\renewcommand{\atop}[2]{\genfrac{}{}{0pt}{}{#1}{#2}}
\newcommand*\dif{\mathop{}\!d}
\newcommand{\mmid}{\; \middle| \;}
\newcommand{\coprime}{\; \bot \;}
\newcommand{\norm}[1]{\left\| #1 \right\|}
\newenvironment{spmatrix}
  {\left(\begin{smallmatrix}}
  {\end{smallmatrix}\right)}

  % inspiration from: https://tex.stackexchange.com/questions/8720/overbrace-underbrace-but-with-an-arrow-instead#37758
\newcommand{\overarrow}[2]{
  \overset{\makebox[0pt]{\begin{tabular}{@{}c@{}}#2\\[0pt]\ensuremath{\uparrow}\end{tabular}}}{\ensuremath{#1}}
}
\newcommand{\underarrow}[2]{
  \underset{\makebox[0pt]{\begin{tabular}{@{}c@{}}\downarrow\\[0pt]\ensuremath{#2}\end{tabular}}}{\ensuremath{#1}}
}


	% highlighting shortcuts
\newcommand{\hlimpo}[1]{\textcolor{red}{\textbf{#1}}}
\newcommand{\hlwarn}[1]{\textcolor{yellow}{\textbf{#1}}}
\newcommand{\hldefn}[1]{\textcolor{blue}{\index{#1}\textbf{#1}}}
\newcommand{\hlnotea}[1]{\textcolor{green}{\textbf{#1}}}
\newcommand{\hlnoteb}[1]{\textcolor{cyan}{\textbf{#1}}}
\newcommand{\hlb}[2]{\colorbox{#1!30!background}{#2}}
\newcommand{\hlbnotea}[1]{\hlb{green}{#1}}
\newcommand{\hlbnoteb}[1]{\hlb{cyan}{#1}}
\newcommand{\hlbnotec}[1]{\hlb{yellow}{#1}}
\newcommand{\hlbnoted}[1]{\hlb{magenta}{#1}}
\newcommand{\hlbnotee}[1]{\hlb{red}{#1}}
\newcommand{\WTP}{\textcolor{bwhite}{WTP} }
\newcommand{\WTS}{\textcolor{bwhite}{WTS} }

  % stars on important stuff
\newcommand{\imponote}{\faStar}
\newcommand{\vimponote}{\faStar\faStar}
\newcommand{\vvimponote}{\faStar\faStar\faStar}

% Document header formatting
\makeatletter
\pagestyle{fancy}
\fancyhead{}
\fancyhead[RO]{\textsl{\@title} \enspace \thepage}
\fancyhead[LE]{\thepage \enspace \textsl{\leftmark \enspace \rightmark}}
\makeatother
\renewcommand{\chaptermark}[1]{\markboth{#1}{}}
\renewcommand{\sectionmark}[1]{\markright{#1}}

% Comment the two lines below if you want to print the document
\pagecolor{background}
\color{foreground}

\usepackage{tikz-3dplot}

\DeclareMathOperator{\std}{std}

\begin{document}
\hypersetup{pageanchor=false}
\maketitle
\hypersetup{pageanchor=true}
\begin{fullwidth}
\tableofcontents
\end{fullwidth}

\newpage
\begin{fullwidth}
  \renewcommand{\listtheoremname}{\faBook\ \slshape List of Definitions}
  \listoftheorems[ignoreall,show={defn}]
\end{fullwidth}

\newpage 
\begin{fullwidth}
  \renewcommand{\listtheoremname}{\faCoffee\ \slshape List of Theorems}
  \listoftheorems[ignoreall,
    show={axiom,lemma,thm,crly,propo,marginthm,marginpropo,marginlemma,marginaxiom,margincrly}
  ]
\end{fullwidth}

\newpage
\begin{fullwidth}
  \renewcommand{\listtheoremname}{\faCodeBranch\ \slshape List of Procedures}
  \listoftheorems[ignoreall, show={procedure}]
\end{fullwidth}


\chapter*{Preface}%
\addcontentsline{toc}{chapter}{Preface}
\label{chp:preface}
% chapter preface

\nocite{karigiannis2019}

This course is a post-requisite of MATH 235/245 (\href{http://www.ucalendar.uwaterloo.ca/1819/COURSE/course-MATH.html\#MATH235}{Linear Algebra II}) and AMATH 231 (\href{http://www.ucalendar.uwaterloo.ca/1819/COURSE/course-AMATH.html#AMATH231}{Calculus IV}) or MATH 247 (\href{http://www.ucalendar.uwaterloo.ca/1819/COURSE/course-MATH.html#MATH247}{Advanced Calculus III}). In other words, familiarity with vector spaces and calculus is expected.

The course is spiritually separated into two parts. The first part shall be called \textbf{Exterior Differential Calculus}, which allows for a natural, metric-independent generalization of \hlnotea{Stokes' Theorem}, \hlnotea{Gauss's Theorem}, and \hlnotea{Green's Theorem}. Our end goal of this part is to arrive at Stokes' Theorem, that renders the \hlnotea{Fundamental Theorem of Calculus} as a special case of the theorem.

The second part of the course shall be called in the name of the course: \textbf{Differential Geometry}. This part is dedicated to studying geometry using techniques from differential calculus, integral calculus, linear algebra, and multilinear algebra.

% chapter preface (end)

\newgeometry{letterpaper}
\part{Exterior Differential Calculus}
\restoregeometry

\chapter{Lecture 1 Jan 7th}%
\label{chp:lecture_1_jan_7th}
% chapter lecture_1_jan_7th

\section{Linear Algebra Review}%
\label{sec:linear_algebra_review}
% section linear_algebra_review

\begin{defn}[Linear Map]\index{Linear Map}\label{defn:linear_map}
  Let $V, W$ be finite dimensional real vector spaces. A map $T : V \to W$ is called \hlnoteb{linear} if $\forall a, b \in \mathbb{R}$, $\forall v \in V$ and $\forall w \in W$,
  \begin{equation*}
    T(av + bw) = aT(v) + bT(w).
  \end{equation*}
  We define L(U, W) to be the set of all linear maps from $V$ to $W$.
\end{defn}

\begin{note}
  \begin{itemize}
    \item Note that $L(U, W)$ is itself a finite dimensional real vector space.
    \item The structure of the vector space $L(V, W)$ is such that $\forall T, S \in L(V, W)$, and $\forall a, b \in \mathbb{R}$, we have
      \begin{equation*}
        aT + bS : V \to W
      \end{equation*}
      and
      \begin{equation*}
        (aT + bS)(v) = aT(v) + bS(v).
      \end{equation*}
    \item A special case: when $W = V$, we usually write
      \begin{equation*}
        L(V, W) = L(V),
      \end{equation*}
      and we call this the \hldefn{space of linear operators on $V$}.
  \end{itemize}
\end{note}

Now suppose $\dim(V) = n$ for some $n \in \mathbb{N}$. This means that there exists a basis $\{ e_1 , \ldots, e_n \}$ of $V$ with $n$ elements.

\begin{defn}[Basis]\index{Basis}\label{defn:basis}
  A basis $\mathcal{B} = \{ e_1, \ldots, e_n \}$ of an $n$-dimensional vector space $V$ is a subset of $V$ where
  \begin{enumerate}
    \item $\mathcal{B}$ \hlnotea{spans} $V$, i.e. $\forall v \in V$
      \begin{equation*}
        v = \sum_{i=1}^{n} v^i e_i. \quad \footnotemark
      \end{equation*}
      \footnotetext{We shall use a different convention when we write a linear combination. In particular, we use $v^i$ to represent the $i$\textsuperscript{th} coefficient of the linear combination instead of $v_i$. Note that this should not be confused with taking powers, and should be clear from the context of the discussion.}
    \item $e_1, \ldots, e_n$ are linearly independent, i.e.
      \begin{equation*}
        v^i e_i = 0 \implies v^i = 0 \text{ for every } i.
      \end{equation*}
  \end{enumerate}
\end{defn}

\begin{note}
  We shall abusively write
  \begin{equation*}
    v^i e_i = \sum_{i} v^i e_i.
  \end{equation*}
  Again, this should be clear from the context of the discussion.
\end{note}

The two conditions that define a basis implies that any $v \in V$ can be expressed as $v^i e_i$, where $v^i \in \mathbb{R}$.

\begin{defn}[Coordinate Vector]\index{Coordinate Vector}\label{defn:coordinate_vector}
  The $n$-tuple $(v^1, \ldots, v^n) \in \mathbb{R}^n$ is called the \hlnoteb{coordinate vector} $[v]_{\mathcal{B}} \in \mathbb{R}^n$ of $v$ with respect to the basis $\mathcal{B} = \{ e_1, \ldots, e_n \}$.
\end{defn}

\begin{note}
  It is clear that the coordinate vector $[v]_{\mathcal{B}}$ is dependent on the basis $\mathcal{B}$. Note that we shall also assume that the basis is ``ordered'', which is somewhat important since the same basis (set-wise) with a different ``ordering'' may give us a completely different coordinate vector.
\end{note}

\begin{eg}
  Let $V = \mathbb{R}^n$, and $\hat{e}_i = ( 0, \ldots, 0, 1, 0, \ldots, 0 )$, where $1$ is the $i$\textsuperscript{th} compoenent of $\hat{e}_1$. Then
  \begin{equation*}
    \mathcal{B}_{\std} = \left\{ \hat{e}_1, \ldots, \hat{e}_n \right\}
  \end{equation*}
  is called the \hldefn{standard basis} of $\mathbb{R}^n$.
\end{eg}

\begin{note}
  Let $v = (v^1, \ldots, v^n) \in \mathbb{R}^n$. Then
  \begin{equation*}
    v = v^1 \hat{e}_1 + \hdots v^n \hat{e}_n.
  \end{equation*}
  So $\mathbb{R}^n \ni [v]_{\mathcal{B}_{\std}} = v \in V = \mathbb{R}^n$.

  This is a privilege enjoyed by the $n$-dimensional vector space $\mathbb{R}^n$.
\end{note}

Now if we choose a \hldefn{non-standard basis} of $\mathbb{R}^n$, say $\tilde{\mathcal{B}}$, then $[v]_{\tilde{\mathcal{B}}} \neq v$.

\begin{note}
  It does not make sense to ask if a standard basis exists for an arbitrary space, as we have seen above. A geometrical way of wrestling with this notion is as follows:

  \begin{figure}[ht]
    \centering
    \begin{tikzpicture}
      \draw[->] (0, 0, 0) -- (2, 0, 0) node[right] {$x$};
      \draw[->] (0, 0, 0) -- (0, 2, 0) node[above] {$z$};
      \draw[->] (0, 0, 0) -- (0, 0, 2) node[below,left] {$y$};

      \draw[-,fill=be-blue,fill opacity=0.5] (1.5, 0, 0.5) -- (0.5, 0, 1.5) -- (-1.5, 1, -0.5) -- (-0.5, 1, -1.5) -- (1.5, 0, 0.5);
      \node[above,right] at (1.5, 1.5, 0) {a $2$-dimensional subspace in $\mathbb{R}^3$};
    \end{tikzpicture}
    \caption{An arbitrary $2$-dimensional subspace in a $3$-dimensional space}
    \label{fig:an_arbitrary_2-dimensional_subspace_in_a_3-dimensional_space}
  \end{figure}

  While the subspace is embedding in a vector space of which has a standard basis, we cannot establish a ``standard'' basis for this $2$-dimensional subspace. In laymen terms, we cannot tell which direction is up or down, positive or negative for the subspace, without making assumptions.
\end{note}

However, since we are still in a finite-dimensional vector space, we can still make a connection to a Euclidean space of the same dimension.

\begin{defn}[Linear Isomorphism]\index{Linear Isomorphism}\label{defn:linear_isomorphism}
  Let $V$ be $n$-dimensional, and $\mathcal{B} = \left\{ e_1, \ldots, e_n \right\}$ be some basis of $V$. The map
  \begin{equation*}
    v = v^i e_i \mapsto [v]_{\mathcal{B}}
  \end{equation*}
  from $V$ to $\mathbb{R}^n$ is a \hlnoteb{linear isomorphism} of vector spaces.
\end{defn}

\begin{ex}
  Prove that the said linear isomorphism is indeed linear and bijective\sidenote{i.e. we are right in calling it linear and being an isomorphism}.
\end{ex}

\begin{note}
  Any $n$-dimensional real vecotr space is isomorphic to $\mathbb{R}^n$, but not \hlnotea{canonically} so, as it requires the knowledge of the basis that is arbitrarily chosen. In other words, a different set of basis would give us a different isomorphism.
\end{note}

% section linear_algebra_review (end)

\section{Orientation}%
\label{sec:orientation}
% section orientation

Consider an $n$-dimensional vector space $V$. Recall that for any linear operator $T \in L(V)$, we may associate a real number $\det(T)$, called the \hldefn{determinant} of $T$, such that $T$ is said to be \hldefn{invertible} iff $\det(T) \neq 0$.

\begin{defn}[Same and Opposite Orientations]\index{Same orientation}\index{Opposite orientation}\label{defn:same_and_opposite_orientations}
  Let
  \begin{equation*}
    \mathcal{B} = \left\{ e_1, \ldots, e_n \right\} \quad \text{ and } \quad \tilde{\mathcal{B}} = \left\{ \tilde{e}_1, \ldots, \tilde{e}_n \right\}
  \end{equation*}
  be two ordered bases of $V$. Let $T \in L(V)$ be the linear operator defined by
  \begin{equation*}
    T(e_i) = \tilde{e}_i
  \end{equation*}
  for each $i = 1, 2, \ldots, n$. This mapping is clearly invertible, and so $\det(T) \neq 0$, and $T^{-1}$ is also linear, such that $T^{-1} \left( \tilde{e}_i \right) = e_i$, for each $i$.

  We say that $\mathcal{B}$ and $\tilde{\mathcal{B}}$ determine the \hlnoteb{same orientation} if $\det(T) > 0$, and we say that they determine the \hlnoteb{opposite orientations} if $\det(T) < 0$.
\end{defn}

\begin{note}
  \begin{itemize}
    \item This notion of orientation only works in real vector spaces, as, for instance, in a complex vector space, there is no sense of ``positivity'' or ``negativity''.
    \item Whenever we talk about same and opposite orientation(s), we are usually talking about 2 sets of bases. It makes sense to make a comparison to the standard basis in a Euclidean space, and determine that the compared (non-)standard basis is ``positive'' (same direction) or ``negative'' (opposite), but, again, in an arbitrary space, we do not have this convenience.
  \end{itemize}
\end{note}

\begin{ex}[A1Q1]
  Show that any $n$-dimensional real vector space $V$ admits exactly 2 orientations.
\end{ex}

\begin{eg}
  On $\mathbb{R}^n$, consider the standard basis
  \begin{equation*}
    \mathcal{B}_{\std} = \{ \hat{e}_1, \ldots, \hat{e}_n \}.
  \end{equation*}
  The orientation determined by $\mathcal{B}_{\std}$ is called the \hldefn{standard orientation} of $\mathbb{R}^n$.
\end{eg}

\begin{defn}[Dual Space]\index{Dual Space}\label{defn:dual_space}
  Let $V$ be an $n$-dimensional vector space. Then $\mathbb{R}$ is a $1$-dimensional real vector space. Thus we have that $L(V, \mathbb{R})$ is also a real vector space\sidenote{Note that $L(V, \mathbb{R})$ is also finite dimensional since both the domain and codomain are finite dimensional.}. The \hlnoteb{dual space} $V^*$ of $V$ is defined to be
  \begin{equation*}
    V^* := L(V, \mathbb{R}).
  \end{equation*}
\end{defn}

Let $\mathcal{B}$ be a basis of $V$. For all $i = 1, 2, \ldots, n$, let $e^i \in V^*$ such that
\begin{equation*}
  e^i(e_j) = \delta^i_j = \begin{cases}
    1 & i = j \\
    0 & i \neq j
  \end{cases}.
\end{equation*}
This $\delta^i_j$ is known as the \hldefn{Kronecker Delta}.

In general, we have that for every $v = v^j e_j \in V$, where $v^i \in \mathbb{R}$, by the linearity of $e^i$, we have
\begin{equation*}
  e^i(v) = e^i(v^j e_j) = v^j e^i(e_j) = v_j \delta^i_j = v^i.
\end{equation*}
So each of the $e^i$, when applied on $v$, gives us the $i$\textsuperscript{th} component of $[v]_{\mathcal{B}}$, where $\mathcal{B}$ is a basis of $V$.

% section orientation (end)

% chapter lecture_1_jan_7th (end)

\appendix

\backmatter

\pagestyle{plain}

\nobibliography*
\bibliography{references}

\printindex

\end{document}

