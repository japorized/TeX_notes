% !TEX TS-program = pdflatex
\documentclass[notoc,notitlepage]{tufte-book}
% \nonstopmode % uncomment to enable nonstopmode

\usepackage{classnotetitle}

\title{ACTSC371 --- Introduction to Investments}
\author{Johnson Ng}
\subtitle{Class notes for Fall 2019}
\credentials{BMath (Hons), Pure Mathematics major, Actuarial Science Minor}
\institution{University of Waterloo}

\setcounter{secnumdepth}{3}
\setcounter{tocdepth}{3}

\renewcommand{\baselinestretch}{1.1}
\usepackage{geometry}
\geometry{letterpaper}
\usepackage[parfill]{parskip}
\usepackage{graphicx}

% Essential Packages
\usepackage{makeidx}
\makeindex
\usepackage{enumitem}
\usepackage[T1]{fontenc}
\usepackage{natbib}
\bibliographystyle{apalike}
\usepackage{ragged2e}
\usepackage{etoolbox}
\usepackage{amssymb}
\usepackage{fontawesome}
\usepackage{amsmath}
\usepackage{mathrsfs}
\usepackage{mathtools}
\usepackage{xparse}
\usepackage{tkz-euclide}
\usetkzobj{all}
\usepackage[utf8]{inputenc}
\usepackage{csquotes}
\usepackage[english]{babel}
\usepackage{marvosym}
\usepackage{pgf,tikz}
\usepackage{pgfplots}
\usepackage{fancyhdr}
\usepackage{array}
\usepackage{faktor}
\usepackage{float}
\usepackage{xcolor}
\usepackage{centernot}
\usepackage{silence}
  \WarningFilter*{latex}{Marginpar on page \thepage\space moved}
\usepackage{tcolorbox}
\tcbuselibrary{skins,breakable}
\usepackage{longtable}
\usepackage[amsmath,hyperref]{ntheorem}
\usepackage{hyperref}
\usepackage[noabbrev,capitalize,nameinlink]{cleveref}

% xcolor (scheme: base16 eighties)
\definecolor{base16-eighties-dark}{HTML}{2D2D2D}
\definecolor{base16-eighties-light}{HTML}{D3D0C8}
\definecolor{base16-eighties-magenta}{HTML}{CD98CD}
\definecolor{base16-eighties-red}{HTML}{F47678}
\definecolor{base16-eighties-yellow}{HTML}{E2B552}
\definecolor{base16-eighties-green}{HTML}{98CD97}
\definecolor{base16-eighties-lightblue}{HTML}{61CCCD}
\definecolor{base16-eighties-blue}{HTML}{6498CE}
\definecolor{base16-eighties-brown}{HTML}{D47B4E}
\definecolor{base16-eighties-gray}{HTML}{747369}

% hyperref Package Settings
\hypersetup{
    bookmarks=true,         % show bookmarks bar?
    unicode=true,          % non-Latin characters in Acrobat’s bookmarks
    pdftoolbar=false,        % show Acrobat’s toolbar?
    pdfmenubar=false,        % show Acrobat’s menu?
    pdffitwindow=true,     % window fit to page when opened
    colorlinks=true,
    allcolors=base16-eighties-magenta,
}

% tikz
\usepgfplotslibrary{polar}
\usepgflibrary{shapes.geometric}
\usetikzlibrary{angles,patterns,calc,decorations.markings}
\tikzset{midarrow/.style 2 args={
        decoration={markings,
            mark= at position #2 with {\arrow{#1}} ,
        },
        postaction={decorate}
    },
    midarrow/.default={latex}{0.5}
}
\def\centerarc[#1](#2)(#3:#4:#5)% Syntax: [draw options] (center) (initial angle:final angle:radius)
    { \draw[#1] ($(#2)+({#5*cos(#3)},{#5*sin(#3)})$) arc (#3:#4:#5); }

% enumitems
\newlist{inlinelist}{enumerate*}{1}
\setlist*[inlinelist,1]{%
  label=(\roman*),
}

% Theorem Style Customization
\setlength\theorempreskipamount{2ex}
\setlength\theorempostskipamount{3ex}

\makeatletter
\let\nobreakitem\item
\let\@nobreakitem\@item
\patchcmd{\nobreakitem}{\@item}{\@nobreakitem}{}{}
\patchcmd{\nobreakitem}{\@item}{\@nobreakitem}{}{}
\patchcmd{\@nobreakitem}{\@itempenalty}{\@M}{}{}
\patchcmd{\@xthm}{\ignorespaces}{\nobreak\ignorespaces}{}{}
\patchcmd{\@ythm}{\ignorespaces}{\nobreak\ignorespaces}{}{}

\renewtheoremstyle{break}%
  {\item{\theorem@headerfont
          ##1\ ##2\theorem@separator}\hskip\labelsep\relax\nobreakitem}%
  {\item{\theorem@headerfont
          ##1\ ##2\ (##3)\theorem@separator}\hskip\labelsep\relax\nobreakitem}
\makeatother

% ntheorem + framed
\makeatletter

% ntheorem Declarations
\theorempreskip{10pt}
\theorempostskip{5pt}
\theoremstyle{break}

\newtheorem*{solution}{\faPencil $\enspace$ Solution}
\newtheorem*{remark}{Remark}
\newtheorem{eg}{Example}[section]
\newtheorem{ex}{Exercise}[section]

    % definition env
\theoremprework{\textcolor{base16-eighties-blue}{\hrule height 2pt}}
\theoremheaderfont{\color{base16-eighties-blue}\normalfont\bfseries}
\theorempostwork{\textcolor{base16-eighties-blue}{\hrule height 2pt}}
\theoremindent10pt
\newtheorem{defn}{\faBook \enspace Definition}

    % definition env no num
\theoremprework{\textcolor{base16-eighties-blue}{\hrule height 2pt}}
\theoremheaderfont{\color{base16-eighties-blue}\normalfont\bfseries}
\theorempostwork{\textcolor{base16-eighties-blue}{\hrule height 2pt}}
\theoremindent10pt
\newtheorem*{defnnonum}{\faBook \enspace Definition}

    % theorem envs
\theoremprework{\textcolor{base16-eighties-magenta}{\hrule height 2pt}}
\theoremheaderfont{\color{base16-eighties-magenta}\normalfont\bfseries}
\theorempostwork{\textcolor{base16-eighties-magenta}{\hrule height 2pt}}
\theoremindent10pt
\newtheorem{thm}{\faCoffee \enspace Theorem}

\theoremprework{\textcolor{base16-eighties-magenta}{\hrule height 2pt}}
\theorempostwork{\textcolor{base16-eighties-magenta}{\hrule height 2pt}}
\theoremindent10pt
\newtheorem{propo}[thm]{\faTint \enspace Proposition}

\theoremprework{\textcolor{base16-eighties-magenta}{\hrule height 2pt}}
\theorempostwork{\textcolor{base16-eighties-magenta}{\hrule height 2pt}}
\theoremindent10pt
\newtheorem{crly}[thm]{\faSpaceShuttle \enspace Corollary}

\theoremprework{\textcolor{base16-eighties-magenta}{\hrule height 2pt}}
\theorempostwork{\textcolor{base16-eighties-magenta}{\hrule height 2pt}}
\theoremindent10pt
\newtheorem{lemma}[thm]{\faTree \enspace Lemma}

\theoremprework{\textcolor{base16-eighties-magenta}{\hrule height 2pt}}
\theorempostwork{\textcolor{base16-eighties-magenta}{\hrule height 2pt}}
\theoremindent10pt
\newtheorem{axiom}[thm]{\faShield \enspace Axiom}

    % theorem envs without counter
\theoremprework{\textcolor{base16-eighties-magenta}{\hrule height 2pt}}
\theoremheaderfont{\color{base16-eighties-magenta}\normalfont\bfseries}
\theorempostwork{\textcolor{base16-eighties-magenta}{\hrule height 2pt}}
\theoremindent10pt
\newtheorem*{thmnonum}{\faCoffee \enspace Theorem}

\theoremprework{\textcolor{base16-eighties-magenta}{\hrule height 2pt}}
\theorempostwork{\textcolor{base16-eighties-magenta}{\hrule height 2pt}}
\theoremindent10pt
\newtheorem*{propononum}{\faTint \enspace Proposition}

\theoremprework{\textcolor{base16-eighties-magenta}{\hrule height 2pt}}
\theorempostwork{\textcolor{base16-eighties-magenta}{\hrule height 2pt}}
\theoremindent10pt
\newtheorem*{crlynonum}{\faSpaceShuttle \enspace Corollary}

\theoremprework{\textcolor{base16-eighties-magenta}{\hrule height 2pt}}
\theorempostwork{\textcolor{base16-eighties-magenta}{\hrule height 2pt}}
\theoremindent10pt
\newtheorem*{lemmanonum}{\faTree \enspace Lemma}

\theoremprework{\textcolor{base16-eighties-magenta}{\hrule height 2pt}}
\theorempostwork{\textcolor{base16-eighties-magenta}{\hrule height 2pt}}
\theoremindent10pt
\newtheorem*{axiomnonum}{\faShield \enspace Axiom}

    % proof env
\theoremprework{\textcolor{base16-eighties-brown}{\hrule height 2pt}}
\theoremheaderfont{\color{base16-eighties-brown}\normalfont\bfseries}
\theorempostwork{\textcolor{base16-eighties-brown}{\hrule height 2pt}}
\newtheorem*{proof}{\faPencil \enspace Proof}

    % note and notation env
\theoremprework{\textcolor{base16-eighties-yellow}{\hrule height 2pt}}
\theoremheaderfont{\color{base16-eighties-yellow}\normalfont\bfseries}
\theorempostwork{\textcolor{base16-eighties-yellow}{\hrule height 2pt}}
\newtheorem*{note}{\faQuoteLeft \enspace Note}

\theoremprework{\textcolor{base16-eighties-yellow}{\hrule height 2pt}}
\theorempostwork{\textcolor{base16-eighties-yellow}{\hrule height 2pt}}
\newtheorem*{notation}{\faPaw \enspace Notation}

    % warning env
\theoremprework{\textcolor{base16-eighties-red}{\hrule height 2pt}}
\theoremheaderfont{\color{base16-eighties-red}\normalfont\bfseries}
\theorempostwork{\textcolor{base16-eighties-red}{\hrule height 2pt}}
\theoremindent10pt
\newtheorem*{warning}{\faBug \enspace Warning}

% more environments
\newtcolorbox{redquote}{
  blanker,enhanced,breakable,standard jigsaw,
  opacityback=0,
  coltext=base16-eighties-light,
  left=5mm,right=5mm,top=2mm,bottom=2mm,
  colframe=base16-eighties-red,
  boxrule=0pt,leftrule=3pt,
  fontupper=\itshape
}
\newtcolorbox{bluequote}{
  blanker,enhanced,breakable,standard jigsaw,
  opacityback=0,
  coltext=base16-eighties-light,
  left=5mm,right=5mm,top=2mm,bottom=2mm,
  colframe=base16-eighties-blue,
  boxrule=0pt,leftrule=3pt,
  fontupper=\itshape
}
\newtcolorbox{greenquote}{
  blanker,enhanced,breakable,standard jigsaw,
  opacityback=0,
  coltext=base16-eighties-light,
  left=5mm,right=5mm,top=2mm,bottom=2mm,
  colframe=base16-eighties-green,
  boxrule=0pt,leftrule=3pt,
  fontupper=\itshape
}
\newtcolorbox{yellowquote}{
  blanker,enhanced,breakable,standard jigsaw,
  opacityback=0,
  coltext=base16-eighties-light,
  left=5mm,right=5mm,top=2mm,bottom=2mm,
  colframe=base16-eighties-yellow,
  boxrule=0pt,leftrule=3pt,
  fontupper=\itshape
}
\newtcolorbox{magentaquote}{
  blanker,enhanced,breakable,standard jigsaw,
  opacityback=0,
  coltext=base16-eighties-light,
  left=5mm,right=5mm,top=2mm,bottom=2mm,
  colframe=base16-eighties-magenta,
  boxrule=0pt,leftrule=3pt,
  fontupper=\itshape
}

% ntheorem listtheorem style
\makeatother
\newlength\widesttheorem
\AtBeginDocument{
  \settowidth{\widesttheorem}{Proposition A.1.1.1\quad}
}

\makeatletter
\def\thm@@thmline@name#1#2#3#4{%
        \@dottedtocline{-2}{0em}{2.3em}%
                   {\makebox[\widesttheorem][l]{#1 \protect\numberline{#2}}#3}%
                   {#4}}
\@ifpackageloaded{hyperref}{
\def\thm@@thmline@name#1#2#3#4#5{%
    \ifx\#5\%
        \@dottedtocline{-2}{0em}{2.3em}%
            {\makebox[\widesttheorem][l]{#1 \protect\numberline{#2}}#3}%
            {#4}
    \else
        \ifHy@linktocpage\relax\relax
            \@dottedtocline{-2}{0em}{2.3em}%
                {\makebox[\widesttheorem][l]{#1 \protect\numberline{#2}}#3}%
                {\hyper@linkstart{link}{#5}{#4}\hyper@linkend}%
        \else
            \@dottedtocline{-2}{0em}{2.3em}%
                {\hyper@linkstart{link}{#5}%
                  {\makebox[\widesttheorem][l]{#1 \protect\numberline{#2}}#3}\hyper@linkend}%
                    {#4}%
        \fi
    \fi}
}

\makeatletter
\def\thm@@thmline@noname#1#2#3#4{%
        \@dottedtocline{-2}{0em}{5em}%
                   {{\protect\numberline{#2}}#3}%
                   {#4}}
\@ifpackageloaded{hyperref}{
\def\thm@@thmline@noname#1#2#3#4#5{%
    \ifx\#5\%
        \@dottedtocline{-2}{0em}{5em}%
            {{\protect\numberline{#2}}#3}%
            {#4}
    \else
        \ifHy@linktocpage\relax\relax
            \@dottedtocline{-2}{0em}{5em}%
                {{\protect\numberline{#2}}#3}%
                {\hyper@linkstart{link}{#5}{#4}\hyper@linkend}%
        \else
            \@dottedtocline{-2}{0em}{5em}%
                {\hyper@linkstart{link}{#5}%
                  {{\protect\numberline{#2}}#3}\hyper@linkend}%
                    {#4}%
        \fi
    \fi}
}

\theoremlisttype{allname}

\AtBeginDocument{\renewcommand\contentsname{Table of Contents}}

% Heading formattings
% chapter format
\titleformat{\chapter}%
  {\huge\rmfamily\itshape\color{base16-eighties-magenta}}% format applied to label+text
  {\llap{\colorbox{base16-eighties-magenta}{\parbox{1.5cm}{\hfill\itshape\huge\textcolor{base16-eighties-dark}{\thechapter}}}}}% label
  {5pt}% horizontal separation between label and title body
  {}% before the title body
  []% after the title body

% section format
\titleformat{\section}%
  {\normalfont\Large\rmfamily\itshape\color{base16-eighties-blue}}% format applied to label+text
  {\llap{\colorbox{base16-eighties-blue}{\parbox{1.5cm}{\hfill\itshape\textcolor{base16-eighties-dark}{\thesection}}}}}% label
  {5pt}% horizontal separation between label and title body
  {}% before the title body
  []% after the title body

% subsection format
\titleformat{\subsection}%
  {\normalfont\large\itshape\color{base16-eighties-green}}% format applied to label+text
  {\llap{\colorbox{base16-eighties-green}{\parbox{1.5cm}{\hfill\textcolor{base16-eighties-dark}{\thesubsection}}}}}% label
  {1em}% horizontal separation between label and title body
  {}% before the title body
  []% after the title body

% Sidenote enhancements
\def\mathmarginnote#1{%
  \tag*{\rlap{\hspace\marginparsep\smash{\parbox[t]{\marginparwidth}{%
  \footnotesize#1}}}}
}

% Custom table columning
\newcolumntype{L}[1]{>{\raggedright\let\newline\\\arraybackslash\hspace{0pt}}m{#1}}
\newcolumntype{C}[1]{>{\centering\let\newline\\\arraybackslash\hspace{0pt}}m{#1}}
\newcolumntype{R}[1]{>{\raggedleft\let\newline\\\arraybackslash\hspace{0pt}}m{#1}}

% Custom math operator
% \DeclareMathOperator{\rem}{rem}
\DeclareMathOperator*{\argmax}{arg\,max}
\DeclareMathOperator*{\argmin}{arg\,min}
\DeclareMathOperator{\re}{Re}
\DeclareMathOperator{\im}{Im}
\DeclareMathOperator{\caparg}{Arg}
\DeclareMathOperator{\Ind}{Ind}
\DeclareMathOperator{\Res}{Res}

% Graph styles
\pgfplotsset{compat=1.15}
\usepgfplotslibrary{fillbetween}
\pgfplotsset{four quads/.append style={axis x line=middle, axis y line=
middle, xlabel={$x$}, ylabel={$y$}, axis equal }}
\pgfplotsset{four quad complex/.append style={axis x line=middle, axis y line=
middle, xlabel={$\re$}, ylabel={$\im$}, axis equal }}

% Shortcuts
\newcommand{\floor}[1]{\lfloor #1 \rfloor}      % simplifying the writing of a floor function
\newcommand{\ceiling}[1]{\lceil #1 \rceil}      % simplifying the writing of a ceiling function
\newcommand{\dotp}{\, \cdotp}			        % dot product to distinguish from \cdot
\newcommand{\qed}{\hfill\ensuremath{\square}}   % Q.E.D sign
\newcommand{\abs}[1]{\left|#1\right|}						% absolute value
\newcommand{\lra}[1]{\langle \; #1 \; \rangle}
\newcommand{\at}[2]{\Big|_{#1}^{#2}}
\newcommand{\Arg}[1]{\caparg #1}
\renewcommand{\bar}[1]{\mkern 1.5mu \overline{\mkern -1.5mu #1 \mkern -1.5mu} \mkern 1.5mu}
\newcommand{\quotient}[2]{\faktor{#1}{#2}}
\newcommand{\cyclic}[1]{\left\langle #1 \right\rangle}
	% highlighting shortcuts
\newcommand{\hlimpo}[1]{\textcolor{base16-eighties-red}{\textbf{#1}}}
\newcommand{\hlwarn}[1]{\textcolor{base16-eighties-yellow}{\textbf{#1}}}
\newcommand{\hldefn}[1]{\textcolor{base16-eighties-blue}{\index{#1}\textbf{#1}}}
\newcommand{\hlnotea}[1]{\textcolor{base16-eighties-green}{\textbf{#1}}}
\newcommand{\hlnoteb}[1]{\textcolor{base16-eighties-lightblue}{\textbf{#1}}}
\newcommand{\hlnotec}[1]{\textcolor{base16-eighties-brown}{\textbf{#1}}}
\newcommand{\WTP}{\textcolor{base16-eighties-brown}{WTP} }
\newcommand{\WTS}{\textcolor{base16-eighties-brown}{WTS} }
\newcommand{\ind}[2]{\Ind_{#2}\left( #1 \right)}
\newcommand{\notimply}{\centernot\implies}
\newcommand{\res}[2]{\underset{#2}{\Res} #1 }
\newcommand{\tworow}[3]{\begin{tabular}{@{}#1@{}} #2 \\ #3 \end{tabular}}
\renewcommand{\epsilon}{\varepsilon}
\newcommand{\lrarrow}{\leftrightarrow}
\newcommand{\larrow}{\leftarrow}
\newcommand{\rarrow}{\rightarrow}
\renewcommand{\atop}[2]{\genfrac{}{}{0pt}{}{#1}{#2}}
\newcommand*\dif{\mathop{}\!d}

  % inspiration from: https://tex.stackexchange.com/questions/8720/overbrace-underbrace-but-with-an-arrow-instead#37758
\newcommand{\overarrow}[2]{
  \overset{\makebox[0pt]{\begin{tabular}{@{}c@{}}#2\\[0pt]\ensuremath{\uparrow}\end{tabular}}}{#1}
}
\newcommand{\underarrow}[2]{
  \underset{\makebox[0pt]{\begin{tabular}{@{}c@{}}\downarrow\\[0pt]\ensuremath{#2}\end{tabular}}}{#1}
}

% Document header formatting
\renewcommand{\chaptermark}[1]{\markboth{#1}{}}
\renewcommand{\sectionmark}[1]{\markright{#1}}
\makeatletter
\pagestyle{fancy}
\fancyhead{}
\fancyhead[RO]{\textsl{\@title} \enspace \thepage}
\fancyhead[LE]{\thepage \enspace \textsl{\leftmark \enspace - \enspace \rightmark}}
\makeatother

% Comment the two lines below if you want to print the document
\pagecolor{base16-eighties-dark}
\color{base16-eighties-light}


\begin{document}
\hypersetup{pageanchor=false}
\maketitle
\hypersetup{pageanchor=true}
\begin{fullwidth}
\tableofcontents
\end{fullwidth}

\newpage
\begin{fullwidth}
  \renewcommand{\listtheoremname}{\faBook\ \slshape List of Definitions}
  \listoftheorems[ignoreall,show={defn}]
  \addcontentsline{toc}{chapter}{List of Definitions}
\end{fullwidth}

\newpage 
\begin{fullwidth}
  \renewcommand{\listtheoremname}{\faCoffee\ \slshape List of Theorems}
  \listoftheorems[ignoreall,
    show={axiom,lemma,thm,crly,propo,marginthm,marginpropo,marginlemma,marginaxiom,margincrly}
  ]
  \addcontentsline{toc}{chapter}{List of Theorems}
\end{fullwidth}


\chapter*{Preface}%
\label{chp:preface}
\addcontentsline{toc}{chapter}{Preface}
% chapter preface

The main text used by this note is the optional textbook
recommended by the course instructor, titled Investments.
\cite{bodie2015}
This note also pulls information from lectures.
This note shall also follow the general structure of the textbook,
instead of my usual approach that is a per-lecture basis.

% chapter preface (end)

\chapter{Introduction and Overview}%
\label{chp:introduction_and_overview}
% chapter introduction_and_overview

\paragraph{The Environment of Investing}\label{paragraph:the_environment_of_investing}

\begin{itemize}
  \item \hlnotea{Capital investment} needs funds, which gives rise to capital
    markets.
  \item \hlnotea{Capital markets} exist for a plethora of financial instruments
    that meets the needs of investors and users of capital.
  \item Each of the above instruments, starting with stocks and bonds,
    are created and evolved to respond to the needs of these users.
\end{itemize}

\section{A Short History of Investing}%
\label{sec:a_short_history_of_investing}
% section a_short_history_of_investing

The financial market has gone through some turbulent times,
and increasingly frequent as of recent.
Many of the notable crashes in recent history have been fueled
by irrationality, in that the public has failed to value a commodity
for its intrinsic or `actual' value.

On a related note, according to the course instructor,
when a crash happens in the East,
where new days come earlier, `just like dominoes',
the crash will sweep its way to the West.
In such cases, monetary institutions can brace for impact.

\paragraph{The Crash of 1929 and the Great Depression}
Fortunes were rapidly made by supposedly brilliant investors,
and this did not escape the attention of the media.
The public went into a mania for investing,
which ended up a phenomenal rise in the stock market averages,
which then led to the \hldefn{Crash of 1929}.
This paved way to the \hldefn{Great Depression}.

\paragraph{The One-day Panic in 1987}
The \hldefn{One-day Panic in 1987}, aka \hldefn{Black Monday},
\cite{wiki:blackmonday1987} attracted global attention as well.
However, unlike the Crash of 1929, no economic collapse followed.
This is likely due to more informed decision-making by monetary officials,
and a stronger economic situation.
The financial environment also turned out to be neutral for the year 1987.

This crash, in turn, set stage for the economic and financial boom of the 1990s.

\paragraph{Technology Bubble in the 1990s}
With high pubic interest in the stock market,
and nightly news reports on levels of market indices on various mediums,
the stock market really soared in the 1990s.
The most prominent increase was confined to the technology stage.
Technological companies like Dell Computer and Cisco Systems
grew unbelievably fast.  
However, this growth attracted the attention of unsophisticated investors.

This growth of these companies,
as measured by the \hlnotea{Nasdaq market index},
was described as a ``\hlnotea{bubble}'',
a term used to describe the unwarranted inflation in asset values.
When the bubble finally `burst',
naïve investors lose money in the collapse as they enter the bubbling market
long after the initial gain,
and even experienced investors insisted on staying
due to not wanting to lose out on gains.
This eventually lead to the new millennium `\hlnotea{bear market}'.

\paragraph{The New Millennium `Bear Market'} 
The great `bull market' \sidenote{A market that is optimistic.} in the 1990s
became a `bear market' \sidenote{A market that is pessimistic.}
for the new millennium.
However, the decline ended in October 2002,
at a level of more than 50\% below the all-time high in 1999.

\paragraph{Our `Trustworthy' Banks}
A new bubble came along in the first decade of the new millennium,
this time in real assets over financial ones, particularly in commodities
such as copper, oil and many food staples, and especially in \hlnotec{real estate}.
In the West, due to miscalculations of the \hlnotea{Federal Reserve} about
a risk of deflation, interest rates were pegged at high values
and mortgages soared.
As a result, mortgage officers and banks complied with
individuals with poor credit risks, allowing these individuals to finance
homes that are supposed to be unaffordable to them.

The mortgages that were issued were then resold by banks through
\hlnotea{mortgage-backed securities},
and these banks also devised instruments with credit-backed obligations
and circulated them.
Investors of various scales traded these securities without understanding
the risk that entails, much of it due to poor credit risks in the market.
By the end of 2007, the cracks appeared, and it was revealed
that many of the world's largest banks,
who were deeply entangled in these instruments,
were effectively bankrupt.
The main concentration of these holdings was in the United States,
the British, and the European banks; the Canadian banks,
who also participated in the action earlier on,
have largely divested themselves of the credit instruments by the collapse.

% section a_short_history_of_investing (end)

\section{The Economy and Investment}%
\label{sec:the_economy_and_investment}
% section the_economy_and_investment

When investors trade stocks, it is usually traded with another investor,
who has the opposite idea of what the value of the company is ---
buyers think that the value of the stock is higher than the share price,
while the sellers think that the value of the stock is lower.

The price in the market is important
in establishing a fair valuation of the shares.
This is relevant to the corporation when it issues new shares,
when it requires new capital.
The company itself is usually uninvolved in the trading;
the company is usually only interested in knowing what the trade price
says about the sentiment of investors about its financial prospects.

Shares in companies are used for companies to raise funds,
so that they can expand and purchase physical assets.
Investors have extra capital that companies need, and generally so
because individuals have more funds than required for immediate needs.

To obtain capital, those with a deficit issue \hlnotea{securities},
which are bought by those with excess funds.
We shall see a more formal introduction to securities later on,
but we shall talk about stocks and bonds here, which are issued by
private corporations.

\hlnotea{Bonds} are notes that acknowledge indebtedness and specify
the terms of repayment; \hlnotea{stocks} are instruments that convey
ownership rights to their holders,
which no guarantee of any fixed, or even positive, return.

Stocks allow investors to participate in business activities
while away from the drawbacks of individual ownership and partnership.
They are relatively liquid, in that investors can fairly quickly
extract the true value of the shares.
Shares also offer limited liability, so that the greatest loss possible
is the investment itself, in the case where the business goes out.

\subsection{Real Investment VS Financial Investment}%
\label{sub:real_investment_vs_financial_investment}
% subsection real_investment_vs_financial_investment

\begin{defn}[Financial Investment]\index{Financial Investment}\label{defn:financial_investment}
  The investment of individuals in stocks and bonds of corporations
  is known as \hlnoteb{financial investment}.
\end{defn}

\begin{defn}[Real Investment]\index{Real Investment}\label{defn:real_investment}
  A \hlnoteb{real investment}is when a corporation takes capital
  and invests it in productive assets.
\end{defn}

\begin{remark}
  \begin{enumerate}
    \item Financial investment occurs as investors enter the securities markets
      and exchange cash for financial instruments.
      In this case, since only cash is exchanged between investors,
      no new capital reaches the corporations,
      and so no real investment occurs.
    \item Real investments can be reinvestment profits,
      but major real investment requires issues of new debt or equity
      instruments.
  \end{enumerate}
\end{remark}

\begin{defn}[Real Assets]\index{Real Assets}\label{defn:real_assets}
  \hlnoteb{Real assets} determine the productive capacity of the economy.
\end{defn}

\begin{remark}
  Real investments are channeled into real assets.
\end{remark}

\begin{eg}
  Real assets can be:
  \begin{itemize}
    \item land;
    \item buildings;
    \item machines;
    \item knowledge necessary to produce goods; and
    \item workers.
  \end{itemize}
\end{eg}

\begin{defn}[Financial Assets]\index{Financial Assets}\label{defn:financial_assets}
  \hlnoteb{Financial assets} are related to financial investments,
  such as stocks and bonds.
\end{defn}

\begin{remark}
  \begin{enumerate}
    \item Financial assets do not represent a society's wealth.
      \begin{itemize}
        \item Shares of stock represent only ownership rights to assets,
          not the productive capacity of the economy.
      \end{itemize}
    \item Financial assets contribute to the productive capacity of the economy
      \hlnotec{indirectly}; it allows for separation of the ownership and
      management of the firm and facilitate the transfer of funds to
      enterprises.
      \begin{itemize}
        \item When real assets used by a firm generate income,
          the income is allocated to investors by their ownership of the
          financial assets, usually by the percentage that they hold.
        \item Bondholders, for instance, get a flow of income
          based on the \hlnotea{interest rate} and \hlnotea{par value}
          of the bond.
        \item Equityholders and stockholders get any residual income
          after bondholders and other creditors are paid.
      \end{itemize}
    \item Financial assets contribute to the wealth of individuals or firms
      that holds them, since they are claims on the income generated by
      the real assets or on the income of the corporation that issues
      the instruments.
    \item Real assets are income-generating assets;
      financial assets are the allocation of income or wealth among investors.
      In a sense, financial assets can be viewed as a means by which
      individuals hold their claims on real assets.
  \end{enumerate}
\end{remark}

% subsection real_investment_vs_financial_investment (end)

\subsection{Role of Financial Assets and Markets}%
\label{sub:role_of_financial_assets_and_markets}
% subsection role_of_financial_assets_and_markets

Financial assets and the markets where they are traded
play several roles that ensure the `efficient' allocation
of capital to real assets in the economy.

\paragraph{Informational Role} Stock prices reflect a \hlnotee{collective assessment}
of the investors of the current performance and future prospects of a firm.
\begin{itemize}
  \item When the market is more optimistic about the firm,
    share prices of the firm will rise.
  \item Higher prices eases the raising of capital for the firm,
    which then further encourages investment.
\end{itemize}

In this manner, stock prices serve to allocate capital in market economies,
directing capital to firms (and other applications) with
the `greatest \hlnotec{perceived}' potential.
However, this is not the absolute most `efficient' way of allocating capital.
Time and again, we have seen how this allocation goes to places where they
are not in the best interest of the market, or simply ended up in failure
and loss of money (e.g. the dot-com bubble).

This, however, is not to say that this model of our financial market is chosen
arbitrarily. In fact, alternatives such as having a central planner or
letting politicians make these decisions certainly have the same pitfalls.
The willfulness of the market is cannot and will not be determined
by a small group of people, in some sense.

\paragraph{Consumption Timing} One does not have to immediately,
or any time in the near future, consume their earnings in the presence
of financial assets. We can, in a sense, `store' our wealth in financial assets,
so that we may then `consume' them at a later time.

An example use-case is to save for retirement, where we are old and unable
to generate income as we did when we are younger.
During our youth, where we can make high-earnings, we can invest your savings
in financial assets.
Once we are old, we can then sell these assets to provide funds for consumption
needs, or continue trading for bonds for a more reliable stream of income.

In other words, financial markets allow us to \hlnotee{separate decisions
concerning current consumption from constraits that otherwise
would be imposed by current earnings}.

\paragraph{Allocation of Risk} Often, the most important decision to be made when
creating an \hlnotea{investment portfolio} is the choosing of a \hlnotea{risky asset}.
Higher risks usually gives higher return in investments,
and so the above decision is not uncommon.
To soften the risk, we typically \hlnotea{diversify} our investment
by buying some other assets, typically those that are less risky.
This process of investing in a risky investment while making sure that
one does not crash too hard in the event of a failed investment is called
an allocation of risk, and financial assets allow this practice.

\paragraph{Separation of Ownership and Management}
We see that many large corporations are not owner-operated.
Often, corporate executives are selected by boards of directors,
who oversee the management of the firm with respect to the interest
of the actual owners --- the shareholders. \sidenote{Certainly, some of the
shareholders can also be executives or directors themselves.}
This guarantees a level of stability to the firm; for instance,
if the stockholders decide that they no longer wish
to hold ownership of the firm,
selling their ownership away has no impact on the management of the firm.

\begin{defn}[The Agency Problem]\index{Agency Problem}\label{defn:the_agency_problem}
  The possible substitution of personal interest for those of the owners
  is known as \hlnoteb{the agency problem}.
\end{defn}

In our setting, the managers themselves are the agents that are supposed
to act in the interest of the shareholders.
However, there is a risk, perhaps a moral one, that the manager may not
act as they are supposed to.

Several mechanisms have appeared to quell this problem:
\begin{enumerate}
  \item Tying the income of managers to the success of the firm through
    compensation plans.
    \begin{itemize}
      \item This typically comes in the form of stock options.
        \marginnote{Overuse of options causes its own agency problem.
        It creates a moral hazard for managers to engage in risky projects,
        or manipulate information to inflate stock prices, only to cash out
        and leave the company before prices return to normal levels.}
    \end{itemize}
  \item Boards of directors force out under-performing management teams.
  \item Outsiders such as security analysts or mutual funds and pension funds
    monitor the firm, putting pressure on the management.
  \item Bad performers are subject to the threat of a takeover.
    \begin{itemize}
      \item (\hlnotea{Internal takeover}) Unsatisfied shareholders
        can launch a \hlnotea{proxy contest} to take control of the firm.
        \sidenote{However, this threat is usually minimal. Statistically,
        most proxy fights have failed.}
      \item (\hlnotea{External takeover}) An under-performing firm is at risk
        of being acquired by another firm that may wish to take out its
        competitors, or diversify their business by replacing management
        with their own.
    \end{itemize}
\end{enumerate}

% subsection role_of_financial_assets_and_markets (end)

% section the_economy_and_investment (end)

% chapter introduction_and_overview (end)

\appendix

\backmatter

\fancyhead[LE]{\thepage \enspace \textsl{\leftmark}}

% \nobibliography*
\bibliography{references}

\printindex

\end{document}
% vim:tw=80:fdm=syntax
