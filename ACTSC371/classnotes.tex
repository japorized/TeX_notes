% !TEX TS-program = pdflatex
\documentclass[notoc,notitlepage]{tufte-book}
% \nonstopmode % uncomment to enable nonstopmode

\usepackage{classnotetitle}

\title{ACTSC371 --- Introduction to Investments}
\author{Johnson Ng}
\subtitle{Class notes for Fall 2019}
\credentials{BMath (Hons), Pure Mathematics major, Actuarial Science Minor}
\institution{University of Waterloo}

\setcounter{secnumdepth}{3}
\setcounter{tocdepth}{3}

\renewcommand{\baselinestretch}{1.2}
\usepackage{geometry}
\geometry{letterpaper}
\usepackage[parfill]{parskip}
\usepackage{graphicx}

% Essential Packages
\usepackage{makeidx}
\makeindex
\usepackage{enumitem}
\usepackage[T1]{fontenc}
\usepackage{natbib}
\bibliographystyle{apalike}
\usepackage{ragged2e}
\usepackage{etoolbox}
\usepackage{amssymb}
\usepackage{eso-pic}
\usepackage[fixed]{fontawesome5}
\usepackage{todonotes}
\usepackage{apptools, chngcntr}
\usepackage{amsmath}
\usepackage{mathrsfs}
\usepackage{stmaryrd}
\usepackage{mathtools}
\usepackage{tocloft}
\usepackage{tocbibind}
\usepackage{xparse}
\usepackage{tkz-euclide}
\usetkzobj{all}
\usepackage[utf8]{inputenc}
\usepackage{csquotes}
\usepackage[english]{babel}
\usepackage{marvosym}
\usepackage{pgf,tikz}
\usepackage{tikz-cd}
\usepackage{ifthen}
\usepackage{pgfplots}
\usepackage{fancyhdr}
\usepackage{array}
\usepackage{float}
\usepackage{xcolor}
\usepackage{soul}
\usepackage{centernot}
\usepackage{silence}
  \WarningFilter*{latex}{Marginpar on page \thepage\space moved}
\usepackage{tcolorbox}
\tcbuselibrary{skins,breakable}
\usepackage{longtable,booktabs}
\usepackage[amsmath,hyperref,thmmarks]{ntheorem}
\usepackage{thmtools}
\usepackage{hyperref}
\usepackage[noabbrev,capitalize,nameinlink]{cleveref}

\newcommand{\personalcolor}{false}
\ifthenelse{\equal{\personalcolor}{true}}{
  \usepackage{colorscheme-chaos}
}{
  \usepackage{colorscheme-student}
}

% hyperref Package Settings
\hypersetup{
    unicode=true,          % non-Latin characters in Acrobat’s bookmarks
    pdftoolbar=false,        % show Acrobat’s toolbar?
    pdfmenubar=false,        % show Acrobat’s menu?
    pdffitwindow=true,     % window fit to page when opened
    colorlinks=true,
    allcolors=magenta,
}

% tikz
\usepgfplotslibrary{polar}
\usepgflibrary{shapes.geometric}
\usetikzlibrary{angles,patterns,calc,decorations.markings,arrows.meta,tikzmark,bending}
\tikzset{midarrow/.style 2 args={
        decoration={markings,
            mark= at position #2 with {\arrow{#1}} ,
        },
        postaction={decorate}
    },
    midarrow/.default={latex}{0.5}
}
\def\centerarc[#1](#2)(#3:#4:#5)% Syntax: [draw options] (center) (initial angle:final angle:radius)
    { \draw[#1] ($(#2)+({#5*cos(#3)},{#5*sin(#3)})$) arc (#3:#4:#5); }
% from https://tex.stackexchange.com/questions/67573/tikz-shift-and-rotate-in-3d
\newcommand{\rotateRPY}[4][0/0/0]% point to be saved to \savedxyz, roll, pitch, yaw
{   \pgfmathsetmacro{\rollangle}{#2}
    \pgfmathsetmacro{\pitchangle}{#3}
    \pgfmathsetmacro{\yawangle}{#4}

    % to what vector is the x unit vector transformed, and which 2D vector is this?
    \pgfmathsetmacro{\newxx}{cos(\yawangle)*cos(\pitchangle)}% a
    \pgfmathsetmacro{\newxy}{sin(\yawangle)*cos(\pitchangle)}% d
    \pgfmathsetmacro{\newxz}{-sin(\pitchangle)}% g
    \path (\newxx,\newxy,\newxz);
    \pgfgetlastxy{\nxx}{\nxy};

    % to what vector is the y unit vector transformed, and which 2D vector is this?
    \pgfmathsetmacro{\newyx}{cos(\yawangle)*sin(\pitchangle)*sin(\rollangle)-sin(\yawangle)*cos(\rollangle)}% b
    \pgfmathsetmacro{\newyy}{sin(\yawangle)*sin(\pitchangle)*sin(\rollangle)+ cos(\yawangle)*cos(\rollangle)}% e
    \pgfmathsetmacro{\newyz}{cos(\pitchangle)*sin(\rollangle)}% h
    \path (\newyx,\newyy,\newyz);
    \pgfgetlastxy{\nyx}{\nyy};

    % to what vector is the z unit vector transformed, and which 2D vector is this?
    \pgfmathsetmacro{\newzx}{cos(\yawangle)*sin(\pitchangle)*cos(\rollangle)+ sin(\yawangle)*sin(\rollangle)}
    \pgfmathsetmacro{\newzy}{sin(\yawangle)*sin(\pitchangle)*cos(\rollangle)-cos(\yawangle)*sin(\rollangle)}
    \pgfmathsetmacro{\newzz}{cos(\pitchangle)*cos(\rollangle)}
    \path (\newzx,\newzy,\newzz);
    \pgfgetlastxy{\nzx}{\nzy};

    % transform the point given by #1
    \foreach \x/\y/\z in {#1}
    {   \pgfmathsetmacro{\transformedx}{\x*\newxx+\y*\newyx+\z*\newzx}
        \pgfmathsetmacro{\transformedy}{\x*\newxy+\y*\newyy+\z*\newzy}
        \pgfmathsetmacro{\transformedz}{\x*\newxz+\y*\newyz+\z*\newzz}
        \xdef\savedx{\transformedx}
        \xdef\savedy{\transformedy}
        \xdef\savedz{\transformedz}     
    }
}
\tikzset{RPY/.style={x={(\nxx,\nxy)},y={(\nyx,\nyy)},z={(\nzx,\nzy)}}}
\newcommand{\AxisRotator}[1][rotate=0]{%
    \tikz [x=0.25cm,y=0.60cm,line width=.2ex,-stealth,#1] \draw (0,0) arc (-150:150:1 and 1);%
  }

% enumitems
\newlist{inlinelist}{enumerate*}{1}
\setlist*[inlinelist,1]{%
  label=(\roman*),
}

% Theorem Style Customization
\setlength\theorempreskipamount{2ex}
\setlength\theorempostskipamount{3ex}

\makeatletter
\let\nobreakitem\item
\let\@nobreakitem\@item
\patchcmd{\nobreakitem}{\@item}{\@nobreakitem}{}{}
\patchcmd{\nobreakitem}{\@item}{\@nobreakitem}{}{}
\patchcmd{\@nobreakitem}{\@itempenalty}{\@M}{}{}
\patchcmd{\@xthm}{\ignorespaces}{\nobreak\ignorespaces}{}{}
\patchcmd{\@ythm}{\ignorespaces}{\nobreak\ignorespaces}{}{}

\renewtheoremstyle{break}%
  {\item{\theorem@headerfont
          ##1\ ##2\theorem@separator}\hskip\labelsep\relax\nobreakitem}%
  {\item{\theorem@headerfont
          ##1\ ##2\ (##3)\theorem@separator}\hskip\labelsep\relax\nobreakitem}
\makeatother

% ntheorem Declarations
\theorempreskip{10pt}
\theorempostskip{5pt}
\theoremstyle{break}

\theoremsymbol{\faComment}
\newtheorem{remark}{Remark}[section]
\theoremsymbol{}
\newtheorem*{strategy}{\faPaperPlane Strategy}
\newtheorem*{procedure}{\faCodeBranch\ }
\newtheorem{ex}{Exercise}[section]
\theorembodyfont{\normalfont}
\newtheorem*{solution}{\faPencil* Solution}
\theoremsymbol{\faGavel}
\newtheorem{eg}{Example}[section]
\theoremsymbol{}
\theorembodyfont{\it}

    % definition env
\theoremprework{\textcolor{blue}{\hrule height 2pt width \textwidth}}
\theoremheaderfont{\color{blue}\normalfont\bfseries}
\theorempostwork{\textcolor{blue}{\hrule height 2pt width \textwidth}}
\theoremindent10pt
\newtheorem{defn}{\faBook Definition}

    % definition env no num
\theoremprework{\textcolor{blue}{\hrule height 2pt width \textwidth}}
\theoremheaderfont{\color{blue}\normalfont\bfseries}
\theorempostwork{\textcolor{blue}{\hrule height 2pt width \textwidth}}
\theoremindent10pt
\newtheorem*{defnnonum}{\faBook Definition}

\theoremprework{\textcolor{blue}{\hrule height 2pt width \marginparwidth}}
\theoremheaderfont{\color{blue}\normalfont\bfseries}
\theorempostwork{\textcolor{blue}{\hrule height 2pt width \marginparwidth}}
\theoremindent10pt
\newtheorem{margindefn}[defn]{\faBook Definition}

\theoremprework{\textcolor{blue}{\hrule height 2pt width \textwidth}}
\theoremheaderfont{\color{blue}\normalfont\bfseries}
\theorempostwork{\textcolor{blue}{\hrule height 2pt width \textwidth}}
\theoremindent10pt
\newtheorem*{margindefnnonum}{\faBook Definition}

    % theorem envs
\theoremprework{\textcolor{magenta}{\hrule height 2pt width \textwidth}}
\theoremheaderfont{\color{magenta}\normalfont\bfseries}
\theorempostwork{\textcolor{magenta}{\hrule height 2pt width \textwidth}}
\theoremindent10pt
\newtheorem{thm}{\faCoffee Theorem}

\theoremprework{\textcolor{magenta}{\hrule height 2pt width \textwidth}}
\theorempostwork{\textcolor{magenta}{\hrule height 2pt width \textwidth}}
\theoremindent10pt
\newtheorem{propo}[thm]{\faTint Proposition}

\theoremprework{\textcolor{magenta}{\hrule height 2pt width \textwidth}}
\theorempostwork{\textcolor{magenta}{\hrule height 2pt width \textwidth}}
\theoremindent10pt
\newtheorem{crly}[thm]{\faSpaceShuttle Corollary}

\theoremprework{\textcolor{magenta}{\hrule height 2pt width \textwidth}}
\theorempostwork{\textcolor{magenta}{\hrule height 2pt width \textwidth}}
\theoremindent10pt
\newtheorem{lemma}[thm]{\faTree Lemma}

\theoremprework{\textcolor{magenta}{\hrule height 2pt width \textwidth}}
\theorempostwork{\textcolor{magenta}{\hrule height 2pt width \textwidth}}
\theoremindent10pt
\newtheorem{axiom}[thm]{\faShield Axiom}

    % theorem envs without counter
\theoremprework{\textcolor{magenta}{\hrule height 2pt width \textwidth}}
\theoremheaderfont{\color{magenta}\normalfont\bfseries}
\theorempostwork{\textcolor{magenta}{\hrule height 2pt width \textwidth}}
\theoremindent10pt
\newtheorem*{thmnonum}{\faCoffee Theorem}

\theoremprework{\textcolor{magenta}{\hrule height 2pt width \textwidth}}
\theorempostwork{\textcolor{magenta}{\hrule height 2pt width \textwidth}}
\theoremindent10pt
\newtheorem*{propononum}{\faTint Proposition}

\theoremprework{\textcolor{magenta}{\hrule height 2pt width \textwidth}}
\theorempostwork{\textcolor{magenta}{\hrule height 2pt width \textwidth}}
\theoremindent10pt
\newtheorem*{crlynonum}{\faSpaceShuttle Corollary}

\theoremprework{\textcolor{magenta}{\hrule height 2pt width \textwidth}}
\theorempostwork{\textcolor{magenta}{\hrule height 2pt width \textwidth}}
\theoremindent10pt
\newtheorem*{lemmanonum}{\faTree Lemma}

\theoremprework{\textcolor{magenta}{\hrule height 2pt width \textwidth}}
\theorempostwork{\textcolor{magenta}{\hrule height 2pt width \textwidth}}
\theoremindent10pt
\newtheorem*{axiomnonum}{\faShield Axiom}

    % envs on margins
\theoremprework{\textcolor{magenta}{\hrule height 2pt width \marginparwidth}}
\theoremheaderfont{\color{magenta}\normalfont\bfseries}
\theorempostwork{\textcolor{magenta}{\hrule height 2pt width \marginparwidth}}
\theoremindent10pt
\newtheorem{marginthm}[thm]{\faCoffee Theorem}

\theoremprework{\textcolor{magenta}{\hrule height 2pt width \marginparwidth}}
\theorempostwork{\textcolor{magenta}{\hrule height 2pt width \marginparwidth}}
\theoremindent10pt
\newtheorem{marginpropo}[thm]{\faTint Proposition}

\theoremprework{\textcolor{magenta}{\hrule height 2pt width \marginparwidth}}
\theorempostwork{\textcolor{magenta}{\hrule height 2pt width \marginparwidth}}
\theoremindent10pt
\newtheorem{margincrly}[thm]{\faSpaceShuttle Corollary}

\theoremprework{\textcolor{magenta}{\hrule height 2pt width \marginparwidth}}
\theorempostwork{\textcolor{magenta}{\hrule height 2pt width \marginparwidth}}
\theoremindent10pt
\newtheorem{marginlemma}[thm]{\faTree Lemma}

\theoremprework{\textcolor{magenta}{\hrule height 2pt width \marginparwidth}}
\theorempostwork{\textcolor{magenta}{\hrule height 2pt width \marginparwidth}}
\theoremindent10pt
\newtheorem{marginaxiom}[thm]{\faShield Axiom}

    % envs on margins without counter
\theoremprework{\textcolor{magenta}{\hrule height 2pt width \marginparwidth}}
\theoremheaderfont{\color{magenta}\normalfont\bfseries}
\theorempostwork{\textcolor{magenta}{\hrule height 2pt width \marginparwidth}}
\theoremindent10pt
\newtheorem*{marginthmnonum}{\faCoffee Theorem}

\theoremprework{\textcolor{magenta}{\hrule height 2pt width \marginparwidth}}
\theorempostwork{\textcolor{magenta}{\hrule height 2pt width \marginparwidth}}
\theoremindent10pt
\newtheorem*{marginpropononum}{\faTint Proposition}

\theoremprework{\textcolor{magenta}{\hrule height 2pt width \marginparwidth}}
\theorempostwork{\textcolor{magenta}{\hrule height 2pt width \marginparwidth}}
\theoremindent10pt
\newtheorem*{margincrlynonum}{\faSpaceShuttle Corollary}

\theoremprework{\textcolor{magenta}{\hrule height 2pt width \marginparwidth}}
\theorempostwork{\textcolor{magenta}{\hrule height 2pt width \marginparwidth}}
\theoremindent10pt
\newtheorem*{marginlemmanonum}{\faTree Lemma}

\theoremprework{\textcolor{magenta}{\hrule height 2pt width \marginparwidth}}
\theorempostwork{\textcolor{magenta}{\hrule height 2pt width \marginparwidth}}
\theoremindent10pt
\newtheorem*{marginaxiomnonum}{\faShield Axiom}

    % proof env
\theoremprework{\textcolor{green}{\hrule height 2pt width \textwidth}}
\theorembodyfont{\normalfont}
\theoremheaderfont{\color{green}\normalfont\bfseries}
\theorempostwork{\textcolor{green}{\hrule height 2pt width \textwidth}}
\theoremsymbol{\ensuremath{_\square}}
\newtheorem*{proof}{\faPencil* Proof}
\theoremsymbol{}

\theoremprework{\textcolor{green}{\hrule height 2pt width \marginparwidth}}
\theorembodyfont{\normalfont}
\theoremheaderfont{\color{green}\normalfont\bfseries}
\theorempostwork{\textcolor{green}{\hrule height 2pt width \marginparwidth}}
\theoremsymbol{\ensuremath{_\square}}
\newtheorem*{mproof}{\faPencil* Proof}
\theoremsymbol{}

    % note and notation env
\theorembodyfont{\it}

\theoremprework{\textcolor{yellow}{\hrule height 2pt width \textwidth}}
\theoremheaderfont{\color{yellow}\normalfont\bfseries}
\theorempostwork{\textcolor{yellow}{\hrule height 2pt width \textwidth}}
\newtheorem{note}{\faQuoteLeft Note}[section]

\theoremprework{\textcolor{yellow}{\hrule height 2pt width \marginparwidth}}
\theoremheaderfont{\color{yellow}\normalfont\bfseries}
\theorempostwork{\textcolor{yellow}{\hrule height 2pt width \marginparwidth}}
\newtheorem{mnote}[note]{\faQuoteLeft Note}

\theoremprework{\textcolor{yellow}{\hrule height 2pt width \textwidth}}
\theorempostwork{\textcolor{yellow}{\hrule height 2pt width \textwidth}}
\newtheorem*{notation}{\faPaw Notation}

    % warning env
\theoremprework{\textcolor{red}{\hrule height 2pt width \textwidth}}
\theoremheaderfont{\color{red}\normalfont\bfseries}
\theorempostwork{\textcolor{red}{\hrule height 2pt width \textwidth}}
\theoremindent10pt
\newtheorem*{warning}{\faBug Warning}

\theoremprework{\textcolor{red}{\hrule height 2pt width \marginparwidth}}
\theoremheaderfont{\color{red}\normalfont\bfseries}
\theorempostwork{\textcolor{red}{\hrule height 2pt width \marginparwidth}}
\theoremindent10pt
\newtheorem*{marginwarning}{\faBug Warning}

% rule for appendices
\AtAppendix{\counterwithin{defn}{chapter}}
\AtAppendix{\counterwithin{thm}{chapter}}
\AtAppendix{\counterwithin{propo}{chapter}}
\AtAppendix{\counterwithin{lemma}{chapter}}
\AtAppendix{\counterwithin{crly}{chapter}}
\AtAppendix{\counterwithin{axiom}{chapter}}

% more environments
\newtcolorbox{quotebox}[2]{
  blanker,enhanced,breakable,standard jigsaw,
  opacityback=0,
  coltext=\ifblank{#2}{black}{#2},
  left=5mm,right=5mm,top=2mm,bottom=2mm,
  colframe=\ifblank{#1}{bblack}{#1},
  boxrule=0pt,leftrule=3pt,
  fontupper=\itshape
}

\providecommand{\parthook}{}
\patchcmd{\part}{\thispagestyle}{\parthook\thispagestyle}{}{}
\newcommand{\partimage}[2][]{% \parthook[<options>]{<image>}
  \renewcommand{\parthook}{% Update \parthook
    \AddToShipoutPictureBG*{% Add picture to background of THIS page only
      \AtPageLowerLeft{\includegraphics[width=\paperwidth,height=\paperheight,#1]{#2}}}% Insert image
    \renewcommand{\parthook}{}}}% Restore \parthook

\AtBeginDocument{\renewcommand\contentsname{\slshape Table of Contents\normalfont}}
\cftpagenumbersoff{part}

\newcommand{\tuftepart}[1]{\newgeometry{}\part{#1}\restoregeometry}

% Heading formattings
% chapter format
\titleformat{\chapter}%
  {\huge\rmfamily\itshape\color{magenta}}% format applied to label+text
  {\llap{\colorbox{magenta}{\parbox[c][1cm]{3cm}{\hfill\itshape\Huge\textcolor{background}{\thechapter}}}}}% label
  {5pt}% horizontal separation between label and title body
  {\faLeaf}% before the title body
  []% after the title body

% section format
\titleformat{\section}%
  {\normalfont\Large\rmfamily\itshape\color{blue}}% format applied to label+text
  {\llap{\colorbox{blue}{\parbox{3cm}{\hfill\itshape\textcolor{background}{\thesection}}}}}% label
  {5pt}% horizontal separation between label and title body
  {}% before the title body
  []% after the title body

% subsection format
\titleformat{\subsection}%
  {\normalfont\large\itshape\color{green}}% format applied to label+text
  {\llap{\colorbox{green}{\parbox{3cm}{\hfill\textcolor{background}{\thesubsection}}}}}% label
  {1em}% horizontal separation between label and title body
  {}% before the title body
  []% after the title body

% subsubsection format
\titleclass{\subsubsection}{straight}
\titleformat{\subsubsection}%
  {\normalfont\large\itshape\color{yellow}}% format applied to label+text
  {\llap{\colorbox{yellow}{\parbox{3cm}{\hfill\textcolor{background}{\thesubsubsection}}}}}% label
  {1em}% horizontal separation between label and title body
  {}% before the title body
  []% after the title body

% Sidenote enhancements
\def\mathmarginnote#1{%
  \tag*{\rlap{\hspace\marginparsep\smash{\parbox[t]{\marginparwidth}{%
  \footnotesize#1}}}}
}

% Custom table columning
\newcolumntype{L}[1]{>{\raggedright\let\newline\\\arraybackslash\hspace{0pt}}m{#1}}
\newcolumntype{C}[1]{>{\centering\let\newline\\\arraybackslash\hspace{0pt}}m{#1}}
\newcolumntype{R}[1]{>{\raggedleft\let\newline\\\arraybackslash\hspace{0pt}}m{#1}}

% Graph styles
\pgfplotsset{compat=1.15}
\usepgfplotslibrary{fillbetween}
\pgfplotsset{four quads/.append style={axis x line=middle, axis y line=
middle, xlabel={$x$}, ylabel={$y$}, axis equal }}
\pgfplotsset{four quad complex/.append style={axis x line=middle, axis y line=
middle, xlabel={$\re$}, ylabel={$\im$}, axis equal }}
\def\axisdefaultwidth{360pt}
\pgfplotsset{
  tufteaxis/.append style = {thick},tick style = {thick,black},
  %
  % #1 = x, y, or z
  % #2 = the shift value
  /tikz/normal shift/.code 2 args = {%
    \pgftransformshift{%
        \pgfpointscale{#2}{\pgfplotspointouternormalvectorofticklabelaxis{#1}}%
    }%
  },%
  %
  range3frame/.style = {
    tick align        = outside,
    scaled ticks      = false,
    enlargelimits     = false,
    ticklabel shift   = {10pt},
    axis lines*       = left,
    line cap          = round,
    clip              = false,
    xtick style       = {normal shift={x}{10pt}},
    ytick style       = {normal shift={y}{10pt}},
    ztick style       = {normal shift={z}{10pt}},
    x axis line style = {normal shift={x}{10pt}},
    y axis line style = {normal shift={y}{10pt}},
    z axis line style = {normal shift={z}{10pt}},
  }
}

% Shortcuts
\DeclareMathOperator{\id}{id}
\DeclareMathOperator{\Img}{Img}
\DeclareMathOperator{\Res}{Res}
\DeclareMathOperator*{\argmax}{arg\,max}
\DeclareMathOperator*{\argmin}{arg\,min}
\DeclareMathOperator{\re}{Re}
\DeclareMathOperator{\im}{Im}
\DeclareMathOperator{\caparg}{Arg}
\DeclareMathOperator{\Char}{Char}
\DeclareMathOperator{\sgn}{sgn}
\DeclareMathOperator{\Range}{range}

\newcommand{\floor}[1]{\lfloor #1 \rfloor}      % simplifying the writing of a floor function
\newcommand{\ceiling}[1]{\lceil #1 \rceil}      % simplifying the writing of a ceiling function
\newcommand{\dotp}{\, \cdotp}			        % dot product to distinguish from \cdot
\newcommand{\abs}[1]{\left|#1\right|}						% absolute value
\newcommand{\lra}[1]{\left\langle \; #1 \; \right\rangle}
\newcommand{\at}[2]{\Big|_{#1}^{#2}}
\newcommand{\Arg}[1]{\caparg #1}
\renewcommand{\bar}[1]{\mkern 1.5mu \overline{\mkern -1.5mu #1 \mkern -1.5mu} \mkern 1.5mu}
\newcommand{\faktor}[2]{{\raisebox{.2em}{$#1$}\left/\raisebox{-.2em}{$#2$}\right.}}
\newcommand{\quotient}[2]{\faktor{#1}{#2}}
\newcommand{\cyclic}[1]{\left\langle #1 \right\rangle}
\newcommand{\ind}[2]{\Ind_{#2}\left( #1 \right)}
\newcommand{\notimply}{\centernot\implies}
\newcommand{\res}[2]{\underset{#2}{\Res} #1 }
\newcommand{\tworow}[3]{\begin{tabular}{@{}#1@{}} #2 \\ #3 \end{tabular}}
\renewcommand{\epsilon}{\varepsilon}
\renewcommand{\phi}{\varphi}
\newcommand{\lrarrow}{\leftrightarrow}
\newcommand{\larrow}{\leftarrow}
\newcommand{\rarrow}{\rightarrow}
\renewcommand{\atop}[2]{\genfrac{}{}{0pt}{}{#1}{#2}}
\newcommand*\dif{\mathop{}\!d}
\newcommand{\mmid}{\; \middle| \;}
\newcommand{\coprime}{\; \bot \;}
\newcommand{\norm}[1]{\left\| #1 \right\|}
\newenvironment{spmatrix}
  {\left(\begin{smallmatrix}}
  {\end{smallmatrix}\right)}

  % inspiration from: https://tex.stackexchange.com/questions/8720/overbrace-underbrace-but-with-an-arrow-instead#37758
\newcommand{\overarrow}[2]{
  \overset{\makebox[0pt]{\begin{tabular}{@{}c@{}}#2\\[0pt]\ensuremath{\uparrow}\end{tabular}}}{\ensuremath{#1}}
}
\newcommand{\underarrow}[2]{
  \underset{\makebox[0pt]{\begin{tabular}{@{}c@{}}\downarrow\\[0pt]\ensuremath{#2}\end{tabular}}}{\ensuremath{#1}}
}


	% highlighting shortcuts
\newcommand{\hlimpo}[1]{\textcolor{red}{\textbf{#1}}}
\newcommand{\hlwarn}[1]{\textcolor{yellow}{\textbf{#1}}}
\newcommand{\hldefn}[1]{\textcolor{blue}{\index{#1}\textbf{#1}}}
\newcommand{\hlnotea}[1]{\textcolor{green}{\textbf{#1}}}
\newcommand{\hlnoteb}[1]{\textcolor{cyan}{\textbf{#1}}}
\newcommand{\hlb}[2]{\colorbox{#1!30!background}{#2}}
\newcommand{\hlbnotea}[1]{\hlb{green}{#1}}
\newcommand{\hlbnoteb}[1]{\hlb{cyan}{#1}}
\newcommand{\hlbnotec}[1]{\hlb{yellow}{#1}}
\newcommand{\hlbnoted}[1]{\hlb{magenta}{#1}}
\newcommand{\hlbnotee}[1]{\hlb{red}{#1}}
\newcommand{\WTP}{\textcolor{bwhite}{WTP} }
\newcommand{\WTS}{\textcolor{bwhite}{WTS} }

  % stars on important stuff
\newcommand{\imponote}{\faStar}
\newcommand{\vimponote}{\faStar\faStar}
\newcommand{\vvimponote}{\faStar\faStar\faStar}

% Document header formatting
\makeatletter
\pagestyle{fancy}
\fancyhead{}
\fancyhead[RO]{\textsl{\@title} \enspace \thepage}
\fancyhead[LE]{\thepage \enspace \textsl{\leftmark \enspace \rightmark}}
\makeatother
\renewcommand{\chaptermark}[1]{\markboth{#1}{}}
\renewcommand{\sectionmark}[1]{\markright{#1}}

% Comment the two lines below if you want to print the document
\pagecolor{background}
\color{foreground}


\DeclareMathOperator{\BEY}{BEY}
\DeclareMathOperator{\BD}{BD}
\DeclareMathOperator{\eff}{eff}

\begin{document}
\hypersetup{pageanchor=false}
\maketitle
\hypersetup{pageanchor=true}
\begin{fullwidth}
\tableofcontents
\end{fullwidth}

\newpage
\begin{fullwidth}
  \renewcommand{\listtheoremname}{\faBook\ \slshape List of Definitions}
  \listoftheorems[ignoreall,show={defn}]
\end{fullwidth}

\newpage 
\begin{fullwidth}
  \renewcommand{\listtheoremname}{\faCoffee\ \slshape List of Theorems}
  \listoftheorems[ignoreall,
    show={axiom,lemma,thm,crly,propo,marginthm,marginpropo,marginlemma,marginaxiom,margincrly}
  ]
\end{fullwidth}

\newpage
\begin{fullwidth}
  \renewcommand{\listtheoremname}{\faCodeBranch\ \slshape List of Procedures}
  \listoftheorems[ignoreall, show={procedure}]
\end{fullwidth}


\chapter*{Preface}%
\label{chp:preface}
\addcontentsline{toc}{chapter}{Preface}
% chapter preface

The main text used by this note is the optional textbook
recommended by the course instructor, titled Investments.
\cite{bodie2015}
This note also pulls information from lectures.
This note shall also follow the general structure of the textbook,
instead of my usual approach that is a per-lecture basis.

Thus far, I find that this textbook does give a pretty long lecture
for its introduction. Perhaps it tried its best in actually
covering as much ground but not going to deep, so that the reader
can have a feel of the contents that are to come.
However, it also feels like the authors are rather... distracted
in their writing. It is rather difficult to keep track
of what the authors are trying to convey. For instance,
paragraphs that are meant to introduce reasons for a concept
ends up giving an example about how the concept comes into play.

% chapter preface (end)

\chapter{Introduction and Overview}%
\label{chp:introduction_and_overview}
% chapter introduction_and_overview

\paragraph{The Environment of Investing}\label{paragraph:the_environment_of_investing}

\begin{itemize}
  \item \hlnotea{Capital investment} needs funds, which gives rise to capital
    markets.
  \item \hlnotea{Capital markets} exist for a plethora of financial instruments
    that meets the needs of investors and users of capital.
  \item Each of the above instruments, starting with stocks and bonds,
    are created and evolved to respond to the needs of these users.
\end{itemize}

\section{A Short History of Investing}%
\label{sec:a_short_history_of_investing}
% section a_short_history_of_investing

The financial market has gone through some turbulent times,
and increasingly frequent as of recent.
Many of the notable crashes in recent history have been fueled
by irrationality, in that the public has failed to value a commodity
for its intrinsic or `actual' value.

On a related note, according to the course instructor,
when a crash happens in the East,
where new days come earlier, `just like dominoes',
the crash will sweep its way to the West.
In such cases, monetary institutions can brace for impact.

\paragraph{The Crash of 1929 and the Great Depression}
Fortunes were rapidly made by supposedly brilliant investors,
and this did not escape the attention of the media.
The public went into a mania for investing,
which ended up a phenomenal rise in the stock market averages,
which then led to the \hldefn{Crash of 1929}.
This paved way to the \hldefn{Great Depression}.

\paragraph{The One-day Panic in 1987}
The \hldefn{One-day Panic in 1987}, aka \hldefn{Black Monday},
\cite{wiki:blackmonday1987} attracted global attention as well.
However, unlike the Crash of 1929, no economic collapse followed.
This is likely due to more informed decision-making by monetary officials,
and a stronger economic situation.
The financial environment also turned out to be neutral for the year 1987.

This crash, in turn, set stage for the economic and financial boom of the 1990s.

\paragraph{Technology Bubble in the 1990s}
With high pubic interest in the stock market,
and nightly news reports on levels of market indices on various mediums,
the stock market really soared in the 1990s.
The most prominent increase was confined to the technology stage.
Technological companies like Dell Computer and Cisco Systems
grew unbelievably fast.  
However, this growth attracted the attention of unsophisticated investors.

This growth of these companies,
as measured by the \hlnotea{Nasdaq market index},
was described as a ``\hlnotea{bubble}'',
a term used to describe the unwarranted inflation in asset values.
When the bubble finally `burst',
naïve investors lose money in the collapse as they enter the bubbling market
long after the initial gain,
and even experienced investors insisted on staying
due to not wanting to lose out on gains.
This eventually lead to the new millennium `\hlnotea{bear market}'.

\paragraph{The New Millennium `Bear Market'} 
The great `bull market' \sidenote{A market that is optimistic.} in the 1990s
became a `bear market' \sidenote{A market that is pessimistic.}
for the new millennium.
However, the decline ended in October 2002,
at a level of more than 50\% below the all-time high in 1999.

\paragraph{Our `Trustworthy' Banks}
A new bubble came along in the first decade of the new millennium,
this time in real assets over financial ones, particularly in commodities
such as copper, oil and many food staples, and especially in \hlnotec{real estate}.
In the West, due to miscalculations of the \hlnotea{Federal Reserve} about
a risk of deflation, interest rates were pegged at high values
and mortgages soared.
As a result, mortgage officers and banks complied with
individuals with poor credit risks, allowing these individuals to finance
homes that are supposed to be unaffordable to them.

The mortgages that were issued were then resold by banks through
\hlnotea{mortgage-backed securities},
and these banks also devised instruments with credit-backed obligations
and circulated them.
Investors of various scales traded these securities without understanding
the risk that entails, much of it due to poor credit risks in the market.
By the end of 2007, the cracks appeared, and it was revealed
that many of the world's largest banks,
who were deeply entangled in these instruments,
were effectively bankrupt.
The main concentration of these holdings was in the United States,
the British, and the European banks; the Canadian banks,
who also participated in the action earlier on,
have largely divested themselves of the credit instruments by the collapse.

% section a_short_history_of_investing (end)

\section{The Economy and Investment}%
\label{sec:the_economy_and_investment}
% section the_economy_and_investment

When investors trade stocks, it is usually traded with another investor,
who has the opposite idea of what the value of the company is ---
buyers think that the value of the stock is higher than the share price,
while the sellers think that the value of the stock is lower.

The price in the market is important
in establishing a fair valuation of the shares.
This is relevant to the corporation when it issues new shares,
when it requires new capital.
The company itself is usually uninvolved in the trading;
the company is usually only interested in knowing what the trade price
says about the sentiment of investors about its financial prospects.

Shares in companies are used for companies to raise funds,
so that they can expand and purchase physical assets.
Investors have extra capital that companies need, and generally so
because individuals have more funds than required for immediate needs.

To obtain capital, those with a deficit issue \hlnotea{securities},
which are bought by those with excess funds.
We shall see a more formal introduction to securities later on,
but we shall talk about stocks and bonds here, which are issued by
private corporations.

\hlnotea{Bonds} are notes that acknowledge indebtedness and specify
the terms of repayment; \hlnotea{stocks} are instruments that convey
ownership rights to their holders,
which no guarantee of any fixed, or even positive, return.

Stocks allow investors to participate in business activities
while away from the drawbacks of individual ownership and partnership.
They are relatively liquid, i.e. investors can fairly quickly
extract the true value of the shares.
Shares also offer limited liability, so that the greatest loss possible
is the investment itself, in the case where the business goes out.

\subsection{Real Investment VS Financial Investment}%
\label{sub:real_investment_vs_financial_investment}
% subsection real_investment_vs_financial_investment

\begin{defn}[Financial Investment]\index{Financial Investment}\label{defn:financial_investment}
  The investment of individuals in stocks and bonds of corporations
  is known as \hlnoteb{financial investment}.
\end{defn}

\begin{defn}[Real Investment]\index{Real Investment}\label{defn:real_investment}
  A \hlnoteb{real investment}is when a corporation takes capital
  and invests it in productive assets.
\end{defn}

\begin{remark}
  \begin{enumerate}
    \item Financial investment occurs as investors enter the securities markets
      and exchange cash for financial instruments.
      In this case, since only cash is exchanged between investors,
      no new capital reaches the corporations,
      and so no real investment occurs.
    \item Real investments can be reinvestment profits,
      but major real investment requires issues of new debt or equity
      instruments.
  \end{enumerate}
\end{remark}

\begin{defn}[Real Assets]\index{Real Assets}\label{defn:real_assets}
  \hlnoteb{Real assets} determine the productive capacity of the economy.
\end{defn}

\begin{remark}
  Real investments are channeled into real assets.
\end{remark}

\begin{eg}
  Real assets can be:
  \begin{itemize}
    \item land;
    \item buildings;
    \item machines;
    \item knowledge necessary to produce goods; and
    \item workers.
  \end{itemize}
\end{eg}

\begin{defn}[Financial Assets]\index{Financial Assets}\label{defn:financial_assets}
  \hlnoteb{Financial assets} are related to financial investments,
  such as stocks and bonds.
\end{defn}

\begin{remark}
  \begin{enumerate}
    \item Financial assets do not represent a society's wealth.
      \begin{itemize}
        \item Shares of stock represent only ownership rights to assets,
          not the productive capacity of the economy.
      \end{itemize}
    \item Financial assets contribute to the productive capacity of the economy
      \hlnotec{indirectly}; it allows for separation of the ownership and
      management of the firm and facilitate the transfer of funds to
      enterprises.
      \begin{itemize}
        \item When real assets used by a firm generate income,
          the income is allocated to investors by their ownership of the
          financial assets, usually by the percentage that they hold.
        \item Bondholders, for instance, get a flow of income
          based on the \hlnotea{interest rate} and \hlnotea{par value}
          of the bond.
        \item Equityholders and stockholders get any residual income
          after bondholders and other creditors are paid.
      \end{itemize}
    \item Financial assets contribute to the wealth of individuals or firms
      that holds them, since they are claims on the income generated by
      the real assets or on the income of the corporation that issues
      the instruments.
    \item Real assets are income-generating assets;
      financial assets are the allocation of income or wealth among investors.
      In a sense, financial assets can be viewed as a means by which
      individuals hold their claims on real assets.
  \end{enumerate}
\end{remark}

% subsection real_investment_vs_financial_investment (end)

\subsection{Role of Financial Assets and Markets}%
\label{sub:role_of_financial_assets_and_markets}
% subsection role_of_financial_assets_and_markets

Financial assets and the markets where they are traded
play several roles that ensure the `efficient' allocation
of capital to real assets in the economy.

\paragraph{Informational Role} Stock prices reflect a \hlnotee{collective assessment}
of the investors of the current performance and future prospects of a firm.
\begin{itemize}
  \item When the market is more optimistic about the firm,
    share prices of the firm will rise.
  \item Higher prices eases the raising of capital for the firm,
    which then further encourages investment.
\end{itemize}

In this manner, stock prices serve to allocate capital in market economies,
directing capital to firms (and other applications) with
the `greatest \hlnotec{perceived}' potential.
However, this is not the absolute most `efficient' way of allocating capital.
Time and again, we have seen how this allocation goes to places where they
are not in the best interest of the market, or simply ended up in failure
and loss of money (e.g. the dot-com bubble).

This, however, is not to say that this model of our financial market is chosen
arbitrarily. In fact, alternatives such as having a central planner or
letting politicians make these decisions certainly have the same pitfalls.
The willfulness of the market is cannot and will not be determined
by a small group of people, in some sense.

\paragraph{Consumption Timing} One does not have to immediately,
or any time in the near future, consume their earnings in the presence
of financial assets. We can, in a sense, `store' our wealth in financial assets,
so that we may then `consume' them at a later time.

An example use-case is to save for retirement, where we are old and unable
to generate income as we did when we are younger.
During our youth, where we can make high-earnings, we can invest your savings
in financial assets.
Once we are old, we can then sell these assets to provide funds for consumption
needs, or continue trading for bonds for a more reliable stream of income.

In other words, financial markets allow us to \hlnotee{separate decisions
concerning current consumption from constraits that otherwise
would be imposed by current earnings}.

\paragraph{Allocation of Risk} Often, the most important decision to be made when
creating an \hlnotea{investment portfolio} is the choosing of a \hlnotea{risky asset}.
Higher risks usually gives higher return in investments,
and so the above decision is not uncommon.
To soften the risk, we typically \hlnotea{diversify} our investment
by buying some other assets, typically those that are less risky.
This process of investing in a risky investment while making sure that
one does not crash too hard in the event of a failed investment is called
an allocation of risk, and financial assets allow this practice.

\paragraph{Separation of Ownership and Management}
We see that many large corporations are not owner-operated.
Often, corporate executives are selected by boards of directors,
who oversee the management of the firm with respect to the interest
of the actual owners --- the shareholders. \sidenote{Certainly, some of the
shareholders can also be executives or directors themselves.}
This guarantees a level of stability to the firm; for instance,
if the stockholders decide that they no longer wish
to hold ownership of the firm,
selling their ownership away has no impact on the management of the firm.

\begin{defn}[The Agency Problem]\index{Agency Problem}\label{defn:the_agency_problem}
  The possible substitution of personal interest for those of the owners
  is known as \hlnoteb{the agency problem}.
\end{defn}

In our setting, the managers themselves are the agents that are supposed
to act in the interest of the shareholders.
However, there is a risk, perhaps a moral one, that the manager may not
act as they are supposed to.

Several mechanisms have appeared to quell this problem:
\begin{enumerate}
  \item Tying the income of managers to the success of the firm through
    compensation plans.
    \begin{itemize}
      \item This typically comes in the form of stock options.
        \marginnote{Overuse of options causes its own agency problem.
        It creates a moral hazard for managers to engage in risky projects,
        or manipulate information to inflate stock prices, only to cash out
        and leave the company before prices return to normal levels.}
    \end{itemize}
  \item Boards of directors force out under-performing management teams.
  \item Outsiders such as security analysts or mutual funds and pension funds
    monitor the firm, putting pressure on the management.
  \item Bad performers are subject to the threat of a takeover.
    \begin{itemize}
      \item (\hlnotea{Internal takeover}) Unsatisfied shareholders
        can launch a \hlnotea{proxy contest} to take control of the firm.
        \sidenote{However, this threat is usually minimal. Statistically,
        most proxy fights have failed.}
      \item (\hlnotea{External takeover}) An under-performing firm is at risk
        of being acquired by another firm that may wish to take out its
        competitors, or diversify their business by replacing management
        with their own.
    \end{itemize}
\end{enumerate}

% subsection role_of_financial_assets_and_markets (end)

% section the_economy_and_investment (end)

\section{Participants in the Financial Market: Individuals and Financial Intermediaries}%
\label{sec:participants_in_the_financial_market_individuals_and_financial_intermediaries}
% section participants_in_the_financial_market_individuals_and_financial_intermediaries

There are essentially 3 types of participants in financial markets:
\begin{enumerate}
  \item \hlnotea{Households}: typically the net suppliers of capital, as savers.
    \begin{itemize}
      \item Purchase securities issued by firms that need to raise funds.
    \end{itemize}
  \item \hlnotea{Firms}: typically net demanders of capital.
    \begin{itemize}
      \item Raise capital now to pay for investment in real assets.
      \item Income generated by real assets provides returns to investors
        who purchased securities issued by the firm.
    \end{itemize}
  \item \hlnotea{Governments}: can be either; depends on relationship between
    tax revenue and government expenditures.
\end{enumerate}

\subsection{Individuals and Financial Objectives}%
\label{sub:individuals_and_financial_objectives}
% subsection individuals_and_financial_objectives

The objective of investing is to make a return on capital.
There are various possibilities to the kind of return expected.
The amount of return, the risk exposure, and the duration to a return
can vary based on the preferences of the investor,
which may vary depending on the stage of the investor's life.

\marginnote{
  \begin{marginwarning}
    It is easy to get confused between \hlnotee{saving}
    and \hlnotee{safe investment}.
    The textbook gives the following example to clarify their differences:
    \begin{quotebox}{magenta}{foreground}
      Suppose you earn $\$100,000$ a year from your job and spend $\$80,000$
      on consumption. You save $\$20,000$.
      Suppose you decide to invest all $\$20,000$ in risky assets.
      Then you are still saving $\$20,000$ but you are not investing it safely.
    \end{quotebox}
  \end{marginwarning}
}

Some investors are content with a \hlnoted{fixed return}
if the principal is guaranteed; others look for opportunities to
\hlnoted{double their investment} in days.
These extremes are not in the interest of this course.

\paragraph{First significant decision for most individuals --- Education}

The major asset that most people have during their earlier working years
is their power to earn by drawing on their human capital.
The risk of a crippling illness or injury is worse than the risk
associated with their financial wealth.
The most direct way of \hlnotea{hedging} this is by buying insurance.

\paragraph{First major economic asset for most individuals --- Personal Residence} 

This requires an evaluation of potential \hlnotea{appreciation} in residential
values in the light of rental expenses.
When considering real estate investment as an option for diversification,
it may be the case that a personal portfolio is overweighted in real estate.

One of the risks associated with this investment is that
if the asset is also meant as a personal residence,
especially for someone who just owned their first house,
the risk of a downturn in their employer's industry, or other related factors,
may force the investor to give up on their house
(in the event of a need to move).

% subsection individuals_and_financial_objectives (end)

\subsection{The Investment Process}%
\label{sub:the_investment_process}
% subsection the_investment_process

\begin{defn}[Portfolio]\index{Portfolio}\label{defn:portfolio}
  A \hlnoteb{portfolio} is a collection of investment assets.
\end{defn}

\begin{remark}
  \begin{enumerate}
    \item A portfolio is updated or ``rebalanced'' by selling existing
      securities,
      using the proceeds to then purchase new securities.
    \item The size of a portfolio can be increased by investing additional
      funds,
      and can be decreased by selling securities.
  \end{enumerate}
\end{remark}

Investment assets can be categorized into broad asset classes,
e.g. stocks bonds, real estate, commodities, etc.

Investors make 2 types of decisions when constructing their portfolios:
\begin{enumerate}
  \item (\hldefn{Asset allocation}) the decision to choose among the asset
    classes;
  \item (\hldefn{Security selection}) the decision to choose
    which particular securities to hold.
\end{enumerate}

The following are 2 approaches to constructing a portfolio:
\begin{enumerate}
  \item \hldefn{Top-down portfolio}
    \begin{itemize}
      \item Begins with asset allocation.
      \item After deciding on assets, the investor turns to
        security selection.
    \end{itemize}
  \item \hldefn{Bottom-up portfolio}
    \begin{itemize}
      \item The portfolio is constructed first from buying
        securities that seem attractively priced without
        much concern for the resultant asset allocation.
      \item This approach tends result in unintended bets
        in one or another sector of the economy.
    \end{itemize}
\end{enumerate}

A technique that is used for selecting securities is called
\hlnotea{security analysis}, which involves valuation of particular securities
that are being considered for a portfolio.

% subsection the_investment_process (end)

\subsection{Financial Intermediaries}%
\label{sub:financial_intermediaries}
% subsection financial_intermediaries

\hlnoteb{Financial intermediaries} are institutions that stand
between the security issuers (typically firms), and the
ultimate owner of the security (typically investors).
The problem space that they are in has the following problems:
\begin{enumerate}
  \item Corporations and governments do not directly trade
    their securities with individuals. Consequently,
    they do not market their securities to the public.
  \item The small (financial) size of households makes it difficult
    to do direct investment.
    \begin{itemize}
      \item It is unreasonable to do their own advertisement
        to look for prospective borrowers.
      \item It is difficult for an individual investor to diversify
        across borrowers to reduce risk.
      \item Dedicated time and effort needs to be put into
        assessing and monitoring the credit risk of borrowers and
        the market.
    \end{itemize}
\end{enumerate}
The common theme for both sides of the party is that it is not
beneficial for these parties to fully dedicate their times into
playing in this market, more so if their fields are not directly
involved in the financial market.

\begin{eg}[Banks as a financial intermediary]
  A bank raises funds by ``borrowing'' \sidenote{Banks take deposits
  from its customers.} and lending that money to other borrowers.
  Banks give its creditors a certain interest rate and charge
  its debtors an interest rate that is higher.
  The difference, called the \hlnotea{spread}, is the source
  of the bank's profit.

  Through the bank, the lenders and borrowers need not contact
  each other, nor do they need to know each other. The bank
  acts as an intermediary between the two.
  The problem of matching lenders and borrowers is a non-issue
  when each party approach the bank independently.
\end{eg}

Other examples of financial intermediaries are \hlnotea{investment
companies}, \hlnotea{insurance companies}, and \hlnotea{credit unions}.
They all offer similar services but in differing `flavours'.
Their similarities lie in:
\begin{enumerate}
  \item the pooling of resources, so that they can lend
    considerable amounts to borrowers;
  \item achieve significant diversification through the
    pooling process, allowing them to take on risks
    that are unreasonable on an individual level; and
  \item build expertise through day-to-day business, and
    use economies of scale
    \sidenote{Cost advantages that a firms can obtain due to
    their scale of operation.}
    and scope
    \sidenote{Efficiencies formed by variety, not volume.}
    to develop tools to assess and monitor risks.
\end{enumerate}

\begin{eg}[Investment companies as a financial intermediary]
  The `flavour' of problems that investment companies focus on
  are the following:
  \begin{itemize}
    \item household portfolios are not large enough to be spread
      among a wide variety of securities; and
    \item brokerage fees and research costs are expensive (also
      time-wise) to purchase a few shares of many different firms.
  \end{itemize}

  Investment companies can also design portfolios for larger investors
  with particular goals.
\end{eg}

\begin{eg}[Mutual funds as a tool for a financial intermediary]
  \hldefn{Mutual funds} have the advantage of large-scale trading
  and portfolio management.
  Participating investors are given a \hlnotea{prorated share}
  of the total funds, with respect to the size of their investment.
  This solves many of the small investors' problem, with a management
  fee that they are willing to pay to the mutual fund operator.

  Mutual funds are sold in the \hlnotea{retail market},
  and the strategy of these funds are designed to attract large number
  of clients.
\end{eg}

\begin{eg}[Investment bankers]
  \hldefn{Investment bankers} are born to fill the niche for
  firms that perform specialized services for businesses.
  They serve to offer their expertise in finance to firms
  that would prefer such a service over
  maintaining an in-house security issuance division.
  In Canada, they are also known as \hldefn{investment dealers}.
  Some examples in Canada are Scotia Capital, RBC Investments,
  and BMO Nesbitt.

  These bankers provide advice for corporations
  on the pricing of their securities, interest rates, etc.
  Essentially, the bankers handles the marketing of the security
  in the \hldefn{primary market}. \sidenote{The primary market
  is where new issues of securities are offered to the public.
  The stage where investors trade issued securities among themselves
  is called the \hldefn{secondary market}.}

  Aside from their expertise, investment bankers offers their own
  reputation for honestly on the deal. When investors seek to buy
  securities, the branding of the investment banker serves
  as sort of a credibility check, just as one do when buying products
  as a regular consumer.
\end{eg}

% subsection financial_intermediaries (end)

% section participants_in_the_financial_market_individuals_and_financial_intermediaries (end)

\section{Recent Trends}%
\label{sec:recent_trends}
% section recent_trends

\marginnote{I shall come back to this later on.}
There are 4 important trends in the modern investment environment:
\begin{enumerate}
  \item globalization;
  \item financial engineering;
  \item securitization; and
  \item information and computer networks.
\end{enumerate}

% section recent_trends (end)

% chapter introduction_and_overview (end)

\chapter{Financial Markets and Instruments}%
\label{chp:financial_markets_and_instruments}
% chapter financial_markets_and_instruments

The following a several major classes of financial assets
or securities, of which we shall go into more detail,
but will provided a brief description so that we may proceed:
\begin{enumerate}
  \item Debt instruments
    \begin{itemize}
      \item money market instruments
      \item bonds --- an instrument where the bondholder is the creditor,
        and the payment is usually in terms of interest and a final
        maturity payment
    \end{itemize}
  \item Common stock
    \begin{itemize}
      \item the most common form of stocks, hence the name
      \item common stockholders have voting and ownership rights
        of the company
      \item common stockholders can get paid a dividend
    \end{itemize}
  \item Preferred stock
    \begin{itemize}
      \item a special class of stocks, given preference
        due to order of paying back liabilities of the company
        (see below)
      \item has ownership rights but typically no voting rights
      \item preferred stockholders can get paid a dividend,
        typically before common stockholders
    \end{itemize}
  \item Derivative securities
    \begin{itemize}
      \item instruments that are based on, or derived, from
        issued securities
      \item this is typically traded outside of the firm that
        issued the security of which the derivative is based on
    \end{itemize}
\end{enumerate}

The order of how companies pay their liabilities is:
\begin{center}
  creditors $\to$ preferred stockholders $\to$ common stockholders
\end{center}

\section{The Money Market}%
\label{sec:the_money_market}
% section the_money_market

\begin{itemize}
  \item A sub-sector of the fixed income market.
  \item Consists of instruments, called \hlnotea{money market instruments}
    or \hlnotea{debt securities}, that are
    \begin{itemize}
      \item \hlnotec{very-short-term};
      \item \hlnotec{highly marketable};
      \item \hlnotec{liquid}; and
      \item \hlnotec{low-risk}.
    \end{itemize}
  \item Money market instruments are also sometimes called
    \hlnotea{cash equivalents} due to their safety and liquidity.
  \item Money market investors \hlnoted{aim} to generate
    safe, short-term returns.
  \item Most of these securities trade in large denominations,
    making them out of reach for small investors.
  \item \hlnotea{Money market funds}, on the other hand,
    is accessible to small investors.
\end{itemize}

\subsubsection{Money Market Instruments}%
\label{ssub:money_market_instruments}
% subsubsection money_market_instruments

\paragraph{Treasury Bills (T-bills)}\label{para:t_bills}\index{Treasury bills}\index{T-bills}

\begin{itemize}
  \item Most marketable of all Canadian money market instruments.
  \item Issued by the government to raise money.
  \item Usually issued at a bi-weekly \hlnotec{auction} for maturities of
    \begin{enumerate*}
      \item 1 month;
      \item 2 months;
      \item 3 months;
      \item 6 months; and
      \item 1 year.
    \end{enumerate*}
    \begin{itemize}
      \item Chartered banks and authorized dealers can submit
        only \hlnoteb{competitive bids}.
      \item \hldefn{Competitive bid} --- order for a given quantity of bills
        at a specific offer price.
      \item \hldefn{Non-competitive bid} ---
        unconditional offer to purchase at the average price
        of the successful competitive bid; submittable only for bonds.
      \item Government rank-orders bids by offering price,
        and accepts bids in descending order
        \sidenote{It is sensible to accept bids of the highest price,
        since that means that the government needs to pay back less.}
        until the entire issue is absorbed.
    \end{itemize}
  \item Also traded in the secondary market.
  \item Highly liquid.
  \item \hlnotea{Credit risk} is considered to be close to nil;
    governments can raise taxes to repay debt.
  \item Sold at a discount.
  \item Price is quoted based on \nameref{defn:bond_equivalent_yield}.
\end{itemize}


\paragraph{Certificates of Deposit (CD)}\label{para:certificates_of_deposit}

\begin{defn}[Time Deposit]\index{Time Deposit}\label{defn:time_deposit}
  A \hlnoteb{time deposit} is an interest-bearing deposit account
  with a specific date of maturity, and cannot be withdrawn
  before maturity.
  The deposit bearer will pay interest and the principal
  back to the investor only at maturity.
\end{defn}

\begin{remark}
  \begin{enumerate}
    \item Time deposits are known as `fixed deposits' in a certain country in
      the East.
    \item While time deposits cannot be withdrawn by definition,
      some time deposits allow the investor to still withdraw their money
      but as a sign of cancellation,
      and either the full sum is paid back, or a small fee is charged.
  \end{enumerate}
\end{remark}

\begin{itemize}
  \item CD is a time deposit with a chartered bank.
    \sidenote{A \hldefn{chartered bank} is a bank that has received
    regulatory approval to operate. The specifics of what they can do
    as a chartered bank differs from country to country.}
  \item A similar time deposit for smaller amounts is known as a
    \hldefn{guarantee investment certificate (GIC)}.
  \item Both CDs and GICs are non-transferable in Canada;
    but certain banks allow for transfer for those with great denominations.
  \item Transferable CDs in Canada are known as \hldefn{bearer deposit notes}
    (BDNs).
  \item CDs in the US are marketable.
\end{itemize}

\paragraph{Commercial Paper}

\begin{itemize}
  \item Short-term unsecured debt notes.
  \item Typically issued by large, well-known companies.
  \item Often backed by a bank line of credit.
    \sidenote{A \hldefn{line of credit} is an arrangement between a
    financial institution and a customer, and it establishes the maximum
    amount that the customer can borrow. A bank line of credit is a line
    of credit where the financial institution is the bank.}
  \item Range: $\leq 1$ year; longer maturities require registration under
    the \hlnotea{Ontario Securities Act} and thus almost never used.
    \begin{itemize}
      \item Most often issued with less than 1 or 2 months;
        with minimum denominations of $\$50,000$.
      \item This means that small investors can only access commercial papers
        indirectly, via \hlnotea{money market mutual funds}.
    \end{itemize}
  \item Almost all commercial papers are now rated for credit quality
    by rating agencies.
    \sidenote{\hldefn{Rating agencies} are agencies that rate a debtor's ability
    to pay back principals and interests in a timely manner, and their
    likelihood of defaulting.}
  \item Considered a fairly safe asset due to short term.
    \begin{itemize}
      \item If investors doubt the credit-worthiness of the firm,
        the firm may be forced to pick other methods of financing,
        which are most costly. (E.g.
        \href{https://en.wikipedia.org/wiki/Olympia_and_York}{Olympia and York
        -- March 1992})
    \end{itemize}
  \item Financial firms like banks have also started issuing commercial papers,
    usually called \hldefn{asset-backed commercial paper}.
    \begin{itemize}
      \item These are short-term instruments used to raise funds,
        for the institution to invest in other assets.
      \item These assets are collateral for the commercial paper --- hence
        the namesake.
      \item A notorious example of where the investment goes was into subprime
        mortgages, where we saw in summer of 2007.
    \end{itemize}
\end{itemize}

\paragraph{Bankers' Acceptance}\index{Bankers' Acceptance}\label{para:bankers_acceptance}

\begin{itemize}
  \item Is an order to a bank, by a bank's client, to pay an amount to a holder
    at a future date.
    \begin{itemize}
      \item Bank endorses the order for payment as ``accepted'', hence the name.
      \item Bank assumes responsibility for eventual payment.
      \item This makes Bankers' Acceptances second only to T-bills
        in terms of security.
    \end{itemize}
  \item Typically issued for 6 months.
  \item Once issued, Bankers' Acceptances are tradable in the secondary market.
    \begin{itemize}
      \item Traders may substitute the bank's credit standing for their own.
      \item Sold at discount of the face value of the payment order; hence
        yield is calculated similarly.
    \end{itemize}
\end{itemize}

\paragraph{Eurodollars}\index{Eurodollars}\label{para:eurodollars}

\begin{itemize}
  \item US dollar-denominated deposits at foreign banks or
    foreign branches of US banks.
    \begin{itemize}
      \item Not under regulation of the US Federal Reserve Board.
      \item Despite the name, Eurodollars need not be based on
        European banks, but it is where Eurodollars began.
    \end{itemize}
  \item Most Eurodollar deposits are large.
  \item Most Eurodollars are time deposits with less than 6 months' maturity.
  \item The \textbf{\textcolor{blue}{Eurodollar}}
    \hyperref[para:certificates_of_deposit]
    {certificate of deposit}\index{Eurodollar certificate of deposit}
    is a variation of the Eurodollar time deposit.
    \begin{itemize}
      \item Tradable, which a typical time deposit is not.
      \item Considered less liquid and riskier than US CDs;
        hence higher yields.
    \end{itemize}
  \item \hldefn{Eurodollar bonds} --- dollar-denominated bonds in Europe.
    \begin{itemize}
      \item \hlnotec{Not} a money market instrument due to long maturities.
    \end{itemize}
  \item All above Eurodollar-related instruments also exist denominated
    in all major currencies.
    \begin{itemize}
      \item They are called \hldefn{Eurocurrency instruments}; e.g.
        they are called Euro-Canadian dollars in Canadian Dollars.
      \item Constitute a minor portion of the Eurocurrency market,
        which is dominated by Eurodollars.
    \end{itemize}
\end{itemize}

\paragraph{Repos and Reverse Repos}\index{Repos}\index{Reverse Repos}\index{Repurchase agreements}\index{RP}\label{para:repos_and_reverse_repos}

\begin{itemize}
  \item Are short-term, typically overnight, borrowing.
  \item Frequently used by dealers in government securities.
  \item Sold to an \hlnotea{institutional investor} as collateral,
    with the promise to buyback the next day
    with an \hlnotea{overnight interest}.
  \item Considered very safe; backed by government securities.
  \item A \hldefn{term repo} is a similar instrument,
    except that the duration of the agreement can be 30 days or more.
  \item A \hldefn{reverse repo} is when
    the dealer finds an investor holding government securities and buys them,
    with the agreement to sell them back.
\end{itemize}

\paragraph{Federal Funds}\index{Federal Funds}\label{para:federal_funds}

\begin{itemize}
  \item Aka \hldefn{fed funds}.
  \item A minimum balance in a reserve account that a member bank in the US
    is required to have with the Federal Reserve.
  \item Banks with excess funds lend to those with a shortage at the
    so-called \hldefn{federal funds rate}, usually as overnight transactions.
  \item Fed funds market was meant to help banks meet the minimum requirement,
    but many large banks consider the fund as a component of their total
    funding.
  \item The fed fund rate is recognized as a \hlnotee{key barometer} of monetary
    policy.
\end{itemize}

\paragraph{Brokers' Call Loans}\index{Brokers' Call Loans}\label{para:brokers_call_loans}

\begin{itemize}
  \item Individuals who buy stocks on margin
    \sidenote{This means that the investor borrows some money to buy a stock.}
    borrow part of the funds from their broker.
    The broker may borrow the funds from a bank,
    agreeing to pay back on call.
  \item Chartered banks make these call loans on investment firms
    to finance their inventory of securities.
  \item Interest rate on these loans is usually closely related to
    the rate of short-term T-bills.
\end{itemize}

\paragraph{The LIBOR Market}\index{LIBOR}\label{para:libor} 

\begin{itemize}
  \item Full name: London Interbank Offered Rate.
  \item Is the rate at which large banks in London are willing to lend each
    other money.
  \item Became the premier short-term interest rate quoted in the European money
    market.
  \item Used as a reference rate for many transactions. (e.g. LIBOR $+ 2\%$)
  \item May be tied to currencies other than USD;
    e.g. British pound, yen, euros.
  \item \hldefn{EURIBOR} (\hldefn{European Interbank Offered Rate})
    is a similar rate at which banks in the Eurozone are willing to lend
    euros among themselves.
\end{itemize}

% subsubsection money_market_instruments (end)

\subsubsection{Yields on Money Market Instruments}%
\label{ssub:yields_on_money_market_instruments}
% subsubsection yields_on_money_market_instruments

\begin{defn}[Par Value / Face Value]\index{Par Value}\index{Face Value}\label{defn:par_value}\label{defn:face_value}
  Given a T-bill, the \hlnoteb{par value} (or \hlnoteb{face value})
  of the T-bill is the value of the T-bill at maturity.
\end{defn}

\begin{defn}[Bond Equivalent Yield (BEY)]\index{Bond Equivalent Yield}\index{BEY}\label{defn:bond_equivalent_yield}
  The \hlnoteb{Bond Equivalent Yield (BEY)}, denoted as $r_{\BEY}$,
  is a calculation for restating semi-annual,
  quarterly or monthly discount T-bill into an annual yield.
  In particular, there are 2 standards that are used
  in the market to calculate $r_{\BEY}$.
  Let $F$ be the par value, $P$ be the selling price
  of the T-bill, and $n$ the maturity of the T-bill in days of a year.
  In Canada, we use the \hldefn{exact method}, which states that
  \begin{equation*}
    r_{\BEY} = \frac{F - P}{P} \times \frac{365}{n}.
  \end{equation*}

  In the US, the BEY is known as the \hldefn{Bond Discount Rate} (BD),
  denoted by $r_{\BD}$, where
  \begin{equation*}
    r_{\BD} = \frac{F - P}{F} \times \frac{360}{n}.
  \end{equation*}
\end{defn}

\begin{remark}
  \begin{enumerate}
    \item Perhaps the best way to think about the naming is to
      think of the name as being archaic; ``bonds'' were issued for 1 year
      in the old days. These calculations were devised
      to calculate the rate of a similar instrument with a shorter period,
      in a way so that they may be comparable to the one-year T-bill.
    \item The BEY is \hlnotec{not} an accurate measure
      of the effective rate of return.
      This is because the BEY uses simple interest
      rather than compounded interest.
      The next example illustrates this.
    \item The BEY is also known as a \hldefn{annual percentage rate} (APR),
      since is uses simple interest procedure to \hlnotea{annualize},
      instead of compound interest.
      \sidenote{In the case of compound interest, we call such a rate
      the \hldefn{annual percentage yield}, also known, perhaps more
      commonly to us, as the \hldefn{effective annual rate}.}
  \end{enumerate}
\end{remark}

\begin{eg}[BEY is not an accurate measure of the effective rate of return]
  Consider a $\$1,000$ T-bill priced at $\$960$ now,
  maturing in 6 months.
  Then the effective rate of interest is
  \begin{equation*}
    \left(1 + \frac{1000 - 960}{960}\right)^2 - 1 \approx 0.0851 = 8.51\%.
  \end{equation*}
  On the other hand, the BEY is
  \begin{equation*}
    r_{\BEY} = \frac{1000 - 960}{960} \cdot \frac{365}{182}
          \approx 8.356\%.
  \end{equation*}
\end{eg}

\begin{propo}[Relationship between BEY and BD]\label{propo:relationship_between_bey_and_bd}
  Let $F$ be the par value, $P$ be the selling price of the T-bill,
  and $n$ the maturity of the T-bill in days of a year.
  Then
  \begin{equation*}
    r_{\BEY} = \frac{365 \cdot r_{\BD}}{360 - r_{\BD} \cdot n}.
  \end{equation*}
\end{propo}

\begin{proof}
  Note that
  \begin{equation*}
    r_{\BEY} = \frac{F - P}{P} \cdot \frac{365}{n}
  \end{equation*}
  and
  \begin{equation*}
    r_{\BD} = \frac{F - P}{F} \cdot \frac{360}{n}.
  \end{equation*}
  Equating the two by $\frac{F - P}{n}$, we get
  \begin{equation*}
    \frac{r_{\BEY} \cdot P}{365} = \frac{r_{\BD} \cdot F}{360},
  \end{equation*}
  which thus
  \begin{equation*}
    r_{\BEY} = \frac{r_{\BD} \cdot F \cdot 365}{360 \cdot P}.
  \end{equation*}
  Notice that by the formula of $r_{\BD}$, we have
  \begin{equation*}
    P = F \left( 1 - r_{\BD} \cdot \frac{n}{360} \right).
  \end{equation*}
  It follows that
  \begin{equation*}
    r_{\BEY} = \frac{r_{\BD} \cdot \cancel{F} \cdot 365}
            {360 \cdot \cancel{F} \left( 1 - r_{\BD} \frac{n}{360} \right)}
            = \frac{365 \cdot r_{\BD}}{360 - r_{\BD} \cdot n}.
  \end{equation*}
\end{proof}

\begin{propo}[Relationship between the Effective Annual Rate and the BEY]\label{propo:relationship_between_the_effective_annual_rate_and_the_bey}
  Let $r_{\eff}$ be the effective annual interest rate.
  Then
  \begin{equation*}
    r_{\eff} = \left( 1 + \frac{r_{\BEY}}{365 / n} \right)^{\frac{365}{n}} - 1,
  \end{equation*}
  where $n$ is the number of days to maturity of the T-bill
  at which $r_{\BEY}$ is based upon.
\end{propo}

\begin{proof}
  First, notice that if we let $t$ be the period at which the bill is in place,
  and if we label the compound interest rate as $i_c$ and the simple interest
  rate as $i_s$, then we can equate
  \begin{equation*}
    (1 + i_c)^t = (1 + i_s t)
  \end{equation*}
  to find one interest rate from the other.
  It is important to note that since the period of both sides are the same,
  the period at which interest is applied are the same.
  In particular, since $r_{\BEY}$ is a type of annual simple interest,
  it follows that $i_c = r_{\eff}$, and if we pick $t = \frac{n}{365}$, then
  \begin{equation*}
    (1 + r_{\eff})^{\frac{n}{365}} = (1 + \frac{r_{\BEY}}{365 / n})
  \end{equation*}
  and so
  \begin{equation*}
    r_{\eff} = \left( 1 + \frac{r_{\BEY}}{365 / n} \right)^{\frac{365}{n}} - 1
  \end{equation*}
  as required.
\end{proof}

\begin{remark}
  \begin{enumerate}
    \item By the definition of the \nameref{defn:bond_equivalent_yield},
      we observe that
      \begin{equation*}
        \frac{r_{\BEY}}{365 / n} = \frac{F - P}{P},
      \end{equation*}
      where $F$ is the face value of the T-bill and $P$ its purchase price.
      Thus
      \begin{equation*}
        (1 + r_{\eff})^{\frac{n}{365}} = 1 + \frac{F - P}{P},
      \end{equation*}
      which also agrees with our notion of the effective annual rate.
    \item Since $r_{\BEY}$ is an annualized rate, we also call
      $\frac{r_{\BEY}}{365 / n}$ a `deannualized' rate.
  \end{enumerate}
\end{remark}

% subsubsection yields_on_money_market_instruments (end)

% section the_money_market (end)

\section{The Bond Market}%
\label{sec:the_bond_market}
% section the_bond_market

\begin{itemize}
  \item Composed of borrowing instruments that have terms longer than those
    traded in the money market.
  \item These borrowing instruments are either called \hldefn{debt instruments}
    or \hldefn{bonds}.
    \begin{itemize}
      \item Most of these instruments promise either
        \begin{itemize}
          \item a fixed stream of income; or
          \item a stream of income determined according to some formula.
        \end{itemize}
      \item Because of the above, they are also said
        to make up the \hldefn{fixed-income capital market}.
        \begin{itemize}
          \item In practice, the flow of income may not be fixed,
            thus the term ``fixed income'' is somewhat inappropriate.
        \end{itemize}
    \end{itemize}
\end{itemize}

\begin{eg}
  The following are some examples of:
  \begin{itemize}
    \item Publicly issued instruments:
      \begin{itemize*}
        \item Canada bonds
        \item Provincial government bonds
        \item Municipal bonds
      \end{itemize*}
    \item Privately issued instruments:
      \begin{itemize*}
        \item Corporate bonds
        \item International Bonds
        \item Mortgage-backed securities
      \end{itemize*}
  \end{itemize}
\end{eg}

\paragraph{Government of Canada Bonds}\label{para:government_of_canada_bonds}

\begin{itemize}
  \item The Canadian government issues non-marketable and marketable
    debt securities to borrow funds.
  \item Non-marketable debt securities
    \begin{itemize}
      \item Both the following examples are
        \begin{itemize}
          \item issued every year starting November 1
          \item sale period may be a few months
          \item has little interest rate risk
        \end{itemize}
      \item \hldefn{Canada Savings Bonds} (CSBs) ---
        \begin{itemize}
          \item can be cashed any time prior to maturity at
            face value plus accrued interest
          \item valuation is complex due to redemption feature
        \end{itemize}
      \item \hldefn{Canada Permium Bonds} (CPBs) ---
        \begin{itemize}
          \item can only be cashed in November of succeeding years
          \item interest rate rises if the realized holding period is longer
        \end{itemize}
    \end{itemize}
  \item Marketable debt securities
    \begin{itemize}
      \item \hldefn{Government of Canada Bonds} or \hldefn{Canadas}
        or \hldefn{Canada bonds} ---
        \begin{itemize}
          \item varying maturities at issue; range up to 40 years
          \item considered part of the money market when their
            term becomes less than 3 years
          \item generally noncallable
          \item make semi-annual coupon payments
          \item competitive coupon rates to ensure their issue at or near par
            value
          \item there are listings that show the prices, periods and yields
            of these bonds
          \item more detailed listings may also include
            \begin{itemize}
              \item the change in yield and/or price from previous
                \hlnotea{close}
              \item \hldefn{bid} and \hldefn{ask} prices,
                which respectively means the price at which
                an investor can sell or buy the asset on the market
            \end{itemize}
        \end{itemize}
    \end{itemize}
\end{itemize}

% section the_bond_market (end)

% chapter financial_markets_and_instruments (end)

\appendix

\backmatter

\fancyhead[LE]{\thepage \enspace \textsl{\leftmark}}

% \nobibliography*
\bibliography{references}

\printindex

\end{document}
% vim:tw=80:fdm=syntax
