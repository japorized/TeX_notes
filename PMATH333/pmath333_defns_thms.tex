% Document Head
\documentclass[11pt, oneside]{book}
\usepackage{geometry}
\geometry{letterpaper}
\usepackage[parfill]{parskip}
\usepackage{graphicx}

% Essential Packages
\usepackage{ragged2e}
\usepackage{amssymb}
\usepackage{amsmath}
\usepackage{mathrsfs}
\usepackage[utf8]{inputenc}
\usepackage[english]{babel}
\usepackage[hyperref]{ntheorem}

% Theorem Style Customization
\setlength\theorempreskipamount{2ex}
\setlength\theorempostskipamount{3ex}

% hyperref Package Settings
\usepackage{hyperref}
\hypersetup{
	colorlinks = true,
	linkcolor = magenta
}

% ntheorem Declarations
\theoremstyle{break}
\newtheorem{thm}{Theorem}[section]
\newtheorem*{proof}{Proof}
\newtheorem{crly}{Corollary}[thm]
\newtheorem{lemma}[thm]{Lemma}
\newtheorem{propo}{Proposition}[section]
\newtheorem*{remark}{Remark}
\newtheorem*{note}{Note}
\newtheorem{defn}{Definition}[section]
\newtheorem{eg}{Example}[section]

% ntheorem listtheorem style
\makeatletter
\def\thm@@thmline@name#1#2#3#4{%
        \@dottedtocline{-2}{0em}{2.3em}%
                   {\makebox[\widesttheorem][l]{#1 \protect\numberline{#2}}#3}%
                   {#4}}
\@ifpackageloaded{hyperref}{
\def\thm@@thmline@name#1#2#3#4#5{%
    \ifx\\#5\\%
        \@dottedtocline{-2}{0em}{2.3em}%
            {\makebox[\widesttheorem][l]{#1 \protect\numberline{#2}}#3}%
            {#4}
    \else
        \ifHy@linktocpage\relax\relax
            \@dottedtocline{-2}{0em}{2.3em}%
                {\makebox[\widesttheorem][l]{#1 \protect\numberline{#2}}#3}%
                {\hyper@linkstart{link}{#5}{#4}\hyper@linkend}%
        \else
            \@dottedtocline{-2}{0em}{2.3em}%
                {\hyper@linkstart{link}{#5}%
                  {\makebox[\widesttheorem][l]{#1 \protect\numberline{#2}}#3}\hyper@linkend}%
                    {#4}%
        \fi
    \fi}
}
\makeatother
\newlength\widesttheorem
\AtBeginDocument{
  \settowidth{\widesttheorem}{Proposition A.1.1.1\quad}
}

\theoremlisttype{allname}

% Shortcuts
\newcommand{\bb}[1]{\mathbb{#1}}		% using bb instead of mathbb

% Main Body
\title{UW W17 PMATH333 - Definitions and Theorems}
\author{Johnson Ng}

\begin{document}
\maketitle
\tableofcontents

\chapter*{List of Definitions}
\theoremlisttype{allname}
\listtheorems{defn}

\chapter*{List of Theorems}
\theoremlisttype{allname}
\listtheorems{lemma,thm,crly,propo}

\appendix

\chapter{ZF Set Theory and the Axiom of Choice}
\section{Introduction}
\begin{eg}[Russel's Paradox]\label{russel_paradox}
	Let X be the set of all sets, and let $S = \{ A \in X | A \notin A\}$.\\
	Note for example that $Z \notin Z \implies Z \in S$, and $X \in X \implies X \notin S$.\\
	Thus we have $S \in S \iff S \notin S$.
\end{eg}

To ensure that mathematical paradoxes (like the above) can no longer arise, mathematicians considered the following questions, and with these questions, rough answers are provided:
\begin{enumerate}
	\item What exactly is an allowable mathematical object? \\
	A: Every mathematical object is a mathematical set, and a mathematical set can be constructed using certain rules, for e.g. the now widely accepted Zermelo-Fraenkel Set Theory and the Axiom of Choice. While the Axiom of Choice is still highly criticized even today (e.g. the highly controversial \href{https://en.wikipedia.org/wiki/Banach–Tarski_paradox}{Banach-Tarski Paradox}), the Zermelo-Fraenkel Set Theory is widely welcomed, but not without critics. We shall call the Zermelo-Fraenkel Set Theory and the Axiom of Choice as the ZFC Axioms of Set Theory.
	\item What exactly is an allowable mathematical statement? \\
	A: Every mathematical statement can be expressed in a formal symbolic language, which uses symbols rather than words from any spoken language.
	\item What exactly is allowable in a mathematical proof? \\
	A: Every mathematical proof is a finite list of ordered pairs $(\mathscr{S}_n, \mathscr{F}_n)$ (which we can think of as proven theorems), where each $\mathscr{S}_n$ is a finite set of formulas (called the $\textit{premises}$) and each $\mathscr{F}_n$ is a single formula (called the $\textit{conclusion}$), which that each pair $(\mathscr{S}_n, \mathscr{F}_n)$ can be obtained from previous pairs $(\mathscr{S}_i, \mathscr{F}_i)$ with $i < n$, using certain proof rules.
\end{enumerate}

In the remainder of this appendix, we shall look more into the first 2 questions.


\section{ZFC Axioms of Set Theory}

\begin{defn}[Mathematical Symbols]
	We allow ourselves to use only the following symbols from the following symbol set: \\
	\begin{center}
		\begin{tabular}{c l}
			$\neg$		&	not \\
			$\land$		&	and \\
			$\lor$		&	or \\
			$\implies$	&	implies \\
			$\iff$	&	if and only if \\
			$=$			&	equals \\
			$\in$		&	is an element of \\
			$\forall$	&	for all \\
			$\exists$	&	there exists \\
			$()\quad\{\}\quad[]$	&	parenthesis
		\end{tabular}
	\end{center}
	along with some variable symbols such as $x, y, z, u, v, w,... \text{ or } x_1, x_2, x_3,...$
\end{defn}

\begin{defn}[Formula]
	A formula (in the formal symbolic language of first order set theory) is a non-empty finite string of symbols, from the above list, which can be obtained using finitely many applications following the three rules below:
	\begin{enumerate}
		\item If x and y are variable symbols, then each of the following strings are formulas.
		\[
			x = y, \quad x \in y
		\]
		\item If F and G are formulas then each of the following strings are formulas.
		\[
			\neg F, \quad (F \land G), \quad (F \lor G), \quad (F \implies G), \quad (F \iff G)
		\]
		\item If x is a variable symbol and F is a formula then each of the following is a formula.
		\[
			\forall x \in F, \quad \exists x \in F
		\]
	\end{enumerate}
\end{defn}

\begin{defn}[Free or Bounded Variable]
	Let x be a variable symbol and let F be a formula. For each occurrence of the symbol x, which does not immediately follow a quantifier, in the formula F, we define whether the occurrence of x is free or bound inductively as follows:
	\begin{enumerate}
		\item If F is a formula of one of the forms y = z or $y \in z$, where y and z are variable symbols (possibly equal to x), then every occurence of x in F is free, and no occurrence is bound.

		\item If F is a formula of one of the forms $\neg H, (H \land G), (H \lor G), (H \implies G), (H \iff G)$, where G and H are formulas, then each occurence of the symbol x is either an occurrence in the formula G or an occurrence in the formula H, and each free (respectively, bound) occurence of x in G remains free (respectively, bound) in F, and similarly for each free (or bound) occurrence of x in G. In other words, wlog, if x is bounded in G, then it is bounded in F, and vice versa.
		\item If F is a formula of one of the forms $\forall y \in G \text{ or } \exists y \in G$, where G is a formula and y is a variable symbol. If y is different from x, then each free (or bound) occurrence of x in G remains free (or bound) in the formula G, and if y = x then every free occurrence of x in G becomes bound in F, and every bound occurrence of x in G remains bound in F.
	\end{enumerate}
\end{defn}

\begin{defn}[Is Bound By and Binds]
	When a quantifier symbol occurs in a given formula F, and is followed by the variable symbol x and then by the formula G, any free occurence of x in G will become bound in the given formula F (by the 3rd definition above). We shall say that the occurrence of x is bound by (that occurrence of) the quantifier symbol, or that (the occurrence of) the quantifier symbol binds the occurence of x.
\end{defn}

\begin{defn}[Free Variable, Statement, Statement About]
	A $\textbf{free variable}$ in a formula F is any variable symbol that has at least one free occurence in F. A formula F with no free variables is called a $\textbf{statement}$. When the free variables in F all lie in the set $\{x_1, x_2, ..., x_n\}$, we shall write F as $F(x_1, x_2, ..., x_n)$ and we shall say that F is a $\textbf{statement about}$ the variables $x_1, x_2, ..., x_n$.
\end{defn}

\begin{defn}[Unique Existence]
	When F(x) is a statement about x, we sometimes write F(y) as a short form for the formula $\forall x(x=y\implies F(x))$, and we sometimes write
	\[
		\exists!y \quad F(y)
	\]
	which we read as "there exists a unique y such that F(y)", as a short form for the formula
	\[
		(\exists y \quad F(y) \land \forall z \quad F(z)) \implies z = y))
	\] which is, in turn, for the formula
	\[
		\exists y \bigg( \forall x \Big(x = y \implies F(x) \Big) \land \forall z \Big(\forall x (x = z \implies F(x)) \implies z = y \Big) \bigg)
	\]
\end{defn}

\begin{remark}[The ZFC Axioms of Set Theory (informal)]
	Every mathematical set can be constructed using specific rules, which we shall use the ZFC Axioms of Set Theory. Below is a list of the ZFC Axioms, stated informally.
	\begin{itemize}
		\item Empty Set Axiom: There exists an empty set $\emptyset$ with no elements.
		\item Extension Axiom: 2 sets are equal if and only if they have the same elements.
		\item Separation Axiom: If u is a set and F(x) is a statement about x, $\{x \in u : F(x)\}$ is a set.
		\item Pair Axiom: If u and v are sets then {u, v} is a set.
		\item Union Axiom: If u is a set then $\cup u = \bigcup\limits_{v \in u} v$ is a set.
		\item Power Set Axiom: If u is a set then $\mathcal{P}(u) = \{v : v \in u\}$ is a set.
		\item Axiom of Infinity: If we define the natural numbers to be the sets $0 = \emptyset, 1 = \{0\}, 2 = \{0, 1\}, 3 = \{0, 1, 2\}$ and so on, then $\bb{N} = \{0, 1, 2, 3, ...\}$ is a set.
		\item Replacement Axiom: If u is a ste and F(x, y) is a statement about x and y with the property that $\forall x \> \exists! y \> F(x,y)$ then $\{y : \exists x \in u \> F(x,y)\}$ is a set.
		\item Axiom of Choice: Given a set u of non-empty pairwise disjoint sets, there exists a set which contains exactly one element from each of the sets in u.
	\end{itemize}
\end{remark}

\begin{defn}[Empty Set Axiom]
	The Empty Set Axiom is the formula
	\[
		\exists u \> \forall x \quad \neg x \in u
	\]
\end{defn}

\begin{defn}[Extension Axiom]
	The Extension Axiom is the formula
	\[
		\forall u \> \forall v \> \Big(u = v \iff \forall x \> (x \in u \iff x \in v) \Big)
	\]
\end{defn}

\begin{thm}[Uniqueness of the Empty Set]
	The empty set is unique.
\end{thm}

\begin{defn}[$\emptyset$]
	We denote the unique empty set by $\emptyset$.
\end{defn}

\begin{defn}[Subset]
	Given sets u and v, we say that u is a $\textbf{subset}$ of v, and write $u \subseteq v$, when $\forall x (x \in u \implies x \in v)$
\end{defn}

\begin{defn}[Separation Axiom]
	For any statement F(x) about x, the following formula is an axiom.
	\[
		\forall u \> \exists v \> \forall x \Big(x \in v \iff (x \in u \land F(x)) \Big)
	\]
	More generally, for any statement $F(x, u_1, u_2, ..., u_n)$ about $x, u_1, u_2, ..., u_n$ where $n \geq 0$, the following formula is an axiom.
	\[
		\forall u \> \forall u_1 \hdots \forall u_n \> \exists v \> \forall x \Big( x \in v \iff (x \in i \land F(x, u_1, ..., u_n)) \Big)
	\]
	Any axiom of this form is called the Separation Axiom.
\end{defn}

\begin{note}
	It is important to realize that a Separation Axiom only allows us to construct a subset of a given set u. So, e.g., we cannot use the Separation Axiom to show that the collection $S = \{x : \neg x \in x\}$, which is used to formulate \hyperref[russel_paradox]{Russel's Paradox}, is a set.
\end{note}

\begin{defn}[Pair Axiom]
	The Pair Axiom is the formula
	\[
		\forall u \> \forall v \> \exists w \> \forall x \Big( x \in w \iff (x = u \lor x = v) \Big)
	\]
\end{defn}

\begin{defn}[Union Axiom]\label{union_axiom}
	The Union Axiom is the formula
	\[
		\forall u \> \exists w \> \forall x \Big( x \in w \iff \exists v (v \in u \land x \in v) \Big)
	\]
\end{defn}

\begin{defn}[Union]
	Given a set u, by the Union Axiom there exists a set w with the property that $\forall x \Big( x \in w \iff \exists v (v \in u \land x \in v) \Big)$, and by the Extension Axiom, this set w is unique. We call the set w the $\textbf{union}$ of the elements in u, and denote it by
	\[
		\cup u = \bigcup_{v \in u} v.
	\]Given two sets u and v, we define the union of u and v to be the set
	\[
		u \cup v := \bigcup \{u, v\}.
	\]
	Given three sets u, v, and w, note that \{z\} = \{z, z\} is a set and so $\{x, y, z\} = \{x, y\} \cup \{z\}$ is also a set. More generally, if $u_1, u_2, ..., u_n$ are sets then $\{u_1, u_2, ..., u_n\}$ is a set and we define the union of the sets $u_1, u_2, ..., u_n$ to be
	\[
		u_1 \cup u_2 \cup \hdots \cup u_n = \bigcup_{k = 1}^{n} u_k = \bigcup \{u_1, u_2, ..., u_n\}
	\]
\end{defn}

\begin{defn}[Intersection]
	Given a st u, we define the intersection of the elements in u to be the set
	\[
		\bigcap u = \biggl\{ x \in \bigcup u \> \Big| \> \forall v (v \in u \implies x \in v) \biggr\}
	\] Given two sets u and v, we define the intersection of u and v to be the set
	\[
		u \cap v = \bigcap \{u, v\}
	\] and more generally, given sets $u_1, u_2, ..., u_n$, we define the intersection of $u_1, u_2, ..., u_n$ to be the set
	\[
	 	u_1 \cap u_2 \cap \hdots \cap u_n = \bigcap_{k=1}^{n} u_k = \bigcap \{u_1, u_2, ..., u_n\}
	\] 
\end{defn}

\begin{defn}[Power Set Axiom]
	The Power Set Axiom is the formula
	\[
		\forall u \> \exists w \> \forall v (v \in w \iff v \subseteq u)
	\]
\end{defn}

\begin{defn}[Power Set]
	Given a set u, the set w is with the property that $\forall v (v \in w \iff v \subseteq u)$ (which exists by the Power Set Axiom and is unique by the Extension Axiom) is called the power set of u and is denoted by $\mathcal{P}(u)$, so we have
	\[
		\mathcal{P}(u) = \{v | v \subseteq u\}
	\]
\end{defn}

\begin{defn}[Ordered Pair]
	Given two sets x and y, we define the ordered pair (x, y) to be the set
	\[
		(x, y) = \{\{x\}, \{x, y\}\}.
	\]
	Given two sets u and v, note that if $x \in u$ and $y \in v$ then we have $\{x\} \in \mathcal{P}(u \cup v)$ and $\{x, y\} \in \mathcal{P}(u \cup v)$ and so $(x, y) = \{\{x\}, \{x, y\}\} \in \mathcal{P}(\mathcal{P}(u \cup v))$. We define the product $u \times v$ to be the set
	\[
		u \times v = \{(x, y) | x \in u \land y \in v\},
	\] i.e.
	\[
		u \times v = \Bigr\{ z \in \mathcal{P}(\mathcal{P}(u \cup v)) | \exists x \exists y \big( (x \in u \land y \in v) \land z = (x, y) \big) \Bigr\}
	\]
\end{defn}

\begin{defn}[Successor, Inductive]
	We define
	\[
		0 = \emptyset, \quad	1 = \{0\}, \quad	2 = \{0, 1\} = 1 \cup \{1\}, \quad	3 = \{0, 1, 2\} = 2 \cup \{2\},
	\] and so on. For a set x, we define the successor of x to be the set
	\[
		x + 1 = x \cup \{x\}.
	\] A set u is called inductive when it has the property that
	\[
		(0 \in u \land \forall x (x \in u \implies x + 1 \in u))
	\]
\end{defn}

\begin{defn}[Axiom of Infinity]
	The Axiom of Infinity is the formula
	\[
		\exists u (0 \in u \land \forall x (x \in u \implies x + 1 \in u))
	\] so the Axiom of Infinity states that there exists an inductive set.
\end{defn}

\begin{thm}[Existence \& Uniqueness of an Inductive Set]
	$\exists w := \{x | x \in v \text{ for every inductive set v }\}$ \\
	Moreover, this set w is an inductive set.
\end{thm}

\begin{defn}[Natural Numbers]
	The unique set w in the above theorem is called the set of natural numbers, and we denote it by $\bb{N}$. We write
	\begin{align*}
		\bb{N} &= \{x | x \in v \text{ for every inductive set v }\} \\
			   &= \{0, 1, 2, 3, ...\}
	\end{align*}
	For $x, y \in \bb{N}$, we write x < y when $x \in y$ and write $x \leq y$ when $x < y \lor x = y$.
\end{defn}

\begin{remark}
	For a formula F, we write $\forall x \in u \> F$ as a shorthand notation for the formula $\forall x (x \in u \implies F)$. Similarly, we write $\exists x \in u \> F$ as a shorthand notation for $\exists x (x \in u \land F)$
\end{remark}

\begin{thm}[Principle of Induction]
	Let F(x) be a statement about x. SPS that
	\begin{enumerate}
		\item F(0), and
		\item $\forall x \in \bb{N} (F(x) \implies F(x+1))$.
	\end{enumerate}
	Then $\forall x \in \bb{N} \> F(x)$
\end{thm}

\begin{remark}
	The expression F(0) is short for $\forall x (x = 0 \implies F(x))$, which in turn is short for $\forall x (\forall y \> \neg y \in x \implies F(x))$.Similarly, F(x + 1) is short for the formula $\forall y (y = x + 1 \implies F(y))$, where F(y) is short for $\forall x (x = y \implies F(x))$.
\end{remark}

\begin{defn}[Replacement Axiom]
	Given a statement F(x, y) about x and y, the following formula is an axiom:
	\[
		\forall u \Big( \forall x \exists! y \> F(x,y) \implies \exists w \forall y \big( y \in w \iff \exists x \in u \> F(x, y) \big) \Big)
	\]
	where $\exists! y \> F(x, y)$ is short for $\exists y \Big(F(x, y) \land \forall z \big(F(x, z) \implies z = y)\Big)$ with F(x ,z) short for the formula $\forall y (y = z \implies F(x, y))$. More generally, given a statement $F(x, y, u_1, ..., u_n)$ about $x, y, u_1, ..., u_n$ with $n \geq 0$, the following formula is an axiom:
	\[
		\forall u \forall u_1 \hdots \forall u_n \Big( \forall x \exists!y \> F(x, y, u_1, ..., u_n) \implies \exists w \forall y \big( y \in w \iff \exists x \in u \> F(x, y, u_1, ..., u_n) \big) \Big)
	\]
	An axiom of this form is called a Replacement Axiom.
\end{defn}

\begin{defn}[Axiom of Choice]
	The Axiom of Choice is the formula given by
	\[
		\forall u \Big( \big( \neg \phi \in u \land \forall x \in u \> \forall y \in u (\neg x = y \implies x \cap y = \emptyset) \big) \implies \exists w \forall v \in u \> \exists!x \in v \> x \in w \Big)
	\]
\end{defn}

From this point on, we will be using upper-case letters to denote sets, instead of lower-case as per the statements above.


\section{Relations, Equivalence Relations, Functions and Recursion}

\begin{defn}[Binary Relation]
	A binary relation R on a set X is a subset $R \subseteq X \times X$. More generally, a binary relation is any set R whose elements are ordered pairs. For a binary relation R, we usually write xRy instead of $(x, y) \in R$.
\end{defn}

\begin{defn}[Domain, Range, Image, Inverse Image, Inverse, Composition]
	Let R and S be binary relations. \\
	The domain of R is
	\[
		\text{Domain}(R) = \{x | \exists y \> xRy\}
	\]
	and the range of R is
	\[
		\text{Range}(R) = \{x | \exists y \> xRy\}.
	\]
	For any set A, the image of A under R is
	\[
		R(A) = \{y | \exists x \in A \> xRy\}
	\]
	and the inverse image of A under R is
	\[
		R^{-1}(A) = \{x | \exists y \in A \> xRy\}.
	\]
	The inverse of R is
	\[
		R^{-1} = \{(y, x) | (x, y) \in R\}
	\]
	and the composition S composed with R is
	\[
		S \circ R = \{(x, z) | \exists y \> xRy \land ySz \}
	\]
\end{defn}

\begin{thm}[Domain, Range, Image and Inverse Image as Sets]
	Let A be a set and let R be a binary relation. Then Domain(R), Range(R), R(A) and $R^{-1}(A)$ are sets.
\end{thm}

\begin{thm}[Inverse and Composition as Binary Relations]
	Let A be a set and let R and S be binary relations. Then $R^{-1}$ and $S \circ R$ are binary relations.
\end{thm}

\begin{defn}[Equivalence Relation]
	An equivalence relation on a set X is a binary relation R on X such that
	\begin{enumerate}
		\item R is $\textbf{reflexive}$, i.e. $\forall x \in X \> xRx$
		\item R is $\textbf{symmetric}$, i.e. $\forall x, y \in X \> (xRy \implies yRx)$, and
		\item R is $\textbf{transitive}$, i.e. $\forall x, y, z \in R \> \big( (xRy \land yRz) \implies xRz \big)$.
	\end{enumerate}
\end{defn}

\begin{defn}[Equivalence Class]
	Let R be an equivalence relation on the set X. For $a \in X$, the equivalence class of a modulo R is the set
	\[
		[a]_R = \{x \in X | xRa \}
	\]
\end{defn}

\begin{defn}[Partition]
	A partition of a set X is a set S of non-empty pairwise disjoint sets whose union is X, that is a set S such that
	\begin{enumerate}
		\item $\forall X, Y \in S \> (X \neq Y \implies X \cap Y = \emptyset)$
		\item $\bigcup S = X$.
	\end{enumerate}
\end{defn}

\begin{thm}[Correspondence of Equivalence Relations and Partitions]
	Given a set X, we have the following correspondence between equivalence relations on X and partitions of X.
	\begin{enumerate}
		\item Given an equivalence relation R on X, the set of all equivalence classes
		\[
			S_R = \{[a]_R | a \in X\}
		\] is a partition of X.
		\item Given a partition S of X, the relation $R_S$ on X is defined by
		\[
			R_S = \{(x, y) \in X \times X | \exists A \in S (x \in A \land y \in A)\}
		\] is an equivalence relation on X.
		\item Given an equivalence relation R on X we have $R_{S_R} = R$, and a given partition S of X, we have $S_{R_S} = S$.
	\end{enumerate}
\end{thm}

\begin{note}[Set of All Equivalence Classes]
	Given an equivalence relation R on X, the set of all equivalence classes, which we denote by $S_R$ in the above theorem, is usually denoted by $X/R$, so
	\[
		X/R = \{[a]_R | a \in X\}
	\]
\end{note}

\begin{defn}[Set of Representatives]
	Let R be an equivalence relation. A set of representatives for R is a subset of X which contains exactly one element from each equivalence class in X/R.
\end{defn}

\begin{remark}
	Notice that the AC is equivalent to the statement that every equivalence relation has a set of representatives.
\end{remark}

\begin{defn}[Function]
	Get sets X and Y, a function from X to Y is a binary relation $f \subseteq X \times Y$ with the property that
	\[
		\forall x \in X \> \exists! y \in Y \> (x,y) \in f
	\]
	More generally, a function is a binary relation with the property that
	\[
		\forall x \in Domain(f) \> \exists! y \> (x, y) \in f.
	\]
	For a function f, we usually write y = f(x) instead of xfy. It is customary to use the notation $f: X \mapsto Y$ when X = Domain(f) and Y is any set with $Range(f) \subseteq Y$.
\end{defn}

\begin{defn}[One-to-one \& Onto]
	Let $f: X \mapsto Y$. THe function f is called one-to-one (or injective) when
	\[
		\forall y \in Y \> \exists \text{ at most one } x \in X \> y = f(x)
	\]
	and f is called onto (or surjective) when
	\[
		\forall y \in Y \> \exists \text{ at least one } x \in X \> y = f(x)
	\]
\end{defn}

\begin{defn}[Left and Right Inverses]
	Let $f: X \mapsto Y$. Let $I_X$ and $I_Y$ denote the identity function on X and Y respectively. A left inverse of f is a function $g: Y \mapsto X$ such that $g \circ f = I_X$. A right inverse of f is a function $H: X \mapsto Y$ such that $f \circ H = I_Y$. Note that if f has a left inverse g and a right inverse H, then we have $g = g \circ I_Y = g \circ f \circ H = I_X \circ H = H$. In this case, we say that g is the (unique two-sided) inverse of f.
\end{defn}

\begin{thm}[Surjective and Injective VS Inverses]
	Let $f: X \mapsto Y$. Then
	\begin{enumerate}
		\item f is one-to-one if and only if f has a left inverse.
		\item f is onto if and only if f has a right inverse.
		\item f is one-to-one and onto if and only if f has a (two-sided) inverse.
	\end{enumerate}
\end{thm}

\begin{defn}[Invertible]
	A function $f: X \mapsto Y$ is called invertible (or bijective) when it is one-to-one and onto, or equivalently, when it has a (unique two-sided) inverse.
\end{defn}

\begin{thm}[The Recursion Theorem]\label{recursion_theorem}
	\begin{enumerate}
		\item Let A be a set, let $a \in A$, and let $g : A \times \bb{N} \mapsto A$. Then there exists a unique function $f: \bb{N} \mapsto A$ such that
		\[
			f(0) = a \text{ and } f(n+1) = g(f(n), n) \text{ for all } n \in \bb{N}
		\]
		\item Let A and B be sets, let $g: A \mapsto B$, and let $h: A \times B \times \bb{N} \mapsto B$. Then there exists a unique function $f:A \times \bb{N} \mapsto B$ such that for all $a \in A$ we have
		\[
			f(a, 0) = g(a) \text{ and } f(a, n + 1) = h(a, f(a, n), n) \text{ for all } n \in \bb{N}
		\]
	\end{enumerate}
\end{thm}


\section{Construction of Integers, Rational, Real and Complex Numbers}

\begin{defn}[Sum and Product]
	By Part(2) of the \hyperref[recursion_theorem]{Recursion Theorem}, there is a unique function $s: \bb{N} \times \bb{N} \mapsto \bb{N}$ such that for all $a, b \in \bb{N}$ we have
	\[
		s(a, 0) = a, \quad s(1, b + 1) = s(a, b) + 1.
	\]
	We call s(a, b) the sum of a and b $\in \bb{N}$ and write it as
	\[
		a + b = s(a, b).
	\]
	Also, there is a unique function $p: \bb{N} \times \bb{N} \mapsto \bb{N}$ such that for all a, b $\in \bb{N}$ we have
	\[
		p(a, 0) = 0, \quad p(a, b + 1) = p(a, b) + a
	\]
	We call p(a, b) the product of a and b in $\bb{N}$, and we write it as
	\[
		a \cdot b = p(a, b)
	\]
\end{defn}

\begin{defn}[Integers]
	We define the set of integers to be the set
	\[
		\bb{Z} = (\bb{N} \times \bb{N})/R
	\]
	where R is the equivalence relation given by
	\[
		(a, b)R(c,d) \iff a + d = b + c
	\]
	For [(a, b)] and [(c,d)] in $\bb{Z}$, we define
	\begin{gather*}
		[(a, b)] \leq [(c, d)] \iff b + c \leq a + d \\
		[(a, b)] + [(c ,d)] \iff [(a + c, b + d)] \\
		[(a, b)] \cdot [(c, d)] = [(ac + bd, ad + bc)]
	\end{gather*}
	For $n \in \bb{N}$ we write n = [(n, 0)] and -n = [(0, n)], so that every element of $\bb{Z}$ can be written as $\pm n$ for some $n \in \bb{N}$, and we can identity $\bb{N}$ with a subset of $\bb{Z}$
\end{defn}

\begin{defn}[Rational Numbers]
	We define the set of reational numbers to be the set
	\[
		\bb{Q} = (\bb{N} \times \bb{Z}^+)/R
	\]
	where $\bb{Z}^+ = \{x \in \bb{N} | x \neq 0\}$ and R is the equivalence relation given by
	\[
		(a, b)R(c, d) \iff ad = bc
	\]
	For [(a, b)] and [(c, d)] in $\bb{Q}$ we define
	\begin{gather*}
		[(a, b)] \leq [(c, d)] \iff a \cdot d \leq b \cdot c \\
		[(a, b)] + [(c ,d)] \iff [(a \cdot d + b \cdot c, b \cdot d)] \\
		[(a, b)] \cdot [(c, d)] = [(ac, bd)]
	\end{gather*}
	For $a \in \bb{N}$ and $b \in \bb{Z}^+$, it is customary to write $\frac{a}{b} = [(a, b)]$. Also for $a \in \bb{Z}$ we write a = [(a, 1)], and we identify $\bb{Z}$ with a subset of $\bb{Q}$
\end{defn}

\begin{defn}[Real Numbers]
	We define the set of real numbers of the the set
	\[
		\bb{R} = \{x \subseteq \bb{Q} | x \neq \emptyset, x \neq \bb{Q}, \forall a \in x \> \forall b \in \bb{Q}(b \leq a \implies b \in x), \forall a \in x \> \exists b \in x \> a < b \}
	\]
	For $x, y \in \bb{R}$ we define
	\begin{gather*}
		x \leq y \iff x \subseteq y \\
		x + y = \{ a + b | a, b \in \bb{Q}, a \in x, b \in y\}
	\end{gather*}
	For $0 \leq x, y \in \bb{R}$ we define
	\[
		x \cdot y = \{ a \cdot b | 0 \leq a, b \in \bb{Q}, a \in x, b \in y\} \cup \{c \in \bb{Q} | c < 0\},
	\]
	and YOU can try to, similarly, define $x \cdot y$ in the case that x < 0 and y < 0.
\end{defn}

\begin{defn}[Complex Numbers]
	We define the set of complex numbers to be the set
	\[
		\bb{C} = \bb{R} \times \bb{R}.
	\]
	We define addition and multiplication in $\bb{C}$ by
	\begin{align*}
		(a, b) + (c, d) &= (a + c, b + d) \\
		(a, b) \cdot (c, d) &= (ac - bd, ad + bc).
	\end{align*}
	We write $i = (0, 1).$ For $x \in \bb{R}$ we write x = (x, 0) and identify $\bb{R}$ with a subset of $\bb{C}$.
\end{defn}

\end{document}
% Document End