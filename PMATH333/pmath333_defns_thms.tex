% Document Head
\documentclass[11pt, oneside]{book}
\usepackage{geometry}
\geometry{letterpaper}
\usepackage[parfill]{parskip}
\usepackage{graphicx}

% Essential Packages
\usepackage{ragged2e}
\usepackage{amssymb}
\usepackage{amsmath}
\usepackage{mathrsfs}
\usepackage[utf8]{inputenc}
\usepackage[english]{babel}
\usepackage[hyperref]{ntheorem}

% Theorem Style Customization
\setlength\theorempreskipamount{2ex}
\setlength\theorempostskipamount{3ex}

% hyperref Package Settings
\usepackage{hyperref}
\hypersetup{
	colorlinks = true,
	linkcolor = magenta
}

% ntheorem Declarations
\theoremstyle{break}
\newtheorem{thm}{Theorem}[section]
\newtheorem*{proof}{Proof}
\newtheorem{crly}{Corollary}[thm]
\newtheorem{lemma}[thm]{Lemma}
\newtheorem{propo}{Proposition}[section]
\newtheorem*{remark}{Remark}
\newtheorem*{note}{Note}
\newtheorem{defn}{Definition}[section]
\newtheorem{eg}{Example}[section]

% ntheorem listtheorem style
\makeatletter
\def\thm@@thmline@name#1#2#3#4{%
        \@dottedtocline{-2}{0em}{2.3em}%
                   {\makebox[\widesttheorem][l]{#1 \protect\numberline{#2}}#3}%
                   {#4}}
\@ifpackageloaded{hyperref}{
\def\thm@@thmline@name#1#2#3#4#5{%
    \ifx\\#5\\%
        \@dottedtocline{-2}{0em}{2.3em}%
            {\makebox[\widesttheorem][l]{#1 \protect\numberline{#2}}#3}%
            {#4}
    \else
        \ifHy@linktocpage\relax\relax
            \@dottedtocline{-2}{0em}{2.3em}%
                {\makebox[\widesttheorem][l]{#1 \protect\numberline{#2}}#3}%
                {\hyper@linkstart{link}{#5}{#4}\hyper@linkend}%
        \else
            \@dottedtocline{-2}{0em}{2.3em}%
                {\hyper@linkstart{link}{#5}%
                  {\makebox[\widesttheorem][l]{#1 \protect\numberline{#2}}#3}\hyper@linkend}%
                    {#4}%
        \fi
    \fi}
}
\makeatother
\newlength\widesttheorem
\AtBeginDocument{
  \settowidth{\widesttheorem}{Proposition A.1.1.1\quad}
}

\theoremlisttype{allname}

% Shortcuts
\newcommand{\bb}[1]{\mathbb{#1}}		% using bb instead of mathbb
\newcommand{\floor}[1]{\lfloor #1 \rfloor}	% simplifying the writing of a floor function
\newcommand{\ceiling}[1]{\lceil #1 \rceil}	% simplifying the writing of a ceiling function

% Main Body
\title{UW W17 PMATH333 - Definitions and Theorems}
\author{Johnson Ng}

\begin{document}
\maketitle

% =================================
% 		Preface
% =================================
\chapter*{Preface}
PMATH333 is offered as a course that attempts to bridge the gap for students who have taken the regular math courses instead of the advanced math courses, in particular for MATH147, MATH148 and MATH247 in UW. This set of notes is taken from the W2017 term.

\tableofcontents

\chapter*{List of Definitions}
\theoremlisttype{allname}
\listtheorems{defn}

\chapter*{List of Theorems}
\theoremlisttype{allname}
\listtheorems{lemma,thm,crly,propo}


% =================================
% 		Chapter 1
% =================================
\chapter{The Real Number System}

\section{Ordered Field Axioms}
Please review \hyperref[apdxA]{Appendix A}. We shall use all of the set notations that are introduced in Appendix A. We will also introduce one more notation.

\begin{defn}[Removal]
	Let A and B be sets. The set A remove B, denoted as $A \setminus B$, is the set
	\[
		A \setminus B = \{ x | x \in A \land x \notin B \}
	\]
\end{defn}

\begin{defn}[Disjoint]
	Let A and B be sets. We say that A and B are disjoint when $A \cap B = \emptyset$
\end{defn}

\begin{thm}[Properties of Sets]
	Let $A, B, C \subseteq X$. Then
	\begin{enumerate}
		\item (Idempotence) $A \cup A = A, A \cap A = A$
		\item (Identity) $A \cup \emptyset = A, A \cap \emptyset = \emptyset, A \cup X = X, A \cap X = A$
		\item (Associativity) $(A \cup B) \cup C = A \cup (B \cup C)$ and $(A \cap B) \cap C = A \cap (B \cap C)$
		\item (Commutativity) $A \cup B = B \cup A$ and $A \cap B = B \cap A$
		\item (Distributivity) $A \cap (B \cup C) = (A \cap B) \cup (A \cap C)$ and $A \cup (B \cap C) = (A \cup B) \cap (A \cup C)$
		\item (De Morgan's Laws) $X \setminus (A \cup B) = (X \setminus A) \cap (X \setminus B)$ and $X \setminus (A \cap B) = (X \setminus A) \cup (X \setminus B)$
	\end{enumerate}
\end{thm}

\begin{defn}[Intervals]
	For $a, b \in \bb{R}$ with $a \leq b$ we write
	\begin{center}
		\begin{tabular}{c c}
			$(a, b) = \{x \in \bb{R} | a < x < b\}$,	&	$[a, b] = \{x \in \bb{R} | a \leq x \leq b \}$,\\
			$(a, b] = \{x \in \bb{R} | a < x \leq b \}$,	&	$[a, b) = \{x \in \bb{R} | a \leq x < b \}$,\\
			$(a, \infty) = \{x \in \bb{R} | a < x \}$,	&	$[a, \infty) = \{x \in \bb{R} | a \leq x\}$,\\
			$(-\infty, b) = \{x \in \bb{R} | x \leq b\}$,	&	$(-\infty, b] = \{x \in \bb{R} | x < b\}$,\\
			\multicolumn{2}{c}{$(-\infty, \infty) = \bb{R}$}
		\end{tabular}
	\end{center}

	An interval in $\bb{R}$ is any set of one of the above forms. In the case that a = b, we have $(a, b) = [a, b) = (a, b] = \emptyset$ and $[a, b] = \{a\}$, and these intervals are called $\textbf{denegerate}$ intervals. The intervals $\emptyset, (a, b), (a, \infty), (-\infty, b) \text{ and } (\infty, \infty)$ are called open intervals. The intervals $\emptyset, [a, b], [a, \infty), (-\infty, b] \text{ and } (-\infty, \infty)$ are called closed intervals.
\end{defn}

\begin{remark}
	Note on how the intervals $\emptyset$ and $(-\infty, \infty)$ are both open and closed intervals.
\end{remark}

\begin{defn}[Ring]
	A ring is a set F with two distinct elements $0, 1 \in F$ and two binary operations + and $\cdot$ such that
	\begin{enumerate}
		\item (Additive Associativity) For all $x, y, z \in F$ we have (x + y) = z = x + (y + z),
		\item (Additive Commutativity) For all $x, y \in F$ we have x + y = y + x,
		\item (Additive Identity) For all $x \in F$ we have 0 + x = x.
		\item (Additive Inverse) $\forall x \in F \> \exists! y \in F \> x + y = 0$
		\item (Multiplicative Associativity) $\forall x, y, z \in F$ we have $(x \cdot y) \cdot z = x \cdot (y \cdot z)$,
		\item (Multiplicative Identity) $\forall x \in F$ we have $1 \cdot x = x = x \cdot 1$,
		\item (Distributivity) $\forall x, y, z \in F$ we have $x \cdot (y + z) = (x \cdot y) + (x \cdot z)$
	\end{enumerate}
\end{defn}

\begin{defn}[Commutative Ring]
	A ring F is a commutative ring if it has the following (additional) property: \\
	(Multiplicative Commutativity) $\forall x, y \in F$ we have $x \cdot y = y \cdot x$.
\end{defn}

\begin{defn}[Field]
	A commutative ring F is a field if it has the following (additional) property: \\
	(Multiplicative Inverse) $\forall x \neq 0 \in F \> \exists! y \in F$ such that $x \cdot y = 1$.
\end{defn}

\begin{remark}
	For the sake of simplicity, we will write $x \cdot y = xy$ for any x and y.
\end{remark}

\begin{thm}[$\bb{Q} \text{ and } \bb{R}$ as Fields]
	$\bb{Q} \text{ and } \bb{R}$ are fields.
\end{thm}

\begin{remark}
	Note that $\bb{Z}$ and $\bb{N}$ are not fields since their elements do have have a multiplicative inverse. They are, however, commutative rings.
\end{remark}

\begin{remark}[Some shorthand notations]
	Let F be a friend and let $a ,b \in F$. We denote the unique additive inverse of a by -a and we write a - b = a + (-b). When $a \neq -$, we denote the unique multiplicatie inverse of a by $a^{-1}$ and we write $b \div a = \frac{b}{a} = ba^{-1}$.
\end{remark}

\begin{thm}[Cancellations \& Identities]
	Lt F be a field. Then $\forall x, y z \in F$ we have
	\begin{enumerate}
		\item (Additive Cancellation) $x + y = x + z \implies x = z$
		\item (Uniqueness of Additive Identity) $x + y = x \implies y = 0$
		\item (Multiplicative Cancellation) $xy = xz \implies (x = 0 \lor y = z)$
		\item (Uniqueness of Multiplicative Identity) $xy = x \implies y = 1$
		\item (No Zero Divisors) $xy = 0 \implies (x = 0 \lor y = 0)$
	\end{enumerate}
\end{thm}

\begin{thm}[Properties of Fields]
	Let F be a field. Then for all $x, y \in F$ we have
	\begin{center}
		\begin{tabular}{c c c c}
			$0 \cdot x = 0$	&	-(-x) = x 	&	-(x + y) = -x - y 	&	(-1)x = x \\
			(-x)y = -(xy) &	(-x)(-y) = xy &	$(a^{-1})^{-1} = a$	&	$(ab)^{-1} = a^{-1}b^{-1}$ \\
			\multicolumn{4}{c}{$(-a)^{-1} = -a^{-1}$}
		\end{tabular}
	\end{center}
\end{thm}

\begin{defn}[Order]
	An order on a set X is a binary relation $\leq$ on X such that
	\begin{enumerate}
		\item (Totality) $\forall x, y \in X (x \leq y \lor y \leq x)$
		\item (Antisymmetry) $\forall x, y \in X \> (x \leq y \land y \leq x) \implies x = y$
		\item (Transitivity) $\forall x, y, z \in X \> (x \leq y \land y \leq z) \implies x \leq z$
	\end{enumerate}
\end{defn}

\begin{remark}[Order defined using the $<$ operator]
	Note that we may also make a definition of the above using < instead of $\leq$. Then the properties that will define an order will be:
	\begin{enumerate}
		\item (Trichotomy Property) $\forall x, y \in X (x < y \lor y < x \lor x = y)$
		\item (Transitive Property) $\forall x, y, z \in X (x < y \land y < z) \implies x < z$
		\item (Additive Property) $\forall x, y, z \in X \> x < y \implies x + z < y + z$
		\item (Multiplicative Property) $\forall a, b, c \in X$ we have
		\begin{enumerate}
			\item $a < b \land c > 0 \implies ac < bc$ 
			\item $a < b \land c < 0 \implies bc < ac$

		\end{enumerate}
	\end{enumerate}
\end{remark}

\begin{remark}[Non-negative and Non-positive]
	 Let $a \in \bb{R}$. We say that a is non-negative when $0 \leq a$ and that a is non-positive $a \leq 0$.
\end{remark}

\begin{remark}
	Some ways of writing the order symbol. Let $a, b, c \in \bb{R}$
	\begin{itemize}
		\item b > a is equivalent to b < a
		\item $b \leq a$ is equivalent to $b < a \lor b = a$
		\item If a < b and b < c, we can write a < b < c.
	\end{itemize}
\end{remark}

\begin{thm}[$\bb{N}, \bb{Z}, \bb{Q}$ and $\bb{R}$ are ordered]
	Each of $\bb{N}, \bb{Z}, \bb{Q}$ and $\bb{R}$  is an ordered set using the standard order $\leq$. Under the inclusions $\bb{N} \subseteq \bb{Z} \subseteq \bb{Q} \subseteq \bb{R}$ the orders coincide (e.g. when $a, b \in \bb{N}$ we have $a \leq b$ in $\bb{N}$ if and only if $a \leq b$ in $\bb{R}$)
\end{thm}

\begin{defn}[Ordered Field]
	An ordered field is a field F with an order $\leq$ such that for all $x ,y ,z \in F$
	\begin{enumerate}
		\item $x \leq y \implies x + z \leq y + z$, and
		\item $0 \leq x \land 0 \leq y \implies 0 \leq xy$.
	\end{enumerate}
\end{defn}

\begin{thm}[$\bb{Q}$ and $\bb{R}$ as Ordered Fields]
	$\bb{Q}$ and $\bb{R}$ are ordered fields.
\end{thm}

\begin{thm}[Properties of Ordered Fields]
	Let F be an ordered field. Then $\forall x, y, z \in F$ we have
	\begin{enumerate}
		\item $x > 0 \implies -x < 0$ and $x < 0 \implies -x > 0$
		\item $x \neq 0 \implies x^2 > 0$ and in particular 1 > 0
		\item $0 < x < y \implies 0 < \frac{1}{y} < \frac{1}{x}$
	\end{enumerate}
\end{thm}

\begin{defn}[Absolute Value]
	Let F be an ordered field. For $a \in F$ we define the absolute value of a to be
	\begin{gather*}
		|a| =
		\begin{cases}
			 a & \text{ if } a \geq 0, \\
			-a & \text{ if } a \leq 0.
		\end{cases}
	\end{gather*}
\end{defn}

\begin{thm}[Properties of Absolute Values]
	Let F be an ordered field. Then for all $x, y, z \in F$ we have
	\begin{enumerate}
		\item (Positive Definiteness) $|x| \geq 0$ and $|x| = 0 \iff x = 0$
		\item (Symmetry) $|x - y| = |y - x|$
		\item (Multiplicativeness) $|xy| = |x||y|$
		\item (Triangle Inequality) $|x + y| \leq |x| + |y|$
		\item (Approximation) $|x - y| \leq z \implies y - z \leq x \leq y + z$
	\end{enumerate}
\end{thm}

\begin{thm}[Basic Order Properties in $\bb{Z}$]
	\begin{enumerate}
		\item $\forall n \in \bb{Z} (n \in \bb{N} \iff n \geq 0)$
		\item $\forall k, n \in \bb{Z} (k \leq n \iff k < n + 1)$
	\end{enumerate}
\end{thm}


\section{Completeness Axiom}

\begin{defn}[Upper and Lower Bounds]
	Let X be an ordered set and let $A \subseteq X$
	\begin{enumerate}
		\item We say that A is bounded above (in X) when $\exists b \in X \> \forall a \in A \> a \leq b$, in which case we call b the upper bound of A.
		\item We say that A is bounded below (in X) when $\exists c \in X \> \forall a \in A \> c \leq a$, in which case we call c the lower bound of A.
	\end{enumerate}
	We say that A is bounded when it is bounded above and below.
\end{defn}

\begin{defn}[Supremum and Infimum]
	Let X be an ordered set and let $A \subseteq X$.
	\begin{enumerate}
		\item We say that A has a supremum (or the least upper bound) when
			\[
				\exists b \in X (\forall a \in A \> a \leq b) \quad \forall c \in X (\forall a \in A \> a \leq c) \quad b < c.
			\]
			We write b = sup A. \\
			Now if b = sup A and $b \in A$, we call b the maximum of A, and denote it as b = max A. \\
		\item We say that A has an infimum (or the greatest lower bound) when
			\[
				\exists d \in X (\forall a \in A \> d \leq a) \quad \forall c \in X (\forall a \in A \> c \leq a) \quad c < d.
			\]
			We write d = inf A. \\
			Now if d = inf A and $d \in A$, we call d the minimum of A, and denote it as d = min A.
	\end{enumerate}
\end{defn}

\begin{thm}[Approximation Property of Supremum and Infimum]
	Let $\emptyset \neq A \subseteq \bb{R}$.
	\begin{enumerate}
		\item $b = \sup A \implies \forall 0 < \epsilon \in \bb{R} \> \exists x \in A \> (b - \epsilon < x \leq b)$
		\item $c = \inf A \implies \forall 0 < \epsilon \in \bb{R} \> \exists x \in A \> (c \leq x < c + \epsilon)$
	\end{enumerate}
\end{thm}

\begin{thm}[Completeness Properties of $\bb{R}$]
	\begin{enumerate}
		\item $\forall \emptyset \neq A \subseteq \bb{R}$, if A is bounded above, then A has a supremum in $\bb{R}$
		\item $\forall \emptyset \neq A \subseteq \bb{R}$, if A is bounded below, then A has an infimum in $\bb{R}$
	\end{enumerate}
\end{thm}

\begin{thm}[Well-Ordering Properties of $\bb{Z}$ in $\bb{R}$]
	\begin{enumerate}
		\item Every nonempty subset of $\bb{Z}$ which is bounded above in $\bb{R}$ has a maximum.
		\item Every nonempty subset of $\bb{Z}$ which is bounded below in $\bb{R}$ has a minimum. In particular, every nonempty subset of $\bb{N}$ has a minimum.
	\end{enumerate}
\end{thm}

\begin{thm}[Floor and Ceiling Properties of $\bb{Z}$ in $\bb{R}$]
	\begin{enumerate}
		\item (Floor Properties) $\forall x \in \bb{R} \> \exists! n \in \bb{Z} \> (x - 1 < n \leq x)$
		\item (Ceiling Properties) $\forall x \in \bb{R} \> \exists! n \in \bb{Z} \> (x \leq n < x + 1)$
	\end{enumerate}	
\end{thm}

\begin{defn}[Floor and Ceiling Functions]
	Given $x \in \bb{R}$ we define the floor of x to be the unique $n \in \bb{Z}$ with $x - 1 < n \leq x$ and denote the floor of x by $\floor{x}$. The function $f: \bb{R} \to \bb{Z}$ given by $f(x) = \floor{x}$ is called the floor function.

	Similarly, we define the ceiling of x to be the unique $n \in \bb{Z}$ with $x \leq n < x + 1$ and denote the ceiling of x by $\ceiling{x}$. The function $f: \bb{R} \to \bb{Z}$ given by $f(x) = \ceiling{x}$ is called the ceiling function.
\end{defn}

\begin{thm}[Archimedean Properties of $\bb{Z}$ in $\bb{R}$]
	\begin{enumerate}
		\item $\forall x \in \bb{R} \> \exists n \in \bb{Z} \> (n > x)$
		\item $\forall x \in \bb{R} \> \exists m \in \bb{Z} \> (m < x)$
	\end{enumerate}
\end{thm}

\begin{thm}[Density of $\bb{Q}$]
	\begin{gather*}
		\forall a, b \in \bb{R} (a < b) \quad \exists q \in \bb{Q} (a < q < b)
	\end{gather*}
\end{thm}


% =================================
% 		Chapter 2
% =================================
\chapter{Sequences}\label{chp2}

\section{Limits of Sequences}

\begin{defn}[Sequence]
	For $p \in \bb{Z}$, let $\bb{Z}_{\geq p} = \{k \in \bb{Z} | k \geq p\}$. A sequence in a set A is a function of the form $x: \bb{Z}_{\geq p} \to A$ for some $p \in \bb{Z}$. Given a sequence $x: \bb{Z}_{\geq p} \to A$, the k-th term of the sequence is the element $x_k = x(k) \in A$, and we denote the sequence x by
	\[
		\langle x_k \rangle_{k \geq p} = \{x_k | k \geq p\} = \{x_p, x_{p + 1}, x_{p + 2}, ...\}
	\]
	Note that the range of the sequence $\langle x_k \rangle_{k \geq p}$ is the set $\{x_k\}_{k \geq p} = \{x_k | k \geq p\}$.
\end{defn}

\begin{remark}
	While the notation $\{x_k\}_{k \geq p}$ is more commonly used, since this set of notes works a lot between sequences and sets, we shall use the notation $\langle x_k \rangle_{k \geq p}$ to denote a sequence instead to make a clear distinction between the two.
\end{remark}

\begin{defn}[Subsequence]
	Let $\langle x_k \rangle_{k \geq p}$ be a sequence. A subsequence of $\langle x_k \rangle_{k \geq p}$ is a sequence of the form $\langle x_{k_n} \rangle_{n \in \bb{N}}$ such that $k_1 < k_2 < k_3 < \hdots$ and $x_{k_1} < x_{k_2} < x_{k_3} < \hdots$, where $x_{k_l} = x_m$ for all $n \geq l \in \bb{N}$ and a unique $k \geq m \in \bb{Z}_{\geq p}$.
\end{defn}

\begin{remark}
	In other words, a subsequence $\langle x_{k_n} \rangle_{n \in \bb{N}}$ is constructed from $\langle x_k \rangle_{k \geq p}$ by "removing" from $x_p, x_{p + 1}, x_{p + 2}, ...$ all the $x_m$'s except for those such that $m = k_l$ for some l.
\end{remark}

\begin{defn}[Extended Ordered Field]
	Let F be an ordered field. We can define the extended ordered field $\hat{F}$ to be the set $\hat{F} = F \cup \{-\infty, \infty\}$, such that $\forall a \in F, -\infty < a < \infty$.

	We also define, $\forall a \in F$:
	\begin{itemize}
		\item $a + \infty = \infty$,
		\item $a - \infty = -\infty$,
		\item if $a > 0$, then $a \cdot \infty = \infty$, and
		\item if $a < 0$, then $a \cdot \infty = -\infty$.
	\end{itemize}

	We define some indeterminant forms:
	\[
		\infty - \infty, \; \infty \cdot 0, \; \frac{\infty}{\infty}, \; \frac{\infty}{0}, \; \frac{0}{\infty}
	\]
	We extend the order relation < on F such that $-\infty < \infty$.
\end{defn}

\begin{defn}[Convergence, Divergence and Limits of a Sequence]
	Let F be an extended ordered field. and $\langle x_k \rangle_{k \geq p}$ be a sequence in F. For $a \in F$, we say that the sequence $\langle x_k \rangle_{k \geq p}$ converseges to a (or that the limit of $\langle x_k \rangle_{k \geq p}$ is equal to a), and we write $x_k \to a (\text{as } k \to \infty)$, or we write $\lim_{k \to \infty} = a$, when
	\[
		\forall 0 < \epsilon \in F \; \exists m \in \bb{Z} \; \forall k \in \bb{Z}_{\geq p} \; (k \geq m \implies |x_k - a| \leq \epsilon).
	\]
	We say that the sequence $\langle x_k \rangle_{k \geq p}$ diverges (in F) when it does not converge (to any $a \in F$). We say that $\langle x_k \rangle_{k \geq p}$ diverges to infinity, or that the limit of $\langle x_k \rangle_{k \geq p}$ is equal to infinity, and we write $x_k \to \infty (\text{as } k \to \infty$, or we write $\lim_{k \to \infty}x_k = \infty$, when
	\[
		\forall r \in F \; \exists m \in \bb{Z} \; \forall k \in \bb{Z}_{\geq p} \; (k \geq m \implies x_k \geq r)
	\]
	Similarly, we say that $\langle x_k \rangle_{k \geq p}$ diverges to $-\infty$, or that the limit of $\langle x_k \rangle_{k \geq p}$ is equal to negative infinity, and we write $x_k \to -\infty (\text{as } k \to \infty)$, or we write $\lim_{k \to \infty}x_k = -\infty$, when
	\[
		\forall r \in F \; \exists m \in \bb{Z} \; \forall k \in \bb{Z}_{\geq p} \; (k \geq m \implies x_k \leq r)
	\]
\end{defn}

\begin{thm}[Independence of Limit from Initial Terms]
	Let $\langle x_k \rangle_{k \geq p}$ be a sequence in a subfield F of $\bb{R}$.
	\begin{enumerate}
		\item If $q \geq p$ and $y_k = x_k$ for all $k \geq q$, then $\langle x_k \rangle_{k \geq p}$ converges iff $\langle y_k \rangle_{k \geq q}$ converges, and in this case $\lim_{k \to \infty} x_k = \lim_{k \to \infty} y_k$.

		(Note that in this statement, $\langle y_k \rangle_{k \geq q}$ is a subsequence of $\langle x_k \rangle_{k \geq p}$, such that it takes on all the elements of the sequence after some $q \geq p$.)
		\item If $l \geq 0$ and $y_k = x_{k + l}$ for all $k \geq p$, then $\langle x_k \rangle_{k \geq p}$ converges iff $\langle y_k \rangle_{k \geq p}$ converges, and in this case $\lim_{k \to \infty} x_k = \lim_{k \to \infty} y_k$.

		(Note that in this statement, $\langle x_k \rangle_{k \geq p}$ is a subsequence of $\langle y_k \rangle_{k \geq p}$ instead, such that $\langle x_k \rangle_{k \geq p}$ takes on all the values of $\langle y_k \rangle_{k \geq p}$ from k + l.)
	\end{enumerate}
\end{thm}

\begin{remark}
	Because of the above theorem, we often simply denote $\langle x_k \rangle_{k \geq p}$ as $\langle x_k \rangle$
\end{remark}

\begin{thm}[Uniqueness of Limit]
	Let $\langle x_k \rangle$ be a sequence in an ordered field F. If $\langle x_k \rangle$ has a limit (finite or infinite) then its limit is unique.
\end{thm}


\section{Limit Theorems}

\begin{thm}[Basic Limits]
	In any ordered field F, for $a \in F$ we have
	\[
		\lim_{k \to infty} a = a, \quad \lim_{k \to \infty} k = \infty, \quad \lim_{k \to \infty} \frac{1}{k} = 0
	\]
\end{thm}

\begin{thm}[Operations on Limits]
	Let $\langle x_k \rangle$ and $\langle y_k \rangle$ be sequnces in an ordefred field F and let $c \in F$. SPS that $\langle x_k \rangle$ and $\langle y_k \rangle$ both converge with $x_k \to a$ and $y_k \to b$. Then
	\begin{enumerate}
		\item $ cx_k \to ca$
		\item $(x_k + y_k) \to a + b$
		\item $(x_k - y_k) \to a - b$
		\item $x_k y_k \to ab$
		\item If $b \neq 0$, then $\frac{x_k}{y_k} \to \frac{a}{b}$
	\end{enumerate}
\end{thm}

\begin{thm}[Extended Operations on Limits]
	Let $\langle x_k \rangle$ and $\langle y_k \rangle$ be sequences in F. SPS that $\lim_{k \to \infty} x_k = u$ and $\lim_{k \to \infty} y_k = v$, where $u, v \in \hat{F}$.
	\begin{enumerate}
		\item If u + v is defined in $\hat{F}$, then $\lim_{k \to \infty} (x_k + y_k) = u + v$.
		\item If u - v is defined in $\hat{F}$, then $\lim_{k \to \infty} (x_k - y_k) = u - v$.
		\item If uv is defined in $\hat{F}$, then $\lim_{k \to \infty} x_k y_k = uv$.
		\item If $\frac{u}{v}$ is defined in $\hat{F}$, then $\lim_{k \to \infty} \frac{x_k}{y_k} = \frac{u}{v}$
	\end{enumerate}
\end{thm}

\begin{thm}[Monotonic Surjective Functions]
	Let I and J be intervqals in a subfield $F \subseteq \bb{R}$. SPS $f: I \to J$ is increasing and surjective. Let $\langle x_k \rangle$ be a sequence in I. Then
	\begin{enumerate}
		\item If $x_K \to a \in I$, then $f(x_k) \to f(a) \in J$.
		\item If $x_k \to u \in F \cup \{\infty\}$ is the right endpoint of I, then $f(x_k) \to v \in F \cup \{\infty\}$ is the right endpoint in J.
		\item If $x_k \to u \in F \cup \{-\infty\}$ is the left endpoint of I, then $f(x_k) \to v \in F \cup \{-\infty\}$ is the left endpoint in J.
	\end{enumerate}
\end{thm}

\begin{thm}[Basic Elementary functions Acting on Limits]
	Let $\langle x_k \rangle$ be a sequence in $\bb{R}$ ad let $b \in \bb{R}$. Then
	\begin{enumerate}
		\item $x_k \to a > 0 \implies x_k^b \to a^b$ and \\
			\[
				x_k \to \infty \implies \lim_{k \to \infty} x_k^b =
				\begin{cases}
					\infty 	& b > 0 \\
					1		& b = 0 \\
					0 		& b < 0
				\end{cases}
			\]
		\item $(x_k \to a \land b > 0) \implies b^{x_k} \to b^a$ and \\
			\[
				(x_k \to \infty \land b > 0) \implies \lim_{k \to \infty} b^{x_k} =
				\begin{cases}
					\infty 	& b > 1 \\
					1		& b = 1 \\
					0		& 0 < b < 1
				\end{cases}
			\]
		\item $(x_k \to a > 0 \land b > 0) \implies \log_b x_k \to \log_b a$ and \\
			\[
				(x_k \to \infty \land b > 0) \implies \lim_{k \to \infty} \log_b x_k = 
				\begin{cases}
					\infty 	& b > 1 \\
					0		& b = 1 \\
					-\infty & 0 < b < 1
				\end{cases}
			\]
		\item $x_k \to a \implies (\sin x_k \to \sin a \land \cos x_k \to \cos a)$ and \\
			  $(x_k \to a (\forall t \in \bb{Z} \> a \neq \frac{\pi}{2} + 2 \pi t)) \implies \tan x_k \to \tan a$
		\item $x_k \to a \in [-1, 1] \implies (\arcsin x_k = \arcsin a \land \arccos x_k = \arccos a)$, \\
			  $x_k \to a \implies \arctan x_k \to \arctan a$, \\
			  $x_k \to \infty \implies \arctan x_k \to \frac{\pi}{2}$, and \\
			  $x_k \to -\infty \implies \arctan x_k \to -\frac{\pi}{2}$
	\end{enumerate}
\end{thm}

\begin{thm}[Comparison Theorem for Sequences]
	Let $\langle x_k \rangle$ and $\langle y_k \rangle$ be sequences in a subfield $F \subseteq \bb{R}$. SPS that $x_k \leq y_k$ for all k. Then
	\begin{enumerate}
		\item $(x_k \to a \land y_k \to b) \implies a \leq b$
		\item $x_k \to \infty \implies y_k \to \infty$
		\item $y_k \to -\infty \implies x_k \to -\infty$
	\end{enumerate}
\end{thm}

\begin{thm}[Squeeze Theorem for Sequences]
	Let $\langle x_k \rangle$, $\langle y_k \rangle$ and $\langle z_k \rangle$ be sequences in a subfield $F \subseteq \bb{R}$.
	\begin{enumerate}
		\item $(\forall k \in \bb{N} \; x_k \leq y_k \leq x_k \> \land \> x_k \to a \land z_k \to a) \implies y_k \to a$
		\item $(\forall k \in \bb{N} \; |x_k| \leq y_k \> \land \> y_k \to 0) \implies x_k \to 0$
	\end{enumerate}
\end{thm}

\begin{defn}[Bounds]
	Let $\langle x_k \rangle$ be a sequence in an ordered set X. We say that
	\begin{enumerate}
		\item $\langle x_k \rangle$ is bounded above iff the set $\{x_k | n \in \bb{N} \}$ is bounded above;
		\item $\langle x_k \rangle$ is bounded below iff the set $\{x_k | n \in \bb{N} \}$ is bounded below.
	\end{enumerate}
\end{defn}


\section{Bolzano-Weierstrass Theorem}

\begin{defn}[Increasing, Decreasing, and Monotonic Sequences]
	Let $\langle x_k \rangle$ be a sequence in a subfiend $F \subseteq \bb{R}$. We say that
	\begin{enumerate}
		\item $\langle x_k \rangle$ is increasing iff $\forall k, l \in \bb{Z}_{\geq p} \> (k \leq l \implies x_k \leq x_l)$
		\item $\langle x_k \rangle$ is strictly increasing iff $\forall k, l \in \bb{Z}_{\geq p} \> (k < l \implies x_k < x_l)$
		\item $\langle x_k \rangle$ is decreasing iff $\forall k, l \in \bb{Z}_{\geq p} \> (k \leq l \implies x_k \geq x_l)$
		\item $\langle x_k \rangle$ is strictly decreasing iff $\forall k, l \in \bb{Z}_{\geq p} \> (k < l \implies x_k > x_l)$
	\end{enumerate}
	We say that $\langle x_k \rangle$ is monotonic when it is either increasing or decreasing only.
\end{defn}

\begin{thm}[Monotone Convergence Theorem]
	Let $\langle x_k \rangle$ be a sequence in $\bb{R}$.
	\begin{enumerate}
		\item SPS $\langle x_k \rangle$ is increasing. If $\langle x_k \rangle$ is bounded above, then $x_k \to \sup \{x_k\}$. If $\langle x_k \rangle$ is not bounded above, then $x_k \to \infty$.
		\item SPS $\langle x_k \rangle$ is decreasing. If $\langle x_k \rangle$ is bounded below, then $x_k \to \inf \{x_k\}$. If $\langle x_k \rangle$ is not bounded below, then $x_k \to -\infty$.
	\end{enumerate}
\end{thm}

\begin{thm}[Nested Interval Theorem]
	Let $I_0, I_1, I_2, ...$ be nonempty, closed, and bounded intervals in $\bb{R}$. SPS $I_0 \supseteq I_1 \supseteq I_2 \supseteq \hdots$, then $\bigcap_{k = 0}^\infty I_k \neq \emptyset$. Moreover, if the lengths of these intervals satisfy $|I_k| \to 0$ as $k \to \infty$, then $\bigcap_{k = 0}^\infty I_k$ is a single point.
\end{thm}

\begin{defn}[Rearrangement of a Sequence]
	Let $\langle x_k \rangle_{k \geq p}$ be a sequence in a set X. Given a bijective function $F: \bb{Z}_{\geq q} \to \bb{Z}_{\geq p}$ such that $f(l) = k_l$ and let $y_l = x_{k_l}$ for $l \geq q$. Then the sequence $\langle y_l \rangle_{l \geq q}$ is called a rearrangement of the sequence $\langle x_k \rangle_{k \geq p}$.
\end{defn}

\begin{thm}[Convergence of Subsequences and Rearrangements]
	Let $\langle x_k \rangle$ be a sequence in a subfield $F \subseteq \bb{R}$. SPS that $x_k \to a$. Then
	\begin{enumerate}
		\item every subsequence of $\langle x_k \rangle$ converges to a; and
		\item every rearrangement of $\langle x_k \rangle$ converges to a.
	\end{enumerate}
\end{thm}

\begin{thm}[Bolzano-Weierstrass Theorem]
	Every bounded sequence of $\bb{R}$ has a convergent subsequence.
\end{thm}


\section{Cauchy Sequences}

\begin{defn}[Cauchy]
	Let $\langle x_k \rangle_{k \geq p}$ be a seuqnece in a subfield $F \subseteq \bb{R}$. We say that $\langle x_k \rangle$ is Cauchy when
	\[
		\forall \epsilon > 0 \; \exists m \in \bb{Z} \; \forall k, l \in \bb{Z}_{k \geq p}\; (k, l \geq m \implies |x_k - x_l| \leq \epsilon)
	\]
\end{defn}

\begin{thm}[Cauchy Criterion for Convergence]
	Let $\langle x_k \rangle$ be a sequnce of $\bb{R}$. Then $\langle x_k \rangle$ is Cauchy iff $\langle x_k \rangle$ converges (to some point $a \in \bb{R}$).
\end{thm}


% =================================
% 		Chapter 3
% =================================
\chapter{Functions on \texorpdfstring{$\bb{R}$}{R}}


\section{Two-Sided Limits}

\begin{defn}[Limit Point]
	Let $U \subseteq F$ where F is an ordered field. Let $f: U \to F$. For $a \in F$ we say that a is a limit point of U when
	\[
		\forall \epsilon > 0 \; \exists x \in U \; 0 < |x - a| < \epsilon
	\]

	When a is a limit point of A, we make the following definitions.
	\begin{enumerate}
		\item For $b \in F$ we say that the limit of f(x) as x tends to a is equal to b, and write $\lim\limits_{x \to a} f(x) = b$ when
			\[
				\forall \epsilon > 0 \; \exists \delta > 0 \; \forall x \in U \; (0 < |x - a| \leq \delta \implies |f(x) - b| \leq \epsilon).
			\]
		\item We say that the limit of f(x) as x tends to a is equal to infinity, and write $\lim\limits_{x \to a} f(x) = \infty$ when
			\[
				\forall r \in F \; \exists \delta > 0 \; \forall x \in U \; (0 < |x - a| \leq \delta \implies f(x) \geq r).
			\]
		\item We say that the limit of f(x) as x tends to a is equal to negative infinity, and write $\lim\limits_{x \to a}f(x) = -\infty$ when
			\[
				\forall r \in F \; \exists \delta > 0 \; \forall x \in U \; (0 < |x - a| \leq \delta \implies f(x) \leq r).
			\]
	\end{enumerate}
\end{defn}

\begin{defn}[Limit Point from Above and Below]
	Let $U \subseteq F$ where F is an ordered field. Let $f: U \to F$.

	For $a \in F$, we say that a is a $\textbf{limit point from below}$ when
	\[
		\forall \delta > 0 \; \exists x \in U \; a - \delta < \leq x < a
	\]
	When a is a limit point of U from below and $b \in F$, we define:
	\begin{enumerate}
		\item $\lim\limits_{x \to a^-} f(x) = b \iff \forall \epsilon > 0 \; \exists \delta > 0 \; \forall x \in U \; (a - \delta \leq x < a \implies |f(x) - b| \leq \epsilon)$.
		\item $\lim\limits_{x \to a^-} f(x) = \infty \iff \forall r \in F \; \exists \delta > 0 \; \forall x \in U \; (a - \delta \leq x < a \implies f(x) \geq r)$.
		\item $\lim\limits_{x \to a^-} f(x) = -\infty \iff \forall r \in F \; \exists \delta > 0 \; \forall x \in U \; (a - \delta \leq x < a \implies f(x) \leq r)$.
	\end{enumerate}

	For $a \in F$, we say that a is a $\textbf{limit point from above}$ when
	\[
		\forall \delta > 0 \; \exists x \in U \; a < x \leq a + \delta
	\]
	When a is a limit point of U from above and $b \in F$, we define:
	\begin{enumerate}
		\item $\lim\limits_{x \to a^+} f(x) = b \iff \forall \epsilon > 0 \; \exists \delta > 0 \; \forall x \in U \; (a < x \leq a + \delta \implies |f(x) - b| \leq \epsilon)$.
		\item $\lim\limits_{x \to a^+} f(x) = \infty \iff \forall r \in F \; \exists \delta > 0 \; \forall x \in U \; (a < x \leq a + \delta \implies f(x) \geq r)$.
		\item $\lim\limits_{x \to a^+} f(x) = -\infty \iff \forall r \in F \; \exists \delta > 0 \; \forall x \in U \; (a < x \leq a + \delta \implies f(x) \leq r)$			
	\end{enumerate}
\end{defn}

\begin{defn}[Infinity As A Limit Point]
	Let $U \subseteq F$ where F is an ordered field. Let $f: U \to F$.

	We say that infinity is a limit point (from below) when U is not bounded above, i.e $\forall m \in F \; \exists x \in U \; x \geq m$. When U is not bounded above and $b \in F$, we make the following definitions:
	\begin{enumerate}
		\item $\lim\limits_{x \to \infty} f(x) = b \iff \forall \epsilon > 0 \; \exists m \in F \; \forall x \in U \; (x \geq m \implies |f(x) - b| \leq \epsilon).$
		\item $\lim\limits_{x \to \infty} f(x) = \infty \iff \forall r \in F \; \exists m \in F \; \forall x \in U \; (x \geq m \implies f(x) \geq r).$
		\item $\lim\limits_{x \to \infty} f(x) = -\infty \iff \forall r \in F \; \exists m \in F \; \forall x \in U \; (x \geq m \implies f(x) \leq r).$
	\end{enumerate}

	We say that negative infinity is a limit point (from above) when U is not bounded below, i.e. $\forall m \in F \; \exists x \in U \; x \leq m$. When U is not bounded below and $b \in F$, we make the following definitions:
	\begin{enumerate}
		\item $\lim\limits_{x \to -\infty} f(x) = b \iff \forall \epsilon > 0 \; \exists m \in F \; \forall x \in U \; (x \leq m \implies |f(x) - b| \leq \epsilon).$
		\item $\lim\limits_{x \to -\infty} f(x) = \infty \iff \forall r \in F \; \exists m \in F \; \forall x \in U \; (x \leq m \implies f(x) \geq r).$
		\item $\lim\limits_{x \to -\infty} f(x) = -\infty \iff \forall r \in F \; \exists m \in F \; \forall x \in U \; (x \leq m \implies f(x) \leq r).$
	\end{enumerate}
\end{defn}

\begin{thm}[Two-sided Limits]
	Let F be a subfield of $\bb{R}$. Let $A \subseteq F$. Let $f: A \to F$. Let $a \in F$. SPS that a is a limit point of A both from above and below. Then $\forall u \in F$, we have $\lim\limits_{x \to a} f(x) = u \iff \lim\limits_{x \to a^-} f(x) = u = \lim\limits_{x \to a^+} f(x)$.
\end{thm}

\begin{thm}[Sequential Characterization of Limits of Functions]
	Let F be a subfield of $\bb{R}$, let $A \subseteq F$, let $f: A \to F$ and $u \in F$.
	\begin{enumerate}
		\item When $a \in F$ is a limit point of A, $\lim\limits_{x \to a} f(x) = u$ iff for every sequence $\langle x_k \rangle$ in $A \setminus \{a\}$ with $x_k \to a$ we have $f(x_k) \to u$.
		\item When a is a limit point of A from below, $\lim\limits_{x \to a^-} f(x) = u$ iff for every sequence $\langle x_k \rangle$ in $\{x \in A | x < a\}$ with $x_k \to a$ we have $f(x_k) \to u$.
		\item When a is a limit point of A from above, $\lim\limits_{x \to a^+} f(x) = u$ iff for every sequence $\langle x_k \rangle$ in $\{x \in A | x > a\}$ with $x_k \to a$ we have $f(x_k) \to u$.
		\item When A is not bounded above, $\lim\limits_{x \to \infty} f(x) = u$ iff for every sequence $\langle x_k \rangle$ in A with $x_k \to \infty$ we have $f(x_k) \to u$.
		\item When A is not bounded below, $\lim\limits_{x \to -\infty} f(x) = u$ iff for every sequence $\langle x_k \rangle$ in A with $x_k \to -\infty$ we have $f(x_k) \to u$.
	\end{enumerate}
\end{thm}

\begin{remark}
	It follows form the Sequential Characterization of Limits of Functions that all the theorems about limits of sequences (see \hyperref[chp2]{Chapter 2}) imply analogous theorems in the more general setting of limits of functions. We will state those theorems for easier reference.
\end{remark}

\begin{thm}[Local Determination of Limits]
	Let F be a subfield of $\bb{R}$, let $A, B \subseteq F$, let $f: A \to F$ and $g: B \to F$. SPS that $a \in F$ is a limit point of both sets A and B, and that for some $\delta > 0$ we have $C = \{x \in A \; | \; 0 < |x - a| \leq \delta\} \subseteq \{x \in B \; | \; 0 < |x - a| \leq \delta\}$ and that f(x) = g(x) for all $x \in C$. Then for $u \in \hat{F}$
		\[
			\lim_{x \to a} g(x) = u \iff \lim_{x \to a} f(x) = u
		\]
	Analogous results hold for limits $x \to a^\pm$ and $x \to \pm \infty$.
\end{thm}

\begin{thm}[Uniquness of Limits]
	Let F be a subfield of $\bb{R}$, let $A \subseteq F$, let $f: A \to F$ and let $a \in F$ be a limit point of A. For $u, v \hat{F}$,
		\[
			\lim_{x \to a} f(x) = u \land \lim_{x \to a} f(x) = v \implies u = v
		\]
	Analogous result holds for limits $x \in a^\pm$ and $x \to \pm \infty$.
\end{thm}

\begin{thm}[Extended Operations on Limits]
	Let F be a subfield of $\bb{R}$, let $A \subseteq F$, let $f, g: A \to F$, and let $a \in F$ be a limit point of A. Let $u ,v \in \cap{F}$, and SPS that $\lim_{x \to a} f(x) = u$ and $\lim_{x \to a} g(x) = v$. Then
	\begin{enumerate}
		\item $u \pm v \in \hat{F} \implies \lim_{x \to a} (f \pm g)(x) = u \pm v$
		\item $uv \in \hat{F} \implies \lim_{x \to a} (f \cdot g)(x) = uv$
		\item $\frac{u}{v} \in \hat{F} \implies \lim_{x \to a} (\frac{f}{g})(x) = \frac{u}{v}$
	\end{enumerate}
	Analogous results hold for limits $x \to a^\pm$ and $x \to \pm \infty$.
\end{thm}

\begin{thm}[Basic Elementary Functions Acting on Limits]
	For $A \subseteq \bb{R}$, let $f: A \to \bb{R}$ and $a, b \in \bb{R}$ with a as a limit point of A. We have
	\begin{enumerate}
		\item $\lim_{x \to a} f(x) = b > 0 \implies \lim_{x \to a} f(x)^c = b^c$ \\
			  $\lim_{x \to a} f(x) = \infty \implies \lim_{x \to a} f(x)^c = \begin{cases}
			  	\infty 	& c > 0 \\
			  	1		& c = 0 \\
			  	0		& c < 0
			  \end{cases}$ \\
			  $\forall x \in A \; (f(x) > 0 \land \lim_{x \to a} f(x) = 0) \implies \lim_{x \to a} f(x)^c = \begin{cases}
			  	0		& c > 0 \\
			  	1		& c = 0 \\
			  	\infty 	& c < 0
			  \end{cases}$
		\item $\lim_{x \to a} f(x) = \infty \land c > 0 \implies \lim_{x \to a} c^{f(x)} = \begin{cases}
				\infty		& c > 1 \\
				1			& c = 1 \\
				0 			& c < 1
			  \end{cases}$ \\
			  $\lim_{x \to a} f(x) = b \land c > 0 \implies \lim_{x \to a} c^{f(x)} = c^b$ \\
			  $\lim_{x \to a} f(x) = -\infty \land c > 0 \implies \lim_{x \to a} c^{f(x)} = \begin{cases}
			  	\infty 		& c < 1 \\
			  	1			& c = 1 \\
			  	0 			& c > 1
			  \end{cases}$
		\item $\lim_{x \to a} f(x) = b > 0 \land c > 0 \implies \lim_{x \to a} \log_c f(x) = \log_c b$ \\
			  $\lim_{x \to a} f(x) = \infty \land c > 0 \implies \lim_{x \to a} \log_c f(x) = \begin{cases}
			  	\infty 	& c > 1 \\
			  	0 		& c = 1 \\
			  	-\infty & c < 1
			  \end{cases}$ \\
			  $\forall x \in A \; (f(x) > 0 \land \lim\limits_{x \to a} f(x) = 0 \land 1 \neq c > 0) \implies \lim\limits_{x \to a} \log_c f(x) = \begin{cases}
			  	-\infty 	& c > 1 \\
			  	\infty 		& c < 1
			  \end{cases}$
		\item $\lim_{x \to a} f(x) = b \implies (\lim_{x \to a} \sin f(x) = \sin b \land \lim_{x \to a} \cos f(x) = \cos b)$ \\
			  The limits $\lim_{x \to \pm \infty} \sin x, \lim_{x \to \pm \infty} \cos x, \text{ and } \lim_{x \to \pm \infty} \tan x$ do not exists.
		\item $\forall x \in A \; (f(x) \in [-1, 1] \land \lim_{x \to a} f(x) = b) \implies \lim_{x \to a} \arcsin f(x) = \arcsin b$. \\
			  $\lim_{x \to a} f(x) = b \in \bb{R} \implies \lim_{x \to a} \arctan f(x) = \arctan b$ \\
			  $\lim_{x \to a} f(x) = \infty \implies \lim_{x \to a} \arctan f(x) = \frac{\pi}{2}$ and \\
			  $\lim_{x \to a} f(x) = -\infty \implies \lim_{x \to a} \arctan f(x) = -\frac{\pi}{2}$.
	\end{enumerate}

	Similar results hold for limits $x \to a^\pm$ and $x \to \pm \infty$ (unless stated otherwise in the above statements).
\end{thm}

\begin{thm}[Comparison Theorem for Functions]
	Let F be a subfield of $\bb{R}$, let $A \subseteq F$, let $f, g: A \to F$ and let $a \in F$ be a limit point of A. SPS that $\forall x \in A \; f(x) \leq g(x)$. Then
	\begin{enumerate}
		\item $\exists u, v \in \hat{F} \; (\lim\limits_{x \to a} f(x) = u \land \lim\limits_{x \to a} g(x) = v) \implies u \leq v.$
		\item $\lim\limits_{x \to a} f(x) = \infty \implies \lim\limits_{x \to a} g(x) = \infty,$ and
		\item $\lim\limits_{x \to a} g(x) = -\infty \implies \lim\limits_{x \to a} f(x) = -\infty.$
 	\end{enumerate}
 	Similar results hold for when $x \to a^\pm$ and $x \to \pm \infty$.
\end{thm}

\begin{thm}[Squeeze Theorem for Functions]
	Let F be a subfield of $\bb{R}$, let $A \subseteq F$, let $f, g, h: A \to F$, let $b \in F$ and let a be a limit point of A. We have that
	\begin{enumerate}
		\item $\forall x \in A \; \Big(f(x) \leq g(x) \leq h(x) \land \lim\limits_{x \to a} f(x) = b = \lim\limits_{x \to a} h(x) \Big) \implies \lim\limits_{x \to a} g(x) = b.$
		\item $\forall x \in A \; \Big(|f(x)| \leq g(x) \land \lim\limits_{x \to a} g(x) = 0 \Big) \implies \lim\limits_{x \to a} f(x) = 0$
	\end{enumerate}
\end{thm}


\section{Continuity}

\begin{defn}[Continuity]
	Let F be a subfield of $\bb{R}$, let $A \in F$, and let $f: A \to F$. For all $a \in A$, we say that f is continuous at a iff
	\[
		\forall a \in A \; \forall \epsilon > 0 \; \exists \delta > 0 \; \forall x \in A \; (|x - a| \leq \delta \implies |f(x) - f(a)| \leq \epsilon)
	\]
	(Note that $\delta$ depends on $a, f, \text{ and, especially, } \epsilon$ in general).

	f is said to be continuous (in A) when f is continuous at every point $a \in A$.
\end{defn}

\begin{defn}[Limit Point and Continuity]
	Let F be a subfield of $\bb{R}$, let $A \subseteq F$, let $f: A \to F$ and let $a \in A$. Then
	\begin{enumerate}
		\item if a is not a limit point of A then f is continuous on a; and
		\item if a is a limit point of A, then f is continuous on a iff $\lim_{x \to a} f(x) = f(a)$.
	\end{enumerate}
\end{defn}

\begin{thm}[Sequential Characterization of Continuity]
	Let F be a subfield of $\bb{R}$, let $A \subseteq F$, let $f: A \to F$ and let $a \in A$. Then f is continuous at a iff for every sequence $\langle x_k \rangle$ in A with $x_k \to a$ we have $f(x_k) \to f(a)$.
\end{thm}

\begin{thm}[Operations on Continuous Functions]
	Let F be a subfield of $\bb{R}$, let $A \subseteq F$, let $f, g: A \to F$, let $a \in A$ and let $c \in F$. SPS that f and g are continuous at a. Then the functions cf, f $\pm$ g and fg are call continuous at a, and if $g(a) \neq 0$, then the function $\frac{f}{g}$ is continuous at a.
\end{thm}

\begin{thm}[Composition of Continuous Functions]
	Let F be a subfield of $\bb{R}$, let $A, b \subseteq \bb{R}$, and let $f: A \to \bb{R}$ and $g: B \to \bb{R}$. Let $h = g \circ f : C \to \bb{R}$ where $C = A \cap f^{-1}(B)$
	\begin{enumerate}
		\item If f is continuous at $a \in C$ and g is continuous at f(a), then h is continuous at a.
		\item If f is continuous (in A) and g is continuous (in B), then h is continuous (in C).
	\end{enumerate}
\end{thm}

\begin{crly}[Continuity of Elementary Functions]
	Every elementary function is continuous (in their respective domain).
\end{crly}

\begin{thm}[Functions Acting on Limits]
	Let F be a subfield of $\bb{R}$, let $A, B \subseteq F$, let $f: A \to F$ and $g: B \to F$, and let $h = g \circ f : C \to F$ where $C = A \cap f^{-1}(B)$. Let s be a limit point of C (hence also of A) and let b be a limit point of B. Let $c \in F$. SPS that $\lim\limits_{x \to a} f(x) = a \land \lim\limits_{y \to b} g(y) = c$. SPS either that $\forall x \in C \setminus \{a\} \; f(x) \neq b$ or g is continuous at b. Then $\lim\limits_{x \to a} h(x) = c$.
\end{thm}

\begin{thm}[Intermediate Value Theorem]
	Let I be an interval in $\bb{R}$ and let $f: I \to \bb{R}$ be continuus (in I). Let $a, b \in I$ with $a \leq b$ and let $y \in \bb{R}$.
	\[
		\min \{f(a), f(b)\} \leq y \leq \max \{f(a), f(b)\} \implies \exists x \in [a, b] \; (f(x) = y)
	\]
\end{thm}

\begin{defn}[Maximum, Minimum and Extreme Values]
	Let F be a subfield of $\bb{R}$, let $A \subseteq F$, and let $f: A \to F$. For $a \in A$, fi $f(x) \geq f(x)$ for every $x \in A$, then we say that f(a) is the maximum value of f or that f attains its maximum value at a. Similarly for $b \in A$, if $f(b) \leq f(x)$ for every $x \in A$, we say that f(b) is the minimum value of f, or that f attains its minimum at b.

	We say that f attains its extreme values in A when f attains its maximum value at some point $a \in A$ and f attains its minimum value at some point $b \in A$.
\end{defn}

\begin{thm}[Extreme Value Theorem]\label{EVT}
	Let $, b \in \bb{R}$ with $a < b$, and let $f: [a, b] \to \bb{R}$ be continuous. Then f attains its extreme values in $[a, b]$.
\end{thm}

\begin{defn}[Uniform Continuity]
	Let F be a subfield of $\bb{R}$, let $A \subseteq F$, and let $f: A \to F$. We say that f is uniformly continuous in A when
	\[
		\forall \epsilon > 0 \; \exists \delta > 0 \; \forall a \in A \; \forall x \in A \; (|x - a| \leq \delta \implies |f(x) - f(a)| \leq \epsilon)
	\]
\end{defn}

\begin{thm}[Closed Bounded Intervals and Uniform Continuity]
	Let $a, b \in \bb{R}$ with a < b and let $f: [a, b] \to \bb{R}$. If f is continuous (on [a, b]) then f is uniformly continuous (on [a, b]).
\end{thm}

% =================================
% 		Chapter 4
% =================================
\chapter{Differentiability on \texorpdfstring{$\bb{R}$}{R}}


\section{The Derivative}

\begin{defn}[Differentiable on a Point]
	Let F be a subfield of $\bb{R}$, let $A \subseteq F$, and let $a \in A$ be a limit point of A. We say that f is differentiable at a when the limit
	\[
		\lim_{x \to a} \frac{f(x) - f(a)}{x - a}
	\]
	exists in F. In this case we call the limit the derivative of f at a, and we denote that by $f'(a)$, so we have
	\[
		f'(a) = \lim_{x \to a} \frac{f(x) - f(a)}{x - a}.
	\]
	When $a \in A$ is a limit point of A from the right, we sya that f is differentiable from the right at a and that $f'_+(a)$ is the derivative from the right of f at a, when
	\[
		f'_+(a) = \lim_{x \to a^+} \frac{f(x) - f(a)}{x - a}.
	\]
	Similarly, when $a \in A$ is a limit point of A from the left, we say that f is differentiable from the left at a and that $f'_{-}(a)$ is the derivative from the left of f at a when
	\[
		f'_{-}(a) = \lim_{x \to a^-} \frac{f(x) - f(a)}{x - a}.
	\]
\end{defn}

\begin{defn}[Differentiable in a Domain]
	We say that f is differentiable (in A) when f is differentiable at every point $a \in A$, or that f is in $C^1$. In this case, the derivative of f is the function $f': A \to F$ defined by
	\[
		f'(x) = \lim_{u \to x} \frac{f(u) - f(x)}{u - x}.
	\]
\end{defn}

\begin{defn}[Differentiable n-times \& n-th Derivative]
	When $f'$ is differentiable at a, we denote the derivative of $f'$ at a by $f''(a)$, and we call $f''(a)$ the second derivative of f at a. When $f''(a)$ exists for every a in A, we say that f is twice differentiable (in A), or that it is in $C^2$, and we call the function $f'': A \to F$ is called the second derivative of f. Similarly, $f'''(a)$ is he derivative of $f''$ at a, and so on.

	More generally, for any function $f: A \to F$, we define its derivative to be the function $f': B \to F$ where $B = \{a \in A | f \text{ is differentaible at a}\}$, and we define its second derivative to be the function $f'': C \to F$ where $C = \{a \in B | f' \text{ is differentiable at a}\}$, and so on.
\end{defn}

\begin{remark}
	Note that
	\[
		\lim_{x \to a} \frac{f(x) - f(a)}{x - a} = \lim_{h \to 0} \frac{f(a + h) - f(a)}{h}.
	\]
	To be precise, the limit on the left exists in F iff the limit on the right exists in F, and in this case the two limits are equal.
\end{remark}

\begin{thm}[Definition of Differentiablity in $\epsilon$-$\delta$]
	Let F be a subfield of $\bb{R}$, let $A \subseteq F$, let $f: A \to F$, and let $a \in A$ be a limit point of A. Then f is differentiable at a with derivative $f'(a) \iff$
	\[
		\forall \epsilon > 0 \; \exists \delta > 0 \; \forall x \in A \; (|x - a| \leq \delta \implies |f(x) - f(a) - f'(a)(x-a)| \leq \epsilon)
	\]
\end{thm}

\begin{defn}[Linearization \& Tangent Line]
	When $f: A \to F$ is differentiable at a with derivative $f'(a)$, the function
	\[
		l(x) = f(a) + f'(a)(x-a)
	\]
	is called the linearization of f at a.

	Note that the graph $y = l(x)$ of the linearization is the line through the point $(a, f(a))$ with slope $f'(a)$. This line is called the tangent line to the graph $y = f(x)$ at the point $(a, f(a))$.
\end{defn}

\begin{thm}[Differentiablility $\implies$ Continuity]
	Let F be a subfield of $\bb{R}$, let $A \subseteq F$, let $f: A \to F$ and let $a \in A$ be a limit point of A.
	\[
		\text{f is differentiable at a} \implies \text{f is continuous at a}.
	\]
\end{thm}


\section{Differentiability Theorems}

\begin{thm}[Local Determination of the Derivative]
	Let F be a subfield of $\bb{R}$, let $A, B \subseteq F$, let $f: A \to F$ and $g:B \to F$, and let $a \in A \cap B$ be a limit point of both A and B. SPS that for some $\delta > 0$ we have $\{ x \in A \; | \; |x - a| \leq \delta \} \subset \{x \in B \; | \; |x - a| \leq \delta\}$. If g is differentiable at a then so is f, and we have $f'(a) = g'(a)$.
\end{thm}

\begin{thm}[Operations on Derivatives]
	Let F be a subfield of $\bb{R}$, let $A \subseteq F$, let $f, g: A \to F$, let $a \in A$ be a limit point of A, and let $c \in F$. SPS that f and g are differentiable at a. Then
	\begin{enumerate}
		\item (Linearity) the functions cf and f $\pm$ g are differentiable at a with
			\[
				(cf)'(a) = cf'(a), \quad (f \pm g)'(a) = f'(a) \pm g'(a)
			\]
		\item (Product Rule) the function fg is differentiable at a with
			\[
				(fg)'(a) = f'(a)g(a) + f(a)g'(a)
			\]
		\item (Reciprocal Rule) if $g(a) \neq 0$ then the function 1/g is differentiable at a with
			\[
				(\frac{1}{g})'(a) = -\frac{g'(a)}{g(a)^2}
			\]
		\item (Quotient Rule) if $g(a) \neq 0$ then the function f/g is differentiable at a with
			\[
				(\frac{f}{g})'(a) = \frac{f'(a)g(a) - f(a)g'(a)}{g(a)^2}
			\]
	\end{enumerate}
\end{thm}

\begin{thm}[Chain Rule]
	Let F be a subfield of $\bb{R}$, let $A, B \subseteq F$, let $f: A \to F$, let $g:B \to F$ and let $h = g \circ f: C \to F$ where $C = A \cap f^{-1}(B)$. Let $a \in C$ be a limit point of C (hence also for A) and let $b = f(a) \in B$. SPS that f is differentiable at a and g is differentiable at b. Then h is differentiable at a with
	\[
		h'(a) = g'\big( f(a) \big)f'(a)
	\]
\end{thm}


\section{Inverse Function Theorems}

\begin{thm}[Monotonic Functions]
	Let F be a subfield of $\bb{R}$, let $A \subseteq F$ and let $f: A \to F$. Then f is monotonic iff f has the property that for all $a, b, c \in A$, if b lies between a and c, then f(b) lies between f(a) and f(c).
\end{thm}

\begin{thm}[Continuity and Strictly Monotonous Functions]
	Let I be an interval in $\bb{R}$, let $f: I \to \bb{R}$ and let J = f(I).
	\begin{enumerate}
		\item If f is continuous then its range J = f(I) is an interval in $\bb{R}$.
		\item If f is injective and continuous, then f is strictly monotonic.
		\item If $f: I \to J$ is strictly monotonic, then so is its inverse $g: J \to I$.
	\end{enumerate}
\end{thm}

\begin{thm}[Inverse Function Theorem]
	Let I be an interval in $\bb{R}$, let $f: I \to \bb{R}$ and let $J = f(I)$.
	\begin{enumerate}
		\item If f is bijective and continuous, then its inverse g is continuous.
		\item If f is bijective and continuous, and f is differentiable at a with $f'(a) \neq 0$, then its inverse g is differentiable a b = f(a) with $g'(b) = \frac{1}{f'(a)}$.
	\end{enumerate}
\end{thm}

\begin{thm}[Derivatives of the Basic Elementary Functions]
	The basic elementary functions have the following derivatives.
	\begin{enumerate}
		\item $(x^a)' = a x^{a-1}$ where $a, x \in \bb{R}$ and $x^{a - 1}$ is defined.
		\item $(a^x)' = (\ln a) \cdot a^x$ where $a > 0$ and $x \in \bb{R}$ and \\
		 	  $(\log_a x)' = \frac{1}{\ln a} \cdot \frac{1}{x}$ where $0 < a \neq 1$ and $x > 0$, and in particular \\
		 	  $(e^x)' = e^x$ for all $x \in \bb{R}$ and $(\ln x)' = \frac{1}{x}$ for all $x > 0$.
		\item $(\sin x)' = \cos x$ and $(\cos x)' = -\sin x$ for all $x \in \bb{R}$, and \\
			  $(\tan x)' = \sec^2 x$ and $(\sec x)' = \sec x \tan x$ for all $x \in \bb{R}$ with $x \neq \frac{\pi}{2} + k\pi, k \in \bb{Z}$,\\
			  $(\cot x)' = -\csc^2 x$ and $(\csc x)' = -\cot x \csc x$ for all $x \in \bb{R}$ with $x \neq \pi + k \pi, k \in \bb{Z}$.
		\item $(\arcsin x)' = \frac{1}{\sqrt{1 - x^2}}$ and $(\arccos x)' = \frac{-1}{\sqrt{1 - x^2}}$ for $|x| < 1$, \\
			  $(\sec^{-1} x)'' = \frac{1}{x \sqrt{x^2 - 1}}$ and $(\csc^{-1} x)' = \frac{-1}{x \sqrt{x^2 - 1}}$ for $|x| > 1$, and \\
			  $(\arctan x)' = \frac{1}{1 + x^2}$ and $(\cot^{-1} x)' = -\frac{1}{1 + x^2}$ for all $x \in \bb{R}$.
	\end{enumerate}
\end{thm}


\section{Mean Value Theorem}

\begin{defn}[Local Maximum and Minimum]
	Let F be a subfield of $\bb{R}$, let $A \subseteq F$, let $f: A \to F$ and let $a \in A$. We say that f has a local maximum value at a when
	\[
		\exists \delta > 0 \; \forall x \in A \; \big(|x-a| \leq \delta \implies f(x) \leq f(a) \big)
	\]
	Similarly, we say that f has a local minimum value at a when
	\[
		\exists \delta > 0 \; \forall x \in A \; \big( |x-a| \leq \delta \implies f(x) \geq f(a) \big)
	\]
\end{defn}

\begin{thm}[Fermat's Theorem]
	Let F be a subfield of $\bb{R}$, let $A \subseteq F$, let $f: A \to F$. SPS that $a \in A$ is a limit point of A, both from above and from below. SPS that f is differentiable at a and that f has a local maximum or minimum value at a. Then $f'(a) = 0$.
\end{thm}

\begin{thm}[Rolle's Theorem]
	SPS that $a, b \in \bb{R}$ with a < b. If $f: [a, b] \to \bb{R}$ is continuous on [a, b] and differentiable on (a, b), with f(a) = 0 = f(b), then there exists a point $c \in (a, b)$ such that $f'(c) = 0$.
\end{thm}

\begin{remark}
	The continuity hypothesis and differentiability hypothesis in Rolle's Theorem cannot be relaxed at even one point in [a, b].
\end{remark}

\begin{thm}[Mean Value Theorem]
	Let $a, b \in \bb{R}$ with a < b. If $f: [a, b] \to \bb{R}$ is differentiable on (a, b) and continuous on [a, b], then there exists a point $c \in (a, b)$ with
	\[
		f'(c) = \frac{f(b) - f(a)}{b - a}
	\]
\end{thm}

\begin{thm}[Cauchy/Generalized Mean Value Theorem]
	Let $a, b \in \bb{R}$ with a < b. If $f, g: [a, b] \to \bb{R}$ are differentiable on (a, b) and continuous on [a, b], then there exists a point $c \in (a, b)$ such that
	\[
		g'(c)\big( f(b) - f(a) \big) = f'(c)\big( g(b) - g(a) \big)
	\]
\end{thm}

\begin{crly}[Trichotomy of The Derivative]
	Let $a, b \in \bb{R}$ with a < b. Let $f: [a, b] \to \bb{R}$. SPS that f is differentiable in (a, b) and continuous at a and b.
	\begin{enumerate}
		\item $\forall x \in (a, b) \; f'(x) \geq 0 \implies $ f is increasing on [a, b].
		\item $\forall x \in (a, b) \; f'(x) > 0 \implies $ f is strictly increasing on [a, b].
		\item $\forall x \in (a, b) \; f'(x) \leq 0 \implies $ f is decreasing on [a, b].
		\item $\forall x \in (a, b) \; f'(x) < 0 \implies $ f is strictly decreasing on [a, b].
		\item $\forall x \in (a, b) \; f'(x) = 0 \implies $ f is constant on [a,b].
		\item If $g: [a ,b] \to \bb{R}$ is continuous at a and b and differentiable in 9a, b) with $g'(x) = f'(x)$ for all $x \in (a, b)$, then for some $c \in \bb{R}$ we have $g(x) = f(x) + c$ for all $x \in (a, b)$.
	\end{enumerate}
\end{crly}

\begin{crly}[The Second Derivative Test]
	Let I be an interval in $\bb{R}$, let $f: I \to \bb{R}$ and let $a \in I$. SPS that f is differentiable in I with $f'(a) = 0$.
	\begin{enumerate}
		\item If $f''(a) > 0$ then f has a local minimum at a.
		\item If $f''(a) < 0$ then f has a local maximum at a.
	\end{enumerate}
\end{crly}


\section{Taylor's Theorem and l'Hôpital's Rule}

\begin{thm}[l'Hôpital's Rule]
	Let I be a non degenerate interval in $\bb{R}$. Let $a \in I$, or let a be an endpoint of I. Let $f, g: I \setminus \{a\} \to \bb{R}$. SPS that f and g are differentiable in $I \setminus \{a\}$ with $g'(x) \neq 0$ for all $x \in I \setminus \{a\}$. SPS further that $\lim\limits_{x \to a} f(x) = \lim\limits_{x \to a} g(x) = 0 \text{ or } \pm \infty$, and suppose that $\lim\limits_{x \to a} \frac{f'(x)}{g'(x)} \in \hat{\bb{R}}$. Then
	\[
		\lim_{x \to a} \frac{f(x)}{g(x)} = \lim_{x \to a} \frac{f'(x)}{g'(x)}
	\]
\end{thm}

\begin{thm}[Taylor's Formula]
	Let $n \in \bb{N}$ and let $a ,b \in \hat{\bb{R}}$ with a < b. If $f: (a, b) \to \bb{R}$ and $f^{(n + 1)}$ exists on (a, b), then for each pair of points $x, x_0 \in (a, b)$ there is a number c between x and $x_0$ such that
	\[
		f(x) = f(x_0)) + \sum_{k=1}^{n}\frac{f^{(k)} (x_0)}{k!}(x - x_0)^k + \frac{f^{(n + 1)}(c)}{(n + 1)!}(x - x_0)^{n + 1}
	\]
\end{thm}

% =================================
% 		Chapter 5
% =================================
\chapter{Integrability on \texorpdfstring{$\bb{R}$}{R}}

\section{The Riemann Integral}

\begin{defn}[Partition \& Subintervals]
	A partition of the closed interval [a, b] is a set $X = \{x_0, x_1, ..., x_n\}$ with
	\[
		a = x_0 < x_1 < x_2 < \hdots < x_n = b.
	\]
	The intervals $[x_{i-1}, x_i]$ are called the subintervals of [a, b], and we write
	\[
		\Delta_i x = x_i - x_{i-1}
	\]
	for the size of the $i^{th}$ subinterval. Note that
	\[
		\sum_{i=1}^{n} \Delta_i x = b - a
	\]
	The size (or norm) of the partition X, denoted by $\Delta X$ is
	\[
		\Delta X = \max \{\Delta_i x | 1 \leq i \leq n\}
	\]
	i.e. $\Delta X$ is the largest subinterval in X.
\end{defn}

\begin{defn}[The Riemann Sum]
	Let X be a partition of [a, b], and let $f: [a, b] \to \bb{R}$ be bounded. A Riemann sum for f on X is a sum of the form
	\[
		S = \sum_{i=1}^{n} f(t_i) \Delta_i x \quad \text{for some } t_i \in [x_{i - 1}, x_i].
	\]
	The points $t_i$ are called sample points.
\end{defn}

\begin{defn}[(Riemann) Integrable]
	Let $f: [a, b] \to \bb{R}$ be bounded. We say that f is (Riemann) integrable on [a, b] when
	\begin{gather*}
		\exists I \in \bb{R} \; \forall \epsilon > 0 \; \exists \delta > 0 \; \forall X = \{x_0, x_1, ..., x_n\} \\
		\Delta X < \delta \implies \bigg(\forall t_i \in [x_{i -1}, x_i] \quad \Bigg|\sum_{i=1}^{n} f(t_i) \Delta_i x - I \Bigg| < \epsilon \bigg)
	\end{gather*}
	where $X = \{x_0, x_1, ..., x_n\}$ is a partition on [a, b].

	The number I is unique, and it is called the (Riemann) integral of f on [a, b], and we write
	\[
		I = \int_{a}^{b} f, \text{ or } I = \int_{a}^{b} f(x) dx
	\]
\end{defn}


\section{Upper and Lower Riemann Sums}

\begin{defn}[Upper and Lower Riemann Sums]
	Let X be a partition for [a, b] and let $f: [a, b] \to \bb{R}$ be bounded. The upper Riemann sum for f on X, denoted by U(f, X), is
	\[
		U(f, X) = \sum_{i=1}^{n} M_i \Delta_i X \quad \text{where } M_i = \sup \{f(t) \; | \; t \in [x_{i-1}, x_i] \}
	\]
	and the lower Riemann sum for f on X, denoted L(f, X), is
	\[
		L(f, X) = \sum_{i = 1}^{n} m_i \Delta_i X \quad \text{where } m_i = \inf \{f(t) \; | \; t \in [x_{i-1}, x_i]\}
	\]
\end{defn}

\begin{remark}
	\begin{itemize}
		\item U(f, X) and L(f, X) are not necessarily Riemann sums themselves, since we do not always have $M_i = f(t_i)$ or $m_i = f(s_i)$, for any $t_i, s_i \in [x_{i-1}, x_i]$.
		\item If f is increasing, then $M_i = f(x_i)$ and $m_i = f(x_{i-1})$, and in this case U(f, X) and L(f, X) are Riemann sums.
		\item Similarly, if f is decreasing, then $M_i = f(x_{i-1})$ and $m_i = f(x_i)$, and U(f, X) and L(f, X) are Riemann sums.
		\item If f is continuous, then by the \hyperref[EVT]{Extreme Value Theorem}, we have $M_i = f(t_i)$ and $m_i = f(s_i)$ for some $t_i, s_i \in [x_{i-1}, x_i]$, and so in this case U(f, X) and L(f, X) are again Riemann sums.
	\end{itemize}
\end{remark}

\begin{thm}[Upper and Lower Riemann Sums and the Riemann Sum]
	Let X be a partition of [a, b] and let $f: [a, b] \to \bb{R}$ be bounded. Then
	\begin{align*}
		U(f, X) &= \sup \left\{\sum_{i=1}^{n} f(t_i) \Delta_i x \; \Bigg| \; t_i \in [x_{i-1}, x_i] \right\} \\
		L(f, X) &= \inf \left\{\sum_{i=1}^{n} f(t_i) \Delta_i x \; \Bigg| \; t_i \in [x_{i-1}, x_i] \right\}
	\end{align*}
	In particular, for every Riemann sum S for f on X we have
	\[
		L(f, X) \leq \sum_{i=1}^{n} f(t_i) \Delta_i x \leq U(f, X)
	\]
	for every $t_i \in [x_{i-1}, x_i]$.
\end{thm}

\begin{thm}[Partition Refinement]
	Let $f: [a, b] \to \bb{R}$ be bounded with upper and lower bounds M and m. Let $X = \{x_0, x_1, ..., x_n\}$ and $Y = \{y_0, y_1, ..., y_m\}$ be partitions of [a, b] such that $\{x_0, x_1, ..., x_n\} \subseteq \{y_0, y_1, ..., y_m\}$ (or $m \geq n$). Then
	\begin{gather*}
		0 \leq L(f, Y) - L(f, X) \leq (M - m) \Delta X \\
		0 \leq U(f, X) - U(f, Y) \leq (M - m) \Delta X
	\end{gather*}
\end{thm}

\begin{remark}
	Let X and Y be partitions of [a, b] with $X \subset Y$. Then
	\[
		L(f, X) \leq L(f, Y) \leq U(f, Y) \leq U(f, X)
	\]
\end{remark}

\begin{defn}[Upper and Lower Integral]
	Let $f: [a, b] \to \bb{R}$ be bounded. The upper integral of f on [a, b], denoted by U(f), is given by
	\[
		U(f) = \inf \{U(f, X) | \text{ X is a partition of [a, b]} \}
	\]
	and the lower integral of f on [a, b], denoted by L(f), is given by
	\[
		L(f) = \sup \{L(f, X) | \text{ X is a partition of [a, b]} \}.
	\]
\end{defn}

\begin{thm}[Upper Integral $\geq$ Lower Integral]
	Let $f: [a, b] \to \bb{R}$. We always have $L(f) \leq U(f)$.
\end{thm}

\begin{thm}[Equivalent Definitions of Integrability]
	Let $f: [a, b] \to \bb{R}$ be bounded. Then TFAE:
	\begin{enumerate}
		\item f is integrable on [a, b].
		\item $\forall \epsilon > 0 \; \exists \text{ partition } X \; U(f, X) - L(f, X) < \epsilon$.
		\item L(f) = U(f).
	\end{enumerate}
\end{thm}


\section{Evaluating Integrals of Continuous Functions}

\begin{thm}[Continuous Functions are Integrable]
	Let $f: [a, b] \to \bb{R}$ be continuous. Then f is integrable on [a, b].
\end{thm}

\begin{lemma}[Summation Formulas]
	We have
	\[
		\sum_{i=1}^{n} 1 = n, \quad \sum_{i = 1}^{n} i = \frac{n(n+1)}{2}, \quad \sum_{i=1}^{n} i^2 = \frac{n(n+1)(2n+1)}{6}, \quad \sum_{i=1}^{n} i^3 = \frac{n^2 (n+1)^2}{4}
	\]
\end{lemma}


\section{Basic Properties of Integrals}

\begin{thm}[Linearity]
	Let f and g be integrable on [a, b] and let $c \in \bb{R}$. Then f + g and cf are both integrable on [a, b] and
	\[
		\int_{a}^{b} (f+g) = \int_{a}^{b} f + \int_{a}^{b} g
	\]
	and
	\[
		\int_{a}^{b} cf = c \int_{a}^{b} f
	\]
\end{thm}

\begin{thm}[Comparison Theorem for Integrals]
	Let f and g be integrable on [a, b]. If $f(x) \leq g(x)$ for all $x \in [a, b]$ then
	\[
		\int_{a}^{b} f \leq \int_{a}^{b} g.
	\]
\end{thm}

\begin{thm}[Additivity]
	Let $a < b < c$ and let $f: [a, c] \to \bb{R}$ be bounded. Then f is integrable on [a, c] iff f is integrable both on [a, b] and on [b, c], and in this case
	\[
		\int_{a}^{b} f + \int_{b}^{c} f = \int_{a}^{c} f.
	\]
\end{thm}

\begin{defn}[Integral at a Point \& Integral from the Right]
	We define $\int_{a}^{a} f = 0$ and for a < b we define $\int_{b}^{a} f = -\int_{a}^{b} f$.
\end{defn}

\begin{note}
	Using the above definition, the Additivity Theorem extends to the case that a, b, c $\in \bb{R}$ are not in increasing order: for any a, b, c $\in \bb{R}$, if f is integrable on $[\max \{a, b, c\}, \min \{a, b, c\}]$ then
	\[
		\int_{a}^{b} f + \int_{b}^{c} f = \int_{a}^{c} f.
	\]
\end{note}

\begin{thm}[Estimation]
	Let f be integrable on [a, b]. Then $|f|$ is integrable on [a, b] and
	\[
		\left|\int_{a}^{b} f \right| \leq \int_{a}^{b} |f|
	\]
\end{thm}


\section{Fundamental Theorem of Calculus}

\begin{note}
	For a function F, defined on an interval containing [a, b], we write
	\[
		[F(x)]_a^b = F(b) - F(a).
	\]
\end{note}

\begin{thm}[The Fundamental Theorem of Calculus]
	\begin{enumerate}
		\item Let f be integrable on [a, b]. Define $F: [a, b] \to \bb{R}$ by
			\[
				F(x) = \int_{a}^{x} f = \int_{a}^{x} f(t) dt
			\]
			Then F is continuous on [a, b]. Moreover, if f is continuous at a point $x \in [a, b]$, then F is differentiable at x and
			\[
				F'(x) = f(x).
			\]
		\item Let f be integrable on [a, b]. Let F be differentiable on [a, b] with $F' = f$. Then
			\[
				\int_{a}^{b} f = \lbrack F(x) \rbrack_a^b = F(b) - F(a)
			\]
	\end{enumerate}
\end{thm}

\begin{defn}[Antiderivative]
	A function F such that $F' = f$ on an interval is called an antiderivative of f on the interval.
\end{defn}


% =================================
% 		Chapter 6
% =================================
\chapter{Sequences and Series}

% =================================
% 		Chapter 7
% =================================
\chapter{Sequences and Series of Functions}

% =================================
% 		Chapter 8
% =================================
\chapter{Topology in Euclidean Space}


% =================================
% 		Appendix
% =================================
\appendix

\chapter{ZF Set Theory and the Axiom of Choice}\label{apdxA}
\section{Introduction}
\begin{eg}[Russel's Paradox]\label{russel_paradox}
	Let X be the set of all sets, and let $S = \{ A \in X | A \notin A\}$.\\
	Note for example that $Z \notin Z \implies Z \in S$, and $X \in X \implies X \notin S$.\\
	Thus we have $S \in S \iff S \notin S$.
\end{eg}

To ensure that mathematical paradoxes (like the above) can no longer arise, mathematicians considered the following questions, and with these questions, rough answers are provided:
\begin{enumerate}
	\item What exactly is an allowable mathematical object? \\
	A: Every mathematical object is a mathematical set, and a mathematical set can be constructed using certain rules, for e.g. the now widely accepted Zermelo-Fraenkel Set Theory and the Axiom of Choice. While the Axiom of Choice is still highly criticized even today (e.g. the highly controversial \href{https://en.wikipedia.org/wiki/Banach–Tarski_paradox}{Banach-Tarski Paradox}), the Zermelo-Fraenkel Set Theory is widely welcomed, but not without critics. We shall call the Zermelo-Fraenkel Set Theory and the Axiom of Choice as the ZFC Axioms of Set Theory.
	\item What exactly is an allowable mathematical statement? \\
	A: Every mathematical statement can be expressed in a formal symbolic language, which uses symbols rather than words from any spoken language.
	\item What exactly is allowable in a mathematical proof? \\
	A: Every mathematical proof is a finite list of ordered pairs $(\mathscr{S}_n, \mathscr{F}_n)$ (which we can think of as proven theorems), where each $\mathscr{S}_n$ is a finite set of formulas (called the $\textit{premises}$) and each $\mathscr{F}_n$ is a single formula (called the $\textit{conclusion}$), which that each pair $(\mathscr{S}_n, \mathscr{F}_n)$ can be obtained from previous pairs $(\mathscr{S}_i, \mathscr{F}_i)$ with $i < n$, using certain proof rules.
\end{enumerate}

In the remainder of this appendix, we shall look more into the first 2 questions.


\section{ZFC Axioms of Set Theory}

\begin{defn}[Mathematical Symbols]
	We allow ourselves to use only the following symbols from the following symbol set: \\
	\begin{center}
		\begin{tabular}{c l}
			$\neg$		&	not \\
			$\land$		&	and \\
			$\lor$		&	or \\
			$\implies$	&	implies \\
			$\iff$	&	if and only if \\
			$=$			&	equals \\
			$\in$		&	is an element of \\
			$\forall$	&	for all \\
			$\exists$	&	there exists \\
			$()\quad\{\}\quad[]$	&	parenthesis
		\end{tabular}
	\end{center}
	along with some variable symbols such as $x, y, z, u, v, w,... \text{ or } x_1, x_2, x_3,...$
\end{defn}

\begin{defn}[Formula]
	A formula (in the formal symbolic language of first order set theory) is a non-empty finite string of symbols, from the above list, which can be obtained using finitely many applications following the three rules below:
	\begin{enumerate}
		\item If x and y are variable symbols, then each of the following strings are formulas.
		\[
			x = y, \quad x \in y
		\]
		\item If F and G are formulas then each of the following strings are formulas.
		\[
			\neg F, \quad (F \land G), \quad (F \lor G), \quad (F \implies G), \quad (F \iff G)
		\]
		\item If x is a variable symbol and F is a formula then each of the following is a formula.
		\[
			\forall x \in F, \quad \exists x \in F
		\]
	\end{enumerate}
\end{defn}

\begin{defn}[Free or Bounded Variable]
	Let x be a variable symbol and let F be a formula. For each occurrence of the symbol x, which does not immediately follow a quantifier, in the formula F, we define whether the occurrence of x is free or bound inductively as follows:
	\begin{enumerate}
		\item If F is a formula of one of the forms y = z or $y \in z$, where y and z are variable symbols (possibly equal to x), then every occurence of x in F is free, and no occurrence is bound.

		\item If F is a formula of one of the forms $\neg H, (H \land G), (H \lor G), (H \implies G), (H \iff G)$, where G and H are formulas, then each occurence of the symbol x is either an occurrence in the formula G or an occurrence in the formula H, and each free (respectively, bound) occurence of x in G remains free (respectively, bound) in F, and similarly for each free (or bound) occurrence of x in G. In other words, wlog, if x is bounded in G, then it is bounded in F, and vice versa.
		\item If F is a formula of one of the forms $\forall y \in G \text{ or } \exists y \in G$, where G is a formula and y is a variable symbol. If y is different from x, then each free (or bound) occurrence of x in G remains free (or bound) in the formula G, and if y = x then every free occurrence of x in G becomes bound in F, and every bound occurrence of x in G remains bound in F.
	\end{enumerate}
\end{defn}

\begin{defn}[Is Bound By and Binds]
	When a quantifier symbol occurs in a given formula F, and is followed by the variable symbol x and then by the formula G, any free occurence of x in G will become bound in the given formula F (by the 3rd definition above). We shall say that the occurrence of x is bound by (that occurrence of) the quantifier symbol, or that (the occurrence of) the quantifier symbol binds the occurence of x.
\end{defn}

\begin{defn}[Free Variable, Statement, Statement About]
	A $\textbf{free variable}$ in a formula F is any variable symbol that has at least one free occurence in F. A formula F with no free variables is called a $\textbf{statement}$. When the free variables in F all lie in the set $\{x_1, x_2, ..., x_n\}$, we shall write F as $F(x_1, x_2, ..., x_n)$ and we shall say that F is a $\textbf{statement about}$ the variables $x_1, x_2, ..., x_n$.
\end{defn}

\begin{defn}[Unique Existence]
	When F(x) is a statement about x, we sometimes write F(y) as a short form for the formula $\forall x(x=y\implies F(x))$, and we sometimes write
	\[
		\exists!y \quad F(y)
	\]
	which we read as "there exists a unique y such that F(y)", as a short form for the formula
	\[
		(\exists y \quad F(y) \land \forall z \quad F(z)) \implies z = y))
	\] which is, in turn, for the formula
	\[
		\exists y \bigg( \forall x \Big(x = y \implies F(x) \Big) \land \forall z \Big(\forall x (x = z \implies F(x)) \implies z = y \Big) \bigg)
	\]
\end{defn}

\begin{remark}[The ZFC Axioms of Set Theory (informal)]
	Every mathematical set can be constructed using specific rules, which we shall use the ZFC Axioms of Set Theory. Below is a list of the ZFC Axioms, stated informally.
	\begin{itemize}
		\item Empty Set Axiom: There exists an empty set $\emptyset$ with no elements.
		\item Extension Axiom: 2 sets are equal if and only if they have the same elements.
		\item Separation Axiom: If u is a set and F(x) is a statement about x, $\{x \in u : F(x)\}$ is a set.
		\item Pair Axiom: If u and v are sets then {u, v} is a set.
		\item Union Axiom: If u is a set then $\cup u = \bigcup\limits_{v \in u} v$ is a set.
		\item Power Set Axiom: If u is a set then $\mathcal{P}(u) = \{v : v \in u\}$ is a set.
		\item Axiom of Infinity: If we define the natural numbers to be the sets $0 = \emptyset, 1 = \{0\}, 2 = \{0, 1\}, 3 = \{0, 1, 2\}$ and so on, then $\bb{N} = \{0, 1, 2, 3, ...\}$ is a set.
		\item Replacement Axiom: If u is a ste and F(x, y) is a statement about x and y with the property that $\forall x \> \exists! y \> F(x,y)$ then $\{y : \exists x \in u \> F(x,y)\}$ is a set.
		\item Axiom of Choice: Given a set u of non-empty pairwise disjoint sets, there exists a set which contains exactly one element from each of the sets in u.
	\end{itemize}
\end{remark}

\begin{defn}[Empty Set Axiom]
	The Empty Set Axiom is the formula
	\[
		\exists u \> \forall x \quad \neg x \in u
	\]
\end{defn}

\begin{defn}[Extension Axiom]
	The Extension Axiom is the formula
	\[
		\forall u \> \forall v \> \Big(u = v \iff \forall x \> (x \in u \iff x \in v) \Big)
	\]
\end{defn}

\begin{thm}[Uniqueness of the Empty Set]
	The empty set is unique.
\end{thm}

\begin{defn}[$\emptyset$]
	We denote the unique empty set by $\emptyset$.
\end{defn}

\begin{defn}[Subset]
	Given sets u and v, we say that u is a $\textbf{subset}$ of v, and write $u \subseteq v$, when $\forall x (x \in u \implies x \in v)$
\end{defn}

\begin{defn}[Separation Axiom]
	For any statement F(x) about x, the following formula is an axiom.
	\[
		\forall u \> \exists v \> \forall x \Big(x \in v \iff (x \in u \land F(x)) \Big)
	\]
	More generally, for any statement $F(x, u_1, u_2, ..., u_n)$ about $x, u_1, u_2, ..., u_n$ where $n \geq 0$, the following formula is an axiom.
	\[
		\forall u \> \forall u_1 \hdots \forall u_n \> \exists v \> \forall x \Big( x \in v \iff (x \in i \land F(x, u_1, ..., u_n)) \Big)
	\]
	Any axiom of this form is called the Separation Axiom.
\end{defn}

\begin{note}
	It is important to realize that a Separation Axiom only allows us to construct a subset of a given set u. So, e.g., we cannot use the Separation Axiom to show that the collection $S = \{x : \neg x \in x\}$, which is used to formulate \hyperref[russel_paradox]{Russel's Paradox}, is a set.
\end{note}

\begin{defn}[Pair Axiom]
	The Pair Axiom is the formula
	\[
		\forall u \> \forall v \> \exists w \> \forall x \Big( x \in w \iff (x = u \lor x = v) \Big)
	\]
\end{defn}

\begin{defn}[Union Axiom]\label{union_axiom}
	The Union Axiom is the formula
	\[
		\forall u \> \exists w \> \forall x \Big( x \in w \iff \exists v (v \in u \land x \in v) \Big)
	\]
\end{defn}

\begin{defn}[Union]
	Given a set u, by the Union Axiom there exists a set w with the property that $\forall x \Big( x \in w \iff \exists v (v \in u \land x \in v) \Big)$, and by the Extension Axiom, this set w is unique. We call the set w the $\textbf{union}$ of the elements in u, and denote it by
	\[
		\cup u = \bigcup_{v \in u} v.
	\]Given two sets u and v, we define the union of u and v to be the set
	\[
		u \cup v := \bigcup \{u, v\}.
	\]
	Given three sets u, v, and w, note that \{z\} = \{z, z\} is a set and so $\{x, y, z\} = \{x, y\} \cup \{z\}$ is also a set. More generally, if $u_1, u_2, ..., u_n$ are sets then $\{u_1, u_2, ..., u_n\}$ is a set and we define the union of the sets $u_1, u_2, ..., u_n$ to be
	\[
		u_1 \cup u_2 \cup \hdots \cup u_n = \bigcup_{k = 1}^{n} u_k = \bigcup \{u_1, u_2, ..., u_n\}
	\]
\end{defn}

\begin{defn}[Intersection]
	Given a st u, we define the intersection of the elements in u to be the set
	\[
		\bigcap u = \biggl\{ x \in \bigcup u \> \Big| \> \forall v (v \in u \implies x \in v) \biggr\}
	\] Given two sets u and v, we define the intersection of u and v to be the set
	\[
		u \cap v = \bigcap \{u, v\}
	\] and more generally, given sets $u_1, u_2, ..., u_n$, we define the intersection of $u_1, u_2, ..., u_n$ to be the set
	\[
	 	u_1 \cap u_2 \cap \hdots \cap u_n = \bigcap_{k=1}^{n} u_k = \bigcap \{u_1, u_2, ..., u_n\}
	\] 
\end{defn}

\begin{defn}[Power Set Axiom]
	The Power Set Axiom is the formula
	\[
		\forall u \> \exists w \> \forall v (v \in w \iff v \subseteq u)
	\]
\end{defn}

\begin{defn}[Power Set]
	Given a set u, the set w is with the property that $\forall v (v \in w \iff v \subseteq u)$ (which exists by the Power Set Axiom and is unique by the Extension Axiom) is called the power set of u and is denoted by $\mathcal{P}(u)$, so we have
	\[
		\mathcal{P}(u) = \{v | v \subseteq u\}
	\]
\end{defn}

\begin{defn}[Ordered Pair]
	Given two sets x and y, we define the ordered pair (x, y) to be the set
	\[
		(x, y) = \{\{x\}, \{x, y\}\}.
	\]
	Given two sets u and v, note that if $x \in u$ and $y \in v$ then we have $\{x\} \in \mathcal{P}(u \cup v)$ and $\{x, y\} \in \mathcal{P}(u \cup v)$ and so $(x, y) = \{\{x\}, \{x, y\}\} \in \mathcal{P}(\mathcal{P}(u \cup v))$. We define the product $u \times v$ to be the set
	\[
		u \times v = \{(x, y) | x \in u \land y \in v\},
	\] i.e.
	\[
		u \times v = \Bigr\{ z \in \mathcal{P}(\mathcal{P}(u \cup v)) | \exists x \exists y \big( (x \in u \land y \in v) \land z = (x, y) \big) \Bigr\}
	\]
\end{defn}

\begin{defn}[Successor, Inductive]
	We define
	\[
		0 = \emptyset, \quad	1 = \{0\}, \quad	2 = \{0, 1\} = 1 \cup \{1\}, \quad	3 = \{0, 1, 2\} = 2 \cup \{2\},
	\] and so on. For a set x, we define the successor of x to be the set
	\[
		x + 1 = x \cup \{x\}.
	\] A set u is called inductive when it has the property that
	\[
		(0 \in u \land \forall x (x \in u \implies x + 1 \in u))
	\]
\end{defn}

\begin{defn}[Axiom of Infinity]
	The Axiom of Infinity is the formula
	\[
		\exists u (0 \in u \land \forall x (x \in u \implies x + 1 \in u))
	\] so the Axiom of Infinity states that there exists an inductive set.
\end{defn}

\begin{thm}[Existence \& Uniqueness of an Inductive Set]
	$\exists w := \{x | x \in v \text{ for every inductive set v }\}$ \\
	Moreover, this set w is an inductive set.
\end{thm}

\begin{defn}[Natural Numbers]
	The unique set w in the above theorem is called the set of natural numbers, and we denote it by $\bb{N}$. We write
	\begin{align*}
		\bb{N} &= \{x | x \in v \text{ for every inductive set v }\} \\
			   &= \{0, 1, 2, 3, ...\}
	\end{align*}
	For $x, y \in \bb{N}$, we write x < y when $x \in y$ and write $x \leq y$ when $x < y \lor x = y$.
\end{defn}

\begin{remark}
	For a formula F, we write $\forall x \in u \> F$ as a shorthand notation for the formula $\forall x (x \in u \implies F)$. Similarly, we write $\exists x \in u \> F$ as a shorthand notation for $\exists x (x \in u \land F)$
\end{remark}

\begin{thm}[Principle of Induction]
	Let F(x) be a statement about x. SPS that
	\begin{enumerate}
		\item F(0), and
		\item $\forall x \in \bb{N} (F(x) \implies F(x+1))$.
	\end{enumerate}
	Then $\forall x \in \bb{N} \> F(x)$
\end{thm}

\begin{remark}
	The expression F(0) is short for $\forall x (x = 0 \implies F(x))$, which in turn is short for $\forall x (\forall y \> \neg y \in x \implies F(x))$.Similarly, F(x + 1) is short for the formula $\forall y (y = x + 1 \implies F(y))$, where F(y) is short for $\forall x (x = y \implies F(x))$.
\end{remark}

\begin{defn}[Replacement Axiom]
	Given a statement F(x, y) about x and y, the following formula is an axiom:
	\[
		\forall u \Big( \forall x \exists! y \> F(x,y) \implies \exists w \forall y \big( y \in w \iff \exists x \in u \> F(x, y) \big) \Big)
	\]
	where $\exists! y \> F(x, y)$ is short for $\exists y \Big(F(x, y) \land \forall z \big(F(x, z) \implies z = y)\Big)$ with F(x ,z) short for the formula $\forall y (y = z \implies F(x, y))$. More generally, given a statement $F(x, y, u_1, ..., u_n)$ about $x, y, u_1, ..., u_n$ with $n \geq 0$, the following formula is an axiom:
	\[
		\forall u \forall u_1 \hdots \forall u_n \Big( \forall x \exists!y \> F(x, y, u_1, ..., u_n) \implies \exists w \forall y \big( y \in w \iff \exists x \in u \> F(x, y, u_1, ..., u_n) \big) \Big)
	\]
	An axiom of this form is called a Replacement Axiom.
\end{defn}

\begin{defn}[Axiom of Choice]
	The Axiom of Choice is the formula given by
	\[
		\forall u \Big( \big( \neg \phi \in u \land \forall x \in u \> \forall y \in u (\neg x = y \implies x \cap y = \emptyset) \big) \implies \exists w \forall v \in u \> \exists!x \in v \> x \in w \Big)
	\]
\end{defn}

From this point on, we will be using upper-case letters to denote sets, instead of lower-case as per the statements above.


\section{Relations, Equivalence Relations, Functions and Recursion}

\begin{defn}[Binary Relation]
	A binary relation R on a set X is a subset $R \subseteq X \times X$. More generally, a binary relation is any set R whose elements are ordered pairs. For a binary relation R, we usually write xRy instead of $(x, y) \in R$.
\end{defn}

\begin{defn}[Domain, Range, Image, Inverse Image, Inverse, Composition]
	Let R and S be binary relations. \\
	The domain of R is
	\[
		\text{Domain}(R) = \{x | \exists y \> xRy\}
	\]
	and the range of R is
	\[
		\text{Range}(R) = \{x | \exists y \> xRy\}.
	\]
	For any set A, the image of A under R is
	\[
		R(A) = \{y | \exists x \in A \> xRy\}
	\]
	and the inverse image of A under R is
	\[
		R^{-1}(A) = \{x | \exists y \in A \> xRy\}.
	\]
	The inverse of R is
	\[
		R^{-1} = \{(y, x) | (x, y) \in R\}
	\]
	and the composition S composed with R is
	\[
		S \circ R = \{(x, z) | \exists y \> xRy \land ySz \}
	\]
\end{defn}

\begin{thm}[Domain, Range, Image and Inverse Image as Sets]
	Let A be a set and let R be a binary relation. Then Domain(R), Range(R), R(A) and $R^{-1}(A)$ are sets.
\end{thm}

\begin{thm}[Inverse and Composition as Binary Relations]
	Let A be a set and let R and S be binary relations. Then $R^{-1}$ and $S \circ R$ are binary relations.
\end{thm}

\begin{defn}[Equivalence Relation]
	An equivalence relation on a set X is a binary relation R on X such that
	\begin{enumerate}
		\item R is $\textbf{reflexive}$, i.e. $\forall x \in X \> xRx$
		\item R is $\textbf{symmetric}$, i.e. $\forall x, y \in X \> (xRy \implies yRx)$, and
		\item R is $\textbf{transitive}$, i.e. $\forall x, y, z \in R \> \big( (xRy \land yRz) \implies xRz \big)$.
	\end{enumerate}
\end{defn}

\begin{defn}[Equivalence Class]
	Let R be an equivalence relation on the set X. For $a \in X$, the equivalence class of a modulo R is the set
	\[
		[a]_R = \{x \in X | xRa \}
	\]
\end{defn}

\begin{defn}[Partition]
	A partition of a set X is a set S of non-empty pairwise disjoint sets whose union is X, that is a set S such that
	\begin{enumerate}
		\item $\forall X, Y \in S \> (X \neq Y \implies X \cap Y = \emptyset)$
		\item $\bigcup S = X$.
	\end{enumerate}
\end{defn}

\begin{thm}[Correspondence of Equivalence Relations and Partitions]
	Given a set X, we have the following correspondence between equivalence relations on X and partitions of X.
	\begin{enumerate}
		\item Given an equivalence relation R on X, the set of all equivalence classes
		\[
			S_R = \{[a]_R | a \in X\}
		\] is a partition of X.
		\item Given a partition S of X, the relation $R_S$ on X is defined by
		\[
			R_S = \{(x, y) \in X \times X | \exists A \in S (x \in A \land y \in A)\}
		\] is an equivalence relation on X.
		\item Given an equivalence relation R on X we have $R_{S_R} = R$, and a given partition S of X, we have $S_{R_S} = S$.
	\end{enumerate}
\end{thm}

\begin{note}[Set of All Equivalence Classes]
	Given an equivalence relation R on X, the set of all equivalence classes, which we denote by $S_R$ in the above theorem, is usually denoted by $X/R$, so
	\[
		X/R = \{[a]_R | a \in X\}
	\]
\end{note}

\begin{defn}[Set of Representatives]
	Let R be an equivalence relation. A set of representatives for R is a subset of X which contains exactly one element from each equivalence class in X/R.
\end{defn}

\begin{remark}
	Notice that the AC is equivalent to the statement that every equivalence relation has a set of representatives.
\end{remark}

\begin{defn}[Function]\label{fn_apdxA}
	Get sets X and Y, a function from X to Y is a binary relation $f \subseteq X \times Y$ with the property that
	\[
		\forall x \in X \> \exists! y \in Y \> (x,y) \in f
	\]
	More generally, a function is a binary relation with the property that
	\[
		\forall x \in Domain(f) \> \exists! y \> (x, y) \in f.
	\]
	For a function f, we usually write y = f(x) instead of xfy. It is customary to use the notation $f: X \to Y$ when X = Domain(f) and Y is any set with $Range(f) \subseteq Y$.
\end{defn}

\begin{defn}[One-to-one \& Onto]
	Let $f: X \to Y$. THe function f is called one-to-one (or injective) when
	\[
		\forall y \in Y \> \exists \text{ at most one } x \in X \> y = f(x)
	\]
	and f is called onto (or surjective) when
	\[
		\forall y \in Y \> \exists \text{ at least one } x \in X \> y = f(x)
	\]
\end{defn}

\begin{defn}[Left and Right Inverses]
	Let $f: X \to Y$. Let $I_X$ and $I_Y$ denote the identity function on X and Y respectively. A left inverse of f is a function $g: Y \to X$ such that $g \circ f = I_X$. A right inverse of f is a function $H: X \to Y$ such that $f \circ H = I_Y$. Note that if f has a left inverse g and a right inverse H, then we have $g = g \circ I_Y = g \circ f \circ H = I_X \circ H = H$. In this case, we say that g is the (unique two-sided) inverse of f.
\end{defn}

\begin{thm}[Surjective and Injective VS Inverses]
	Let $f: X \to Y$. Then
	\begin{enumerate}
		\item f is one-to-one if and only if f has a left inverse.
		\item f is onto if and only if f has a right inverse.
		\item f is one-to-one and onto if and only if f has a (two-sided) inverse.
	\end{enumerate}
\end{thm}

\begin{defn}[Invertible]
	A function $f: X \to Y$ is called invertible (or bijective) when it is one-to-one and onto, or equivalently, when it has a (unique two-sided) inverse.
\end{defn}

\begin{thm}[The Recursion Theorem]\label{recursion_theorem}
	\begin{enumerate}
		\item Let A be a set, let $a \in A$, and let $g : A \times \bb{N} \to A$. Then there exists a unique function $f: \bb{N} \to A$ such that
		\[
			f(0) = a \text{ and } f(n+1) = g(f(n), n) \text{ for all } n \in \bb{N}
		\]
		\item Let A and B be sets, let $g: A \to B$, and let $h: A \times B \times \bb{N} \to B$. Then there exists a unique function $f:A \times \bb{N} \to B$ such that for all $a \in A$ we have
		\[
			f(a, 0) = g(a) \text{ and } f(a, n + 1) = h(a, f(a, n), n) \text{ for all } n \in \bb{N}
		\]
	\end{enumerate}
\end{thm}


\section{Construction of Integers, Rational, Real and Complex Numbers}

\begin{defn}[Sum and Product]
	By Part(2) of the \hyperref[recursion_theorem]{Recursion Theorem}, there is a unique function $s: \bb{N} \times \bb{N} \to \bb{N}$ such that for all $a, b \in \bb{N}$ we have
	\[
		s(a, 0) = a, \quad s(1, b + 1) = s(a, b) + 1.
	\]
	We call s(a, b) the sum of a and b $\in \bb{N}$ and write it as
	\[
		a + b = s(a, b).
	\]
	Also, there is a unique function $p: \bb{N} \times \bb{N} \to \bb{N}$ such that for all a, b $\in \bb{N}$ we have
	\[
		p(a, 0) = 0, \quad p(a, b + 1) = p(a, b) + a
	\]
	We call p(a, b) the product of a and b in $\bb{N}$, and we write it as
	\[
		a \cdot b = p(a, b)
	\]
\end{defn}

\begin{defn}[Integers]
	We define the set of integers to be the set
	\[
		\bb{Z} = (\bb{N} \times \bb{N})/R
	\]
	where R is the equivalence relation given by
	\[
		(a, b)R(c,d) \iff a + d = b + c
	\]
	For [(a, b)] and [(c,d)] in $\bb{Z}$, we define
	\begin{gather*}
		[(a, b)] \leq [(c, d)] \iff b + c \leq a + d \\
		[(a, b)] + [(c ,d)] \iff [(a + c, b + d)] \\
		[(a, b)] \cdot [(c, d)] = [(ac + bd, ad + bc)]
	\end{gather*}
	For $n \in \bb{N}$ we write n = [(n, 0)] and -n = [(0, n)], so that every element of $\bb{Z}$ can be written as $\pm n$ for some $n \in \bb{N}$, and we can identity $\bb{N}$ with a subset of $\bb{Z}$
\end{defn}

\begin{defn}[Rational Numbers]
	We define the set of reational numbers to be the set
	\[
		\bb{Q} = (\bb{N} \times \bb{Z}^+)/R
	\]
	where $\bb{Z}^+ = \{x \in \bb{N} | x \neq 0\}$ and R is the equivalence relation given by
	\[
		(a, b)R(c, d) \iff ad = bc
	\]
	For [(a, b)] and [(c, d)] in $\bb{Q}$ we define
	\begin{gather*}
		[(a, b)] \leq [(c, d)] \iff a \cdot d \leq b \cdot c \\
		[(a, b)] + [(c ,d)] \iff [(a \cdot d + b \cdot c, b \cdot d)] \\
		[(a, b)] \cdot [(c, d)] = [(ac, bd)]
	\end{gather*}
	For $a \in \bb{N}$ and $b \in \bb{Z}^+$, it is customary to write $\frac{a}{b} = [(a, b)]$. Also for $a \in \bb{Z}$ we write a = [(a, 1)], and we identify $\bb{Z}$ with a subset of $\bb{Q}$
\end{defn}

\begin{defn}[Real Numbers]
	We define the set of real numbers of the the set
	\[
		\bb{R} = \{x \subseteq \bb{Q} | x \neq \emptyset, x \neq \bb{Q}, \forall a \in x \> \forall b \in \bb{Q}(b \leq a \implies b \in x), \forall a \in x \> \exists b \in x \> a < b \}
	\]
	For $x, y \in \bb{R}$ we define
	\begin{gather*}
		x \leq y \iff x \subseteq y \\
		x + y = \{ a + b | a, b \in \bb{Q}, a \in x, b \in y\}
	\end{gather*}
	For $0 \leq x, y \in \bb{R}$ we define
	\[
		x \cdot y = \{ a \cdot b | 0 \leq a, b \in \bb{Q}, a \in x, b \in y\} \cup \{c \in \bb{Q} | c < 0\},
	\]
	and YOU can try to, similarly, define $x \cdot y$ in the case that x < 0 and y < 0.
\end{defn}

\begin{defn}[Complex Numbers]
	We define the set of complex numbers to be the set
	\[
		\bb{C} = \bb{R} \times \bb{R}.
	\]
	We define addition and multiplication in $\bb{C}$ by
	\begin{align*}
		(a, b) + (c, d) &= (a + c, b + d) \\
		(a, b) \cdot (c, d) &= (ac - bd, ad + bc).
	\end{align*}
	We write $i = (0, 1).$ For $x \in \bb{R}$ we write x = (x, 0) and identify $\bb{R}$ with a subset of $\bb{C}$.
\end{defn}



\chapter{Functions and Cardinality}


\section{Functions}
\begin{defn}[Range, Image, and Inverse Image]
	Let X and Y be sets and let $f: X \to Y$. Recall (see \hyperref[fn_apdxA]{Function in Appendix A}) that the domain of f and the range of f are the sets
	\[
		\text{Domain}(f) = X, \quad \text{Range}(f) = f(X) = \{f(x) | x \in X\}
	\]
	For $A \subseteq X$, the image of A under f is the set
	\[
		f(A) = \{f(x) | x \in A\}
	\]
	For $B \subseteq Y$, the inverse image of B under f is the set
	\[
		f^{-1}(B) = \{x \in X | f(x) \in B\}
	\]
\end{defn}

\begin{defn}[Composite Function]
	Let X, Y and Z be sets. Let $f: X \to Y$ and let $g: Y \to Z$. We define the composite function $g \circ f: X \to Z$ by $(g \circ f)(x) = g \big( f(x) \big)$ for all $x \in X$
\end{defn}

\begin{defn}[Bijection]
	Let X and Y be sets. Let $f: X \to Y$. We say that f is a bijection, or that f is bijective, if f is both one-to-one and onto (or that f is both injective and surjective).
\end{defn}

\begin{thm}[Bijectiveness and Inverse of the Composite Function]
	Let X, Y and Z be sets. Let $f: X \to Y$ and $g: Y \to Z$. Then
	\begin{enumerate}
		\item if f and g are both injective then so is $g \circ f$,
		\item if f and g are both surjective then so is $g \circ f$, and
		\item if f and g are both invertible then so is $g \circ f$, and in this case $(g \circ f)^{-1} = f^{-1} \circ g^{-1}$
	\end{enumerate}
\end{thm}

\begin{defn}[Identity Function]
	For a set X, we define the identity function on X to be the function $I_X: X \to X$ given by $I_X(x) = x$ for all $x \in X$. Note that for $f: X \to Y$ we have $f \circ I_X = f$ and $I_Y \circ f = f$.
\end{defn}

\begin{thm}[Bijectiveness and Invertability of Functions]
	Let X and Y be nonempty sets and let $f: X \to Y$. Then
	\begin{enumerate}
		\item f is injective if and only if f has a left inverse,
		\item f is surjective if and only if f has a right inverse, and
		\item f is bijective if and only if f has a left inverse g and a right inverse h, and in this case we have g = h = $f^{-1}$.
	\end{enumerate}
\end{thm}

\begin{crly}[Relationship between Injection and Surjection]
	Let X and Y be sets. Then there exists an injective map $f: X \to Y$ if and only if there exists a surjective map $g: Y \to X$.
\end{crly}


\section{Cardinality}

\begin{defn}[Equal Cardinality]
	Let A and B be sets. We say that A and B have the same cardinality, and wrtie |A| = |B|, when there exists a bijective map $f: A \to B$.

	We say that the cardinality of A is less than or equal to the cardinality of B, and write $|A| \leq |B|$, when there exists an injective map $f: A \to B$.

	We say that the cardinality of A is less than the cardinality of B, and write $|A| < |B|$, when $|A| \leq |B| \land |A| \neq |B|$ (i.e. there exists an injective map from A to B but no surjective map from A to B).

	We also write $|A| \geq |B|$ when $|B| \leq |A|$ and $|A| > |B|$ when $|B| < |A|$.
\end{defn}

\begin{defn}[Properties for Cardinality of Sets]
	For all sets A, B, and C,
	\begin{enumerate}
		\item |A| = |A|,
		\item if |A| = |B|, then |B| = |A|,
		\item if |A| = |B| and |B| = |C|, then |A| = |C|,
		\item $|A| \leq |B| \iff (|A| = |B| \lor |A| < |B|)$, and
		\item $|A| \leq |B| \land |B| \leq |C| \implies |A| \leq |C|$.
	\end{enumerate}
\end{defn}

\begin{defn}[Finiteness and Countability of Sets]
	Let A be a set. For each $n \in \bb{N}$, let $S_n = \{0, 1, 2, ..., n - 1\}$. For $n \in \bb{N}$, we say that the cardinality of A is equal to n, or that A has n elements, and write |A| = n, when $|A| = |S_n|$. We say that A is finite when |A| = n for some $n \in \bb{N}$. We say that A is infinite when A is not finite. We say that A is countable when $|A| = |\bb{N}|$.
\end{defn}

\begin{remark}
	Note that a set A is said to be countable when A is of the form $A = \{a_0, a_1, a_2, ...\}$ where all its element are distinct.
\end{remark}

\begin{thm}
	Let A be a set. Then the following are equivalent.
	\begin{enumerate}
		\item A is infinite.
		\item A contains a countable subset.
		\item $|\bb{N}| \leq |A|$
		\item There exists a map $f: A \to A$ which is injective but not surjective.
	\end{enumerate}
\end{thm}

\begin{crly}
	Let A and B be sets.
	\begin{enumerate}
		\item If A is countable then A is infinite.
		\item When $|A| \leq |B|$, if B is finite then so is A, and if A is infinite, so is B.
		\item If |A| = n and |B| = m, then |A| = |B| iff n = m.
		\item If |A| = n and |B| = m, then $|A| \leq |B| \iff n \leq m$.
		\item When one of the two sets A or B is finite. If $|A| \leq |B| \land |B| \leq |A| \implies |A| = |B|$.
	\end{enumerate}
\end{crly}

\begin{thm}[$|\bb{N}|$ as a Threshold for Finiteness and Countability]
	Let A be a set. $|A| \leq |\bb{N}| \iff$ A is finite or countable.
\end{thm}

\begin{thm}
	Let A be a set. Then
	\begin{enumerate}
		\item $|A| < |\bb{N}| \iff$ A is finite,
		\item $|\bb{N}| < |A| \iff$ A is neither finite nor countable, and
		\item $|A| \leq |\bb{N}| \land |\bb{N}| \leq |A| \implies |A| = |\bb{N}|$.
	\end{enumerate}
\end{thm}

\begin{defn}[Countability and $\aleph_0$]
	Let A be a set. When A is countable we write $|A| = \aleph_0$.

	When A is finite we write $|A| < \aleph_0$.

	When A is infinite we write $|A| \geq \aleph_0$.

	When A is either finite or countable we write $|A| \leq \aleph_0$, and say that A is at most countable.

	When A is neither finite nor countable we write $|A| > \aleph_0$, and say that A is uncountable.
\end{defn}

\begin{thm}[Set Cartesian Product and Union, and $\bb{Q}$ are Countable]
	\begin{enumerate}
		\item If A and B are countable sets, then so is $A \times B$.
		\item If A and B are countable sets, then so is $A \cup B$.
		\item If $A_0, A_1, A_2, ...$ are countable sets, then so is $\bigcup_{k = 0}^{\infty} A_k$.
		\item $\bb{Q}$ is countable.
	\end{enumerate}
\end{thm}

\begin{remark}
	For a set A, we let $2^A$ denote the set of all functions from A to $S_2 = \{0, 1\}$, i.e.
	\[
		2^A = \{f | f: A \to S_2 \}
	\]
\end{remark}

\begin{thm}[$\bb{R}$ as an Uncountable Set]
	\begin{enumerate}
		\item For every set A, $|\mathcal{P}(A| = |2^A|$.
		\item For every set A, $|A| < |\mathcal{P}(A)|$.
		\item $\bb{R}$ is uncountable.
	\end{enumerate}
\end{thm}

\begin{thm}[Cantor-Schröder-Bernstein Theorem]
	Let A and B be sets.
	\[
		|A| \leq |B| \land |B| \leq |A| \implies |A| = |B|.
	\]
\end{thm}

\end{document}
% Document End