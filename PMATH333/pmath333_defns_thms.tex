% Document Head
\documentclass[11pt, oneside]{book}
\usepackage{geometry}
\geometry{letterpaper}
\usepackage[parfill]{parskip}
\usepackage{graphicx}

% Essential Packages
\usepackage{ragged2e}
\usepackage{amssymb}
\usepackage{amsmath}
\usepackage{mathrsfs}
\usepackage[utf8]{inputenc}
\usepackage[english]{babel}
\usepackage[hyperref]{ntheorem}

% Theorem Style Customization
\setlength\theorempreskipamount{2ex}
\setlength\theorempostskipamount{3ex}

% hyperref Package Settings
\usepackage{hyperref}
\hypersetup{
	colorlinks = true,
	linkcolor = magenta
}

% ntheorem Declarations
\theoremstyle{break}
\newtheorem{thm}{Theorem}[section]
\newtheorem*{proof}{Proof}
\newtheorem{crly}{Corollary}[thm]
\newtheorem{lemma}[thm]{Lemma}
\newtheorem{propo}{Proposition}[section]
\newtheorem*{remark}{Remark}
\newtheorem*{note}{Note}
\newtheorem{defn}{Definition}[section]
\newtheorem{eg}{Example}[section]

% ntheorem listtheorem style
\makeatletter
\def\thm@@thmline@name#1#2#3#4{%
        \@dottedtocline{-2}{0em}{2.3em}%
                   {\makebox[\widesttheorem][l]{\protect\numberline{#2}}#3}%
                   {#4}}
\@ifpackageloaded{hyperref}{
\def\thm@@thmline@name#1#2#3#4#5{%
    \ifx\\#5\\%
        \@dottedtocline{-2}{0em}{2.3em}%
            {\makebox[\widesttheorem][l]{\protect\numberline{#2}}#3}%
            {#4}
    \else
        \ifHy@linktocpage\relax\relax
            \@dottedtocline{-2}{0em}{2.3em}%
                {\makebox[\widesttheorem][l]{\protect\numberline{#2}}#3}%
                {\hyper@linkstart{link}{#5}{#4}\hyper@linkend}%
        \else
            \@dottedtocline{-2}{0em}{2.3em}%
                {\hyper@linkstart{link}{#5}%
                  {\makebox[\widesttheorem][l]{\protect\numberline{#2}}#3}\hyper@linkend}%
                    {#4}%
        \fi
    \fi}
}
\makeatother
\newlength\widesttheorem
\AtBeginDocument{
  \settowidth{\widesttheorem}{Proposition 10\quad}
}

\theoremlisttype{allname}

% Shortcuts
\newcommand{\bb}[1]{\mathbb{#1}}		% using bb instead of mathbb

% Main Body
\title{UW W17 PMATH333: Definitions and Theorems}
\author{Johnson Ng}

\begin{document}
\maketitle
\tableofcontents

\chapter*{List of Definitions}
\theoremlisttype{allname}
\listtheorems{defn}

\chapter*{List of Theorems}
\theoremlisttype{allname}
\listtheorems{lemma, thm, crly, propo}

\appendix

\chapter{Zermelo-Fraenkel Set Theory and the Axiom of Choice}
\section{Introduction}
\begin{eg}[Russel's Paradox]\label{russel_paradox}
	Let X be the set of all sets, and let $S = \{ A \in X | A \notin A\}$.\\
	Note for example that $Z \notin Z \implies Z \in S$, and $X \in X \implies X \notin S$.\\
	Thus we have $S \in S \iff S \notin S$.
\end{eg}

To ensure that mathematical paradoxes (like the above) can no longer arise, mathematicians considered the following questions, and with these questions, rough answers are provided:
\begin{enumerate}
	\item What exactly is an allowable mathematical object? \\
	A: Every mathematical object is a mathematical set, and a mathematical set can be constructed using certain rules, for e.g. the now widely accepted Zermelo-Fraenkel Set Theory and the Axiom of Choice. While the Axiom of Choice is still highly criticized even today (e.g. the highly controversial \href{https://en.wikipedia.org/wiki/Banach–Tarski_paradox}{Banach-Tarski Paradox}), the Zermelo-Fraenkel Set Theory is widely welcomed, but not without critics. We shall call the Zermelo-Fraenkel Set Theory and the Axiom of Choice as the ZFC Axioms of Set Theory.
	\item What exactly is an allowable mathematical statement? \\
	A: Every mathematical statement can be expressed in a formal symbolic language, which uses symbols rather than words from any spoken language.
	\item What exactly is allowable in a mathematical proof? \\
	A: Every mathematical proof is a finite list of ordered pairs $(\mathscr{S}_n, \mathscr{F}_n)$ (which we can think of as proven theorems), where each $\mathscr{S}_n$ is a finite set of formulas (called the $\textit{premises}$) and each $\mathscr{F}_n$ is a single formula (called the $\textit{conclusion}$), which that each pair $(\mathscr{S}_n, \mathscr{F}_n)$ can be obtained from previous pairs $(\mathscr{S}_i, \mathscr{F}_i)$ with $i < n$, using certain proof rules.
\end{enumerate}

In the remainder of this appendix, we shall look more into the first 2 questions.

\begin{defn}[Mathematical Symbols]
	We allow ourselves to use only the following symbols from the following symbol set: \\
	\begin{center}
		\begin{tabular}{c l}
			$\neg$		&	not \\
			$\land$		&	and \\
			$\lor$		&	or \\
			$\implies$	&	implies \\
			$\iff$	&	if and only if \\
			$=$			&	equals \\
			$\in$		&	is an element of \\
			$\forall$	&	for all \\
			$\exists$	&	there exists \\
			$()\quad\{\}\quad[]$	&	parenthesis
		\end{tabular}
	\end{center}
	along with some variable symbols such as $x, y, z, u, v, w,... \text{ or } x_1, x_2, x_3,...$
\end{defn}

\begin{defn}[Formula]
	A formula (in the formal symbolic language of first order set theory) is a non-empty finite string of symbols, from the above list, which can be obtained using finitely many applications following the three rules below:
	\begin{enumerate}
		\item If x and y are variable symbols, then each of the following strings are formulas.
		\[
			x = y, \quad x \in y
		\]
		\item If F and G are formulas then each of the following strings are formulas.
		\[
			\neg F, \quad (F \land G), \quad (F \lor G), \quad (F \implies G), \quad (F \iff G)
		\]
		\item If x is a variable symbol and F is a formula then each of the following is a formula.
		\[
			\forall x \in F, \quad \exists x \in F
		\]
	\end{enumerate}
\end{defn}

\begin{defn}[Free or Bounded Variable]
	Let x be a variable symbol and let F be a formula. For each occurrence of the symbol x, which does not immediately follow a quantifier, in the formula F, we define whether the occurrence of x is free or bound inductively as follows:
	\begin{enumerate}
		\item If F is a formula of one of the forms y = z or $y \in z$, where y and z are variable symbols (possibly equal to x), then every occurence of x in F is free, and no occurrence is bound.

		\item If F is a formula of one of the forms $\neg H, (H \land G), (H \lor G), (H \implies G), (H \iff G)$, where G and H are formulas, then each occurence of the symbol x is either an occurrence in the formula G or an occurrence in the formula H, and each free (respectively, bound) occurence of x in G remains free (respectively, bound) in F, and similarly for each free (or bound) occurrence of x in G. In other words, wlog, if x is bounded in G, then it is bounded in F, and vice versa.
		\item If F is a formula of one of the forms $\forall y \in G \text{ or } \exists y \in G$, where G is a formula and y is a variable symbol. If y is different from x, then each free (or bound) occurrence of x in G remains free (or bound) in the formula G, and if y = x then every free occurrence of x in G becomes bound in F, and every bound occurrence of x in G remains bound in F.
	\end{enumerate}
\end{defn}

\begin{defn}[Is Bound By and Binds]
	When a quantifier symbol occurs in a given formula F, and is followed by the variable symbol x and then by the formula G, any free occurence of x in G will become bound in the given formula F (by the 3rd definition above). We shall say that the occurrence of x is bound by (that occurrence of) the quantifier symbol, or that (the occurrence of) the quantifier symbol binds the occurence of x.
\end{defn}

\begin{defn}[Free Variable, Statement, Statement About]
	A $\textbf{free variable}$ in a formula F is any variable symbol that has at least one free occurence in F. A formula F with no free variables is called a $\textbf{statement}$. When the free variables in F all lie in the set $\{x_1, x_2, ..., x_n\}$, we shall write F as $F(x_1, x_2, ..., x_n)$ and we shall say that F is a $\textbf{statement about}$ the variables $x_1, x_2, ..., x_n$.
\end{defn}

\begin{defn}[Unique Existence]
	When F(x) is a statement about x, we sometimes write F(y) as a short form for the formula $\forall x(x=y\implies F(x))$, and we sometimes write
	\[
		\exists!y \quad F(y)
	\]
	which we read as "there exists a unique y such that F(y)", as a short form for the formula
	\[
		(\exists y \quad F(y) \land \forall z \quad F(z)) \implies z = y))
	\] which is, in turn, for the formula
	\[
		\exists y \bigg( \forall x \Big(x = y \implies F(x) \Big) \land \forall z \Big(\forall x (x = z \implies F(x)) \implies z = y \Big) \bigg)
	\]
\end{defn}

\begin{remark}[The ZFC Axioms of Set Theory (informal)]
	Every mathematical set can be constructed using specific rules, which we shall use the ZFC Axioms of Set Theory. Below is a list of the ZFC Axioms, stated informally.
	\begin{itemize}
		\item Empty Set Axiom: There exists an empty set $\emptyset$ with no elements.
		\item Extension Axiom: 2 sets are equal if and only if they have the same elements.
		\item Separation Axiom: If u is a set and F(x) is a statement about x, $\{x \in u : F(x)\}$ is a set.
		\item Pair Axiom: If u and v are sets then {u, v} is a set.
		\item Union Axiom: If u is a set then $\cup u = \bigcup\limits_{v \in u} v$ is a set.
		\item Power Set Axiom: If u is a set then $\mathcal{P}(u) = \{v : v \in u\}$ is a set.
		\item Axiom of Infinity: If we define the natural numbers to be the sets $0 = \emptyset, 1 = \{0\}, 2 = \{0, 1\}, 3 = \{0, 1, 2\}$ and so on, then $\bb{N} = \{0, 1, 2, 3, ...\}$ is a set.
		\item Replacement Axiom: If u is a ste and F(x, y) is a statement about x and y with the property that $\forall x \> \exists! y \> F(x,y)$ then $\{y : \exists x \in u \> F(x,y)\}$ is a set.
		\item Axiom of Choice: Given a set u of non-empty pairwise disjoint sets, there exists a set which contains exactly one element from each of the sets in u, i.e.
	\end{itemize}

	We may write the Axiom of Choice symbolically as:
	\begin{gather*}
		\forall i \in \bb{N} \quad u_i \neq \emptyset \quad \forall j \neq i \in \bb{N} \quad u_i \cap u_j = \emptyset\\
		\exists v = \{x_1, x_2, x_3, ... : \forall k \in \bb{N}, x_k \in u_k \}
	\end{gather*}
\end{remark}

\begin{defn}[Empty Set Axiom]
	The Empty Set Axiom is the formula
	\[
		\exists u \> \forall x \quad \neg x \in u
	\]
\end{defn}

\begin{defn}[Extension Axiom]
	The Extension Axiom is the formula
	\[
		\forall u \> \forall v \> \Big(u = v \iff \forall x \> (x \in u \iff x \in v) \Big)
	\]
\end{defn}

\begin{thm}
	The empty set is unique.
\end{thm}

\begin{defn}[$\emptyset$]
	We denote the unique empty set by $\emptyset$.
\end{defn}

\begin{defn}[Subset]
	Given sets u and v, we say that u is a $\textbf{subset}$ of v, and write $u \subseteq v$, when $\forall x (x \in u \implies x \in v)$
\end{defn}

\begin{defn}[Separation Axiom]
	For any statement F(x) about x, the following formula is an axiom.
	\[
		\forall u \> \exists v \> \forall x \Big(x \in v \iff (x \in u \land F(x)) \Big)
	\]
	More generally, for any statement $F(x, u_1, u_2, ..., u_n)$ about $x, u_1, u_2, ..., u_n$ where $n \geq 0$, the following formula is an axiom.
	\[
		\forall u \> \forall u_1 \hdots \forall u_n \> \exists v \> \forall x \Big( x \in v \iff (x \in i \land F(x, u_1, ..., u_n)) \Big)
	\]
	Any axiom of this form is called the Separation Axiom.
\end{defn}

\begin{note}
	It is important to realize that a Separation Axiom only allows us to construct a subset of a given set u. So, e.g., we cannot use the Separation Axiom to show that the collection $S = \{x : \neg x \in x\}$, which is used to formulate \hyperref[russel_paradox]{Russel's Paradox}, is a set.
\end{note}

\begin{defn}[Pair Axiom]
	The Pair Axiom is the formula
	\[
		\forall u \> \forall v \> \exists w \> \forall x \Big( x \in w \iff (x = u \lor x = v) \Big)
	\]
\end{defn}

\begin{defn}[Union Axiom]\label{union_axiom}
	The Union Axiom is the formula
	\[
		\forall u \> \exists w \> \forall x \Big( x \in w \iff \exists v (v \in u \land x \in v) \Big)
	\]
\end{defn}

\begin{defn}[Union]
	Given a set u, by the Union Axiom there exists a set w with the property that $\forall x \Big( x \in w \iff \exists v (v \in u \land x \in v) \Big)$, and by the Extension Axiom, this set w is unique. We call the set w the $\textbf{union}$ of the elements in u, and denote it by
	\[
		\cup u = \bigcup_{v \in u} v.
	\]Given two sets u and v, we define the union of u and v to be the set
	\[
		u \cup v := \bigcup \{u, v\}.
	\]Given three sets u, v, and w, note that \{z\} = \{z, z\} is a set and so $\{x, y, z\} = \{x, y\} \cup \{z\}$ is also a set. More generally, if $u_1, u_2, ..., u_n$ are sets then $\{u_1, u_2, ..., u_n\}$ is a set and we define the union of the sets $u_1, u_2, ..., u_n$ to be
	\[
		u_1 \cup u_2 \cup \hdots \cup u_n = \bigcup_{k = 1}^{n} u_k = \bigcup \{u_1, u_2, ..., u_n\}
	\]
\end{defn}

\end{document}
% Document End