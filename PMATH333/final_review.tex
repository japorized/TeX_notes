% Document Head
\documentclass[11pt, oneside]{book}
\usepackage{geometry}
\geometry{letterpaper}
\usepackage[parfill]{parskip}
\usepackage{graphicx}

% Essential Packages
\usepackage{ragged2e}
\usepackage{amssymb}
\usepackage{amsmath}
\usepackage{mathrsfs}
\usepackage[utf8]{inputenc}
\usepackage[english]{babel}
\usepackage[hyperref]{ntheorem}

% Theorem Style Customization
\setlength\theorempreskipamount{2ex}
\setlength\theorempostskipamount{3ex}

% hyperref Package Settings
\usepackage{hyperref}
\hypersetup{
	colorlinks = true,
	linkcolor = magenta
}

% ntheorem Declarations
\theoremstyle{break}
\newtheorem{thm}{Theorem}[section]
\newtheorem*{proof}{Proof}
\newtheorem{crly}{Corollary}[thm]
\newtheorem{lemma}[thm]{Lemma}
\newtheorem{propo}{Proposition}[section]
\newtheorem*{remark}{Remark}
\newtheorem*{note}{Note}
\newtheorem{defn}{Definition}[section]
\newtheorem{eg}{Example}[section]

% ntheorem listtheorem style
\makeatletter
\def\thm@@thmline@name#1#2#3#4{%
        \@dottedtocline{-2}{0em}{2.3em}%
                   {\makebox[\widesttheorem][l]{#1 \protect\numberline{#2}}#3}%
                   {#4}}
\@ifpackageloaded{hyperref}{
\def\thm@@thmline@name#1#2#3#4#5{%
    \ifx\\#5\\%
        \@dottedtocline{-2}{0em}{2.3em}%
            {\makebox[\widesttheorem][l]{#1 \protect\numberline{#2}}#3}%
            {#4}
    \else
        \ifHy@linktocpage\relax\relax
            \@dottedtocline{-2}{0em}{2.3em}%
                {\makebox[\widesttheorem][l]{#1 \protect\numberline{#2}}#3}%
                {\hyper@linkstart{link}{#5}{#4}\hyper@linkend}%
        \else
            \@dottedtocline{-2}{0em}{2.3em}%
                {\hyper@linkstart{link}{#5}%
                  {\makebox[\widesttheorem][l]{#1 \protect\numberline{#2}}#3}\hyper@linkend}%
                    {#4}%
        \fi
    \fi}
}
\makeatother
\newlength\widesttheorem
\AtBeginDocument{
  \settowidth{\widesttheorem}{Proposition A.1.1.1\quad}
}

\theoremlisttype{allname}

% Shortcuts
\newcommand{\bb}[1]{\mathbb{#1}}			% using bb instead of mathbb
\newcommand{\floor}[1]{\lfloor #1 \rfloor}	% simplifying the writing of a floor function
\newcommand{\ceiling}[1]{\lceil #1 \rceil}	% simplifying the writing of a ceiling function
\newcommand{\dotp}{\, \cdotp}				% dot product to distinguish from \cdot

% Main Body
\title{Final Exam Content Revision}
\author{Johnson Ng}

\usepackage{hyperref}

\begin{document}
\maketitle

\tableofcontents

\chapter*{List of Theorems}
\theoremlisttype{allname}
\listtheorems{lemma,thm,crly,propo}


\chapter{Important Theorems}
\begin{thm}[Monotone Convergence Theorem]
	
\end{thm}

\begin{proof}
	
\end{proof}

\begin{thm}[Nested Interval Theorem]
	For each $k \in \bb{N}$, if each $I_k$ is closed, bounded, and nonempty. If $I_0 \supseteq I_1 \supseteq I_2 \supseteq \hdots$, then $\bigcap I_k \neq \emptyset$. Moreover, if $|I_k| \to 0$ as $k \to \infty$, $\bigcap I_k$ is a single point.
\end{thm}

\begin{proof}
	
\end{proof}

\begin{thm}[Intermediate Value Theorem]
	Let $a, b \in \bb{R}$ with $a < b$, let $f: [a, b] \to \bb{R}$ be continous.
	\begin{equation*}
		\forall y \in \bb{R} \enspace \min\{f(a), f(b)\} \leq y \leq \max\{f(a), f(b)\} \implies \exists x \in [a, b] \; f(x) = y
	\end{equation*}
\end{thm}

\begin{proof}
	
\end{proof}

\begin{thm}[Closed Bounded Intervals and Uniform Convergence]
	Let $a, b \in \bb{R}$ with $a < b$. If $f: [a, b] \to \bb{R}$ is continous, then f is uniformly continuous.
\end{thm}

\begin{thm}[Equivalent Definitions of Integrability]
	Let Let $a, b \in \bb{R}$ and $f: [a, b] \to \bb{R}$. TFAE:
	\begin{enumerate}
		\item $f$ is integrable on $[a, b]$.
		\item $\forall \epsilon > 0 \exists \; P \text{ that is a partition} \; U(f, P) - L(f, P) < \epsilon$
		\item $U(f) = L(f)$
	\end{enumerate}
\end{thm}

\begin{proof}
	
\end{proof}

\begin{thm}[Continuous Functions are Integrable]
	Let $a, b \in \bb{R}$. Every continuous function $f: [a, b] \to \bb{R}$ is integrable.
\end{thm}

\begin{proof}
	
\end{proof}

\begin{thm}[Cauchy Criterion for Uniform Convergence - Sequence of Functions]
	Let $I \subseteq \bb{R}, \; \forall n \in \bb{Z}^+ \; f_n: I \to \bb{R}$.
	\begin{gather*}
		\exists g: I \to \bb{R} \; f_n \to g \text{ uniformly on } I \\
		\iff \\
		\forall \epsilon > 0 \; \exists N \in \bb{Z}^+ \; \forall x \in I \; \forall k, l \in \bb{Z}^+ \; (k, l \geq N \implies |f_k(x) - f_l(x)| < \epsilon)
	\end{gather*}
\end{thm}

\begin{proof}
	
\end{proof}

\begin{thm}[Uniform Convergence, Limits and Continuity]
	Supposed $f_n \to f$ uniformly on $I \subseteq \bb{R}$, where $\forall n \in \bb{Z}^+ \; f_n: I \to \bb{R}$. If $\lim\limits_{y \to x} f_n(y)$ exists for each n, then
	\begin{equation*}
	 	\lim_{n \to \infty} \lim_{y \to x} f_n(y) = \lim_{y \to x} \lim_{n \to \infty} f_n(y)
	\end{equation*}
	Furthermore, if $f_n$ is continuous on I for each n, then f is continous on I.
\end{thm}

\begin{proof}
	
\end{proof}

\begin{thm}[Equivalent Topological Definitions]
	Let $A \subseteq \bb{R}^n$.
	\begin{enumerate}
		\item A is closed iff $A' \subseteq A$
		\item $\bar{A} = A' \cup A$
		\item $A^0$ is equal to the set of all interior points of A
		\item $\partial A = \bar{A} \setminus A^0$
	\end{enumerate}
\end{thm}

\begin{proof}
	
\end{proof}

\begin{thm}[Heine-Borel Theorem]
	Every subset of $\bb{R}^n$ is compact iff it is closed and bounded.
\end{thm}

\begin{proof}
	
\end{proof}

\chapter{Important Examples}


\chapter{FAQ for Self}


\chapter{Food for Thought}
\end{document}
% Document End