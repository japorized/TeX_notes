\documentclass[notoc,notitlepage]{tufte-book}
% \nonstopmode % uncomment to enable nonstopmode

\usepackage{classnotetitle}

\title{PMATH433/733 - Model Theory and Set Theory}
\author{Johnson Ng}
\subtitle{Classnotes for Fall 2018}
\credentials{BMath (Hons), Pure Mathematics major, Actuarial Science Minor}
\institution{University of Waterloo}

\usepackage{stmaryrd}
\setcounter{secnumdepth}{3}
\setcounter{tocdepth}{3}

\renewcommand{\baselinestretch}{1.1}
\usepackage{geometry}
\geometry{letterpaper}
\usepackage[parfill]{parskip}
\usepackage{graphicx}

% Essential Packages
\usepackage{makeidx}
\makeindex
\usepackage{enumitem}
\usepackage[T1]{fontenc}
\usepackage{natbib}
\bibliographystyle{apalike}
\usepackage{ragged2e}
\usepackage{etoolbox}
\usepackage{amssymb}
\usepackage{fontawesome}
\usepackage{amsmath}
\usepackage{mathrsfs}
\usepackage{mathtools}
\usepackage{xparse}
\usepackage{tkz-euclide}
\usetkzobj{all}
\usepackage[utf8]{inputenc}
\usepackage{csquotes}
\usepackage[english]{babel}
\usepackage{marvosym}
\usepackage{pgf,tikz}
\usepackage{pgfplots}
\usepackage{fancyhdr}
\usepackage{array}
\usepackage{faktor}
\usepackage{float}
\usepackage{xcolor}
\usepackage{centernot}
\usepackage{silence}
  \WarningFilter*{latex}{Marginpar on page \thepage\space moved}
\usepackage{tcolorbox}
\tcbuselibrary{skins,breakable}
\usepackage{longtable}
\usepackage[amsmath,hyperref]{ntheorem}
\usepackage{hyperref}
\usepackage[noabbrev,capitalize,nameinlink]{cleveref}

% xcolor (scheme: base16 eighties)
\definecolor{base16-eighties-dark}{HTML}{2D2D2D}
\definecolor{base16-eighties-light}{HTML}{D3D0C8}
\definecolor{base16-eighties-magenta}{HTML}{CD98CD}
\definecolor{base16-eighties-red}{HTML}{F47678}
\definecolor{base16-eighties-yellow}{HTML}{E2B552}
\definecolor{base16-eighties-green}{HTML}{98CD97}
\definecolor{base16-eighties-lightblue}{HTML}{61CCCD}
\definecolor{base16-eighties-blue}{HTML}{6498CE}
\definecolor{base16-eighties-brown}{HTML}{D47B4E}
\definecolor{base16-eighties-gray}{HTML}{747369}

% hyperref Package Settings
\hypersetup{
    bookmarks=true,         % show bookmarks bar?
    unicode=true,          % non-Latin characters in Acrobat’s bookmarks
    pdftoolbar=false,        % show Acrobat’s toolbar?
    pdfmenubar=false,        % show Acrobat’s menu?
    pdffitwindow=true,     % window fit to page when opened
    colorlinks=true,
    allcolors=base16-eighties-magenta,
}

% tikz
\usepgfplotslibrary{polar}
\usepgflibrary{shapes.geometric}
\usetikzlibrary{angles,patterns,calc,decorations.markings}
\tikzset{midarrow/.style 2 args={
        decoration={markings,
            mark= at position #2 with {\arrow{#1}} ,
        },
        postaction={decorate}
    },
    midarrow/.default={latex}{0.5}
}
\def\centerarc[#1](#2)(#3:#4:#5)% Syntax: [draw options] (center) (initial angle:final angle:radius)
    { \draw[#1] ($(#2)+({#5*cos(#3)},{#5*sin(#3)})$) arc (#3:#4:#5); }

% enumitems
\newlist{inlinelist}{enumerate*}{1}
\setlist*[inlinelist,1]{%
  label=(\roman*),
}

% Theorem Style Customization
\setlength\theorempreskipamount{2ex}
\setlength\theorempostskipamount{3ex}

\makeatletter
\let\nobreakitem\item
\let\@nobreakitem\@item
\patchcmd{\nobreakitem}{\@item}{\@nobreakitem}{}{}
\patchcmd{\nobreakitem}{\@item}{\@nobreakitem}{}{}
\patchcmd{\@nobreakitem}{\@itempenalty}{\@M}{}{}
\patchcmd{\@xthm}{\ignorespaces}{\nobreak\ignorespaces}{}{}
\patchcmd{\@ythm}{\ignorespaces}{\nobreak\ignorespaces}{}{}

\renewtheoremstyle{break}%
  {\item{\theorem@headerfont
          ##1\ ##2\theorem@separator}\hskip\labelsep\relax\nobreakitem}%
  {\item{\theorem@headerfont
          ##1\ ##2\ (##3)\theorem@separator}\hskip\labelsep\relax\nobreakitem}
\makeatother

% ntheorem + framed
\makeatletter

% ntheorem Declarations
\theorempreskip{10pt}
\theorempostskip{5pt}
\theoremstyle{break}

\newtheorem*{solution}{\faPencil $\enspace$ Solution}
\newtheorem*{remark}{Remark}
\newtheorem{eg}{Example}[section]
\newtheorem{ex}{Exercise}[section]

    % definition env
\theoremprework{\textcolor{base16-eighties-blue}{\hrule height 2pt}}
\theoremheaderfont{\color{base16-eighties-blue}\normalfont\bfseries}
\theorempostwork{\textcolor{base16-eighties-blue}{\hrule height 2pt}}
\theoremindent10pt
\newtheorem{defn}{\faBook \enspace Definition}

    % definition env no num
\theoremprework{\textcolor{base16-eighties-blue}{\hrule height 2pt}}
\theoremheaderfont{\color{base16-eighties-blue}\normalfont\bfseries}
\theorempostwork{\textcolor{base16-eighties-blue}{\hrule height 2pt}}
\theoremindent10pt
\newtheorem*{defnnonum}{\faBook \enspace Definition}

    % theorem envs
\theoremprework{\textcolor{base16-eighties-magenta}{\hrule height 2pt}}
\theoremheaderfont{\color{base16-eighties-magenta}\normalfont\bfseries}
\theorempostwork{\textcolor{base16-eighties-magenta}{\hrule height 2pt}}
\theoremindent10pt
\newtheorem{thm}{\faCoffee \enspace Theorem}

\theoremprework{\textcolor{base16-eighties-magenta}{\hrule height 2pt}}
\theorempostwork{\textcolor{base16-eighties-magenta}{\hrule height 2pt}}
\theoremindent10pt
\newtheorem{propo}[thm]{\faTint \enspace Proposition}

\theoremprework{\textcolor{base16-eighties-magenta}{\hrule height 2pt}}
\theorempostwork{\textcolor{base16-eighties-magenta}{\hrule height 2pt}}
\theoremindent10pt
\newtheorem{crly}[thm]{\faSpaceShuttle \enspace Corollary}

\theoremprework{\textcolor{base16-eighties-magenta}{\hrule height 2pt}}
\theorempostwork{\textcolor{base16-eighties-magenta}{\hrule height 2pt}}
\theoremindent10pt
\newtheorem{lemma}[thm]{\faTree \enspace Lemma}

\theoremprework{\textcolor{base16-eighties-magenta}{\hrule height 2pt}}
\theorempostwork{\textcolor{base16-eighties-magenta}{\hrule height 2pt}}
\theoremindent10pt
\newtheorem{axiom}[thm]{\faShield \enspace Axiom}

    % theorem envs without counter
\theoremprework{\textcolor{base16-eighties-magenta}{\hrule height 2pt}}
\theoremheaderfont{\color{base16-eighties-magenta}\normalfont\bfseries}
\theorempostwork{\textcolor{base16-eighties-magenta}{\hrule height 2pt}}
\theoremindent10pt
\newtheorem*{thmnonum}{\faCoffee \enspace Theorem}

\theoremprework{\textcolor{base16-eighties-magenta}{\hrule height 2pt}}
\theorempostwork{\textcolor{base16-eighties-magenta}{\hrule height 2pt}}
\theoremindent10pt
\newtheorem*{propononum}{\faTint \enspace Proposition}

\theoremprework{\textcolor{base16-eighties-magenta}{\hrule height 2pt}}
\theorempostwork{\textcolor{base16-eighties-magenta}{\hrule height 2pt}}
\theoremindent10pt
\newtheorem*{crlynonum}{\faSpaceShuttle \enspace Corollary}

\theoremprework{\textcolor{base16-eighties-magenta}{\hrule height 2pt}}
\theorempostwork{\textcolor{base16-eighties-magenta}{\hrule height 2pt}}
\theoremindent10pt
\newtheorem*{lemmanonum}{\faTree \enspace Lemma}

\theoremprework{\textcolor{base16-eighties-magenta}{\hrule height 2pt}}
\theorempostwork{\textcolor{base16-eighties-magenta}{\hrule height 2pt}}
\theoremindent10pt
\newtheorem*{axiomnonum}{\faShield \enspace Axiom}

    % proof env
\theoremprework{\textcolor{base16-eighties-brown}{\hrule height 2pt}}
\theoremheaderfont{\color{base16-eighties-brown}\normalfont\bfseries}
\theorempostwork{\textcolor{base16-eighties-brown}{\hrule height 2pt}}
\newtheorem*{proof}{\faPencil \enspace Proof}

    % note and notation env
\theoremprework{\textcolor{base16-eighties-yellow}{\hrule height 2pt}}
\theoremheaderfont{\color{base16-eighties-yellow}\normalfont\bfseries}
\theorempostwork{\textcolor{base16-eighties-yellow}{\hrule height 2pt}}
\newtheorem*{note}{\faQuoteLeft \enspace Note}

\theoremprework{\textcolor{base16-eighties-yellow}{\hrule height 2pt}}
\theorempostwork{\textcolor{base16-eighties-yellow}{\hrule height 2pt}}
\newtheorem*{notation}{\faPaw \enspace Notation}

    % warning env
\theoremprework{\textcolor{base16-eighties-red}{\hrule height 2pt}}
\theoremheaderfont{\color{base16-eighties-red}\normalfont\bfseries}
\theorempostwork{\textcolor{base16-eighties-red}{\hrule height 2pt}}
\theoremindent10pt
\newtheorem*{warning}{\faBug \enspace Warning}

% more environments
\newtcolorbox{redquote}{
  blanker,enhanced,breakable,standard jigsaw,
  opacityback=0,
  coltext=base16-eighties-light,
  left=5mm,right=5mm,top=2mm,bottom=2mm,
  colframe=base16-eighties-red,
  boxrule=0pt,leftrule=3pt,
  fontupper=\itshape
}
\newtcolorbox{bluequote}{
  blanker,enhanced,breakable,standard jigsaw,
  opacityback=0,
  coltext=base16-eighties-light,
  left=5mm,right=5mm,top=2mm,bottom=2mm,
  colframe=base16-eighties-blue,
  boxrule=0pt,leftrule=3pt,
  fontupper=\itshape
}
\newtcolorbox{greenquote}{
  blanker,enhanced,breakable,standard jigsaw,
  opacityback=0,
  coltext=base16-eighties-light,
  left=5mm,right=5mm,top=2mm,bottom=2mm,
  colframe=base16-eighties-green,
  boxrule=0pt,leftrule=3pt,
  fontupper=\itshape
}
\newtcolorbox{yellowquote}{
  blanker,enhanced,breakable,standard jigsaw,
  opacityback=0,
  coltext=base16-eighties-light,
  left=5mm,right=5mm,top=2mm,bottom=2mm,
  colframe=base16-eighties-yellow,
  boxrule=0pt,leftrule=3pt,
  fontupper=\itshape
}
\newtcolorbox{magentaquote}{
  blanker,enhanced,breakable,standard jigsaw,
  opacityback=0,
  coltext=base16-eighties-light,
  left=5mm,right=5mm,top=2mm,bottom=2mm,
  colframe=base16-eighties-magenta,
  boxrule=0pt,leftrule=3pt,
  fontupper=\itshape
}

% ntheorem listtheorem style
\makeatother
\newlength\widesttheorem
\AtBeginDocument{
  \settowidth{\widesttheorem}{Proposition A.1.1.1\quad}
}

\makeatletter
\def\thm@@thmline@name#1#2#3#4{%
        \@dottedtocline{-2}{0em}{2.3em}%
                   {\makebox[\widesttheorem][l]{#1 \protect\numberline{#2}}#3}%
                   {#4}}
\@ifpackageloaded{hyperref}{
\def\thm@@thmline@name#1#2#3#4#5{%
    \ifx\#5\%
        \@dottedtocline{-2}{0em}{2.3em}%
            {\makebox[\widesttheorem][l]{#1 \protect\numberline{#2}}#3}%
            {#4}
    \else
        \ifHy@linktocpage\relax\relax
            \@dottedtocline{-2}{0em}{2.3em}%
                {\makebox[\widesttheorem][l]{#1 \protect\numberline{#2}}#3}%
                {\hyper@linkstart{link}{#5}{#4}\hyper@linkend}%
        \else
            \@dottedtocline{-2}{0em}{2.3em}%
                {\hyper@linkstart{link}{#5}%
                  {\makebox[\widesttheorem][l]{#1 \protect\numberline{#2}}#3}\hyper@linkend}%
                    {#4}%
        \fi
    \fi}
}

\makeatletter
\def\thm@@thmline@noname#1#2#3#4{%
        \@dottedtocline{-2}{0em}{5em}%
                   {{\protect\numberline{#2}}#3}%
                   {#4}}
\@ifpackageloaded{hyperref}{
\def\thm@@thmline@noname#1#2#3#4#5{%
    \ifx\#5\%
        \@dottedtocline{-2}{0em}{5em}%
            {{\protect\numberline{#2}}#3}%
            {#4}
    \else
        \ifHy@linktocpage\relax\relax
            \@dottedtocline{-2}{0em}{5em}%
                {{\protect\numberline{#2}}#3}%
                {\hyper@linkstart{link}{#5}{#4}\hyper@linkend}%
        \else
            \@dottedtocline{-2}{0em}{5em}%
                {\hyper@linkstart{link}{#5}%
                  {{\protect\numberline{#2}}#3}\hyper@linkend}%
                    {#4}%
        \fi
    \fi}
}

\theoremlisttype{allname}

\AtBeginDocument{\renewcommand\contentsname{Table of Contents}}

% Heading formattings
% chapter format
\titleformat{\chapter}%
  {\huge\rmfamily\itshape\color{base16-eighties-magenta}}% format applied to label+text
  {\llap{\colorbox{base16-eighties-magenta}{\parbox{1.5cm}{\hfill\itshape\huge\textcolor{base16-eighties-dark}{\thechapter}}}}}% label
  {5pt}% horizontal separation between label and title body
  {}% before the title body
  []% after the title body

% section format
\titleformat{\section}%
  {\normalfont\Large\rmfamily\itshape\color{base16-eighties-blue}}% format applied to label+text
  {\llap{\colorbox{base16-eighties-blue}{\parbox{1.5cm}{\hfill\itshape\textcolor{base16-eighties-dark}{\thesection}}}}}% label
  {5pt}% horizontal separation between label and title body
  {}% before the title body
  []% after the title body

% subsection format
\titleformat{\subsection}%
  {\normalfont\large\itshape\color{base16-eighties-green}}% format applied to label+text
  {\llap{\colorbox{base16-eighties-green}{\parbox{1.5cm}{\hfill\textcolor{base16-eighties-dark}{\thesubsection}}}}}% label
  {1em}% horizontal separation between label and title body
  {}% before the title body
  []% after the title body

% Sidenote enhancements
\def\mathmarginnote#1{%
  \tag*{\rlap{\hspace\marginparsep\smash{\parbox[t]{\marginparwidth}{%
  \footnotesize#1}}}}
}

% Custom table columning
\newcolumntype{L}[1]{>{\raggedright\let\newline\\\arraybackslash\hspace{0pt}}m{#1}}
\newcolumntype{C}[1]{>{\centering\let\newline\\\arraybackslash\hspace{0pt}}m{#1}}
\newcolumntype{R}[1]{>{\raggedleft\let\newline\\\arraybackslash\hspace{0pt}}m{#1}}

% Custom math operator
% \DeclareMathOperator{\rem}{rem}
\DeclareMathOperator*{\argmax}{arg\,max}
\DeclareMathOperator*{\argmin}{arg\,min}
\DeclareMathOperator{\re}{Re}
\DeclareMathOperator{\im}{Im}
\DeclareMathOperator{\caparg}{Arg}
\DeclareMathOperator{\Ind}{Ind}
\DeclareMathOperator{\Res}{Res}

% Graph styles
\pgfplotsset{compat=1.15}
\usepgfplotslibrary{fillbetween}
\pgfplotsset{four quads/.append style={axis x line=middle, axis y line=
middle, xlabel={$x$}, ylabel={$y$}, axis equal }}
\pgfplotsset{four quad complex/.append style={axis x line=middle, axis y line=
middle, xlabel={$\re$}, ylabel={$\im$}, axis equal }}

% Shortcuts
\newcommand{\floor}[1]{\lfloor #1 \rfloor}      % simplifying the writing of a floor function
\newcommand{\ceiling}[1]{\lceil #1 \rceil}      % simplifying the writing of a ceiling function
\newcommand{\dotp}{\, \cdotp}			        % dot product to distinguish from \cdot
\newcommand{\qed}{\hfill\ensuremath{\square}}   % Q.E.D sign
\newcommand{\abs}[1]{\left|#1\right|}						% absolute value
\newcommand{\lra}[1]{\langle \; #1 \; \rangle}
\newcommand{\at}[2]{\Big|_{#1}^{#2}}
\newcommand{\Arg}[1]{\caparg #1}
\renewcommand{\bar}[1]{\mkern 1.5mu \overline{\mkern -1.5mu #1 \mkern -1.5mu} \mkern 1.5mu}
\newcommand{\quotient}[2]{\faktor{#1}{#2}}
\newcommand{\cyclic}[1]{\left\langle #1 \right\rangle}
	% highlighting shortcuts
\newcommand{\hlimpo}[1]{\textcolor{base16-eighties-red}{\textbf{#1}}}
\newcommand{\hlwarn}[1]{\textcolor{base16-eighties-yellow}{\textbf{#1}}}
\newcommand{\hldefn}[1]{\textcolor{base16-eighties-blue}{\index{#1}\textbf{#1}}}
\newcommand{\hlnotea}[1]{\textcolor{base16-eighties-green}{\textbf{#1}}}
\newcommand{\hlnoteb}[1]{\textcolor{base16-eighties-lightblue}{\textbf{#1}}}
\newcommand{\hlnotec}[1]{\textcolor{base16-eighties-brown}{\textbf{#1}}}
\newcommand{\WTP}{\textcolor{base16-eighties-brown}{WTP} }
\newcommand{\WTS}{\textcolor{base16-eighties-brown}{WTS} }
\newcommand{\ind}[2]{\Ind_{#2}\left( #1 \right)}
\newcommand{\notimply}{\centernot\implies}
\newcommand{\res}[2]{\underset{#2}{\Res} #1 }
\newcommand{\tworow}[3]{\begin{tabular}{@{}#1@{}} #2 \\ #3 \end{tabular}}
\renewcommand{\epsilon}{\varepsilon}
\newcommand{\lrarrow}{\leftrightarrow}
\newcommand{\larrow}{\leftarrow}
\newcommand{\rarrow}{\rightarrow}
\renewcommand{\atop}[2]{\genfrac{}{}{0pt}{}{#1}{#2}}
\newcommand*\dif{\mathop{}\!d}

  % inspiration from: https://tex.stackexchange.com/questions/8720/overbrace-underbrace-but-with-an-arrow-instead#37758
\newcommand{\overarrow}[2]{
  \overset{\makebox[0pt]{\begin{tabular}{@{}c@{}}#2\\[0pt]\ensuremath{\uparrow}\end{tabular}}}{#1}
}
\newcommand{\underarrow}[2]{
  \underset{\makebox[0pt]{\begin{tabular}{@{}c@{}}\downarrow\\[0pt]\ensuremath{#2}\end{tabular}}}{#1}
}

% Document header formatting
\renewcommand{\chaptermark}[1]{\markboth{#1}{}}
\renewcommand{\sectionmark}[1]{\markright{#1}}
\makeatletter
\pagestyle{fancy}
\fancyhead{}
\fancyhead[RO]{\textsl{\@title} \enspace \thepage}
\fancyhead[LE]{\thepage \enspace \textsl{\leftmark \enspace - \enspace \rightmark}}
\makeatother

% Comment the two lines below if you want to print the document
\pagecolor{base16-eighties-dark}
\color{base16-eighties-light}


\newcommand{\class}[1]{\left\llbracket \enspace #1 \enspace \right\rrbracket}
\DeclareMathOperator{\Ord}{Ord }
\DeclareMathOperator{\Set}{Set }

\begin{document}
\hypersetup{pageanchor=false}
\maketitle
\hypersetup{pageanchor=true}
\tableofcontents

\chapter*{\faBook \enspace List of Definitions}
\addcontentsline{toc}{chapter}{List of Definitions}
\theoremlisttype{all}
\listtheorems{defn}

\chapter*{\faCoffee \enspace List of Theorems}
\addcontentsline{toc}{chapter}{List of Theorems}
\theoremlisttype{allname}
\listtheorems{axiom,lemma,thm,crly,propo}

\chapter*{Foreword}%
\label{chp:foreword}
% chapter foreword

This course has a ratio of about 1:3 for naive set theory to model theory.

% chapter foreword (end)

\chapter{Lecture 1 Sep 06th}%
\label{chp:lecture_1_sep_06th}
% chapter lecture_1_sep_06th

\section{Introduction to Set Theory}%
\label{sec:introduction_to_set_theory}
% section introduction_to_set_theory

\newthought{In this course}, we shall focus only on \hlnotea{practical set theory}, which is more commonly knowned as \hlnotea{naive set theory}. In practical set theory, we look at set theory as a language of mathematics. Some of the examples of which we look into in this flavour of set theory are (transfinite) induction and recursion, and the measuring of the sizes of sets.

Another approach to set theory, one that is deemed required in order to learn set theory is a more formal way, is to look at set theory as the foundations of mathematics. Such an approach is more axiomatic, rigorous, and grounding as compared to practical set theory. This course will try to work around going into these topics, as they can take a life of their own, and within the context of this course, the topics that will be explored using this approach are not required.

% section introduction_to_set_theory (end)

\section{Ordinals}%
\label{sec:ordinals}
% section ordinals

\subsection{Zermelo-Fraenkel Axioms}%
\label{sub:zermelo_fraenkel_axioms}
% subsection zermelo_fraenkel_axioms

We use the natural numbers, i.e.
\begin{equation*}
  0, 1, 2, 3, 4, ...
\end{equation*}
to \hlnotea{count} finite sets. There are two related meanings attached to the word ``count'' here:
\begin{itemize}
  \item enumeration; and
  \item measuring (of sizes)
\end{itemize}

In order to facilitate the introduction to certain axioms that we shall need, let our current goal be to develop an infinitary generalization of the natural numbers, so as to be able to enumerate and measure arbitrary sets.\marginnote{We have that
\begin{align*}
  \text{ enumeration } &\to \text{ ordinals } \\
  \text{ measuring } &\to \text{ cardinals }
\end{align*}
where $\to$ represents ``leads to'' here.}

\newthought{To construct} the natural numbers, we require 3 basic notions that shall remain undefined but understood:
\begin{itemize}
  \item a set;
  \item membership, denoted by $\in$; and
  \item equality.
\end{itemize}

One such construction is
\begin{align*}
  0 &:= \emptyset, \text{ the empty set } \\
  1 &:= \{0\} = \{ \emptyset \}, \text{ the set whose only member is } 0 \\
  2 &:= \{0, 1\} = \{ \emptyset, \{ \emptyset \} \}, \text{ the set whose only members are } 0 \text{ and } 1.
\end{align*}

\begin{defn}[Successor]\index{Successor}
\label{defn:successor}
  Given a natural number $n$, the \hlnoteb{successor} of $n$ is the natural number next to $n$, which can be obtained by
  \begin{equation*}
    S(n) := n \cup \{ n \}.
  \end{equation*}
\end{defn}

We can use the definition of a successor to construct the rest of the natural numbers.

\begin{eg}
  Just to verify to ourselves that the definition indeed works, observe that
  \begin{equation*}
    S(1) = 2 = \{ \emptyset, \{ \emptyset \} \} = \{ \emptyset \} \cup \{ \{ \emptyset \} \}.
  \end{equation*}
  So to construct the natural number $3$, we see that
  \begin{align*}
    S(2) &= 3 = \{ 0, 1, 2 \} = 2 \cup \{ 2 \} = \{ \emptyset, \{ \emptyset \} \} \cup \{ \{ \emptyset, \{ \emptyset \} \} \} \\
         &= \{ \emptyset, \{ \emptyset \}, \{ \emptyset, \{ \emptyset \} \} \}
  \end{align*}
\end{eg}

Looking at these, we start wondering to ourselves: how do we know that $\emptyset$ exists in the first place? How do we know that we can use $\cup$ and what does it even mean? Now it is meaningless if we cannot take that $\emptyset$ always exists, nor is it meaningful if we cannot take the $\cup$ of sets. And so, to allow us to continue, or even start with these notions, we require axioms.

\begin{axiom}[Empty Set Axiom]
\index{Empty Set Axiom}
\label{axiom:empty_set_axiom}
  There exists a set, denoted by $\emptyset$, with no members.
\end{axiom}

With this axiom, we can indeed construct $0$. To get $1$ from $0$, we have that $1$ is a set whose only member is zero, and so if we take a member from $1$, that member must be $0$.

\begin{axiom}[Pairset Axiom]
\index{Pairset Axiom}
\label{axiom:pairset_axiom}
  Given set $x, y$, there exists a set, denoted by $\{x , y\}$, whose only members are $x$ and $y$. In other words,
  \begin{equation*}
    t \in \{ x, y \} \lrarrow ( t = x \, \lor \, t = y )
  \end{equation*}
\end{axiom}

Now note that in \cref{axiom:pairset_axiom}, if $x = y$, then the set $\{x, y\}$ has only $x$ as its member. For example, we realize that $1 = \{ 0, 0 \} = \{ 0 \}$. But why exactly does this equality make sense? What exactly does ``realize'' mean?

\begin{axiom}[Axiom of Extension]
\index{Axiom of Extension}
\label{axiom:axiom_of_extension}
  Given sets $x, y$, $x = y$ if and only if $x$ and $y$ have the same members.
\end{axiom}

Now, using the above 3 axioms, we are guaranteed that
\begin{align*}
  0 &= \emptyset \text{ exists by the Empty Set Axiom } \\
  1 &= \{ \emptyset \} \text{ exists by the Pairset Axiom } \\
  2 &= \{ \emptyset, \{ \emptyset \} \} \text{ exists by the Pairset Axiom }
\end{align*}

Now we've constructed $3$ to be the set whose only members are $0, 1$, and $2$. So far, within our axioms, there is no such thing as $\{0, 1, 2\}$, which is what our $3$ is supposed to be. We now require the following axiom:

\begin{axiom}[Union Set Axiom]
\index{Union Set Axiom}
\label{axiom:union_set_axiom}
  Given a set $x$, there exists a set denoted by $\cup x$, whose members are precisely the members of the members of $x$, i.e.
  \begin{equation*}
    t \in \cup x \lrarrow ( t \in y \text{ for some } y \in x )
  \end{equation*}
\end{axiom}

So, by this axiom, we have that given any $n$, $S(n) = \cup \{ n, \{ n \} \}$, or in other words,
\begin{equation*}
  t \in S(n) \lrarrow t \in n \, \lor \, t = n.
\end{equation*}

With all of the above axioms, we can iteratively construct each and every natural number in a rigorous manner. However, our goal is to construct infinitely many of them.

It is tempting to simply take the infinitude of natural numbers simply as an axiom, i.e.

\begin{quotebox}{base16-eighties-magenta}\label{sp:natural_numbers_axiom}
  There exists a set whose members are precisely the natural numbers.
\end{quotebox}

There is a certain rule to which we set down axioms, and that is, axioms must be expressable in a ``finitary'' manner, i.e. they must be expressible using first-order logic.

\begin{defn}[Definite Condition]\index{Definite Condition}
\label{defn:definite_condition}
  We define a \hlnoteb{definite condition} as follows:
  \begin{itemize}
    \item $x \in y$ and $x = y$ are definite conditions, where $x$ and $y$ are both indeterminants, standing for sets, or are sets themselves;
    \item if $P$ and $Q$ are definite conditions, then so are
      \begin{itemize}
        \item not $P$, denoted as $\neg P$;
        \item $P$ and $Q$, denoted as $P \land Q$;
        \item $P$ or $Q$, denoted as $P \lor Q$;
        \item for all $x$, $P$, denoted as $\forall x P$; and
        \item there exists $x$, $P$, denoted as $\exists x P$.
      \end{itemize}
  \end{itemize}
\end{defn}

\begin{eg}
  \begin{equation*}
    x \in 1, 0 \in 2 , 2 \in 0
  \end{equation*}
  are all definite conditions. Note, however, that $2 \in 0$ is false.
\end{eg}

\begin{note}
  ``If $P$ then $Q$'', which is also written as $P \rarrow Q$, is also a definite condition since it is ``\hlnotea{equivalent}''\sidenote{We have yet to define what equivalent statements are but we shall take this for granted for now.} to the statement $\neg P \lor Q$.

  Consequently, $P if and only if Q$, which is also expressed as $P \lrarrow Q$, can be written as
  \begin{equation*}
    (\neg P \lor Q) \land (\neg Q \lor P)
  \end{equation*}
\end{note}

Now, with this definition, and first-order logic notations in mind, we can write:

\marginnote{
  \begin{ex}
    Write \cref{axiom:axiom_of_extension} in first-order logic notation.
  \end{ex}

  \begin{solution}
    \begin{gather*}
      \forall x \; \forall y \\
      ( x = y ) \lrarrow ( \forall t \; ( ( t \in x ) \lrarrow ( t \in y ) ) )
    \end{gather*}
  \end{solution}
}
\begin{itemize}
  \item Empty Set Axiom: $\exists x \; \forall t \; \neg ( t \in x )$
  \item Pairset Axiom: $\forall x \; \forall y \; \exists p \; \forall t \; ( t \in p \lrarrow ( ( t = x ) \lor ( t = y ) ) )$
    \item Union Set Axiom: $\forall x \; \exists z \; \forall t \; ( ( t \in z ) \lrarrow ( \exists y \; ( (y \in x) \land ( t \in y ) ) ) )$
\end{itemize}

Note that the statement that we proposed as an axiom for the set of natural numbers in page \pageref{sp:natural_numbers_axiom} is not definite, although that itself is not obvious.

For example, we may try to write
\begin{equation*}
  \exists x \; ( \forall t \; ( ( t \in x ) \lrarrow ( (t = 0) \lor (t = 1) \lor ( t = 2 ) \lor ... ) ) )
\end{equation*}
and then notice that we do not have the notion of ``$...$'' within the ``tools'' that we are allowed to use.

% subsection zermelo_fraenkel_axioms (end)

% section ordinals (end)

% chapter lecture_1_sep_06th (end)

\chapter{Lecture 2 Sep 11th}%
\label{chp:lecture_2_sep_11th}
% chapter lecture_2_sep_11th

\section{Ordinals (Continued)}%
\label{sec:ordinals_continued}
% section ordinals_continued

\subsection{Zermelo-Fraenkel Axioms (Continued)}%
\label{sub:zermelo_fraenkel_axioms_continued}
% subsection zermelo_fraenkel_axioms_continued

We stopped at the discussion about allowing for an infinite set, so that we can construct our set of infinite natural numbers. The idea here is to take \textit{the smallest set that contains $0$ and is preserved by the successor function}\sidenote{Q: Why the smallest set?}

\begin{axiom}[Infinity Axiom]
\index{Infinity Axiom}
\label{axiom:infinity_axiom}
  There exists a set $I$ that contains $0$ and is preserved by the successor function. We may express this as
  \begin{equation*}
    \exists I ( ( 0 \in I ) \land \forall x ( x \in I \rarrow S(x) \in I ) )
  \end{equation*}
  where we have defined that $S(x) \in I$ means
  \begin{equation*}
    \exists y ( \forall t ( t \in y \lrarrow (t \in x) \lor ( t = x ) ) \land (y \in I) )
  \end{equation*}
  We call $I$ the \hldefn{successor set}.
\end{axiom}

Since we want the smallest of such successor sets, we can try taking the intersection of all successor sets. But before we can do that, we require more axiomatic statements.

\begin{defn}[Subsets]\index{Subsets}
\label{defn:subsets}
  $x \subseteq y$ means that every element of $x$ is an element of $y$, i.e.
  \begin{equation*}
    \forall t ( ( t \in x ) \rarrow ( t \in y ) )
  \end{equation*}
\end{defn}

With a definition of a subset, we can define the Powerset Axiom.

\begin{axiom}[Powerset Axiom]
\index{Powerset Axiom}
\label{axiom:powerset_axiom}
  Given a set $x$, there exists a set $\mathcal{P}(x)$ that contains all subsets of $x$, i.e.
  \begin{equation*}
    \forall t ( ( t \subseteq \mathcal{P}(x) ) \lrarrow \left( t \subseteq x \right) )
  \end{equation*}
\end{axiom}

We also require the following axiom.

\begin{axiom}[(Bounded) Separation Axiom]
\index{Bounded Separation Axiom}\index{Separation Axiom}
\label{axiom:bounded_separation_axiom}
Given a set $x$ and a definition condition $P$, there exists a set whose elements are precisely the members of $x$ that satisfies $P$, i.e.\marginnote{There are two important aspects to the Bounded Separation Axiom:
  \begin{itemize}
    \item it is bounded by the set $x$; and
    \item $P$ is a definite condition.
  \end{itemize}
}
  \begin{equation*}
    \forall x \; \exists y \; \forall t \; ( ( t \in y ) \lrarrow \forall y \; ( ( t \in x ) \land P(t) ) )
  \end{equation*}
  where
  \begin{equation*}
    y = \{ z \in x \mid P(z) \}.
  \end{equation*}
\end{axiom}

\begin{ex}[Set Intersection]\index{Set Intersection}
  Prove that given a non-empty set $x$, there exists a set $\cap x$ satisfying
  \begin{equation*}
    \forall t \; ( ( t \in \cap x ) \lrarrow \forall y \; ( ( y \in x ) \rarrow (t \in y) ) )
  \end{equation*}
\end{ex}

\begin{proof}
  \hlwarn{to be solved}
\end{proof}

\begin{defn}[Natural Numbers]\index{Natural Numbers}
\label{defn:natural_numbers}
Let $I$ be a successor set. The set of natural numbers is\sidenote{We can also write $J \subseteq I$ as $J \in \mathcal{P}(I)$ and invoke the Bounded Separation Axiom.}
  \begin{equation*}
    \omega := \cap \{ J \subseteq I : J \text{ is a successor set } \}
  \end{equation*}
\end{defn}

\begin{note}
  $J$ being a successor set can be expressed by the definite condition
  \begin{equation*}
    ( 0 \in J ) \land \forall x \; ( x \in J \rarrow S(x) \in J ),
  \end{equation*}
  so we can write the definite condition in the above definition by
  \begin{equation*}
    \omega := \cap \{ J \subseteq I : ( 0 \in J ) \lor \forall x \; ( x \in J \rarrow S(x) \in J ) \}
  \end{equation*}
\end{note}

\begin{ex}
  Show that the definition of $\omega$ does not actually depend on $I$, i.e. if given $I_1$ and $I_2$ such that we have
  \begin{align*}
    \omega_1 &= \cap \{ J \subseteq I_1 : J \text{ is a successor set } \} \\
    \omega_2 &= \cap \{ J \subseteq I_2 : J \text{ is a successor set } \}
  \end{align*}
  we have
  \begin{equation*}
    \omega_1 = \omega_2.
  \end{equation*}
\end{ex}

\begin{proof}
  \hlwarn{to be solved}
\end{proof}

Another useful axiom that we will use later is the following:

\begin{axiom}[Replacement Axiom]
\index{Replacement Axiom}
\label{axiom:replacement_axiom}
  Suppose $P$ is a binary definite condition\sidenote{A \hldefn{binary definite condition} has only two variables.} such that for every set $x$, there is a unique $y$ satisfying $P(x, y)$. Given a set $A$, there is a set $B$ such that $t \in B$ if and only if there is an $a \in A$ with $P(a, t)$.
\end{axiom}

\begin{note}
  The slogan for the Replacement Axiom is:
  \begin{quotebox}{base16-eighties-red}
    The image of a set under a definite operation exists.
  \end{quotebox}
\end{note}

These eight axioms, along with another ninth axiom called the Regularity Axiom\sidenote{We shall not discuss too much about this. According to the lecture and the lecture notes, the Regularity Axiom states that every set has a minimal element. On Wikipedia, the axiom states that every set has an element that does not intersect with the set itself.}, constitutes the \hlnotea{Zermelo-Fraenkel Set Theory}.

Note that all axioms, save the \hyperref[axiom:axiom_of_extension]{Extensionality}, assert the existence of sets.

% subsection zermelo_fraenkel_axioms_continued (end)

\subsection{Classes}%
\label{sub:classes}
% subsection classes

There are times where we are interested in a collection of sets that do not form a set themselves.

\begin{eg}[Russell's Paradox]\index{Russell's Paradox}
  There is no set containing all sets.

  \begin{proof}
    Suppose such a set exists, and call it $U$. Now consider the set
    \begin{equation*}
      R := \{ x \in U : x \notin x \},
    \end{equation*}
    which exists by \hyperref[axiom:bounded_separation_axiom]{Bounded Separation}. Observe that
    \begin{gather*}
      R \in R \implies R \notin R \quad \text{\Lightning} \\
      \implies R \notin R \implies R \in R \quad \text{\Lightning}
    \end{gather*}
    Thus such a set $U$ cannot exist.
  \end{proof}
\end{eg}

To talk about such collections, that may or may not be sets, we define \textit{classes}.

\begin{defn}[Class]\index{Class}
\label{defn:class}
  A class is any collection of sets defined by definite property, i.e. given any difinite condition $P$,
  \begin{equation*}
    \class{ z \mid P(z) }
  \end{equation*}
  is the class of all sets satisfying $P$.
\end{defn}

Here, instead of \hyperref[axiom:bounded_separation_axiom]{Bounded Separation}, we have what is called \hlnotea{unbounded separation}.

\begin{note}
  We shall use $\class{}$ rather than $\{ \; \; \}$ to emphasize that we are talking about classes, i.e. we may be talking about non-sets.
\end{note}

\begin{eg}
  \begin{equation*}
    \Set := \class{ z \mid z = z }
  \end{equation*}
  is the universal class of all sets.
\end{eg}

\begin{note}
  \begin{itemize}
    \item Every set is a class.
      \begin{proof}
        Suppose $x$ is a set. We may write
        \begin{equation*}
          x = \class{ z \mid z \in x }.
        \end{equation*}\qed
      \end{proof}

    \item Some classes are not sets; these are called \hldefn{proper classes}. E.g. the universal class of all sets, and
      \begin{equation*}
        \text{Russell} := \class{ z \mid z \notin z }.
      \end{equation*}
  \end{itemize}
\end{note}

% subsection classes (end)

% section ordinals_continued (end)

% chapter lecture_2_sep_11th (end)

\chapter{Lecture 3 Sep 13th}%
\label{chp:lecture_3_sep_13th}
% chapter lecture_3_sep_13th

\section{Ordinals (Continued 2)}%
\label{sec:ordinals_continued_2}
% section ordinals_continued_2

\subsection{Cartesian Products and Function}%
\label{sub:cartesian_products_and_function}
% subsection cartesian_products_and_function

\begin{defn}[Ordered Pairs]\index{Ordered Pairs}
\label{defn:ordered_pairs}
  Given sets $x, y$, an \hlnoteb{ordered pair} of $x$ and $y$ is defined as\sidenote{This invokes the Pairset Axiom thrice.}\marginnote{Why did we not define an ordered pair as
  \begin{equation*}
    (x, y) = \{ \{x\}, \{y \} \}
  \end{equation*}
  instead?}
  \begin{equation*}
    (x, y) = \{ \{ x \}, \{ x, y \} \}
  \end{equation*}
\end{defn}

\begin{note}
  Note that we must have
  \begin{equation*}
    ( (x, y) = (x', y') ) \iff ( x = x' \land y = y' ).
  \end{equation*}

  \begin{proof}
    The $(\impliedby)$ direction is clear by Extensionality. For the other direction, we shall break it into 2 cases:

    \noindent\textbf{Case 1}: $x = y$. Then $\{x, y\} = \{x\}$ by Extensionality, and so
    \begin{equation*}
      (x, y) = \{ \{ x \} \}
    \end{equation*}
    Therefore, we have that
    \begin{equation*}
      \{ \{ x \} \} = (x, y) = (x', y') = \{ \{ x' \}, \{ x', y' \} \}
    \end{equation*}
    So we have
    \begin{equation*}
      \{ x \} = \{ x' \} \implies x = x'
    \end{equation*}
    and
    \begin{equation*}
      \{ x \} = \{ x', y' \} \implies y' = x = y.
    \end{equation*}
    Thus we have
    \begin{equation*}
      x = x' \land y = y'.
    \end{equation*}

    \noindent \textbf{Case 2}: Suppose $x \neq y$ and $x' \neq y'$ \sidenote{If any of them are equal, Case 1 would apply.} We have
    \begin{equation*}
      \{ \{ x \}, \{ x, y \} \} = \{ \{ x' \}, \{ x', y' \} \}
    \end{equation*}
    Then
    \begin{equation*}
      \{ x \} = \{ x' \} \lor \{ x \} = \{ x', y' \}
    \end{equation*}
    The latter leads to a contradiction, since it would imply
    \begin{equation*}
      x' = x = y'.
    \end{equation*}
    Thus $x = x'$. Also, we have
    \begin{equation*}
      \{ x, y \} = \{ x' \} \lor \{ x, y \} = \{ x', y' \}
    \end{equation*}
    Now the former leads to a contradiction since it would imply that
    \begin{equation*}
      x = x' = y.
    \end{equation*}
    Now since $x = x'$, it must be that $y = y'$, otherwise $y = x' = x$ would contradict our assumption. Therefore, we have that
    \begin{equation*}
      x = x' \land y = y'.
    \end{equation*}\qed
  \end{proof}
\end{note}

With ordered pairs, we can build Cartesian products:

\begin{defn}[Cartesian Product]\index{Cartesian Product}
\label{defn:cartesian_product}
  Given classes $X$ and $Y$, the \hlnoteb{Cartesian Product} of $X$ and $Y$ is defined as
  \begin{equation*}
    X \times Y := \class{ z : z = (x, y), x \in X, y \in Y }
  \end{equation*}
\end{defn}

\begin{note}
  We can express this definition using definite conditions;
  \begin{fullwidth}
  \begin{gather*}
    \forall x, y \Bigg( (x \in X) \land (y \in Y) \land \Big( \exists a, b ( \forall t ( t \in a \lrarrow t = x ) ) \land \forall t ( t \in b \lrarrow (t = x) \lor (t = y) ) \Big) \land \\
    \forall t \Big( t \in z \lrarrow \big( (t = a) \lor (t = b) \big) \Big) \Bigg)
  \end{gather*}
  \end{fullwidth}
\end{note}

\begin{note}
  \begin{itemize}
    \item A Cartesian product is a class.
    \item If $A$ is a set and $B$ is a class, and $B \subseteq A$, then $B$ is also a set. This is easy to show: observe that by Extentionality,
      \begin{equation*}
        B = \{ a \in A \mid a \in B \}.
      \end{equation*}
      By Bounded Separation Axiom, $B$ is a set\sidenote{This statement can be rephrased as: subclasses of a set are subsets.}.
  \end{itemize}
\end{note}

Consequently, Cartesian products of sets are sets themselves; if $X$ and $Y$ are sets, we want to show that $X \times Y$ is a set so it is sufficient to show that it is contained in one. Recall that
\begin{equation*}
  (x, y) = \{ \{ x \}, \{ x, y \} \}
\end{equation*}
and $\{ x, y \} \subset X \cup Y$ which means $\{ x, y \} \in \mathcal{P}(X)$, and we observe that $\{ x \} \in \mathcal{P}(X \cup Y)$. So $(x, y) \in \mathcal{P}(X \cup Y)$. Therefore, $X \times Y \subset \mathcal{P} ( \mathcal{P} (X \cup Y) )$, and we show to ourselves that $X \times Y$ is indeed a set.

\begin{defn}[Definite Operation]\index{Definite Operation}
\label{defn:definite_operation}
  Given classes $X$ and $Y$, a \hlnoteb{definite operation} $f: X \to Y$ is a subclass $\Gamma(f) \subseteq X \times Y$ such that
  \begin{equation*}
    \forall x \in X \; \exists ! y \in Y \; (x, y) \in \Gamma(f).
  \end{equation*}
\end{defn}

\begin{note}
  We write $f(x) = y$ to mean $(x, y) \in \Gamma(f)$. We also refer to $\Gamma(f)$ as the \hldefn{graph of $f$}.
\end{note}

\begin{eg}
  The successor function $S : \Set \to \Set$ is a definite operation such that\marginnote{To show that $S$ is a definite operation, we need to show that $S$ is a definite condition.}
  \begin{equation*}
    S(x) = x \cup \{x\}
  \end{equation*}
  This is true since is can be expressed as
  \begin{equation*}
    \forall t ( t \in y \lrarrow ( t \in x \lor t = x ) ).
  \end{equation*}
\end{eg}

\begin{note}
  If $X$ and $Y$ are sets and $f$ is a definite operation, then $\Gamma(f) \subseteq X \times Y$ is a set. In such a case, we call $f$ a function.
\end{note}

\begin{defn}[Functions]\index{Functions}
\label{defn:functions}
  A function is a definite operation $f : X \to Y$ where $X$ and $Y$ are both sets.
\end{defn}

We can now restate the Replacement Axiom.

\begin{axiom}[Replacement Axiom (Restated)]
\index{Replacement Axiom}
\label{axiom:replacement_axiom_restated}
  If $f : X \to Y$ is a definite operation, and $A \subseteq X$ is a set, then $\exists B \subseteq Y$ that is a set such that $t \in B$ if and only if $t = f(a)$ for some $a \in A$.
\end{axiom}

% subsection cartesian_products_and_function (end)

\subsection{The Natural Numbers}%
\label{sub:the_natural_numbers}
% subsection the_natural_numbers

\begin{thm}[Induction Principle]
\index{Induction Principle}
\label{thm:induction_principle}
  Suppose $J \subseteq \omega$, $0 \in J$ and whenever $n \in J$, $S(n) \in J$. Then $J = \omega$.
\end{thm}

\begin{proof}
  By assumption, $J$ is a successor set, therefore $\omega \subseteq J$ by definition. Thus, sinec $J \subseteq \omega$, we have $J = \omega$.\qed
\end{proof}

\begin{lemma}[Properties of the Natural Numbers]
\label{lemma:properties_of_the_natural_numbers}
  Suppose $n \in \omega$. We have
  \begin{enumerate}
    \item $n \subseteq \omega$;
    \item $\forall m \in n \quad m \subseteq n$;
    \item $n \notin n$;
    \item $n = 0 \veebar 0 \in n$; and
    \item $y \in n \implies S(y) \in n \veebar S(y) = n$.
  \end{enumerate}
\end{lemma}

\begin{proof}
  \begin{enumerate}
    \item Let\sidenote{We construct this $J$ and show that it is a successor set. Note that if $J = \omega$, our proof is complete.}
      \begin{equation*}
        J := \{ n \in \omega : n \subseteq \omega \} \subseteq \omega.
      \end{equation*}
      Note that $\emptyset \subseteq \omega$ and so $0 \subseteq \omega$. By membership, $0 \in J$.
      
      Suppose $m \in J$. Consider $S(m) = m \cup \{ m \}$. Since $J \subseteq \omega$, $m \in \omega$. Since $m \in \omega$, $\{ m \} \subseteq \omega$. Therefore $S(m) = m \cup \{ m \} \subseteq \omega$, and so $S(m) \in J$. So $J$ is a successor set. And thus by \hyperref[thm:induction_principle]{Induction Principle}, $J = \omega$.

    \item Let
      \begin{equation*}
        J := \{ n \in \omega: \forall m \in n, m \subseteq n \}.
      \end{equation*}
      It is vacuously true that $0 \in J$ since $\emptyset$ is a subset of every $n \in J$. Suppose $n \in J$. Then $\forall m \in n$, we have $m \subseteq n$. Consider $S(n) = n \cup \{ n \}$. Note that $n \in S(n)$ and $n \subseteq S(n)$. For $x \in S(n)$ such that $x \neq n$, we must have that $x \in n$. By assumption, $x \subseteq n \subseteq S(n)$. Therefore, $S(n) \in J$, and so $J$ is a successor set. By the Induction Principle, $J = \omega$.

    \item Let
      \begin{equation*}
        J := \{ n \in \omega : n \notin n \}.
      \end{equation*}
      We have $0 = \emptyset \notin \emptyset$. So $0 \in J$.

      Let $n \in J$. Consider $S(n) = n \cup \{n\}$. In particular, note that $n \in S(n)$. Suppose, for contradiction, that $S(n) \in S(n)$. Then $S(n) = n$ or $S(n) \in n$.

      $S(n) = n \implies n \in S(n) = n$ \Lightning $n \notin n$.

      $S(n) \in n \implies S(n) \subseteq n$ by part 2 $\implies n \in n$ \Lightning $n \notin n$.

      Thus $S(n) \notin S(n)$ and so $S(n) \in J$. So $J$ is a successor set, and so by the \hyperref[thm:induction_principle]{Induction Principle}, $J = \omega$.

    \item It suffices to show that
      \begin{equation*}
        \omega = \{0\} \cup \{ n \in \omega: 0 \in n \}.
      \end{equation*}
      Let $J =$ RHS. We have that $0 \in J$. Suppose $n \in J$ such that $n \neq 0$. Then $0 \in n$. Since $n \subseteq S(n) = n \cup \{n\}$, we have that $0 \in S(n)$. Therefore, $S(n) \in J$. So $J$ is a successor set, and so by the Induction Principle, $J = \omega$ as required.

    \item Let
      \begin{equation*}
        J := \{ n \in \omega : y \in n \implies S(y) \in n \veebar S(y) = n \}.
      \end{equation*}
      $0 \in J$ is vacuously true, since there are no $y \in 0$. Suppose $n \in J$. Let $y \in S(n) = n \cup \{ n \}$. We have two choices: either $y \in n$ or $y = n$. If $y \in n$, then $S(y) \in n \veebar S(y) = n$, since $n \in J$. We have that

      $S(y) \in n \subseteq S(n)$ in which case we are done; and

      $(sy) \subseteq n \in S(n)$.

      Otherwise, if $y \notin n$, then $y = n$. Then we simply have $S(y) = S(n)$. Thus $J$ is a succesor set and so by the Induction Principle, $J = \omega$.
  \end{enumerate}\qed
\end{proof}

% subsection the_natural_numbers (end)

\subsection{Well-Orderings}%
\label{sub:well_orderings}
% subsection well_orderings

\begin{defn}[Strict Partially Ordered Set]\index{Strict Partially Ordered Set}\index{Strict Poset}
\label{defn:strict_partially_ordered_set}
  A \hlnoteb{strict partially ordered set} (or \hlnoteb{strict poset}\sidenote{This is my unofficial terminology}) is a set $E$ together with $R \subseteq E^2 = E \times E$ such that
  \begin{enumerate}
    \item (\hlnotea{anti-reflexive}) $\forall a \in E \quad (a, a) \notin R$;
    \item (\hlnotea{anti-symmetric}) $\forall a, b \in E \quad (a, b) \in R \land (b, a) \in R \implies a = b$; and
    \item (\hlnotea{transitivity}) $\forall a, b, c \in E \quad (a, b), (b, c) \in R \implies (a, c) \in R$.
  \end{enumerate}
\end{defn}

\begin{defn}[Strict Totally Ordered Set]\index{Strict Totally Ordered Set}\index{Strict Linearly Ordered Set}
\label{defn:strict_totally_ordered_set}
  A strict poset is \hlnoteb{total} (or \hlnoteb{linear}) if
  \begin{equation*}
    \forall a, b \in E \quad (a, b) \in R \veebar (b, a) \in R
  \end{equation*}
\end{defn}

\begin{defn}[Well-Order]\index{Well-Order}
\label{defn:well_order}
  A strict linear order is \hlnoteb{well-ordered} if
  \begin{equation*}
    \forall X \subseteq E ( X \neq \emptyset ) \quad \exists a \in X \quad \forall b \in X ( b \neq a ) \quad (a, b) \in R
  \end{equation*}
  i.e. every nonempty subset of $E$ has a \hlimpo{least element}.
\end{defn}

We shall prove the following next lecture.\sidenote{Anti-reflexivity and Anti-symmetry were proven in this lecture, but I am moving it to the next for ease of reading.}

\begin{propononum}[$\omega$ is Strictly Well-ordered]
\label{propo:_omega_is_strictly_well_ordered}
  $(\omega, \in)$ is a strict well-ordering.
\end{propononum}

% subsection well_orderings (end)

% section ordinals_continued_2 (end)

% chapter lecture_3_sep_13th (end)

\chapter{Lecture 4 Sep 18th}%
\label{chp:lecture_4_sep_18th}
% chapter lecture_4_sep_18th

\section{Ordinals (Continued 3)}%
\label{sec:ordinals_continued_3}
% section ordinals_continued_3

\subsection{Well-Orderings (Continued)}%
\label{sub:well_orderings_continued}
% subsection well_orderings_continued

\marginnote{
  \begin{marginlemmanonum}[\cref{lemma:properties_of_the_natural_numbers}]
  Suppose $n \in \omega$. We have
  \begin{enumerate}
    \item $n \subseteq \omega$;
    \item $\forall m \in n \quad m \subseteq n$;
    \item $n \notin n$;
    \item $n = 0 \veebar 0 \in n$; and
    \item $y \in n \implies S(y) \in n \veebar S(y) = n$.
  \end{enumerate}
\end{marginlemmanonum}
}
\begin{propo}[$\omega$ is Strictly Well-Ordered]
\label{propo:_omega_is_strictly_well_ordered}
  $(\omega, \in)$ is a strict well-ordering.
\end{propo}

\begin{proof}
  By \cref{lemma:properties_of_the_natural_numbers}, we have that $\forall n \in \omega$, $n \notin n$. (\hlnotea{anti-reflexivity \faCheck}).

  \noindent $\forall n, m \in \omega$, suppose, for contradiction, that $n \in m$ and $m \in n$. Again, by \cref{lemma:properties_of_the_natural_numbers}, we have $n \subseteq m$ and $m \subseteq n$, which implies that $n = m$. Thus, we have $n \in m = n$ and $m \in n = m$, a contradiction to the fact that $n \notin n$ and $m \notin m$ (\hlnotea{anti-symmetry \faCheck}).

  \noindent $\forall x, y, z \in \omega$ such that $x \in y$ and $y \in z$, by \cref{lemma:properties_of_the_natural_numbers}, $y \in z \implies y \subseteq z \implies x \in z$ (\hlnotea{transitivity \faCheck}).

  To show totality of the relation, let $n \in \omega$. WTS for any $m \in \omega$, either
  \begin{equation*}
    m \in n, \quad m = n, \text{ or } n \in m.
  \end{equation*}
  Let\sidenote{We construct $J$ such that $J$ will contain all the possible cases, and use this fact to prove that $J = \omega$ so these 3 cases are the only scenarios that can happen.}
  \begin{equation*}
    J = \underset{ \in n }{n} \cup \underset{= n}{ \{ n \} } \cup \underset{> n}{ \{ m \in \omega : n \in m \} }.
  \end{equation*}
  \noindent\underline{Case 1: $n = 0$.} In this case, we have\sidenote{Note that $0 = \emptyset$.}
  \begin{equation*}
    J = \emptyset \cup \{ \emptyset \} \cup \{ m \in \omega : 0 \in m \}
  \end{equation*}
  As a consequence of \cref{lemma:properties_of_the_natural_numbers} (4), we have that $J = \omega$.

  \noindent\underline{Case 2: $n \neq 0$.} Again, by \cref{lemma:properties_of_the_natural_numbers} (4), since $n \neq 0$, we must have $0 \in n \subseteq J$ and so $0 \in J$. Now suppose that $m \in J$.

  \underline{Case 2(a): $m \in n$.} Then by \cref{lemma:properties_of_the_natural_numbers} (5), $S(m) \in n$ or $S(m) = n$.\\
  $S(m) \in n \implies S(m) \in J$\\
  $S(m) \in n \implies S(m) \in J$

  \underline{Case 2(b): $m = n$.} Then $S(m) = S(n) = n \cup \{ n \}$. And so $n \in S(m)$, which implies $S(m) \in J$.

  \underline{Case 2(c): $n \in m$} Then since $S(m) = m \cup \{ m \}$, we have that $m \in m \subseteq S(m)$. Therefore $S(m) \in J$.

  Therefore, $J$ is a sucessor subset of $\omega$. Thus by the Induction Principle, $J = \omega$. (\hlnotea{totality \faCheck})

  To prove that $\in$ is a well-ordering, suppose $X \subseteq \omega$ is non-empty. Suppose, for contradiction, that $X$ has no $\in$-least element. Now consider
  \begin{equation*}
    J = \{ n \in \omega : S(n) \cap X = \emptyset \}
  \end{equation*}
  \noindent\underline{Claim: $J$ is a successor set.}\sidenote{Since we want to prove that $\in$ is a well-ordering, we can suppose that there is a non-empty subset of $\omega$ that is not empty, and has no $\in$-least element. The core idea here is that, by the construction of $J$, if $J = \omega$, then all elements of $\omega$ would be disjoint from $X$, forcing $X$ to be the empty set.}

  By \cref{lemma:properties_of_the_natural_numbers} (4), $0$ is the $\in$-least element of $\omega$. If $0 \in X$, then $0$ would be $\in$-least in $X$, contradicting our supposition. Thus $0 \notin X$, And so
  \begin{equation*}
    S(0) \cap X = ( 0 \cup \{ 0 \} ) \cap X = \{ 0 \} \cap X = \emptyset
  \end{equation*}
  since $0 \notin X$. Thus $0 \in J$.

  Suppose $n \in J$. By construction of $J$, we have $S(n) \cap X = \emptyset$. Observe that
  \begin{equation*}
    S(S(n)) \cap X = ( S(n) \cup \{ S(n) \} ) \cap X.
  \end{equation*}
  Now if RHS of the above is non-empty (aiming for contradiction), then we may have $S(n) \in X$. Then $S(n)$ would be the $\in$-least element in $X$, a contradiction. If $m \in S(n)$, we have that $m \notin X$ since $S(n) \cap X = \emptyset$. Thus $SS(n) \cap X = \emptyset$ and so $S(n) \in J$. Therefore, by the Induction Principle, $J = \omega$.

  We observe that $\forall n \in \omega$,
  \begin{equation*}
    \emptyset = S(n) \cap X) = ( n \cup \{ n \} ) \cap X
  \end{equation*}
  $\implies n \notin X$, and so we must have $X = \emptyset$ (\hlnotea{well-ordered \faCheck}).\qed
\end{proof}

\begin{note}
  Given $n, m \in \omega$, we often write $n < m$ to mean $n \in m$.
\end{note}

\begin{defn}[Ordinals]\index{Ordinals}
\label{defn:ordinals}
  An \hlnoteb{ordinal} is a set $\alpha$ satisfying:
  \begin{enumerate}
    \item $x \in \alpha \implies x \subseteq \alpha$;
    \item $(\alpha, \in)$ is a strict well-ordering.
  \end{enumerate}
\end{defn}

\begin{eg}
  $\omega$ is an ordinal: $\forall n \in \omega$, by \cref{lemma:properties_of_the_natural_numbers}, $n \subseteq \omega$, and $\omega$ is proven to have a strict well-ordering under $\in$.
\end{eg}

\begin{eg}
  Every natural number is an ordinal (\textit{finite ordinals}): by \cref{lemma:properties_of_the_natural_numbers} (2), the first property is satisfied; well-ordering follows from the property of $\omega$.
\end{eg}

Let $\Ord$ denote the class of all ordinals. We shall show later that $\Ord$ is a proper class.

\begin{ex}
  Verify that for a set to be an ordinal is a definite condition.

  \begin{fullwidth}
  \resizebox{\linewidth}{!}{
  \begin{equation*}
    \forall t ( t \in \Ord \lrarrow ( \underbrace{ \forall x ( x \in t \rarrow \forall a ( a \in x \rarrow a \in t ) ) }_{x \in t \implies x \subseteq t} \land \underbrace{ \forall s ( s \subseteq t \land s \neq \emptyset \rarrow \exists a ( a \in s \rarrow \forall b ( b \in s \land b \neq a \rarrow (a, b) \in (\in)) ) ) }_{(t, \in) \text{ is a strict well-ordering }} ) )
  \end{equation*}
  }
  \end{fullwidth}
\end{ex}

\begin{lemma}[Proper Subsets of an Ordinal Are Its Elements]
\label{lemma:proper_subsets_of_an_ordinal_are_its_elements}
  If $\alpha, \beta \in \Ord$ and $\alpha \subsetneq \beta$, then $\alpha \in \beta$.
\end{lemma}

\begin{proof}
  We shall prove that $\alpha$ is the least element in $\beta$ that is not in $\alpha$ itself.\sidenote{We shall construct a subset of $\beta \setminus \alpha$ and show that $\alpha$ is its element.}

  Let $D := \beta \setminus \alpha = \{ x \in \beta : x \notin \alpha \} \subset \beta$ \sidenote{Exists by Bounded Separation Axiom.}. Since $\alpha \subsetneq \beta$, $D \neq \emptyset$. Since $\beta \in \Ord$, $(\beta, \in)$ has a strict well-ordering, and so $D$ has a least element, $d$. Note that $d \in \beta$, and since $\beta \in \Ord$, $d \subseteq \beta$.

  \noindent\underline{Claim: $\alpha = d$.}\sidenote{If $\alpha = d$, then $\alpha$ is the said least element.} WTS $\alpha \subseteq d$. $\forall x \in \alpha$, we have $x, d \in \beta$. Then since $(\beta, \in)$ is a strict well-ordering, we have either
  \begin{equation*}
    x < d, \quad x = d, \; \text{ or } d < x
  \end{equation*}
  Note that $x \neq d$, otherwise $x = d \in D = \beta \setminus \alpha$.

  \sidenote{
    This is an errorneous proof.
    \begin{marginwarning}
      \begin{gather*}
      d < x \land x < \alpha \\
      \underset{\text{transitivity}}{\implies} d < \alpha \implies d \in \alpha
      \end{gather*}

      This argument is errorneous because we do not yet know if $\alpha \in \beta$.
    \end{marginwarning}} If $d < x$, then $d \in x$ (by our notation). Now since $\alpha \in \Ord$, $x < \alpha \implies x \in \subseteq$, and so $d \in \alpha$, which is yet another contradiction ($d \in D = \beta \setminus \alpha$).

    Thus we must have $x < d$, i.e. $x \in d$. So $\alpha \subseteq d$.

    WTS $d \subseteq \alpha$. Suppose not. Then let $x \in d \setminus \alpha$. Then since $d \in D = \beta \setminus \alpha$, we have $x \in \beta \setminus \alpha$, which then contradicts the minimality of $d$. Therefore, $d = \alpha$ as required.\qed
\end{proof}

\begin{propo}[Properties of Ordinals]
\label{propo:properties_of_ordinals}
\marginnote{
  Some of proofs of these properties are available in the course notes.
  \begin{ex}
    Prove \cref{item:properties_of_ordinals_3}, \cref{item:properties_of_ordinals_4}, and \cref{item:properties_of_ordinals_5} of \cref{propo:properties_of_ordinals}.
  \end{ex}
}
  \begin{enumerate}
    \item Every member of an ordinal is an ordinal.\label{item:properties_of_ordinals_1}
    \item $\alpha \in \Ord \implies \alpha \notin \alpha$.\label{item:properties_of_ordinals_2}
    \item $\alpha \in \Ord \implies S(\alpha) \in \Ord$.\label{item:properties_of_ordinals_3}
    \item $\alpha, \beta \in \Ord \implies \alpha \cap \beta \in \Ord$.\label{item:properties_of_ordinals_4}
    \item $\alpha, \beta \in \Ord \implies \alpha \in \beta \lor \alpha = \beta \lor \beta \in \alpha$.\label{item:properties_of_ordinals_5}
    \item $E \subseteq \Ord$ a subset $\implies (E, \in)$ is a strict well-ordering.\label{item:properties_of_ordinals_6}
    \item $\Ord$ is a proper class.\label{item:properties_of_ordinals_7}
  \end{enumerate}
\end{propo}

\begin{proof}
  \begin{enumerate}
    \item Suppose $x \in \beta \in \Ord$. WTS $x \in \Ord$, and we shall show that $x$ satisfies \cref{defn:ordinals}.

      Since $\beta \in \Ord, \, x \in \beta \implies x \subseteq \beta$. Thus $(x, \in)$ is a strict well-ordering (through inheriting the property). So it suffices to show that $y \in x \implues y \subseteq x$. So let $y \in x$, and let $t \in x$ \sidenote{To show that $y \subseteq x$, we need to show that $\forall t \in y$, $t \in x$.}. Observe that
      \begin{align*}
        t \in y &\implies t < y \\
        y \in x &\implies y < x
      \end{align*}
      and $t, y, x \in \beta \in \Ord$. Therefore, by transitivity, we have $t < y < x \implies t \in x$.

    \item Suppose not, i.e. $\alpha \in \alpha$. Then $\alpha \subseteq \alpha \in \Ord$, and so $(\alpha, \in)$ is a strict well-ordering, i.e. $\alpha \notin \alpha$, a contradiction.

    \setcounter{enumi}{5}
    \item Suppose $A \subseteq E$ and $A \neq \emptyset$. Let $\alpha \in A$.

      \noindent\underline{Case 1: $\alpha \cap A = \emptyset$.} Then $\forall \beta \in \alpha \implies \beta \notin A$. Therefore $\alpha$ is $\in$-least in $A$.

      \noindent\underline{Case 2: $a \cap A \neq \emptyset$.} Let $A' = \alpha \cap A \subseteq \alpha$. Since $\alpha \in A \subseteq E \subseteq \Ord$, we have $(\alpha, \in)$ is a strict well-ordering, and so $A'$ has a strict well-ordering as well, and thus it must have a $\in$-least element, $x$. Then $x$ is the $\in$-least element in $A$.

    \item If $\Ord$ is a set, then by \cref{item:properties_of_ordinals_6}, $(\Ord, \in)$ is a strict well-ordering. Also, by \cref{item:properties_of_ordinals_1}, every element of $\Ord$ is a subset of $\Ord$. Therefore, $\Ord$ satisfies \cref{defn:ordinals}, and so $\Ord \in \Ord$, which contradicts \cref{item:properties_of_ordinals_2}. Therefore $\Ord \notin \Set$.
  \end{enumerate}\qed
\end{proof}

% subsection well_orderings_continued (end)

% section ordinals_continued_3 (end)

% chapter lecture_4_sep_18th (end)

\appendix

\backmatter

\pagestyle{plain}

\nobibliography*
% \bibliography{references}

\printindex

\end{document}

