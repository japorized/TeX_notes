\chapter{Lecture 7 May 16th 2018}%
\label{chp:lecture_7_may_16th_2018}
% chapter lecture_7_may_16th_2018

\section{Subgroups (Continued 3)}%
\label{sec:subgroups_continued_3}
% section subgroups_continued_3

\subsection{Order of Elements (Continued)}%
\label{sub:order_of_elements_continued}
% subsection order_of_elements_continued

\begin{eg}
  Consider $(\mathbb{Z}, +)$ . Note that $\forall k \in \mathbb{Z}$, we can write $k = k \cdot 1 = \underbrace{1 + 1 + \hdots + 1}_{k \text{times}}$. So we have that $(\mathbb{Z} , +) = \lra{1}$. Similarly, we would have $(\mathbb{Z} , +) = \lra{-1}$.

\noindent However, observe that $\forall n \in \mathbb{Z}$ with $n \neq \pm 1$, there is no $k \in \mathbb{Z} $ such that $k \cdot n = 1$. Therefore, $\pm 1$ are the only \hlnotea{generators} of $\mathbb{Z}$.
\end{eg}

\newthought{Let} $G$ be a group and $g \in G$. Suppose $\exists k \in \mathbb{Z}$ with $k \neq 0$ such that $g^k = 1$. Then $g^{-k} = ( g^k )^{-1} = 1$. Thus wlog, we can assume that $k \geq 1$. By the \hlnotea{Well Ordering Principle}, $\exists n \in \mathbb{N}$ such that $n$ is the smallest, such that $g^n = 1$.

With that, we may have the following definition:

\begin{defn}[Order of an Element]\index{Order of an Element}
\label{defn:order_of_an_element}
  Let $G$ be a group and $g \in G$. If $n$ is the smallest positive integer such that $g^n = 1$, we say that the order of $g$ is $n$, denoted by $o(g) = n$.

  \noindent If no such $n$ exists, then we say that $g$ has infinite order and write $o(g) = \infty$.
\end{defn}

\begin{propo}[Properties of Elements of Finite Order]
\label{propo:properties_of_elements_of_finite_order}
  Let $G$ be a group with $g \in G$ where $o(g) = n \in \mathbb{N}$. Then
  \begin{enumerate}
    \item $g^k = 1 \iff n | k$;
    \item $g^k = g^m \iff k \equiv m \mod n$; and
    \item $\lra{g} = \{1, g, g^2, ..., g^{n - 1} \}$ where each $g^i$ is distinct from others.
  \end{enumerate}
\end{propo}

\begin{proof}
  \begin{enumerate}
    \item $(\impliedby)$ If $n | k$, then $k = nq$ for some $q \in \mathbb{Z}$. Then
      \begin{equation*}
        g^k = g^{nq} = (g^n)^q = 1^q = 1
      \end{equation*}

      $(\implies)$ Suppose $g^k = 1$. Since $k \in \mathbb{Z}$, the \hlnotea{Division Algorithm}, we can write $k = nq + r$ with $q, r \in \mathbb{Z}$ and $0 \leq r < n$. Note $g^n = 1$. Thus
      \begin{equation*}
        g^r = g^{k - nq}  = g^k (g^n)^{-q} = 1 \cdot 1 = 1.
      \end{equation*}
      Since $0 \leq r < n$, we must have that $r = 0$. Thus $n | k$.

    \item $(\implies)$ $g^k = g^m \implies g^{k - m} = 1 \overset{\text{by } 1}{\implies} n | ( k - m ) \iff k \equiv m \mod n$
    
      $(\impliedby)$ $k \equiv m \mod n \implies \exists q \in \mathbb{Z} \enspace k = qnm$. The result follows from 1.

    \item $(\supseteq)$ is clear by definition of $\lra{g} = \{g^k : k \in \mathbb{Z}\}$.

      To prove $(\subseteq)$, let $x = g^k \in \lra{g}$ for some $k \in \mathbb{Z}$. By the \hlnotea{Division Algorithm}, $k = nq + r$ for some $q, r \in \mathbb{Z}$ and $0 \leq r < n$. Then
      \begin{equation*}
        x = g^k = g^{nq + r} = g^{nq} g^r \overset{\text{by } 1}{=} g^r.
      \end{equation*}
      Since $0 \leq r < n$, we have that $x \in \{1, g, g^2, ..., g^{n - 1} \}$. Thus $\lra{g} = \{1, g, g^2, ..., g^{n - 1} \}$.

      It remains to show that all the elements in $\lra{g}$ are distinct. Suppose $g^k = g^m$ for some $k, m \in \mathbb{Z}$ with $0 \leq k, m < n$. By 2, we have that $k \equiv m \mod 2$. Therefore, $k = m$.

      We can also use 1 by the fact that $g^{k - m} = 1$ from assumption to complete the uniqueness proof.
  \end{enumerate} \qed
\end{proof}

\begin{propo}[Property of Elements of Infinite Order]
\label{propo:property_of_elements_of_infinite_order}
  Let $G$ be a group, and $g \in G$ such that $o(g) = \infty$. Then
  \begin{enumerate}
    \item $g^k = 1 \iff k = 0$;
    \item $g^k = g^r \iff k = m$;
    \item $\lra{g} = \{..., g^{-2}, g^{-1} 1, g, g^2, ...\}$ where each $g^i$ is distinct from others.
  \end{enumerate}
\end{propo}

\begin{proof}
  It suffices to prove 1, since 2 easily becomes true with 1, and 2 $\implies$ 3.

  \begin{enumerate}
    \item $(\impliedby) \; g^0 = 1$

      $(\implies)$ Suppose for contradiction that $g^k = 1$ for some $k \in \mathbb{Z} \; k \neq 0$. Then $g^{-k} = (g^k)^{-1} = 1$. Then we can assume that $k \geq 1$. This, however, implies that $o(g)$ is finite, which contradicts our assumption. Thus $k = 0$.

    \item \begin{equation*}
      g^k = g^m \iff g^{k - m} = 1 \overset{\text{by } 1}{\iff} k - m = 0 \iff k = m
    \end{equation*}
  \end{enumerate} \qed
\end{proof}

\begin{propo}[Orders of Powers of the Element]
\label{propo:orders_of_powers_of_the_element}
  Let $G$ be a group, and $g \in G$ with $o(g) = n \in \mathbb{N}$. We have that
  \begin{equation*}
    \forall d \in \mathbb{N} \enspace d \; | \; n \implies o(g^d) = \frac{n}{d}
  \end{equation*}
\end{propo}

\begin{proof}
  Let $k = \frac{n}{d}$. Note that $(g^d)^k = g^n = 1$. It remains to show that $k$ is the smallest such positive integer. Suppose $\exists r \in \mathbb{N} \enspace (g^d)^r = 1$. Since $o(g) = n$, then $n \; | \; dr$. Then $\exists q \in \mathbb{Z} \enspace dr = nq$ by definition of divisibility. $\because n = dk$ and $d \neq 0$, we have
  \begin{align*}
    dr = dkq \overset{d \neq 0}{\implies} r = kq \implies r > k \quad \because r, k \in \mathbb{N} \implies q \in \mathbb{N}
  \end{align*}\qed
\end{proof}

% subsection order_of_elements_continued (end)

\subsection{Cyclic Groups}%
\index{Cyclic Group}
\label{sub:cyclic_groups}
% subsection cyclic_groups

Recall the definition of a cyclic groups.

\begin{defn*}[Cyclic Groups]
  Let $G$ be a group and $g \in G$. Then we call $\lra{g}$ the \hlnoteb{cyclic subgroup} of $G$ generated by $g$. If $G = \lra{g}$ for some $g \in G$, then we say that $G$ is a \hlnoteb{cyclic group}, and $g$ is a \hldefn{generator} of $G$.
\end{defn*}

\begin{propo}[Cyclic Groups are Abelian]
\label{propo:cyclic_groups_are_abelian}
  All cyclic groups are abelian.
\end{propo}

\begin{proof}
  Note that a cyclic group $G$ is of the form $G = \lra{g}$. So
  \begin{gather*}
    \forall a, b \in G \enspace \exists m, n \in \mathbb{Z} \enspace a = g^m \, \land \, b = g^n \\
    a \cdot b = g^m g^n = g^{m + n} = g^{n + m} = g^n g^m = b \cdot a
  \end{gather*}\qed
\end{proof}

% subsection cyclic_groups (end)

% section subgroups_continued_3 (end)

% chapter lecture_7_may_16th_2018 (end)
