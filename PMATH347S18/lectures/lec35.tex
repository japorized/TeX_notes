\chapter{Lecture 35 Jul 25th 2018}%
\label{chp:lecture_35_jul_25th_2018}
% chapter lecture_35_jul_25th_2018

\section{Factorizations in Integral Domains (Continued 5)}%
\label{sec:factorizations_in_integral_domains_continued_5}
% section factorizations_in_integral_domains_continued_5

\subsection{Gauss' Lemma (Continued 2)}%
\label{sub:gauss_lemma_continued_2}
% subsection gauss_lemma_continued_2

We have shown in \cref{eg:polynomials_with_integer_coefficients_is_not_a_pid} that $\mathbb{Z}[x]$ is not a PID. Our goal now is the show that, in spite of that, $\mathbb{Z}[x]$ is a UFD.

\begin{note}
  Recall the following results from the recent lectures: Let $R$ be a UFD with $F$ being its field of fractions. We have
  \begin{itemize}
    \item $l(x) \in R[x]$ is irreducible $\implies \con(l) \sim 1$ (\cref{lemma:non_trivial_irreducible_polynomials_are_primitive});
    \item $\con(fg) \sim \con(f) \con(g)$ (\cref{lemma:role_of_the_content});
    \item $l(x)$ is irreducible in $R[x] \implies l(x)$ is irreducible in $F[x]$\\ (\cref{thm:reducibility_in_the_field_of_fractions}).
  \end{itemize}
\end{note}

\begin{note}
  Recall that the contrapositive of \cref{thm:reducibility_in_the_field_of_fractions} is: if $l(x)$ is reducible in $F[x]$, then $l(x)$ is reducible in $R[x]$.

  In other words, for $f(x) \in R[x]$, if $f(x) = g(x) h(x) \in F[x]$, then $\exists \tilde{g}(x), \, \tilde{h}(x) \in R[x]$ such that
  \begin{equation*}
    f(x) = \tilde{g}(x) \tilde{h}(x) \in R[x].
  \end{equation*}
\end{note}

\begin{eg}
  $2x^2 + 7x + 3 \in \mathbb{Z}[x]$, which we observe that
  \begin{align*}
    2x^2 + 7x + 3 &= \left(x + \frac{1}{2}\right)(2x + 6) \\
                  &= (2x + 1)(x + 3).
  \end{align*}
\end{eg}

We want to take advantage of the fact that $\mathbb{Q}[x]$ is a UFD to show that $\mathbb{Z}[x]$ is also a UFD.

Recall from \cref{eq:irreducibility_in_field_of_fractions_does_not_imply_irreducibility_in_the_original_ring} that $2x + 4 \in \mathbb{Q}[x]$ is irreducible, but is reducible in $\mathbb{Z}[x]$. Therefore, we have that the converse of \cref{thm:reducibility_in_the_field_of_fractions} is not true.

\begin{propo}
\label{propo:the_missing_condition_tying_irreducibility}
Let $R$ be a UFD with field of fractions $F$. TFAE:
\begin{enumerate}
  \item $f(x)$ is irreducible in $R[x]$;
  \item $f(x)$ is primitive and irreducible in $F[x]$.
\end{enumerate}
\end{propo}

\begin{proof}
  $(1) \implies (2)$ follows from \cref{lemma:non_trivial_irreducible_polynomials_are_primitive}, \cref{thm:gauss_lemma} and\\\noindent \cref{thm:reducibility_in_the_field_of_fractions}.

  \noindent $(2) \implies (1)$: Suppose that $f(x)$ is primitive and irreducibile in $F[x]$ but reducible in $R[x]$. Then a non-trivial factorization of $f(x) \in R[x]$ must take the form $f(x) = dg(x)$ with $d \in R$ and $d \not\sim 1$ \sidenote{Note that we cannot have both factors to have degree $\geq 1$, otherwise this would be a non-trivial factorization in $F[x]$, contradicting the irreducibility of $f(x)$ in $F[x]$.}. Since $d \mid f(x)$, $d \not\sim 1$ must then divide each of the coefficients of $f(x)$, which contradicts the assumption that $f(x)$ is primitive.\qed
\end{proof}

\begin{thm}[Polynomial Ring of a UFD is also a UFD]
\label{thm:polynomial_ring_of_a_ufd_is_also_a_ufd}
  If $R$ is a UFD, then the polynomial ring $R[x]$ is also a UFD.
\end{thm}

\begin{proof}
  By \cref{thm:integral_domain_that_satisfies_accp_has_a_polynomial_ring_that_satisfies_accp}, since $R$ is a UFD and hence satisfies ACCP \sidenote{See note on page \pageref{note:chain_of_definitions}.}, we have $R[x]$ also satisfies ACCP. Then by \cref{thm:ufd_and_accp}, to complete the proof, it suffices to show that every irreducible element $l(x) \in R[x]$ is prime. To show that an irreducible element $l(x) \in R[x]$ is prime, we need to show that if $l(x) \mid f(x) g(x)$ in $R[x]$, then $l(x) \mid f(x)$ or $l(x) \mid g(x)$.

  \textbf{Claim}: It suffices to show that
  \begin{equation*}
    l(x) \mid f_1(x) g_1(x) \implies l(x) \mid f_1(x) \, \lor \, l(x) \mid g_1(x)
  \end{equation*}
  where $f_1(x)$ and $g_1(x)$ are primitive, then given any non-primitive $f(x)$ and $g(x)$ such that $l(x) \mid f(x) g(x)$, we can reduce it to the primitive case, which then $l(x) \mid f(x)$ or $l(x) \mid g(x)$.

  Suppose $l(x) \mid f(x) g(x)$, which then $\exists h(x) \in R[x]$ such that $l(x) h(x) = f(x) g(x)$. Note that at this point, it is not necessary that $f(x)$ and $g(x)$ are primitive. Then by \cref{lemma:role_of_the_content}, we may write
  \begin{align*}
    f(x) &= \con(f) f_1(x) \\
    g(x) &= \con(g) g_1(x) \\
    h(x) &= \con(h) h_1(x)
  \end{align*}
  for some primitive polynomials $f_1(x), \, g_1(x)$ and $h_1(x)$ in $R[x]$. Since $l(x)$ is irreducible, by \cref{lemma:non_trivial_irreducible_polynomials_are_primitive}, we have $\con(l) \sim 1$. It thus follows that $\con(h) \sim \con(f) \con(g)$. Since
  \begin{equation*}
    \con(h) h_1(x) = \con(f) \con(g) f_1(x) g_1(x),
  \end{equation*}
  we have that
  \begin{equation*}
    h_1(x) l(x) \sim f_1(x) g_1(x).
  \end{equation*}
  Then we have that $l(x) \mid f_1(x) g_1(x)$, and so by the assumption, we have that $l(x) \mid f_1(x)$ or $l(x) \mid g_1(x)$, and so we have $l(x) \mid f(x)$ or $l(x) \mid g(x)$.

  We may now assume that $l(x) \mid f(x) g(x)$ where $f(x), \, g(x)$ are primitive in $R[x]$. Let $F$ denote the field of fractions of $R$, and consider $R \subseteq F$ is a subring of $F$. Then by extension, we have that $l(x) \mid f(x) g(x)$ in $F[x]$. Since $l(x)$ is irreducible in $R[x]$, we also have that $l(x)$ is irreducible in $F[x]$, by \cref{thm:reducibility_in_the_field_of_fractions}. Then by \cref{propo:euclid_s_lemma_for_polynomials}, since $F[x]$ is a field, we have $l(x) \mid f(x)$ or $l(x) \mid g(x)$.

  Suppose that $l(x) \mid f(x)$ in $F[x]$, say $\exists k(x) \in F[x]$ such that
  \begin{equation*}
    f(x) = l(x) k(x).
  \end{equation*}
  If $d \in R$ is the product of all denominators of the non-zero coefficients of $k(x)$, then $k_0(x) = dk(x) \in R[x]$, and so we have
  \begin{equation*}
    df(x) = dl(x) k(x) = l(x) k_0(x).
  \end{equation*}
  Since $f(x)$ is primitive and $l(x)$ is irreducible, by \cref{lemma:non_trivial_irreducible_polynomials_are_primitive} and \cref{thm:gauss_lemma}, we have
  \begin{equation}\label{eq:thm109_eq1}
    d \sim \con(df) \sim \con(lk_0) \sim \con(l) \con(k_0) \sim \con(k_0).
  \end{equation}
  Now if we write $k_0(x) = \con(k_0) k_1(x)$ using \cref{lemma:role_of_the_content}, for some primitive $k_1(x) \in R[x]$, then
  \begin{equation*}
    df(x) = l(x) k_0(x) = \con(k_0) l(x) k_1(x).
  \end{equation*}
  Then from \cref{eq:thm109_eq1}, we have
  \begin{equation*}
    f(x) \sim l(x) k_1(x).
  \end{equation*}
  Thus we have $l(x) \mid f(x)$ in $R[x]$. Similarly so, if $l(x) \mid g(x)$ in $F[x]$, we can show that $l(x) \mid g(x)$ in $R[x]$. It follows that $l(x)$ is therefore prime and so $R[x]$ is a UFD.\qed
\end{proof}

\newthought{Let $R$ be a UFD}, and $x_1, ..., x_n$ be $n$ commuting variables, i.e. $\forall i, \, j \in \{1, ..., n\}$ we have
\begin{equation*}
  x_i x_j = x_j x_i.
\end{equation*}
We may then inductively define the ring $R[x_1, ..., x_n]$ of polynomials in $n$ variables by
\begin{equation*}
  R[x_1, ..., x_n] = \left( R[x_1, ..., x_{n - 1}] \right)[x_n]
\end{equation*}
for $n \geq 1$. Then, as a direct corollary of \cref{thm:polynomial_ring_of_a_ufd_is_also_a_ufd}, we have:

\begin{crly}[Multiparametered Polynomial Ring of a UFD is also a UFD]
\label{crly:multiparametered_polynomial_ring_of_a_ufd_is_also_a_ufd}
If $R$ is a UFD, then $\forall n \in \mathbb{N}$, $R[x_1, ..., x_n]$ is also a UFD.
\end{crly}

Now since $\mathbb{Z}$ is a UFD, we have, therefore:

\begin{crly}[Polynomial Ring over Integers is a UFD]
\label{crly:polynomial_ring_over_integers_is_a_ufd}
  $\mathbb{Z}[x]$ and $\mathbb{Z}[x_1, ..., x_n]$ are UFDs.
\end{crly}

Another application of \hyperref[thm:gauss_lemma]{Gauss' Lemma} is:

\begin{thm}[Eisenstein's Criterion of {$\mathbb{Z}[x]$}]
\index{Eisenstein's Criterion of $\mathbb{Z}[x]$}
\label{thm:eisenstein_s_criterion_of_z_x_}
Let $f(x) = a_n x^n + \hdots + a_0 \in \mathbb{Z}[x]$ and $p$ a prime. Suppose that
\begin{equation*}
  p \nmid a_n, \quad p \nmid a_i \text{ for } 0 \leq i \leq n - 1 \quad \text{ and } \quad p^2 \nmid a_o.
\end{equation*}
Then $f(x)$ is irreducible in $\mathbb{Q}[x]$. In particular, if $f(x)$ is primitive, then $f(x)$ is irreducible in $\mathbb{Z}[x]$.\sidenote{e.g. $f(x)$ is monic $\implies f(x)$ is primitive.}
\end{thm}

\begin{proof}
  \hlimpo{Take PMATH348!!}\sidenote{And so we have a teaser right at the end!!}
\end{proof}

% subsection gauss_lemma_continued_2 (end)

% section factorizations_in_integral_domains_continued_5 (end)

% chapter lecture_35_jul_25th_2018 (end)
