\chapter{Lecture 24 Jun 27th 2018}%
\label{chp:lecture_24_jun_27th_2018}
% chapter lecture_24_jun_27th_2018

\section{Rings (Continued 4)}%
\label{sec:rings_continued_4}
% section rings_continued_4

\subsection{Isomorphism Theorems for Rings (Continued)}%
\label{sub:isomorphism_theorems_for_rings_continued}
% subsection isomorphism_theorems_for_rings_continued

\begin{thm}[Chinese Remainder Theorem]
\index{Chinese Remainder Theorem}
\label{thm:chinese_remainder_theorem}
  Let $A$ and $B$ be ideals of $R$.
  \begin{enumerate}
    \item $A + B = R \implies \faktor{R}{(A \cap B)} \cong \faktor{R}{A} \times \faktor{R}{B}$
    \item $A + B = R \, \land \, A \cap B = \{0\} \implies R \cong \faktor{R}{A} \times \faktor{R}{B}$
  \end{enumerate}
\end{thm}

\begin{proof}
  It suffices to prove $(1)$ since if $(1)$ is true and $A \cap B = \{0\}$, then $(2)$ immediately follows.

  Define
  \begin{equation*}
    \Theta: R \to \faktor{R}{A} \times \faktor{R}{B} \qquad r \mapsto ( r + A, r + B )
  \end{equation*}
  Then $\Theta$ is a ring homomorphism \sidenote{
  \begin{ex}
    Prove that $\Theta$ is a ring homomorphism.
  \end{ex}
  }.

  \begin{proof}[$\Theta$ is a ring homomorphism]
    $\forall r, s \in R$, we have
    \begin{align*}
      \Theta(rs) &= (rs + A, rs + B) \\
                 &\overset{(*)}{=} (r + A, r + B)(s + A, s + B) \\
                 &= \Theta(r) \Theta(s)
    \end{align*}
    where $(*)$ is by \cref{propo:construction_of_the_quotient_ring}. Also by the same proposition, we have
    \begin{equation*}
      \Theta(1) = (1 + A, 1 + B).
    \end{equation*}
    Then,
    \begin{align*}
      \Theta(r + s) &= (r + s + A, r + s + B) \\
                    &\overset{(\dagger)}{=} (r + A, r + B) + (s + A, s + B) \\
                    &= \Theta(r) + \Theta(s)
    \end{align*}
    where $(\dagger)$ is by \cref{propo:properties_of_the_additive_quotient_group}.
  \end{proof}
  Note that $\ker \Theta = A \cap B$, since
  \begin{equation*}
    \ker \Theta = \{ r \in R : \Theta(r) = (A, B) \} = \{r \in A \, \land \, r \in B \} = A \cap B.
  \end{equation*}
  To show that $\Theta$ is surjective, let $(s + A, t + B) \in \faktor{R}{A} \times \faktor{R}{B}$ with $s, \, t \in R$. Since $A + B = R$, $\exists a \in A, \, \exists b \in B$ such that $a + b = 1$. Let $r = sb + ta$. Then
  \begin{align*}
    s - r &= s - sb - ta = s(1 - b) - ta = sa - ta = (s - t) a \in A \\
    t - r &= t - sb - ta = t(1 - a) - sb = tb - sb = (t - s) b \in B
  \end{align*}
  and so by \cref{propo:properties_of_the_additive_quotient_group},
  \begin{equation*}
    s + A = r + A \text{ and } t + B = r + B.
  \end{equation*}
  Therefore
  \begin{equation*}
    \Theta(r) = (r + A, r + B) = (s + A, t + B),
  \end{equation*}
  and so $\Theta$ is surjecive. Then by the \cref{thm:first_isomorphism_theorem_for_rings},
  \begin{equation*}
    \faktor{R}{(A \cap B)} \cong \faktor{R}{A} \times \faktor{R}{B}.
  \end{equation*}\qed
\end{proof}

\newthought{Why is} \cref{thm:chinese_remainder_theorem} called the Chinese Remainder Theorem?

Let $m, n \in \mathbb{N}$ with $\gcd(m, n) = 1$. Then we know that
\begin{equation*}
  m \mathbb{Z} \cap n \mathbb{Z} = mn \mathbb{Z}.
\end{equation*}
Also, $m \mathbb{Z} + n \mathbb{Z} = \mathbb{Z}$ since $1 = ma + nb$ for some $a, b \in \mathbb{Z}$ by \hlnotea{Bezout's Lemma}. And so:

\begin{crly}
\label{crly:why_crt}
  \begin{enumerate}
    \item If $m, n \in \mathbb{N}$ with $\gcd(m, n) = 1$, then
      \begin{equation*}
        \faktor{\mathbb{Z}}{mn \mathbb{Z}} \cong \faktor{\mathbb{Z}}{m \mathbb{Z}} \times \faktor{\mathbb{Z}}{n \mathbb{Z}}
      \end{equation*}
      i.e.
      \begin{equation*}
        \mathbb{Z}_{mn} \cong \mathbb{Z}_m \times \mathbb{Z}_n
      \end{equation*}

    \item If $m, n \in \mathbb{N}$ with $m, n \geq 2$ and $\gcd(m, n) = 1$, then
      \begin{equation*}
        \phi(mn) = \phi(m)\phi(n)
      \end{equation*}
      where $\phi(m) = \abs{\mathbb{Z}_m^*}$ is \hldefn{Euler's $\phi$-function}\index{Euler's Totient Function}.
  \end{enumerate}
\end{crly}

\newthought{Let} $p$ be a prime. Recall that one consequence of \hyperref[thm:lagrange_s_theorem]{Lagrange's Theorem} is that every group $G$ of order $p$ is cyclic, i.e. $G \cong C_p$.

An analogous notion in rings is the following:

\begin{propo}[Ring With Prime Order Is Isomorphic to Integer Modulo Prime]
\label{propo:ring_with_prime_order_is_isomorphic_to_integer_modulo_prime}
  If $R$ is a non-trivial ring with $\abs{R} = p$ where $p$ is prime, then $R \cong \mathbb{Z}_p$.
\end{propo}

\begin{proof}
  Define
  \begin{equation*}
    \Theta : \mathbb{Z}_p \to R \qquad [k] \mapsto k \cdot 1_R.
  \end{equation*}
  Note that since $R$ is an additive group with $\abs{R} = p$, by \hyperref[thm:lagrange_s_theorem]{Lagrange's Theorem}, $o(1_R) = 1$ or $p$. Since $R$ is non-trivial, we have that $1_R \neq 0$ by the remark on the definition of a \hyperref[defn:trivial_ring]{trivial ring}, and so $o(1_R) \neq 1$. Thus $o(1_R) = p$. Then, by \cref{propo:implications_of_the_characteristic}, we have
  \begin{equation*}
    [k] = [m] \iff p \, | \, (k - m) \iff (k - m) 1_R = 0 \iff k \cdot 1_R = m \cdot 1_R
  \end{equation*}
  in $R$. Thus, $\Theta$ is well-defined and injective. $\Theta$ is also a ring homomorphism \sidenote{
  \begin{ex}
    Prove that $\Theta$ is a ring homomorphism.
  \end{ex}
  }.

  \begin{proof}[$\Theta$ is a ring homomorphism]
    $\forall [a], [b] \in \mathbb{Z}$, we have
    \begin{align*}
      \Theta([a][b]) &= \Theta( [ab] ) = ab \cdot 1_R \\
                     &= ( a \cdot 1_R )( b \cdot 1_R ) = \Theta([a])\Theta([b]).
    \end{align*}
    \begin{equation*}
      \Theta([1]) = 1 \cdot 1_R = 1_R
    \end{equation*}
    and
    \begin{align*}
      \Theta([a] + [b]) &= \Theta([ a + b ]) = (a + b) \cdot 1_R \\
                        &= a \cdot 1_R + b \cdot 1_R = \Theta([a]) + \Theta([b]).
    \end{align*}
    So $\Theta$ is a ring homomorphism.
  \end{proof}
  Now because $\abs{\mathbb{Z}_p} = p = \abs{R}$ and $\Theta$ is injective, $\Theta$ must be surjective. Therefore $\Theta$ is a ring isomorphism and hence $R \cong \mathbb{Z}_p$ as required. \qed
\end{proof}

% subsection isomorphism_theorems_for_rings_continued (end)

% section rings_continued_4 (end)

\section{Commutative Rings}%
\label{sec:commutative_rings}
% section commutative_rings

\subsection{Integral Domain and Fields}%
\label{sub:integral_domain_and_fields}
% subsection integral_domain_and_fields

\begin{defn}[Units]\index{Units}
\label{defn:units_and_group_of_units}
  Let $R$ be a ring. We say that $u \in R$ is a \hlnoteb{unit} if $u$ has a multiplicative inverse in $R$, and denote it by $u^{-1}$. We have
  \begin{equation*}
    uu^{-1} = 1 = u^{-1} u
  \end{equation*}
\end{defn}

\begin{note}
  If $u$ is a unit in $R$, and $r, s \in R$, we have
  \begin{align*}
    ur = us &\implies r = s \quad \text{\hlnotea{( Right Cancellation )}} \\
    ru = su &\implies r = s \quad \text{\hlnotea{( Left Cancellation )}}
  \end{align*}
  Let $R^*$ denote the set of all units in $R$. We know that the definition of a ring is that $R$ is ``almost'' a group under multiplication except that its elements do not necessarily have multiplicative inverses. Since $R^* \subseteq R$ is the set that contains all units, i.e. all elements with multiplicative inverses in $R$, we have that $(R^*, \cdot)$ is a group. This is called the \hldefn{Group of Units} of $R$.
\end{note}

\begin{eg}
  Note that $2$ is a unit in $\mathbb{Q}$, but it is not a unit in $\mathbb{Z}$. We have that
  \begin{equation*}
    \mathbb{Q}^* = \mathbb{Q} \setminus \{0\} \text{ and } \mathbb{Z}^* = \{ \pm 1 \}
  \end{equation*}
\end{eg}

\begin{eg}
  Consider the ring of \hldefn{Gaussian Integers},
  \begin{equation*}
    \mathbb{Z}[i] = \{a + bi : a, b \in \mathbb{Z}, \, i^2 = -1 \} \leq \mathbb{C}.
  \end{equation*}
  Then
  \begin{equation*}
    \mathbb{Z}[i]^* = \{ \pm 1, \pm i \} \leq \mathbb{C}.
  \end{equation*}

  \begin{proof}
    \textbf{$\mathbb{Z}[i] \leq \mathbb{C}$} : \\
    Note that $1 = 1 + 0 \cdot i \in \mathbb{Z}[i]$. $\forall x, y \in \mathbb{Z}[i]$, write
    \begin{equation*}
      x = a + bi \quad y = c + di
    \end{equation*}
    for some $a, b, c, d \in \mathbb{Z}$. Observe that
    \begin{equation}\label{eq:q5_a_1}
      xy = (a + bi)(c + di) = (ac - bd) + i (bc + ad) \in \mathbb{Z}[i]
    \end{equation}
    since $(ab - bd), (bc + ad) \in \mathbb{Z}$. Also, with a similar reason
    \begin{equation*}
      x - y = a + bi - c - di = (a - c) + i (b - d) \in \mathbb{Z}[i].
    \end{equation*}
    Therefore, by the Subring Test, $\mathbb{Z}[i]$ is a subring of $\mathbb{C}$.

    \textbf{$\mathbb{Z}[i]^* = \left\{ \pm 1, \, \pm i \right\}$} : \\
    In order for $x \in \mathbb{Z}[i]^*$, $x \neq 0$, to have an inverse, we must have that
      \begin{equation*}
        (ac - bd) + i ( bc + ad ) = 1
      \end{equation*}
      from \cref{eq:q5_a_1}, where we shall note that $a^2 + b^2 \neq 0$. Observe that
      \begin{align*}
        &( ac - bd ) + i ( bc + ad ) = 1 \\
        &\iff \begin{pmatrix}
            ac - bd \\ bc + ad 
        \end{pmatrix} = \begin{pmatrix}
              1 \\ 0
        \end{pmatrix} \\
        &\iff \begin{pmatrix}
            a & -b \\ b & a
        \end{pmatrix} \begin{pmatrix}
          c \\ d
        \end{pmatrix} = \begin{pmatrix}
          1 \\ 0
        \end{pmatrix} \\
        &\iff \begin{pmatrix}
          c \\ d
        \end{pmatrix} = \frac{1}{a^2 + b^2} \begin{pmatrix}
          1 \\ 0
        \end{pmatrix} \begin{pmatrix}
          a & b \\ -b & a
        \end{pmatrix} = \frac{1}{a^2 + b^2} \begin{pmatrix}
          a \\ -b
        \end{pmatrix} \\
        &\iff c + id = \frac{a}{a^2 + b^2} + i \left( \frac{-b}{a^2 + b^2} \right)
      \end{align*}
      Since $c, d \in \mathbb{Z}$, we must have that
      \begin{equation*}
        \frac{a}{a^2 + b^2}, \, \frac{-b}{a^2 + b^2} \in \mathbb{Z}.
      \end{equation*}
      Also, since $a^2 + b^2 > a, \, b$, we have that
      \begin{equation*}
        \frac{a}{a^2 + b^2}, \, \frac{-b}{a^2 + b^2} \in \mathbb{Z} \iff a^2 + b^2 = 1.
      \end{equation*}
      Since $a, \, b \in \mathbb{Z}$, we have that the only integer solutions to $a^2 + b^2 = 1$ are
      \begin{equation*}
        (1, \, 0) , \, (0, \, 1) , \, ( -1, \, 0 ) , \, (0, \, -1),
      \end{equation*}
      which corresponds to
      \begin{equation*}
        x = \pm 1, \, \pm i.
      \end{equation*}
      Since we began with $x$ begin an arbitrary element in $\mathbb{Z}[i]^*$, we have that
      \begin{equation*}
        \mathbb{Z}[i]^* = \{ \pm 1, \, \pm i \}
      \end{equation*}
      as required.
    \qed
  \end{proof}
\end{eg}

\begin{defn}[Division Ring and Field]\index{Division Ring}\index{Field}
\label{defn:division_ring_and_field}
  A non-trivial ring $R$ is a \hlnoteb{division ring} if
  \begin{equation*}
    R^* = R \setminus \{0\}.
  \end{equation*}
  A commutative division ring is a \hlnoteb{field}.
\end{defn}

\begin{eg}
  $\mathbb{Q}, \, \mathbb{R}, \, \mathbb{C}$ are fields but $\mathbb{Z}$ is not.
\end{eg}

\begin{eg}
  $\mathbb{Z}_n$ is a field $\iff \, n$ is prime.
\end{eg}

\begin{remark}
\marginnote{This remark is not as useful or spectacular within this course, but it will be once we go into PMATH348 contents.} If $R$ is a division ring or a field, then its only ideals are $\{0\}$ or $R$, since if $A \neq \{0\}$ is an ideal of $R$, then $\exists a \in A$, $a \neq 0$, such that $1 = aa^{-1} \in A$, which implies that $A = R$ by \cref{propo:the_only_ideal_with_the_multiplicative_identity_is_the_ring_itself}.
\end{remark}

\begin{remark}
  It can be shown that every finite division ring is a field, and this is known as \href{https://en.wikipedia.org/wiki/Wedderburn_Theorem}{Wedderburn's Theorem}.
\end{remark}

\newthought{Note that} if $n = ab$ for some integer $n$ with $0 < a, b < n$, then in $\mathbb{Z}$ we have
\begin{equation*}
  [a][b] = [n] = [0]
\end{equation*}
but $[a] \neq [0] \neq [b]$ by our definition of $a, b$.

\begin{defn}[Zero Divisor]\index{Zero Divisor}
\label{defn:zero_divisor}
  Let $R$ be a non-trivial ring. If $0 \neq a \in R$, then $a$ is called a \hlnoteb{zero divisor} if $\exists 0 \neq b \in R$ such that $ab = 0$.
\end{defn}

% subsection integral_domain_and_fields (end)

% section commutative_rings (end)

% chapter lecture_24_jun_27th_2018 (end)
