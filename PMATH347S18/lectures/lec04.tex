\chapter{Lecture 4 May 09 2018}
  \label{chapter:lecture_4_may_09_2018}

\section{Groups (Continued)} % (fold)
\label{sec:groups_continued}

\subsection{Groups (Continued)} % (fold)
\label{sub:groups_continued}

\begin{propo}[Cancellation Laws]\label{propo:cancellation_laws}
  Let $G$ be a group and $g, h, f \in G$. Then
  \begin{enumerate}
    \item \begin{enumerate}
        \item (\hlnoteb{Right Cancellation}) $gh = gf \implies h = f$
        \item (\hlnoteb{Left Cancellation}) $hg = fg \implies h = f$
      \end{enumerate}
    \item The equation $ax = b$ and $ya = b$ have unique solution for $x, y \in G$.
  \end{enumerate}
\end{propo}

\begin{proof}
  \begin{enumerate}
    \item \begin{enumerate}
      \item By left multiplication and associativity,
        \begin{equation*}
          gh = gf \iff g^{-1} gh = g^{-1} gf \iff h = f
        \end{equation*}
      \item By right multiplication and associativity,
        \begin{equation*}
          hg = fg \iff hgg^{-1} = fgg^{-1} \iff h = f
        \end{equation*}
    \end{enumerate}

    \item Let $x = a^{-1} b$. Then
      \begin{equation*}
        a x = a (a^{-1} b) = (aa^{-1}) b = b.
      \end{equation*}
      If $\exists u \in G$ that is another solution, then
      \begin{equation*}
        au = b = ax \implies u = x
      \end{equation*}
      by Left Cancellation. The proof for $ya = b$ is similar by letting $y = ba^{-1}$.
  \end{enumerate}\qed
\end{proof}

% subsection groups_continued (end)

\subsection{Cayley Tables} % (fold)
\label{sub:cayley_tables}

For a finite group, defining its operation by means of a table is sometimes convenient.

\begin{defn}[Cayley Table]\label{defn:cayley_table}
\index{Cayley Table}
  Let $G$ be a group. Given $x, y \in G$, let the product $xy$ be an entry of a table in the row corresponding to $x$ and column corresponding to $y$. Such a table is called a \hlnoteb{Cayley Table}.
\end{defn}

\begin{note}
  By \autoref{propo:cancellation_laws}, the entries in each row (and respectively, column) of a Cayley Table are all distinct.
\end{note}

\begin{eg}
  Consider the group $(\mathbb{Z}_2, +)$. Its Cayley Table is
  \begin{center}
    \begin{tabular}{c|c|c}
      $\mathbb{Z}_2$ & $[0]$ & $[1]$ \\
      \hline
      $[0]$     & $[0]$ & $[1]$ \\
      $[1]$     & $[1]$ & $[0]$ 
    \end{tabular}
  \end{center}
  where note that we must have $[1] + [1] = [0]$; otherwise if $[1] + [1] = [1]$ then $[1]$ does not have its additive inverse, which contradicts the fact that it is in the group.
\end{eg}

\marginnote {
  If we replace $1$ by $[0]$ and $-1$ by $[1]$, the Cayley Tables of $\mathbb{Z}_2$ and $\mathbb{Z}^*$ are the same. In thie case, we say that $\mathbb{Z}_2$ and $\mathbb{Z}^*$ are \hlnotea{isomorphic}, which we denote by $\mathbb{Z}_2 \cong \mathbb{Z}^*$.
}

\begin{eg}
  Consider the group $\mathbb{Z}^* = \{1. -1\}$. Its Cayley Table (under multiplication) is
  \begin{center}
    \begin{tabular}{c|c|c}
      $\mathbb{Z}^*$ & $1$    & $-1$ \\
      \hline
      $1$              & $1$  & $-1$ \\
      $-1$             & $-1$ & $1$
    \end{tabular}
  \end{center}
\end{eg}

\begin{eg}\label{eg:cyclic_group_cayley_table}
  Given $n \in \mathbb{N}$, the \hldefn{Cyclic Group} of order $n$ is defined by
  \begin{equation*}
    C_n = \{1, a, a^2, ..., a^{n - 1}\} \quad \text{with } a^n = 1.
  \end{equation*}
  We write $C_n = \langle a : a^n = 1 \rangle$ and $a$ is called a generator of $C_n$. The Cayley Table of $C_n$ is
  \begin{center}
    \begin{tabular}{c | c c c c c c}
      $C_n$     & $1$       & $a$       & $a^2$  & $\hdots$ & $a^{n - 2}$ & $a^{n - 1}$ \\
      \hline
      $1$       & $1$       & $a$       & $a^2$  & $\hdots$ & $a^{n - 2}$ & $a^{n - 1}$ \\
      $a$       & $a$       & $a^2$     & $a^3$  & $\hdots$ & $a^{n - 1}$ & $1$ \\
      $a^2$     & $a^2$     & $a^3$     & $a^4$  & $\hdots$ & $1$         & $a$ \\
      \vdots    & \vdots    & \vdots    & \vdots &          & \vdots      & \vdots \\
      $a^{n-2}$ & $a^{n-2}$ & $a^{n-1}$ & $1$    & $\hdots$ & $a^{n-4}$   & $a^{n-3}$ \\
      $a^{n-1}$ & $a^{n-1}$ & $1$       & $a$    & $\hdots$ & $a^{n-3}$   & $a^{n-2}$
    \end{tabular}
  \end{center}
\end{eg}

\begin{propo}\label{propo:small_groups}
  Let $G$ be a group. Up to isomorphism, we have
  \begin{enumerate}
    \item if $\abs{G} = 1$, then $G \cong \{1\}$.
    \item if $\abs{G} = 2$, then $G \cong C_2$.
    \item if $\abs{G} = 3$, then $G \cong C_3$.
    \item if $\abs{G} = 4$, then either $G \cong C_4$ or $G \cong K_4 \cong C_2 \times C_2$ \marginnote{$K_n$ is known as the \hldefn{Klein n-group}}.
  \end{enumerate}
\end{propo}

\begin{proof}
  \begin{enumerate}
    \item If $\abs{G} = 1$, then it can only be $G = \{1\}$ where $1$ is the identity element.
    \item $\abs{G} = 2 \implies G = \{1, g\}$ with $g \neq 1$. The Cayley Table of $G$ is thus
      \begin{center}
        \begin{tabular}{c | c c}
        $G$ & $1$ & $g$ \\
        \hline
        $1$ & $1$ & $g$ \\
        $g$ & $g$ & $1$
        \end{tabular}
      \end{center}
      where we note that $g^2 = 1$; otherwise if $g^2 = g$, then we would have $g = 1$ by \autoref{propo:cancellation_laws}, which contradicts the fact that $g \neq 1$. Comparing the above Cayley Table with that of $C_2$, we see that $G = \langle g : g^2 = 1 \rangle \cong C_2$.
    \item $\abs{G} = 3 \implies G = \{1, g, h\}$ with $g \neq 1 \neq h$ and $g \neq h$. We can then start with the following Cayley Table:
      \begin{center}
        \begin{tabular}{c | c c c}
        $G$ & $1$ & $g$ & $h$ \\
        \hline
        $1$ & $1$ & $g$ & $h$ \\
        $g$ & $g$ &     &     \\
        $h$ & $h$ &     &     
        \end{tabular}
      \end{center}
      We know that by \autoref{propo:cancellation_laws}, $gh \neq g$ and $gh \neq h$. Thus $gh = 1$. Similarly, we get that $hg = 1$.

      \underline{Claim:} Entries in a row (or column) must be distinct. Suppose not. Then say $g^2 = 1$. But since $gh = 1$, by \autoref{propo:cancellation_laws}, we have that $h = g$, which is a contradiction.

      With that, we can proceed to fill in the rest of the entries: with $g^2 = h$ and $h^2 = g$. Therefore,
      \begin{center}
        \begin{tabular}{c | c c c}
        $G$ & $1$ & $g$ & $h$ \\
        \hline
        $1$ & $1$ & $g$ & $h$ \\
        $g$ & $g$ & $h$ & $1$ \\
        $h$ & $h$ & $1$ & $g$
        \end{tabular}
      \end{center}

      Recall that the Cayley Table for $C_3$ is:
      \begin{center}
        \begin{tabular}{c | c c c}
        $C_3$ & $1$   & $a$   & $a^2$ \\
        \hline
        $1$   & $1$   & $a$   & $a^2$ \\
        $a$   & $a$   & $a^2$ & $1$ \\
        $a^2$ & $a^2$ & $1$   & $a$
        \end{tabular}
      \end{center}
      $\therefore G \cong C_3$ (by identifying $g = a$ and $h = a^2$).

    \item Suppose $G$ is a group of order $4$, i.e. $\abs{G} = 4$. Then, let $G = \{1, g h, f\}$, where $1, g, h, f$ are distinct. We can then draw the following Cayley Table, wherein the blank entries will be discussed.

          \begin{center}
            \begin{tabular}{c|c|c|c|c}
              $G$   & 1 & g & h & f \\
              \hline
              1     & 1 & g & h & f \\
              g     & g &   &   &   \\
              h     & h &   &   &   \\
              f     & f &   &   &  
            \end{tabular}
          \end{center}

          We know that $gh \neq h$, otherwise by Right Cancellation, we would have $g = 1$, which is not true since $1$ and $g$ are distinct elements of $G$. Thus, $gh$ is either $f$ or $1$.

          \underline{Case 1: $g^2 = 1$} \\
          If $g^2 = 1$, then $fg = h$. Otherwise, if $fg = f$, we would have $g = 1$ which is a contradiction to the fact that $1$ and $g$ are distinct. Consequently, $hg = f$. Similarly, since $gf = f$ would contradict the fact that $g \neq 1$ through Right Cancellation, we have $gh = f$ and $gf = h$. We now have the following form of the Cayley Table:
          \begin{center}
            \begin{tabular}{c|c|c|c|c}
              $G$   & 1 & g & h & f \\
              \hline
              1     & 1 & g & h & f \\
              g     & g & 1 & f & h \\
              h     & h & f &   &   \\
              f     & f & h &   &  
            \end{tabular}
          \end{center}

          Now there are 2 options, either $h^2 = 1$ or $h^2 = g$.

          \underline{Case 1-1: $h^2 = 1$} \\
          If $h^2 = 1$, then through elimination, $hf = g$, $fh = g$ and $f^2 = 1$. We then have the following Cayley Table:
          
          \begin{center}
            \begin{tabular}{c|c|c|c|c}
              $G$   & 1 & g & h & f \\
              \hline
              1     & 1 & g & h & f \\
              g     & g & 1 & f & h \\
              h     & h & f & 1 & g \\
              f     & f & h & g & 1
            \end{tabular}
          \end{center}

          This is clearly the Cayley Table of the Klein $4$ group.

          \underline{Case 1-2: $h^2 = g$} \\
          If $h^2 = g$, then through elimination, $hf = 1$, $fh = 1$ and $f^2 = g$. We then have the following Cayley Table:

          \begin{center}
            \begin{tabular}{c|c|c|c|c}
              $G$   & 1 & g & h & f \\
              \hline
              1     & 1 & g & h & f \\
              g     & g & 1 & f & h \\
              h     & h & f & g & 1 \\
              f     & f & h & 1 & g
            \end{tabular}
          \end{center}

          We can rearrange the elements and hence the Cayley Table to the following:

          \begin{center}
            \begin{tabular}{c|c|c|c|c}
              $G$   & 1 & f & g & h \\
              \hline
              1     & 1 & f & g & h \\
              f     & f & g & h & 1 \\
              g     & g & h & 1 & f \\
              h     & h & 1 & f & g
            \end{tabular}
          \end{center}

          which is the Cayley Table of $C_4$.

          Now note that the following case will cover for 2 cases, i.e. $g^2 = h$ and $g^2 = f$, since we can proceed with the argument without loss of generality.

          \underline{Case 2: $g^2 = f$} \\
          If $g^2 = f$, we have that $hg = 1$, since we can only have distinct elements in a column and in a row. Consequently, we have $fg = h$. Similarly, we must have that $gh = 1$ and consequently $gf = h$. Thus we have the following Cayley Table:

          \begin{center}
            \begin{tabular}{c|c|c|c|c}
              $G$   & 1 & g & h & f \\
              \hline
              1     & 1 & g & h & f \\
              g     & g & f & 1 & h \\
              h     & h & 1 &   &   \\
              f     & f & h &   &  
            \end{tabular}
          \end{center}

          Note that $h^2 \neq g$, because we would then have $fh = f$, which would imply $h = 1$ through Left Cancellation, a contradiction to the fact that $h \neq 1$. Thus $h^2 = g$. Again, since we can only have distinct elements in a row (and a column), we will end up with the following Cayley Table:

          \begin{center}
            \begin{tabular}{c|c|c|c|c}
              $G$   & 1 & g & h & f \\
              \hline
              1     & 1 & g & h & f \\
              g     & g & f & 1 & h \\
              h     & h & 1 & f & g \\
              f     & f & h & g & 1
            \end{tabular}
          \end{center}

          We can rearrange the elements to get the following Cayley Table:

          \begin{center}
            \begin{tabular}{c|c|c|c|c}
              $G$   & 1 & h & f & g \\
              \hline
              1     & 1 & h & f & g \\
              h     & h & f & g & 1 \\
              f     & f & g & 1 & h \\
              g     & g & 1 & h & f
            \end{tabular}
          \end{center}

          in which we observe is the Cayley Table for $C_4$.

          Since we have explored all the possibilities, we have that the only possible groups of order $4$ is the cyclic group $C_4$ and the Klein $4$ group $K_4$.
  \end{enumerate}\qed
\end{proof}

% subsection cayley_tables (end)

% section groups_continued (end)

\section{Subgroups}
\label{sec:subgroups}

\subsection{Subgroups}
\label{sub:subgroups}

\begin{defn}[Subgroup]\label{defn:subgroup}
\index{Subgroup}
  Let $G$ be a group and $H \subseteq G$. If $H$ itself is a group, then we say that $H$ is a subgroup of $G$
\end{defn}

% subsection subgroups (end)

% section subgroups (end)

% chapter lecture_4_may_09_2018 (end)
