\chapter{Lecture 3 May 07th 2018}
  \label{chapter:lecture_3_may_07th_2018}

\section{Groups} % (fold)
\label{sec:groups}

\subsection{Groups} % (fold)
\label{sub:groups}

\begin{defn}[Groups]\label{defn:groups}
\index{Groups}
  Let $G$ be a set and $*$ an operation on $G \times G$. We say that $G = (G, *)$ is a \hlnoteb{group} if it satisfies\sidenote{If you wonder why the uniqueness is not specified for \hlnoteb{Identity} and \hlnoteb{Inverse}, see \cref{propo:uniqueness_of_group_identity_and_group_element_inverse}.}
  \begin{enumerate}
    \item \hlnoteb{Closure}: $\forall a, b \in G \quad a * b \in G$
    \item \hlnoteb{Associativity}: $\forall a, b, c \in G \quad a * (b * c) = (a * b) * c$
    \item \hlnoteb{Identity}: $\exists e \in G \enspace \forall a \in G \quad a * e = a = e * a$
    \item \hlnoteb{Inverse}: $\forall a \in G \enspace \exists b \in G \quad a * b = e = b * a$
  \end{enumerate}
\end{defn}

\begin{defn}[Abelian Group]\label{defn:abelian_group}
\index{Abelian Group}
  A group $G$ is said to be abelian if $\forall a, b \in G$, we have $a * b = b * a$.
\end{defn}

\begin{propo}[Group Identity and Group Element Inverse]\label{propo:uniqueness_of_group_identity_and_group_element_inverse}
  Let $G$ be a group and $a \in G$.
  \begin{enumerate}
    \item The identity of $G$ is unique.
    \item The inverse of $a$ is unique.
  \end{enumerate}
\end{propo}

\begin{proof}
  \begin{enumerate}
    \item If $e_1, e_2 \in G$ are both identities of $G$, then we have
      \begin{equation*}
        e_1 \overset{(1)}{=} e_1 * e_2 \overset{(2)}{=} e_2
      \end{equation*}
      where $(1)$ is because $e_2$ is an identity and $(2)$ is because $e_1$ is an identity.

    \item Let $a \in G$. If $b_1, b_2 \in G$ are both the inverses of $a$, then we have
      \begin{equation*}
        b_1 = b_1 * e = b_1 * (a * b_2) \overset{(1)}{=} e * b_2 = b_2
      \end{equation*}
      where $(1)$ is by associativity.
  \end{enumerate}
\end{proof}

\begin{eg}
  The sets $(\mathbb{Z}, +), \, (\mathbb{Q}, +), \, (\mathbb{R}, +)$, and $(\mathbb{C}, +)$ are all abelian, wehre the additive identity is $0$, and the additive inverse of an element $r$ is $(-r)$.
\end{eg}

\begin{note}
  $(\mathbb{N}, +)$ is not a group for neither does it have an identity nor an inverse for any of its elements.
\end{note}

\begin{eg}
  The sets $(\mathbb{Q}, \cdot), \, (\mathbb{R}, \cdot)$ and $(\mathbb{C}, \cdot)$ are \hlwarn{not} groups, since $0$ has no multiplicative inverse in $\mathbb{Q}, \mathbb{R}$ or $\mathbb{C}$.
\end{eg}

We may define that for a set $S$, let $S^* \subseteq S$ contain all the elements of $S$ that has a multiplicative inverse. For example, $\mathbb{Q}^* = \mathbb{Q} \setminus \{0\}$. Then, $(\mathbb{Q}, \cdot), (\mathbb{R}, \cdot)$ and $(\mathbb{C}, \cdot)$ are groups and are in fact abelian, where the multiplicative identity is $1$ and the multiplicative of an element $r$ is $\frac{1}{r}$.

\begin{eg}
  The set $\big( M_n(\mathbb{R}), + \big)$ is an abelian group, where the additive identity is the zero matrix, $0 \in M_n(\mathbb{R})$, and the additive inverse of an element $M = [a_{ij}] \in M_n(\mathbb{R})$ is $-M = [-a_{ij}] \in M_n(\mathbb{R})$.
\end{eg}

\newthought{Consider} the set $M_n(\mathbb{R})$ under the matrix mutiplication operation that we have introduced in \nameref{chapter:lecture_1_may_02nd_2018}. We found that the identity matrix is
\begin{equation*}
  I = \begin{bmatrix}
    1 & 0 & \hdots & 0 \\
    0 & 1 & \hdots & 0 \\
    \vdots & \vdots & & \vdots \\
    0 & 0 & \hdots & 1
  \end{bmatrix} \in M_n(\mathbb{R}).
\end{equation*}
But since not all elements of $M_n(\mathbb{R})$ have a multiplicative inverse\sidenote{The multiplicative inverse of a matrix does not exist if its determinant is $0$.}, $(M_n(\mathbb{R}), \cdot)$ is not a group.

\newthought{We can try} to do something similar as to what we did before: by excluding the elements that do not have an inverse. In this case, we exclude elements whose determinant is $0$. We define the following set

\begin{defn}[General Linear Group]\index{General Linear Group}
\label{defn:general_linear_group}
  The \hlnoteb{general linear group of degree $n$ over $\mathbb{R}$} is defined as
  \begin{equation*}
    GL_n(\mathbb{R}) := \{ M \in M_n(\mathbb{R}) \, : \, \det M \neq 0 \}
  \end{equation*}
\end{defn}

Note that $\because \det I = 1 \neq 0$, we have that $I \in GL_n(\mathbb{R})$. \\
Also, $\forall A, B \in GL_n(\mathbb{R} )$, we have that $\because \det A \neq 0 \, \land \, \det B \neq 0$,
\begin{equation*}
  \det AB = \det A \det B \neq 0,
\end{equation*}
and therefore $AB \in GL_n(\mathbb{R} )$. Finally, $\forall M \in GL_n(\mathbb{R})$, $\exists M^{-1} \in GL_n(\mathbb{R})$ such that
\begin{equation*}
  MM^{-1} = I = M^{-1} M
\end{equation*}
since $\det M \neq 0$. $\therefore (GL_n(\mathbb{R}), \cdot)$ is a group.

\newthought{Since} we have introduced permutations in \nameref{chapter:lecture_2_may_04th_2018}, we shall formalize the purpose of its introduction below.

\begin{eg}
  Consider $S_n$, the set of all permutations on $\{1, 2, ..., n\}$. By \cref{propo:properties_of_Sn}, we know that $S_n$ is a group. We call $S_n$ the \hldefn{symmetry group} \hlnoteb{of degree $n$}. For $n \geq 3$, the group $S_n$ is not abelian\sidenote{Let us make this an exercise.
  \begin{ex}
    For $n \geq 3$, prove that the group $S_n$ is not abelian.
  \end{ex}}.
\end{eg}

\newthought{Now that} we have a fairly good idea of the basic concept of a group, we will now proceed to look into handling multiple groups. One such operation is known as the \hldefn{direct product}.

\begin{eg}
  \label{eg:direct_product}
  Let $G$ and $H$ be groups. Their direct product is the set $G \times H$ with the component-wise operation defined by
  \begin{equation*}
    (g_1, h_1) * (g_2, h_2) = (g_1 *_G g_2, h_1 *_H h_2)
  \end{equation*}
  where $g_1, g_2 \in G$, $h_1, h_2 \in H$, $*_G$ is the operation on $G$, and $*_H$ is the operation on $H$.

  The \hlnoteb{closure} and \hlnoteb{associativity} property follow immediately from the definition of the operation. The identity is $(1_G, \, 1_H)$ where $1_G$ is the identity of $G$ and $1_H$ is the identity of $H$. The inverse of an element $(g_1, \, h_1) \in G \times H$ is $(g_1^{-1}, \, h_1^{-1})$.
\end{eg}

By induction, we can show that if $G_1, G_2, ..., G_n$ are groups, then so is $G_1 \times G_2 \times \hdots \times G_n$.

To facilitate our writing, use shall use the following notations:

\begin{notation}
  Given a group $G$ and $g_1, g_2 \in G$, we often denote its identity by $1$, and write $g_1 * g_2 = g_1 g_2$. Also, we denote the unique inverse of an element $g \in G$ as $g^{-1}$.

  We will write $g^0 = 1$. Also, for $n \in \mathbb{N}$, we define
  \begin{equation*}
     g^n = \underbrace{g * g * \hdots * g}_{n \text{ times}}
  \end{equation*}
  and
  \begin{equation*}
    g^{-n} = (g^{-1})^n
  \end{equation*}
\end{notation}

With the above notations,

\begin{propo}\label{propo:group_notations}
  Let $G$ be a group and $g, h \in G$. We have \marginnote{
    \begin{ex}
      Prove \cref{propo:group_notations} as an exercise.
    \end{ex}
  }
  \begin{enumerate}
    \item $(g^{-1})^{-1} = g$
    \item $(gh)^{-1} = h^{-1} g^{-1}$
    \item $g^n g^m = g^{n + m}$ for all $n, m \in \mathbb{Z}$
    \item $(g^n)^m = g^{nm}$ for all $n, m \in \mathbb{Z}$
  \end{enumerate}
\end{propo}

\begin{warning}
  In general, it is not true that if $g, h \in G$, then $(gh)^n = g^n h^n$. For example,
  \begin{equation*}
    (gh)^2 = ghgh \quad \text{but} \quad g^2 h^2 = gghh.
  \end{equation*}
  The two are only equal if and only if $G$ is abelian.
\end{warning}

% subsection groups (end)

% section groups (end)

% chapter lecture_3_may_07th_2018 (end)
