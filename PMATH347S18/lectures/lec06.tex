\chapter{Lecture 6 May 14th 2018}
\label{chp:lecture_6_may_14th_2018}
% chapter Lecture 6 May 14th 2018

\section{Subgroups (Continued 2)}
\label{sec:subgroups_continued_2}
% section Subgroups (Continued 2)

\subsection{Alternating Groups}
\label{sub:alternating_groups}
% subsection Alternating Groups

Recall that $\forall \sigma \in S_n$, with $\sigma \neq \epsilon$, $\sigma$ can be uniquely decomposed (up to the order) as disjoint cycles of length at least $2$. We will now present a related concept.

\begin{defn}[Transposition]\label{defn:transposition}
\index{Transposition}
  A \hlnoteb{transposition} $\sigma \in S_n$ is a cycle of length $2$, i.e. $\sigma = \begin{pmatrix} a & b \end{pmatrix}$, where $a, b \in \{1, ..., n\}$ and $a\ neq b$.
\end{defn}

\begin{eg}
  We have that\sidenote{If we apply the permutations on the right hand side, we have that
    \begin{gather*}
      1 \quad 2 \quad 3 \quad 4 \quad 5 \\
      \downarrow \\
      1 \quad 2 \quad 3 \quad 5 \quad 4 \\
      \downarrow \\
      1 \quad 4 \quad 3 \quad 5 \quad 2 \\
      \downarrow \\
      2 \quad 4 \quad 3 \quad 5 \quad 1
    \end{gather*}
  }
  \begin{equation*}
    \begin{pmatrix} 1 & 2 & 4 & 5 \end{pmatrix} = \begin{pmatrix} 1 & 2 \end{pmatrix} \begin{pmatrix} 2 & 4 \end{pmatrix} \begin{pmatrix} 4 & 5 \end{pmatrix}
  \end{equation*}
  Also, we can show that\sidenote{
  \begin{ex}
    Show that \autoref{eq:transposition_eg} is true.
  \end{ex}

  \begin{ex}
    Play around with the same idea and create a few of your own transpositions. Note that you will only be able to get an odd number of tranpositions (why?).
  \end{ex}
  }
  \begin{equation}\label{eq:transposition_eg}
    \begin{pmatrix} 1 & 2 & 4 & 5 \end{pmatrix} = \begin{pmatrix} 2 & 3 \end{pmatrix} \begin{pmatrix} 1 & 2 \end{pmatrix} \begin{pmatrix} 2 & 5 \end{pmatrix} \begin{pmatrix} 1 & 3 \end{pmatrix} \begin{pmatrix} 2 & 4 \end{pmatrix}
  \end{equation}
\end{eg}

Observe that the factorization into transpositions are \hlimpo{not unique or disjoint}. However, the following property is true.

\begin{thm}[Parity Theorem]\label{thm:parity_theorem}
\index{Parity Theorem}
  If a permutations $\sigma$ has $2$ factorizations
  \begin{equation*}
    \sigma = \gamma_1 \gamma_2 \hdots \gamma_r = \mu_1 \mu_2 \hdots \mu_s,
  \end{equation*}
  where each $\gamma_i$ and $\mu_j$ are transpositions, then $r \equiv s \mod 2$.
\end{thm}

\begin{proof}
  \hlwarn{This is the bonus question in A2. Proof shall be included after the end of the assignment.}
\end{proof}

\begin{defn}[Odd and Even Permutations]\label{defn:odd_and_even_permutations}
\index{Odd Permutations}\index{Even Permutations}
  A permutation $\sigma$ is even (or odd) if it can be written as a product of an even (or odd) number of transpositions. By \autoref{thm:parity_theorem}, a permutation must either be even or odd, but not both.
\end{defn}

\begin{thm}[Alternating Group]\label{thm:alternating_group}
\index{Alternating Group}
  For $n \geq 2$, let $A_n$ denote the set of all even permutations in $S_n$. Then
  \begin{enumerate}
    \item $\epsilon \in A_n$
    \item $\forall \sigma, \tau \in A_n \enspace \sigma \tau \in A_n$ and $\exists \sigma^{-1} \in A_n$ such that $\sigma \sigma^{-1} = \epsilon = \sigma^{-1} \sigma$
    \item $\abs{A_n} = \frac{1}{2} n!$
  \end{enumerate}
\end{thm}

\begin{note}
  From items 1 and 2, we know that $A_n$ si a subgroup of $S_n$. $A_n$ is called the \hlnoteb{alternating subgroup of degree $n$}.
\end{note}

\begin{proof}
  \begin{enumerate}
    \item We have that $\epsilon = \begin{pmatrix} 1 & 2 \end{pmatrix} \begin{pmatrix} 1 & 2 \end{pmatrix}$. Thus $\epsilon$ is even and so $\epsilon \in A_n$.
    \item $\forall \sigma, \tau \in A_n$, we may write
      \begin{align*}
        \sigma &= \sigma_1 \sigma_2 \hdots \sigma_r \quad \text{and} \\
        \tau   &= \tau_1 \tau_2 \hdots \tau_s,
      \end{align*}
      where $\sigma_i, \tau_j$ are transpositions, and $r, s$ are even integers. Then
      \begin{equation*}
        \sigma \tau = \sigma_1 \sigma_2 \hdots \sigma_r \tau_1 \tau_2 \hdots \tau_s
      \end{equation*}
      is a product of $(r + s)$ transpositions, and thus $\sigma \tau$ is even. THus $\sigma \tau \in A_n$.

      For the inverse, note that since $\sigma_i$ is a transposition, we have that $\sigma_i^2 = \epsilon$ and thus $\sigma_i^{-1} = \sigma_i$. It follows that
      \begin{align*}
        \sigma^{-1} &= (\sigma_1 \sigma_2 \hdots \sigma_r)^{-1} \\
          &= \sigma_r^{-1} \sigma_{r - 1}^{-1} \hdots \sigma_2^{-1} \sigma_1^{-1} \\
          &= \sigma_r \sigma_{r - 1} \hdots \sigma_2 \sigma_1
      \end{align*}
      which is an even permutation and
      \begin{equation*}
        \sigma \sigma^{-1} = \sigma_1 \sigma_2 \hdots \sigma_r \sigma_r \hdots \sigma_2 \sigma_1 = \epsilon.
      \end{equation*}
      Thus $\exists \sigma^{-1} \in A_n$ such that it is the inverse of $\sigma$.
    \item Let $O_n$ denote the set of odd permutations in $S_n$.\marginnote{For the proof of 3, we know that $\abs{S_n} = n!$, which is twice of the suggested order of $A_n$. Since we took out the even permutations of $S_n$, we just need to make the rest of the permutations, the odd permutations, into a set and prove that $A_n$ and this new set has the same size. One way to show this is by creating a bijection between the two.
    
        Also, note that the set of all odd permutations of $S_n$ is not a group, since
        \begin{itemize}
          \item there is no identity element in this set; and
          \item this set is not closed under map composition.
        \end{itemize}
    
        We have shown that $\epsilon$ is an even permutation, and so by the \hyperref[thm:parity_theorem]{Parity Theorem}, it cannot be an odd permutation, and there is only one identity in $S_n$. The set is not closed under map composition since if we compose two odd permutations, we would get an even permutation, which does not belong to this set.
    } Then we have $S_n = A_n \cup O_n$, and by the \hyperref[thm:parity_theorem]{Parity Theorem}, we have that $A_n \cap O_n = \emptyset$. Since $\abs{S_n} = n!$, to prove that $\abs{A_n} = \frac{1}{2} n!$, it suffices to show that $\abs{A_n} = \abs{O_n}$.
    
    Let $\gamma = \begin{pmatrix} 1 & 2 \end{pmatrix}$ and $f : A_n \to O_n$ such that $f(\sigma) = \gamma \sigma$. Since $\sigma$ is even, $\gamma \sigma$ is odd, and so $f$ is well-defined.
    
    Also, if $\gamma \sigma_1 = \gamma \sigma_2$, then by \hyperref[propo:cancellation_laws]{Cancellation Laws}, $\sigma_1 = \sigma_2$, and hence $f$ is injective.
    
    Finally, $\forall \tau \in O_n$, we have that $\gamma \tau = \sigma \in A_n$. Note that
  \begin{equation*}
    f(\sigma) = \gamma \sigma = \gamma \gamma \tau = \tau.
  \end{equation*}
  Therefore, $f$ is surjective.

  It follows that $\abs{A_n} = \abs{O_n}$. \qed
  \end{enumerate}
\end{proof}

% subsection Alternating Groups (end)

\subsection{Order of Elements}
\label{sub:order_of_elements}
% subsection Order of Elements

\begin{notation}
  If $G$ is a group and $g \in G$, we denote
  \begin{equation*}
    \lra{g} = \{ g^k : k \in \mathbb{Z} \}.
  \end{equation*}
  Note that $1 = g^0 \in \lra{g}$.

  If $x = g^m, y = g^n \in \lra{g}$ where $m, n \in \mathbb{Z}$, then
  \begin{equation*}
    xy = g^m g^n = g^{m + n} \in \lra{g}
  \end{equation*}
  and we have $\exists x^{-1} = g^{-m} \in \lra{g}$ such that
  \begin{equation*}
    xx^{-1} = g^m g^{-m} = g^0 = 1.
  \end{equation*}
\end{notation}

Along with the \hlnoteb{Subgroup Test}, we have the following proposition:

\begin{propo}[Cyclic Group as A Subgroup]\label{propo:cyclic_group_as_a_subgroup}
  If $G$ is a group and $g \in G$, then $\lra{g}$ is a subgroup of $G$.
\end{propo}

\begin{defn}[Cyclic Groups]\label{defn:cyclic_groups}
\index{Cyclic Group}
  Let $G$ be a group and $g \in G$. Then we call $\lra{g}$ the \hlnoteb{cyclic subgroup} of $G$ generated by $g$. If $G = \lra{g}$ for some $g \in G$, then we say that $G$ is a \hlnoteb{cyclic group}, and $g$ is a \hldefn{generator} of $G$.
\end{defn}

% subsection Order of Elements (end)

% section Subgroups (Continued 2) (end)

% chapter Lecture 6 May 14th 2018 (end)
