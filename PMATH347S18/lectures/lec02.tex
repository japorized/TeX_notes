\chapter{Lecture 2 May 04th 2018}
  \label{chapter:lecture_2_may_04th_2018}

\section{Introduction (Continued)} % (fold)
\label{sec:introduction_continued}

\subsection{Permutations} % (fold)
\label{sub:permutations}

\begin{defn}[Injectivity]\label{defn:injectivity}
\index{Injectivity}
  Let $f: X \to Y$ be a function. We say that $f$ is \hlnoteb{injective} (or \hldefn{one-to-one}) if $f(x_1) = f(x_2)$ implies $x_1 = x_2$.
\end{defn}

\begin{defn}[Surjectivity]\label{defn:surjectivity}
\index{Surjectivity}
  Let $f: X \to Y$ be a function. We say that $f$ is \hlnoteb{surjective} (or \hldefn{onto}) if $\forall y \in Y \enspace \exists x \in X \enspace f(x) = y$.
\end{defn}

\begin{defn}[Bijectivity]\label{defn:bijectivity}
\index{Bijectivity}
  Let $f: X \to Y$ be a function. We say that $f$ is \hlnoteb{bijective} if it is both \hlnoteb{injective} and \hlnoteb{surjective}.
\end{defn}

\begin{defn}[Permutations]\label{defn:permutations}
\index{Permutations}
  Given a non-empty set $L$, a permutation of $L$ is a bijection from $L$ to $L$. The set of all permutations of $L$ is denoted by $S_L$.
\end{defn}

\begin{eg}
  \label{eg:permutations_first}
  Consider the set $L = \{1, 2, 3\}$, which has the following $6$ different permutations: \marginnote{\begin{note}
    \begin{equation*}
      \begin{pmatrix} 1 & 2 & 3 \\ 1 & 3 & 2 \end{pmatrix}
    \end{equation*}
    indicates the bijection $\sigma: \{1, 2, 3\} \to \{1, 2, 3\}$ with $\sigma(1) = 1$, $\sigma(2) = 3$ and $\sigma(3) = 2$.
  \end{note}}

  \begin{gather*}
     \begin{pmatrix} 1 & 2 & 3 \\ 1 & 2 & 3 \end{pmatrix} \quad \begin{pmatrix} 1 & 2 & 3 \\ 1 & 3 & 2 \end{pmatrix} \quad \begin{pmatrix} 1 & 2 & 3 \\ 2 & 1 & 3 \end{pmatrix} \\
     \begin{pmatrix} 1 & 2 & 3 \\ 2 & 3 & 1 \end{pmatrix} \quad \begin{pmatrix} 1 & 2 & 3 \\ 3 & 1 & 2 \end{pmatrix} \quad \begin{pmatrix} 1 & 2 & 3 \\ 3 & 2 & 1 \end{pmatrix}
   \end{gather*} 
\end{eg}

\newthought{For $n \in \mathbb{N}$}, we denote $S_n := S_{\{1, 2, ..., n\}}$, the set of all permutations of $\{1, 2, ..., n\}$. \cref{eg:permutations_first} shows the elements of the set $S_3$.

\begin{defn}[Order]\label{defn:order}
\index{Order}
  The \hlnoteb{order} of a set $A$, denoted by $\abs{A}$, is the cardinality of the set.
\end{defn}

\begin{eg}
  \label{eg:order_of_prev_eg}
  We have seen that the order of $S_3$, $\abs{S_3}$ is $6 = 3!$.
\end{eg}

\begin{propo}\label{propo:order_of_Sn_is_n}
  $\abs{S_n} = n!$
\end{propo}

\begin{proof}
  $\forall \sigma \in S_n$, there are $n$ choices for $\sigma(1)$, $n - 1$ choices for $\sigma(2)$, ..., $2$ choices for $\sigma(n - 1)$, and finally $1$ choice for $\sigma(n)$. \qed
\end{proof}

\paragraph{Do elements of $S_n$ share the same properties as what we've seen in the numbers?} Given $\sigma, \tau \in S_n$, we can \hlnotea{compose} the 2 together to get a third element in $S_n$, namely $\sigma \tau$ (wlog), where $\sigma \tau : \{1, ..., n\} \to \{1, ..., n\}$ is given by $\forall x \in \{1, ..., n\}$, $x \mapsto \sigma( \tau(x) )$.

It is important to note that $\because \sigma, \tau$ are \hlimpo{both bijective}, $\sigma \tau$ is also bijective. Thus, together with the fact that $\sigma \tau : \{1, ..., n\} \to \{1, ..., n\}$, we have that $\sigma \tau \in S_n$ by definition of $S_n$.

$\therefore \forall \sigma, \tau \in S_n, \; \sigma \tau, \tau \sigma \in S_n$, but $\sigma \tau \neq \tau \sigma$ in general. The following is an example of the stated case:

\begin{eg}
  \label{eg:commutativity_of_Sn}
  Let
  \begin{equation*}
    \sigma = \begin{pmatrix}
      1 & 2 & 3 & 4 \\
      3 & 4 & 1 & 2
    \end{pmatrix}, \text{ and } 
    \tau = \begin{pmatrix}
      1 & 2 & 3 & 4 \\
      2 & 4 & 3 & 1
    \end{pmatrix}.
  \end{equation*}
  Compute $\sigma \tau$ and $\tau \sigma$ to show that they are not equal.

  \begin{solution}
    \begin{equation*}
      \sigma \tau = \begin{pmatrix}
        1 & 2 & 3 & 4 \\
        4 & 2 & 1 & 3
      \end{pmatrix} \text{ but } 
      \tau \sigma = \begin{pmatrix}
        1 & 2 & 3 & 4 \\
        3 & 1 & 2 & 4
      \end{pmatrix}
    \end{equation*}
  \end{solution}
\end{eg}

Perhaps what is interesting is the question of: \textbf{when does commutativity occur?} One such case is when $\sigma$ and $\tau$ have support sets that are disjoint\sidenote{This is proven in A1}.

On the other hand, the associative property holds\sidenote{
  \begin{ex}
    Prove this as an exercise.
  \end{ex}
}, i.e.
\begin{equation*}
  \forall \sigma, \tau, \mu \in S_n \enspace \sigma (\tau \mu) = (\sigma \tau) \mu
\end{equation*}

The set $S_n$ also has an identity element\sidenote{
  \begin{ex}
    Verify that the given identity element is indeed the identity, i.e.
    \begin{equation*}
      \forall \sigma \in S_n \enspace \sigma \epsilon = \sigma = \epsilon \sigma.
    \end{equation*}
  \end{ex}
}, namely
\begin{equation*}
  \epsilon = \begin{pmatrix}
    1 & 2 & \hdots & n \\
    1 & 2 & \hdots & n
  \end{pmatrix}
\end{equation*}

Finally, $\forall \sigma \in S_n$, since $\sigma$ is a bijection, we have that its inverse function, $\sigma^-1$ is also a bijection, and thus satisfies the requirements to be in $S_n$. We call $\sigma^{-1} \in S_n$ to be the \hldefn{inverse permutation} of $\sigma$, such that
\begin{equation*}
  \forall x, y \in \{1, ..., n\} \quad \sigma^{-1}(x) = y \iff \sigma(y) = x.
\end{equation*}
It follows, immediately, that
\begin{equation*}
  \sigma \big( \sigma^{-1}(x) \big) = x \, \land \, \sigma^{-1} \big( \sigma(y) \big) = y.
\end{equation*}
$\therefore$ We have that 
\begin{equation*}
  \sigma \sigma^{-1} = \epsilon = \sigma^{-1} \sigma.
\end{equation*}

\begin{eg}
  \label{eg:inverse_permutation}
  Find the inverse of
  \begin{equation*}
    \sigma = \begin{pmatrix}
      1 & 2 & 3 & 4 & 5 \\
      4 & 5 & 1 & 2 & 3
    \end{pmatrix}
  \end{equation*}

  \begin{solution}
    By rearranging the image in ascending order, using them now as the object and their respective objects as their image, construct
    \begin{equation*}
      \tau = \begin{pmatrix}
        1 & 2 & 3 & 4 & 5 \\
        3 & 4 & 5 & 1 & 2
      \end{pmatrix}.
    \end{equation*}
    It can easily (although perhaps not so prettily) be shown that
    \begin{equation*}
      \sigma \tau = \epsilon = \tau \sigma.
    \end{equation*}
  \end{solution}
\end{eg}

With all the above, we have for ourselves the following proposition:

\begin{propo}[Properties of $S_n$]\label{propo:properties_of_Sn}
  We have\sidenote{These properties show that $S_n$ is a group, which will be defined later.}
  \begin{enumerate}
    \item $\forall \sigma, \tau \in S_n \enspace \sigma \tau, \tau \sigma \in S_n$.
    \item $\forall \sigma, \tau, \mu \in S_n \enspace \sigma (\tau \mu) = (\sigma \tau) \mu$.
    \item $\exists \epsilon \in S_n \enspace \forall \sigma \in S_n \enspace \sigma \epsilon = \sigma = \epsilon \sigma$.
    \item $\forall \sigma \in S_n \enspace \exists! \sigma^{-1} \in S_n \enspace \sigma \sigma^{-1} = \epsilon = \sigma^{-1} \sigma$.
  \end{enumerate}
\end{propo}

\newthought{Consider}
\begin{equation*}
  \sigma = \begin{pmatrix}
    1 & 2 & 3 & 4 & 5 & 6 & 7 & 8 & 9 & 10 \\
    3 & 1 & 7 & 6 & 9 & 4 & 2 & 5 & 8 & 10
  \end{pmatrix} \in S_{10}
\end{equation*}
If we represent the action of $\sigma$ geometrically, we get

\begin{tabular}{c c c}
  \begin{tikzcd}
   & 1 \arrow[rd] &  \\
  2 \arrow[ru] &  & 3 \arrow[ld] \\
   & 7 \arrow[lu] & 
  \end{tikzcd}
  &
  \begin{tikzcd}
  4 \arrow[dd, bend left] \\
   \\
  6 \arrow[uu, bend left]
  \end{tikzcd}
  &
  \begin{tikzcd}
   & 5 \arrow[rdd] &  \\
   &  &  \\
  8 \arrow[ruu] &  & 9 \arrow[ll]
  \end{tikzcd}
  \\ &
  \begin{tikzcd}
    10 \arrow[loop below]
  \end{tikzcd}
\end{tabular}
We observe that $\sigma$ can be \hlnotea{decomposed} into one $4$-cycle, $\begin{pmatrix} 1 & 3 & 7 & 2 \end{pmatrix}$, one $2$-cycle, $\begin{pmatrix} 4 & 6 \end{pmatrix}$, one $3$-cycle, $\begin{pmatrix} 5 & 9 & 8 \end{pmatrix}$, and one $1$-cycle, $\begin{pmatrix} 10 \end{pmatrix}$.

Note that these cycles are (pairwise) \hlnotea{disjoint}, and we can write\sidenote{We generally do not include the $1$-cycle and assume that by excluding them, it is known that any number that is supposed to appear loops back to themselves.}
\begin{equation*}
  \sigma = \begin{pmatrix} 1 & 3 & 7 & 2 \end{pmatrix}\begin{pmatrix} 4 & 6 \end{pmatrix}\begin{pmatrix} 5 & 9 & 8 \end{pmatrix}
\end{equation*}
Note that we may also write
\begin{align*}
  \sigma &= \begin{pmatrix} 4 & 6 \end{pmatrix}\begin{pmatrix} 5 & 9 & 8 \end{pmatrix}\begin{pmatrix} 1 & 3 & 7 & 2 \end{pmatrix} \\
    &= \begin{pmatrix} 6 & 4 \end{pmatrix}\begin{pmatrix} 9 & 8 & 5 \end{pmatrix}\begin{pmatrix} 7 & 2 & 1 & 3 \end{pmatrix}
\end{align*}
It is interesting to note that the cycles can rotate their ``elements'' in a \hlnotea{cyclic} manner, i.e.
\begin{gather*}
  \begin{pmatrix} 1 & 3 & 7 & 2 \end{pmatrix} = \begin{pmatrix} 7 & 2 & 1 & 3 \end{pmatrix} \neq \begin{pmatrix} 1 & 2 & 7 & 3 \end{pmatrix}.
\end{gather*}
Although the decomposition of the cycle notation is not unique (i.e. you may rearrange them), each individual cycle is unique, and is proven below\sidenote{See bonus question of A1. Proof will be included in the notes once the assignment is over.}.

\begin{thm}[Cycle Decomposition Theorem]\label{thm:cycle_decomposition_theorem}
\index{Cycle Decomposition Theorem}
  If $\sigma \in S_n$, $\sigma \neq \epsilon$, then $\sigma$ is a product of (one or more) disjoint cycles of length at least $2$. This factorization is unique up to the order of the factors.
\end{thm}

\begin{note}[Convention]
 Every permutation in $S_n$ can be regarded as a permutation of $S_{n + 1}$ by fixing the permutation of $n + 1$. Therefore, we have that
 \begin{equation*}
   S_1 \subseteq S_2 \subseteq \hdots \subseteq S_n \subseteq S_{n + 1} \subseteq \hdots
 \end{equation*}
\end{note}

% subsection permutations (end)

% section introduction_continued (end)

% chapter lecture_2_may_04th_2018 (end)
