\documentclass[notoc,notitlepage]{tufte-book}
\setcounter{secnumdepth}{3}
\setcounter{tocdepth}{3}

\renewcommand{\baselinestretch}{1.1}

\usepackage{tikz-cd}
\setcounter{secnumdepth}{3}
\setcounter{tocdepth}{3}

\renewcommand{\baselinestretch}{1.1}
\usepackage{geometry}
\geometry{letterpaper}
\usepackage[parfill]{parskip}
\usepackage{graphicx}

% Essential Packages
\usepackage{makeidx}
\makeindex
\usepackage{enumitem}
\usepackage[T1]{fontenc}
\usepackage{natbib}
\bibliographystyle{apalike}
\usepackage{ragged2e}
\usepackage{etoolbox}
\usepackage{amssymb}
\usepackage{fontawesome}
\usepackage{amsmath}
\usepackage{mathrsfs}
\usepackage{mathtools}
\usepackage{xparse}
\usepackage{tkz-euclide}
\usetkzobj{all}
\usepackage[utf8]{inputenc}
\usepackage{csquotes}
\usepackage[english]{babel}
\usepackage{marvosym}
\usepackage{pgf,tikz}
\usepackage{pgfplots}
\usepackage{fancyhdr}
\usepackage{array}
\usepackage{faktor}
\usepackage{float}
\usepackage{xcolor}
\usepackage{centernot}
\usepackage{silence}
  \WarningFilter*{latex}{Marginpar on page \thepage\space moved}
\usepackage{tcolorbox}
\tcbuselibrary{skins,breakable}
\usepackage{longtable}
\usepackage[amsmath,hyperref]{ntheorem}
\usepackage{hyperref}
\usepackage[noabbrev,capitalize,nameinlink]{cleveref}

% xcolor (scheme: base16 eighties)
\definecolor{base16-eighties-dark}{HTML}{2D2D2D}
\definecolor{base16-eighties-light}{HTML}{D3D0C8}
\definecolor{base16-eighties-magenta}{HTML}{CD98CD}
\definecolor{base16-eighties-red}{HTML}{F47678}
\definecolor{base16-eighties-yellow}{HTML}{E2B552}
\definecolor{base16-eighties-green}{HTML}{98CD97}
\definecolor{base16-eighties-lightblue}{HTML}{61CCCD}
\definecolor{base16-eighties-blue}{HTML}{6498CE}
\definecolor{base16-eighties-brown}{HTML}{D47B4E}
\definecolor{base16-eighties-gray}{HTML}{747369}

% hyperref Package Settings
\hypersetup{
    bookmarks=true,         % show bookmarks bar?
    unicode=true,          % non-Latin characters in Acrobat’s bookmarks
    pdftoolbar=false,        % show Acrobat’s toolbar?
    pdfmenubar=false,        % show Acrobat’s menu?
    pdffitwindow=true,     % window fit to page when opened
    colorlinks=true,
    allcolors=base16-eighties-magenta,
}

% tikz
\usepgfplotslibrary{polar}
\usepgflibrary{shapes.geometric}
\usetikzlibrary{angles,patterns,calc,decorations.markings}
\tikzset{midarrow/.style 2 args={
        decoration={markings,
            mark= at position #2 with {\arrow{#1}} ,
        },
        postaction={decorate}
    },
    midarrow/.default={latex}{0.5}
}
\def\centerarc[#1](#2)(#3:#4:#5)% Syntax: [draw options] (center) (initial angle:final angle:radius)
    { \draw[#1] ($(#2)+({#5*cos(#3)},{#5*sin(#3)})$) arc (#3:#4:#5); }

% enumitems
\newlist{inlinelist}{enumerate*}{1}
\setlist*[inlinelist,1]{%
  label=(\roman*),
}

% Theorem Style Customization
\setlength\theorempreskipamount{2ex}
\setlength\theorempostskipamount{3ex}

\makeatletter
\let\nobreakitem\item
\let\@nobreakitem\@item
\patchcmd{\nobreakitem}{\@item}{\@nobreakitem}{}{}
\patchcmd{\nobreakitem}{\@item}{\@nobreakitem}{}{}
\patchcmd{\@nobreakitem}{\@itempenalty}{\@M}{}{}
\patchcmd{\@xthm}{\ignorespaces}{\nobreak\ignorespaces}{}{}
\patchcmd{\@ythm}{\ignorespaces}{\nobreak\ignorespaces}{}{}

\renewtheoremstyle{break}%
  {\item{\theorem@headerfont
          ##1\ ##2\theorem@separator}\hskip\labelsep\relax\nobreakitem}%
  {\item{\theorem@headerfont
          ##1\ ##2\ (##3)\theorem@separator}\hskip\labelsep\relax\nobreakitem}
\makeatother

% ntheorem + framed
\makeatletter

% ntheorem Declarations
\theorempreskip{10pt}
\theorempostskip{5pt}
\theoremstyle{break}

\newtheorem*{solution}{\faPencil $\enspace$ Solution}
\newtheorem*{remark}{Remark}
\newtheorem{eg}{Example}[section]
\newtheorem{ex}{Exercise}[section]

    % definition env
\theoremprework{\textcolor{base16-eighties-blue}{\hrule height 2pt}}
\theoremheaderfont{\color{base16-eighties-blue}\normalfont\bfseries}
\theorempostwork{\textcolor{base16-eighties-blue}{\hrule height 2pt}}
\theoremindent10pt
\newtheorem{defn}{\faBook \enspace Definition}

    % definition env no num
\theoremprework{\textcolor{base16-eighties-blue}{\hrule height 2pt}}
\theoremheaderfont{\color{base16-eighties-blue}\normalfont\bfseries}
\theorempostwork{\textcolor{base16-eighties-blue}{\hrule height 2pt}}
\theoremindent10pt
\newtheorem*{defnnonum}{\faBook \enspace Definition}

    % theorem envs
\theoremprework{\textcolor{base16-eighties-magenta}{\hrule height 2pt}}
\theoremheaderfont{\color{base16-eighties-magenta}\normalfont\bfseries}
\theorempostwork{\textcolor{base16-eighties-magenta}{\hrule height 2pt}}
\theoremindent10pt
\newtheorem{thm}{\faCoffee \enspace Theorem}

\theoremprework{\textcolor{base16-eighties-magenta}{\hrule height 2pt}}
\theorempostwork{\textcolor{base16-eighties-magenta}{\hrule height 2pt}}
\theoremindent10pt
\newtheorem{propo}[thm]{\faTint \enspace Proposition}

\theoremprework{\textcolor{base16-eighties-magenta}{\hrule height 2pt}}
\theorempostwork{\textcolor{base16-eighties-magenta}{\hrule height 2pt}}
\theoremindent10pt
\newtheorem{crly}[thm]{\faSpaceShuttle \enspace Corollary}

\theoremprework{\textcolor{base16-eighties-magenta}{\hrule height 2pt}}
\theorempostwork{\textcolor{base16-eighties-magenta}{\hrule height 2pt}}
\theoremindent10pt
\newtheorem{lemma}[thm]{\faTree \enspace Lemma}

\theoremprework{\textcolor{base16-eighties-magenta}{\hrule height 2pt}}
\theorempostwork{\textcolor{base16-eighties-magenta}{\hrule height 2pt}}
\theoremindent10pt
\newtheorem{axiom}[thm]{\faShield \enspace Axiom}

    % theorem envs without counter
\theoremprework{\textcolor{base16-eighties-magenta}{\hrule height 2pt}}
\theoremheaderfont{\color{base16-eighties-magenta}\normalfont\bfseries}
\theorempostwork{\textcolor{base16-eighties-magenta}{\hrule height 2pt}}
\theoremindent10pt
\newtheorem*{thmnonum}{\faCoffee \enspace Theorem}

\theoremprework{\textcolor{base16-eighties-magenta}{\hrule height 2pt}}
\theorempostwork{\textcolor{base16-eighties-magenta}{\hrule height 2pt}}
\theoremindent10pt
\newtheorem*{propononum}{\faTint \enspace Proposition}

\theoremprework{\textcolor{base16-eighties-magenta}{\hrule height 2pt}}
\theorempostwork{\textcolor{base16-eighties-magenta}{\hrule height 2pt}}
\theoremindent10pt
\newtheorem*{crlynonum}{\faSpaceShuttle \enspace Corollary}

\theoremprework{\textcolor{base16-eighties-magenta}{\hrule height 2pt}}
\theorempostwork{\textcolor{base16-eighties-magenta}{\hrule height 2pt}}
\theoremindent10pt
\newtheorem*{lemmanonum}{\faTree \enspace Lemma}

\theoremprework{\textcolor{base16-eighties-magenta}{\hrule height 2pt}}
\theorempostwork{\textcolor{base16-eighties-magenta}{\hrule height 2pt}}
\theoremindent10pt
\newtheorem*{axiomnonum}{\faShield \enspace Axiom}

    % proof env
\theoremprework{\textcolor{base16-eighties-brown}{\hrule height 2pt}}
\theoremheaderfont{\color{base16-eighties-brown}\normalfont\bfseries}
\theorempostwork{\textcolor{base16-eighties-brown}{\hrule height 2pt}}
\newtheorem*{proof}{\faPencil \enspace Proof}

    % note and notation env
\theoremprework{\textcolor{base16-eighties-yellow}{\hrule height 2pt}}
\theoremheaderfont{\color{base16-eighties-yellow}\normalfont\bfseries}
\theorempostwork{\textcolor{base16-eighties-yellow}{\hrule height 2pt}}
\newtheorem*{note}{\faQuoteLeft \enspace Note}

\theoremprework{\textcolor{base16-eighties-yellow}{\hrule height 2pt}}
\theorempostwork{\textcolor{base16-eighties-yellow}{\hrule height 2pt}}
\newtheorem*{notation}{\faPaw \enspace Notation}

    % warning env
\theoremprework{\textcolor{base16-eighties-red}{\hrule height 2pt}}
\theoremheaderfont{\color{base16-eighties-red}\normalfont\bfseries}
\theorempostwork{\textcolor{base16-eighties-red}{\hrule height 2pt}}
\theoremindent10pt
\newtheorem*{warning}{\faBug \enspace Warning}

% more environments
\newtcolorbox{redquote}{
  blanker,enhanced,breakable,standard jigsaw,
  opacityback=0,
  coltext=base16-eighties-light,
  left=5mm,right=5mm,top=2mm,bottom=2mm,
  colframe=base16-eighties-red,
  boxrule=0pt,leftrule=3pt,
  fontupper=\itshape
}
\newtcolorbox{bluequote}{
  blanker,enhanced,breakable,standard jigsaw,
  opacityback=0,
  coltext=base16-eighties-light,
  left=5mm,right=5mm,top=2mm,bottom=2mm,
  colframe=base16-eighties-blue,
  boxrule=0pt,leftrule=3pt,
  fontupper=\itshape
}
\newtcolorbox{greenquote}{
  blanker,enhanced,breakable,standard jigsaw,
  opacityback=0,
  coltext=base16-eighties-light,
  left=5mm,right=5mm,top=2mm,bottom=2mm,
  colframe=base16-eighties-green,
  boxrule=0pt,leftrule=3pt,
  fontupper=\itshape
}
\newtcolorbox{yellowquote}{
  blanker,enhanced,breakable,standard jigsaw,
  opacityback=0,
  coltext=base16-eighties-light,
  left=5mm,right=5mm,top=2mm,bottom=2mm,
  colframe=base16-eighties-yellow,
  boxrule=0pt,leftrule=3pt,
  fontupper=\itshape
}
\newtcolorbox{magentaquote}{
  blanker,enhanced,breakable,standard jigsaw,
  opacityback=0,
  coltext=base16-eighties-light,
  left=5mm,right=5mm,top=2mm,bottom=2mm,
  colframe=base16-eighties-magenta,
  boxrule=0pt,leftrule=3pt,
  fontupper=\itshape
}

% ntheorem listtheorem style
\makeatother
\newlength\widesttheorem
\AtBeginDocument{
  \settowidth{\widesttheorem}{Proposition A.1.1.1\quad}
}

\makeatletter
\def\thm@@thmline@name#1#2#3#4{%
        \@dottedtocline{-2}{0em}{2.3em}%
                   {\makebox[\widesttheorem][l]{#1 \protect\numberline{#2}}#3}%
                   {#4}}
\@ifpackageloaded{hyperref}{
\def\thm@@thmline@name#1#2#3#4#5{%
    \ifx\#5\%
        \@dottedtocline{-2}{0em}{2.3em}%
            {\makebox[\widesttheorem][l]{#1 \protect\numberline{#2}}#3}%
            {#4}
    \else
        \ifHy@linktocpage\relax\relax
            \@dottedtocline{-2}{0em}{2.3em}%
                {\makebox[\widesttheorem][l]{#1 \protect\numberline{#2}}#3}%
                {\hyper@linkstart{link}{#5}{#4}\hyper@linkend}%
        \else
            \@dottedtocline{-2}{0em}{2.3em}%
                {\hyper@linkstart{link}{#5}%
                  {\makebox[\widesttheorem][l]{#1 \protect\numberline{#2}}#3}\hyper@linkend}%
                    {#4}%
        \fi
    \fi}
}

\makeatletter
\def\thm@@thmline@noname#1#2#3#4{%
        \@dottedtocline{-2}{0em}{5em}%
                   {{\protect\numberline{#2}}#3}%
                   {#4}}
\@ifpackageloaded{hyperref}{
\def\thm@@thmline@noname#1#2#3#4#5{%
    \ifx\#5\%
        \@dottedtocline{-2}{0em}{5em}%
            {{\protect\numberline{#2}}#3}%
            {#4}
    \else
        \ifHy@linktocpage\relax\relax
            \@dottedtocline{-2}{0em}{5em}%
                {{\protect\numberline{#2}}#3}%
                {\hyper@linkstart{link}{#5}{#4}\hyper@linkend}%
        \else
            \@dottedtocline{-2}{0em}{5em}%
                {\hyper@linkstart{link}{#5}%
                  {{\protect\numberline{#2}}#3}\hyper@linkend}%
                    {#4}%
        \fi
    \fi}
}

\theoremlisttype{allname}

\AtBeginDocument{\renewcommand\contentsname{Table of Contents}}

% Heading formattings
% chapter format
\titleformat{\chapter}%
  {\huge\rmfamily\itshape\color{base16-eighties-magenta}}% format applied to label+text
  {\llap{\colorbox{base16-eighties-magenta}{\parbox{1.5cm}{\hfill\itshape\huge\textcolor{base16-eighties-dark}{\thechapter}}}}}% label
  {5pt}% horizontal separation between label and title body
  {}% before the title body
  []% after the title body

% section format
\titleformat{\section}%
  {\normalfont\Large\rmfamily\itshape\color{base16-eighties-blue}}% format applied to label+text
  {\llap{\colorbox{base16-eighties-blue}{\parbox{1.5cm}{\hfill\itshape\textcolor{base16-eighties-dark}{\thesection}}}}}% label
  {5pt}% horizontal separation between label and title body
  {}% before the title body
  []% after the title body

% subsection format
\titleformat{\subsection}%
  {\normalfont\large\itshape\color{base16-eighties-green}}% format applied to label+text
  {\llap{\colorbox{base16-eighties-green}{\parbox{1.5cm}{\hfill\textcolor{base16-eighties-dark}{\thesubsection}}}}}% label
  {1em}% horizontal separation between label and title body
  {}% before the title body
  []% after the title body

% Sidenote enhancements
\def\mathmarginnote#1{%
  \tag*{\rlap{\hspace\marginparsep\smash{\parbox[t]{\marginparwidth}{%
  \footnotesize#1}}}}
}

% Custom table columning
\newcolumntype{L}[1]{>{\raggedright\let\newline\\\arraybackslash\hspace{0pt}}m{#1}}
\newcolumntype{C}[1]{>{\centering\let\newline\\\arraybackslash\hspace{0pt}}m{#1}}
\newcolumntype{R}[1]{>{\raggedleft\let\newline\\\arraybackslash\hspace{0pt}}m{#1}}

% Custom math operator
% \DeclareMathOperator{\rem}{rem}
\DeclareMathOperator*{\argmax}{arg\,max}
\DeclareMathOperator*{\argmin}{arg\,min}
\DeclareMathOperator{\re}{Re}
\DeclareMathOperator{\im}{Im}
\DeclareMathOperator{\caparg}{Arg}
\DeclareMathOperator{\Ind}{Ind}
\DeclareMathOperator{\Res}{Res}

% Graph styles
\pgfplotsset{compat=1.15}
\usepgfplotslibrary{fillbetween}
\pgfplotsset{four quads/.append style={axis x line=middle, axis y line=
middle, xlabel={$x$}, ylabel={$y$}, axis equal }}
\pgfplotsset{four quad complex/.append style={axis x line=middle, axis y line=
middle, xlabel={$\re$}, ylabel={$\im$}, axis equal }}

% Shortcuts
\newcommand{\floor}[1]{\lfloor #1 \rfloor}      % simplifying the writing of a floor function
\newcommand{\ceiling}[1]{\lceil #1 \rceil}      % simplifying the writing of a ceiling function
\newcommand{\dotp}{\, \cdotp}			        % dot product to distinguish from \cdot
\newcommand{\qed}{\hfill\ensuremath{\square}}   % Q.E.D sign
\newcommand{\abs}[1]{\left|#1\right|}						% absolute value
\newcommand{\lra}[1]{\langle \; #1 \; \rangle}
\newcommand{\at}[2]{\Big|_{#1}^{#2}}
\newcommand{\Arg}[1]{\caparg #1}
\renewcommand{\bar}[1]{\mkern 1.5mu \overline{\mkern -1.5mu #1 \mkern -1.5mu} \mkern 1.5mu}
\newcommand{\quotient}[2]{\faktor{#1}{#2}}
\newcommand{\cyclic}[1]{\left\langle #1 \right\rangle}
	% highlighting shortcuts
\newcommand{\hlimpo}[1]{\textcolor{base16-eighties-red}{\textbf{#1}}}
\newcommand{\hlwarn}[1]{\textcolor{base16-eighties-yellow}{\textbf{#1}}}
\newcommand{\hldefn}[1]{\textcolor{base16-eighties-blue}{\index{#1}\textbf{#1}}}
\newcommand{\hlnotea}[1]{\textcolor{base16-eighties-green}{\textbf{#1}}}
\newcommand{\hlnoteb}[1]{\textcolor{base16-eighties-lightblue}{\textbf{#1}}}
\newcommand{\hlnotec}[1]{\textcolor{base16-eighties-brown}{\textbf{#1}}}
\newcommand{\WTP}{\textcolor{base16-eighties-brown}{WTP} }
\newcommand{\WTS}{\textcolor{base16-eighties-brown}{WTS} }
\newcommand{\ind}[2]{\Ind_{#2}\left( #1 \right)}
\newcommand{\notimply}{\centernot\implies}
\newcommand{\res}[2]{\underset{#2}{\Res} #1 }
\newcommand{\tworow}[3]{\begin{tabular}{@{}#1@{}} #2 \\ #3 \end{tabular}}
\renewcommand{\epsilon}{\varepsilon}
\newcommand{\lrarrow}{\leftrightarrow}
\newcommand{\larrow}{\leftarrow}
\newcommand{\rarrow}{\rightarrow}
\renewcommand{\atop}[2]{\genfrac{}{}{0pt}{}{#1}{#2}}
\newcommand*\dif{\mathop{}\!d}

  % inspiration from: https://tex.stackexchange.com/questions/8720/overbrace-underbrace-but-with-an-arrow-instead#37758
\newcommand{\overarrow}[2]{
  \overset{\makebox[0pt]{\begin{tabular}{@{}c@{}}#2\\[0pt]\ensuremath{\uparrow}\end{tabular}}}{#1}
}
\newcommand{\underarrow}[2]{
  \underset{\makebox[0pt]{\begin{tabular}{@{}c@{}}\downarrow\\[0pt]\ensuremath{#2}\end{tabular}}}{#1}
}

% Document header formatting
\renewcommand{\chaptermark}[1]{\markboth{#1}{}}
\renewcommand{\sectionmark}[1]{\markright{#1}}
\makeatletter
\pagestyle{fancy}
\fancyhead{}
\fancyhead[RO]{\textsl{\@title} \enspace \thepage}
\fancyhead[LE]{\thepage \enspace \textsl{\leftmark \enspace - \enspace \rightmark}}
\makeatother

% Comment the two lines below if you want to print the document
\pagecolor{base16-eighties-dark}
\color{base16-eighties-light}


% Main Body
\title{PMATH347S18 - Groups \& Rings}
\author{Johnson Ng}

\begin{document}
\hypersetup{pageanchor=false}
\maketitle
\hypersetup{pageanchor=true}
\tableofcontents

\chapter*{List of Definitions}
\theoremlisttype{all}
\listtheorems{defn}

\chapter*{List of Theorems}
\theoremlisttype{allname}
\listtheorems{axiom,lemma,thm,crly,propo}

\chapter{Lecture 1 May 02nd 2018}
  \label{chapter:lecture_1_may_02nd_2018}

\section{Introduction} % (fold)
\label{sec:introduction}

\subsection{Numbers} % (fold)
\label{sub:numbers}

The following are some of the number sets that we are already familiar with:
\begin{gather*}
  \mathbb{N} = \{1, 2, 3, ...\} \qquad \mathbb{Z} = \{.., -2, -1, 0, 1, 2, ...\} \\
  \mathbb{Q} = \left\{\frac{a}{b} : a \in \mathbb{Z}, b \in \mathbb{N} \right\} \qquad \mathbb{R} = \text{ set of real numbers} \\
  \mathbb{C} = \{a + bi : a, b \in \mathbb{R}, i = \sqrt{-1} \} = \text{ set of complex numbers} 
\end{gather*}
For $n \in \mathbb{Z}$, let $\mathbb{Z}_n$ denote the set of integers modulo $n$, i.e.
\begin{equation*}
  \mathbb{Z}_n = \{ [0], [1], ..., [n - 1] \}
\end{equation*}
where the $[r]$, $0 \leq r \leq n - 1$, are the congruence classes, i.e.
\begin{equation*}
  [r] = \{z \in \mathbb{Z} : z \equiv r \mod n\}
\end{equation*}

These sets share some common properties, e.g. $+$ and $\times$. Let's try to break that down to make further observation.

\newthought{Note that} for $R = \mathbb{N}, \, \mathbb{Z}, \, \mathbb{Q}, \, \mathbb{R}, \, \mathbb{C},$ or $\mathbb{Z}_n$, $R$ has 2 operations, i.e. addition and multiplication.

\paragraph{Addition} If $r_1, r_2, r_3 \in R$, then
\begin{itemize}
  \item (\hldefn{closure}) $r_1 + r_2 \in R$
  \item (\hldefn{associativity}) $r_1 + (r_2 + r_3) = (r_1 + r_2) + r_3$
\end{itemize}
Also, if $R \neq \mathbb{N}$, then $\exists 0 \in R$ (the \hldefn{additive identity}) such that
\begin{equation*}
  \forall r \in R \quad r + 0 = r = 0 + r.
\end{equation*}
Also, $\forall r \in R$, $\exists (-r) \in R$ such that
\begin{equation*}
  r + (-r) = 0 = (-r) + r.
\end{equation*}

\paragraph{Multiplication} For $r_1, r_2, r_3 \in R$, we have
\begin{itemize}
  \item (\hlnoteb{closure}) $r_1 r_2 \in R$
  \item (\hlnoteb{associativity}) $r_1 (r_2 r_3) = (r_1 r_2) r_3$
\end{itemize}
Also, $\exists 1 \in R$ (a.k.a the \hldefn{mutiplicative identity}), such that
\begin{equation*}
  \forall r \in R \quad r \cdot 1 = r = 1 \cdot r.
\end{equation*}
Finally, for $R = \mathbb{Q}, \, \mathbb{R},$ or $\mathbb{C}$, $\forall r \in R, \, \exists r^{-1} \in R$ such that
\begin{equation*}
  r \cdot r^{-1} = 1 = r^{-1} \cdot r.
\end{equation*}
Note that for $R = \mathbb{Z}_n$, where $n \in \mathbb{Z}$, not all $[r] \in \mathbb{Z}_n$ have a multiplicative inverse. For example, for $[2] \in \mathbb{Z}_4$, there is no $[x] \in \mathbb{Z}_4$ such that $[2][x] = [1]$.\sidenote{This is best proven using techniques introduced in MATH135/145.}

% subsection numbers (end)

\subsection{Matrices}
  \label{sub:matrices}

For $n \in \mathbb{N} \setminus \{1\}$, an $n \times n$ matrix over $\mathbb{R}$ \sidenote{$\mathbb{R}$ can be replaced by $\mathbb{Q}$ or $\mathbb{C}$.} is an $n \times n$ array that can be expressed as follows:
\begin{equation*}
  A = [a_{ij}] = \begin{bmatrix}
    a_{11} & a_{12} & \hdots & a_{1n} \\
    a_{21} & a_{22} & \hdots & a_{2n} \\
    \vdots & \vdots &        & \vdots \\
    a_{n1} & a_{n2} & \hdots & a_{nn}
  \end{bmatrix}
\end{equation*}
where for $1 \leq i, j \leq n$, $a_{ij} \in \mathbb{R}$. We denote $M_n(\mathbb{R})$ as the set of all $n \times n$ matrices over $\mathbb{R}$.

As in \cref{sub:numbers}, we can perform \hlnotea{addition and multiplication} on $M_n(\mathbb{R})$.

\paragraph{Matrix Addition} Given $A = [a_{ij}], B = [b_{ij}], C = [c_{ij}] \in M_n(\mathbb{R})$, we define matrix addition as
\begin{equation*}
  A + B = [a_{ij} + b_{ij}],
\end{equation*}
which immediately gives the \hlnoteb{closure property}, since $a_{ij} + b_{ij} \in \mathbb{R}$ and hence $A + B \in M_n(\mathbb{R})$. Also, by this definition, we also immediately obtain the \hlnoteb{associativity property}, i.e.
\begin{equation*}
  A + (B + C) = (A + B) + C.
\end{equation*}
We define the zero matrix as
\begin{equation*}
  0 = \begin{bmatrix}
    0      &   0    & \hdots &   0 \\
    0      &   0    & \hdots &   0 \\
    \vdots & \vdots &        & \vdots \\
    0      &   0    & \hdots &   0
  \end{bmatrix}.
\end{equation*}
Then we have that $0$ is the \hlnoteb{additive identity}, i.e.
\begin{equation*}
  A + 0 = A = 0 + A.
\end{equation*}
Finally, $\forall A \in M_n(\mathbb{R})$, $\exists (-A) \in M_n(\mathbb{R})$ (the \hlnoteb{additive inverse}) such that
\begin{equation*}
  A + (-A) = 0 - (-A) + A.
\end{equation*}

Note that in this case, we also have that that the operation is \hlnoteb{commutative}, i.e.
\begin{equation*}
  A + B = B + A.
\end{equation*}

\paragraph{Matrix Multiplication} Given $A = [a_{ij}], B = [b_{ij}], C = [c_{ij}] \in M_n(\mathbb{R})$, we define the matrix multiplication as
\begin{equation*}
  AB = [d_{ij}] \text{ where } c_{ij} = \sum_{k=1}^{n} a_{ik} b_{kj} \in \mathbb{R}.
\end{equation*}
Clearly, $AB \in M_n(\mathbb{R})$, i.e. it is \hlnoteb{closed under matrix multiplication}. Also, we have that, under such a defintion, matrix multiplication is \hlnoteb{associative}, i.e.
\begin{equation*}
  A(BC) = (AB)C.
\end{equation*}
Define the identity matrix, $I \in M_n(\mathbb{R})$, as follows:
\begin{equation*}
  I = \begin{bmatrix}
    1      &   0    & \hdots & 0 \\
    0      &   1    & \hdots & 0 \\
    \vdots & \vdots &        & \vdots \\
    0      &   0    & \hdots & 1
  \end{bmatrix}.
\end{equation*}
Then we have that $I$ is the \hlnoteb{multiplicative identity}, since
\begin{equation*}
  AI = A = IA.
\end{equation*}
However, contrary to matrix addition, $\forall A \in M_n(\mathbb{R})$, it is not always true that $\exists A^{-1} \in M_n(\mathbb{R})$ such that\marginnote{This is especially true if the \hlnotea{determinant} of $A$ is $0$.}
\begin{equation*}
  AA^{-1} = I = A^{-1} A.
\end{equation*}

Also, we can always find some $A, B \in M_n(\mathbb{R})$ such that
\begin{equation*}
  AB \neq BA,
\end{equation*}
i.e. matrix multiplication is not always commutative.

\newthought{The common properties} of the operations from above: \hlimpo{closure, associativity, and existence of an inverse}, are not unique to just addition and multiplication. We shall see in the next lecture that there are other operations where these properties will continue to hold, e.g. \hlnoteb{permutations}.

% subsection matrices (end)

% section introduction (end)

% chapter lecture_1_may_02nd_2018 (end)

\chapter{Lecture 2 May 04th 2018}
  \label{chapter:lecture_2_may_04th_2018}

\section{Introduction (Continued)} % (fold)
\label{sec:introduction_continued}

\subsection{Permutations} % (fold)
\label{sub:permutations}

\begin{defn}[Injectivity\index{Injectivity}]\label{defn:injectivity}
  Let $f: X \to Y$ be a function. We say that $f$ is \hlnoteb{injective} (or \hldefn{one-to-one}) if $f(x_1) = f(x_2)$ implies $x_1 = x_2$.
\end{defn}

\begin{defn}[Surjectivity\index{Surjectivity}]\label{defn:surjectivity}
  Let $f: X \to Y$ be a function. We say that $f$ is \hlnoteb{surjective} (or \hldefn{onto}) if $\forall y \in Y \enspace \exists x \in X \enspace f(x) = y$.
\end{defn}

\begin{defn}[Bijectivity\index{Bijectivity}]\label{defn:bijectivity}
  Let $f: X \to Y$ be a function. We say that $f$ is \hlnoteb{bijective} if it is both \hlnoteb{injective} and \hlnoteb{surjective}.
\end{defn}

\begin{defn}[Permutations\index{Permutations}]\label{defn:permutations}
  Given a non-empty set $L$, a permutation of $L$ is a bijection from $L$ to $L$. The set of all permutations of $L$ is denoted by $S_L$.
\end{defn}

\begin{eg}
  \label{eg:permutations_first}
  Consider the set $L = \{1, 2, 3\}$, which has the following $6$ different permutations: \marginnote{\begin{note}
    \begin{equation*}
      \begin{pmatrix} 1 & 2 & 3 \\ 1 & 3 & 2 \end{pmatrix}
    \end{equation*}
    indicates the bijection $\sigma: \{1, 2, 3\} \to \{1, 2, 3\}$ with $\sigma(1) = 1$, $\sigma(2) = 3$ and $\sigma(3) = 2$.
  \end{note}}

  \begin{gather*}
     \begin{pmatrix} 1 & 2 & 3 \\ 1 & 2 & 3 \end{pmatrix} \quad \begin{pmatrix} 1 & 2 & 3 \\ 1 & 3 & 2 \end{pmatrix} \quad \begin{pmatrix} 1 & 2 & 3 \\ 2 & 1 & 3 \end{pmatrix} \\
     \begin{pmatrix} 1 & 2 & 3 \\ 2 & 3 & 1 \end{pmatrix} \quad \begin{pmatrix} 1 & 2 & 3 \\ 3 & 1 & 2 \end{pmatrix} \quad \begin{pmatrix} 1 & 2 & 3 \\ 3 & 2 & 1 \end{pmatrix}
   \end{gather*} 
\end{eg}

\newthought{For $n \in \mathbb{N}$}, we denote $S_n := S_{\{1, 2, ..., n\}}$, the set of all permutations of $\{1, 2, ..., n\}$. \cref{eg:permutations_first} shows the elements of the set $S_3$.

\begin{defn}[Order\index{Order}]\label{defn:order}
  The \hlnoteb{order} of a set $A$, denoted by $\abs{A}$, is the cardinality of the set.
\end{defn}

\begin{eg}
  \label{eg:order_of_prev_eg}
  We have seen that the order of $S_3$, $\abs{S_3}$ is $6 = 3!$.
\end{eg}

\begin{propo}\label{propo:order_of_Sn_is_n}
  $\abs{S_n} = n!$
\end{propo}

\begin{proof}
  $\forall \sigma \in S_n$, there are $n$ choices for $\sigma(1)$, $n - 1$ choices for $\sigma(2)$, ..., $2$ choices for $\sigma(n - 1)$, and finally $1$ choice for $\sigma(n)$. \qed
\end{proof}

\paragraph{Do elements of $S_n$ share the same properties as what we've seen in the numbers?} Given $\sigma, \tau \in S_n$, we can \hlnotea{compose} the 2 together to get a third element in $S_n$, namely $\sigma \tau$ (wlog), where $\sigma \tau : \{1, ..., n\} \to \{1, ..., n\}$ is given by $\forall x \in \{1, ..., n\}$, $x \mapsto \sigma( \tau(x) )$.

It is important to note that $\because \sigma, \tau$ are \hlimpo{both bijective}, $\sigma \tau$ is also bijective. Thus, together with the fact that $\sigma \tau : \{1, ..., n\} \to \{1, ..., n\}$, we have that $\sigma \tau \in S_n$ by definition of $S_n$.

$\therefore \forall \sigma, \tau \in S_n, \; \sigma \tau, \tau \sigma \in S_n$, but $\sigma \tau \neq \tau \sigma$ in general. The following is an example of the stated case:

\begin{eg}
  \label{eg:commutativity_of_Sn}
  Let
  \begin{equation*}
    \sigma &= \begin{pmatrix}
      1 & 2 & 3 & 4 \\
      3 & 4 & 1 & 2
    \end{pmatrix}, \text{ and } 
    \tau &= \begin{pmatrix}
      1 & 2 & 3 & 4 \\
      2 & 4 & 3 & 1
    \end{pmatrix}.
  \end{equation*}
  Compute $\sigma \tau$ and $\tau \sigma$ to show that they are not equal.

  \begin{solution}
    \begin{equation*}
      \sigma \tau &= \begin{pmatrix}
        1 & 2 & 3 & 4 \\
        4 & 2 & 1 & 3
      \end{pmatrix} \text{ but } 
      \tau \sigma &= \begin{pmatrix}
        1 & 2 & 3 & 4 \\
        3 & 1 & 2 & 4
      \end{pmatrix}
    \end{equation*}
  \end{solution}
\end{eg}

Perhaps what is interesting is the question of: \textbf{when does commutativity occur?} One such case is when $\sigma$ and $\tau$ have support sets that are disjoint\sidenote{This is proven in A1}.

On the other hand, the associative property holds\sidenote{
  \begin{ex}
    Prove this as an exercise.
  \end{ex}
}, i.e.
\begin{equation*}
  \forall \sigma, \tau, \mu \in S_n \enspace \sigma (\tau \mu) = (\sigma \tau) \mu
\end{equation*}

The set $S_n$ also has an identity element\sidenote{
  \begin{ex}
    Verify that the given identity element is indeed the identity, i.e.
    \begin{equation*}
      \forall \sigma \in S_n \enspace \sigma \epsilon = \sigma = \epsilon \sigma.
    \end{equation*}
  \end{ex}
}, namely
\begin{equation*}
  \epsilon = \begin{pmatrix}
    1 & 2 & \hdots & n \\
    1 & 2 & \hdots & n
  \end{pmatrix}
\end{equation*}

Finally, $\forall \sigma \in S_n$, since $\sigma$ is a bijection, we have that its inverse function, $\sigma^-1$ is also a bijection, and thus satisfies the requirements to be in $S_n$. We call $\sigma^{-1} \in S_n$ to be the \hldefn{inverse permutation} of $\sigma$, such that
\begin{equation*}
  \forall x, y \in \{1, ..., n\} \quad \sigma^{-1}(x) = y \iff \sigma(y) = x.
\end{equation*}
It follows, immediately, that
\begin{equation*}
  \sigma \big( \sigma^{-1}(x) \big) = x \, \land \, \sigma^{-1} \big( \sigma(y) \big) = y.
\end{equation*}
$\therefore$ We have that 
\begin{equation*}
  \sigma \sigma^{-1} = \epsilon = \sigma^{-1} \sigma.
\end{equation*}

\begin{eg}
  \label{eg:inverse_permutation}
  Find the inverse of
  \begin{equation*}
    \sigma = \begin{pmatrix}
      1 & 2 & 3 & 4 & 5 \\
      4 & 5 & 1 & 2 7 3
    \end{pmatrix}
  \end{equation*}

  \begin{solution}
    By rearranging the image in ascending order, using them now as the object and their respective objects as their image, construct
    \begin{equation*}
      \tau = \begin{pmatrix}
        1 & 2 & 3 & 4 & 5 \\
        3 & 4 & 5 & 1 & 2
      \end{pmatrix}.
    \end{equation*}
    It can easily (although perhaps not so prettily) be shown that
    \begin{equation*}
      \sigma \tau = \epsilon = \tau \sigma.
    \end{equation*}
  \end{solution}
\end{eg}

With all the above, we have for ourselves the following proposition:

\begin{propo}[Properties of $S_n$\index{Properties of $S_n$}]\label{propo:properties_of_Sn}
  We have
  \begin{enumerate}
    \item $\forall \sigma, \tau \in S_n \enspace \sigma \tau, \tau \sigma \in S_n$.
    \item $\forall \sigma, \tau, \mu \in S_n \enspace \sigma (\tau \mu) = (\sigma \tau) \mu$.
    \item $\exists \epsilon \in S_n \enspace \forall \sigma \in S_n \enspace \sigma \epsilon = \sigma = \epsilon \sigma$.
    \item $\forall \sigma \in S_n \enspace \exists! \sigma^{-1} \in S_n \enspace \sigma \sigma^{-1} = \epsilon = \sigma^{-1} \sigma$.
  \end{enumerate}
\end{propo}

\newthought{Consider}
\begin{equation*}
  \sigma = \begin{pmatrix}
    1 & 2 & 3 & 4 & 5 & 6 & 7 & 8 & 9 & 10 \\
    3 & 1 & 7 & 6 & 9 & 4 & 2 & 5 & 8 & 10
  \end{pmatrix} \in S_{10}
\end{equation*}
If we represent the action of $\sigma$ geometrically, we get

\begin{tabular}{c c c}
\begin{tikzcd}
 & 1 \arrow[rd] &  \\
2 \arrow[ru] &  & 3 \arrow[ld] \\
 & 7 \arrow[lu] & 
\end{tikzcd}
&
\begin{tikzcd}
4 \arrow[dd, bend left] \\
 \\
6 \arrow[uu, bend left]
\end{tikzcd}
&
\begin{tikzcd}
 & 5 \arrow[rdd] &  \\
 &  &  \\
8 \arrow[ruu] &  & 9 \arrow[ll]
\end{tikzcd}
\\ &
\begin{tikzcd}
  10 \arrow[loop below]
\end{tikzcd}
\end{tabular}
We observe that $\sigma$ can be \hlnotea{decomposed} into one $4$-cycle, $\begin{pmatrix} 1 & 3 & 7 & 2 \end{pmatrix}$, one $2$-cycle, $\begin{pmatrix} 4 & 6 \end{pmatrix}$, one $3$-cycle, $\begin{pmatrix} 5 & 9 & 8 \end{pmatrix}$, and one $1$-cycle, $\begin{pmatrix} 10 \end{pmatrix}$.

Note that these cycles are (pairwise) \hlnotea{disjoint}, and we can write\sidenote{We generally do not include the $1$-cycle and assume that by excluding them, it is known that any number that is supposed to appear loops back to themselves.}
\begin{equation*}
  \sigma = \begin{pmatrix} 1 & 3 & 7 & 2 \end{pmatrix}\begin{pmatrix} 4 & 6 \end{pmatrix}\begin{pmatrix} 5 & 9 & 8 \end{pmatrix}
\end{equation*}
Note that we may also write
\begin{align*}
  \sigma &= \begin{pmatrix} 4 & 6 \end{pmatrix}\begin{pmatrix} 5 & 9 & 8 \end{pmatrix}\begin{pmatrix} 1 & 3 & 7 & 2 \end{pmatrix} \\
    &= \begin{pmatrix} 6 & 4 \end{pmatrix}\begin{pmatrix} 9 & 8 & 5 \end{pmatrix}\begin{pmatrix} 7 & 2 & 1 & 3 \end{pmatrix}
\end{align*}
It is interesting to note that the cycles can rotate their ``elements'' in a \hlnotea{cyclic} manner, i.e.
\begin{gather*}
  \begin{pmatrix} 1 & 3 & 7 & 2 \end{pmatrix} = \begin{pmatrix} 7 & 2 & 1 & 3 \end{pmatrix} \neq \begin{pmatrix} 1 & 2 & 7 & 3 \end{pmatrix}.
\end{gather*}
Although the decomposition of the cycle notation is not unique (i.e. you may rearrange them), each individual cycle is unique, and is proven below\sidenote{See bonus question of A1. Proof will be included in the notes once the assignment is over.}.

\begin{thm}[Cycle Decomposition Theorem\index{Cycle Decomposition Theorem}]\label{thm:cycle_decomposition_theorem}
  If $\sigma \in S_n$, $\sigma \neq \epsilon$, then $\sigma$ is a product of (one or more) disjoint cycles of length at least $2$. This factorization is unique up to the order of the factors.
\end{thm}

\begin{note}[Convention]
 Every permutation in $S_n$ can be regarded as a permutation of $S_{n + 1}$ by fixing the permutation of $n + 1$. Therefore, we have that
 \begin{equation*}
   S_1 \subseteq S_2 \subseteq \hdots \subseteq S_n \subseteq S_{n + 1} \subseteq \hdots
 \end{equation*}
\end{note}

% subsection permutations (end)

% section introduction_continued (end)

% chapter lecture_2_may_04th_2018 (end)

\nobibliography*
\bibliography{bibliography}

\printindex
\end{document}