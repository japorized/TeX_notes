\documentclass[notoc,notitlepage]{tufte-book}
\setcounter{secnumdepth}{3}
\setcounter{tocdepth}{3}

\renewcommand{\baselinestretch}{1.1}

\usepackage{classnotetitle}

\title{PMATH347S18 - Groups \& Rings}
\author{Johnson Ng}
\subtitle{Classnotes for Spring 2018}
\credentials{BMath (Hons), Pure Mathematics major, Actuarial Science Minor}
\institution{University of Waterloo}

\usepackage{tikz-cd}
\setcounter{secnumdepth}{3}
\setcounter{tocdepth}{3}

\renewcommand{\baselinestretch}{1.1}
\usepackage{geometry}
\geometry{letterpaper}
\usepackage[parfill]{parskip}
\usepackage{graphicx}

% Essential Packages
\usepackage{makeidx}
\makeindex
\usepackage{enumitem}
\usepackage[T1]{fontenc}
\usepackage{natbib}
\bibliographystyle{apalike}
\usepackage{ragged2e}
\usepackage{etoolbox}
\usepackage{amssymb}
\usepackage{fontawesome}
\usepackage{amsmath}
\usepackage{mathrsfs}
\usepackage{mathtools}
\usepackage{xparse}
\usepackage{tkz-euclide}
\usetkzobj{all}
\usepackage[utf8]{inputenc}
\usepackage{csquotes}
\usepackage[english]{babel}
\usepackage{marvosym}
\usepackage{pgf,tikz}
\usepackage{pgfplots}
\usepackage{fancyhdr}
\usepackage{array}
\usepackage{faktor}
\usepackage{float}
\usepackage{xcolor}
\usepackage{centernot}
\usepackage{silence}
  \WarningFilter*{latex}{Marginpar on page \thepage\space moved}
\usepackage{tcolorbox}
\tcbuselibrary{skins,breakable}
\usepackage{longtable}
\usepackage[amsmath,hyperref]{ntheorem}
\usepackage{hyperref}
\usepackage[noabbrev,capitalize,nameinlink]{cleveref}

% xcolor (scheme: base16 eighties)
\definecolor{base16-eighties-dark}{HTML}{2D2D2D}
\definecolor{base16-eighties-light}{HTML}{D3D0C8}
\definecolor{base16-eighties-magenta}{HTML}{CD98CD}
\definecolor{base16-eighties-red}{HTML}{F47678}
\definecolor{base16-eighties-yellow}{HTML}{E2B552}
\definecolor{base16-eighties-green}{HTML}{98CD97}
\definecolor{base16-eighties-lightblue}{HTML}{61CCCD}
\definecolor{base16-eighties-blue}{HTML}{6498CE}
\definecolor{base16-eighties-brown}{HTML}{D47B4E}
\definecolor{base16-eighties-gray}{HTML}{747369}

% hyperref Package Settings
\hypersetup{
    bookmarks=true,         % show bookmarks bar?
    unicode=true,          % non-Latin characters in Acrobat’s bookmarks
    pdftoolbar=false,        % show Acrobat’s toolbar?
    pdfmenubar=false,        % show Acrobat’s menu?
    pdffitwindow=true,     % window fit to page when opened
    colorlinks=true,
    allcolors=base16-eighties-magenta,
}

% tikz
\usepgfplotslibrary{polar}
\usepgflibrary{shapes.geometric}
\usetikzlibrary{angles,patterns,calc,decorations.markings}
\tikzset{midarrow/.style 2 args={
        decoration={markings,
            mark= at position #2 with {\arrow{#1}} ,
        },
        postaction={decorate}
    },
    midarrow/.default={latex}{0.5}
}
\def\centerarc[#1](#2)(#3:#4:#5)% Syntax: [draw options] (center) (initial angle:final angle:radius)
    { \draw[#1] ($(#2)+({#5*cos(#3)},{#5*sin(#3)})$) arc (#3:#4:#5); }

% enumitems
\newlist{inlinelist}{enumerate*}{1}
\setlist*[inlinelist,1]{%
  label=(\roman*),
}

% Theorem Style Customization
\setlength\theorempreskipamount{2ex}
\setlength\theorempostskipamount{3ex}

\makeatletter
\let\nobreakitem\item
\let\@nobreakitem\@item
\patchcmd{\nobreakitem}{\@item}{\@nobreakitem}{}{}
\patchcmd{\nobreakitem}{\@item}{\@nobreakitem}{}{}
\patchcmd{\@nobreakitem}{\@itempenalty}{\@M}{}{}
\patchcmd{\@xthm}{\ignorespaces}{\nobreak\ignorespaces}{}{}
\patchcmd{\@ythm}{\ignorespaces}{\nobreak\ignorespaces}{}{}

\renewtheoremstyle{break}%
  {\item{\theorem@headerfont
          ##1\ ##2\theorem@separator}\hskip\labelsep\relax\nobreakitem}%
  {\item{\theorem@headerfont
          ##1\ ##2\ (##3)\theorem@separator}\hskip\labelsep\relax\nobreakitem}
\makeatother

% ntheorem + framed
\makeatletter

% ntheorem Declarations
\theorempreskip{10pt}
\theorempostskip{5pt}
\theoremstyle{break}

\newtheorem*{solution}{\faPencil $\enspace$ Solution}
\newtheorem*{remark}{Remark}
\newtheorem{eg}{Example}[section]
\newtheorem{ex}{Exercise}[section]

    % definition env
\theoremprework{\textcolor{base16-eighties-blue}{\hrule height 2pt}}
\theoremheaderfont{\color{base16-eighties-blue}\normalfont\bfseries}
\theorempostwork{\textcolor{base16-eighties-blue}{\hrule height 2pt}}
\theoremindent10pt
\newtheorem{defn}{\faBook \enspace Definition}

    % definition env no num
\theoremprework{\textcolor{base16-eighties-blue}{\hrule height 2pt}}
\theoremheaderfont{\color{base16-eighties-blue}\normalfont\bfseries}
\theorempostwork{\textcolor{base16-eighties-blue}{\hrule height 2pt}}
\theoremindent10pt
\newtheorem*{defnnonum}{\faBook \enspace Definition}

    % theorem envs
\theoremprework{\textcolor{base16-eighties-magenta}{\hrule height 2pt}}
\theoremheaderfont{\color{base16-eighties-magenta}\normalfont\bfseries}
\theorempostwork{\textcolor{base16-eighties-magenta}{\hrule height 2pt}}
\theoremindent10pt
\newtheorem{thm}{\faCoffee \enspace Theorem}

\theoremprework{\textcolor{base16-eighties-magenta}{\hrule height 2pt}}
\theorempostwork{\textcolor{base16-eighties-magenta}{\hrule height 2pt}}
\theoremindent10pt
\newtheorem{propo}[thm]{\faTint \enspace Proposition}

\theoremprework{\textcolor{base16-eighties-magenta}{\hrule height 2pt}}
\theorempostwork{\textcolor{base16-eighties-magenta}{\hrule height 2pt}}
\theoremindent10pt
\newtheorem{crly}[thm]{\faSpaceShuttle \enspace Corollary}

\theoremprework{\textcolor{base16-eighties-magenta}{\hrule height 2pt}}
\theorempostwork{\textcolor{base16-eighties-magenta}{\hrule height 2pt}}
\theoremindent10pt
\newtheorem{lemma}[thm]{\faTree \enspace Lemma}

\theoremprework{\textcolor{base16-eighties-magenta}{\hrule height 2pt}}
\theorempostwork{\textcolor{base16-eighties-magenta}{\hrule height 2pt}}
\theoremindent10pt
\newtheorem{axiom}[thm]{\faShield \enspace Axiom}

    % theorem envs without counter
\theoremprework{\textcolor{base16-eighties-magenta}{\hrule height 2pt}}
\theoremheaderfont{\color{base16-eighties-magenta}\normalfont\bfseries}
\theorempostwork{\textcolor{base16-eighties-magenta}{\hrule height 2pt}}
\theoremindent10pt
\newtheorem*{thmnonum}{\faCoffee \enspace Theorem}

\theoremprework{\textcolor{base16-eighties-magenta}{\hrule height 2pt}}
\theorempostwork{\textcolor{base16-eighties-magenta}{\hrule height 2pt}}
\theoremindent10pt
\newtheorem*{propononum}{\faTint \enspace Proposition}

\theoremprework{\textcolor{base16-eighties-magenta}{\hrule height 2pt}}
\theorempostwork{\textcolor{base16-eighties-magenta}{\hrule height 2pt}}
\theoremindent10pt
\newtheorem*{crlynonum}{\faSpaceShuttle \enspace Corollary}

\theoremprework{\textcolor{base16-eighties-magenta}{\hrule height 2pt}}
\theorempostwork{\textcolor{base16-eighties-magenta}{\hrule height 2pt}}
\theoremindent10pt
\newtheorem*{lemmanonum}{\faTree \enspace Lemma}

\theoremprework{\textcolor{base16-eighties-magenta}{\hrule height 2pt}}
\theorempostwork{\textcolor{base16-eighties-magenta}{\hrule height 2pt}}
\theoremindent10pt
\newtheorem*{axiomnonum}{\faShield \enspace Axiom}

    % proof env
\theoremprework{\textcolor{base16-eighties-brown}{\hrule height 2pt}}
\theoremheaderfont{\color{base16-eighties-brown}\normalfont\bfseries}
\theorempostwork{\textcolor{base16-eighties-brown}{\hrule height 2pt}}
\newtheorem*{proof}{\faPencil \enspace Proof}

    % note and notation env
\theoremprework{\textcolor{base16-eighties-yellow}{\hrule height 2pt}}
\theoremheaderfont{\color{base16-eighties-yellow}\normalfont\bfseries}
\theorempostwork{\textcolor{base16-eighties-yellow}{\hrule height 2pt}}
\newtheorem*{note}{\faQuoteLeft \enspace Note}

\theoremprework{\textcolor{base16-eighties-yellow}{\hrule height 2pt}}
\theorempostwork{\textcolor{base16-eighties-yellow}{\hrule height 2pt}}
\newtheorem*{notation}{\faPaw \enspace Notation}

    % warning env
\theoremprework{\textcolor{base16-eighties-red}{\hrule height 2pt}}
\theoremheaderfont{\color{base16-eighties-red}\normalfont\bfseries}
\theorempostwork{\textcolor{base16-eighties-red}{\hrule height 2pt}}
\theoremindent10pt
\newtheorem*{warning}{\faBug \enspace Warning}

% more environments
\newtcolorbox{redquote}{
  blanker,enhanced,breakable,standard jigsaw,
  opacityback=0,
  coltext=base16-eighties-light,
  left=5mm,right=5mm,top=2mm,bottom=2mm,
  colframe=base16-eighties-red,
  boxrule=0pt,leftrule=3pt,
  fontupper=\itshape
}
\newtcolorbox{bluequote}{
  blanker,enhanced,breakable,standard jigsaw,
  opacityback=0,
  coltext=base16-eighties-light,
  left=5mm,right=5mm,top=2mm,bottom=2mm,
  colframe=base16-eighties-blue,
  boxrule=0pt,leftrule=3pt,
  fontupper=\itshape
}
\newtcolorbox{greenquote}{
  blanker,enhanced,breakable,standard jigsaw,
  opacityback=0,
  coltext=base16-eighties-light,
  left=5mm,right=5mm,top=2mm,bottom=2mm,
  colframe=base16-eighties-green,
  boxrule=0pt,leftrule=3pt,
  fontupper=\itshape
}
\newtcolorbox{yellowquote}{
  blanker,enhanced,breakable,standard jigsaw,
  opacityback=0,
  coltext=base16-eighties-light,
  left=5mm,right=5mm,top=2mm,bottom=2mm,
  colframe=base16-eighties-yellow,
  boxrule=0pt,leftrule=3pt,
  fontupper=\itshape
}
\newtcolorbox{magentaquote}{
  blanker,enhanced,breakable,standard jigsaw,
  opacityback=0,
  coltext=base16-eighties-light,
  left=5mm,right=5mm,top=2mm,bottom=2mm,
  colframe=base16-eighties-magenta,
  boxrule=0pt,leftrule=3pt,
  fontupper=\itshape
}

% ntheorem listtheorem style
\makeatother
\newlength\widesttheorem
\AtBeginDocument{
  \settowidth{\widesttheorem}{Proposition A.1.1.1\quad}
}

\makeatletter
\def\thm@@thmline@name#1#2#3#4{%
        \@dottedtocline{-2}{0em}{2.3em}%
                   {\makebox[\widesttheorem][l]{#1 \protect\numberline{#2}}#3}%
                   {#4}}
\@ifpackageloaded{hyperref}{
\def\thm@@thmline@name#1#2#3#4#5{%
    \ifx\#5\%
        \@dottedtocline{-2}{0em}{2.3em}%
            {\makebox[\widesttheorem][l]{#1 \protect\numberline{#2}}#3}%
            {#4}
    \else
        \ifHy@linktocpage\relax\relax
            \@dottedtocline{-2}{0em}{2.3em}%
                {\makebox[\widesttheorem][l]{#1 \protect\numberline{#2}}#3}%
                {\hyper@linkstart{link}{#5}{#4}\hyper@linkend}%
        \else
            \@dottedtocline{-2}{0em}{2.3em}%
                {\hyper@linkstart{link}{#5}%
                  {\makebox[\widesttheorem][l]{#1 \protect\numberline{#2}}#3}\hyper@linkend}%
                    {#4}%
        \fi
    \fi}
}

\makeatletter
\def\thm@@thmline@noname#1#2#3#4{%
        \@dottedtocline{-2}{0em}{5em}%
                   {{\protect\numberline{#2}}#3}%
                   {#4}}
\@ifpackageloaded{hyperref}{
\def\thm@@thmline@noname#1#2#3#4#5{%
    \ifx\#5\%
        \@dottedtocline{-2}{0em}{5em}%
            {{\protect\numberline{#2}}#3}%
            {#4}
    \else
        \ifHy@linktocpage\relax\relax
            \@dottedtocline{-2}{0em}{5em}%
                {{\protect\numberline{#2}}#3}%
                {\hyper@linkstart{link}{#5}{#4}\hyper@linkend}%
        \else
            \@dottedtocline{-2}{0em}{5em}%
                {\hyper@linkstart{link}{#5}%
                  {{\protect\numberline{#2}}#3}\hyper@linkend}%
                    {#4}%
        \fi
    \fi}
}

\theoremlisttype{allname}

\AtBeginDocument{\renewcommand\contentsname{Table of Contents}}

% Heading formattings
% chapter format
\titleformat{\chapter}%
  {\huge\rmfamily\itshape\color{base16-eighties-magenta}}% format applied to label+text
  {\llap{\colorbox{base16-eighties-magenta}{\parbox{1.5cm}{\hfill\itshape\huge\textcolor{base16-eighties-dark}{\thechapter}}}}}% label
  {5pt}% horizontal separation between label and title body
  {}% before the title body
  []% after the title body

% section format
\titleformat{\section}%
  {\normalfont\Large\rmfamily\itshape\color{base16-eighties-blue}}% format applied to label+text
  {\llap{\colorbox{base16-eighties-blue}{\parbox{1.5cm}{\hfill\itshape\textcolor{base16-eighties-dark}{\thesection}}}}}% label
  {5pt}% horizontal separation between label and title body
  {}% before the title body
  []% after the title body

% subsection format
\titleformat{\subsection}%
  {\normalfont\large\itshape\color{base16-eighties-green}}% format applied to label+text
  {\llap{\colorbox{base16-eighties-green}{\parbox{1.5cm}{\hfill\textcolor{base16-eighties-dark}{\thesubsection}}}}}% label
  {1em}% horizontal separation between label and title body
  {}% before the title body
  []% after the title body

% Sidenote enhancements
\def\mathmarginnote#1{%
  \tag*{\rlap{\hspace\marginparsep\smash{\parbox[t]{\marginparwidth}{%
  \footnotesize#1}}}}
}

% Custom table columning
\newcolumntype{L}[1]{>{\raggedright\let\newline\\\arraybackslash\hspace{0pt}}m{#1}}
\newcolumntype{C}[1]{>{\centering\let\newline\\\arraybackslash\hspace{0pt}}m{#1}}
\newcolumntype{R}[1]{>{\raggedleft\let\newline\\\arraybackslash\hspace{0pt}}m{#1}}

% Custom math operator
% \DeclareMathOperator{\rem}{rem}
\DeclareMathOperator*{\argmax}{arg\,max}
\DeclareMathOperator*{\argmin}{arg\,min}
\DeclareMathOperator{\re}{Re}
\DeclareMathOperator{\im}{Im}
\DeclareMathOperator{\caparg}{Arg}
\DeclareMathOperator{\Ind}{Ind}
\DeclareMathOperator{\Res}{Res}

% Graph styles
\pgfplotsset{compat=1.15}
\usepgfplotslibrary{fillbetween}
\pgfplotsset{four quads/.append style={axis x line=middle, axis y line=
middle, xlabel={$x$}, ylabel={$y$}, axis equal }}
\pgfplotsset{four quad complex/.append style={axis x line=middle, axis y line=
middle, xlabel={$\re$}, ylabel={$\im$}, axis equal }}

% Shortcuts
\newcommand{\floor}[1]{\lfloor #1 \rfloor}      % simplifying the writing of a floor function
\newcommand{\ceiling}[1]{\lceil #1 \rceil}      % simplifying the writing of a ceiling function
\newcommand{\dotp}{\, \cdotp}			        % dot product to distinguish from \cdot
\newcommand{\qed}{\hfill\ensuremath{\square}}   % Q.E.D sign
\newcommand{\abs}[1]{\left|#1\right|}						% absolute value
\newcommand{\lra}[1]{\langle \; #1 \; \rangle}
\newcommand{\at}[2]{\Big|_{#1}^{#2}}
\newcommand{\Arg}[1]{\caparg #1}
\renewcommand{\bar}[1]{\mkern 1.5mu \overline{\mkern -1.5mu #1 \mkern -1.5mu} \mkern 1.5mu}
\newcommand{\quotient}[2]{\faktor{#1}{#2}}
\newcommand{\cyclic}[1]{\left\langle #1 \right\rangle}
	% highlighting shortcuts
\newcommand{\hlimpo}[1]{\textcolor{base16-eighties-red}{\textbf{#1}}}
\newcommand{\hlwarn}[1]{\textcolor{base16-eighties-yellow}{\textbf{#1}}}
\newcommand{\hldefn}[1]{\textcolor{base16-eighties-blue}{\index{#1}\textbf{#1}}}
\newcommand{\hlnotea}[1]{\textcolor{base16-eighties-green}{\textbf{#1}}}
\newcommand{\hlnoteb}[1]{\textcolor{base16-eighties-lightblue}{\textbf{#1}}}
\newcommand{\hlnotec}[1]{\textcolor{base16-eighties-brown}{\textbf{#1}}}
\newcommand{\WTP}{\textcolor{base16-eighties-brown}{WTP} }
\newcommand{\WTS}{\textcolor{base16-eighties-brown}{WTS} }
\newcommand{\ind}[2]{\Ind_{#2}\left( #1 \right)}
\newcommand{\notimply}{\centernot\implies}
\newcommand{\res}[2]{\underset{#2}{\Res} #1 }
\newcommand{\tworow}[3]{\begin{tabular}{@{}#1@{}} #2 \\ #3 \end{tabular}}
\renewcommand{\epsilon}{\varepsilon}
\newcommand{\lrarrow}{\leftrightarrow}
\newcommand{\larrow}{\leftarrow}
\newcommand{\rarrow}{\rightarrow}
\renewcommand{\atop}[2]{\genfrac{}{}{0pt}{}{#1}{#2}}
\newcommand*\dif{\mathop{}\!d}

  % inspiration from: https://tex.stackexchange.com/questions/8720/overbrace-underbrace-but-with-an-arrow-instead#37758
\newcommand{\overarrow}[2]{
  \overset{\makebox[0pt]{\begin{tabular}{@{}c@{}}#2\\[0pt]\ensuremath{\uparrow}\end{tabular}}}{#1}
}
\newcommand{\underarrow}[2]{
  \underset{\makebox[0pt]{\begin{tabular}{@{}c@{}}\downarrow\\[0pt]\ensuremath{#2}\end{tabular}}}{#1}
}

% Document header formatting
\renewcommand{\chaptermark}[1]{\markboth{#1}{}}
\renewcommand{\sectionmark}[1]{\markright{#1}}
\makeatletter
\pagestyle{fancy}
\fancyhead{}
\fancyhead[RO]{\textsl{\@title} \enspace \thepage}
\fancyhead[LE]{\thepage \enspace \textsl{\leftmark \enspace - \enspace \rightmark}}
\makeatother

% Comment the two lines below if you want to print the document
\pagecolor{base16-eighties-dark}
\color{base16-eighties-light}


\DeclareMathOperator{\img}{im }
\DeclareMathOperator{\sign}{sgn}
\DeclareMathOperator{\ch}{ch }


\pagecolor{white}
\color{base16-eighties-dark}

\hypersetup{
	colorlinks=false,
  allbordercolors=base16-eighties-magenta,
  pdfborderstyle={/S/U/W 1}
}

\begin{document}
\hypersetup{pageanchor=false}
\maketitle
\hypersetup{pageanchor=true}
\tableofcontents

\chapter*{\faBook List of Definitions}
\theoremlisttype{all}
\listtheorems{defn}

\chapter*{\faCoffee List of Theorems}
\theoremlisttype{allname}
\listtheorems{axiom,lemma,thm,crly,propo}

\chapter*{Foreword}%
% chapter 

\section*{Usage}%
% section 

\begin{itemize}
  \item Notes are presented in two columns: main notes on the left, and sidenotes on the right. Main notes will have a larger margin.
  \item The following is the color code for the notes: \\
    \begin{tabular}{l l}
    \color{base16-eighties-blue}{Blue} & Definitions \\
    \hlimpo{Red}                       & Important points \\
    \hlwarn{Yellow}                    & Points to watch out for / comment for incompletion \\
    \hlnotea{Green}                    & External definitions, theorems, etc. \\
    \hlnoteb{Light Blue}               & Regular highlighting \\
    \hlnotec{Brown}                    & Secondary highlighting
    \end{tabular}
  \item The following is the color code for boxes, that begin and end with a line of the same color: \\
    \begin{tabular}{l l}
    \color{base16-eighties-blue}{Blue}       & Definitions \\
    \hlimpo{Red}                             & Warning \\
    \hlwarn{Yellow}                          & Notes, remarks, etc. \\
    \hlnotec{Brown}                          & Proofs \\
    \textcolor{base16-eighties-magenta}{Magenta} & Theorems, Propositions, Lemmas, etc.
    \end{tabular}
  \item Hyperlinks are underlined in \textcolor{base16-eighties-magenta}{magenta}. If your PDF reader supports it, you can follow the links to either be redirected to an external website, or a theorem, definition, etc., in the same document. Note that this is only reliable if you have the full set of notes as a single document, which you can find on: \\ \url{https://japorized.github.io/TeX_notes}
\end{itemize}

\chapter{Lecture 1 May 02nd 2018}
  \label{chapter:lecture_1_may_02nd_2018}

\section{Introduction} % (fold)
\label{sec:introduction}

\subsection{Numbers} % (fold)
\label{sub:numbers}

The following are some of the number sets that we are already familiar with:
\begin{gather*}
  \mathbb{N} = \{1, 2, 3, ...\} \qquad \mathbb{Z} = \{.., -2, -1, 0, 1, 2, ...\} \\
  \mathbb{Q} = \left\{\frac{a}{b} : a \in \mathbb{Z}, b \in \mathbb{N} \right\} \qquad \mathbb{R} = \text{ set of real numbers} \\
  \mathbb{C} = \{a + bi : a, b \in \mathbb{R}, i = \sqrt{-1} \} = \text{ set of complex numbers} 
\end{gather*}
For $n \in \mathbb{Z}$, let $\mathbb{Z}_n$ denote the set of integers modulo $n$, i.e.
\begin{equation*}
  \mathbb{Z}_n = \{ [0], [1], ..., [n - 1] \}
\end{equation*}
where the $[r]$, $0 \leq r \leq n - 1$, are the congruence classes, i.e.
\begin{equation*}
  [r] = \{z \in \mathbb{Z} : z \equiv r \mod n\}
\end{equation*}

These sets share some common properties, e.g. $+$ and $\times$. Let's try to break that down to make further observation.

\newthought{Note that} for $R = \mathbb{N}, \, \mathbb{Z}, \, \mathbb{Q}, \, \mathbb{R}, \, \mathbb{C},$ or $\mathbb{Z}_n$, $R$ has 2 operations, i.e. addition and multiplication.

\paragraph{Addition} If $r_1, r_2, r_3 \in R$, then
\begin{itemize}
  \item (\hldefn{closure}) $r_1 + r_2 \in R$
  \item (\hldefn{associativity}) $r_1 + (r_2 + r_3) = (r_1 + r_2) + r_3$
\end{itemize}
Also, if $R \neq \mathbb{N}$, then $\exists 0 \in R$ (the \hldefn{additive identity}) such that
\begin{equation*}
  \forall r \in R \quad r + 0 = r = 0 + r.
\end{equation*}
Also, $\forall r \in R$, $\exists (-r) \in R$ such that
\begin{equation*}
  r + (-r) = 0 = (-r) + r.
\end{equation*}

\paragraph{Multiplication} For $r_1, r_2, r_3 \in R$, we have
\begin{itemize}
  \item (\hlnoteb{closure}) $r_1 r_2 \in R$
  \item (\hlnoteb{associativity}) $r_1 (r_2 r_3) = (r_1 r_2) r_3$
\end{itemize}
Also, $\exists 1 \in R$ (a.k.a the \hldefn{mutiplicative identity}), such that
\begin{equation*}
  \forall r \in R \quad r \cdot 1 = r = 1 \cdot r.
\end{equation*}
Finally, for $R = \mathbb{Q}, \, \mathbb{R},$ or $\mathbb{C}$, $\forall r \in R, \, \exists r^{-1} \in R$ such that
\begin{equation*}
  r \cdot r^{-1} = 1 = r^{-1} \cdot r.
\end{equation*}
Note that for $R = \mathbb{Z}_n$, where $n \in \mathbb{Z}$, not all $[r] \in \mathbb{Z}_n$ have a multiplicative inverse. For example, for $[2] \in \mathbb{Z}_4$, there is no $[x] \in \mathbb{Z}_4$ such that $[2][x] = [1]$.\sidenote{This is best proven using techniques introduced in MATH135/145.}

% subsection numbers (end)

\subsection{Matrices}
  \label{sub:matrices}

For $n \in \mathbb{N} \setminus \{1\}$, an $n \times n$ matrix over $\mathbb{R}$ \sidenote{$\mathbb{R}$ can be replaced by $\mathbb{Q}$ or $\mathbb{C}$.} is an $n \times n$ array that can be expressed as follows:
\begin{equation*}
  A = [a_{ij}] = \begin{bmatrix}
    a_{11} & a_{12} & \hdots & a_{1n} \\
    a_{21} & a_{22} & \hdots & a_{2n} \\
    \vdots & \vdots &        & \vdots \\
    a_{n1} & a_{n2} & \hdots & a_{nn}
  \end{bmatrix}
\end{equation*}
where for $1 \leq i, j \leq n$, $a_{ij} \in \mathbb{R}$. We denote $M_n(\mathbb{R})$ as the set of all $n \times n$ matrices over $\mathbb{R}$.

As in \cref{sub:numbers}, we can perform \hlnotea{addition and multiplication} on $M_n(\mathbb{R})$.

\paragraph{Matrix Addition} Given $A = [a_{ij}], B = [b_{ij}], C = [c_{ij}] \in M_n(\mathbb{R})$, we define matrix addition as
\begin{equation*}
  A + B = [a_{ij} + b_{ij}],
\end{equation*}
which immediately gives the \hlnoteb{closure property}, since $a_{ij} + b_{ij} \in \mathbb{R}$ and hence $A + B \in M_n(\mathbb{R})$. Also, by this definition, we also immediately obtain the \hlnoteb{associativity property}, i.e.
\begin{equation*}
  A + (B + C) = (A + B) + C.
\end{equation*}
We define the zero matrix as
\begin{equation*}
  0 = \begin{bmatrix}
    0      &   0    & \hdots &   0 \\
    0      &   0    & \hdots &   0 \\
    \vdots & \vdots &        & \vdots \\
    0      &   0    & \hdots &   0
  \end{bmatrix}.
\end{equation*}
Then we have that $0$ is the \hlnoteb{additive identity}, i.e.
\begin{equation*}
  A + 0 = A = 0 + A.
\end{equation*}
Finally, $\forall A \in M_n(\mathbb{R})$, $\exists (-A) \in M_n(\mathbb{R})$ (the \hlnoteb{additive inverse}) such that
\begin{equation*}
  A + (-A) = 0 - (-A) + A.
\end{equation*}

Note that in this case, we also have that that the operation is \hlnoteb{commutative}, i.e.
\begin{equation*}
  A + B = B + A.
\end{equation*}

\paragraph{Matrix Multiplication} Given $A = [a_{ij}], B = [b_{ij}], C = [c_{ij}] \in M_n(\mathbb{R})$, we define the matrix multiplication as
\begin{equation*}
  AB = [d_{ij}] \text{ where } c_{ij} = \sum_{k=1}^{n} a_{ik} b_{kj} \in \mathbb{R}.
\end{equation*}
Clearly, $AB \in M_n(\mathbb{R})$, i.e. it is \hlnoteb{closed under matrix multiplication}. Also, we have that, under such a defintion, matrix multiplication is \hlnoteb{associative}, i.e.
\begin{equation*}
  A(BC) = (AB)C.
\end{equation*}
Define the identity matrix, $I \in M_n(\mathbb{R})$, as follows:
\begin{equation*}
  I = \begin{bmatrix}
    1      &   0    & \hdots & 0 \\
    0      &   1    & \hdots & 0 \\
    \vdots & \vdots &        & \vdots \\
    0      &   0    & \hdots & 1
  \end{bmatrix}.
\end{equation*}
Then we have that $I$ is the \hlnoteb{multiplicative identity}, since
\begin{equation*}
  AI = A = IA.
\end{equation*}
However, contrary to matrix addition, $\forall A \in M_n(\mathbb{R})$, it is not always true that $\exists A^{-1} \in M_n(\mathbb{R})$ such that\marginnote{This is especially true if the \hlnotea{determinant} of $A$ is $0$.}
\begin{equation*}
  AA^{-1} = I = A^{-1} A.
\end{equation*}

Also, we can always find some $A, B \in M_n(\mathbb{R})$ such that
\begin{equation*}
  AB \neq BA,
\end{equation*}
i.e. matrix multiplication is not always commutative.

\newthought{The common properties} of the operations from above: \hlimpo{closure, associativity, and existence of an inverse}, are not unique to just addition and multiplication. We shall see in the next lecture that there are other operations where these properties will continue to hold, e.g. \hlnoteb{permutations}.

% subsection matrices (end)

% section introduction (end)

% chapter lecture_1_may_02nd_2018 (end)

\chapter{Lecture 2 May 04th 2018}
  \label{chapter:lecture_2_may_04th_2018}

\section{Introduction (Continued)} % (fold)
\label{sec:introduction_continued}

\subsection{Permutations} % (fold)
\label{sub:permutations}

\begin{defn}[Injectivity]\label{defn:injectivity}
\index{Injectivity}
  Let $f: X \to Y$ be a function. We say that $f$ is \hlnoteb{injective} (or \hldefn{one-to-one}) if $f(x_1) = f(x_2)$ implies $x_1 = x_2$.
\end{defn}

\begin{defn}[Surjectivity]\label{defn:surjectivity}
\index{Surjectivity}
  Let $f: X \to Y$ be a function. We say that $f$ is \hlnoteb{surjective} (or \hldefn{onto}) if $\forall y \in Y \enspace \exists x \in X \enspace f(x) = y$.
\end{defn}

\begin{defn}[Bijectivity]\label{defn:bijectivity}
\index{Bijectivity}
  Let $f: X \to Y$ be a function. We say that $f$ is \hlnoteb{bijective} if it is both \hlnoteb{injective} and \hlnoteb{surjective}.
\end{defn}

\begin{defn}[Permutations]\label{defn:permutations}
\index{Permutations}
  Given a non-empty set $L$, a permutation of $L$ is a bijection from $L$ to $L$. The set of all permutations of $L$ is denoted by $S_L$.
\end{defn}

\begin{eg}
  \label{eg:permutations_first}
  Consider the set $L = \{1, 2, 3\}$, which has the following $6$ different permutations: \marginnote{\begin{note}
    \begin{equation*}
      \begin{pmatrix} 1 & 2 & 3 \\ 1 & 3 & 2 \end{pmatrix}
    \end{equation*}
    indicates the bijection $\sigma: \{1, 2, 3\} \to \{1, 2, 3\}$ with $\sigma(1) = 1$, $\sigma(2) = 3$ and $\sigma(3) = 2$.
  \end{note}}

  \begin{gather*}
     \begin{pmatrix} 1 & 2 & 3 \\ 1 & 2 & 3 \end{pmatrix} \quad \begin{pmatrix} 1 & 2 & 3 \\ 1 & 3 & 2 \end{pmatrix} \quad \begin{pmatrix} 1 & 2 & 3 \\ 2 & 1 & 3 \end{pmatrix} \\
     \begin{pmatrix} 1 & 2 & 3 \\ 2 & 3 & 1 \end{pmatrix} \quad \begin{pmatrix} 1 & 2 & 3 \\ 3 & 1 & 2 \end{pmatrix} \quad \begin{pmatrix} 1 & 2 & 3 \\ 3 & 2 & 1 \end{pmatrix}
   \end{gather*} 
\end{eg}

\newthought{For $n \in \mathbb{N}$}, we denote $S_n := S_{\{1, 2, ..., n\}}$, the set of all permutations of $\{1, 2, ..., n\}$. \cref{eg:permutations_first} shows the elements of the set $S_3$.

\begin{defn}[Order]\label{defn:order}
\index{Order}
  The \hlnoteb{order} of a set $A$, denoted by $\abs{A}$, is the cardinality of the set.
\end{defn}

\begin{eg}
  \label{eg:order_of_prev_eg}
  We have seen that the order of $S_3$, $\abs{S_3}$ is $6 = 3!$.
\end{eg}

\begin{propo}\label{propo:order_of_Sn_is_n}
  $\abs{S_n} = n!$
\end{propo}

\begin{proof}
  $\forall \sigma \in S_n$, there are $n$ choices for $\sigma(1)$, $n - 1$ choices for $\sigma(2)$, ..., $2$ choices for $\sigma(n - 1)$, and finally $1$ choice for $\sigma(n)$. \qed
\end{proof}

\paragraph{Do elements of $S_n$ share the same properties as what we've seen in the numbers?} Given $\sigma, \tau \in S_n$, we can \hlnotea{compose} the 2 together to get a third element in $S_n$, namely $\sigma \tau$ (wlog), where $\sigma \tau : \{1, ..., n\} \to \{1, ..., n\}$ is given by $\forall x \in \{1, ..., n\}$, $x \mapsto \sigma( \tau(x) )$.

It is important to note that $\because \sigma, \tau$ are \hlimpo{both bijective}, $\sigma \tau$ is also bijective. Thus, together with the fact that $\sigma \tau : \{1, ..., n\} \to \{1, ..., n\}$, we have that $\sigma \tau \in S_n$ by definition of $S_n$.

$\therefore \forall \sigma, \tau \in S_n, \; \sigma \tau, \tau \sigma \in S_n$, but $\sigma \tau \neq \tau \sigma$ in general. The following is an example of the stated case:

\begin{eg}
  \label{eg:commutativity_of_Sn}
  Let
  \begin{equation*}
    \sigma &= \begin{pmatrix}
      1 & 2 & 3 & 4 \\
      3 & 4 & 1 & 2
    \end{pmatrix}, \text{ and } 
    \tau &= \begin{pmatrix}
      1 & 2 & 3 & 4 \\
      2 & 4 & 3 & 1
    \end{pmatrix}.
  \end{equation*}
  Compute $\sigma \tau$ and $\tau \sigma$ to show that they are not equal.

  \begin{solution}
    \begin{equation*}
      \sigma \tau &= \begin{pmatrix}
        1 & 2 & 3 & 4 \\
        4 & 2 & 1 & 3
      \end{pmatrix} \text{ but } 
      \tau \sigma &= \begin{pmatrix}
        1 & 2 & 3 & 4 \\
        3 & 1 & 2 & 4
      \end{pmatrix}
    \end{equation*}
  \end{solution}
\end{eg}

Perhaps what is interesting is the question of: \textbf{when does commutativity occur?} One such case is when $\sigma$ and $\tau$ have support sets that are disjoint\sidenote{This is proven in A1}.

On the other hand, the associative property holds\sidenote{
  \begin{ex}
    Prove this as an exercise.
  \end{ex}
}, i.e.
\begin{equation*}
  \forall \sigma, \tau, \mu \in S_n \enspace \sigma (\tau \mu) = (\sigma \tau) \mu
\end{equation*}

The set $S_n$ also has an identity element\sidenote{
  \begin{ex}
    Verify that the given identity element is indeed the identity, i.e.
    \begin{equation*}
      \forall \sigma \in S_n \enspace \sigma \epsilon = \sigma = \epsilon \sigma.
    \end{equation*}
  \end{ex}
}, namely
\begin{equation*}
  \epsilon = \begin{pmatrix}
    1 & 2 & \hdots & n \\
    1 & 2 & \hdots & n
  \end{pmatrix}
\end{equation*}

Finally, $\forall \sigma \in S_n$, since $\sigma$ is a bijection, we have that its inverse function, $\sigma^-1$ is also a bijection, and thus satisfies the requirements to be in $S_n$. We call $\sigma^{-1} \in S_n$ to be the \hldefn{inverse permutation} of $\sigma$, such that
\begin{equation*}
  \forall x, y \in \{1, ..., n\} \quad \sigma^{-1}(x) = y \iff \sigma(y) = x.
\end{equation*}
It follows, immediately, that
\begin{equation*}
  \sigma \big( \sigma^{-1}(x) \big) = x \, \land \, \sigma^{-1} \big( \sigma(y) \big) = y.
\end{equation*}
$\therefore$ We have that 
\begin{equation*}
  \sigma \sigma^{-1} = \epsilon = \sigma^{-1} \sigma.
\end{equation*}

\begin{eg}
  \label{eg:inverse_permutation}
  Find the inverse of
  \begin{equation*}
    \sigma = \begin{pmatrix}
      1 & 2 & 3 & 4 & 5 \\
      4 & 5 & 1 & 2 & 3
    \end{pmatrix}
  \end{equation*}

  \begin{solution}
    By rearranging the image in ascending order, using them now as the object and their respective objects as their image, construct
    \begin{equation*}
      \tau = \begin{pmatrix}
        1 & 2 & 3 & 4 & 5 \\
        3 & 4 & 5 & 1 & 2
      \end{pmatrix}.
    \end{equation*}
    It can easily (although perhaps not so prettily) be shown that
    \begin{equation*}
      \sigma \tau = \epsilon = \tau \sigma.
    \end{equation*}
  \end{solution}
\end{eg}

With all the above, we have for ourselves the following proposition:

\begin{propo}[Properties of $S_n$]\label{propo:properties_of_Sn}
  We have\sidenote{These properties show that $S_n$ is a group, which will be defined later.}
  \begin{enumerate}
    \item $\forall \sigma, \tau \in S_n \enspace \sigma \tau, \tau \sigma \in S_n$.
    \item $\forall \sigma, \tau, \mu \in S_n \enspace \sigma (\tau \mu) = (\sigma \tau) \mu$.
    \item $\exists \epsilon \in S_n \enspace \forall \sigma \in S_n \enspace \sigma \epsilon = \sigma = \epsilon \sigma$.
    \item $\forall \sigma \in S_n \enspace \exists! \sigma^{-1} \in S_n \enspace \sigma \sigma^{-1} = \epsilon = \sigma^{-1} \sigma$.
  \end{enumerate}
\end{propo}

\newthought{Consider}
\begin{equation*}
  \sigma = \begin{pmatrix}
    1 & 2 & 3 & 4 & 5 & 6 & 7 & 8 & 9 & 10 \\
    3 & 1 & 7 & 6 & 9 & 4 & 2 & 5 & 8 & 10
  \end{pmatrix} \in S_{10}
\end{equation*}
If we represent the action of $\sigma$ geometrically, we get

\begin{tabular}{c c c}
  \begin{tikzcd}
   & 1 \arrow[rd] &  \\
  2 \arrow[ru] &  & 3 \arrow[ld] \\
   & 7 \arrow[lu] & 
  \end{tikzcd}
  &
  \begin{tikzcd}
  4 \arrow[dd, bend left] \\
   \\
  6 \arrow[uu, bend left]
  \end{tikzcd}
  &
  \begin{tikzcd}
   & 5 \arrow[rdd] &  \\
   &  &  \\
  8 \arrow[ruu] &  & 9 \arrow[ll]
  \end{tikzcd}
  \\ &
  \begin{tikzcd}
    10 \arrow[loop below]
  \end{tikzcd}
\end{tabular}
We observe that $\sigma$ can be \hlnotea{decomposed} into one $4$-cycle, $\begin{pmatrix} 1 & 3 & 7 & 2 \end{pmatrix}$, one $2$-cycle, $\begin{pmatrix} 4 & 6 \end{pmatrix}$, one $3$-cycle, $\begin{pmatrix} 5 & 9 & 8 \end{pmatrix}$, and one $1$-cycle, $\begin{pmatrix} 10 \end{pmatrix}$.

Note that these cycles are (pairwise) \hlnotea{disjoint}, and we can write\sidenote{We generally do not include the $1$-cycle and assume that by excluding them, it is known that any number that is supposed to appear loops back to themselves.}
\begin{equation*}
  \sigma = \begin{pmatrix} 1 & 3 & 7 & 2 \end{pmatrix}\begin{pmatrix} 4 & 6 \end{pmatrix}\begin{pmatrix} 5 & 9 & 8 \end{pmatrix}
\end{equation*}
Note that we may also write
\begin{align*}
  \sigma &= \begin{pmatrix} 4 & 6 \end{pmatrix}\begin{pmatrix} 5 & 9 & 8 \end{pmatrix}\begin{pmatrix} 1 & 3 & 7 & 2 \end{pmatrix} \\
    &= \begin{pmatrix} 6 & 4 \end{pmatrix}\begin{pmatrix} 9 & 8 & 5 \end{pmatrix}\begin{pmatrix} 7 & 2 & 1 & 3 \end{pmatrix}
\end{align*}
It is interesting to note that the cycles can rotate their ``elements'' in a \hlnotea{cyclic} manner, i.e.
\begin{gather*}
  \begin{pmatrix} 1 & 3 & 7 & 2 \end{pmatrix} = \begin{pmatrix} 7 & 2 & 1 & 3 \end{pmatrix} \neq \begin{pmatrix} 1 & 2 & 7 & 3 \end{pmatrix}.
\end{gather*}
Although the decomposition of the cycle notation is not unique (i.e. you may rearrange them), each individual cycle is unique, and is proven below\sidenote{See bonus question of A1. Proof will be included in the notes once the assignment is over.}.

\begin{thm}[Cycle Decomposition Theorem]\label{thm:cycle_decomposition_theorem}
\index{Cycle Decomposition Theorem}
  If $\sigma \in S_n$, $\sigma \neq \epsilon$, then $\sigma$ is a product of (one or more) disjoint cycles of length at least $2$. This factorization is unique up to the order of the factors.
\end{thm}

\begin{note}[Convention]
 Every permutation in $S_n$ can be regarded as a permutation of $S_{n + 1}$ by fixing the permutation of $n + 1$. Therefore, we have that
 \begin{equation*}
   S_1 \subseteq S_2 \subseteq \hdots \subseteq S_n \subseteq S_{n + 1} \subseteq \hdots
 \end{equation*}
\end{note}

% subsection permutations (end)

% section introduction_continued (end)

% chapter lecture_2_may_04th_2018 (end)

\chapter{Lecture 3 May 07th 2018}
  \label{chapter:lecture_3_may_07th_2018}

\section{Groups} % (fold)
\label{sec:groups}

\subsection{Groups} % (fold)
\label{sub:groups}

\begin{defn}[Groups]\label{defn:groups}
\index{Groups}
  Let $G$ be a set and $*$ an operation on $G \times G$. We say that $G = (G, *)$ is a \hlnoteb{group} if it satisfies\sidenote{If you wonder why the uniqueness is not specified for \hlnoteb{Identity} and \hlnoteb{Inverse}, see \cref{propo:uniqueness_of_group_identity_and_group_element_inverse}.}
  \begin{enumerate}
    \item \hlnoteb{Closure}: $\forall a, b \in G \quad a * b \in G$
    \item \hlnoteb{Associativity}: $\forall a, b, c \in G \quad a * (b * c) = (a * b) * c$
    \item \hlnoteb{Identity}: $\exists e \in G \enspace \forall a \in G \quad a * e = a = e * a$
    \item \hlnoteb{Inverse}: $\forall a \in G \enspace \exists b \in G \quad a * b = e = b * a$
  \end{enumerate}
\end{defn}

\begin{defn}[Abelian Group]\label{defn:abelian_group}
\index{Abelian Group}
  A group $G$ is said to be abelian if $\forall a, b \in G$, we have $a * b = b * a$.
\end{defn}

\begin{propo}[Group Identity and Group Element Inverse]\label{propo:uniqueness_of_group_identity_and_group_element_inverse}
  Let $G$ be a group and $a \in G$.
  \begin{enumerate}
    \item The identity of $G$ is unique.
    \item The inverse of $a$ is unique.
  \end{enumerate}
\end{propo}

\begin{proof}
  \begin{enumerate}
    \item If $e_1, e_2 \in G$ are both identities of $G$, then we have
      \begin{equation*}
        e_1 \overset{(1)}{=} e_1 * e_2 \overset{(2)}{=} e_2
      \end{equation*}
      where $(1)$ is because $e_2$ is an identity and $(2)$ is because $e_1$ is an identity.

    \item Let $a \in G$. If $b_1, b_2 \in G$ are both the inverses of $a$, then we have
      \begin{equation*}
        b_1 = b_1 * e = b_1 * (a * b_2) \overset{(1)}{=} e * b_2 = b_2
      \end{equation*}
      where $(1)$ is by associativity.
  \end{enumerate}
\end{proof}

\begin{eg}
  The sets $(\mathbb{Z}, +), \, (\mathbb{Q}, +), \, (\mathbb{R}, +)$, and $(\mathbb{C}, +)$ are all abelian, wehre the additive identity is $0$, and the additive inverse of an element $r$ is $(-r)$.
\end{eg}

\begin{note}
  $(\mathbb{N}, +)$ is not a group for neither does it have an identity nor an inverse for any of its elements.
\end{note}

\begin{eg}
  The sets $(\mathbb{Q}, \cdot), \, (\mathbb{R}, \cdot)$ and $(\mathbb{C}, \cdot)$ are \hlwarn{not} groups, since $0$ has no multiplicative inverse in $\mathbb{Q}, \mathbb{R}$ or $\mathbb{C}$.
\end{eg}

We may define that for a set $S$, let $S^* \subseteq S$ contain all the elements of $S$ that has a multiplicative inverse. For example, $\mathbb{Q}^* = \mathbb{Q} \setminus \{0\}$. Then, $(\mathbb{Q}, \cdot), (\mathbb{R}, \cdot)$ and $(\mathbb{C}, \cdot)$ are groups and are in fact abelian, where the multiplicative identity is $1$ and the multiplicative of an element $r$ is $\frac{1}{r}$.

\begin{eg}
  The set $\big( M_n(\mathbb{R}), + \big)$ is an abelian group, where the additive identity is the zero matrix, $0 \in M_n(\mathbb{R})$, and the additive inverse of an element $M = [a_{ij}] \in M_n(\mathbb{R})$ is $-M = [-a_{ij}] \in M_n(\mathbb{R})$.
\end{eg}

\newthought{Consider} the set $M_n(\mathbb{R})$ under the matrix mutiplication operation that we have introduced in \nameref{chapter:lecture_1_may_02nd_2018}. We found that the identity matrix is
\begin{equation*}
  I = \begin{bmatrix}
    1 & 0 & \hdots & 0 \\
    0 & 1 & \hdots & 0 \\
    \vdots & \vdots & & \vdots \\
    0 & 0 & \hdots & 1
  \end{bmatrix} \in M_n(\mathbb{R}).
\end{equation*}
But since not all elements of $M_n(\mathbb{R})$ have a multiplicative inverse\sidenote{The multiplicative inverse of a matrix does not exist if its determinant is $0$.}, $(M_n(\mathbb{R}), \cdot)$ is not a group.

\newthought{We can try} to do something similar as to what we did before: by excluding the elements that do not have an inverse. In this case, we exclude elements whose determinant is $0$. We define the following set

\begin{defn}[General Linear Group]\index{General Linear Group}
\label{defn:general_linear_group}
  The \hlnoteb{general linear group of degree $n$ over $\mathbb{R}$} is defined as
  \begin{equation*}
    GL_n(\mathbb{R}) := \{ M \in M_n(\mathbb{R}) \, : \, \det M \neq 0 \}
  \end{equation*}
\end{defn}

Note that $\because \det I = 1 \neq 0$, we have that $I \in GL_n(\mathbb{R})$. \\
Also, $\forall A, B \in GL_n(\mathbb{R} )$, we have that $\because \det A \neq 0 \, \land \, \det B \neq 0$,
\begin{equation*}
  \det AB = \det A \det B \neq 0,
\end{equation*}
and therefore $AB \in GL_n(\mathbb{R} )$. Finally, $\forall M \in GL_n(\mathbb{R})$, $\exists M^{-1} \in GL_n(\mathbb{R})$ such that
\begin{equation*}
  MM^{-1} = I = M^{-1} M
\end{equation*}
since $\det M \neq 0$. $\therefore (GL_n(\mathbb{R}), \cdot)$ is a group.

\newthought{Since} we have introduced permutations in \nameref{chapter:lecture_2_may_04th_2018}, we shall formalize the purpose of its introduction below.

\begin{eg}
  Consider $S_n$, the set of all permutations on $\{1, 2, ..., n\}$. By \cref{propo:properties_of_Sn}, we know that $S_n$ is a group. We call $S_n$ the \hldefn{symmetry group} \hlnoteb{of degree $n$}. For $n \geq 3$, the group $S_n$ is not abelian\sidenote{Let us make this an exercise.
  \begin{ex}
    For $n \geq 3$, prove that the group $S_n$ is not abelian.
  \end{ex}}.
\end{eg}

\newthought{Now that} we have a fairly good idea of the basic concept of a group, we will now proceed to look into handling multiple groups. One such operation is known as the \hldefn{direct product}.

\begin{eg}
  \label{eg:direct_product}
  Let $G$ and $H$ be groups. Their direct product is the set $G \times H$ with the component-wise operation defined by
  \begin{equation*}
    (g_1, h_1) * (g_2, h_2) = (g_1 *_G g_2, h_1 *_H h_2)
  \end{equation*}
  where $g_1, g_2 \in G$, $h_1, h_2 \in H$, $*_G$ is the operation on $G$, and $*_H$ is the operation on $H$.

  The \hlnoteb{closure} and \hlnoteb{associativity} property follow immediately from the definition of the operation. The identity is $(1_G, \, 1_H)$ where $1_G$ is the identity of $G$ and $1_H$ is the identity of $H$. The inverse of an element $(g_1, \, h_1) \in G \times H$ is $(g_1^{-1}, \, h_1^{-1})$.
\end{eg}

By induction, we can show that if $G_1, G_2, ..., G_n$ are groups, then so is $G_1 \times G_2 \times \hdots \times G_n$.

To facilitate our writing, use shall use the following notations:

\begin{notation}
  Given a group $G$ and $g_1, g_2 \in G$, we often denote its identity by $1$, and write $g_1 * g_2 = g_1 g_2$. Also, we denote the unique inverse of an element $g \in G$ as $g^{-1}$.

  We will write $g^0 = 1$. Also, for $n \in \mathbb{N}$, we define
  \begin{equation*}
     g^n = \underbrace{g * g * \hdots * g}_{n \text{ times}}
  \end{equation*}
  and
  \begin{equation*}
    g^{-n} = (g^{-1})^n
  \end{equation*}
\end{notation}

With the above notations,

\begin{propo}\label{propo:group_notations}
  Let $G$ be a group and $g, h \in G$. We have \marginnote{
    \begin{ex}
      Prove \cref{propo:group_notations} as an exercise.
    \end{ex}
  }
  \begin{enumerate}
    \item $(g^{-1})^{-1} = g$
    \item $(gh)^{-1} = h^{-1} g^{-1}$
    \item $g^n g^m = g^{n + m}$ for all $n, m \in \mathbb{Z}$
    \item $(g^n)^m = g^{nm}$ for all $n, m \in \mathbb{Z}$
  \end{enumerate}
\end{propo}

\begin{warning}
  In general, it is not true that if $g, h \in G$, then $(gh)^n = g^n h^n$. For example,
  \begin{equation*}
    (gh)^2 = ghgh \quad \text{but} \quad g^2 h^2 = gghh.
  \end{equation*}
  The two are only equal if and only if $G$ is abelian.
\end{warning}

% subsection groups (end)

% section groups (end)

% chapter lecture_3_may_07th_2018 (end)

\chapter{Lecture 4 May 09 2018}
  \label{chapter:lecture_4_may_09_2018}

\section{Groups (Continued)} % (fold)
\label{sec:groups_continued}

\subsection{Groups (Continued)} % (fold)
\label{sub:groups_continued}

\begin{propo}[Cancellation Laws]\label{propo:cancellation_laws}
  Let $G$ be a group and $g, h, f \in G$. Then
  \begin{enumerate}
    \item \begin{enumerate}
        \item (\hlnoteb{Right Cancellation}) $gh = gf \implies h = f$
        \item (\hlnoteb{Left Cancellation}) $hg = fg \implies h = f$
      \end{enumerate}
    \item The equation $ax = b$ and $ya = b$ have unique solution for $x, y \in G$.
  \end{enumerate}
\end{propo}

\begin{proof}
  \begin{enumerate}
    \item \begin{enumerate}
      \item By left multiplication and associativity,
        \begin{equation*}
          gh = gf \iff g^{-1} gh = g^{-1} gf \iff h = f
        \end{equation*}
      \item By right multiplication and associativity,
        \begin{equation*}
          hg = fg \iff hgg^{-1} = fgg^{-1} \iff h = f
        \end{equation*}
    \end{enumerate}

    \item Let $x = a^{-1} b$. Then
      \begin{equation*}
        a x = a (a^{-1} b) = (aa^{-1}) b = b.
      \end{equation*}
      If $\exists u \in G$ that is another solution, then
      \begin{equation*}
        au = b = ax \implies u = x
      \end{equation*}
      by Left Cancellation. The proof for $ya = b$ is similar by letting $y = ba^{-1}$.
  \end{enumerate}\qed
\end{proof}

% subsection groups_continued (end)

\subsection{Cayley Tables} % (fold)
\label{sub:cayley_tables}

For a finite group, defining its operation by means of a table is sometimes convenient.

\begin{defn}[Cayley Table]\label{defn:cayley_table}
\index{Cayley Table}
  Let $G$ be a group. Given $x, y \in G$, let the product $xy$ be an entry of a table in the row corresponding to $x$ and column corresponding to $y$. Such a table is called a \hlnoteb{Cayley Table}.
\end{defn}

\begin{note}
  By \autoref{propo:cancellation_laws}, the entries in each row (and respectively, column) of a Cayley Table are all distinct.
\end{note}

\begin{eg}
  Consider the group $(\mathbb{Z}_2, +)$. Its Cayley Table is
  \begin{center}
    \begin{tabular}{c|c|c}
      $\mathbb{Z}_2$ & $[0]$ & $[1]$ \\
      \hline
      $[0]$     & $[0]$ & $[1]$ \\
      $[1]$     & $[1]$ & $[0]$ 
    \end{tabular}
  \end{center}
  where note that we must have $[1] + [1] = [0]$; otherwise if $[1] + [1] = [1]$ then $[1]$ does not have its additive inverse, which contradicts the fact that it is in the group.
\end{eg}

\marginnote {
  If we replace $1$ by $[0]$ and $-1$ by $[1]$, the Cayley Tables of $\mathbb{Z}_2$ and $\mathbb{Z}^*$ are the same. In thie case, we say that $\mathbb{Z}_2$ and $\mathbb{Z}^*$ are \hlnotea{isomorphic}, which we denote by $\mathbb{Z}_2 \cong \mathbb{Z}^*$.
}

\begin{eg}
  Consider the group $\mathbb{Z}^* = \{1. -1\}$. Its Cayley Table (under multiplication) is
  \begin{center}
    \begin{tabular}{c|c|c}
      $\mathbb{Z}^*$ & $1$    & $-1$ \\
      \hline
      $1$              & $1$  & $-1$ \\
      $-1$             & $-1$ & $1$
    \end{tabular}
  \end{center}
\end{eg}

\begin{eg}\label{eg:cyclic_group_cayley_table}
  Given $n \in \mathbb{N}$, the \hldefn{Cyclic Group} of order $n$ is defined by
  \begin{equation*}
    C_n = \{1, a, a^2, ..., a^{n - 1}\} \quad \text{with } a^n = 1.
  \end{equation*}
  We write $C_n = \langle a : a^n = 1 \rangle$ and $a$ is called a generator of $C_n$. The Cayley Table of $C_n$ is
  \begin{center}
    \begin{tabular}{c | c c c c c c}
      $C_n$     & $1$       & $a$       & $a^2$  & $\hdots$ & $a^{n - 2}$ & $a^{n - 1}$ \\
      \hline
      $1$       & $1$       & $a$       & $a^2$  & $\hdots$ & $a^{n - 2}$ & $a^{n - 1}$ \\
      $a$       & $a$       & $a^2$     & $a^3$  & $\hdots$ & $a^{n - 1}$ & $1$ \\
      $a^2$     & $a^2$     & $a^3$     & $a^4$  & $\hdots$ & $1$         & $a$ \\
      \vdots    & \vdots    & \vdots    & \vdots &          & \vdots      & \vdots \\
      $a^{n-2}$ & $a^{n-2}$ & $a^{n-1}$ & $1$    & $\hdots$ & $a^{n-4}$   & $a^{n-3}$ \\
      $a^{n-1}$ & $a^{n-1}$ & $1$       & $a$    & $\hdots$ & $a^{n-3}$   & $a^{n-2}$
    \end{tabular}
  \end{center}
\end{eg}

\begin{propo}\label{propo:small_groups}
  Let $G$ be a group. Up to isomorphism, we have
  \begin{enumerate}
    \item if $\abs{G} = 1$, then $G \cong \{1\}$.
    \item if $\abs{G} = 2$, then $G \cong C_2$.
    \item if $\abs{G} = 3$, then $G \cong C_3$.
    \item if $\abs{G} = 4$, then either $G \cong C_4$ or $G \cong K_4 \cong C_2 \times C_2$ \marginnote{$K_n$ is known as the \hldefn{Klein n-group}}.
  \end{enumerate}
\end{propo}

\begin{proof}
  \begin{enumerate}
    \item If $\abs{G} = 1$, then it can only be $G = \{1\}$ where $1$ is the identity element.
    \item $\abs{G} = 2 \implies G = \{1, g\}$ with $g \neq 1$. The Cayley Table of $G$ is thus
      \begin{center}
        \begin{tabular}{c | c c}
        $G$ & $1$ & $g$ \\
        \hline
        $1$ & $1$ & $g$ \\
        $g$ & $g$ & $1$
        \end{tabular}
      \end{center}
      where we note that $g^2 = 1$; otherwise if $g^2 = g$, then we would have $g = 1$ by \autoref{propo:cancellation_laws}, which contradicts the fact that $g \neq 1$. Comparing the above Cayley Table with that of $C_2$, we see that $G = \langle g : g^2 = 1 \rangle \cong C_2$.
    \item $\abs{G} = 3 \implies G = \{1, g, h\}$ with $g \neq 1 \neq h$ and $g \neq h$. We can then start with the following Cayley Table:
      \begin{center}
        \begin{tabular}{c | c c c}
        $G$ & $1$ & $g$ & $h$ \\
        \hline
        $1$ & $1$ & $g$ & $h$ \\
        $g$ & $g$ &     &     \\
        $h$ & $h$ &     &     
        \end{tabular}
      \end{center}
      We know that by \autoref{propo:cancellation_laws}, $gh \neq g$ and $gh \neq h$. Thus $gh = 1$. Similarly, we get that $hg = 1$.

      \underline{Claim:} Entries in a row (or column) must be distinct. Suppose not. Then say $g^2 = 1$. But since $gh = 1$, by \autoref{propo:cancellation_laws}, we have that $h = g$, which is a contradiction.

      With that, we can proceed to fill in the rest of the entries: with $g^2 = h$ and $h^2 = g$. Therefore,
      \begin{center}
        \begin{tabular}{c | c c c}
        $G$ & $1$ & $g$ & $h$ \\
        \hline
        $1$ & $1$ & $g$ & $h$ \\
        $g$ & $g$ & $h$ & $1$ \\
        $h$ & $h$ & $1$ & $g$
        \end{tabular}
      \end{center}

      Recall that the Cayley Table for $C_3$ is:
      \begin{center}
        \begin{tabular}{c | c c c}
        $C_3$ & $1$   & $a$   & $a^2$ \\
        \hline
        $1$   & $1$   & $a$   & $a^2$ \\
        $a$   & $a$   & $a^2$ & $1$ \\
        $a^2$ & $a^2$ & $1$   & $a$
        \end{tabular}
      \end{center}
      $\therefore G \cong C_3$ (by identifying $g = a$ and $h = a^2$).

    \item \hlwarn{Proof will be added once assignment 1 is over}
  \end{enumerate}
\end{proof}

% subsection cayley_tables (end)

% section groups_continued (end)

\section{Subgroups}
\label{sec:subgroups}

\subsection{Subgroups}
\label{sub:subgroups}

\begin{defn}[Subgroup]\label{defn:subgroup}
\index{Subgroup}
  Let $G$ be a group and $H \subseteq G$. If $H$ itself is a group, then we say that $H$ is a subgroup of $G$
\end{defn}

% subsection subgroups (end)

% section subgroups (end)

% chapter lecture_4_may_09_2018 (end)

\chapter{Lecture 5 May 11th 2018}
\label{chp:lecture_5_may_11th_2018}

\section{Subgroups (Continued)}
\label{sec:subgroups_continued}
% section Subgroups (Continued)

\subsection{Subgroups (Continued)}
\label{sub:subgroups_continued}
% subsection Subgroups (Continued)

\begin{note}[Recall: definition of a subgroup]
  Let $G$ be a group and $H \subseteq G$. If $H$ itself is a group, then we say that $H$ is a subgroup of $G$.
\end{note}

\begin{note}
  Since $G$ is a group, $\forall h_1, h_2, h_3 \in H \subseteq G$, we have $h_1 (h_2 h_3) = (h_1 h_2) h_3$. So $H$ is a subgroup of $G$ if it satisfies the following conditions, which we shall hereafter refer to as the Subgroup Test.

\noindent\hldefn{Subgroup Test} \\
  \marginnote{Note that the identity in $H$ must also be the identity in $G$. This is because if $h_1, h_1^{-1} \in H$, then $h_1 h_1^{-1} = 1_H$, but $h_1, h_1^{-1} \in G$ as well, and so $h_1 h_1^{-1} = 1_G$. Thus $1_H = 1_G$.}
  \begin{enumerate}
    \item $h_1 h_2 \in H$
    \item $1_G \in H$
    \item $\exists h_1^{-1} \in H$ such that $h_1 h_1^{-1} = 1_G$
  \end{enumerate}
\end{note}

\begin{eg}
  Given a group $G$, it is clear that $\{1\}$ and $G$ are both subgroups of $G$.
\end{eg}

\begin{eg}
  We have the following chain of groups:
  \begin{equation*}
    (\mathbb{Z}, +) \subseteq (\mathbb{Q}, +) \subseteq (\mathbb{R}, +) \subseteq (\mathbb{C}, +)
  \end{equation*}
\end{eg}

Recall that the general linear group is defined as:
\begin{equation*}
  GL_n(\mathbb{R}) = (GL_n(\mathbb{R}), \cdot) = \{A \in M_n(\mathbb{R}) : \det A \neq 0 \}
\end{equation*}

\begin{defn}[Special Linear Group]\label{defn:special_linear_group}
\index{Special Linear Group}
  The \hlnoteb{special linear group} of order $n$ of $\mathbb{R}$ is defined as
  \begin{equation*}
    SL_n(\mathbb{R}) = (SL_n(\mathbb{R}), \cdot) = \{A \in M_n(\mathbb{R}) : \det A = 1 \}
  \end{equation*}
\end{defn}

\begin{eg}\label{eg:special_linear_group_as_a_subgroup}
  Clearly, $SL_n(\mathbb{R}) \subseteq GL_n(\mathbb{R})$. Note that the identity matrix $I$ must be in $SL_n(\mathbb{R})$ since $\det I = 1$. Also, $\forall A, B \in SL_n(\mathbb{R})$, we have that
  \begin{equation*}
    \det AB = \det A \det B = 1
  \end{equation*}
  $\therefore AB \in SL_n(\mathbb{R})$. Also, since $\det A^{-1} = \frac{1}{\det A} = 1$, we also have that $\A^{-1} \in SL_n(\mathbb{R})$. We see that $SL_n(\mathbb{R})$ satisfies the \hlnoteb{Subgroup Test}, and hence it is a subgroup of $GL_n(\mathbb{R})$.
\end{eg}

\begin{defn}[Center of a Group]\label{defn:center_of_a_group}
\index{Center of a Group}
  Given a group $G$, the \hlnoteb{the center of a group $G$} is defined as
  \begin{equation*}
    Z(G) = \{z \in G \, : \, \forall g \in G \enspace zg = gz \}
  \end{equation*}
\end{defn}

\begin{eg}
  For a group $G$, $Z(G)$ is an abelian subgroup of $G$.

  \begin{proof}
    Clearly, $1_G \in Z(G)$. Let $y, z \in G$. $\forall g \in G$, we have that
    \begin{equation*}
      (yz)g = y(zg) = y(gz) = (yg)z = (gy)z = g(yz)
    \end{equation*}
    Therefore $yz \in Z(G)$ and so $Z(G)$ is closed under its operation. Also, $\forall h 
    in G$, we can write $h = (h^{-1})^{-1} = g^{-1}$. Since $z \in Z(G)$, we have that $\forall g \in G$,
    \begin{align*}
      zg = gz \iff (zg)^{-1} = (gz)^{-1} &\iff g^{-1} z^{-1} = z^{-1} g^{-1} \\
          &\iff hz^{-1} = z^{-1} h
    \end{align*}
    Therefore $z^{-1} \in Z(G)$. By the \hlnoteb{Subgroup Test}, it follows that $Z(G)$ is a subgroup of $G$.

    Finally, since $Z(G) \subseteq G$, by its definition, we have that $\forall x, y \in Z(G)$, $x, y \in G$ as well, and we have that $xy = yx$. Therefore, $Z(G)$ is abelian. \qed
  \end{proof}
\end{eg}

\begin{propo}[Intersection of Subgroups is a Subgroup]\label{propo:intersection_of_subgroups_is_a_subgroup}
  Let $H$ and $K$ be subgroups of a group $G$. Then their intersection
  \begin{equation*}
    H \cap K = \{g \in G : g \in H \, \land \, g \in K\}
  \end{equation*}
  is also a subgroup of $G$.
\end{propo}

\begin{proof}
  Since $H$ and $K$ are subgroups, we have that $1 \in H$ and $1 \in K$ and hence $1 \in H \cap K$. Let $a, b \in H \cap K$. Since $H$ and $K$ are subgroups, we have that $ab \in H$ and $ab \in K$. Therefore, $ab \in H \cap K$. Similarly, since $a^{-1} \in H$ and $a^{-1} \in K$, $a^{-1} \in H \cap K$. By the \hlnoteb{Subgroup Test}, $H \cap K$ is a subgroup of $G$. \qed
\end{proof}

\begin{propo}[Finite Subgroup Test]\label{propo:finite_subgroup_test}
\index{Finite Subgroup Test}
\marginnote{This result says that if $H$ is a finite nonempty subset, then we only need to prove that it is closed under its operation to prove that it is a subgroup. The other two conditions in the \hlnoteb{Subgroup Test} are automatically implied.}
  If $H$ is a finite nonempty subset of a group $G$, then $H$ is a subgroup if and only if $H$ is closed under its operation.
\end{propo}

\begin{proof}
  The forward direction of the proof is trivially true, since $H$ must satisfy the closure property for it to be a subgroup.

  For the converse, since $H \neq \emptyset$, let $h \in H$. Since $H$ is closed under its operation, we have that
  \begin{equation*}
    h, h^2, h^3, ...
  \end{equation*}
  are all in $H$. Since $H$ is finite, not all of the $h^n$'s are distinct. Then, $\forall n \in \mathbb{N}$, there must $\exists m \in \mathbb{N}$ such that $h^n = h^{n + m}$. Then by \autoref{propo:cancellation_laws}, $h^m = 1$ and so $1 \in H$. Also, because $1 = h^{m - 1} h$, we have that $h^{-1} = h^{m - 1}$, and thus the inverse of $h$ is also in $H$. Therefore, $H$ is a subgroup of $G$ as requried. \qed
\end{proof}

% subsection Subgroups (Continued) (end)

% section Subgroups (Continued) (end)

% chapter lecture_5_may_11th_2018 (end)

\chapter{Lecture 6 May 14th 2018}
\label{chp:lecture_6_may_14th_2018}
% chapter Lecture 6 May 14th 2018

\section{Subgroups (Continued 2)}
\label{sec:subgroups_continued_2}
% section Subgroups (Continued 2)

\subsection{Alternating Groups}
\label{sub:alternating_groups}
% subsection Alternating Groups

Recall that $\forall \sigma \in S_n$, with $\sigma \neq \epsilon$, $\sigma$ can be uniquely decomposed (up to the order) as disjoint cycles of length at least $2$. We will now present a related concept.

\begin{defn}[Transposition]\label{defn:transposition}
\index{Transposition}
  A \hlnoteb{transposition} $\sigma \in S_n$ is a cycle of length $2$, i.e. $\sigma = \begin{pmatrix} a & b \end{pmatrix}$, where $a, b \in \{1, ..., n\}$ and $a\ neq b$.
\end{defn}

\begin{eg}
  We have that\sidenote{If we apply the permutations on the right hand side, we have that
    \begin{gather*}
      1 \quad 2 \quad 3 \quad 4 \quad 5 \\
      \downarrow \\
      1 \quad 2 \quad 3 \quad 5 \quad 4 \\
      \downarrow \\
      1 \quad 4 \quad 3 \quad 5 \quad 2 \\
      \downarrow \\
      2 \quad 4 \quad 3 \quad 5 \quad 1
    \end{gather*}
  }
  \begin{equation*}
    \begin{pmatrix} 1 & 2 & 4 & 5 \end{pmatrix} = \begin{pmatrix} 1 & 2 \end{pmatrix} \begin{pmatrix} 2 & 4 \end{pmatrix} \begin{pmatrix} 4 & 5 \end{pmatrix}
  \end{equation*}
  Also, we can show that\sidenote{
  \begin{ex}
    Show that \autoref{eq:transposition_eg} is true.
  \end{ex}

  \begin{ex}
    Play around with the same idea and create a few of your own transpositions. Note that you will only be able to get an odd number of tranpositions (why?).
  \end{ex}
  }
  \begin{equation}\label{eq:transposition_eg}
    \begin{pmatrix} 1 & 2 & 4 & 5 \end{pmatrix} = \begin{pmatrix} 2 & 3 \end{pmatrix} \begin{pmatrix} 1 & 2 \end{pmatrix} \begin{pmatrix} 2 & 5 \end{pmatrix} \begin{pmatrix} 1 & 3 \end{pmatrix} \begin{pmatrix} 2 & 4 \end{pmatrix}
  \end{equation}
\end{eg}

Observe that the factorization into transpositions are \hlimpo{not unique or disjoint}. However, the following property is true.

\begin{thm}[Parity Theorem]\label{thm:parity_theorem}
\index{Parity Theorem}
  If a permutations $\sigma$ has $2$ factorizations
  \begin{equation*}
    \sigma = \gamma_1 \gamma_2 \hdots \gamma_r = \mu_1 \mu_2 \hdots \mu_s,
  \end{equation*}
  where each $\gamma_i$ and $\mu_j$ are transpositions, then $r \equiv s \mod 2$.
\end{thm}

\begin{proof}
  \hlwarn{This is the bonus question in A2. Proof shall be included after the end of the assignment.}
\end{proof}

\begin{defn}[Odd and Even Permutations]\label{defn:odd_and_even_permutations}
\index{Odd Permutations}\index{Even Permutations}
  A permutation $\sigma$ is even (or odd) if it can be written as a product of an even (or odd) number of transpositions. By \autoref{thm:parity_theorem}, a permutation must either be even or odd, but not both.
\end{defn}

\begin{thm}[Alternating Group]\label{thm:alternating_group}
\index{Alternating Group}
  For $n \geq 2$, let $A_n$ denote the set of all even permutations in $S_n$. Then
  \begin{enumerate}
    \item $\epsilon \in A_n$
    \item $\forall \sigma, \tau \in A_n \enspace \sigma \tau \in A_n$ and $\exists \sigma^{-1} \in A_n$ such that $\sigma \sigma^{-1} = \epsilon = \sigma^{-1} \sigma$
    \item $\abs{A_n} = \frac{1}{2} n!$
  \end{enumerate}
\end{thm}

\begin{note}
  From items 1 and 2, we know that $A_n$ is a subgroup of $S_n$. $A_n$ is called the \hlnoteb{alternating subgroup of degree $n$}.
\end{note}

\begin{proof}
  \begin{enumerate}
    \item We have that $\epsilon = \begin{pmatrix} 1 & 2 \end{pmatrix} \begin{pmatrix} 1 & 2 \end{pmatrix}$. Thus $\epsilon$ is even and so $\epsilon \in A_n$.
    \item $\forall \sigma, \tau \in A_n$, we may write
      \begin{align*}
        \sigma &= \sigma_1 \sigma_2 \hdots \sigma_r \quad \text{and} \\
        \tau   &= \tau_1 \tau_2 \hdots \tau_s,
      \end{align*}
      where $\sigma_i, \tau_j$ are transpositions, and $r, s$ are even integers. Then
      \begin{equation*}
        \sigma \tau = \sigma_1 \sigma_2 \hdots \sigma_r \tau_1 \tau_2 \hdots \tau_s
      \end{equation*}
      is a product of $(r + s)$ transpositions, and thus $\sigma \tau$ is even. Thus $\sigma \tau \in A_n$.

      For the inverse, note that since $\sigma_i$ is a transposition, we have that $\sigma_i^2 = \epsilon$ and thus $\sigma_i^{-1} = \sigma_i$. It follows that
      \begin{align*}
        \sigma^{-1} &= (\sigma_1 \sigma_2 \hdots \sigma_r)^{-1} \\
          &= \sigma_r^{-1} \sigma_{r - 1}^{-1} \hdots \sigma_2^{-1} \sigma_1^{-1} \\
          &= \sigma_r \sigma_{r - 1} \hdots \sigma_2 \sigma_1
      \end{align*}
      which is an even permutation and
      \begin{equation*}
        \sigma \sigma^{-1} = \sigma_1 \sigma_2 \hdots \sigma_r \sigma_r \hdots \sigma_2 \sigma_1 = \epsilon.
      \end{equation*}
      Thus $\exists \sigma^{-1} \in A_n$ such that it is the inverse of $\sigma$.
    \item Let $O_n$ denote the set of odd permutations in $S_n$.\marginnote{For the proof of 3, we know that $\abs{S_n} = n!$, which is twice of the suggested order of $A_n$. Since we took out the even permutations of $S_n$, we just need to make the rest of the permutations, the odd permutations, into a set and prove that $A_n$ and this new set has the same size. One way to show this is by creating a bijection between the two.
    
        Also, note that the set of all odd permutations of $S_n$ is not a group, since
        \begin{itemize}
          \item there is no identity element in this set; and
          \item this set is not closed under map composition.
        \end{itemize}
    
        We have shown that $\epsilon$ is an even permutation, and so by the \hyperref[thm:parity_theorem]{Parity Theorem}, it cannot be an odd permutation, and there is only one identity in $S_n$. The set is not closed under map composition since if we compose two odd permutations, we would get an even permutation, which does not belong to this set.
    } Then we have $S_n = A_n \cup O_n$, and by the \hyperref[thm:parity_theorem]{Parity Theorem}, we have that $A_n \cap O_n = \emptyset$. Since $\abs{S_n} = n!$, to prove that $\abs{A_n} = \frac{1}{2} n!$, it suffices to show that $\abs{A_n} = \abs{O_n}$.
    
    Let $\gamma = \begin{pmatrix} 1 & 2 \end{pmatrix}$ and $f : A_n \to O_n$ such that $f(\sigma) = \gamma \sigma$. Since $\sigma$ is even, $\gamma \sigma$ is odd, and so $f$ is well-defined.
    
    Also, if $\gamma \sigma_1 = \gamma \sigma_2$, then by \hyperref[propo:cancellation_laws]{Cancellation Laws}, $\sigma_1 = \sigma_2$, and hence $f$ is injective.
    
    Finally, $\forall \tau \in O_n$, we have that $\gamma \tau = \sigma \in A_n$. Note that
  \begin{equation*}
    f(\sigma) = \gamma \sigma = \gamma \gamma \tau = \tau.
  \end{equation*}
  Therefore, $f$ is surjective.

  It follows that $\abs{A_n} = \abs{O_n}$. \qed
  \end{enumerate}
\end{proof}

% subsection Alternating Groups (end)

\subsection{Order of Elements}
\label{sub:order_of_elements}
% subsection Order of Elements

\begin{notation}
  If $G$ is a group and $g \in G$, we denote
  \begin{equation*}
    \lra{g} = \{ g^k : k \in \mathbb{Z} \}.
  \end{equation*}
  Note that $1 = g^0 \in \lra{g}$.

  If $x = g^m, y = g^n \in \lra{g}$ where $m, n \in \mathbb{Z}$, then
  \begin{equation*}
    xy = g^m g^n = g^{m + n} \in \lra{g}
  \end{equation*}
  and we have $\exists x^{-1} = g^{-m} \in \lra{g}$ such that
  \begin{equation*}
    xx^{-1} = g^m g^{-m} = g^0 = 1.
  \end{equation*}
\end{notation}

Along with the \hlnoteb{Subgroup Test}, we have the following proposition:

\begin{propo}[Cyclic Group as A Subgroup]\label{propo:cyclic_group_as_a_subgroup}
  If $G$ is a group and $g \in G$, then $\lra{g}$ is a subgroup of $G$.
\end{propo}

\begin{defn}[Cyclic Groups]\label{defn:cyclic_groups}
\index{Cyclic Group}
  Let $G$ be a group and $g \in G$. Then we call $\lra{g}$ the \hlnoteb{cyclic subgroup} of $G$ generated by $g$. If $G = \lra{g}$ for some $g \in G$, then we say that $G$ is a \hlnoteb{cyclic group}, and $g$ is a \hldefn{generator} of $G$.
\end{defn}

% subsection Order of Elements (end)

% section Subgroups (Continued 2) (end)

% chapter Lecture 6 May 14th 2018 (end)

\chapter{Lecture 7 May 16th 2018}%
\label{chp:lecture_7_may_16th_2018}
% chapter lecture_7_may_16th_2018

\section{Subgroups (Continued 3)}%
\label{sec:subgroups_continued_3}
% section subgroups_continued_3

\subsection{Order of Elements (Continued)}%
\label{sub:order_of_elements_continued}
% subsection order_of_elements_continued

\begin{eg}
  Consider $(\mathbb{Z}, +)$ . Note that $\forall k \in \mathbb{Z}$, we can write $k = k \cdot 1 = \underbrace{1 + 1 + \hdots + 1}_{k \text{ times}}$. So we have that $(\mathbb{Z} , +) = \lra{1}$. Similarly, we would have $(\mathbb{Z} , +) = \lra{-1}$.

\noindent However, observe that $\forall n \in \mathbb{Z}$ with $n \neq \pm 1$, there is no $k \in \mathbb{Z} $ such that $k \cdot n = 1$. Therefore, $\pm 1$ are the only \hlnotea{generators} of $\mathbb{Z}$.
\end{eg}

\newthought{Let} $G$ be a group and $g \in G$. Suppose $\exists k \in \mathbb{Z}$ with $k \neq 0$ such that $g^k = 1$. Then $g^{-k} = ( g^k )^{-1} = 1$. Thus wlog, we can assume that $k \geq 1$. By the \hlnotea{Well Ordering Principle}, $\exists n \in \mathbb{N}$ such that $n$ is the smallest, such that $g^n = 1$.

With that, we may have the following definition:

\begin{defn}[Order of an Element]\index{Order of an Element}
\label{defn:order_of_an_element}
  Let $G$ be a group and $g \in G$. If $n$ is the smallest positive integer such that $g^n = 1$, we say that the order of $g$ is $n$, denoted by $o(g) = n$.

  \noindent If no such $n$ exists, then we say that $g$ has infinite order and write $o(g) = \infty$.
\end{defn}

\begin{propo}[Properties of Elements of Finite Order]
\label{propo:properties_of_elements_of_finite_order}
  Let $G$ be a group with $g \in G$ where $o(g) = n \in \mathbb{N}$. Then
  \begin{enumerate}
    \item $g^k = 1 \iff n | k$;
    \item $g^k = g^m \iff k \equiv m \mod n$; and
    \item $\lra{g} = \{1, g, g^2, ..., g^{n - 1} \}$ where each $g^i$ is distinct from others.\sidenote{This also means that the order of the group is the same as the order of the generator.}
  \end{enumerate}
\end{propo}

\begin{proof}
  \begin{enumerate}
    \item $(\impliedby)$ If $n | k$, then $k = nq$ for some $q \in \mathbb{Z}$. Then
      \begin{equation*}
        g^k = g^{nq} = (g^n)^q = 1^q = 1
      \end{equation*}

      $(\implies)$ Suppose $g^k = 1$. Since $k \in \mathbb{Z}$, the \hlnotea{Division Algorithm}, we can write $k = nq + r$ with $q, r \in \mathbb{Z}$ and $0 \leq r < n$. Note $g^n = 1$. Thus
      \begin{equation*}
        g^r = g^{k - nq}  = g^k (g^n)^{-q} = 1 \cdot 1 = 1.
      \end{equation*}
      Since $0 \leq r < n$, we must have that $r = 0$. Thus $n | k$.

    \item $(\implies)$ $g^k = g^m \implies g^{k - m} = 1 \overset{\text{by } 1}{\implies} n | ( k - m ) \iff k \equiv m \mod n$
    
      $(\impliedby)$ $k \equiv m \mod n \implies \exists q \in \mathbb{Z} \enspace k = qnm$. The result follows from 1.

    \item $(\supseteq)$ is clear by definition of $\lra{g} = \{g^k : k \in \mathbb{Z}\}$.

      To prove $(\subseteq)$, let $x = g^k \in \lra{g}$ for some $k \in \mathbb{Z}$. By the \hlnotea{Division Algorithm}, $k = nq + r$ for some $q, r \in \mathbb{Z}$ and $0 \leq r < n$. Then
      \begin{equation*}
        x = g^k = g^{nq + r} = g^{nq} g^r \overset{\text{by } 1}{=} g^r.
      \end{equation*}
      Since $0 \leq r < n$, we have that $x \in \{1, g, g^2, ..., g^{n - 1} \}$. Thus $\lra{g} = \{1, g, g^2, ..., g^{n - 1} \}$.

      It remains to show that all the elements in $\lra{g}$ are distinct. Suppose $g^k = g^m$ for some $k, m \in \mathbb{Z}$ with $0 \leq k, m < n$. By 2, we have that $k \equiv m \mod 2$. Therefore, $k = m$.

      We can also use 1 by the fact that $g^{k - m} = 1$ from assumption to complete the uniqueness proof.
  \end{enumerate} \qed
\end{proof}

\begin{propo}[Property of Elements of Infinite Order]
\label{propo:property_of_elements_of_infinite_order}
  Let $G$ be a group, and $g \in G$ such that $o(g) = \infty$. Then
  \begin{enumerate}
    \item $g^k = 1 \iff k = 0$;
    \item $g^k = g^r \iff k = m$;
    \item $\lra{g} = \{..., g^{-2}, g^{-1} 1, g, g^2, ...\}$ where each $g^i$ is distinct from others.
  \end{enumerate}
\end{propo}

\begin{proof}
  It suffices to prove 1, since 2 easily becomes true with 1, and 2 $\implies$ 3.

  \begin{enumerate}
    \item $(\impliedby) \; g^0 = 1$

      $(\implies)$ Suppose for contradiction that $g^k = 1$ for some $k \in \mathbb{Z} \; k \neq 0$. Then $g^{-k} = (g^k)^{-1} = 1$. Then we can assume that $k \geq 1$. This, however, implies that $o(g)$ is finite, which contradicts our assumption. Thus $k = 0$.

    \item \begin{equation*}
      g^k = g^m \iff g^{k - m} = 1 \overset{\text{by } 1}{\iff} k - m = 0 \iff k = m
    \end{equation*}
  \end{enumerate} \qed
\end{proof}

\begin{propo}[Orders of Powers of the Element]
\label{propo:orders_of_powers_of_the_element}
  Let $G$ be a group, and $g \in G$ with $o(g) = n \in \mathbb{N}$. We have that
  \begin{equation*}
    \forall d \in \mathbb{N} \enspace d \; | \; n \implies o(g^d) = \frac{n}{d}
  \end{equation*}
\end{propo}

\begin{proof}
  Let $k = \frac{n}{d}$. Note that $(g^d)^k = g^n = 1$. It remains to show that $k$ is the smallest such positive integer. Suppose $\exists r \in \mathbb{N} \enspace (g^d)^r = 1$. Since $o(g) = n$, then $n \; | \; dr$. Then $\exists q \in \mathbb{Z} \enspace dr = nq$ by definition of divisibility. $\because n = dk$ and $d \neq 0$, we have
  \begin{align*}
    dr = dkq \overset{d \neq 0}{\implies} r = kq \implies r > k \quad \because r, k \in \mathbb{N} \implies q \in \mathbb{N}
  \end{align*}\qed
\end{proof}

% subsection order_of_elements_continued (end)

\subsection{Cyclic Groups}%
\index{Cyclic Group}
\label{sub:cyclic_groups}
% subsection cyclic_groups

Recall the definition of a cyclic groups.

\begin{defn*}[Cyclic Groups]
  Let $G$ be a group and $g \in G$. Then we call $\lra{g}$ the \hlnoteb{cyclic subgroup} of $G$ generated by $g$. If $G = \lra{g}$ for some $g \in G$, then we say that $G$ is a \hlnoteb{cyclic group}, and $g$ is a \hldefn{generator} of $G$.
\end{defn*}

\begin{propo}[Cyclic Groups are Abelian]
\label{propo:cyclic_groups_are_abelian}
  All cyclic groups are abelian.
\end{propo}

\begin{proof}
  Note that a cyclic group $G$ is of the form $G = \lra{g}$. So
  \begin{gather*}
    \forall a, b \in G \enspace \exists m, n \in \mathbb{Z} \enspace a = g^m \, \land \, b = g^n \\
    a \cdot b = g^m g^n = g^{m + n} = g^{n + m} = g^n g^m = b \cdot a
  \end{gather*}\qed
\end{proof}

% subsection cyclic_groups (end)

% section subgroups_continued_3 (end)

% chapter lecture_7_may_16th_2018 (end)

\chapter{Lecture 8 May 18th 2018}%
\label{chp:lecture_8_may_18th_2018}
% chapter lecture_8_may_18th_2018

\section{Subgroups (Continued 4)}%
\label{sec:subgroups_continued_4}
% section subgroups_continued_4

\subsection{Cyclic Groups (Continued)}%
\label{sub:cyclic_groups_continued}
% subsection cyclic_groups_continued

\begin{note}
  Consider the converse of \cref{propo:cyclic_groups_are_abelian}: Are abelian groups cyclic? \hlimpo{No!} For example, $K_4 \cong C_2 \times C_2$ is abelian but not cyclic, since no one element can generate the entire group.
\end{note}

\begin{propo}[Subgroups of Cyclic Groups are Cyclic]
\label{propo:subgroups_of_cyclic_groups_are_cyclic}
  Every subgroup of a cyclic group is cyclic.
\end{propo}

\begin{proof}
  Let $G = \lra{g}$ and $H$ be a subgroup of $G$.
  \begin{align*}
    H = \{1\} &\implies H = \lra{1} \\
    H \neq \{1\} & \implies \exists k \neq 0 \in \mathbb{Z} \enspace g^k \in H \\
                 & \implies g^{-k} \in H \quad (\because H \text{ is a group })
  \end{align*}
  We may assume that $k \in \mathbb{N}$. By the \hlnotea{Well Ordering Principle}, let $m \in \mathbb{N}$ be the smallest positive integer such that $g^m \in H$. We will now show that $H = \lra{g^m}$.

  \begin{align*}
    g^m \in H &\implies \lra{g^m} \subseteq H \\
    \because H \subseteq G = \lra{g} &\quad \forall h \in H \; \exists k \in \mathbb{Z} \; h = g^k \\
    \hlnotea{Division Algorithm} &: \exists q, r \in \mathbb{Z} \; 0 \leq r < m \quad k = mq + r \\
    h = g^k \implies g^r = g^{k - mq} &= g^k (g^m)^{-q} \in H \\
    r \neq 0 \implies \exists 0 < r < m &\quad g^r \in H \quad \text{\Lightning} \quad m \text{ is the smallest +ve integer } \\
    \implies g^k \in \lra{g^m} & \implies H \subseteq \lra{g^m}
  \end{align*}
  Finally,
  \begin{equation*}
    \lra{g^m} \subseteq H \, \land \, H \subseteq \lra{g^m} &\implies H = \lra{g^m}
  \end{equation*}\qed
\end{proof}

\begin{propo}[Other generators in the same group]
\label{propo:other_generators_in_the_same_group}
  Let $G = \lra{g}$ with $o(g) = n \in \mathbb{N}$. We have
\marginnote{If we have $k$ such that $g^k \in G$, and $k$ and $n$ are coprimes, then $g^k$ is also a generator of $G$.}
  \begin{equation*}
    G = \lra{g^k} \iff \gcd(k, n) = 1
  \end{equation*}
\end{propo}

\begin{proof}
  For $(\implies)$,
  \begin{align*}
    G = \lra{g^k} &\implies g \in \lra{g^k} \implies \exists x \in \mathbb{Z} \quad g = g^{kx} \\
      &\implies 1 = g^{kx - 1} \implies n \, | \, (kx - 1) \quad (\because \cref{propo:properties_of_elements_of_finite_order}) \\
      &\implies \exists y \in \mathbb{Z} \quad kx - 1 = ny \quad (\because \hlnotea{Division Algorithm}) \\
      &\implies 1 = kx + ny
  \end{align*}
  Then
  \begin{gather*}
    \because 1 \, | \, kx \enspace \land \enspace 1 \, | \, ny \enspace \land \enspace 1 = kx + ny \\
    \gcd(k, n) = 1 \qquad (\because \hlnotea{gcd Characterization})
  \end{gather*}

  For $(\impliedby)$, note that $g \in G \implies \lra{g^k} \subseteq G$. It suffices to show that $G \subseteq \lra{g^k}$, i.e. $g \in \lra{g^k}$.
  \begin{align*}
    \gcd(k, n) = 1 &\implies \exists x, y \in \mathbb{Z} \enspace 1 = kx + ny \quad (\because \hlnotea{Bezout's Lemma}) \\
        &\implies g = g^1 = g^{kx + ny} = (g^k)^x (g^n)^y = (g^k)^x \in \lra{g^k}
  \end{align*}\qed
\end{proof}

\begin{thm}[Fundamental Theorem of Finite Cyclic Groups]
\label{thm:fundamental_theorem_of_finite_cyclic_groups}
  \marginnote{This is a significant result that classifies the structure of a cyclic group (hence its name). The theorem tells us that for a group with finite order, it has only finitely many subgroups, and the order of each of these subgroups are multiples of $n$. Inversely, there are no subgroups of $G$ where its order is some integer that does not divide $n$.
  
\noindent  \hlimpo{Note:} It is clear that $d \in \mathbb{N}$ and $d \leq n$.

In a sense, this theorem is more powerful than \cref{propo:subgroups_of_cyclic_groups_are_cyclic}.
  }
  Let $G = \lra{g}$ with $o(g) = n \in \mathbb{N}$.
  \begin{enumerate}
    \item $H$ is a subgroup of $G \implies \exists d \in \mathbb{N} \enspace d \, | \, n \quad H = \lra{g^d} \implies \abs{H} \, | \, n$.
    \item $k \, | \, n \implies \lra{g^{\frac{k}{n}}}$ is the unique subgroup of $G$ of order $k$.
  \end{enumerate}
\end{thm}

\begin{proof}
  \begin{enumerate}
    \item Note
      \begin{equation*}
        \cref{propo:subgroups_of_cyclic_groups_are_cyclic} \implies \exists m \in \mathbb{N} \enspace H = \lra{g^m} 
      \end{equation*}
      Let $d = \gcd(m, n)$. Want to show that $H = \lra{g^d}$.
      \begin{align*}
        d = \gcd(m, n) &\implies d \, | \, m \implies \exists k \in \mathbb{Z} \enspace m = dk \\
          &\implies g^m = g^{dk} = (g^d)^k \in \lra{g^d} \implies H \subseteq \lra{g^d} \\
        d = \gcd(m, n) &\implies \exists x, y \in \mathbb{Z} \quad d = mx + ny \quad (\because \hlnotea{Bezout's Lemma}) \\
          &\implies g^d = g^{mx + ny} = (g^m)^x (g^n)^y = (g^m)^x (1) \in H \\
          &\implies \lra{g^d} \subseteq H \\
          &\therefore H = \lra{g^d}
      \end{align*}
      Note: $d = \gcd(m, n) \implies d \, | \, n \implies \abs{H} = o(g^d) = \frac{n}{d}$ \\ $\because \cref{propo:orders_of_powers_of_the_element}$. Thus $\abs{H} \, | \, n$.

    \item Let $K$ be a subgroup of $G$ with order $k$ such that $k \, | \, n$. By 1, we have $K = \lra{g^d}$ with $d \, | \, n$. Note that
    \begin{equation*}
      k = \abs{K} \overset{(1)}{=} o(g^d) \overset{(2)}{=} \frac{n}{d}
    \end{equation*}
    where $(1)$ is by \cref{propo:properties_of_elements_of_finite_order} and $(2)$ is by \cref{propo:orders_of_powers_of_the_element}. Thus $d = \frac{n}{k}$ and $K = \lra{g^{\frac{n}{k}}}$
  \end{enumerate}\qed
\end{proof}

% subsection cyclic_groups_continued (end)

% section subgroups_continued_4 (end)

% chapter lecture_8_may_18th_2018 (end)

\chapter{Lecture 9 May 22nd 2018}%
\label{chp:lecture_9_may_22nd_2018}
% chapter lecture_9_may_22nd_2018

\section{Subgroups (Continued 5)}%
\label{sec:subgroups_continued_5}
% section subgroups_continued_5

\subsection{Examples of Non-Cyclic Groups}%
\label{sub:examples_of_non_cyclic_groups}
% subsection examples_of_non_cyclic_groups

\begin{eg}
  The Klein $4$-group is
  \begin{equation*}
    K_4 = \{1, a, b, c\} \enspace \text{where } a^2 = b^2 = c^2 = 1 \text{ and } ab = c.
  \end{equation*}
  We may also write
  \begin{equation*}
    K_4 = \lra{ a, b : a^2 = 1 = b^2, \, ab = ba }.
  \end{equation*}
  Note that we can replace $( a, \, b )$ by $( a, \, c )$ or $( b, \, c )$.
\end{eg}

\begin{eg}
  The symmetric group of degree $3$ is
  \begin{equation*}
    S_3 = \{\epsilon, \sigma, \sigma^2, \tau, \tau \sigma, \tau \sigma^2 \}
  \end{equation*}
  where $\sigma^3 = \epsilon = \tau^2$ and $\sigma \tau = \tau \sigma^2$. We may also express $S_3$ as
  \begin{equation*}
    S_3 = \lra{ \sigma, \tau : \sigma^3 = \epsilon = \tau^2, \, \sigma \tau = \tau \sigma^2 }
  \end{equation*}
\end{eg}

\begin{defn}[Dihedral Group]\index{Dihedral Group}
\label{defn:dihedral_group}
\marginnote{Recall from Assignment 1 that the dihedral group is a set of rigid motions for transforming a regular polygon back to its original position while changing the index of its vertices.}
  For $n \geq 2$, the \hlnoteb{dihedral group} of order $2n$ is
  \begin{equation*}
    D_{2n} = \{1, a, ..., a^{n - 1}, b, ba, ..., b^{n - 1}\ }
  \end{equation*}
  where $a^n = 1 = b^2$ and $aba = b$. Note that $a$ represents a rotation of $\frac{2 \pi}{n}$ radians, and $b$ represents a reflection through the $x$-axis
\end{defn}

\begin{eg}
  We may write the dihedral group as
  \begin{equation*}
    D_{2n} = \lra{ a, b : a^n = 1 = b^2, \, aba = b }
  \end{equation*}
\end{eg}

\begin{ex}
  Prove the following:
  \begin{enumerate}
    \item $D_4 \cong K_4$
    \item $D_6 \cong S_3$
  \end{enumerate}
\end{ex}

% subsection examples_of_non_cyclic_groups (end)

% section subgroups_continued_5 (end)

\section{Normal Subgroup}%
\label{sec:normal_subgroup}
% section normal_subgroup

\subsection{Homomorphism and Isomorphism}%
\label{sub:homomorphism_and_isomorphism}
% subsection homomorphism_and_isomorphism

\begin{defn}[Homomorphism]\index{Homomorphism}
\label{defn:homomorphism}
  Let $G, H$ be groups. A mapping
  \begin{equation*}
    \alpha : G \to H
  \end{equation*}
  is called a \hlnoteb{homomorphism} if $\forall a, b \in G$,\sidenote{
  Note that $ab$ uses the operation of $G$ while $\alpha(a)\alpha(b)$ uses the operation of $H$.}
  \begin{equation*}
    \alpha(ab) = \alpha(a)\alpha(b).
  \end{equation*}
\end{defn}

\begin{eg}[A classical example]\label{eg:homomorphism_classical_eg}
  Consider the determinant map:\marginnote{Note that $\mathbb{R}^*$ is the set of real numbers that has a multiplicative inverse.}
  \begin{equation*}
    \det : GL_n(\mathbb{R}) \to \mathbb{R}^* \quad \text{given by } A \to \det A
  \end{equation*}
  Since
  \marginnote{This is a classical example to show a homomorphism, especially since the group $GL_n(\mathbb{R})$ uses \hlnoteb{matrix multiplication} while $\mathbb{R}^*$ uses regular \hlnoteb{arithmetic multiplication}.}
  \begin{equation*}
    \det AB = \det A \det B
  \end{equation*}
  we have that the determinant map is a homomorphism.
\end{eg}

\begin{propo}[Properties of Homomorphism]
\label{propo:properties_of_homomorphism}
  Let $\alpha: G \to H$ be a group homomorphism. Then
  \begin{enumerate}
    \item $\alpha(1_G) = 1_H$
    \item $\forall g \in G \enspace \alpha(g^{-1}) = \alpha(g)^{-1}$
    \item $\forall g \in G \; \forall k \in \mathbb{Z} \enspace \alpha(g^k) = \alpha(g)^k$
  \end{enumerate}
\end{propo}

\begin{proof}
  \begin{enumerate}
    \item Note that
      \begin{equation*}
        \alpha(1_G) \alpha(g) = \alpha(1_G \cdot g) = \alpha(g) = \alpha(g \cdot 1_G) = \alpha(g) \alpha(1_G)
      \end{equation*}
      Thus it must be that $\alpha(1_G) = 1_H$ for only the identity of $H$ satisfies this equation.

    \item Since $H$ is a group, we know that
      \begin{equation*}
        1_H = \alpha(g)\alpha(g)^{-1}.
      \end{equation*}
      Now with part 1, we have that
      \begin{equation*}
        \alpha(g)\alpha(g^{-1}) = \alpha(gg^{-1}) = \alpha(1_G) = 1_H = \alpha(g)\alpha(g)^{-1}.
      \end{equation*}
      By \cref{propo:cancellation_laws}, we have that $\alpha(g^{-1}) = \alpha(g)^{-1}$.

    \item This is simply a result of applying the definition repeatedly, which we can then perform an induction procedure to complete the proof.\qed
  \end{enumerate}
\end{proof}

\begin{defn}[Isomorphism]\index{Isomorphism}
\label{defn:isomorphism}
  Let $G, H$ be groups. Consider a mapping
  \begin{equation*}
    \alpha: G \to H
  \end{equation*}
  We say that $\alpha$ is an \hlnoteb{isomorphism} if it is a homomorphism and bijective.

  If $\alpha$ is an isomorphism, we say that $G$ is \hldefn{isomorphic to} to $H$, or that $G$ and $H$ are \hldefn{isomorphic}, and denote that by $G \cong H$.
\end{defn}

\begin{propo}[Isomorphism as an Equivalence Relation]
\label{propo:isomorphism_as_an_equivalence_relation}\index{Equivalence Relation}
  \begin{enumerate}
    \item \hlnotea{(Reflexive)} The identity map $G \to G$ is an isomorphism.
    \item \hlnotea{(Symmetric)} If $\sigma : G \to H$ is an isomorphism, then the inverse map $\sigma^{-1} : H \to G$ is also an isomorphism.
    \item \hlnotea{(Transitive)} If $\sigma : G \to H$ and $\tau : H \to K$, then the composition map $\tau \sigma : G \to K$ is also an isomorphism.
  \end{enumerate}
\end{propo}

\begin{proof}
  \begin{enumerate}
    \item The identity map is clearly bijective. For all $g_1, g_2 \in G$, we have that
      \begin{equation*}
        \alpha(g_1 g_2) = g_1 g_2 = \alpha(g_1)\alpha(g_2).
      \end{equation*}
      Thus the identity map is a homomorphism, and hence an isomorphism.
    
    \item Since $\sigma$ is a bijective map, its inverse $\sigma^{-1}$ exists and is also a bijective map. Since $\sigma$ is bijective, we have that
      \begin{equation*}
        \forall h_1, h_2 \in H \enspace \exists ! g_1, g_2 \in G \quad \sigma(g_1) = h_1, \, \sigma(g_2) = h_2.
      \end{equation*}
      Note that since $\sigma$ has a bijective inverse, we also have
      \begin{equation*}
        g_1 = \sigma^{-1}(h_1) \text{ and } g_2 = \sigma^{-1}(h_2).
      \end{equation*}
      Then since $\sigma$ is a homomorphism,
      \begin{align*}
        \sigma^{-1}(h_1 h_2) &= \sigma^{-1}(\sigma(g_1)\sigma(g_2)) = \sigma^{-1}(\sigma(g_1 g_2)) \\
          &= g_1 g_2 = \sigma^{-1}(h_1) \sigma^{-1}(h_2).
      \end{align*}

    \item We know that the composition map of two bijective map is bijective. Let $g_1, g_2 \in G$, then since both $\tau$ and $\sigma$ are homomorphisms
      \begin{equation*}
        \tau \sigma (g_1 g_2) = \tau( \sigma(g_1) \sigma(g_2) ) = \tau \sigma(g_1) \tau \sigma(g_2),
      \end{equation*}
      where we note that $\sigma(g_1), \sigma(g_2) \in H$.
\end{enumerate}\qed
\end{proof}

\begin{eg}
  Let $\mathbb{R}^+ = \{r \in \mathbb{R} : r \geq 0 \}$. Show that $(\mathbb{R}, +) \cong (\mathbb{R}^+, \cdot)$.

  \begin{solution}
    Consider the map
    \begin{equation*}
      \alpha : (\mathbb{R}, +) \to (\mathbb{R}^+, \cdot) \quad r \mapsto e^r,
    \end{equation*}
    where $e$ is the natural exponent. Note that the exponential map from $\mathbb{R}$ to $\mathbb{R}^+$ is bijective\sidenote{The image of the map covers all positive real numbers while taking all real numbers, which is the perfect candidate as a map here.}. Also, $\forall r, s \in \mathbb{R}$ we have that
    \begin{equation*}
      \alpha(r + s) = e^{r + s} = e^r e^s = \alpha(r) \alpha(s).
    \end{equation*}
    Therefore, $\alpha$ is an isomorphism and $(\mathbb{R}, +) \cong (\mathbb{R}^+, \cdot)$.\qed
  \end{solution}
\end{eg}

\begin{eg}
  Show that $(\mathbb{Q}, +) \not\cong (\mathbb{Q}^*, \cdot)$.

  \begin{solution}
    Suppose, for contradiction, that $\tau : (\mathbb{Q}, +) \to (\mathbb{Q}^*, \cdot)$ is an isomorphism. In particular, we have that $\tau$ is onto. Then $\exists q \in \mathbb{Q}$ such that $\tau(q) = 2$. Let $\tau(\frac{q}{2}) = \alpha$. Since $\tau$ is an isomorphism, we have
    \begin{equation*}
      \alpha^2 = \tau(\frac{q}{2}) \tau(\frac{q}{2}) = \tau(\frac{q}{2} + \frac{q}{2}) = \tau(q) = 2.
    \end{equation*}
    But that implies that $\alpha = \sqrt{2}$, which is clearly not rational. Thus, we know that there is no such $\tau$ and
    \begin{equation*}
      (\mathbb{Q}, +) \not\cong (\mathbb{Q}^*, \cdot)
    \end{equation*}
    as required. \qed
  \end{solution}
\end{eg}

% subsection homomorphism_and_isomorphism (end)

\subsection{Cosets and Lagrange's Theorem}%
\label{sub:cosets_and_lagrange_s_theorem}
% subsection cosets_and_lagrange_s_theorem

\begin{defn}[Coset]\index{Coset}
\label{defn:coset}
  Let $H$ be a subgroup of a group $G$.
  \begin{equation*}
    \forall a \in G \quad Ha = \{ha : h \in H\} \quad \text{is the right coset of H generated by } a
  \end{equation*}
  and
  \begin{equation*}
    \forall a \in G \quad aH = \{ah : h \in H\} \quad \text{is the left coset of H generated by } a
  \end{equation*}
\end{defn}

\begin{note}
  Note that $1H = H = H1$. Also, since $a1 = a$ and $1 \in H$, we have that $a \in aH$, and similarly so for $a \in Ha$.

  In general, $aH$ and $Ha$ are not subgroups of $G$. See example

  Also, in general, $aH \neq Ha$, since not all groups are abelian.
\end{note}

\begin{propo}[Properties of Cosets]
\label{propo:properties_of_cosets}
  Let $H$ be a subgroup of $G$, and let $a, b \in G$. Then
  \begin{enumerate}
    \item $Ha = Hb \iff ab^{-1} \in H$. In particular, $Ha = H \iff a \in H$.
    \item $a \in Hb \implies Ha = Hb$.
    \item $Ha = Hb \veebar Ha \cap Hb = \emptyset$.\sidenote{$\veebar \equiv $ XOR} Then the distinct right cosets of $H$ forms a partition of $G$.\sidenote{Note that this is true because by definition, we iterate over all elements of $G$ to construct the cosets of the subgroup $H$. The earlier part of this statement implies that cosets must be distinct (otherwise, they are the same set), and so if we take the union of these cosets, by iterating through all elements of $G$, we get that
    \begin{equation*}
      \bigcup_{a \in G} Ha = G.
    \end{equation*}
    Summarizing the above argument, we observe that the distinct cosets partitions $G$.
    }
  \end{enumerate}
  We can create an analogued version of this proposition for the left cosets.
\end{propo}

\begin{proof}
  \begin{enumerate}
    \item For $(\implies)$,
      \begin{align*}
        Ha = Hb &\implies a = 1a \in Ha = Hb \\
                &\implies \exists h \in H \enspace a = hb \\
                &\implies ab^{-1} = h \in H.
      \end{align*}
      For $(\impliedby)$,
      \begin{align*}
        ab^{-1} \in H &\implies \forall h \in H \enspace ha = h(ab^{-1})b \in Hb \\
            &\implies Ha \subseteq Hb \\
        ab^{-1} \in H &\implies (ab^{-1})^{-1} = ba^{-1} \in H \\
            &\implies \forall h \in H \enspace hb = h(ba^{-1})a \in Ha \\
            &\implies Hb \subseteq Ha
      \end{align*}
      Let $b = 1$. Then
      \begin{equation*}
        Ha = H \iff a \in H \qquad \because 1^{-1} = 1
      \end{equation*}

    \item Note
      \begin{equation*}
        a \in Hb \implies \exists h \in H \enspace a = hb \implies ab^{-1} \in H \overset{\text{by 1}}{\implies} Ha = Hb
      \end{equation*}

    \item Trivially, if $Ha \cap Hb = \emptyset$, we are done.
      \begin{align*}
        &Ha \cap Hb \neq \emptyset \\ 
        &\implies \exists x \in Ha \cap Hb \\
        &\implies ( x \in Ha \overset{\text{by 1}}{\implies} Hx = Hb ) \, \land \, ( x \in Hb \overset{\text{by 1}}{\implies} Hx = Hb ) \\
        &\implies Ha = Hb
      \end{align*}
  \end{enumerate}\qed
\end{proof}

By \cref{propo:properties_of_cosets}, we have that $G$ can be written as a disjoint union of cosets of a subgroup $H$. We now define the following terminology that we shall use for the upcoming content.

\begin{defn}[Index]\index{Index}
\label{defn:index}
  Let $H$ be a subgroup of a group $G$. We call the number of disjoint cosets of $H$ in $G$ as the \hlnoteb{index} of $H$ in $G$, and denote this number by $[G : H]$.
\end{defn}

% subsection cosets_and_lagrange_s_theorem (end)

% section normal_subgroup (end)

% chapter lecture_9_may_22nd_2018 (end)

\chapter{Lecture 10 May 23rd 2018}%
\label{chp:lecture_10_may_23rd_2018}
% chapter lecture_10_may_23rd_2018

\section{Normal Subgroup (Continued)}%
\label{sec:normal_subgroup_continued}
% section normal_subgroup_continued

\subsection{Cosets and Lagrange's Theorem (Continued)}%
\label{sub:cosets_and_lagrange_s_theorem_continued}
% subsection cosets_and_lagrange_s_theorem_continued

\begin{thm}[Lagrange's Theorem]
\index{Lagrange's Theorem}
\label{thm:lagrange_s_theorem}
  Let $H$ be a subgroup of a \hlimpo{finite} group $G$. Then
  \begin{equation*}
    \abs{H} \, \Big| \, \abs{G} \text{ and } [G : H] = \frac{\abs{G}}{\abs{H}}
  \end{equation*}
\end{thm}

\begin{proof}
  Since $G$ is finite, there can only be finitely many cosets of $H$. Let $k = [G : H]$ and $Ha_1, Ha_2, ..., Ha_k$ be the distinct right cosets of $H$ in $G$. By \cref{propo:properties_of_cosets}, we have that these cosets partition $G$, i.e.
  \begin{equation*}
    G = \bigcup_{i = 1}^{k} Ha_i.
  \end{equation*}
  Note that by the definition of a right coset, the map 
  \begin{equation*}
    H \to Hb \; \text{ defined by } \; h \mapsto hb
  \end{equation*}
  is a surjection from $H$ to $Hb$. By \hyperref[propo:cancellation_laws]{Cancellation Laws}, the map is injective, since if $hb_1 = hb_2$, then $b_1 = b_2$. Therefore, for $i = 1, ..., k$,
  \begin{equation*}
    \abs{H} = \abs{Ha_i}.
  \end{equation*}
  Then we have
  \begin{equation*}
    \abs{G} = k \abs{H} \implies \abs{H} \, \Big| \, \abs{G} \, \land \, [G : H] = k = \frac{\abs{G}}{\abs{H}}
  \end{equation*}\qed
\end{proof}

\begin{crly}
\label{crly:lagrange_s_theorem_crly1}
  \begin{enumerate}
    \item If $G$ is a finite group and $g \in G$, then $o(g) \, \Big| \, G$.
    \item If $G$ is a finite group and $\abs{G} = n$, then $g^n = 1$.
  \end{enumerate}
\end{crly}

\begin{proof}
  \begin{enumerate}
    \item Let $H = \lra{g}$. Then by \autoref{thm:lagrange_s_theorem}, $o(g) = \abs{H} \, \Big| \, \abs{G}$.

    \item For some $g \in G$, let $o(g) = m \in \mathbb{Z} \setminus \{0\}$. Then by 1, $m \, | \, n$ and so $g^n = (g^m)^{\frac{n}{m}} = 1$.
  \end{enumerate}\qed
\end{proof}

\begin{note}
  Let $n \in \mathbb{N} \setminus \{1\}$. \hldefn{Euler's Totient Function}, or more generally written as \hldefn{Euler's $\phi$-function} is defined as
  \begin{equation}\label{eq:euler_s_totient_function}
    \phi(n) \equiv \Big| \big\{k \in \{1, ..., n - 1\} \, : \, \gcd(k, n) = 1 \big\} \Big|.
  \end{equation}
  Note that the set $\mathbb{Z}_n^*$ under multiplication has a similar definition to the set on the RHS, since the only numbers from $1$ to $n$ that has an inverse are those that are coprime with $n$. Thus $\phi(n) = \abs{\mathbb{Z}_n^*}$.

  With \cref{crly:lagrange_s_theorem_crly1}, we have \hldefn{Euler's Theorem} that states that
  \begin{equation}\label{eq:euler_s_theorem}
    \forall a \in \mathbb{Z} \enspace \gcd(a, n) = 1 \implies a^{\phi(n)} \equiv 1 \mod n.
  \end{equation}
  If $n = p$ where $p$ is some prime number, then Euler's Theorem implies \hldefn{Fermat's Little Theorem}, i.e. $a^{p - 1} \equiv 1 \mod p$.
\end{note}

\begin{crly}
\label{crly:lagrange_s_theorem_crly2}
  If $p$ is prime, then every group $G$ of order $p$ is cyclic. In fact, $g = \lra{g}$ fpr $g \neq 1 \in G$. Hence, the only subgroup of $G$ are $\{1\}$ and $G$ itself.
\end{crly}

\begin{proof}
  Let $g \in G$ such that $g \neq 1$. By \cref{crly:lagrange_s_theorem_crly1}, $o(g) \, | \, p$. Since $g \neq 1$ and $p$ is prime, by \hlnotea{uniqueness of prime factorization}, it must be that $o(g) = p$. Thus we can write $G = \lra{g}$. If $H$ is a subgroup of $G$, then by \hyperref[thm:lagrange_s_theorem]{Lagrange's Theorem}, we have $\abs{H} \, \big| \, p$. Since $p$ is prime, we either have $\abs{H} = 1$ or $p$. In other words, we either have that $H = \{1\}$ or $H = G$, respectively. \qed
\end{proof}

\begin{crly}
\label{crly:lagrange_s_theorem_crly3}
  Let $H$ and $K$ be finite subgroups of $G$. If $\gcd(\abs{H}, \abs{K}) = 1$, then $H \cap K = \{1\}$.
\end{crly}

\begin{proof}
  Since $H \cap K$ is a subgroup of $H$ and of $K$, by \autoref{thm:lagrange_s_theorem}, $\abs{H \cap K} \Big| \abs{H} \, \land \, \abs{H \cap K} \Big| \abs{K}$. By assumption that $\gcd(\abs{H}, \abs{K}) = 1$, we have\sidenote{$\abs{H \cap K}$ is a common divisor for $\abs{H}$ and $\abs{K}$. But $\gcd(\abs{H}, \abs{K}) = 1$} that $\abs{H \cap K} = 1$, and hence $\abs{H \cap K} = \{ 1 \}$. \qed
\end{proof}

% subsection cosets_and_lagrange_s_theorem_continued (end)

\subsection{Normal Subgroup}%
\label{sub:normal_subgroup}
% subsection normal_subgroup

We have seen that given $H$ is a subgroup of a group $G$ and $g \in G$, $gH$ and $Hg$ are generally not the same.

\begin{defn}[Normal Subgroup]\index{Normal Subgroup}
\label{defn:normal_subgroup}
  Let $H$ be a subgroup of a group $G$. If $\forall g \in G$, we have $Hg = gH$, then we say that $H$ is a \hlnoteb{normal subgroup} of $G$, and write
  \begin{equation*}
    H \triangleleft G
  \end{equation*}
\end{defn}

\begin{eg}
  $\{1\} \triangleleft G$ and $G \triangleleft G$.
\end{eg}

\begin{eg}
  The \hyperref[defn:center_of_a_group]{center}, $Z(G)$, of a group $G$ is an abelian group. By \cref{defn:normal_subgroup},
  \begin{equation*}
    Z(G) \triangleleft G.
  \end{equation*}
\end{eg}

\begin{eg}
  If $G$ is abelian, then every subgroup of $G$ is normal in $G$.
\end{eg}

\begin{propononum}[Normality Test]
  Let $H$ be a subgroup of $G$. The following are equivalent:
  \begin{enumerate}
    \item $H \triangleleft G$;
    \item $\forall g \in G \quad gHg^{-1} \subseteq H$;
    \item $\forall g \in G \quad gHg^{-1} = H$ \sidenote{This means that
    \begin{equation*}
      H \triangleleft G \iff H \text{ is the only conjugate of } H
    \end{equation*}}
  \end{enumerate}
\end{propononum}

% subsection normal_subgroup (end)

% section normal_subgroup_continued (end)

% chapter lecture_10_may_23rd_2018 (end)

\chapter{Lecture 11 May 25th 2018}%
\label{chp:lecture_11_may_25th_2018}
% chapter lecture_11_may_25th_2018

The following theorem is useful for A2. The proof is not provided in this lecture, but expect the corollary to be restated and proven in a later lecture.

\begin{crlynonum}
  Let $G$ be a finite group and $H, K \triangleleft G$, $H - K = \{1\}$ and $\abs{H} \abs{K} = \abs{G}$. Then $G \cong H \times K$.
\end{crlynonum}

\section{Normal Subgroup (Continued 2)}%
\label{sec:normal_subgroup_continued_2}
% section normal_subgroup_continued_2

\subsection{Normal Subgroup (Continued)}%
\label{sub:normal_subgroup_continued}
% subsection normal_subgroup_continued

\begin{note}[Recall]
  Recall the definition of a normal subgroup as in \cref{defn:normal_subgroup}. Let $H$ be a subgroup of $G$. If $gH = Hg$ for all $g \in G$, then $H \triangleleft G$.
\end{note}

\begin{propo}[Normality Test]
\index{Normality Test}
\label{propo:normality_test}
  Let $H$ be a subgroup of a group $G$. The following are equivalent:
  \marginnote{
    \begin{note}
  Note that item 3 is indeed a stronger statement that item 2. But since the statements are equivalent, while using the \hlnoteb{Normality Test}, if we can show that item 2 is true, item 3 is automatically true.
    \end{note}
    }
  \begin{enumerate}
    \item $H \triangleleft G$
    \item $\forall g \in G \enspace gHg^{-1} \subseteq H$
    \item $\forall g \in G \enspace gHg^{-1} = H$
  \end{enumerate}
\end{propo}

\begin{proof}
  $( 1 ) \implies ( 2 )$:
  \begin{align*}
    x \in gHg^{-1} &\implies \exists h \in H \enspace x = ghg^{-1} \\
      &\implies \exists h_1 \in H \enspace gh = h_1 g \qquad \because gh \in gH = Hg \\
      &\implies x = ghg^{-1} = h_1 gg^{-1} = h_1 \in H \\
      &\implies gHg^{-1} \subseteq H
  \end{align*}

  \noindent$( 2 ) \implies ( 3 )$:
  \begin{align*}
    ( 2 ) &\implies \forall g \in G \quad gHg^{-1} \subseteq H \\
      &\implies \exists g^{-1} \in G \quad g^{-1}Hg \subseteq H \\
      &\implies H \subseteq gHg^{-1} \\
      &\overset{(2)}{\implies} gHg^{-1} = H
  \end{align*}

  \noindent$( 3 ) \implies ( 1 )$:
  \begin{align*}
    ( 3 ) &\implies \forall g \in G \quad gHg^{-1} = H \\
      &\implies \forall x \in gH \quad xg^{-1} \in gHg^{-1} = H \\
      &\implies x \in Hg \qquad \because gg^{-1} = 1 \\
      &\implies gH \subseteq Hg
  \end{align*}
  Using a similar argument, we would have $Hg \subseteq Hg$. And so $gH = Hg$ as required.\qed
\end{proof}

\begin{eg}
  Let $G = GL_n(\mathbb{R})$ and $H = SL_n(\mathbb{R})$.\index{General Linear Group}\index{Special Linear Group}\sidenote{Recall \cref{defn:general_linear_group} and \cref{defn:special_linear_group}.} For $A \in G$ and $B \in H$ we have
  \begin{equation*}
    \det ABA^{-1} = \det A \det B \det A^{-1} = \det A (1) \frac{1}{\det A} = 1.
  \end{equation*}
  Thus $\forall A \in G, \, ABA^{-1} \in H$. By \cref{propo:normality_test}, $H \triangleleft G$, i.e. $SL_n(\mathbb{R}) \triangleleft GL_n(\mathbb{R})$.\sidenote{
  \begin{note}
    The normality is true for any field, not just $\mathbb{R}$.
  \end{note}
  }
\end{eg}

\begin{propo}[Subgroup of Index 2 is Normal]
\label{propo:subgroup_of_index_2_is_normal}
  \begin{equation*}
    \forall H \text{ subgroup of } G \, \land \, [G : H] = 2 \implies H \triangleleft G
  \end{equation*}
\end{propo}

\begin{proof}
  Let $a \in G$.
  \begin{align*}
    a \in H &\implies aH = H = Ha \\
    a \notin H &\implies G = H \cup Ha \implies Ha = G \setminus H \quad \because \cref{propo:properties_of_cosets} \\
    a \notin H &\implies G = H \cup aH \implies aH = G \setminus H \quad \because \cref{propo:properties_of_cosets}
  \end{align*}
  That implies that $aH = Ha$ for any $a \in G$. Hence, by \cref{propo:normality_test}, $H \triangleleft G$.\qed
\end{proof}

\begin{eg}
  Let $A_n$ be the \hldefn{Alternating Group} contained by $S_n$.\sidenote{Recall the definition of alternating group from \cref{thm:alternating_group} and $S_n$ from \cref{defn:permutations}} By \cref{propo:subgroup_of_index_2_is_normal}, since $[S_n : A_n] = 2$ because $S_n = A_n \cup O_n$ and $O_n$ is a coset of $A_n$, we have that
  \begin{equation*}
    A_n \triangleleft S_n.
  \end{equation*}
\end{eg}

\begin{eg}
  Let
  \begin{equation*}
    D_{2n} = \{1, a, a^2, ..., a^{n - 1}, b, ba, ba^2, ..., ba^{n - 1} \}
  \end{equation*}
  be the \hldefn{Dihedral Group} of order $2n$. Since $[D_{2n} : \lra{a}] = 2$,\sidenote{The coset of $\lra{a}$ is $b\lra{a}$.} we have that
  \begin{equation*}
    \lra{a} \triangleleft D_{2n} \quad \because \cref{propo:normality_test}.
  \end{equation*}
\end{eg}

\newthought{Let} $H$ and $K$ be subgroups of a group $G$. Recall an earlier discussion: $H \cap K$ is the largest subgroup contained in both $H$ and $K$.

What is the ``smallest'' subgroup that contains both $H$ and $K$? Since $H \cap K$ is the largest, it makes sense to think about $H \cup K$. However,
\begin{equation*}
  H \cup K \text{ is a subgroup of } G \iff H \subseteq K \veebar K \subseteq H
\end{equation*}
While we know that $H \cup K$ can indeed be such a subgroup, the price of the restriction is too high, since it is overly restrictive.

A more ``useful'' construction turns out to be the \hlnoteb{product} of the subgroups.

\begin{defn}[Product of Groups]\index{Product of Groups}
\label{defn:product_of_groups}
  \begin{equation*}
    HK := \{ hk \, : \, h \in H, \, k \in K \}
  \end{equation*}
\end{defn}

However, $HK$ is not necessarily a subgroup. For example, for $h_1 k_1, h_2 k_2 \in HK$, it is not necessary that $h_1 k_1 h_2 k_2 \in HK$, since $k_1 h_2$ is not necessarily equal to $h_2 k_1$.

\begin{lemma}[Product of Groups as a Subgroup]
\label{lemma:product_of_groups_as_a_subgroup}
  Let $H$ and $K$ be subgroups of $G$. The following are equivalent:
  \begin{enumerate}
    \item $HK$ is a subgroup of $G$
    \item $HK = KH$ \sidenote{If one of $H$ or $K$ is normal, then the lemma immediately kicks in.}
    \item $KH$ is a subgroup of $G$
  \end{enumerate}
\end{lemma}

\begin{proof}
  It suffices to prove $(1) \iff (2)$, since $(1) \iff (3)$ simply through exchanging $H$ and $K$.

  \noindent $(1) \implies (2)$: Let $kh \in KH$ such that $k \in K$ and $h \in H$. Their inverses are $k^{-1} \in K$ and $h^{-1} \in H$, since $K$ and $H$ are groups. Note that
  \begin{equation*}
    kh = (h^{-1} k^{-1})^{-1} \in HK \quad \because HK \text{ is a subgroup of } G.
  \end{equation*}
  Therefore $kh \in HK$, which implies $KH \subseteq HK$. By a similar argument, we can arrive at $HK \subseteq KH$ and so $HK = KH$.

  \noindent $(2) \implies (1)$: Note that 1 = 1 \cdot 1 \in HK. For all $hk \in HK$, $(hk)^{-1} = k^{-1} h^{-1} \in KH = HK$. For $h_1 k_1, h_2 k_2 \in HK$, note that $k_1 h_2 \in KH = HK$, so there exists $h k \in HK$ such that $k_1 h_2 = hk$. Therefore,
  \begin{equation*}
    h_1 k_1 h_2 k_2 = h_1 h k k_2 \in HK.
  \end{equation*}
  By the \hlnoteb{Subgroup Test}, $HK$ is a subgroup of $G$.\qed
\end{proof}

\begin{propo}[Product of Normal Subgroups is Normal]
\label{propo:product_of_normal_subgroups_is_normal}
  Let $H$ and $K$ be subgroups of $G$.
  \begin{enumerate}
    \item $H \triangleleft G \, \lor \, K \triangleleft G \implies HK = KH$ is a subgroup of $G$
    \item $H, K \triangleleft G \implies HK = KH \triangleleft G$
  \end{enumerate}
\end{propo}

\begin{proof}
  \begin{enumerate}
    \item Without loss of generality, suppose $H \triangleleft G$. Then
      \begin{equation}\label{eq:product_of_normal_subgroups_is_normal_pf_1}
        HK = \bigcup_{k \in K} Hk = \bigcup_{k \in K} kH = KH
      \end{equation}
      By \cref{lemma:product_of_groups_as_a_subgroup}, $HK = KH$ is a subgroup of $G$.

    \item Suppose $H, K \triangleleft G$. Then
      \begin{align*}
        \forall g \in G \enspace \forall hk \in HK \quad g^{-1}( hk )g = (g^{-1} hg)(g^{-1} k g) \in HK
      \end{align*}
      Thus $gHKg^{-1} \subseteq HK$. Thus by \cref{propo:normality_test}, we have that $HK \triangleleft G$.
  \end{enumerate}\qed
\end{proof}

\begin{note}
  Note that \cref{eq:product_of_normal_subgroups_is_normal_pf_1} is a weaker statement than the regular normality that we have defined, since it only requires all elements of $K$ to work instead of the entire $G$.
\end{note}

With that, we define the following notion:

\begin{defn}[Normalizer]\index{Normalizer}
\label{defn:normalizer}
  Let $H$ be a subgroup of $G$. The \hlnoteb{normalizer of $H$}, denoted by $N_G(H)$, is defined to be
  \begin{equation*}
    N_G(H) := \{ g \in G \, : \, gH = Hg \}
  \end{equation*}
\end{defn}

\begin{note}
  By the above definition, we immediately see that $H \triangleleft G \iff N_G(H) = G$ by \cref{eq:product_of_normal_subgroups_is_normal_pf_1}. Observe that since we only needed $kH = Hk$ in \cref{eq:product_of_normal_subgroups_is_normal_pf_1} for all $k \in K$, we have that $k \in N_G(H)$.
\end{note}

We shall now provide the following corollary which shall be proven in the next lecture.

\begin{crlynonum}
  Let $H$ and $K$ be subgroups of a group $G$.
  \begin{equation*}
    K \subseteq N_G(H) \, \lor \, H \subseteq N_G(K) \implies HK = KH \text{ is a subgroup of } G
  \end{equation*}
\end{crlynonum}

% subsection normal_subgroup_continued (end)

% section normal_subgroup_continued_2 (end)

% chapter lecture_11_may_25th_2018 (end)

\chapter{Lecture 12 May 28th 2018}%
\label{chp:lecture_12_may_28th_2018}
% chapter lecture_12_may_28th_2018

\section{Normal Subgroup (Continued 3)}%
\label{sec:normal_subgroup_continued_3}
% section normal_subgroup_continued_3

\subsection{Normal Subgroup (Continued 2)}%
\label{sub:normal_subgroup_continued_2}
% subsection normal_subgroup_continued_2

\begin{thm}
\label{thm:product_and_cross_product_of_normal_subgroups_are_isomorphic}
  If $H \triangleleft G$ and $K \triangleleft G$ satisfy $H \cap K = \{ 1 \}$, then
  \begin{equation*}
    HK \cong H \times K
  \end{equation*}
\end{thm}

\begin{proof}
  \underline{Claim 1:}
  \begin{equation*}
    H \triangleleft G \, \land \, K \triangleleft G \, \land \, H \cap K = \{1\} \implies \forall h \in H \enspace \forall k \in K \quad hk = kh
  \end{equation*}

  Consider $x = hkh^{-1}k^{-1}$. Note that since $H \triangleleft G$, by \cref{propo:normality_test}, we have that $\forall g \in G$, $gHg^{-1} = H$. Then $khk^{-1} \in kHk^{-1} = H$. Thus $x = h(kh^{-1}k^{-1}) \in H$. Using a similar argument, we can get that $x \in K$. Since $x \in H \cap K = \{1\}$, we have that $hkh^{-1}k^{-1} = 1$, we have that $hk = kh$ as claimed.

  Note that since $H \triangleleft G$, by \cref{propo:product_of_normal_subgroups_is_normal}, we have that $HK$ is a subgroup of $G$.\sidenote{We do not need the more powerful statement that says that $HK$ is a normal subgroup.} Define $\sigma : H \times K \to HK$ by
  \begin{equation*}
    \forall h \in H \enspace \forall k \in K \qquad \sigma\big( (h, k) \big) = hk
  \end{equation*}

  \noindent\underline{Claim 2:} $\sigma$ is an isomorphism.

  Let $(h, k), (h_1, k_1) \in H \times K$. By Claim 1, note that $h_1 k = kh_1$. Therefore,
  \begin{align*}
    \sigma\big( (h, k)\cdot(h_1, k_1) \big)
      &= \sigma\big( (hh_1, kk_1) \big) = hh_1 kk_1 \\
      &= hkh_1k_1 = \sigma\big( (h, k) \big) \sigma\big( (h_1, k_1) \big)
  \end{align*}
  Thus we see that $\sigma$ is a group homomorphism. Note that by the definition of $HK$, $\sigma$ is a surjection. Also, if $\sigma\big( (h, k) \big) = \sigma\big( (h_1, k_1) \big)$, we have that
  \begin{align*}
    hk = h_1 k_1 &\implies h_1^{-1} h = k_1 k^{-1} \in H \cap K = \{ 1 \} \\
      &\implies h_1^{-1} h = 1 = k_1 k^{-1} \implies h_1 = h \, \land \, k_1 = k.
  \end{align*}
  Thus $\sigma$ is an injection, and hence $\sigma$ is bijective. Therefore, $\sigma$ is an isomorphism. This proves that $HK \cong H \times K$. \qed
\end{proof}

An immediate result is the corollary that we were given in the last class but not proven.

\begin{crly}
\label{crly:group_iso_with_cross_prod_of_its_subgroups}
  Let $G$ be a finite group, $H, K \triangleleft G$ such that $H \cap K = \{1\}$ and $\abs{H} \abs{K} = \abs{G}$. Then $G \cong H \times K$.
\end{crly}

\begin{eg}
  Let $m, n \in \mathbb{N}$ with $\gcd(m ,n) = 1$. Let $G$ be a cyclic group of order $mn$. Write $G = \lra{a}$ with $o(a) = mn$. Let $H = \lra{a^n}$ and $K = \lra{a^m}$. Then we have
  \begin{equation*}
    \abs{H} = o(a^n) = m \, \land \, \abs{K} = o(a^m) = n.
  \end{equation*}
  It follows that $\abs{H}\abs{K} = mn = \abs{G}$. Note that $H \cong C_m$ and $K \cong C_n$. Since $\gcd(m, n) = 1$, by \cref{crly:lagrange_s_theorem_crly3}, we have that $H \cap K = \{1\}$.

  Also, since $G$ is cyclic and thus abelian, we have that $H, K \triangleleft G$. Then by \cref{crly:group_iso_with_cross_prod_of_its_subgroups}, we have that $G \cong C_{mn} \cong C_m \times C_n$.
\end{eg}

% subsection normal_subgroup_continued_2 (end)

% section normal_subgroup_continued_3 (end)

\section{Isomorphism Theorems}%
\label{sec:isomorphism_theorems}
% section isomorphism_theorems

\subsection{Quotient Groups}%
\label{sub:quotient_groups}
% subsection quotient_groups

Let $G$ be a group and $K$ a subgroup of $G$. Given a set
\begin{equation*}
  \{Ka \, : \, a \in G \},
\end{equation*}
how can we create a group out of it?

A ``natural'' way to define an operation on the set of right cosets above is
\begin{equation}\tag{$\dagger$}\label{eq:intro_to_quotients}
  \forall a, b \in G \qquad Ka * Kb = Kab.
\end{equation}
Note that it is entirely possible that for $a_1 \neq a$ and $b_1 \neq b$, we have $Ka = Ka_1$ and $Kb = Kb_1$. In order for \cref{eq:intro_to_quotients} to make sense as an operation, it is necessary that
\begin{equation*}
  Ka = Ka_1 \, \land \, Kb = Kb_1 \implies Kab = Ka_1 b_1.
\end{equation*}
If the condition is satisfied, we say that the ``multiplication'' $KaKb$ is well-defined.

\begin{lemma}[Multiplication of Cosets of Normal Subgroups]
\label{lemma:multiplication_of_cosets_of_normal_subgroups}
  Let $K$ be a subset of $G$. The following are equivalent:
  \begin{enumerate}
    \item $K \triangleleft G$;
    \item $\forall a, b \in G \enspace Ka Kb = Kab$ is well-defined.
  \end{enumerate}
\end{lemma}

\begin{proof}
  $(1) \implies (2)$ Suppose $K \triangleleft G$. Suppose $Ka = Ka_1$ and $Kb = Kb_1$. Then $aa_1^{-1} \in K$ and $bb_1^{-1} \in K$. To show that $Kab = Ka_1 b_1$, it suffices to show that $(ab) (a_1 b_1)^{-1} \in K$. Note that since $K \triangleleft G$, we have that $aKa^{-1} = K$. Therefore,
  \begin{align*}
    ab(a_1 b_1)^{-1} &= ab (b_1^{-1} a_1^{-1}) = a(bb_1^{-1}) a_1^{-1} \\
      &= \big(a ( bb_1^{-1} ) a^{-1}\big) (a a_1^{-1}) \in K.
  \end{align*}
  Therefore $Kab = Ka_1 b_1$ as required.

  \noindent $(2) \implies (1)$ If $a \in G$, we need to show that $\forall k \in K$, $aka^{-1} \in K$. Since $Ka = Ka$ and $Kk = K(1)$ \sidenote{This is cause $1$ is in the same coset.}, by $(2)$, we have that $Kak = Ka(1)$, i.e. $Kak = Ka$. Thus $aka^{-1} = 1 \in K$, implying that $aKa^{-1} \subseteq K$ and hence $K \triangleleft G$. \qed
\end{proof}

% subsection quotient_groups (end)

% section isomorphism_theorems (end)

% chapter lecture_12_may_28th_2018 (end)

\chapter{Lecture 13 May 30 2018}%
\label{chp:lecture_13_may_30_2018}
% chapter lecture_13_may_30_2018

\section{Isomorphism Theorems (Continued)}%
\label{sec:isomorphism_theorems_continued}
% section isomorphism_theorems_continued

\subsection{Quotient Groups (Continued)}%
\label{sub:quotient_groups_continued}
% subsection quotient_groups_continued

\begin{propo}
\label{propo:propo_related_to_quotient_groups}
  Let $K \triangleleft G$ and write $\faktor{G}{K} = \{Ka \, : \, a \ in G\}$ for the set of cosets of $K$.
  \begin{enumerate}
    \item $\faktor{G}{K}$ is a group under the operation $Ka Kb = Kab$.
    \item The mapping $\phi : G \to \faktor{G}{K}$ given by $\phi(a) = Ka$ is a surjective homomorphism.\sidenote{
    \begin{ex}
      Is $\phi$ injective?
    \end{ex}

    \begin{solution}
      We know that we cannot uniquely express a coset, since for $a, b \in Ka$ such that $a \neq b$, we have that $Ka = Kb$.
    \end{solution}
    }
    \item If $[G : K]$ is finite, then $\abs{\faktor{G}{K}} = [G : K]$. In particular, if $\abs{G}$ is finite, then $\abs{\faktor{G}{K}} = \frac{\abs{G}}{\abs{K}}$.
  \end{enumerate}
\end{propo}

\begin{proof}
  \begin{enumerate}
    \item By \cref{lemma:multiplication_of_cosets_of_normal_subgroups}, the operation is well-defined, and $\faktor{G}{K}$ is closed under the operation. The identity of $\faktor{G}{K}$ is $K = K(1)$ since $\forall Ka \in \faktor{G}{K}$,
      \begin{equation*}
        Ka K(1) = Ka = K(1) Ka.
      \end{equation*}
      Also, since
      \begin{equation*}
        Ka Ka^{-1} = K(1) = Ka^{-1} Ka,
      \end{equation*}
      the inverse of $Ka$ is $Ka^{-1}$. Finally, by associativity of $G$, we have that
      \begin{equation*}
        Ka(KbKc) = Kabc = (KaKb)Kc.
      \end{equation*}
      It follows that $\faktor{G}{K}$ is a group.

    \item Clearly, $\phi$ is surjective. For $a, b \in G$,
      \begin{equation*}
        \phi(ab) = Kab = Ka Kb = \phi(a) \phi(b).
      \end{equation*}
      Thus $\phi$ is a surjective homomorphism.

    \item If $[G : K]$ is finite, then by definition of the index $[G : K]$, we have that $[G : K] = \abs{\faktor{G}{K}}$. Also, if $\abs{G}$ is finite, then by \cref{thm:lagrange_s_theorem},
      \begin{equation*}
        \abs{\faktor{G}{K}} = [G : K] = \frac{\abs{G}}{\abs{K}}.
      \end{equation*}
  \end{enumerate}\qed
\end{proof}

\begin{defn}[Quotient Group]\index{Quotient Group}\index{Coset Map}\index{Quotient Map}
\label{defn:quotient_group}
  Let $K \triangleleft G$. The group $\faktor{G}{K}$ of all cosets of $K$ in $G$ is called the \hlnoteb{quotient group} of $G$ by $K$. Also, the mapping
  \begin{equation*}
    \phi: G \to \faktor{G}{K} \text{ defined by } a \mapsto Ka
  \end{equation*}
  is called the \hlnoteb{coset} (pr \hlnoteb{quotient}) \hlnoteb{map}.
\end{defn}

% subsection quotient_groups_continued (end)

\subsection{Isomorphism Theorems}%
\label{sub:isomorphism_theorems}
% subsection isomorphism_theorems

\begin{defn}[Kernel and Image]\index{Kernel}\index{Image of a Homomorphism}
\label{defn:kernel_and_image}
  Let $\alpha: G \to H$ be a group homomorphism. The \hlnoteb{kernel} of $\alpha$ is defined by
  \begin{equation*}
    \ker \alpha := \{g \in G \, : \, \alpha(g) = 1_H \} \subseteq G
  \end{equation*}
  and the image of $\alpha$ is defined by
  \begin{equation*}
    \img \alpha := \alpha(G) = \{\alpha(g) \, : \, g \in G \} \subseteq H.
  \end{equation*}
\end{defn}

\begin{propo}
\label{propo:image_of_hm_is_a_subgroup_n_kernel_of_hm_is_a_normal_subgroup}
  Let $\alpha : G \to H$ be a group homomorphism.
  \begin{enumerate}
    \item $\img \alpha$ is a subgroup of $H$
    \item $\ker \alpha \triangleleft G$
  \end{enumerate}
\end{propo}

\begin{proof}
  \begin{enumerate}
    \item Note that $1_H = \alpha(1_G) \in \alpha(G)$ (i.e. the identity is in $\img \alpha$). Also, for $h_1 = \alpha(g_1)$ and $h_2 = \alpha(g_2)$ in $\alpha(G)$ and $h_1, h_2 \in H$, we have
      \begin{equation*}
        h_1 h_2 = \alpha(g_1) \alpha(g_2) = \alpha(g_1 g_2) \in \alpha(G).
      \end{equation*}
      (i.e. $\img \alpha$ i closed under its operation). By \cref{propo:properties_of_homomorphism}, $\alpha(g)^{-1} = \alpha(g^{-1}) \in \alpha(G)$ (i.e. the inverse of an element is also in $\img \alpha$). Thus by the \hlnotea{Subgroup Test}, we have that $\img \alpha$ is a subgroup of $H$.

    \item For $\ker \alpha$, $\alpha(1_G) = 1_H$. For $k_1, k_2 \in \ker \alpha$, we have
      \begin{equation*}
        \alpha(k_1 k_2) = \alpha(k_1) \alpha(k_2) = 1 \cdot 1 = 1.
      \end{equation*}
      Also,
      \begin{equation*}
        \alpha(k_1^{-1}) = \alpha(k_1)^{-1} = 1^{-1} = 1.
      \end{equation*}
      By the \hlnotea{Subgroup Test}, $\ker \alpha$ is a subgroup of $G$.

      If $g \in G$ and $k \in \ker \alpha$, then
      \begin{equation*}
        \alpha(gkg^{-1}) = \alpha(g) \alpha(k) \alpha(g^{-1}) = \alpha(g) \alpha(g^{-1}) = 1.
      \end{equation*}
      Thus by \cref{propo:normality_test}, $\ker \alpha \triangleleft G$.
  \end{enumerate}\qed
\end{proof}

\begin{eg}
  Consider the determinant map
  \begin{equation*}
    \det : GL_n(\mathbb{R}) \to \mathbb{R}^* \text{ defined by } A \mapsto \det A.
  \end{equation*}
  Then $\ker \det = SL_n(\mathbb{R})$. Then $SL_n(\mathbb{R}) \triangleleft GL_n(\mathbb{R})$, as proven before.
\end{eg}

\begin{eg}
  Define the \hldefn{sign of a permutation} $\sigma \in S_n$ by
  \begin{equation*}
    \sign(\sigma) = \begin{cases}
      1 & \text{if } \sigma \text{ is even;} \\
      -1 & \text{if } \sigma \text{ is odd.}
    \end{cases}
  \end{equation*}
  Then the sign mapping, $\sign : S_n \to \{\pm 1\}$ defined by $\sigma \mapsto \sign(\sigma)$ is a homomorphism.\sidenote{Think about why. It's quite straightforward using the defintion.} Also, $\ker \sign = A_n$. Thus, we have $A_n \triangleleft S_n$, as proven before.
\end{eg}

\begin{propo}[Normal Subgroup as the Kernel]
\label{propo:normal_subgroup_as_the_kernel}
  If $K \triangleleft G$, then $K = \ker \phi$ where $\phi : G \to \faktor{G}{K}$ is the coset map.
\end{propo}

\begin{proof}
  Recall that $\phi : G \to \faktor{G}{K}$ is defined by $g \mapsto Kg$, $\forall g \in G$, and is a group homomorphism. By \cref{propo:properties_of_cosets}, we have
  \begin{equation*}
    Kg = K = K1 \iff g \in K.
  \end{equation*}
  Thus $K = \ker \phi$.\qed
\end{proof}

\begin{thm}[First Isomorphism Theorem]
\index{First Isomorphism Theorem}
\label{thm:first_isomorphism_theorem}
  Let $\alpha: G \to H$ be a group homomorphism. We have
  \begin{equation*}
    \faktor{G}{\ker \alpha} \cong \img \alpha
  \end{equation*}
\end{thm}

\begin{proof}
  Let $K = \ker \alpha$. Since $K \triangleleft G$ (by \cref{propo:image_of_hm_is_a_subgroup_n_kernel_of_hm_is_a_normal_subgroup}), $\faktor{G}{K}$ is a group. Let\sidenote{We must check that the function is well-defined, since cosets are not uniquely represented and so it is likely that a constructed mapping is not well-defined.}
  \begin{equation*}
    \bar{\alpha} : \faktor{G}{K} \to \img \alpha \text{ be defined by } Kg \mapsto \alpha(g)
  \end{equation*}
  Note that
  \begin{equation*}
    Kg = Kg_1 \iff gg_1^{-1} \in K \iff \alpha(gg_1^{-1}) = 1 \iff \alpha(g) = \alpha(g_1).
  \end{equation*}
  Thus $\bar{ \alpha }$ is well-defined and injective. Clearly, $\bar{ \alpha }$ is surjective. It remains to show that $\bar{\alpha}$ is a group homomorphism. $\forall g, h \in G$, we have
  \begin{equation*}
    \bar{\alpha}(Kg Kh) = \bar{\alpha}(Kgh) = \alpha(gh) = \alpha(g) \alpha(h) = \bar{\alpha}(Kg) \bar{\alpha}(Kh).
  \end{equation*}
  Therefore, we have that $\bar{\alpha}$ is an isomorphism and hence $\faktor{G}{\ker \alpha} \cong \img \alpha$ as desired. \qed
\end{proof}

% subsection isomorphism_theorems (end)

% section isomorphism_theorems_continued (end)

% chapter lecture_13_may_30_2018 (end)

\chapter{Lecture 14 Jun 01st 2018}%
\label{chp:lecture_14_jun_01st_2018}
% chapter lecture_14_jun_01st_2018

\section{Isomorphism Theorems (Continued 2)}%
\label{sec:isomorphism_theorems_continued_2}
% section isomorphism_theorems_continued_2

\subsection{Isomorphism Theorems (Continued)}%
\label{sub:isomorphism_theorems_continued}
% subsection isomorphism_theorems_continued

\begin{note}[Recall]
  In \autoref{thm:first_isomorphism_theorem}, we had that for a group homomorphism $\alpha : G \to H$ where $G$ and $H$ are groups,
  \begin{equation*}
    \faktor{G}{\ker \alpha} \cong \img \alpha
  \end{equation*}

  Now let $\alpha : G \to H$ be a group homomorphism, $K = \ker \alpha$, $\phi : G \to \faktor{G}{K}$ be the coset map, and $\bar{\alpha}$ be as defined in the proof of \autoref{thm:first_isomorphism_theorem}. We then have the following commutative diagram to illustrate the relationship between the three groups.
  \begin{center}
  \begin{tikzcd}
    G \arrow[dd, "\phi"'] \arrow[rrr, "\alpha"] &  &  & H \\
     &  &  &  \\
    G/K \arrow[rrruu, "\bar{\alpha}"'] &  &  & 
  \end{tikzcd}
  \end{center}
\end{note}

A natural question to ask after seeing the relationship is: Is $\bar{\alpha} \phi = \alpha$? If it is, is the definition of $\bar{\alpha}$ unique? The answer is: \hlnotec{YES!} on both accounts.

\begin{proof}
  Let $g \in G$. Then
  \begin{equation*}
    \bar{\alpha} \phi (g) = \bar{\alpha} \big( \phi(g) \big) = \bar{\alpha} (Kg) = \alpha(g)
  \end{equation*}

  Suppose $\alpha = \beta \phi$ where $\beta : \faktor{G}{K} \to H$. Then
  \begin{equation*}
    \beta (Kg) \overset{(1)}{=} \beta ( \phi(g) ) = \beta \phi(g) = \alpha(g) = \bar{\alpha} (Kg)
  \end{equation*}
  where $(1)$ is because $\phi$ is surjective by \cref{propo:propo_related_to_quotient_groups}. Therefore, we observe that $\beta = \bar{\alpha}$ for any $Kg \in \faktor{G}{K}$. This proves that $\bar{\alpha}$ is the unique homomorphism such that $\faktor{G}{K} \to H$ satisfying $\alpha = \bar{\alpha} \phi$.\qed
\end{proof}

With that, we have the following proposition.

\begin{propo}\index{factors through}
\label{propo:uniqueness_of_homomorphism_factors}
  Let $\alpha : G \to H$ be a group homomorphism, where $G$ and $H$ are groups. Let $K = \ker \alpha$. Then $\alpha$ factors uniquely as $\alpha = \bar{\alpha} \phi$< where $\phi : G \to \faktor{G}{K}$ is the coset map and $\bar{\alpha} : \faktor{G}{K} \to H$ is defined by
  \begin{equation*}
    \bar{\alpha} (Kg) = \alpha(g).
  \end{equation*}
  Note that $\phi$ is surjective and $\bar{\alpha}$ is injective.

  In such a scenario, we also say that $\alpha$ \hlnoteb{factors through} $\phi$.\sidenote{Reference for the terminology: \url{https://math.stackexchange.com/questions/68941/terminology-a-homomorphism-factors}.}
\end{propo}

\begin{eg}
  Let $G = \lra{g}$ be a cyclic group. Consider $\alpha : \mathbb{Z} \to G$, defined as
  \begin{equation*}
    \forall k \in \mathbb{Z} \quad \alpha(k) = g^k,
  \end{equation*}
  which is a group homomorphism. By definition, $\alpha$ is surjective. Note that
  \begin{equation*}
    \ker \alpha = \{k \in \mathbb{Z} : g^k = 1 \}.
  \end{equation*}
  We have, therefore, two cases to consider.
  \begin{itemize}
    \item \underline{$G$ is an infinite group} \\
      This would imply that $\ker \alpha = \{0\}$ since only $g^0 = 1$. Then by \autoref{thm:first_isomorphism_theorem}, we have that
      \begin{equation*}
        \faktor{\mathbb{Z}}{\ker \alpha} \cong G
      \end{equation*}
      Note that\sidenote{We are assuming that the group $\mathbb{Z}$ here works under the operation of addition, otherwise, if we employ multiplication, then $\mathbb{Z}$ would not be a group and $\alpha$ would not be a group homomorphism.}
      \begin{equation*}
        \faktor{\mathbb{Z}}{\ker \alpha} = \{ ( \ker \alpha ) k : k \in \mathbb{Z} \} = \{ 0 + k : k \in \mathbb{Z} \} = \mathbb{Z}.
      \end{equation*}
      Therefore
      \begin{equation*}
        \mathbb{Z} \cong G
      \end{equation*}

    \item \underline{$G$ is a finite group} \\
      Suppose that $\abs{G} = o(g) = n \in \mathbb{N}$, which is valid by \cref{crly:lagrange_s_theorem_crly1}. Then
      \begin{equation*}
        \ker \alpha = n \mathbb{Z}
      \end{equation*}
      Then by the \autoref{thm:first_isomorphism_theorem}, we have
      \begin{equation*}
        \faktor{\mathbb{Z}}{n \mathbb{Z}} \cong G.
      \end{equation*}
      Observe that
      \begin{equation*}
        \faktor{\mathbb{Z}}{n \mathbb{Z}} = \{ n \mathbb{Z} + k : k \in \mathbb{Z} \} = \mathbb{Z}_n
      \end{equation*}
      since the set in the middle is the definition of the set of integers modulo $n$.\sidenote{This is why we often see texts from various authors using $\faktor{\mathbb{Z}}{n \mathbb{Z}}$ to represent the set of integers modulo $n$.} Therefore,
      \begin{equation*}
        \mathbb{Z}_n \cong G
      \end{equation*}
  \end{itemize}

  Therefore, we have that
  \begin{equation*}
    \mathbb{Z} \cong G \text{ or } \mathbb{Z}_{o(g)} \cong G
  \end{equation*}
\end{eg}

\begin{thm}[Second Isomorphism Theorem]
\index{Second Isomorphism Theorem}
\label{thm:second_isomorphism_theorem}
  Let $H$ and $K$ be the subgroups of a group $G$ with $K \triangleleft G$. Then
  \begin{itemize}
    \item $HK$ is a subgroup of $G$;
    \item $K \triangleleft HK$;
    \item $H \cap K \triangleleft H$; and
    \item $\faktor{HK}{K} \cong \faktor{H}{H \cap K}$.
  \end{itemize}
\end{thm}

\begin{proof}
  Since $K \triangleleft G$, by \cref{lemma:product_of_groups_as_a_subgroup} and \cref{propo:product_of_normal_subgroups_is_normal}, we have that $HK = KH$ is a subgroup of $G$. Consequently, we have $K \triangleleft HK$, since $K$ is clearly a subgroup of $HK$ and $K \triangleleft G$, and so $\forall x \in HK \subseteq G$ we have that $gK = Kg$.

  Consider $\alpha : H \to \faktor{HK}{K}$, defined by\sidenote{Note that $Kh \in \faktor{HK}{K}$ since $h \in H \subseteq HK$.}
  \begin{equation*}
    \alpha(h) = Kh
  \end{equation*}
  Now if $x = kh \in KH = HK$, then
  \begin{equation*}
    Kx = K(kh) = Kh = \alpha(h).
  \end{equation*}
  Therefore, we have that $\alpha$ is surjective. Now by \cref{propo:properties_of_cosets}, observe that
  \begin{equation*}
    \ker \alpha = \{h \in H : Kh = K \} = \{h \in H h \in K \} = H \cap K.
  \end{equation*}
  Then by the \hyperref[thm:first_isomorphism_theorem]{First Isomorphism Theorem}, we have that
  \begin{equation*}
    \faktor{HK}{K} \cong \faktor{H}{H \cap K}.
  \end{equation*}
  Since we have that $\ker \alpha = H \cap K$ and $\ker \alpha \triangleleft H$, we have that $H \cap K \triangleleft H$.\qed
\end{proof}

\begin{thm}[Third Isomorphism Theorem]
\index{Third Isomorphism Theorem}
\label{thm:third_isomorphism_theorem}
  Let $K \subseteq H \subseteq G$ be groups, with $K \triangleleft G$ and $H \triangleleft G$. Then
  \begin{equation*}
    \faktor{H}{K} \, \triangleleft \, \faktor{G}{K} \text{ and } \left( \faktor{G}{K} \right) \Big/ \left( \faktor{H}{K} \right) \cong \faktor{G}{H}
  \end{equation*}
\end{thm}

\begin{proof}
  Define $\alpha : \faktor{G}{K} \to \faktor{G}{H}$ by $\alpha(Kg) = Hg$ for all $g \in G$. Clearly, $\alpha$ is surjective. Now if $Kg = Kg_1$, for any $g, g_1 \in G$, then $gg_1 \in K \subseteq H$. Therefore, $Hg = Hg_1$. Thus $\alpha$ is well-defined. Now
  \begin{equation*}
    \ker \alpha = \{Kg : Hg = H \} = \{ Kg : g \in H \} = \faktor{H}{K}.
  \end{equation*}
  Then
  \begin{equation*}
    \faktor{H}{K} = \ker \alpha \triangleleft \faktor{G}{K}.
  \end{equation*}
  By the \hyperref[thm:first_isomorphism_theorem]{First Isomorphism Theorem}, we have
  \begin{equation*}
    \left( \faktor{G}{K} \right) \Big/ \left( \faktor{H}{K} \right)
  \end{equation*}
  as required. \qed
\end{proof}

% subsection isomorphism_theorems_continued (end)

% section isomorphism_theorems_continued_2 (end)

\newthought{One reason} that we are interested in the symmetric group is that they contain all finite groups.

\begin{thmnonum}[Cayley's Theorem]
  If $G$ is a finite group of order $n$, then $G$ is isomorphic to a subgroup of $S_n$.
\end{thmnonum}

% chapter lecture_14_jun_01st_2018 (end)

\chapter{Lecture 15 Jun 04th 2018}%
\label{chp:lecture_15_jun_04th_2018}
% chapter lecture_15_jun_04th_2018

\section{Group Action}%
\label{sec:group_action}
% section group_action

\subsection{Cayley's Theorem}%
\label{sub:cayley_s_theorem}
% subsection cayley_s_theorem

\begin{thm}[Cayley's Theorem]
\index{Cayley's Theorem}
\label{thm:cayley_s_theorem}
  If $G$ is a finite group of order $n$, then $G$ is isomorphic to a subgroup of $S_n$.
\end{thm}

\begin{proof}
  Since $G$ is finite, let $G = \{g_1, g_2, ..., g_n\}$ and let $S_G$ be the permutation group of $G$. By identifying $g_i$ with $i$, where $1 \leq i \leq n$, we see that $S_G \cong S_n$ \sidenote{$S_G$ is the permutation group of $G$. We can think of $S_G$ as a group of permutations that permutes the index of the elements of $G$. Since there are $n$ indices, there are $n!$ ways to permute the indices, and so $\abs{S_G} = n! = \abs{S_n}$. Then we can certainly find some isomorphism from $S_G$ to $S_n$, and so $S_G \cong S_n$.}. Therefore, it suffices to find an injective homomorphism\sidenote{\hlwarn{Why do we need injectivity?} We need homomorphicity in order to invoke the \hyperref[thm:first_isomorphism_theorem]{First Isomorphism Theorem} so that we can get $G \cong \img \sigma \leq S_G \cong S_n$.} $\sigma : G \to S_G$.

  Consider the function $\mu_a : G \to G$, where $a \in G$, such that $\mu_a(g) = ag$ for all $g \in G$. Clearly, $\mu_a$ is surjective. Suppose $\mu_a = \mu_b$, where $b \in G$. Then $a = \mu_a (1) = \mu_b (1) = b$. Thus $\mu_a$ is also injective. It follows that $\mu_a \in S_G$ by definition.

  Now define the function $\sigma : G \to S_G$ such that $\sigma(a) = \mu_a$. Clearly, $\sigma$ is injective, since $\sigma(a) = \sigma(b) \implies \mu_a = \mu_b$. Observe that $\sigma(ab) = \mu_{ab} = ab = \mu_a \mu_b$. Thus $\sigma$ is a group homomorphism. Note that $\ker \sigma = \{1\}$, the trivial group. It follows from the \hyperref[thm:first_isomorphism_theorem]{First Isomorphism Theorem} that $G \cong \im \sigma \leq S_G \cong S_n$.\sidenote{We shall use $H \leq G$ to denote that $H$ is a subgroup of $G$ from here on.} \sidenote{This is a result from \cref{propo:image_of_hm_is_a_subgroup_n_kernel_of_hm_is_a_normal_subgroup}} \qed
\end{proof}

Cayley's Theorem is, however, too strong at times. We can certainly find a smaller integer $m$ such that $G$ is contained in $S_m$. Consider the following example.

\begin{eg}\label{eg:a_smaller_subgroup_of_Sn_that_is_iso_to_G}
  Let $H \leq G$ with $[G : H] = m < \infty$. Let $X = \{g_1 H, g_2 H, ..., g_m H\}$ be the set of all distinct left cosets of $H$ in $G$ \sidenote{This is simply a consequence of $[G : H] = m$.}. For $a \in G$, define $\lambda_a : X \to X$ by $\lambda_a (gH) = agH$, $gH \in X$.

  Note that $\lambda_a$ is a bijection\sidenote{This is true as shown in the proof above, but it can also serve as a tiny exercise.}, and so $\lambda_a \in S_X$, the permutation group of $X$. Consider the mapping $\tau : G \to S_X$ defined by $\tau (a) = \lambda_a$ for $a \in G$. Note that $\forall a, b \in G$, $\lambda_{ab} = \lambda_a \lambda_b$. Thus $\tau$ is a homomorphism. Note that if $a \in \ker \tau$, then $aH = H$ which implies $a \in H$ by \cref{propo:properties_of_cosets}. Thus $\ker \tau \subseteq H$.
\end{eg}

From the example above, if we apply the \hyperref[thm:first_isomorphism_theorem]{First Isomorphism Theorem}, then
\begin{equation*}
  \faktor{G}{\ker \tau} \cong \img \tau \leq S_X \cong S_m \leq S_n.
\end{equation*}
This is the result that we desired.

\begin{thm}[Extended Cayley's Theorem]
\index{Extended Cayley's Theorem}
\label{thm:extended_cayley_s_theorem}
  Let $H \leq G$ with $[G : H] = m < \infty$. If $G$ has no normal subgroup contained in $H$ except for the trivial subgroup $\{1\}$, then $G$ is isomorphic to a subgroup of $S_m$.
\end{thm}

\begin{proof}
  By our assumption, let $X$ be the set of all distinct left cosets of $H$ in $G$. Then we have that $\abs{X} = m$ and so $S_X \cong S_m$ \sidenote{This is as argued in the proof of \hyperref[thm:cayley_s_theorem]{Cayley's Theorem}.}. From \cref{eg:a_smaller_subgroup_of_Sn_that_is_iso_to_G}, we have that there exists a group homomorphism $\tau : G \to S_X$ with $K := \ker \tau \subseteq H$. So by the \hyperref[thm:first_isomorphism_theorem]{First Isomorphism Theorem}, we have that
  \begin{equation*}
    \faktor{G}{K} \cong \img \tau.
  \end{equation*}
  Since $K \subseteq H$ and $K \triangleleft G$, we have, by assumption, that $K = \{1\}$. It follows that
  \begin{equation*}
    G \cong \img \tau \leq S_X \cong S_m.
  \end{equation*}\qed
\end{proof}

\begin{crly}
\label{crly:subgroup_with_prime_index_dividing_order_of_group_is_normal}
  Let $\abs{G} = m \in \mathbb{N}$ and $p$ the smallest prime such that $p \big| m$. If $H \leq G$ with $[G : H] = p$, then $H \triangleleft G$.
\end{crly}

\begin{proof}
  Let $X$ be the set of all distinct left cosets of $H$ in $G$. We have $\abs{X} = p$ and so $S_X \cong S_p$. Let $\tau : G \to S_X \cong S_p$ be as defined in \cref{eg:a_smaller_subgroup_of_Sn_that_is_iso_to_G}, with $K := \ker \tau \subseteq H$. By the \hyperref[thm:first_isomorphism_theorem]{First Isomorphism Theorem}, we have that
  \begin{equation*}
    \faktor{G}{K} \cong \img \tau \leq S_X \cong S_p,
  \end{equation*}
  i.e. $\faktor{G}{K}$ is isomorphic to a subgroup of $S_p$. Therefore, by \hyperref[thm:lagrange_s_theorem]{Lagrange's Theorem}, we have that $\abs{\faktor{G}{K}} \Bigg| \; p!$.

  Also, since $K \subseteq H$, if $[H : K] = k \in \mathbb{N}$, then
  \begin{equation*}
    \abs{\faktor{G}{K}} \overset{(1)}{=}\frac{\abs{G}}{\abs{K}} = \frac{\abs{G}}{\abs{H}} \cdot \frac{\abs{H}}{\abs{K}} = pk,
  \end{equation*}
  where $(1)$ is by \cref{propo:propo_related_to_quotient_groups}. Therefore we have that $pk \, | p!$ and so $k \, | \, (p - 1)!$.
  
  Note that $k \, | \, \abs{H}$ \sidenote{This is clear since $\abs{H} = k \abs{K}$.}, which divides $\abs{G}$, and $p$ is the smallest prime dividing $\abs{G}$. Thus evrey prime divisor of $k$ must be $\geq p$.\sidenote{By the \hlnotea{Fundamental Theorem of Arithmetic}, and since $k$ is finite, let $k = p_1^{a_1} p_2^{a_2} ...p_m^{a_m}$, where $p_i$'s are distinct primes and $a_i \in \mathbb{N}$ are the multiplicities of the $i$\textsuperscript{th}, and by the \hlnotea{Well-Ordering Principle}, let $p_i < p_{i + 1}$. Then we have, for some $b = b_1^{c_1} b_2^{c_2} \hdots b_j^{c_j} \in \mathbb{N}$ where the $b_i$'s are distint primes, $b_i < b_{i + 1}$, and $c_i \in \mathbb{N} \cup \{0\}$,
  \begin{equation*}
    m = kb = p_1^{a_1} p_2^{a_2} \hdots p_m^{a_m} b_1^{c_1} b_2^{c_2} \hdots b_j^{c_j}.
  \end{equation*}
  Since $p$ is the smallest prime that divides $m$, we have
  \begin{align*}
    p &= \min \{p_1, p_2, ..., p_m, b_1, b_2, ..., b_j\} \\
      &= \min \{ p_1, b_1 \}
  \end{align*}
  } Thus $k = 1$, which implies that $K = H$. Therefore, $H \triangleleft G$ as desired.\qed
\end{proof}

% subsection cayley_s_theorem (end)

\subsection{Group Action}%
\label{sub:group_action}
% subsection group_action

\begin{defn}[Group Action]\index{Group Action}
\label{defn:group_action}
  Let $G$ be a group, $X$ a non-empty set. A \hlnoteb{group action} of $G$ on $X$ is a mapping $G \times X \to X$ denoted as $(a, x) \to ax$ such that
  \begin{enumerate}
    \item $1 \cdot x = x$, $x \in X$
    \item $a \cdot (b \cdot x) = (ab) \cdot x$, $a, b \in G, \, x \in X$
  \end{enumerate}
  In this case, we say $G$ \hldefn{acts on} $X$.
\end{defn}

% subsection group_action (end)

% section group_action (end)

% chapter lecture_15_jun_04th_2018 (end)

\chapter{Lecture 16 Jun 06 2018}%
\label{chp:lecture_16_jun_06_2018}
% chapter lecture_16_jun_06_2018

\section{Group Action (Continued)}%
\label{sec:group_action_continued}
% section group_action_continued

\subsection{Group Action (Continued)}%
\label{sub:group_action_continued}
% subsection group_action_continued

\begin{remark}
  Let $G$ be a group acting on a set $X$. For $a, b \in G$, and $x, y \in X$, we have that
  \begin{equation*}
    a \cdot x = b \cdot y \iff (b^{-1} a) \cdot x = y.
  \end{equation*}
  In particular, we have
  \begin{equation*}
    a \cdot x = a \cdot y \iff x = y.
  \end{equation*}
\end{remark}

For $a \in G$, define $\sigma_a : X \to X$ by $\sigma_a(x) = a \cdot x$ for all $x \in X$. In A3, we will be showing that\sidenote{This will be added after the assignment.}:
\begin{enumerate}
  \item $\sigma_a \in S_X$, the permutation group of $X$; and
  \item The function $\Theta : G \to S_X$ given by $\Theta(a) = \sigma_a$ is a group homomorphism with
    \begin{equation*}
      \ker \Theta = \{a \in G : a \cdot x = x, \, x \in X \}.
    \end{equation*}
\end{enumerate}

Note that the group homomorphism $\Theta : G \to S_X$ gives an \hlimpo{equivalent definition} of a \hldefn{Group Action} of $G$ on $X$. If $X = G$, $\abs{G} = n$ and $\ker \Theta = \{1\}$ \sidenote{This is also called a \hldefn{faithful group action}.}, then the map $\Theta : G \to S_G \cong S_n$ shows that $G$ is isomorphic to a subgroup of $S_n$ \sidenote{
\begin{ex}
  Verify that $G$ is indeed isomorphic to a subgroup of $S_n$ using the given information and the equivalent definition of a group action.
\end{ex}
}, which the equivalent statement of \hyperref[thm:cayley_s_theorem]{Cayley's Theorem}\index{Cayley's Theorem}.

\begin{eg}\label{eg:group_action_by_conjugation}
  If $G$ is a group, let $G$ act on itself by $a \cdot x = a \cdot x \cdot a^{-1}$, for all $a, x \in G$. Note that the axioms of a group action is satisfied:
  \begin{enumerate}
    \item $1 \cdot x = 1 \cdot x \cdot 1^{-1} = x$; and
    \item $a \cdot (b \cdot x) = a \cdot ( b \cdot x \cdot b^{-1} ) \cdot a = ab \cdot x \cdot (ab)^{-1} = (ab) \cdot x$.
  \end{enumerate}
  In this case, we say that $G$ \hlimpo{acts on itself by} \hldefn{conjugation}.
\end{eg}

\begin{defn}[Orbit \& Stabilizer]\index{Orbit}\index{Stabilizer}
\label{defn:orbit_n_stabilizer}
  Let $G$ be a group acting on a set $X$, and $x \in X$. We denote by\marginnote{There is no standardized way of expressing the orbit and the stabilizer, i.e. the notation for orbit and stabilizers will be different across many references.}
  \begin{equation*}
    G \cdot x = \{g \cdot x : \forall g \in G \}
  \end{equation*}
  the \hlnoteb{orbit} of $X$ and
  \begin{equation*}
    S(x) = \{g \in G : g \cdot x = x \} \subseteq G
  \end{equation*}
  the \hlnoteb{stabilizer} of $X$.
\end{defn}

\begin{propo}
\label{propo:stabilizer_is_a_subgroup_and_index_of_stabilizer_is_order_of_orbit}
  Let $G$ be a group acting on a set $X$ an $x \in X$. Let $G \cdot x$ and $S(x)$ be the orbit and stabilizer of $X$ respectively. Then
  \begin{enumerate}
    \item $S(x) \leq G$
    \item there is a bijection from $G \cdot x$ to $\{g S(x) : g \in G \}$ and thus $\abs{G \cdot x} = [G : S(x)]$.
  \end{enumerate}
\end{propo}

\begin{proof}
  \begin{enumerate}
    \item Since $1 \cdot x = x$, we have $1 \in S(x)$. If $g, h \in S(x)$, then
      \begin{equation*}
        gh \cdot x = g \cdot (h \cdot x) = g \cdot x = x
      \end{equation*}
      i.e. $S(x)$ is closed under ``composition of group action''. Also note that
      \begin{equation*}
        g^{-1} \cdot x = g^{-1} \cdot (g \cdot x) = (g^{-1}g) \cdot x = 1 \cdot x = 1.
      \end{equation*}
      Thus the inverse of each element is also in $S(x)$. Therefore, by the \hlnotea{Subgroup Test}, $S(x) \leq G$.

    \item For the sake of simplicity, let us write $S = S(x)$. Consider the map
      \begin{equation*}
        \phi: G \cdot x \to \{g S(x) : g \in G\}
      \end{equation*}
      defined by $\phi(g \cdot x) = gS$ \sidenote{We go with the most simplistic and rather naive kind of function here.}. To verify that the map is well-defined, note that
      \begin{align*}
        g \cdot x = h \cdot x &\iff (h^{-1} g) \cdot x = x = 1 \cdot x \\
                              &\iff \phi(h^{-1} g \cdot x) = \phi( 1 \cdot x ) \\
                              &\iff h^{-1}g S = 1 \cdot S = S \\
                              &\iff gS = hS
      \end{align*}
      We also observe that $\phi$ is injective. It is also clear that $\phi$ is onto, and therefore we have that $\phi$ is a bijection. It follows that
      \begin{equation*}
        \abs{G \cdot x} = \abs{ \{gS : g \in G \} } = [G : S]
      \end{equation*}
  \end{enumerate}\qed
\end{proof}

\begin{thm}[Orbit Decomposition Theorem]
\index{Orbit Decomposition Theorem}
\label{thm:orbit_decomposition_theorem}
  Let $G$ be a group acting on a non-empty finite set $X$. Let
  \begin{equation*}
    X_f = \{x \in X : a \cdot x = x, \forall a \in G \}
  \end{equation*}
  (Note that $x \in X_f \iff \abs{G \cdot x} = 1$)\sidenote{Notice that
  \begin{align*}
    x \in X_f &\iff \forall a \in G \enspace a \cdot x = x \\
      &\iff \forall g \cdot x \in G \cdot x \enspace g \cdot x = x \\
      &\iff \abs{G \cdot x} = 1
  \end{align*}}

  Let $G \cdot x_1, \, G \cdot x_2, \, ..., \, G \cdot x_n$ denote the distinct nonsingleton orbits (i.e. $\abs{G \cdot x_i} > 1$ for all $1 \leq i \leq n$). Then
  \begin{equation*}
    \abs{X} = \abs{X_f} + \sum_{i = 1}^{n} [ G : S(x_i) ].
  \end{equation*}
\end{thm}

\begin{proof}
  Note that for $a, b \in G$ and $x, y \in X$,
  \begin{align*}
    a \cdot x = b \cdot y &\overset{\text{WLOG}}{\iff} (b^{-1}a) \cdot x = y \\
          &\iff y \in G \cdot x \\
          &\overset{(1)}{\iff} G \cdot x = G \cdot y
  \end{align*}
  where $(1)$ is the conclusion after consider the other case where $(a^{-1}b) \cdot y = x$.

  Thus, we see that the two orbits are either disjoint or the same, but not both. It follows that the orbits form a disjoint union of $X$. Since  $x \in X_f \iff \abs{G \cdot x} = 1$, the set $X \setminus X_f$ contains all nonsingleton orbits, which are disjoint. It follows that
  \begin{equation*}
    \abs{X} = \abs{X_f} + \sum_{i = 1}^{n} \abs{G \cdot x_i} \overset{(2)}{=} \abs{X_f} + \sum_{i = 1}^{n} [G : S(x_i)]
  \end{equation*}
  where $(2)$ is by \cref{propo:stabilizer_is_a_subgroup_and_index_of_stabilizer_is_orde}.\qed
\end{proof}

% subsection group_action_continued (end)

% section group_action_continued (end)

% chapter lecture_16_jun_06_2018 (end)

\chapter{Lecture 17 Jun 08 2018}%
\label{chp:lecture_17_jun_08_2018}
% chapter lecture_17_jun_08_2018

\section{Group Action (Continued 2)}%
\label{sec:group_action_continued_2}
% section group_action_continued_2

\subsection{Group Action (Continued 2)}%
\label{sub:group_action_continued_2}
% subsection group_action_continued_2

\begin{note}[Recall \cref{thm:orbit_decomposition_theorem}]
  Let $G$ act on a finite set $X \neq \emptyset$. Let\sidenote{$X_f$ is also called the set of elements of $X$ that are fixed by the action of $G$.}
  \begin{equation*}
    X_f = \{x \in X : a \cdot x = x, \, a \in G \}
  \end{equation*}
  Let $G \cdot x_1, G \cdot x_2, ..., G \cdot x_n$ be distinct nonsingleton orbits (ie. $\abs{G \cdot x_i} > 1$). Then
  \begin{equation*}
    \abs{X} = \abs{X_f} + \sum_{i=1}^{n} [ G : S(x_i) ].
  \end{equation*}
\end{note}

\begin{eg}[Conjugacy Class \& Centralizer]
  Let $G$ be a finite group acting on itself by \hlnoteb{conjugation}. In the context of \cref{thm:orbit_decomposition_theorem}, we have that
  \begin{align*}
    X &= G \\
    G_f &= \{x \in G : gxg^{-1} = x, \, g \in G \} \\
        &= \{x \in G : gx = xg, \, g \in G \} = Z(G),
  \end{align*}
  where we recall that $Z(G)$ is the \hyperref[defn:center_of_a_group]{center of $G$}. Now for any $x \in G$, we have
  \begin{equation*}
    G \cdot x = \{ gxg^{-1} : g \in G \},
  \end{equation*}
  which is known as the \hldefn{conjugacy class} of $x$. We also have
  \begin{equation*}
    S(x) = \{g \in G : gxg^{-1} = x \} = \{g \in G : gx = xg \} = C_G(x),
  \end{equation*}
  which is called the \hldefn{centralizer} of $x$.
\end{eg}

Putting the above example with \cref{thm:orbit_decomposition_theorem}, we have the following corollary.

\begin{crly}[Class Equation]
\index{Class Equation}
\label{crly:class_equation}
  Let $G$ be a finite group and $\{gx_1 g^{-1} : g \in G \}, \, ..., \, \{g x_n g^{-1} : g \in G \}$ denote the distinct nonsingleton conjugacy classes. Then
  \begin{equation*}
    \abs{G} = \abs{Z(G)} + \sum_{i=1}^{n} [ G : C_G(x_i) ].
  \end{equation*}
\end{crly}

\begin{lemma}
\label{lemma:abs_x_equiv_abs_x_f}
  Let $G$ be a group of order $o^m$, where $p$ prime and $m \in \mathbb{N}$, which acts on a finite set $X$. Let
  \begin{equation*}
    X_f = \{ x \in X : a \cdot x = x, \, a \in G \}.
  \end{equation*}
  Then we have
  \begin{equation*}
    \abs{X} \equiv \abs{X_f} \mod p
  \end{equation*}
\end{lemma}

\begin{proof}
  By the \hyperref[thm:orbit_decomposition_theorem]{Orbit Decomposition Theorem}, we have that
  \begin{equation*}
    \abs{X} = \abs{X_f} + \sum_{i=1}^{n} [ G : S(x_i) ],
  \end{equation*}
  where $[ G : S(x_i) ] > 1$ for $1 \leq i \leq n$. For any $x_i$, by \hyperref[thm:lagrange_s_theorem]{Lagrange's Theorem}, $[ G : S(x_i) ] \big| \abs{G} = p^m$. Since $[ G : S(x_i) ] > 1$, we have, by the \hlnotea{Fundamental Theorem of Arithmetic}, that $[ G : S(x_i) ]$ must be a multiple of $p$, i.e. $p$ divides $[ G : S(x_i) ]$, for all $i$. Therefore, $p \, | \, \left( \abs{X} - \abs{X_f} \right)$, i.e.
  \begin{equation*}
    \abs{X} \equiv \abs{X_f} \mod p,
  \end{equation*}
  as required. \qed
\end{proof}

\newthought{Recall} \hyperref[thm:lagrange_s_theorem]{Lagrange's Theorem}: If $G$ is finite and $g \in G$, then
\begin{equation*}
  o(g) \, \big| \, \abs{G}.
\end{equation*}

An interesting question to ask here is: Is the converse true? I.e., given a group $G$ with an integer $m$ such that $m \, \big| \, \abs{G}$, does $G$ contain an element of order $m$?

Consider $K_4$, the Klein $4$-group. Note that all elements of $K_4$ have order at most $2$, but $4 | \abs{K_4} = 4$.

Now if $m$ is some prime, is the converse still true?

\begin{thm}[Cauchy's Theorem]
\index{Cauchy's Theorem}
\label{thm:cauchy}
  Let $p$ be a prime, $G$ be a finite group. If $p \, \big| \abs{G}$, then $G$ contains an element of order $p$.
\end{thm}

\begin{proof}[McKay]
  Let $\abs{G} = n$. Suppose $p \, | \, n$. Let
  \begin{equation*}
    X = \{(a_1, ..., a_p) : a_i \in G, \, a_1 \hdots a_p = 1 \}.
  \end{equation*}
  Note that $X \neq \emptyset$, since $(1, ..., 1) \in X$ (so the proof is not vacuous). Take any $a_1, ..., a_{p - 1} \in G$, then $a_p$ is uniquely determined, i.e.
  \begin{equation*}
    a_p = (a_1 \hdots a_{p - 1})^{-1}.
  \end{equation*}
  Now for each $a_i$, we have $n$ choices, thus $\abs{X} = n^{p - 1}$.\sidenote{Convince yourself why this is true.}

  Let $\mathbb{Z}_p = ( \mathbb{Z}_p, + )$ act on $X$ by ``cycling'', i.e. $\forall k \in \mathbb{Z}_p$,
  \begin{equation*}
    k \cdot (a_1, a_2, ..., a_p) = (a_{k + 1}, a_{k + 2}, ..., a_p, a_1, ..., a_k).
  \end{equation*}
  \sidenote{We want to use \cref{thm:orbit_decomposition_theorem} from here.} Note that \\
  \begin{aligned}
    $(a_1, ..., a_p) \in X_f$ &$\iff$ every cycled shift of $(a_1, ..., a_p)$ is itself \\
      &$\iff$ $a_1 = a_2 = \hdots = a_p$ and $a_1 a_2 ... a_p = 1$
  \end{aligned}
  i.e. all of the components of the $p$-tuple are the same. Now if $(a_1, ..., a_p)$ has at least 2 distinct components, then its orbits must have $p$ elements. In other words, for some $r \in \mathbb{N}$, for each $1 \leq i \leq r$, we have that $[ G : S(x_i) ] = p$. Then, by the \hyperref[thm:orbit_decomposition_theorem]{Orbit Decomposition Theorem},
  \begin{gather*}
    n^{p - 1} = \abs{X} = \abs{X_f} + \sum_{i=1}^{r} [ G : S(x_i) ] \\
    \abs{X_f} = n^{p - 1} - rp.
  \end{gather*}
  We observe that $\abs{X_f}$ is indeed divisible by $p$ and is non-zero, since $(1, ..., 1) \in X_f$. Therefore, there exists some $a \neq 1 \in G$, such that $(a, ..., a) \in X_f$, i.e. $a^p = 1$. We know that $p$ is the smallest power by construction, and therefore $o(a) = p$ as required. \qed
\end{proof}

% subsection group_action_continued_2 (end)

% section group_action_continued_2 (end)

% chapter lecture_17_jun_08_2018 (end)

\chapter{Lecture 18 Jun 13th 2018}%
\label{chp:lecture_18_jun_13th_2018}
% chapter lecture_18_jun_13th_2018

\section{Finite Abelian Groups}%
\label{sec:finite_abelian_groups}
% section finite_abelian_groups

\subsection{Primary Decomposition}%
\label{sub:primary_decomposition}
% subsection primary_decomposition

\begin{note}[Notation]
  Let $G$ be an abelian group and $m \in \mathbb{Z}.$ We define
  \begin{equation*}
    G^{(m)} := \{g \in G : g^m = 1\}
  \end{equation*}
\end{note}

\begin{propo}[Group of Elements of the Same Order is a Subgroup]
\label{propo:group_of_elements_of_the_same_order_is_a_subgroup}
  Let $G$ be an abelian group. Then $G^{(m)} \leq G$.
\end{propo}

\begin{proof}
  Note that $1^m = 1 \in G^{(m)}$. $\forall g, h \in G^{(m)}$, since $G$ is abelian, we have that\sidenote{Pay attention that this is only true if $G$ is abelian.}
  \begin{equation*}
    \left( gh \right)^m = g^m h^m = 1 \cdot 1 = 1.
  \end{equation*}
  Therefore $gh \in G^{(m)}$. Also, for $g \in G^{(m)}$, we have 
  \begin{equation*}
    \left( g^{-1} \right)^m = \left( g^m \right)^{-1} = 1.
  \end{equation*}
  Thus $g^{-1} \in G^{(m)}$. By the \hlnotea{Subgroup Test}, we have that $G^{(m)} \leq G$. \qed
\end{proof}

\begin{propo}[Decomposition of a Finite Abelian Group]
\label{propo:decomposition_of_a_finite_abelian_group}
  Let $G$ be a finite abelian group with $\abs{G} = mk$ such that $\gcd(m, k) = 1$. Then
  \begin{enumerate}
    \item $G \cong G^{(m)} \times G^{(k)}$; and
    \item $\abs{G^{(m)}} = m$ and $\abs{G^{(k)}} = k$.
  \end{enumerate}
\end{propo}

\begin{proof}
  \begin{enumerate}
    \item Since $G$ is abelian, $G^{(m)} \triangleleft G$ and $G^{(k)} \triangleleft G$.

      \underline{Claim 1}: $G^{(m)} \cap G^{(k)} = \{1\}$ \\
      \textbf{Proof of Claim 1:} $\forall g \in G^{(m)} \cap G^{(k)}$, $g^m = 1 = g^k$ \\
      $\because \gcd(m, k) = 1$, by \hlnotea{Bezout's Lemma}, $\exists x, y \in \mathbb{Z} \quad 1 = mx + ky$ \\
      $\implies g = g^1 = g^{mx + ky} = ( g^m )^x ( g^k )^y = 1 \cdot 1 = 1$ \\
      $\implies G^{(m)} \cap G^{(k)} = \{1\}$ as claimed.

      \underline{Claim 2}: $G = G^{(m)}G^{(k)}$ \sidenote{Recall that this is the \hyperref[defn:product_of_groups]{Product}}\\
      $\forall g \in G \enspace \because o(g) = mk \quad 1 = g^{mk} = ( g^k )^m = ( g^m )^k$ \\
      It follows that $g^k \in G^{(m)}$ and $g^m \in G^{(k)}$. From \textbf{Claim 1} and by abelianness, we have that
      \begin{equation*}
        g = g^{mx + ky} = (g^k)^y (g^m)^x \in G^{(m)}G^{(k)}
      \end{equation*}
      Thus $G \subseteq G^{(m)}G^{(k)}$. On the other hand, since $G^{(m)} \triangleleft G$ and $G^{(k)} \triangleleft G$, by \cref{lemma:product_of_groups_as_a_subgroup}, we have that $G^{(m)}G^{(k)} \leq G$ and hence $G^{(m)} G^{(k)} \subseteq G$. Thus $G = G^{(m)}G^{(k)}$ as claimed.

      From \textbf{Claims 1 and 2}, we can conclude by \cref{crly:group_iso_with_cross_prod_of_its_subgroups}\sidenote{Should this not be \cref{thm:product_and_cross_product_of_normal_subgroups_are_isomorphic}?}, that $G \cong G^{(m)} \times G^{(k)}$ as required.

    \item Write $\abs{G^(m)} = m'$ and $\abs{G^{(k)}} = k'$. By part $(1)$, we have that $mk = \abs{G} = m'k'$.

      \underline{Claim 3}: $\gcd(m, k') = 1$ \\
      Suppose not\\
      $\implies \exists p$ prime $ \quad p \, | \, m$ and $p \, | \, k'$\\
      $\implies \exists g \in G^{(k)} \quad o(g) = p \qquad \because \hyperref[thm:cauchy]{\text{Cauchy's Theorem}}$\\
      Now $p \, | \, m \implies \exists q \in \mathbb{Z} \quad m = pq$\\
      $\implies g^m = g^{pq} = 1 \enspace \because o(g) = p$\\
      $\implies g \in G^{(m)}$.\\
      By part $(1)$, we have that $g \in G^{(m)} \cap G^{(k)} = \{1\} \implies g = 1$, which contradicts the fact that $o(g) = p$. Thus $\gcd(m, k') = 1$ as claimed. Similarly, we can get that $\gcd(m', k) = 1$.

      Notice that $mk = m'k' \implies m \, | \, m'k'$ \\
      $\implies m \, | \, m' \quad \because \gcd(m, k') = 1$
      and similarly $k \, | \, k'$. But then $mk = m'k'$ would imply that $m' = m$ and $k' = k$.
  \end{enumerate}\qed
\end{proof}

As a direct consequence of \cref{propo:decomposition_of_a_finite_abelian_group}, we have the following:

\begin{thm}[Primary Decomposition]
\index{Primary Decomposition}
\label{thm:primary_decomposition}
  Let $G$ be a finite abelian group with $\abs{G} = p_1^{n_1} \hdots p_k^{n_k}$, where $p_1, ..., p_k$ are distinct primes, and $n_1, ..., n_k \in \mathbb{N}$. Then
  \begin{enumerate}
    \item $G \cong G^{\left(p_1^{n_1}\right)} \times \hdots \times G^{\left(p_k^{n_k}\right)}$; and
    \item $\forall i \enspace 1 \leq i \leq k \quad \abs{G^{\left(p_i^{n_i}\right)}} = p_i^{n_i}$.
  \end{enumerate}
\end{thm}

% subsection primary_decomposition (end)

\subsection{p-Groups}%
\label{sub:p_groups}
% subsection p_groups

On a related note of the groups $G^{\left(p_i^{n_i}\right)}$, we define the following:

\begin{defn}[p-Group]\index{p-Group}
\label{defn:p_group}
  Let $p$ be a prime. A \hlnoteb{p-group} is a group in which every element has an order that is a non-negative power of $p$.
\end{defn}

\begin{propo}[p-Groups are Finite]
\index{p-Groups are Finite}
\label{propo:p_groups_are_finite}
  A finite group $G$ is a p-group $\iff$ $\abs{G}$ is a power of $p$ (including $p^0$).
\end{propo}

\begin{proof}
  $(\impliedby)$ If $\abs{G} = p^\alpha$ for some $\alpha \in \mathbb{N} \cup \{0\}$ and $g \in G$, by \cref{crly:lagrange_s_theorem_crly1}, $o(g) \, | \, p^\alpha$ \\
  $\implies G$ is a p-group.

  \noindent $(\implies)$ Consider the contrapositive and let $\abs{G} = p^n p_2^{n_2} \hdots p_k^{n_k}$ where $p, p_2, ..., p_k$ are distinct primes, $n \in \mathbb{N} \cup \{0\}$, and $n_2, ..., n_k \in \mathbb{N}$. For $k \geq 2$, by \hyperref[thm:cauchy]{Cauchy's Theorem}, $p_2 \, | \, \abs{G}$\\
  $\implies \exists g_1 \in G \quad o(g_1) = p_2$\\
  $\implies G$ is not a p-group.\\
  Therefore, our desired result follows.\qed
\end{proof}

\newthought{Our end goal} here is to prove to ourselves that all finite abelian groups can be written as cross products of cyclic groups, i.e. if $G$ is an abelian group, then
\begin{equation*}
  G \cong C_1 \times C_2 \times \hdots \times C_n.
\end{equation*}
With \cref{thm:primary_decomposition}, we have that
\begin{equation*}
  G \cong G_1 \times G_2 \times \hdots \times G_n.
\end{equation*}
The following proposition will enable us to get to our goal from our current position:

\begin{propononum}[Finite Abelian p-Groups of order $p$ are Cyclic]
If $G$ is a finite abelian p-group that contains only one subgroup of order $p$, where $p$ is prime, then $G$ is cyclic. In other words, if a finite abelian p-group is not cyclic, then it must have at least $2$ subgroups of order $p$.
\end{propononum}

% subsection p_groups (end)

% section finite_abelian_groups (end)

% chapter lecture_18_jun_13th_2018 (end)

\chapter{Lecture 19 Jun 15th 2018}%
\label{chp:lecture_19_jun_15th_2018}
% chapter lecture_19_jun_15th_2018

\section{Finite Abelian Groups (Continued)}%
\label{sec:finite_abelian_groups_continued}
% section finite_abelian_groups_continued

\subsection{p-Groups (Continued)}%
\label{sub:p_groups_continued}
% subsection p_groups_continued

\begin{note}[Recall]
  Recall the definition of a $p$-group:

  $G$ is a $p$-group if the order of all of its elements is a non-negative power of $p \iff \abs{G} = p^k$ for some $k \in \mathbb{N} \cup \{0\}$.
\end{note}

We shall now proceed to prove the proposition mentioned by the end of last class.

\begin{propo}[Finite Abelian $p$-Groups of Order $p$ are Cyclic]
\label{propo:finite_abelian_p_groups_of_order_p_are_cyclic}
  If $G$ is a finite abelian $p$-group that contains only $1$ subgroup of order $p$, then $G$ is cyclic. In other words, if a finite abelian $p$-group is not cyclic, then $G$ has at least $2$ subgroups of order $p$.
\end{propo}

\begin{proof}
  Since $G$ is finite, let $y \in G$ have maximal order. \\
  \underline{Claim}: $G = \lra{y}$ \\
  \textbf{Proof of Claim}: Suppose not. Since $\lra{y} \triangleleft G$ \sidenote{We have $\lra{y} \leq G$ and $G$ is abelian.}, consider the quotient group $\faktor{G}{\lra{y}}$, which is, therefore, a nontrivial $p$-group, since $\abs{\lra{y}} = p$. By \hyperref[thm:cauchy]{Cauchy's Theorem}, we know that $\exists z \in \faktor{G}{\lra{y}}$ such that $o(z) = p$ \sidenote{Note that we have $\faktor{G}{\lra{y}}$ is a $p$-group $\iff$ $\abs{ \faktor{G}{\lra{y}} } = p^k$ for some $k \in \mathbb{N} \cup \{0\}$. The existence of our chosen $z$ follows from there by Cauchy's Theorem.}. In particular, we have that $z \neq 1$ \sidenote{If $z = 1$, then its order would not be $p$.}. Consider the coset map
  \begin{equation*}
    \pi : G \to \faktor{G}{\lra{y}}.
  \end{equation*}
  Let $x \in G$ such that $\pi(x) = z$ \sidenote{Recall that $\pi$ is surjective by \cref{propo:propo_related_to_quotient_groups}.}. Since
  \begin{equation*}
    \pi( x^p ) = \pi(x)^p = z^p = 1,
  \end{equation*}
  we have that $x^p$ gets mapped to $1$ by $\pi$, i.e. $x^p \in \lra{y}$. \\
  $\implies \exists m \in \mathbb{Z}$ such that $x^p = y^m$. We shall consider two cases: \\
  \noindent\textbf{Case 1}: $p \nmid m$. \\
  $\because p \nmid m$, we have that $\gcd(m, \abs{\lra{y}}) = 1$, and hence by \cref{propo:other_generators_in_the_same_group} \sidenote{ \begin{propononum}[Proposition 18]
  Let $G = \lra{g}$ with $o(g) = n \in \mathbb{N}$. We have
  \begin{equation*}
    G = \lra{g^k} \iff \gcd(k, n) = 1
  \end{equation*}
  \end{propononum} }, we have that $o\left(y^m\right) = o(y)$. Because $y$ has maximal order, we have
  \begin{equation*}
    o( x^p ) \overset{(1)}{<} o(x) \leq o(y) = o(y^m) = o(x^p)
  \end{equation*}
  where note that $(1)$ is true because $x$ would need to take more powers of $p$ than $x^p$ to get back to $1$. We observe that we have arrived at a contradiction. \\
  \noindent\textbf{Case 2}: $p \mid m$.\\
  $p \mid m \implies \exists k \in \mathbb{Z} \enspace m = pk \implies x^p = y^m = y^{pk}$ \\
  $\because G$ is abelian, we have that $\left( xy^-k \right)^p = 1$. \\
  By assumption, there is only one subgroup of $G$ of order $p$, call it $H$. Thus $xy^k \in H$. On the other hand, by the \hyperref[thm:fundamental_theorem_of_finite_cyclic_groups]{Fundamental Theorem of Finite Cyclic Groups} \sidenote{
  \begin{thmnonum}[Theorem 19]
    Let $G = \lra{g}$ with $o(g) = n \in \mathbb{N}$.
    \begin{enumerate}
      \item $H$ is a subgroup of $G \implies \exists d \in \mathbb{N} \enspace d \, | \, n \quad H = \lra{g^d} \implies \abs{H} \, | \, n$.
      \item $k \, | \, n \implies \lra{g^{\frac{k}{n}}}$ is the unique subgroup of $G$ of order $k$.
    \end{enumerate}
  \end{thmnonum}
  }, $\lra{y}$ has only one subgroup of of order $p$, which must be $H$. Therefore, in particular, we have $xy^{-k} \in \lra{y}$ which implies $x \in \lra{y}$. It follows that $z = \pi(x) = 1$ since $\lra{y}$ is the identity in the quotient group $\faktor{G}{\lra{y}}$, which contradicts our choice of $z \neq 1$.

  Therefore, by combining the two cases, we have that $G = \lra{y}$.\qed
\end{proof}

\begin{propo}
\label{propo:p_gp_broken_down}
  Let $G \neq \{1\}$ be a finite abelian $p$-group that contains one subgroup of order $p$. Let $C$ be the cyclic subgroup of $G$ of maximal order. Then $\exists B \leq G$ such that $G = CB$ and $C \cap B = \{1\}$. By \cref{crly:group_iso_with_cross_prod_of_its_subgroups}, we have $G \cong C \times B$.
\end{propo}

\begin{proof}
  We shall prove this result by induction. If $\abs{G} = p$, then $C = G$ by definition and we can choose $B = \{1\}$. The result follows from there. Suppose that the result holds for all groups of order $p^{n - 1}$ with $n \in \mathbb{N}$ and $n \geq 2$. Consider the case for $\abs{G} = p^n$. There are two cases to consider from here.

  \noindent \textbf{Case 1}: If $C = G$, then we can pick $B = \{1\}$ so that the result follows.

  \noindent \textbf{Case 2}: If $C \neq G$, then $G$ is not cyclic. By \cref{propo:finite_abelian_p_groups_of_order_p_are_cyclic}, there exists at least 2 subgroups of $G$ that are of order $p$. Since $C$ is cyclic, by the \hyperref[thm:fundamental_theorem_of_finite_cyclic_groups]{Fundamental Theorem for Finite Cyclic Groups}, we have that $C$ contains exactly one subgroup of order $p$. Then $\exists D \leq G$ such that $\abs{D} = p$ and $D \not\subseteq C$, and consequently $C \cap D = \{1\}$. Now since $G$ is abelian, $D \triangleleft G$ and hence we may consider its coset map:
  \begin{equation*}
    \pi : G \to \faktor{G}{D}.
  \end{equation*}
  If we consider $\pi \restriction_C$, called the \hldefn{restriction} of $\pi$ on $C$ \sidenote{The restriction of $\pi$ on $C$ simply means that we restrict the domain of $\pi$ to work solely for the subset $C$. In plain words, we are only considering the case where $\pi$ is applied onto elements of $C$.}, then $\ker \pi \restriction_C = C \cap D = \{1\}$. Then by the \hyperref[thm:first_isomorphism_theorem]{First Isomorphism Theorem}, we have
  \begin{equation*}
    C = \faktor{C}{\ker \pi \restriction_C} \cong \img \pi \restriction_C = \pi(C).
  \end{equation*}
  Now let $y$ be the generator of the cyclic group $C$. Then since $\pi(C) \cong C$, we have $\pi(C) = \lra{ \pi(y) }$. By assumption on $C$, $\pi(C)$ is the cyclic subgroup of $\faktor{G}{D}$ of maximal order \sidenote{\hlwarn{I need to get some clarification from the professor on this.}}. Since $\abs{ \faktor{G}{D} } = p^{n - 1}$ by\\
  \noindent \hyperref[thm:lagrange_s_theorem]{Lagrange's Theorem}, by the induction hypothesis, $\faktor{G}{D}$ has a subgroup $E$ such that $\pi(C) E = \faktor{G}{D}$ and $\pi(C) \cap E = \{1\}$.

  Therefore, choose $B = \pi^{-1} (E)$, i.e. $\pi(B) = E$.

  \noindent\underline{Claim 1}: $G = CB$ \\
  Note that $D \subseteq B$ \sidenote{\hlwarn{This needs clarification as well.}}. If $x \in G$, $\because \pi(C) \pi(B) = \pi(C) E = \faktor{G}{D}$, we have that $\exists u \in C, \, \exists v \in B$ such that
  \begin{equation*}
    \pi(x) = \pi(u)\pi(v).
  \end{equation*}
  By homomorphicity, we have $\pi(xu^{-1}v^{-1}) = 1$ which implies $xu^{-1}v^{-1} \in D \subseteq B$. Then because $v \in B$, we have that $xu^{-1} \in B$ since $B$ is a group. Then since $G$ is abelian, we have
  \begin{equation*}
    x = u x u^{-1} \in CB.
  \end{equation*}

  \noindent\underline{Claim 2}: $C \cap B = \{1\}$. \\
  Let $x \in C \cap B$. Then $\pi(x) \in \pi(C) \cap \pi(B) = \pi(C) \cap E = \{1\}$. Then, $\because \pi(x) = 1 \in \faktor{C}{D}$ \sidenote{\hlwarn{I need to double check this to make sure that it is indeed $C$ and not $G$, because it does not make sense with $C$ being the one that $D$ is onto.}}, we have that $x \in D$. Therefore, $x \in C \cap D = \{1\}$ which then $x = 1$.

  Since \textbf{Claims 1 \& 2} hold, the result follows by induction.\qed 
\end{proof}

% subsection p_groups_continued (end)

% section finite_abelian_groups_continued (end)

% chapter lecture_19_jun_15th_2018 (end)

\chapter{Lecture 20 Jun 18th 2018}%
\label{chp:lecture_20_jun_18th_2018}
% chapter lecture_20_jun_18th_2018

\section{Finite Abelian Groups (Continued 2)}%
\label{sec:finite_abelian_groups_continued_2}
% section finite_abelian_groups_continued_2

\subsection{p-Groups (Continued 2)}%
\label{sub:p_groups_continued_2}
% subsection p_groups_continued_2

Recall that we had the following subgroup of a group $G$.
\begin{equation*}
  G^{(m)} = \{ g \in G : g^m = 1 \}.
\end{equation*}
We discussed about the Primary Decomposition, \cref{thm:primary_decomposition}, and then arrived at \cref{propo:p_gp_broken_down}. With these, we can have the following theorem:

\begin{thm}[Finite Abelian Groups are Isomorphic to a Direct Product of Cyclic Groups]
\label{thm:finite_abelian_groups_are_isomorphic_to_a_direct_product_of_cyclic_groups}
Let $G \neq \{1\}$ be a finite abelian $p$-group. Then $G$ is isomorpic to a direct product of cylic groups.
\end{thm}

\begin{proof}
  By \cref{propo:p_gp_broken_down}, there is a cyclic group $C_1$ and a subgroup $B_1$ of $G$, such that $G \cong C_1 \times B_1$. Since $B_1 \leq G$, we have that $\abs{B_1} \, \Big| \, \abs{G}$, and so by \cref{thm:lagrange_s_theorem}, $B_1$ is also a $p$-group. If $B_1 \neq \{1\}$, then by \cref{propo:p_gp_broken_down}, there exists a cyclic group $C_2$ and a $B_2 \leq B_1$ such that $B_1 \cong C_2 \times B_2$.

  By continuing this line of argument, we can get $C_1, C_2, ...$ until we get to some $C_k$ with $B_k = \{1\}$, for some $k \in \mathbb{N}$. Then
  \begin{equation*}
    G \cong C_1 \times C_2 \times \hdots \times C_k
  \end{equation*}
  as required. \qed
\end{proof}

\begin{remark}
  We can verify that the decomposition of a finite abelian $p$-group into a direct product of cyclic groups is in fact unique up to their orders.\sidenote{This is the bonus question on A4. It will be included once the assignment is over.}
\end{remark}

Combining the above remark, \cref{thm:primary_decomposition} and \cref{thm:finite_abelian_groups_are_isomorphic_to_a_direct_product_of_cyclic_groups}, we have the following theorem.

\begin{thm}[Finite Abelian Group Structure]
\index{Finite Abelian Group Structure}
\label{thm:finite_abelian_group_structure}
  If $G$ is a finite abelian group, then
  \begin{equation*}
    G \cong C_{p_1^{n_i}} \times \hdots \times C_{p_k^{n_k}}
  \end{equation*}
  where $C_{p_i^{n_i}}$ is a cyclic group of order $p_i^{n_i}$, where $1 \leq i \leq k$. The numbers $p_i^{n_i}$ are uniquely determined up to their order.\sidenote{Note that the $p_i$'s do not have to be unique.}
\end{thm}

\begin{remark}
  Note that if $p_1$ and $p_2$ are distinct primes, then
  \begin{equation*}
    C_{p_1^{n_1}} \times C_{p_2^{n_2}} \cong C_{p_1^{n_1} p_2^{n_2}},
  \end{equation*}
  the cyclic group of order $p_1^{n_1} p_2^{n_2}$. Thus, by combining suitable prime factors together, for a finite abelian group $G$, we can also write
  \begin{equation*}
    G \cong \mathbb{Z}_{m_1} \times \mathbb{Z}_{m_2} \times \hdots \times \mathbb{Z}_{m_r},
  \end{equation*}
  where $m_i \in \mathbb{N}$, $i \leq 1 \leq r$, $m_1 > 1$ and
  \begin{equation*}
    m_1 \, \big| \, m_2 \, \big| \hdots \big| \, m_r
  \end{equation*}
\end{remark}

\begin{eg}
  Conder an abelian group $G$ with order $48$. Since $48 = 2^4 \cdot 3$, an abelian group of order $48$ is isomorphic to $H \times \mathbb{Z}_3$, where $H$ is an abelian group of order $2^4$. The options for $H$ are:
  \begin{gather*}
    \mathbb{Z}_{2^4} \qquad \mathbb{Z}_{2^3} \times \mathbb{Z}_2 \qquad \mathbb{Z}_{2^2} \times \mathbb{Z}_{2^2} \\
    \mathbb{Z}_{2^2} \times \mathbb{Z}_2 \times \mathbb{Z}_2 \qquad \mathbb{Z}_2 \times \mathbb{Z}_2 \times \mathbb{Z}_2 \times \mathbb{Z}_2
  \end{gather*}
  Therefore, we have the following possible decompositions of $G$:
  \begin{align*}
    G &\cong \mathbb{Z}_{2^4} \times \mathbb{Z}_3 \cong \mathbb{Z}_{48} \\
    G &\cong \mathbb{Z}_{2^3} \times \mathbb{Z}_2 \times \mathbb{Z}_3 = \mathbb{Z}_2 \times \mathbb{Z}_{24} \\ 
    G &\cong \mathbb{Z}_{2^2} \times \mathbb{Z}_{2^2} \times \mathbb{Z}_3 = \mathbb{Z}_4 \times \mathbb{Z}_{12} \\ 
    G &\cong \mathbb{Z}_{2^2} \times \mathbb{Z}_2 \times \mathbb{Z}_2 \times \mathbb{Z}_3 = \mathbb{Z}_2 \times \mathbb{Z}_2 \times \mathbb{Z}_{12} \\ 
    G &\cong \mathbb{Z}_2 \times \mathbb{Z}_2 \times \mathbb{Z}_2 \times \mathbb{Z}_2 \times \mathbb{Z}_3 = \mathbb{Z}_2 \times \mathbb{Z}_2 \times \mathbb{Z}_2 \times \mathbb{Z}_6
  \end{align*}
\end{eg}

% subsection p_groups_continued_2 (end)

% section finite_abelian_groups_continued_2 (end)

\section{Rings}%
\label{sec:rings}
% section rings

\subsection{Rings}%
\label{sub:rings}
% subsection rings

\begin{defn}[Ring]\index{Ring}
\label{defn:ring}
  A set $R$ is a ring if $\forall a, b, c \in R$,
  \marginnote{\noindent As daunting as this definition seems, it is much easier to remember if we think of $R$ being an \hlnoteb{abelian group under addition}, \hlnoteb{``almost'' a group under multiplication}, save the fact that the \hlimpo{multiplicative inverse of an element does not necessarily exist}, and with the \hlnoteb{distributive law}. }
  \begin{enumerate}
    \item $a + b \in R$
    \item $a + b = b + a$
    \item $a + (b + c) = (a + b) + c$
    \item $\exists 0 \in R \enspace a + 0 = a = 0 + a$
    \item $\exists (-a) \in R \enspace a + (-a) = 0 = (-a) + a$
    \item $ab \in R$
    \item $a(bc) = (ab)c$
    \item $\exists 1 \in R \enspace 1 \cdot a = a = a \cdot 1$
    \item $a ( b + c ) = ab + ac$ and $(b + c) a = ba + ca$
  \end{enumerate}
  We call $1$ as the \hldefn{Unity} of $R$, $0$ as the \textcolor{base16-eighties-blue}{Zero}\index{Zero of a Ring} of $R$, and $-a$ as the \hlnoteb{negative} of $a$.

  The ring $R$ is called a \hldefn{Commutative Ring} if it also satisfies the following:
  \begin{enumerate}
    \setcounter{enumi}{9}
    \item $ab = ba$.
  \end{enumerate}
\end{defn}

\begin{eg}
  $\mathbb{Z}, \mathbb{Q}, \mathbb{R}$ and $\mathbb{C}$ are commutative rings with the zero being $0$, and unity being $1$.
\end{eg}

\begin{eg}
  For $n \in \mathbb{N}, \, n \geq 2$, $\mathbb{Z}_n$ is a commutative ring with the zero being $[0]$, and unity being $[1]$.
\end{eg}

\begin{eg}
  The set $M_n(\mathbb{R})$ is a ring using matrix addition and matrix multiplication, with zero being the zero matrix $0$, and unity being the identity matrix $I$. We also know that $M_n(\mathbb{R})$ is not commutative.
\end{eg}

\begin{warning}
  Note that since $(R, \cdot)$ is not a group, we no longer have the liberty of using \cref{propo:cancellation_laws}, i.e. we do not have left or right cancellation. For example, in $\mathbb{Z}$, $0 \cdot x = 0 \cdot y \notimply x = y$.
\end{warning}

% subsection rings (end)

% section rings (end)

% chapter lecture_20_jun_18th_2018 (end)

\chapter{Lecture 21 Jun 20th 2018}%
\label{chp:lecture_21_jun_20th_2018}
% chapter lecture_21_jun_20th_2018

\section{Rings (Continued)}%
\label{sec:rings_continued}
% section rings_continued

\subsection{Rings (Continued)}%
\label{sub:rings_continued}
% subsection rings_continued

\begin{note}[Notation]
  Given a ring $R$, to distinguish the difference between multiples in addition and in multiplication, for $n \in \mathbb{N} \, \land \, a \in R$, we write
  \begin{equation*}
    na = \underbrace{a + a + \hdots + a}_{n \text{ times }}
  \end{equation*}
  and
  \begin{equation*}
    a^n = \underbrace{a \cdot a \cdot \hdots \cdot a}_{n \text{ times }}
  \end{equation*}
  respectively. Also, we will define
  \begin{equation*}
    (-n) a = \underbrace{(-a) + (-a) + \hdots + (-a)}_{n \text{ times }}
  \end{equation*}
  and
  \begin{equation*}
    a^{-n} = \left( a^{-1} \right)^n
  \end{equation*}
  if $a^{-1}$ exists.
\end{note}

\begin{note}
  Recall that for a group $G$ and $g \in G$, we have $g^0 = 1$, $g^1 = g$, and $\left(g^{-1}\right)^{-1} = g$. Thus for addition, we have
  \begin{gather*}
    \overarrow{0}{\text{integer}} \cdot a = \underarrow{0}{\text{zero in } R} \qquad 1 \cdot a = a \\
    - (-a) = a
  \end{gather*}

  Also, by \cref{propo:group_notations}, if $n, m \in \mathbb{Z}$, we have
  \begin{gather*}
    m \cdot a + n \cdot a = (m + n) \cdot a \\
    n(ma) = (nm)a \\
    n(a + b) = na + nb
  \end{gather*}
\end{note}

\begin{propo}[More Properties of Rings]
\label{propo:more_properties_of_rings}
  Let $R$ be a ring and $r, s \in \mathbb{R}$.\marginnote{This is a problem in A4.}
  \begin{enumerate}
    \item If $0$ is the zero of $R$, then $0 \cdot r = 0 = r \cdot 0$; \sidenote{i.e. all the $0$'s are zeros of $R$.}
    \item $-r (s) = - (rs) = r (-s)$;
    \item $(-r)(-s) = rs$;
    \item $\forall m, n \in \mathbb{Z}, \, (mr)(ns) = (mn)(rs)$.
  \end{enumerate}
\end{propo}

\begin{defn}[Trivial Ring]\index{Trivial Ring}
\label{defn:trivial_ring}
A \hlnoteb{trivial ring} is a ring of only one element. In this case, we have $1 = 0$, i.e. the unity is the zero and vice versa.
\end{defn}

\begin{remark}
  If $R$ is a ring with $R \neq \{0\}$, since $r = r \cdot 1$ for all $r \in R$, we have $1 \neq 0$. Otherwise, if $1 = 0$, then $r = r \cdot 1 = r \cdot 0 = 0$, i.e. $R = \{0\}$.
\end{remark}

\begin{eg}\label{eg:direct_product_of_rings}
  Let $R_1, R_2, ..., R_n$ be rings. We define component-wise operation on the product
  \begin{equation*}
    R_1 \times R_2 \times \hdots \times R_n
  \end{equation*}
  as follows:
  \begin{align*}
    (r_1, r_2, ..., r_n) + (s_1, s_2, ..., s_n) &= (r_1 + s_1, r_2 + s_2, ..., r_n + s_n) \\
    (r_1, r_2, ..., r_n)(s_1, s_2, ..., s_n) &= (r_1 s_1, r_2 s_2, ..., r_n s_n)
  \end{align*}
  We can check that $R_1 \times R_2 \times \hdots \times R_n$ is a ring with the zro being $(0, 0, ..., 0)$ and the unity being $(1, 1, ..., 1)$. This set
  \begin{equation*}
    R_1 \times R_2 \times \hdots \times R_n
  \end{equation*}
  is called the \hldefn{direct product} of $R_1, R_2, ..., R_n$.
\end{eg}

\begin{defn}[Characteristic of a Ring]\index{Characteristic}
\label{defn:characteristic_of_a_ring}
  If $R$ is a ring, we define the \hlnoteb{characteristic} of $R$, denoted by $\ch(R)$, in terms of the order of $1_R$ in the additive group $(R, +)$, by
  \begin{equation*}
    \ch(R) = \begin{cases}
      n & \text{if } o(1_R) = n \in \mathbb{N} \text{ in } (R, +) \\
      0 & \text{if } o(1_R) = \infty \text{ in } (R, +)
    \end{cases}
  \end{equation*}
\end{defn}

For $k \in \mathbb{Z}$, we write $kR = 0$ to mean that $\forall r \in R$, $kr = 0$.

By \cref{propo:more_properties_of_rings}, we have
\begin{equation*}
  kr = k (1_R \cdot r) = (k 1_R) \cdot r
\end{equation*}
and so $kR = 0$ if and only if $k 1_R = 0$. Then, since $(R, +)$ is a group, by \cref{propo:properties_of_elements_of_finite_order} and \cref{propo:property_of_elements_of_infinite_order}, it follows that:

\begin{propo}[Implications of the Characteristic]
\label{propo:implications_of_the_characteristic}
  Let $R$ be a ring and $k \in \mathbb{Z}$.\sidenote{This is why we defined $\ch(R) = 0$ if $o(1_R) = \infty$}
  \begin{enumerate}
    \item $\ch(R) = n \in \mathbb{N} \implies \left( kR = 0 \iff n \mid k \right)$
    \item $\ch(R) = 0 \implies \left( kR = 0 \iff k = 0 \right)$
  \end{enumerate}
\end{propo}

\begin{eg}
  Each of $\mathbb{Z}, \mathbb{Q}, \mathbb{R}$ and $\mathbb{C}$ has characteristic $0$. For $n \in \mathbb{N}$ with $n \geq 2$, the ring $\mathbb{Z}_n$ has characteristic $n$.
\end{eg}

% subsection rings_continued (end)

\subsection{Subring}%
\label{sub:subring}
% subsection subring

\begin{defn}[Subring]\index{Subring}
\label{defn:subring}
A subset $S$ of a ring $R$ is a subring if $S$ is a ring itself (under the same operations: addition and multiplication).\marginnote{Unlike subgroups, since there is no proper suggestion of a symbolic representation, I shall use $S \leq_r R$ to denote that $S$ is a subring of $R$, in comparison to $\leq$ for subgroups, which has no subscript. Note that this is purely for keeping my writing succinct, and so the subscript $r$ is used simply to indicate that the $\leq$ symbol is for denoting a subring and should not be confused with other $r$'s that may be used in a proof. This notation is also not used in class, and should be avoided during materials outside of this set of notes.}
\end{defn}

Note that properties (2), (3), (7) and (9) from \cref{defn:ring} are automatically satisfied. Thus, to show that $S$ is a subring, it suffices to show the following:

\hldefn{Subring Test}\label{spe:subring_test}
\begin{enumerate}
  \item $0, 1 \in S$
  \item $s, t \in S \implies ( s - t ), st \in S$
\end{enumerate}

\begin{eg}
  We have the following chain of commutative rings:
  \begin{equation*}
    \mathbb{Z} \leq_r \mathbb{Q} \leq_r \mathbb{R} \leq_r \mathbb{C}
  \end{equation*}
\end{eg}

\begin{eg}
  If $R$ is a ring, the \textcolor{base16-eighties-blue}{center}\index{Center of a Ring} $Z(R)$ of $R$ is defined as
  \begin{equation*}
    Z(R) = \{ z \in R : zr = rz, r \in R \}.
  \end{equation*}
  Note taht $0, 1 \in Z(R)$. Also, if $s, t \in Z(R)$, then $\forall r \in R$,
  \begin{equation*}
    (s - t) r = sr - tr = rs - rt = r( s - t )
  \end{equation*}
  and so $(s - t) \in Z(R)$. Also,
  \begin{equation*}
    (st)r = s(tr) = s(rt) = (sr)t = (rs)t = r(st)
  \end{equation*}
  and so $st \in Z(R)$. By the \hlnotea{Subring Test}, $Z(R) \leq_r R$.
\end{eg}

\begin{eg}
  Let
  \begin{equation*}
    \mathbb{Z}[c] = \{a + bi : a, b \in \mathbb{Z}, \, i^2 = -1 \} \subseteq \mathbb{C}.
  \end{equation*}
  It can be shown that $\mathbb{Z}[i] \leq_r \mathbb{C}$, and is called the ring of \hldefn{Gaussian integers}.\sidenote{Proof that the Gaussian integers is a subring is in A4, which shall be included after the assignment is over.}
\end{eg}

% subsection subring (end)

% section rings_continued (end)

% chapter lecture_21_jun_20th_2018 (end)

\chapter{Lecture 22 Jun 22nd 2018}%
\label{chp:lecture_22_jun_22nd_2018}
% chapter lecture_22_jun_22nd_2018

\section{Ring (Continued 2)}%
\label{sec:ring_continued_2}
% section ring_continued_2

\subsection{Ideals}%
\label{sub:ideals}
% subsection ideals

Let $R$ be a ring and $A$ an additive subgroup of $R$. Since $(R, +)$ is abelian, we have that $A \triangleleft R$. Thus, we can talk about the additive quotient group
\begin{gather*}
  \faktor{R}{A} = \{r + a : r \in \mathbb{R} \} \text{ with } \\
  r + A = \{ r + a : a \in A \}
\end{gather*}

Using the properties that we know about cosets and quotient groups, we have the following proposition.

\begin{propo}[Properties of the Additive Quotient Group]
\label{propo:properties_of_the_additive_quotient_group}
  Let $R$ be a ring and $A$ an additive subgroup of $R$. For $r, s \in R$, we have\marginnote{This is just a translation of the properties of cosets and quotient groups, that we are familiar with, into the language of addition. You can (read: should) prove this as an exercise for yourself (read: myself).}
  \begin{enumerate}
    \item $r + A = s + A \iff (r - s) \in A$
    \item $(r + A) + (s + A) = (r + s) + A$
    \item $0 + A = A$ is the additive identity of $\faktor{R}{A}$
    \item $- (r + A) = (-r) + A$ is the additive inverse of $r + A$
    \item $\forall k \in \mathbb{Z} \quad k(r + A) = kr + A$
  \end{enumerate}
\end{propo}

Since $R$ is a ring, it is natural to ask if we could make $\faktor{R}{A}$ into a ring\sidenote{\textit{Ideally} (see what I did there?), we would want $\faktor{R}{A}$ as a ring, just as we had $\faktor{R}{A}$ as a group.}. A natural way to define ``multiplication'' in $\faktor{R}{A}$ is
\begin{equation}\tag{$\dagger$}\label{eq:ideal_quotient_ring_multiplication}
  (r + A)(s + A) = rs + A \quad \forall r, s \in \mathbb{R}
\end{equation}
Note, however, that we would have
\begin{equation*}
  r + A = r_1 + A \qquad s + A = s_1 + A
\end{equation*}
with $r \neq r_1$ and $s \neq s_1$. In order for \cref{eq:ideal_quotient_ring_multiplication} to make sense, it is necessary that
\begin{equation*}
  r + A = r_1 + A \, \land \, s + A = s_1 + A \implies rs + A = r_1 s_1 + A
\end{equation*}
so that this ``multiplication'' is \hlimpo{well-defined}.

\begin{propo}
\label{propo:equivalent_defn_of_a_well_defined_coset_multiplication}
  Let $A$ be an additive subgroup of a ring $R$. Then $\forall a \in A$, define
  \begin{equation*}
    Ra = \{ ra : r \in R \} \quad aR = \{ ar : r \in R \}.
  \end{equation*}
  The following are equivalent (TFAE):
  \begin{enumerate}
    \item $Ra \subseteq A$ and $aR \subseteq A$, $\forall a \in A$;
    \item $\forall r, s \in R$, $(r + A)(s + A) = rs + A$ is well-defined in $\faktor{R}{A}$.
  \end{enumerate}
\end{propo}

\begin{proof}
  $(1) \implies (2)$: If $r + A = r_1 + A$ and $s + A = s_1 + A$, for $r, r_1, s, s_1 \in R$, we need to show that
  \begin{equation*}
    rs + A = r_1 s_1 + A.
  \end{equation*}
  By \cref{propo:properties_of_the_additive_quotient_group}, we have that $(r - r_1), (s - s_1) \in A$, and so by $(1)$, we have
  \begin{align*}
    rs - r_1 s_1 &= rs - r_1 s + r_1 s - r_1 s_1 \\
                 &= (r - r_1)s + r_1 (s - s_1) \\
                 &\in (r - r_1) R + R (s - s_1) \subseteq A
  \end{align*}
  Therefore, by \cref{propo:properties_of_the_additive_quotient_group} again, we have $rs + A = r_1 s_1 + A$.

  \noindent $(2) \implies (1)$: Let $r \in R$ and $a \in A$. We have that
  \begin{align*}
    ra + A &= (r + A)(a + A) \quad \because (2) \\
           &= (r + A)(0 + A) \quad \because a, 0 \in A \\
           &= ( r \cdot 0 ) + A \quad \because (2) \\
           &= 0 + A \quad \because \cref{propo:more_properties_of_rings} \\
           &= A \quad \because \cref{propo:properties_of_the_additive_quotient_group}
  \end{align*}
  Thus $ra \in A$ and so $Ra \subseteq A$. Similarly, we can show that $aR \subseteq A$.\qed
\end{proof}

\begin{defn}[Ideal]\index{Ideal}
\label{defn:ideal}
  An additive subgroup $A$ of a ring $R$ is called an \hlnoteb{ideal} of $R$ if $Ra, aR \subseteq A$, $\forall a \in A$.
\end{defn}

\begin{eg}
  If $R$ is a ring, $\{0\}$ and $R$ are both ideals of $R$.
\end{eg}

\begin{propo}[The Only Ideal with the Multiplicative Identity is the Ring Itself]
\label{propo:the_only_ideal_with_the_multiplicative_identity_is_the_ring_itself}
Let $A$ be an ideal of a ring $R$. If $1 \in A$, then $A = R$.\marginnote{This also shows that if we want a non-trivial ideal, then the ideal should not have $1$.}
\end{propo}

\begin{proof}
  $\forall r \in R$, $\because A$ is an ideal and $1 \in A$, we have $r = r \cdot 1 \in A$. It follows that $R \subseteq A \subseteq R$ and so $R = A$.\qed
\end{proof}

\begin{propo}[Construction of the Quotient Ring]
\label{propo:construction_of_the_quotient_ring}
Let $A$ be an ideal of a ring $R$. Then the additive quotient group $\faktor{R}{A}$ is a ring with the multiplication $(r + A)(s + A) = rs + A$, $\forall r, s \in R$. The unity of $\faktor{R}{A}$ is $1 + A$.
\end{propo}

\begin{proof}
  $\because A$ is an additive subgroup of a ring $R$, $\faktor{R}{A}$ is an additive abelian group. By \cref{propo:equivalent_defn_of_a_well_defined_coset_multiplication}, the multiplication on $\faktor{R}{A}$ is well-defined. The multiplication is associative, since $\forall r, s, q \in R$,
  \begin{align*}
    (r + A)\big( (s + A)(q + A) \big) &= (r + A) ( sq + A ) = (rsq + A) \\
                                      &= (rs + A)(q + A) \\
                                      &= \big( (r + A)(s + A) \big)(q + A).
  \end{align*}
  We also have
  \begin{equation*}
    (r + A)(1 + A) = r + A = (1 + A)(r + A)
  \end{equation*}
  and so the unity of $\faktor{R}{A}$ is $1 + A$. The distributive property is inherited from $R$.\qed
\end{proof}

\begin{defn}[Quotient Ring]\index{Quotient Ring}
\label{defn:quotient_ring}
  Let $A$ be an ideal of a ring $R$. Then the ring $\faktor{R}{A}$ is called the \hlnoteb{quotient ring} of $R$ by $A$.
\end{defn}

\begin{defn}[Principal Ideal]\index{Principal Ideal}
\label{defn:principal_ideal}
Let $R$ be a commutative ring and $A$ an ideal of $R$. If $A = aR = \{ ar : r \in R \} = Ra$ for some $a \in A$, we say that $A$ is a \hlnoteb{principal ideal} \textcolor{base16-eighties-blue}{generated}\index{generator} by $a$, and denote $A = \lra{a}$.
\end{defn}

\begin{eg}
  If $n \in \mathbb{Z}$, then $\lra{n} = n \mathbb{Z}$ is a(n) (principal) ideal of $\mathbb{Z}$, since $\mathbb{Z}$ is commutative.
\end{eg}

\begin{propononum}[Ideals of $\mathbb{Z}$ are Principal Ideals]
\label{propononum:ideals_of_z_are_principal_ideals}
All ideals of $\mathbb{Z}$ are of the form $\lra{a}$ for some $n \in \mathbb{Z}$.
\end{propononum}

We shall prove this in the next lecture.

% subsection ideals (end)

% section ring_continued_2 (end)

% chapter lecture_22_jun_22nd_2018 (end)

\chapter{Lecture 23 Jun 25th 2018}%
\label{chp:lecture_23_jun_25th_2018}
% chapter lecture_23_jun_25th_2018

\section{Ring (Continued 3)}%
\label{sec:ring_continued_3}
% section ring_continued_3

\subsection{Ideals (Continued)}%
\label{sub:ideals_continued}
% subsection ideals_continued

\begin{propo}[Ideals of $\mathbb{Z}$ are Principal Ideals]
\label{propo:ideals_of_z_are_principal_ideals}
  All ideals of $\mathbb{Z}$ are of the form $\lra{n}$ for some $n \in \mathbb{Z}$. 
\end{propo}

\begin{proof}
  Let $A$ be an ideal of $\mathbb{Z}$. If $A = \{0\}$, then $A = \lra{0}$. Otherwise, let $a \in A$ with $a \neq 0$, and $\abs{a}$ be the minimum. Clearly, $\lra{a} = a \mathbb{Z} \subseteq A$. To prove the other inclusion, let $b \in A$. By the \hlnotea{Division Algorithm}, $\exists q, t \in \mathbb{Z}$ with $0 \leq r < \abs{a}$ such that $b = qa + r$. Because $A$ is an ideal, we have $r = b - qa \in A$. Since $\abs{r} < \abs{a}$ which is the minimal case, it must be that $r = 0$. Therefore $b = qa \in \lra{a}$ and so $A \subseteq \lra{a}$.\qed
\end{proof}

% subsection ideals_continued (end)

\subsection{Isomorphism Theorems for Rings}%
\label{sub:isomorphism_theorems_for_rings}
% subsection isomorphism_theorems_for_rings

\begin{defn}[Ring Homomorphism]\index{Ring Homomorphism}\index{Homomorphism}
\label{defn:ring_homomorphism}
  Let $R$ and $S$ be rings. A mapping
  \begin{equation*}
    \Theta : R \to S
  \end{equation*}
  is a ring \hlnoteb{homomorphism} if $\forall a, b \in R$, we have
  \begin{enumerate}
    \item $\Theta(a + b) = \Theta(a) + \Theta(b)$
    \item $\Theta(ab) = \Theta(a) \Theta(b)$
    \item $\Theta(1_R) = 1_S$
  \end{enumerate}
\end{defn}

\begin{note}[Remark]
  $(2) \notimply (3)$ because $\Theta(1_R) \in S$ does not necessarily have a multiplicative inverse, since $S$ is a ring.
\end{note}

\begin{eg}
  The mapping $k \mapsto [k]$ from $\mathbb{Z} \to \mathbb{Z}_n$ is a surjective ring homomorphism.
\end{eg}

\begin{eg}[Direct Product of Rings]\label{eg:direct_product_of_rings}
  If $R_1, R_2$ are rings, the projection
  \begin{equation*}
    \pi_1 : R_1 \times R_2 \to R_1 \text{ defined by } \pi_1 (r_1, r_2) = r_1
  \end{equation*}
  is a surjective ring homomorphism, since
  \begin{enumerate}
    \item $\pi_1(r_1 + r_2, q_1 + q_2) = r_1 + r_2 = \pi_1(r_1, q_1) + \pi_1(r_2, q_2)$;
    \item $\pi_1(r_1 r_2, q_1 q_2) = r_1 r_2 = \pi_1(r_1, q_1) \pi_1(r_2, q_2)$; and
    \item $\pi(1, 1) = 1$.
  \end{enumerate}
  We can a similar $\pi_2 : R_1 \times R_2 \to R_2$ such that $(r_1, r_2) \mapsto r_2$, and we will get that $\pi_2$ is also a surjective ring homomorphism.
\end{eg}

\begin{propo}[Properties of Ring Homomorphisms]
\label{propo:properties_of_ring_homomorphisms}
  Let $\Theta: R \to S$ be a ring homomorphism and let $r \in R$. Then
  \begin{enumerate}
    \item $\Theta(0_R) = 0_S$
    \item $\Theta(-r) = - \Theta(r)$
    \item $\Theta(kr) = k \Theta(r)$
    \item $\forall n \in \mathbb{N} \cup \{0\} \quad \Theta(r^n) = \Theta(r)^n$
    \item $u \in R^* \implies \forall k \in \mathbb{Z} \quad \Theta(u^k) = \Theta(u)^k$
  \end{enumerate}
\end{propo}

\begin{proof}
  \begin{enumerate}
    \item Note that
      \begin{equation*}
        \Theta(r) = \Theta(0_R + r) = \Theta(0_R) + \Theta(r).
      \end{equation*}
      Therefore,
      \begin{equation*}
        \Theta(0_R) = 0_S
      \end{equation*}
      as required.
    \item Note that
      \begin{equation*}
        0_S = \Theta(0_R) = \Theta(r - r) = \Theta(r) + \Theta(-r),
      \end{equation*}
      so
      \begin{equation*}
        \Theta(-r) = -\Theta(r).
      \end{equation*}
    \item Observe that
      \begin{equation*}
        \Theta(kr) = \Theta(\underbrace{r + r + \hdots + r}_{k \text{ times }}) = \underbrace{\Theta(r) + \Theta(r) + \hdots + \Theta(r)}_{k \text{ times }} = k \Theta(r)
      \end{equation*}
  \end{enumerate}
  Item 4 follows by induction on the definition of a ring homomorphism, and Item 5 follows as a result from Item 4 because if $u \in R^*$, then $u^{-1} \in R^*$ such that $uu^{-1} = 1_R$.\qed
\end{proof}

\begin{defn}[Ring Isomorphism]\index{Ring Isomorphism}
\label{defn:ring_isomorphism}
  A mapping of rings $\Theta: R \to S$ is a ring \hlnoteb{isomorphism} if $\Theta$ is a bijective ring homomorphism. In this case, we say that $R$ and $S$ are \hlnoteb{isomorphic} and denote that by $R \cong S$.
\end{defn}

\begin{defn}[Kernel and Image]\index{Kernel}\index{Image}
\label{defn:kernel_and_image}
  Let $\Theta: R \to S$ be a ring homomorphism. The \hlnoteb{kernel} of $\Theta$ is defined by
  \begin{equation*}
    \ker \Theta = \{r \in R : \Theta(r) = 0_S\}
  \end{equation*}
  and the \hlnoteb{image} of $\Theta$ is defined by
  \begin{equation*}
    \img \Theta := \Theta(R) = \{\Theta(r) : r \in R\}.
  \end{equation*}
\end{defn}

\begin{propo}
\label{propo:image_of_hm_is_a_subring_n_kernel_of_hm_is_a_normal_subring}
  Let $\Theta: R \to S$ be a ring homomorphism. Then
  \begin{enumerate}
    \item $\img \Theta \leq_r S$
    \item $\ker \Theta$ is an ideal of $R$
  \end{enumerate}
\end{propo}

\begin{proof}
  \begin{enumerate}
    \item $\Theta(1_R) = 1_S$ by definition of a homomorphism so $\Theta(1_R) \in \img \Theta$. Suppose $s_1 = \Theta(r_1)$ and $s_2 = \Theta(r_2)$, then
      \begin{align*}
        s_1 - s_2 &= \Theta(r_1) - \Theta(r_2) = \Theta(r_1 - r_2) \\
        s_1 s_2 &= \Theta(r_1) \Theta(r_2) = \Theta(r_1 r_2)
      \end{align*}
      are both in $\img \Theta$. By the \hyperref[spe:subring_test]{Subring Test}, $\img \Theta \leq_r S$.

    \item Since $\ker \Theta$ is an additive subgroup of $R$, it suffices to show that $ra, ar \in \ker \Theta$ for all $r \in R$ and $a \in \ker \Theta$. Let $r \in R$ and $a \in \ker \Theta$. Then
      \begin{equation*}
        \Theta(ra) = \Theta(r) \Theta(a) = \Theta(r) \cdot 0 = 0
      \end{equation*}
      So $ra \in \ker \Theta$. Similarly so,
      \begin{equation*}
        \Theta(ar) = \Theta(a) \Theta(r) = 0 \cdot \Theta(r) = 0
      \end{equation*}
      and so $ar \in \ker \Theta$. Therefore, $\ker \Theta$ is an ideal of $R$.
  \end{enumerate}\qed
\end{proof}

\begin{thm}[First Isomorphism Theorem for Rings]
\index{First Isomorphism Theorem}
\label{thm:first_isomorphism_theorem_for_rings}
  Let $\Theta : R \to S$ be a ring homomorphism. Then
  \begin{equation*}
    \faktor{R}{\ker \Theta} \cong \img \Theta.
  \end{equation*}
\end{thm}

\begin{proof}
  Let $A = \ker \Theta$. Since $A$ is an ideal of $R$, we have that $\faktor{R}{A}$ is a ring. Define 
  \begin{equation*}
    \bar{\Theta} : \faktor{R}{A} \to \img \Theta \text{ by } ( r + A ) \mapsto \theta(a).
  \end{equation*}
  Note that
  \begin{equation*}
    r + A = s + A \iff (r - s) \in A \iff \Theta(r - s) = 0 \iff \Theta(r) = \Theta(s).
  \end{equation*}
  Therefore $\bar{\Theta}$ is well-defined and injective. Also, it is clear that $\bar{\Theta}$ is surjective. To show that $\bar{\Theta}$ is a homomorphism, note that $\forall r, s \in R$, we have
  \begin{align*}
    \bar{\Theta}(r + A + s + A) &= \bar{\Theta}(r + s + A) = \Theta(r + s) \\
                                &= \Theta(r) + \Theta(s) = \bar{\Theta}(r + A) + \bar{\Theta}(s + A).
  \end{align*}
  It follows that $\bar{\Theta}$ is a ring isomorphism and so
  \begin{equation*}
    \faktor{R}{\ker \Theta} \cong \img \Theta
  \end{equation*}
  as required. \qed
\end{proof}

\begin{ex}
  Let $A, B \leq_r R$, where $R$ is a ring. Prove that
  \begin{enumerate}
    \item $A \cap B$ is the largest subring of $R$ contained in both $A$ and $B$.
    \item If either $A$ or $B$ is an ideal of $R$, the sum
      \begin{equation*}
        A + B = \{a + b : a \in A, \, b \in B \}
      \end{equation*}
      is a subring of $R$, and is the smallest subring of $R$ that contains both $A$ and $B$.
  \end{enumerate}
\end{ex}

\begin{thm}[Second Isomorphism Theorem for Rings]
\index{Second Isomorphism Theorem}
\label{thm:second_isomorphism_theorem_for_rings}
  Let $A$ be a subring and $B$ an ideal of a ring $R$. Then
  \begin{enumerate}
    \item $A + B \leq_r R$;
    \item $B$ is an ideal of $A + B$;
    \item $A \cap B$ is an ideal of $A$; and
    \item 
      \begin{equation*}
        \faktor{(A + B)}{B} \cong \faktor{A}{(A \cap B)}
      \end{equation*}
  \end{enumerate}
\end{thm}

\begin{thm}[Third Isomorphism Theorem for Rings]
\index{Third Isomorphism Theorem}
\label{thm:third_isomorphism_theorem_for_rings}
  Let $A$ and $B$ be ideals of $R$ with $A \subseteq B$, then $\faktor{B}{A}$ is an ideal of $\faktor{R}{A}$ and
  \begin{equation*}
    \left(\faktor{R}{A}\right)\Big/\left(\faktor{B}{A}\right) \cong \faktor{R}{B}.
  \end{equation*}
\end{thm}

% subsection isomorphism_theorems_for_rings (end)

% section ring_continued_3 (end)

% chapter lecture_23_jun_25th_2018 (end)

\chapter{Lecture 24 Jun 27th 2018}%
\label{chp:lecture_24_jun_27th_2018}
% chapter lecture_24_jun_27th_2018

\section{Rings (Continued 4)}%
\label{sec:rings_continued_4}
% section rings_continued_4

\subsection{Isomorphism Theorems for Rings (Continued)}%
\label{sub:isomorphism_theorems_for_rings_continued}
% subsection isomorphism_theorems_for_rings_continued

\begin{thm}[Chinese Remainder Theorem]
\index{Chinese Remainder Theorem}
\label{thm:chinese_remainder_theorem}
  Let $A$ and $B$ be ideals of $R$.
  \begin{enumerate}
    \item $A + B = R \implies \faktor{R}{(A \cap B)} \cong \faktor{R}{A} \times \faktor{R}{B}$
    \item $A + B = R \, \land \, A \cap B = \{0\} \implies R \cong \faktor{R}{A} \times \faktor{R}{B}$
  \end{enumerate}
\end{thm}

\begin{proof}
  It suffices to prove $(1)$ since if $(1)$ is true and $A \cap B = \{0\}$, then $(2)$ immediately follows.

  Define
  \begin{equation*}
    \Theta: R \to \faktor{R}{A} \times \faktor{R}{B} \qquad r \mapsto ( r + A, r + B )
  \end{equation*}
  Then $\Theta$ is a ring homomorphism \sidenote{
  \begin{ex}
    Prove that $\Theta$ is a ring homomorphism.
  \end{ex}
  }.

  \begin{proof}[$\Theta$ is a ring homomorphism]
    $\forall r, s \in R$, we have
    \begin{align*}
      \Theta(rs) &= (rs + A, rs + B) \\
                 &\overset{(*)}{=} (r + A, r + B)(s + A, s + B) \\
                 &= \Theta(r) \Theta(s)
    \end{align*}
    where $(*)$ is by \cref{propo:construction_of_the_quotient_ring}. Also by the same proposition, we have
    \begin{equation*}
      \Theta(1) = (1 + A, 1 + B).
    \end{equation*}
    Then,
    \begin{align*}
      \Theta(r + s) &= (r + s + A, r + s + B) \\
                    &\overset{(\dagger)}{=} (r + A, r + B) + (s + A, s + B) \\
                    &= \Theta(r) + \Theta(s)
    \end{align*}
    where $(\dagger)$ is by \cref{propo:properties_of_the_additive_quotient_group}.
  \end{proof}
  Note that $\ker \Theta = A \cap B$, since
  \begin{equation*}
    \ker \Theta = \{ r \in R : \Theta(r) = (A, B) \} = \{r \in A \, \land \, r \in B \} = A \cap B.
  \end{equation*}
  To show that $\Theta$ is surjective, let $(s + A, t + B) \in \faktor{R}{A} \times \faktor{R}{B}$ with $s, \, t \in R$. Since $A + B = R$, $\exists a \in A, \, \exists b \in B$ such that $a + b = 1$. Let $r = sb + ta$. Then
  \begin{align*}
    s - r &= s - sb - ta = s(1 - b) - ta = sa - ta = (s - t) a \in A \\
    t - r &= t - sb - ta = t(1 - a) - sb = tb - sb = (t - s) b \in B
  \end{align*}
  and so by \cref{propo:properties_of_the_additive_quotient_group},
  \begin{equation*}
    s + A = r + A \text{ and } t + B = r + B.
  \end{equation*}
  Therefore
  \begin{equation*}
    \Theta(r) = (r + A, r + B) = (s + A, t + B),
  \end{equation*}
  and so $\Theta$ is surjecive. Then by the \cref{thm:first_isomorphism_theorem_for_rings},
  \begin{equation*}
    \faktor{R}{(A \cap B)} \cong \faktor{R}{A} \times \faktor{R}{B}.
  \end{equation*}\qed
\end{proof}

\newthought{Why is} \cref{thm:chinese_remainder_theorem} called the Chinese Remainder Theorem?

Let $m, n \in \mathbb{N}$ with $\gcd(m, n) = 1$. Then we know that
\begin{equation*}
  m \mathbb{Z} \cap n \mathbb{Z} = mn \mathbb{Z}.
\end{equation*}
Also, $m \mathbb{Z} + n \mathbb{Z} = \mathbb{Z}$ since $1 = ma + nb$ for some $a, b \in \mathbb{Z}$ by \hlnotea{Bezout's Lemma}. And so:

\begin{crly}
\label{crly:why_crt}
  \begin{enumerate}
    \item If $m, n \in \mathbb{N}$ with $\gcd(m, n) = 1$, then
      \begin{equation*}
        \faktor{\mathbb{Z}}{mn \mathbb{Z}} \cong \faktor{\mathbb{Z}}{m \mathbb{Z}} \times \faktor{\mathbb{Z}}{n \mathbb{Z}}
      \end{equation*}
      i.e.
      \begin{equation*}
        \mathbb{Z}_{mn} \cong \mathbb{Z}_m \times \mathbb{Z}_n
      \end{equation*}

    \item If $m, n \in \mathbb{N}$ with $m, n \geq 2$ and $\gcd(m, n) = 1$, then
      \begin{equation*}
        \phi(mn) = \phi(m)\phi(n)
      \end{equation*}
      where $\phi(m) = \abs{\mathbb{Z}_m^*}$ is \hldefn{Euler's $\phi$-function}\index{Euler's Totient Function}.
  \end{enumerate}
\end{crly}

\newthought{Let} $p$ be a prime. Recall that one consequence of \hyperref[thm:lagrange_s_theorem]{Lagrange's Theorem} is that every group $G$ of order $p$ is cyclic, i.e. $G \cong C_p$.

An analogous notion in rings is the following:

\begin{propo}[Ring With Prime Order Is Isomorphic to Integer Modulo Prime]
\label{propo:ring_with_prime_order_is_isomorphic_to_integer_modulo_prime}
  If $R$ is a non-trivial ring with $\abs{R} = p$ where $p$ is prime, then $R \cong \mathbb{Z}_p$.
\end{propo}

\begin{proof}
  Define
  \begin{equation*}
    \Theta : \mathbb{Z}_p \to R \qquad [k] \mapsto k \cdot 1_R.
  \end{equation*}
  Note that since $R$ is an additive group with $\abs{R} = p$, by \hyperref[thm:lagrange_s_theorem]{Lagrange's Theorem}, $o(1_R) = 1$ or $p$. Since $R$ is non-trivial, we have that $1_R \neq 0$ by the remark on the definition of a \hyperref[defn:trivial_ring]{trivial ring}, and so $o(1_R) \neq 1$. Thus $o(1_R) = p$. Then, by \cref{propo:implications_of_the_characteristic}, we have
  \begin{equation*}
    [k] = [m] \iff p \, | \, (k - m) \iff (k - m) 1_R = 0 \iff k \cdot 1_R = m \cdot 1_R
  \end{equation*}
  in $R$. Thus, $\Theta$ is well-defined and injective. $\Theta$ is also a ring homomorphism \sidenote{
  \begin{ex}
    Prove that $\Theta$ is a ring homomorphism.
  \end{ex}
  }.

  \begin{proof}[$\Theta$ is a ring homomorphism]
    $\forall [a], [b] \in \mathbb{Z}$, we have
    \begin{align*}
      \Theta([a][b]) &= \Theta( [ab] ) = ab \cdot 1_R \\
                     &= ( a \cdot 1_R )( b \cdot 1_R ) = \Theta([a])\Theta([b]).
    \end{align*}
    \begin{equation*}
      \Theta([1]) = 1 \cdot 1_R = 1_R
    \end{equation*}
    and
    \begin{align*}
      \Theta([a] + [b]) &= \Theta([ a + b ]) = (a + b) \cdot 1_R \\
                        &= a \cdot 1_R + b \cdot 1_R = \Theta([a]) + \Theta([b]).
    \end{align*}
    So $\Theta$ is a ring homomorphism.
  \end{proof}
  Now because $\abs{\mathbb{Z}_p} = p = \abs{R}$ and $\Theta$ is injective, $\Theta$ must be surjective. Therefore $\Theta$ is a ring isomorphism and hence $R \cong \mathbb{Z}_p$ as required. \qed
\end{proof}

% subsection isomorphism_theorems_for_rings_continued (end)

% section rings_continued_4 (end)

\section{Commutative Rings}%
\label{sec:commutative_rings}
% section commutative_rings

\subsection{Integral Domain and Fields}%
\label{sub:integral_domain_and_fields}
% subsection integral_domain_and_fields

\begin{defn}[Units]\index{Units}
\label{defn:units_and_group_of_units}
  Let $R$ be a ring. We say that $u \in R$ is a \hlnoteb{unit} if $u$ has a multiplicative inverse in $R$, and denote it by $u^{-1}$. We have
  \begin{equation*}
    uu^{-1} = 1 = u^{-1} u
  \end{equation*}
\end{defn}

\begin{note}
  If $u$ is a unit in $R$, and $r, s \in R$, we have
  \begin{align*}
    ur = us &\implies r = s \quad \text{\hlnotea{( Right Cancellation )}} \\
    ru = su &\implies r = s \quad \text{\hlnotea{( Left Cancellation )}}
  \end{align*}
  Let $R^*$ denote the set of all units in $R$. We know that the definition of a ring is that $R$ is ``almost'' a group under multiplication except that its elements do not necessarily have multiplicative inverses. Since $R^* \subseteq R$ is the set that contains all units, i.e. all elements with multiplicative inverses in $R$, we have that $(R^*, \cdot)$ is a group. This is called the \hldefn{Group of Units} of $R$.
\end{note}

\begin{eg}
  Note that $2$ is a unit in $\mathbb{Q}$, but it is not a unit in $\mathbb{Z}$. We have that
  \begin{equation*}
    \mathbb{Q}^* = \mathbb{Q} \setminus \{0\} \text{ and } \mathbb{Z}^* = \{ \pm 1 \}
  \end{equation*}
\end{eg}

\begin{eg}
  Consider the ring of \hldefn{Gaussian Integers},
  \begin{equation*}
    \mathbb{Z}[i] = \{a + bi : a, b \in \mathbb{Z}, \, i^2 = -1 \} \leq \mathbb{C}.
  \end{equation*}
  Then
  \begin{equation*}
    \mathbb{Z}[i]^* = \{ \pm 1, \pm i \} \leq \mathbb{C}.
  \end{equation*}

  \begin{proof}
    \textbf{$\mathbb{Z}[i] \leq \mathbb{C}$} : \\
    Note that $1 = 1 + 0 \cdot i \in \mathbb{Z}[i]$. $\forall x, y \in \mathbb{Z}[i]$, write
    \begin{equation*}
      x = a + bi \quad y = c + di
    \end{equation*}
    for some $a, b, c, d \in \mathbb{Z}$. Observe that
    \begin{equation}\label{eq:q5_a_1}
      xy = (a + bi)(c + di) = (ac - bd) + i (bc + ad) \in \mathbb{Z}[i]
    \end{equation}
    since $(ab - bd), (bc + ad) \in \mathbb{Z}$. Also, with a similar reason
    \begin{equation*}
      x - y = a + bi - c - di = (a - c) + i (b - d) \in \mathbb{Z}[i].
    \end{equation*}
    Therefore, by the Subring Test, $\mathbb{Z}[i]$ is a subring of $\mathbb{C}$.

    \textbf{$\mathbb{Z}[i]^* = \left\{ \pm 1, \, \pm i \right\}$} : \\
    In order for $x \in \mathbb{Z}[i]^*$, $x \neq 0$, to have an inverse, we must have that
      \begin{equation*}
        (ac - bd) + i ( bc + ad ) = 1
      \end{equation*}
      from \cref{eq:q5_a_1}, where we shall note that $a^2 + b^2 \neq 0$. Observe that
      \begin{align*}
        &( ac - bd ) + i ( bc + ad ) = 1 \\
        &\iff \begin{pmatrix}
            ac - bd \\ bc + ad 
        \end{pmatrix} = \begin{pmatrix}
              1 \\ 0
        \end{pmatrix} \\
        &\iff \begin{pmatrix}
            a & -b \\ b & a
        \end{pmatrix} \begin{pmatrix}
          c \\ d
        \end{pmatrix} = \begin{pmatrix}
          1 \\ 0
        \end{pmatrix} \\
        &\iff \begin{pmatrix}
          c \\ d
        \end{pmatrix} = \frac{1}{a^2 + b^2} \begin{pmatrix}
          1 \\ 0
        \end{pmatrix} \begin{pmatrix}
          a & b \\ -b & a
        \end{pmatrix} = \frac{1}{a^2 + b^2} \begin{pmatrix}
          a \\ -b
        \end{pmatrix} \\
        &\iff c + id = \frac{a}{a^2 + b^2} + i \left( \frac{-b}{a^2 + b^2} \right)
      \end{align*}
      Since $c, d \in \mathbb{Z}$, we must have that
      \begin{equation*}
        \frac{a}{a^2 + b^2}, \, \frac{-b}{a^2 + b^2} \in \mathbb{Z}.
      \end{equation*}
      Also, since $a^2 + b^2 > a, \, b$, we have that
      \begin{equation*}
        \frac{a}{a^2 + b^2}, \, \frac{-b}{a^2 + b^2} \in \mathbb{Z} \iff a^2 + b^2 = 1.
      \end{equation*}
      Since $a, \, b \in \mathbb{Z}$, we have that the only integer solutions to $a^2 + b^2 = 1$ are
      \begin{equation*}
        (1, \, 0) , \, (0, \, 1) , \, ( -1, \, 0 ) , \, (0, \, -1),
      \end{equation*}
      which corresponds to
      \begin{equation*}
        x = \pm 1, \, \pm i.
      \end{equation*}
      Since we began with $x$ begin an arbitrary element in $\mathbb{Z}[i]^*$, we have that
      \begin{equation*}
        \mathbb{Z}[i]^* = \{ \pm 1, \, \pm i \}
      \end{equation*}
      as required.
    \qed
  \end{proof}
\end{eg}

\begin{defn}[Division Ring and Field]\index{Division Ring}\index{Field}
\label{defn:division_ring_and_field}
  A non-trivial ring $R$ is a \hlnoteb{division ring} if
  \begin{equation*}
    R^* = R \setminus \{0\}.
  \end{equation*}
  A commutative division ring is a \hlnoteb{field}.
\end{defn}

\begin{eg}
  $\mathbb{Q}, \, \mathbb{R}, \, \mathbb{C}$ are fields but $\mathbb{Z}$ is not.
\end{eg}

\begin{eg}
  $\mathbb{Z}_n$ is a field $\iff \, n$ is prime.
\end{eg}

\begin{remark}
\marginnote{This remark is not as useful or spectacular within this course, but it will be once we go into PMATH348 contents.} If $R$ is a division ring or a field, then its only ideals are $\{0\}$ or $R$, since if $A \neq \{0\}$ is an ideal of $R$, then $\exists a \in A$, $a \neq 0$, such that $1 = aa^{-1} \in A$, which implies that $A = R$ by \cref{propo:the_only_ideal_with_the_multiplicative_identity_is_the_ring_itself}.
\end{remark}

\begin{remark}
  It can be shown that every finite division ring is a field, and this is known as \href{https://en.wikipedia.org/wiki/Wedderburn_Theorem}{Wedderburn's Theorem}.
\end{remark}

\newthought{Note that} if $n = ab$ for some integer $n$ with $0 < a, b < n$, then in $\mathbb{Z}$ we have
\begin{equation*}
  [a][b] = [n] = [0]
\end{equation*}
but $[a] \neq [0] \neq [b]$ by our definition of $a, b$.

\begin{defn}[Zero Divisor]\index{Zero Divisor}
\label{defn:zero_divisor}
  Let $R$ be a non-trivial ring. If $0 \neq a \in R$, then $a$ is called a \hlnoteb{zero divisor} if $\exists 0 \neq b \in R$ such that $ab = 0$.
\end{defn}

% subsection integral_domain_and_fields (end)

% section commutative_rings (end)

% chapter lecture_24_jun_27th_2018 (end)

\chapter{Lecture 25 Jun 29th 2018}%
\label{chp:lecture_25_jun_29th_2018}
% chapter lecture_25_jun_29th_2018

\section{Commutative Rings (Continued)}%
\label{sec:commutative_rings_continued}
% section commutative_rings_continued

\subsection{Integral Domain and Fields (Continued)}%
\label{sub:integral_domain_and_fields_continued}
% subsection integral_domain_and_fields_continued

Recall the definition of a \hlnoteb{zero divisor}.

\begin{defnnonum}[Zero Divisor]
\label{defnnonum:zero_divisor}
Let $R$ be a non-trivial ring. If $0 \neq a \in R$, then $a$ is called a \hlnoteb{zero divisor} if $\exists 0 \neq b \in R$ such that $ab = 0$.
\end{defnnonum}

\begin{eg}
  $[2], [3], [6]$ in $\mathbb{Z}_6$ are all zero divisors since
  \begin{equation*}
    [0] = [2][3] = [4][3] = [6][2].
  \end{equation*}
\end{eg}

\begin{eg}
  The matrix $\begin{bmatrix} 1 & 0 \\ 0 & 0 \end{bmatrix}$ is a zero divisor in $M_n(\mathbb{R})$ since
  \begin{equation*}
    \begin{bmatrix}
      1 & 0 \\ 0 & 0
    \end{bmatrix} \begin{bmatrix}
      0 & 0 \\ 0 & 1
    \end{bmatrix} = \begin{bmatrix}
      0 & 0 \\ 0 & 0
    \end{bmatrix}.
  \end{equation*}
\end{eg}

\begin{propo}[Ring Cancellations and Zeros]
\label{propo:ring_cancellations_and_zeros}
  Let $R$ be a ring. TFAE:
  \begin{enumerate}
    \item $\forall ab = 0 \in R \quad a = 0 \lor b = 0$ ;
    \item $\forall ab = ac \in R \land a \neq 0 \implies b = c$ ;
    \item $\forall ba = ca \in R \land a \neq 0 \implies b = c$.
  \end{enumerate}
\end{propo}

\begin{proof}
  It suffices to prove $(1) \iff (2)$, since $(1) \iff (3)$ would have a similar argument.

    \noindent $(1) \implies (2)$: Let $ab = ac$ with $a \neq 0$. Then $a(b - c) = 0$. Then by $(1)$, since $a \neq 0$, $(b - c) = 0 \iff b = c$.

    \noindent $(2) \implies (1)$: Let $ab = 0 \in R$. We now have 2 cases:
    \begin{enumerate}
      \item[Case 1] If $a = 0$, we are done.
      \item[Case 2] If $a \neq 0$, then $ab = 0 = a \cdot 0$, and so by $(2)$, $b = 0$.
    \end{enumerate}\qed
\end{proof}

With that, we can make the following definition.

\begin{defn}[Integral Domain]\index{Integral Domain}
\label{defn:integral_domain}
A commutative ring $R \neq \{0\}$ (i.e. non-trivial ring) is called an \hlnoteb{integral domain} if it has \hlimpo{no zero divisor}, i.e. if $ab = 0 \in R$ then $a = 0$ or $b = 0$.
\end{defn}

\begin{eg}\label{eg:integral_domain_that_is_not_a_field}
  $\mathbb{Z}$ is an integral domain since $ab = 0 \implies a = 0$ or $b = 0$.
\end{eg}

\begin{eg}
  Note that if $p$ is prime, then $p \, | \, ab \implies p \, | \, a \, \lor p \, | \, b$, i.e. $[a][b] = [0]$ in $\mathbb{Z}_p \implies [a] = 0$ or $[b] = 0$. So $\mathbb{Z}_p$ is an integral domain.

  However, for $n$ not prime, with $n = ab$, if we have $n = ab$ such that $1 < a, b < n$, then
  \begin{equation*}
    [a][b] = [0] \text{ in } \mathbb{Z}_n
  \end{equation*}
  but neither $[a]$ nor $[b]$ is $[0]$.

  With that, we have that $\mathbb{Z}_n$ is an integral domain if and onoly if $n$ is prime.
\end{eg}

\begin{propo}[Fields are Integral Domains]
\label{propo:fields_are_integral_domains}
Every field is an integral domain.
\end{propo}

\begin{proof}
  $\forall a, b \in R$, where $R$ is a field, such that $ab = 0$, we want to show that $a = 0$ or $b = 0$. We have 2 cases:

  \noindent \underline{\textbf{Case 1}}: $a = 0$. There is nothing to do since the proof is complete.

  \noindent \underline{\textbf{Case 2}}: $a \neq 0$. Since $a \neq 0 \in R$, we know that $\exists a^{-1} \in R$ since $R$ is a field. And so
  \begin{equation*}
    b = a^{-1} ab = a^{-1} \cdot 1 = 0
  \end{equation*}
  
  Therefore, by definition, the field $R$ is an integral domain.\qed
\end{proof}

\begin{note}
  Using the proof from above, we can show that every subring of a field is an integral domain\sidenote{This will become useful in PMATH348}.
\end{note}

\begin{note}
  The converse of \cref{propo:fields_are_integral_domains} is not true. As shown in \cref{eg:integral_domain_that_is_not_a_field}, $\mathbb{Z}$ is an integral domain but not a field.
\end{note}

However, we have the following partial converse:

\begin{propo}[Finite Integral Domains are Fields]
\label{propo:finite_integral_domains_are_fields}
  Every \hlnotec{finite} integral domain is a field.
\end{propo}

\begin{proof}
  Let $R$ be a finite integral domain, say $\abs{R} = n \in \mathbb{N}$. Let
  \begin{equation*}
    R = \{r_1, r_2, ..., r_n\}.
  \end{equation*}
  Then for some $a \in R$ such that $a \neq 0$, by \cref{propo:ring_cancellations_and_zeros}, the set
  \begin{equation*}
    \{ ar_1, ar_2, ..., ar_n \}
  \end{equation*}
  have distinct elements. Since $R$ is finite and so $\abs{aR} = n$, and $aR \subseteq R$, we have that $aR = R$. In particular, $\exists 1 \in aR$ such that $1 = ab$ for some $b \in R$ \sidenote{We can prove for a more general case by not assuming that $R$ is a commutative ring: We can find $c \in R$ such that $1 = ca$. Then
  \begin{equation*}
    b = (ca)b = c(ab) = c.
  \end{equation*}}. It follows that $ab = 1 = ba$ since $R$ is commutative, which then implies that $a$ is a unit. Therefore, $R$ is a field.\qed
\end{proof}

Recall that the \hyperref[defn:characteristic_of_a_ring]{characteristic} of a ring $R$, denoted by $\ch(R)$, is the order of the unity, $1_R$, in $(R, +)$, and write
\begin{equation*}
  \ch(R) = \begin{cases}
    0 & o(1_R) = \infty \\
    n & o(1_R) = n \in \mathbb{N}
  \end{cases}
\end{equation*}

\begin{propo}[Integral Domains have Zero or Prime Characteristics]
\label{propo:integral_domains_have_zero_or_prime_characteristics}
  The characteristic of any integral domain is $0$ or a prime $p$.
\end{propo}

\begin{proof}
  Let $R$ be an integral domain. We have 2 cases:

  \noindent \underline{\textbf{Case 1}}: $\ch(R) = 0$. Our job is done.
  
  \noindent \underline{\textbf{Case 2}}: $\ch(R) = n \in \mathbb{N}$. Suppose $n \neq p$ a prime, and say $n = ab$ for some $a, b \in R$ such that $1 < a, b < n$. If $1$ is the unity of $R$, then by \cref{propo:more_properties_of_rings}, we have
  \begin{equation*}
    ab = (a \cdot 1)(b \cdot 1) = (ab) (1) = n (1) = 0.
  \end{equation*}
  Since $R$ is an integral domain, we have that either
  \begin{equation*}
    a \cdot 1 = 0 \text{ or } b \cdot 1 = 0.
  \end{equation*}
  This contradicts that fact that $n$ is the characteristic. Therefore, $n$ must be prime.\qed
\end{proof}

\begin{note}
  Let $R$ be an integral domain with $\ch(R) = p$ a prime. For $a, b \in R$, by the \hlnotea{Binomial Theorem}, we have
  \begin{equation*}
    (a + b)^p = \sum_{i=1}^{p} \binom{p}{i} a^{p - i} b^i.
  \end{equation*}
  Since $p$ is prime, we have $p \, | \binom{p}{i} = \frac{p(p-1)\hdots(p-i+1)}{i!}$ for $1 \leq i \leq p - 1$. Therefore, since $\ch(R) = p$, we have that\marginnote{This is known as the \href{https://en.wikipedia.org/wiki/Freshman\'s_dream}{Freshman's Dream}.}
  \begin{equation*}
    (a + b)^p = a^p + b^p
  \end{equation*}
\end{note}

% subsection integral_domain_and_fields_continued (end)

% section commutative_rings_continued (end)

% chapter lecture_25_jun_29th_2018 (end)

\chapter{Lecture 26 Jul 04th 2018}%
\label{chp:lecture_26_jul_04th_2018}
% chapter lecture_26_jul_04th_2018

\section{Commutative Rings (Continued 2)}%
\label{sec:commutative_rings_continued_2}
% section commutative_rings_continued_2

\subsection{Prime Ideals and Maximal Ideals}%
\label{sub:prime_ideals_and_maximal_ideals}
% subsection prime_ideals_and_maximal_ideals

\begin{defn}[Prime Ideals]\index{Prime Ideals}
\label{defn:prime_ideals}
  Let $R$ be a commutative ring. An ideal $P \neq R$ is a prime ideal of $R$ if $r, s \in R$ satisfy: $rs \in R \implies r \in P$ or $s \in P$.
\end{defn}

\begin{eg}
  For $n \in \mathbb{N} \setminus \{1\}$, $n \mathbb{Z} = \lra{n}$ is a prime ideal if and only if $n$ is prime.
\end{eg}

\begin{propo}[Ideal is Prime $\iff$ Quotient of Ring by Ideal is an Integral Domain]
\label{propo:ideal_is_prime_iff_quotient_of_ring_by_ideal_is_an_integral_domain}
If $R$ is a commutative ring, then an ideal $P \neq R$ of $R$ is a prime ideal if and only if $\faktor{R}{P}$ is an integral domain.
\end{propo}

\begin{proof}
  Since $R$ is commutative, so is $\faktor{R}{P}$. Since $P \neq R$, we know that $1 \notin P$ \sidenote{See \cref{propo:the_only_ideal_with_the_multiplicative_identity_is_the_ring_itself}.}, i.e. $0 + P = P \neq 1 + P$, and so $\faktor{R}{P}$ is a non-trivial ring.

  \noindent To prove $(\implies)$, let $(r + P)(s + P) = 0 + P = P$. Since $P$ is an ideal\sidenote{See \cref{propo:equivalent_defn_of_a_well_defined_coset_multiplication}.}, we have that $rs + P = P$ and so $rs \in P$. WLOG, since $P$ is a prime ideal, if $r \in P$, then $r + P = P$. And so $\faktor{R}{P}$ is an integral domain.

  \noindent To prove $(\impliedby)$, let $rs \in P$. Then since $P$ is an ideal,
  \begin{equation*}
    (r + P)(s + P) = rs + P = P.
  \end{equation*}
  Since $\faktor{R}{P}$ is an integral domain, either
  \begin{equation*}
    r + P = P \text{ or } s + P = P
  \end{equation*}
  so $r \in P$ or $s \in P$, which implies that $P$ is a prime ideal.\qed
\end{proof}

\begin{defn}[Maximal Ideals]\index{Maximal Ideals}
\label{defn:maximal_ideals}
  Let $R$ be a (commutative) ring. An ideal $M \neq R$ or $R$ is a maximal ideal if $\forall A$ that is an ideal of $R$, we have that
  \begin{equation*}
    M \subseteq A \subseteq R \implies A = M \text{ or } A + R.
  \end{equation*}
\end{defn}

\begin{propo}[Ideal is Maximal $\iff$ Quotient of Ring by Ideal is a Field]
\label{propo:ideal_is_maximal_iff_quotient_of_ring_by_ideal_is_a_field}
  If $R$ is a commutative ring, then an ideal $M \neq R$ is a maximal ideal if and only if $\faktor{R}{M}$ is a field.
\end{propo}

\begin{proof}
  Similar to the proof of \cref{propo:ideal_is_prime_iff_quotient_of_ring_by_ideal_is_an_integral_domain}, $\faktor{R}{M}$ is a nontrivial commutative ring. Let $r \in R$.

  \noindent $(\implies)$ Suppose $M$ is a maximal ideal. Since $\faktor{R}{M}$ is non-trivial, let $r + M \neq 0 + M \in \faktor{R}{M}$. Let $\lra{r} = rR$ Note that $r \notin M$ and $r \in \lra{r} + M$. Thus, $M \subsetneq \lra{r} + M$. Since $M$ is maximal and $M$ is a proper subset of $\lra{r} + M$, we have that $\lra{r} + M = R$. In particular, we have $1 \in \lra{r} + M$ and so $\exists s \in R$ and $m \in M$ such that $1 = rs + m$. Thus
  \begin{equation*}
    1 + M = rs + M = (r + M)(s + M).
  \end{equation*}
  Therefore $s + M$ is the multiplicative inverse of $r + M$, and so $\faktor{R}{M}$ is a field.

  \noindent $(\impliedby)$ Since $\faktor{R}{M}$ is a non-trivial field, we know $0 + M \neq 1 + M$. Therefore $M \neq R$. Suppose $A$ is an ideal such that $M \subsetneq A \subseteq R$. Choose $r \in A \setminus M$. Since $r \notin M$ and so $r + M \neq 0 + M$ and $\faktor{R}{M}$ is a field, we have that $\exists s + M \in \faktor{R}{M}$ such that $(r + M)(s + M) = 1 + M$. Since $M$ is an ideal, we have
  \begin{equation*}
    rs + M = 1 + M \implies \exists m \in M \quad 1 = rs + m.
  \end{equation*}
  Since $r, m \in A$ and $A$ is an ideal, we have that $1 \in A$ and so $A = R$, implying that $M$ is maximal. \qed
\end{proof}

Combining \cref{propo:fields_are_integral_domains}, \cref{propo:ideal_is_prime_iff_quotient_of_ring_by_ideal_is_an_integral_domain}, and \cref{propo:ideal_is_maximal_iff_quotient_of_ring_by_ideal_is_a_field}, we get the following corollary.

\begin{crly}[Maximal Ideals of a Commutative Rings are Prime]
\label{crly:maximal_ideals_of_a_commutative_rings_are_prime}
  Every maximal ideal of a commutative ring is a prime ideal.
\end{crly}

\begin{note}
  The converse of \cref{crly:maximal_ideals_of_a_commutative_rings_are_prime} is not true.
\end{note}

\begin{eg}
  In $\mathbb{Z}$, $\{0\}$ is a prime ideal, but is clearly not maximal.
\end{eg}

% subsection prime_ideals_and_maximal_ideals (end)

\subsection{Fields of Fractions}%
\label{sub:fields_of_fractions}
% subsection fields_of_fractions

Recall that every subring of a field is an integral domain. The converse is actually true\sidenote{This is in comparison with \cref{propo:fields_are_integral_domains}.}, i.e. every integral domain $R$ is isomorphic to a subring of a field $F$.

Let $R$ be an integral domain and $D = R \setminus \{0\}$. Consider
\begin{equation*}
  X = R \times D = \{ (r, s) : r \in R, s \in D \}
\end{equation*}
We say that
\begin{equation}\label{eq:defining_fractions}
  (r, s) \equiv (r_1, s_1) \in X \iff rs_1 = r_1 s
\end{equation}

\begin{eg}
  Show that \cref{eq:defining_fractions} is an equivalence relation.
  \begin{enumerate}
    \item $(r, s) \equiv (r, s)$
    \item $(r, s) \equiv (r_1, s_1) \iff (r_1, s_1) \equiv (r, s)$
    \item $(r, s) \equiv (r_1, s_1) \, \land \, (r_1, s_1) \equiv (r_2, s_2) \implies (r, s) = (r_2, s_2)$
  \end{enumerate}
\end{eg}

Note that using the above idea, we can construct the smallest field that contains $\mathbb{Z}$, and that field is $\mathbb{Q}$. Motivated by this idea, we make the following definition.

\begin{defn}[Fraction]\index{Fraction}
\label{defn:fraction}
  Let $R$ be an integral domain, $D = R \setminus \{0\}$, and $X = R \times D$. The \hlnoteb{fraction}, $\frac{r}{s}$ to be the equivalent class $[(r, s)]$ of the pair $(r, s) \in X$.
\end{defn}

\newthought{Let} $F$ denote the set of all these fractions, i.e.
\begin{equation*}
  F = \{ [ (r, s) ] : r \in R, s \in D \} = \{ \frac{r}{s} : r \in R, s \in R \setminus \{0\} \}.
\end{equation*}
The addition and multiplication of $F$ are defined by
\begin{gather*}
  \frac{r}{s} + \frac{r_1}{s_1} = \frac{rs_1 + sr_1}{ss_1} \\
  \frac{r}{s} \cdot \frac{r_1}{s_1} = \frac{rr_1}{ss_1}
\end{gather*}
where we note that $ss_1 \neq 0$ since $s, s_1 \in R \setminus \{0\}$ and $R$ is an integral domain.

It can be shown that $F$ is a field\sidenote{Prove this as an easy exercise to ease yourself with the concept.
\begin{ex}
  Prove that $F$ is a field.
\end{ex}
}. Also, we have $R \cong R' = \frac{r}{1} : r \in R \} \subseteq F$.

\begin{thm}[Field of Fractions]
\index{Field of Fractions}
\label{thm:field_of_fractions}
  Let $R$ be an integral domain. Then there is a field $F$ containing fractions $\frac{r}{s}$ with $r, s \in R$ and $s \neq 0$. By identifying that $r = \frac{r}{1}$, for any $r \in R$, we have that $R$ is a subring of $F$. The field $F$ is called the \hlnoteb{field of fractions} of $R$.
\end{thm}

\begin{note}
  We can generalize $D = R \setminus \{0\}$ to any subset $D \subseteq R$ satisfying
  \begin{enumerate}
    \item $1 \in D$
    \item $0 \notin D$
    \item $a, b \in D \implies ab \in D$
  \end{enumerate}
\end{note}

% subsection fields_of_fractions (end)

% section commutative_rings_continued_2 (end)

% chapter lecture_26_jul_04th_2018 (end)

\chapter{Lecture 27 Jul 06th 2018}%
\label{chp:lecture_27_jul_06th_2018}
% chapter lecture_27_jul_06th_2018

\section{Polynomial Ring}%
\label{sec:polynomial_ring}
% section polynomial_ring

\subsection{Polynomials}%
\label{sub:polynomials}
% subsection polynomials

\begin{defn}[Polynomials]
\label{defn:polynomials}
  Let $R$ be a ring and $x$ a variable. Let
  \begin{equation*}
    R[x] = \left\{ f(x) = \sum_{i=0}^{m} a_i x^i : m \in \mathbb{N} \cup \{0\}, \, a_i \in R, 0 \leq i \leq m \right\}.
  \end{equation*}
  Each element in $R[x]$ is called a \hldefn{polynomial} in $x$ over $R$. If $a_m \neq 0$, we say that $f(x)$ has \hldefn{degree} $m$, denoted by $\deg f = m$, and we say that $a_m$ is the \hldefn{leading coefficient} of $f(x)$.

  If $\deg f = 0$, then $f(x) = a_0 \in R$. In this case, we call $f(x)$ a \hldefn{constant polynomial}. Note if
  \begin{equation*}
    f(x) = 0 \iff a_0 = a_1 = ... = a_m = 0,
  \end{equation*}
  we define $\deg 0 = - \infty$, and $f(x)$ is called a \hldefn{zero polynomial}.
\end{defn}

For
\begin{gather*}
  f(x) = a_0 + a_1 x + \hdots + a_m x^m \\
  g(x) = b_0 + b_1 x + \hdots + b_n x^n
\end{gather*}
in $R[x]$. If $m \leq n$, we can define $a_i = 0$ for $m + 1 \leq i \leq n$. Then the addition and multiplication on $R[x]$ can be defined as
\begin{align*}
  f(x) + g(x) &= (a_0 + b_0) + (a_1 + b_1) x + \hdots + (a_n + b_n) x^n \\
  f(x) g(x) &= (a_0 + a_1 x + \hdots + a_m x^m) (b_0 + b_1 x + \hdots + b_n x^n) \\
            &= a_0 b_0 + (a_1 b_0 + a_1 b_0) x + (a_2 b_0 + a_1 b_1 + a_0 b_2) x^2 + \hdots \\
            &\quad + (a_m b_m) x^{m + n} \\
            &= c_0 + c_1 x + \hdots + c_{m + n} x^{m + n}
\end{align*}
where $c_i = a_0 b_i + a_1 b_{i - 1} + \hdots + a_{i - 1} b_1 + a_i b_0$.

\begin{propo}[Ring is a Subring of Its Polynomial Ring]
\label{propo:ring_is_a_subring_of_its_polynomial_ring}
  Let $R$ be a ring and $x$ a variable.
  \begin{enumerate}
    \item $R[x]$ is a ring
    \item $R$ is a subring of $R[x]$
    \item If $Z = Z(R)$ denote the center of $R$, then the center of $R[x]$ is $Z[x]$. In particular, $x$ is in the center of $R[x]$.
  \end{enumerate}
\end{propo}

\begin{proof}
  \begin{enumerate}
    \item \textbf{Checking all 9 properties}: Let 
          \begin{gather*}
            f(x) = a_0 + a_1 x + \hdots + a_m x^m \\
            g(x) = b_0 + b_1 x + \hdots + b_n x^n \\
            h(x) = d_0 + d_1 x + \hdots + d_k x^k
          \end{gather*}
          be in $R[x]$.
      \begin{itemize}
        \item (\textbf{Closed under addition and multiplication})
          Suppose, WLOG, that $m \leq n$. Let $a_i = 0$ for $m + 1 \leq i \leq n$. Then
          \begin{equation*}
            f(x) + g(x) = (a_0 + b_0) + (a_1 + b_1) x + \hdots + (a_n + b_n) x^n
          \end{equation*}
          and we observe that $a_i + b_i \in R$ for $0 \leq i \leq n$ since $R$ is a ring. And so $f(x) + g(x) \in R[x]$. Also, we have
          \begin{align*}
            f(x) g(x) = c_0 + c_1 x + \hdots + c_{m + n} x^{m + n}
          \end{align*}
          where $c_i = a_0 b_i + a_1 b_{i - 1} + \hdots + a_{i - 1} b_1 + a_i b_0 \in R$ for $1 \leq i \leq m + n$. And so $f(x) g(x) \in R[x]$.
        \item (\textbf{Commutativity of Addition}) Suppose, WLOG, that $m \leq n$. Let $a_i = 0$ for $m + 1 \leq i \leq n$. Then
          \begin{align*}
            f(x) + g(x) &= (a_0 + b_0) + (a_1 + b_1) x + \hdots + (a_n + b_n) x^n \\
                        &= (b_0 + a_0) + (b_1 + a_1) x + \hdots + (b_n + a_n) x^n \\
                        &= g(x) + f(x)
          \end{align*}
        \item (\textbf{Zero and Unity}) It is clear that the zero and unity of $R$ are the zero and unity of $R[x]$ respectively, since only
          \begin{equation*}
            f(x) + 0 = f(x) = 0 + f(x)
          \end{equation*}
          and
          \begin{equation*}
            1 f(x) = f(x) = f(x) \cdot 1.
          \end{equation*}
        \item (\textbf{Associativity}) Suppose, WLOG, that $m \leq n \leq k$. Let $a_i = b_j = 0$ for $m + 1 \leq i \leq k$ and $n + 1 \leq j \leq k$. Then
          \begin{align*}
            &f(x) + [ g(x) + h(x) ]\\
            &= f(x) + [ (b_0 + d_0) + (b_1 + d_1) x + \hdots + (b_k d_k) x^k ] \\
            &= (a_0 + b_0 + d_0) + (a_1 + b_1 + d_1) x + \hdots + (a_k + b_k + d_k) x^k \\
            &= [(a_0 + b_0) + (a_1 + b_1) x + \hdots + (a_k + b_k) x^k] + d(x) \\
            &= [ f(x) + g(x) ] + h(x)
          \end{align*}
          and if we use the summation notation for $f(x), g(x)$ and $h(x)$, we have
          \begin{align*}
            f(x) [ g(x) d(x) ] &= f(x) \left[ \left( \sum_{j=0}^{n} b_j x^j \right)\left( \sum_{l=0}^{k} d_l x^l \right) \right] \\
                               &= \left[ \sum_{i=0}^{m} a_i x^i \right] \left[ \sum_{j=0}^{n} \sum_{l=0}^{k} b_j d_l x^{j + l} \right] \\
                               &= \sum_{i=0}^{m} \sum_{j=0}^{n} \sum_{l=0}^{k} a_i b_j d_l x^{i + j + k} \\
                               &= \left[ \sum_{i=0}^{m} \sum_{j=0}^{n} a_i b_j x^{i + j} \right] \left[ \sum_{l=0}^{k} d_l x^l \right] \\
                               &= \left[ \left( \sum_{i=0}^{m} a_i x^i \right) \left( \sum_{j=0}^{n} b_j x^j \right) \right] h(x) \\
                               &= [ f(x) g(x) ] h(x)
          \end{align*}
        \item (\textbf{Inverse}) Since $R$ is a ring, and in particular an additive ring, for each $a_i \in R$, $0 \leq i \leq m$, we have that $\exists (-a_i) \in R$ such that $a_i + (-a_i) = 0$. Particularly, we have that
          \begin{equation*}
            - f(x) = ( - a_0 ) + ( - a_1 ) x + ( - a_2 ) x^2 + \hdots + ( - a_m ) x^m
          \end{equation*}
          is the inverse of $f(x) \in R[x]$.
        \item (\textbf{Distributivity}) Again, using the summation notation, since $R$ is a ring, we have
          \begin{align*}
            &f(x) [ g(x) + h(x) ] \\
            &= \left[ \sum_{i=0}^{m} a_i x^i \right] \left[ \sum_{j=0}^{n} b_j x^j + \sum_{l=0}^{k} d_l x^l \right] \\
            &= \left[ \sum_{i=0}^{m} a_i x^i \right] \left[ \sum_{j=0}^{k} (b_j + d_j) x^j \right] \\
            &= \sum_{i=0}^{m} \sum_{j=0}^{k} a_i (b_j + d_j) x^{i + j} = \sum_{i=0}^{m} \sum_{j=0}^{k} (a_i b_j + a_i d_j) x^{i + j} \\
            &= \sum_{i=0}^{m} \sum_{j=0}^{k} a_i b_j x^{i + j} + \sum_{i=0}^{m} \sum_{j=0}^{k} a_i d_j x^{i + j} \\
            &= \sum_{i=0}^{m} \sum_{j=0}^{n} a_i b_j x^{i + j} + \sum_{i=0}^{m} \sum_{j=0}^{k} a_i d_j x^{i + j} \\
            &= f(x) g(x) + f(x) d(x).
          \end{align*}
          Proof for the other side is similar.
        \end{itemize}
        With that, we have that $R[x]$ is a ring.

      \item We already have that $R$ is a ring, and so it suffices to prove that $R \subseteq R[x]$. This is, however, rather simple, since $\forall r \in R$, we have that $r$ is a constant polynomial, and so $r \in R[x]$, and therefore $R \subseteq R[x]$.

      \item Let
        \begin{gather*}
          f(x) = a_0 + a_1 x + a_2 x^2 + \hdots + a_m x^m \in Z[x] \\
          g(x) = b_0 + b_1 x + b_2 x^2 + \hdots + b_n x^n \in R[x].
        \end{gather*}
        We have that
        \begin{equation*}
          f(x) g(x) = \sum_{i=0}^{m} \sum_{j=0}^{n} a_i b_j x^{i + j}.
        \end{equation*}
        Since $a_i \in Z$ for $0 \leq i \leq n$, we have
        \begin{equation*}
          f(x) g(x) = \sum_{i=0}^{m} \sum_{j=0}^{n} b_j a_i x^{i + j} = \sum_{j=0}^{n} \sum_{i=0}^{m} b_j a_i x^{j + i} = g(x) f(x)
        \end{equation*}
        for any $g(x) \in R[x]$. And so $Z[x] = Z(R[x])$.

        For $\supseteq$, $f(x) \in Z(R[x]) \implies \forall b \in R \subseteq R[x]$ we have $f(x) b = bf(x)$. It follows that
        \begin{equation*}
          \forall 0 \leq i \leq n \quad a_i b = b a_i
        \end{equation*}
        and so $a_i \in Z(R)$, which implies that $Z(R[x]) \subseteq Z[x]$. Therefore, $Z(R[x]) = Z[x]$.
  \end{enumerate}\qed
\end{proof}

\begin{warning}
  Althought $f(x) \in R[x]$ can be used to define a function from $R \to R$, the polynomial is not the same as the function it defines. For example, if $R = \mathbb{Z}_2$, then $\mathbb{Z}_2[x]$ is an infinite set, but there are only $4$ different functions from $\mathbb{Z}_2 \to \mathbb{Z}_2$
\end{warning}

\begin{propo}[Polynomial Ring is an Integral Domain]
\label{propo:polynomial_ring_is_an_integral_domain}
  Let $R$ be an integral domain. Then
  \begin{enumerate}
    \item $R[x]$ is an integral domain.
    \item If $f(x) \neq 0$ and $g(x) \neq 0$ in $R[x]$, then\sidenote{In order to preserve this for when we have the case of $\deg 0$, we have to define $\deg 0 = - \infty$. Otherwise, say if we define $\deg 0 = -1$, then if $\deg f = -1$, then $\deg (fg) = \deg f + \deg g$ would imply that $\deg g = -2$, which is undefined.}
      \begin{equation*}
        \deg (fg) = \deg f + \deg g
      \end{equation*}
    \item The units in $R[x]$ are $R^*$, the units in $R$.
  \end{enumerate}
\end{propo}

\begin{proof}
  We shall prove $(1)$ and $(2)$ together.
  \begin{enumerate}
    \item[1 \& 2.] Suppose $f(x) \neq 0 \neq g(x) \in R[x]$, say
      \begin{gather*}
        f(x) = a_0 + a_1 x + \hdots + a_m x^m \quad a_m \neq 0 \\
        g(x) = b_0 + b_1 x + \hdots + b_n x^n \quad b_n \neq 0.
      \end{gather*}
      Then
      \begin{equation*}
        f(x) g(x) = a_m b_n x^{m + n} + \hdots a_0 b_0.
      \end{equation*}
      Now since $R$ is an integral domain, we have that $a_m b_n \neq 0$ and so $f(x) g(x) \neq 0$. Thus $R[x]$ is an integral domain. Moreover, we see that
      \begin{equation*}
        \deg (fg) = m + n = \deg f + \deg g.
      \end{equation*}

      \setcounter{enumi}{2}
    \item Suppose that $u(x) \in R[x]$ is a unit of $R[x]$ with inverse $u^{-1}(x)$ which we shall write as $v(x)$. Since $u(x) v(x) = 1$, by $(2)$, we have that
      \begin{equation}\label{eq:polynomial_ring_is_an_integral_domain_eq1}
        \deg u + \deg v = \deg 1 = 0.
      \end{equation}
      Now by $(1)$, $R[x]$ is an integral domain, and so since $u(x) v(x) = 1$, we have that $u(x) \neq 0 \neq v(x)$. Therefore, $\deg u, \deg v \geq 0$, which implies that we must have $\deg u = 0 = \deg v$ from \cref{eq:polynomial_ring_is_an_integral_domain_eq1}. Therefore, units in $R[x]$ are from $R^*$.
  \end{enumerate}\qed
\end{proof}

\begin{note}
  Recall that $\mathbb{Z}_n$ is an integral domain if and only if $n = p$ a prime. If $n \neq p$, then, e.g., for $\mathbb{Z}_4[x]$, we have
  \begin{equation*}
    2x\cdot 2x = 4x^2 = 0
  \end{equation*}
  and so
  \begin{equation*}
    \deg (2x) + \deg (2x) \neq \deg (4x^2) = \deg (2x \cdot 2x).
  \end{equation*}
\end{note}

% subsection polynomials (end)

\subsection{Factorization of Polynomials}%
\label{sub:factorization_of_polynomials}
% subsection factorization_of_polynomials

\begin{defn}[Division of Polynomials]\index{Division of Polynomials}
\label{defn:division_of_polynomials}
  Let $R$ be a commutative ring and $f(x), g(x) \in R[x]$. We say that $f(x)$ divides $g(x)$, denoted as $f(x) \, | \, g(x)$ if $\exists q(x) \in R[x]$ such that
  \begin{equation*}
    g(x) = q(x) f(x)
  \end{equation*}
\end{defn}

\begin{defn}[Monic Polynomial]\index{Monic Polynomial}
\label{defn:monic_polynomial}
  Let $R$ be a commutative ring and $f(x) \in R[x]$. $f(x)$ is monic if its leading coefficient is $1$.
\end{defn}

We shall prove the following proposition next class.

\begin{propononum}
  Let $R$ be an integral domain, and $f(x), \, g(x) \in R[x]$ be monic polynomials. If $f(x) \, | \, g(x)$ and $g(x) \, | \, f(x)$, then $f(x) = g(x)$.
\end{propononum}

% subsection factorization_of_polynomials (end)

% section polynomial_ring (end)

% chapter lecture_27_jul_06th_2018 (end)

\chapter{Lecture 28 Jul 09th 2018}%
\label{chp:lecture_28_jul_09th_2018}
% chapter lecture_28_jul_09th_2018

\section{Polynomial Ring (Continued)}%
\label{sec:polynomial_ring_continued}
% section polynomial_ring_continued

\subsection{Factorization of Polynomials (Continued)}%
\label{sub:factorization_of_polynomials_continued}
% subsection factorization_of_polynomials_continued

Since the actual focus of our study right now is really fields instead of just integral domains, we shall use fields in place of integral domains or commutative rings from here on unless explicitly stated otherwise. So we redefine \cref{defn:division_of_polynomials} as follows:

\begin{defnnonum}[Division of Polynomials]
  Let $F$ be a field and consider $F[x]$. For $f(x), g(x) \in F[x]$, we say that $f(x) \, | \, g(x)$ if $\exists q(x) \in F[x]$ such that
  \begin{equation*}
    g(x) = q(x) f(x).
  \end{equation*}
\end{defnnonum}

and restate the last stated proposition as follows:

\begin{propo}[$f(x) \, | \, g(x) \, \land \, g(x) \, | \, f(x) \implies f(x) = g(x)$]
\label{propo:f_div_g_and_g_div_f_implies_f_is_g}
Let $F$ be a field and $f(x), g(x) \in F[x]$ be monic polynomials\sidenote{Note that polynomials being monic is analogous to integers being positive. For example, you (read: I) should try to reiterate the proof below by replacing the monic property with positive integers.}. If $f(x) \, | \, g(x)$ and $g(x) \, | \, f(x)$, then $f(x) = g(x)$.
\end{propo}

\begin{proof}
  Since $f(x) \, | \, g(x)$ and $g(x) \, | \, f(x)$, $\exists r(x), s(x) \in F[x]$ such that
  \begin{equation*}
    g(x) = r(x) f(x) \text{ and } f(x) = s(x) g(x).
  \end{equation*}
  Then
  \begin{equation*}
    f(x) = s(x) r(x) f(x).
  \end{equation*}
  By \cref{propo:polynomial_ring_is_an_integral_domain}, we have that
  \begin{equation*}
    \deg f = \deg s + \deg r + \deg f
  \end{equation*}
  and so
  \begin{equation*}
    \deg s + \deg r = 0 \implies \deg s = \deg r = 0 \quad \because \deg s, \, \deg r \geq 0.
  \end{equation*}
  And so $\exists t \in F$ such that $f(x) = tg(x)$. Since $f(x)$ and $g(x)$ are monic, we must have $t = 1$ and so $f(x) = g(x)$.\qed
\end{proof}

\begin{propo}[Division Algorithm for Polynomials]
\index{Division Algorithm}
\label{propo:division_algorithm_for_polynomials}
  Let $F$ be a field, and $f(x), g(x) \in F[x]$ with $f(x) \neq 0$. Then $\exists ! q(x), \, r(x) \in F[x]$ such that
  \begin{equation*}
    g(x) = q(x) f(x) + r(x)
  \end{equation*}
  with $\deg r < \deg f$.\sidenote{Note that this includes the case for $r = 0$, and this is yet another reason why we defined $\deg 0 = -\infty$.}
\end{propo}

\begin{proof}
  We shall first prove the existence of such a $q(x)$ and $r(x)$. For simplicity, write
  \begin{equation*}
    \deg f = m \text{ and } \deg g = n.
  \end{equation*}
  If $n < m$, then
  \begin{equation*}
    g(x) = 0 f(x) + g(x)
  \end{equation*}
  and we are done. Suppose that $n \geq m$ and proceed by induction of $n$. Write
  \begin{gather*}
    f(x) = a_0 + a_1 x + \hdots + a_m x^m \\
    g(x) = b_0 + b_1 x + \hdots + b_n x^n.
  \end{gather*}
  Consider\sidenote{We are implicitly using the fact that $x \in Z[x]$.}
  \begin{align*}
    g_1 (x) &= g(x) - b_n a_m^{-1} x^{n - m} f(x) \\
            &= (b_n x^n + \hdots + b_0) - b_n a_m^{-1} x^{n - m} ( a_m x^m + \hdots a_0 ) \\
            &= 0 x^n + (b_{n - 1} - b_n a_m^{-1} a_{m - 1}) x^{n - 1} + \hdots,
  \end{align*}
  thus either $g_1 (x) = 0$ or $g_1(x) \neq 0$, but in any case, $\deg g_1 < n$.

  \noindent \textbf{Case 1}: $g_1(x) = 0$. In this case, we have
  \begin{equation*}
    g(x) = b_n a_m^{-1} x^{n - m} f(x)
  \end{equation*}
  and so we can pick
  \begin{gather*}
    q(x) = b_n a_m^{-1} x^{n - m} \\
    r(x) = 0,
  \end{gather*}
  and the result follows.

  \noindent \textbf{Case 2}: $g_1(x) \neq 0$. By induction, we can find some $q_1 (x), r_1(x) \in F[x]$ such that
  \begin{equation*}
    g_1(x) = q_1(x) f(x) + r_1(x)
  \end{equation*}
  with $\deg r_1 < \deg f$. It follows that
  \begin{align*}
    g(x) &= g_1(x) + b_n a_m^{-1} x^{n - m} f(x) \\
         &= q_1(x) f(x) + r_1(x) + b_n a_m^{-1} x^{n - m} f(x).
  \end{align*}
  So pick
  \begin{gather*}
    q(x) = q_1 (x) + b_n a_m^{-1} x^{n - m} \\
    r(x) = r_1 (x) < \deg f,
  \end{gather*}
  and so the result follows.

  To prove uniqueness, suppose we have
  \begin{equation*}
    q_1(x) f(x) + r_1(x) = q_2(x) f(x) + r_2(x)
  \end{equation*}
  with $\deg r_1, \deg r_2 < \deg f$. Then
  \begin{equation*}
    r_2(x) - r_1(x) = [ q_1(x) - q_2(x) ] f(x).
  \end{equation*}
  If $q_1(x) - q_2(x) \neq 0$, then
  \begin{equation*}
    \deg (r_2 - r_1) = \deg (q_1 - q_2) + \deg f \geq \deg f
  \end{equation*}
  which is a contradiction since $\deg (r_2 - r_1) < \deg f$. Thus we must have $q_1(x) = q_2(x)$ and so $r_1 (x) = r_2(x)$.\qed
\end{proof}

\begin{propo}[Properties of the Greatest Common Divisor]\index{Greatest common divisor}
\label{propo:properties_of_the_greatest_common_divisor}
Let $F$ be a field and $f(x), \, g(x) \in F[x]$ with $f(x) \neq 0 \neq g(x)$. Then $\exists ! d(x) \in F[x]$ such that
\begin{enumerate}
  \item $d(x)$ is monic;
  \item $d(x) \, | \, f(x)$ and $d(x) \, | \, g(x)$;
  \item $e(x) \, | \, f(x) \, \land \, e(x) \, | \, g(x) \implies e(x) \, | \, d(x)$;
  \item $\exists u(x), v(x) \in F[x] \quad d(x) = u(x) f(x) + v(x) g(x)$
\end{enumerate}
In this case, we say that $d(x)$ is the \hlnoteb{greatest common divisor} of $f(x)$ and $g(x)$, and denote this by $d(x) = \gcd[ f(x), g(x) ]$.
\end{propo}

\begin{proof}
  Consider the set
  \begin{equation*}
    X = \{ u(x) f(x) + v(x) g(x) : u(x), \, v(x) \in F[x] \}.
  \end{equation*}
  Since $f(x) = 1 \cdot f(x) + 0 \cdot g(x) \in X$, the set $X$ contains non-zero polynomial and thus contains monic polynomials (since $F$ is a field\sidenote{This is cause if we have
  \begin{equation*}
    f(x) = a_m x^m + \hdots + a_0
  \end{equation*}
  Then
  \begin{equation*}
    a_m^{-1} f(x) = x^m + \hdots + a_m^{-1} a_0
  \end{equation*}
  is a moic polynomial in $F[x]$.}). Among all of the monic polynomials, choose
  \begin{equation*}
    d(x) = u(x) f(x) + v(x) g(x)
  \end{equation*}
  to have minimal degree. Then we get $(1)$ and $(4)$ in the bag automatically so. $(3)$ also follows almost immediately, since
  \begin{align*}
    &e(x) \, | \, f(x) \, \land \, e(x) \, | \, g(x) \\
    &\implies \exists a(x), b(x) \in F[x] \quad f(x) = a(x) e(x) \, \land \, g(x) = b(x) e(x) \\
    &\implies d(x) = u(x) f(x) + v(x) = [ u(x) a(x) + v(x) b(x) ] e(x) \\
    &\implies e(x) \, | \, d(x).
  \end{align*}
  It remains to prove $(2)$. By \cref{propo:division_algorithm_for_polynomials}, we have that $\exists q(x), r(x) \in F[x]$ with $\deg r < \deg f$ such that
  \begin{equation*}
    f(x) = q(x) d(x) + r(x).
  \end{equation*}
  Then\marginnote{
  \begin{ex}
    Reiterate this proof for integers, by removing the `$(x)$' and replacing instances of monic polynomials with positive integers.
  \end{ex}
  }
  \begin{align*}
    r(x) &= f(x) - q(x) d(x) = f(x) - q(x) [ u(x) f(x) + v(x) g(x) ] \\
         &= [ 1 - q(x) u(x) ] f(x) - q(x) v(x) g(x).
  \end{align*}
  Note that if $r(x) \neq 0$, then write $k \neq 0 \in F$ as the leading coefficient of $r(x)$. Since $F$ is a field, we have that $\exists k^{-1} \in F$, and so $k^{-1} r(x)$ is a monic polynomial of $X$ with $\deg( k^{-1} r ) < \deg d$, which contradicts the fact that the degree of $d(x)$ is minimal. Thus $r(x) = 0$ and $d(x) \, | \, f(x)$. Using a similar argument, we can show that $d(x) \, | \, g(x)$. Therefore, $(2)$ follows.\qed
\end{proof}

% subsection factorization_of_polynomials_continued (end)

% section polynomial_ring_continued (end)

% chapter lecture_28_jul_09th_2018 (end)

\chapter{Lecture 29 Jul 11th 2018}%
\label{chp:lecture_29_jul_11th_2018}
% chapter lecture_29_jul_11th_2018

\section{Polynomial Ring (Continued 2)}%
\label{sec:polynomial_ring_continued_2}
% section polynomial_ring_continued

\subsection{Factorization of Polynomials (Continued 2)}%
\label{sub:factorization_of_polynomials_continued_2}
% subsection factorization_of_polynomials_continued_2

\begin{note}
  If $d(x)$ and $d_1(x)$ satisfies \cref{propo:properties_of_the_greatest_common_divisor}, then in particular $(3)$ is satisfied, i.e.
  \begin{equation*}
    d(x) \, | \, d_1(x) \text{ and } d_1(x) \, | \, d(x),
  \end{equation*}
  then since $d_1(x) = d(x)$ by \cref{propo:f_div_g_and_g_div_f_implies_f_is_g}. Thus $d(x)$ is unique and is therefore called the greatest common divisor of $f(x)$ and $g(x)$, denoted by $\gcd\Big( f(x), \, g(x) \Big) = d(x)$.
\end{note}

\newthought{Note that} in integers, $p \in \mathbb{Z}$ is prime if $p \geq 2$ and whenever $p = ab$, then $a = \pm 1$ or $b = \pm 1$, where $a, \, b \in \mathbb{Z}$. We can have an ``analogous'' notion with polynomials.

\begin{defn}[Irreducible Polynomials]\index{Irreducible Polynomials}
\label{defn:irreducible_polynomials}
  Let $F$ be a field. A non-zero polynomial $l(x) \in F[x]$ is \hlnoteb{irreducible} if $\deg l \geq 1$ and if
  \begin{equation*}
    l(x) = l_1 (x) l_2 (x)
  \end{equation*}
  for $l_1(x), \, l_2 (x) \in F[x]$, then $\deg l_1 = 0$ or $\deg l_2 = 0$ \sidenote{Note that polynomials of degree $0$ are the units of $F[x]$.}.

  Polynomials that are not irreducible are called \hldefn{reducible polynomials}.
\end{defn}

\begin{propo}[Euclid's Lemma for Polynomials]
\label{propo:euclid_s_lemma_for_polynomials}
Let $F$ be a field and $f(x), g(x) \in F[x]$. If $l(x) \in F[x]$ is irreducible and $l(x) \, | \, a(x) b(x)$, then $l(x) \, | \, a(x)$ or $l(x) \, | \, b(x)$.
\end{propo}
\marginnote{This is a good proof for an exercise.
\begin{ex}
  Prove \cref{propo:euclid_s_lemma_for_polynomials}.
\end{ex}}

\begin{proof}
  Suppose $l(x) \, | \, f(x) g(x)$ and $l(x) \not| \, f(x)$. Since $l(x) \not| f(x)$, we have $\gcd[ l(x), f(x) ] = 1$. Then by \cref{propo:properties_of_the_greatest_common_divisor}, $\exists s(x), t(x) \in F[x]$ such that
  \begin{equation*}
    l(x) s(x) + f(x) t(x) = 1.
  \end{equation*}
  Multiplying the equation by $g(x)$, and since $F[x]$ is a field, we have
  \begin{equation*}
    l(x) s(x) g(x) + f(x) g(x) t(x) = g(x).
  \end{equation*}
  Since $l(x) \, | \, f(x) g(x)$ by assumption, we have that $l(x)$ divides the right hand side, and so it must also divide the left hand side, i.e. $l(x) \, | \, g(x)$.\qed
\end{proof}

\begin{thm}[Unique Factorization Theorem for Polynomials]
\index{Unique Factorization Theorem for Polynomials}
\label{thm:unique_factorization_theorem_for_polynomials}
Let $F$ be a field and $f(x) \in F[x]$ with $\deg f \geq 1$. Then we can write
\begin{equation*}
  f(x) = c l_1(x) l_2(x) \hdots l_m(x)
\end{equation*}
where $c \in F^*$ is a unit, and for $1 \leq i \leq m$, $l_i(x)$ is a irreducible monic polynomial. This factorization is unique up to the order of $l_i$.
\end{thm}
\marginnote{This is, yet again, a good proof for an exercise.
\begin{ex}
  Proof \cref{thm:unique_factorization_theorem_for_polynomials}.
\end{ex}}

\begin{proof}
  We shall only prove for when $f(x)$ is a monic polynomial, for if $f(x)$ is not monic, then it has some leading coefficient $a \neq 1 \in F$. Then since $F$ is a field, we have that $a^{-1} f(x)$ is a monic polynomial for which we can continue our consideration.

  Suppose $f(x)$ is a monic polynomial that has the least degree such that it cannot be expressed as a product of irreducible monic polynomials. Clearly, $f(x)$ cannot be irreducible itself, or it would trivially be expressible as a product of irreducible monic polynomials. Therefore, $\exists s(x), t(x) \in F[x]$ such that
  \begin{equation*}
    f(x) = s(x) t(x)
  \end{equation*}
  where $1 \leq \deg s, \, \deg t \leq \deg f$. Since $f(x)$ is the polynomial of the least degree that cannot be expressed as a product of irreducible monic polnomials, $r(x)$ and $t(x)$ must be expressible as a product of irreducible monic polynomials. But this would contradict the fact that $f(x)$ is not expressible as a product of irreducible monic polynomials, and so $f(x)$ must be
  \begin{equation*}
    f(x) = l_1(x) l_2(x) \hdots l_m(x)
  \end{equation*}
  where $l_i(x)$ is an irreducible monic polynomial, for $1 \leq i \leq m$. For the case where $f(x)$ is not monic, say with $a$ as its leading coefficient, we would have
  \begin{equation*}
    f(x) = a l_1(x) l_2(x) \hdots l_m(x).
  \end{equation*}

  For uniqueness, suppose
  \begin{equation*}
    f(x) = c l_1(x) l_2(x) \hdots l_m(x) = d k_1(x) k_2(x) \hdots k_n(x)
  \end{equation*}
  for units $c, d \in F^*$ and irreducible monic polynomials $l_i, \, k_j$ for $1 \leq i \leq m$ and $1 \leq j \leq n$. Since $l_1(x) \, | \, f(x)$, by \cref{propo:euclid_s_lemma_for_polynomials}, $l_1(x) \, | \, k_j(x)$ for some $1 \leq j \leq n$. Relabelling the indices for the $k_j$'s if necessary, we can have that $l_1(x) \, | \, k_1(x)$. Since $k_1(x)$ is irreducible and monic, we must have that $l_1(x) = k_1(x)$.

  Now if we continue this line of argument for $i = 2, 3, ..., m$, and end up with $l_2(x) = k_2(x), \, l_3(x) = k_3(x), \, \hdots , \, l_m(x) = k_m(x)$, where, WLOG, we suppose that $m \leq n$. However, we must have that $n = m$, otherwise we would have some $k_j$, where $m < j \leq n$ that cannot divide any of the $l_i$'s.\qed
\end{proof}

\newthought{For the sake of comparison with} $\mathbb{Z}$, observe the table below:

{\renewcommand{\arraystretch}{1.5}
\begin{tabular}{l | c | c}
                & $\mathbb{Z}$                                                                & $F[x]$ \\
  \hline
  elements      & $m$                                                                         & $f(x)$ \\
  \hline
  size          & $\abs{m}$                                                                   & $\deg f$ \\
  \hline
  units         & $\{ \pm 1 \}$                                                               & $F^*$ \\
                & $\Big( \mathbb{Z} \setminus \{0\} \Big) \Big/ \{ \pm 1 \} \cong \mathbb{N}$ & $\Big( F[x] \setminus \{0\} \Big) \Big/ F^* \cong \{ h : h \text{ is monic } \}$ \\
  \hline
  unique        & $m = \pm 1 p_1^{\alpha_1} \hdots p_n^{\alpha_n}$                            & $f(x) = c l_1(x)^{\alpha_1} \hdots l_n(x)^{\alpha_n} $ \\
  factorization & $p_i$ prime                                                                 & $\deg f \geq 1$ and $l_i$ are irreducible \\
  \hline
  ideals        & $\lra{n} : n \in \mathbb{N}$                                                & $\lra{h(x)} : h$ monic \\
                & $\faktor{\mathbb{Z}}{\lra{n}}$ is a field                                   & $\faktor{F[x]}{\lra{h(x)}}$ is a field \\
                & iff $n$ prime                                                               & iff $h(x)$ is irreducible
\end{tabular}}

In the next section, we will be investigating if the analogy given in the last row for polynomials holds.

% subsection factorization_of_polynomials_continued_2 (end)

\subsection{Quotient Rings of Polynomials}%
\label{sub:quotient_rings_of_polynomials}
% subsection quotient_rings_of_polynomials

\begin{propo}[Ideals of {$F[x]$} are Principal Ideals]
\label{propo:ideals_of_f_x_are_principal_ideals}\index{Principal Ideal}
If $F$ is a field. Then all ideas of $F[x]$ are of the form
\begin{equation*}
  \lra{h(x)} = h(x) F[x] \enspace \text{ for any } h(x) \in F[x].
\end{equation*}
If $\lra{h(x)} \neq \{0\}$ and $h(x)$ is monic, then it is uniquely determined.
\end{propo}

\begin{proof}
  Let $A$ be an ideal of $F[x]$. If $A = \{0\}$, then $A = \lra{0}$. If $A \neq \{0\}$, then it contains a non-zero polynomial. Since $A$ is an ideal, it has a monic polynomial\sidenote{If $f(x) \in A$ has a leading coefficient $a$, then we know that $a^{-1} \in F$, and so $a^{-1} f(x) \in F f(x) \subseteq A$ is monic.}. Amongst all monic polynomials in $A$, choose $h(x) \in A$ that has the minimal degree. Clearly, $\lra{h(x)} \subseteq A$. To prove for $\supseteq$, note that for $f(x) \in A$, by \cref{propo:division_algorithm_for_polynomials},
  \begin{equation*}
    \exists q(x), r(x) \in F[x] \quad f(x) = q(x) h(x) + r(x) \quad \deg r < \deg h.
  \end{equation*}
  If $r(x) \neq 0$, then let $u \neq 0$ be the leading coefficient of $r(x)$. Then since $A$ is an ideal and $f(x), h(x) \in A$, we have
  \begin{align*}
    u^{-1} r(x) &= u^{-1}\left( f(x) - q(x) h(x) \right) \\
                &= u^{-1} f(x) - u^{-1} q(x) h(x) \in A.
  \end{align*}
  Then we have that $\deg u^{-1} r = \deg r < \deg h$ is a monic polynomial in $A$, contradicting the minimality of $\deg h$. Thus $r(x) = 0$ and so $f(x) = q(x) h(x) \in \lra{h(x)}$. Therefore $A \susbeteq \lra{h(x)}$ and so $A = \lra{h(x)}$.

  Now suppose that $A = \lra{h(x)} = \lra{k(x)}$. Then we must have $h(x) \, | \, k(x)$ and $k(x) \, | \, h(x)$. Since $h(x)$ and $k(x)$ are both monic, by \cref{propo:f_div_g_and_g_div_f_implies_f_is_g}, we have that $h(x) = k(x)$.\qed
\end{proof}

% subsection quotient_rings_of_polynomials (end)

% section polynomial_ring_continued_2 (end)

% chapter lecture_29_jul_11th_2018 (end)

\chapter{Lecture 30 Jun 13th 2018}%
\label{chp:lecture_30_jun_13th_2018}
% chapter lecture_30_jun_13th_2018

\section{Polynomial Ring (Continued 3)}%
\label{sec:polynomial_ring_continued_3}
% section polynomial_ring_continued_3

\subsection{Quotient Rings of Polynomials (Continued)}%
\label{sub:quotient_rings_of_polynomials_continued}
% subsection quotient_rings_of_polynomials_continued

Let $A$ be a non-zero ideal in $F[x]$. By \cref{propo:ideals_of_f_x_are_principal_ideals}, we know that $A$ is a principal ideal\index{Principal Ideal} and can be written as $A = \lra{h(x)}$, for a unique polynomial $h(x) \in F[x]$.

Suppose that $\deg h = m \geq 1$. Consider the quotient ring $R = \faktor{F[x]}{A}$, and so we have
\begin{equation*}
  R = \left\{ \bar{f(x)} : f(x) + A, f(x) \in F[x] \right\}.
\end{equation*}
Write $t = \bar{x} = x + A$. Then by the \hlnoteb{Division Algorithm}\sidenote{This entire part until Proposition 89 might need to be rewritten since I am a little lost as to some of the details regarding the discussion.}, we have
\begin{equation*}
  R = \{ \bar{a_0} + \bar{a_1} t + \hdots + \bar{a_{m - 1}} t^{m - 1} : a_i \in F \}.
\end{equation*}
The map $\theta : F \to R$, given by $a \mapsto \bar{a}$, is an injective homomorphism, since $\theta$ is not a zero map and $\ker \theta$ is an ideal of $F$ \sideote{Note that a field $F$ has only 2 ideals: $\{0\}$ and $F$ itself. Since $\ker \theta \neq F$, we have that $\ker \theta = \{0\}$ and so $\theta$ is injective.}. Since we have $F \cong \theta(F)$ by the \hyperref[thm:first_isomorphism_theorem_for_rings]{First Isomorphism Theorem for Rings}, by identifying $F$ with $\theta(F)$, we can write
\begin{equation*}
  R = \{ a_0 + a_1 t + \hdots a_{m - 1} t^{m - 1} : a_i \in F \}.
\end{equation*}

It is clear that, in $R$, we have
\begin{gather*}
  a_0 + a_1 t + \hdots + a_{m - 1} t^{m - 1} = b_0 + b_1 t + \hdots + b_{m - 1} t^{m - 1} \\
  \iff \\
  \forall i \in \mathbb{Z} \enspace 0 \leq i \leq m - 1 \quad a_i = b_i
\end{gather*}

Finally, in the ring $R$, we have $h(t) = 0$.

The following proposition follows from the above discussion.

\begin{propo}
\label{propo:remainder_ring}
Let $F$ be a field and let $h(x), f(x) \in F[x]$ be monic with $( \deg h, \, \deg f \geq 1 )$. Then the quotient ring $R = F[x] / A$ is given by
\begin{equation*}
  R = \{ a_0 + a_1 t + \hdots + a_{m - 1} t^{m - 1} : a_i \in F, \, h(t) = 0 \}
\end{equation*}
in which each element of $R$ can be uniquely represented in the above form.
\end{propo}

\begin{note}
  In $\mathbb{Z}$, we have that $\mathbb{Z} / \lra{n} = \mathbb{Z}_n = \{ [0], [1], ..., [n-1] \}$ which is analogous to our statement in \cref{propo:remainder_ring} for the case of integers.
\end{note}

\begin{eg}
  Consider $\mathbb{R}[x]$ and let $h(x) = x^2 + 1 \in \mathbb{R}[x]$. Then
  \begin{equation*}
    \mathbb{R}[x] = \{a + bt : a, b \in \mathbb{R}, \, t^2 + 1 = 0 \} \cong \{ a + bi : a, b \in \mathbb{R}, \, i^2 = -1 \} = \mathbb{C}
  \end{equation*}
\end{eg}

\begin{note}
  Recall that $\mathbb{Z}_n$ is a field (or an integral domain) if and only if $n$ is prime.
\end{note}

\begin{propo}
\label{propo:principal_ideals_of_polyms_as_fields}
  Let $F$ be a field nad $h(x) \in F[x]$ be a monic polynomial with $\deg h \geq 1$. TFAE:
  \begin{enumerate}
    \item $F[x] \big/ \lra{h(x)}$ is a field;
    \item $F[x] \big/ \lra{h(x)}$ is an integral domain;
    \item $h(x)$ is irreducible in $F[x]$.
  \end{enumerate}
\end{propo}

\begin{proof}
  $(1) \implies (2)$ since a field is an integral domain (see \cref{propo:fields_are_integral_domains}).

  \noindent $(2) \implies (3)$: Write $A = \lra{h(x)}$, If $h(x) = f(x) g(x)$ for $f(x), \, g(x) \in F[x]$, then
  \begin{align*}
    [ f(x) + A ] [ g(x) + A ] &= f(x) g(x) + A \quad \because A \text{ is an ideal } \\
                              &= h(x) + A = 0 \in F[x] \big/ A.
  \end{align*}
  Then by $(2)$, either $f(x) + A = 0$ or $g(x) + A = 0$, i.e. either $f(x) \in A$ or $g(x) \in A$. But if $f(x) \in A = \lra{h(x)}$, then $f(x) = q(x) h(x)$ for some $q(x) \in F[x]$. Then $h(x) = f(x) g(x) = q(x) h(x) g(x)$, which then implies that $0 = h(x) [ 1 - q(x) g(x) ] \implies q(x) g(x) = 1$ since $F[x]$ is an integral domain. Then we have that $\deg g = 0$. Similarly, if $g(x) \in A$, then we have $\deg f = 0$. Therefore, $h(x)$ is irreducible in $F[x]$ by definition.

  \noindent $(3) \implies (1)$: Note that $F[x] \big/ \lra{h(x)}$ is a commutative ring. To show that it is a field, it suffices to show that every non-zero element of $F[x] \big/ \lra{h(x)}$ has an inverse. Let $f(x) + A \neq 0 \in F[x] \big/ \lra{h(x)}$ with $f(x) \in F[x]$. Then $f(x) \notin A$, and so $h(x) \not| \, f(x)$. Since $h(x)$ is irreducible by $(3)$, we have that
  \begin{equation*}
    d(x) = \gcd[ f(x), h(x) ] = 1.
  \end{equation*}
  Then by \cref{propo:properties_of_the_greatest_common_divisor}, $\exists u(x), v(x) \in F[x]$ such that
  \begin{equation*}
    1 = u(x) h(x) + v(x) f(x).
  \end{equation*}
  Since $h(x) u(x) \in A$, we have that
  \begin{equation*}
    [v(x) + A] [f(x) + A] = 1 + A.
  \end{equation*}
  It follows that $f(x) + A$ has an inverse in $F[x] \big/ \lra{h(x)}$ and thus $F[x] \big/ \lra{h(x)}$ is a field. \qed
\end{proof}

% subsection quotient_rings_of_polynomials_continued (end)

% section polynomial_ring_continued_3 (end)

\section{Factorizations in Integral Domains}%
\label{sec:factorizations_in_integral_domains}
% section factorizations_in_integral_domains

\subsection{Irreducibles and Primes}%
\label{sub:irreducibles_and_primes}
% subsection irreducibles_and_primes

We have discussed much about the similarities between $\mathbb{Z}$ and $F[x]$, and in this chapter, we wish to abstract these similarties and study them in a more general manner to see if other sets that share the same kind of properties. For example, if a set has a \hlnoteb{unique factorization} for elements and the \hlnoteb{principal ideal} being the only ideal of the set, then do we still see the same analogy playing out?

\begin{defn}[Division]\index{Division}
\label{defn:division}
  Let $R$ be an integral domain and $a, \, b \in R$. We say that $a \, | \, b$ if $b = ca$ for some $c \in R$.
\end{defn}

\begin{note}
  Recall that in $\mathbb{Z}$, if $n \, | \, m$ and $m \, | \, n$, then $n = \pm m$, and the ideal generated by them are the same, i.e. $\lra{n} = \lra{m}$.

  Similarly so in $F[x]$< if $f(x) \, | \, g(x)$ and $g(x) \, | \, f(x)$, then $f(x) = cg(x)$ for some $x \in F[x]^* = F^*$, and $\lra{f(x)} = \lra{g(x)}$.
\end{note}

\begin{propo}[Division in an Integral Domain]
\label{propo:division_in_an_integral_domain}
Let $R$ be an integral domain. Then $\forall a, b \in R$, TFAE:\marginnote{This should be an easy exercise.
\begin{ex}
  Prove \cref{propo:division_in_an_integral_domain}.
\end{ex}
}
  \begin{enumerate}
    \item $a \, | \, b$ and $b \, | \, a$;
    \item $a = ub$ for some unit $u \in R$;
    \item $\lra{a} = \lra{b}$.
  \end{enumerate}
\end{propo}

\begin{defn}[Association]\index{Association}
\label{defn:association}
  Let $R$ be an integral domain. $\forall a, b \in R$, we say that $a$ is \hldefn{associated to} $b$, denoted by $a \sim b$, if $a \, | \, b$ and $b \, | \, a$.
\end{defn}

\begin{note}
  By \cref{propo:division_in_an_integral_domain}, we have that $a \sim a$ for any $a \in R$.

  \noindent Also, $a \sim b \iff b \sim a$.

  \noindent We also have $a \sim b \, \land \, b \sim c \implies a \sim c$.
  
  In other words, $\sim$ is an equivalence relation\index{Equivalence Relation} in $R$. Also, it can be shown that\sidenote{More exercise is always good.
  \begin{ex}
    Prove that the two statements following this is true.
  \end{ex}
  }
  \begin{enumerate}
    \item $a \sim a' \, \land \, b \sim b' \implies ab \sim a' b'$.
    \item $a \sim a' \, \land \, b \sim b' \implies ( a \, | \, b \iff b \, | \, a )$
  \end{enumerate}
\end{note}

\begin{eg}
  Let $R = \mathbb{Z}[\sqrt{3}] = \{ m + n \sqrt{3} : m, n \in \mathbb{Z} \}$. Note that this is an integral domain\sidenote{For $(a + b\sqrt{3}), \, (c + d\sqrt{3}) \in R$ such that
  \begin{equation*}
    (a + b \sqrt{3})(c + d \sqrt{3}) = 0
  \end{equation*}
  we would have that
  \begin{gather*}
    (a + b\sqrt{3})(a - b \sqrt{3})(c + d\sqrt{3})(c - d\sqrt{3}) = 0 \\
    ( a^2 - 3b^2 )( c^2 - 3d^2 ) = 0.
  \end{gather*}
  Since $\mathbb{Z}$ is an integral domain, suppose $a^2 - 3b^2 = 0$. If $b = 0$, then $a = 0$ and we are done. If $b \neq 0$, then we have $3 = \left( \frac{a}{b} \right)^2$, and we notice that $\sqrt{3}$ is irrational. Thus it can only be that $b = 0$. Therefore, $a + b\sqrt{3} = 0$, implying that there are no zero divisors in $R = \mathbb{Z}[\sqrt{3}]$.
  }. Observe that
  \begin{equation*}
    (2 + \sqrt{3})(2 - \sqrt{3}) = 1 \implies 2 + \sqrt{3} \text{ is a unit in } R.
  \end{equation*}
  Then we would have
  \begin{equation*}
    3 + 2 \sqrt{3} = (2 + \sqrt{3}) \sqrt{3}
  \end{equation*}
  and so by \cref{propo:division_in_an_integral_domain}, we have
  \begin{equation*}
    3 + 2 \sqrt{3} \sim \sqrt{3} \in \mathbb{Z}[\sqrt{3}].
  \end{equation*}
\end{eg}

% subsection irreducibles_and_primes (end)

% section factorizations_in_integral_domains (end)

% chapter lecture_30_jun_13th_2018 (end)


\nobibliography*
\bibliography{bibliography}

\pagestyle{empty}
\printindex

\chapter{List of Symbols}
% chapter List of Symbols

\begin{tabular}{l l}
  $M_n(\mathbb{R})$  & set of $n \times n$ matrices over $\mathbb{R}$ \\
  $\mathbb{Z}_n^*$   & set of integers modulo $n$; each element has its multiplicative inverse \\
  $S_n$              & symmetry group of degree $n$ \\
  $D_{2n}$           & dihedral group of degree $n$; a subset of $S_n$ \\
  $K_n$              & Klein $n$-group \\
  $A_n$              & alternating group of degree $n$; a subset of $S_n$ \\
  $\abs{D_{2n}}$     & order of the dihedral group; the size of the dihedral group \\
  $\begin{pmatrix} 1 & 2 & \hdots & n \end{pmatrix}$ & An $n$-cycle \\
  $\det A$           & determinant of matrix $A$ \\
  $GL_n(\mathbb{R})$ & \tworow{l}{general linear group of degree $n$;}{the set that contains elements of $M_n(\mathbb{R})$ with non-zero determinant} \\
  $SL_n(\mathbb{R})$ & \tworow{l}{special linear group of order $n$;}{the set that contains elements of $GL_n(\mathbb{R})$ with determinant of $1$} \\
  $Z(G)$             & center of group $G$ \\
  $\lra{g}$          & cyclic group with generator $g$ \\
  $n \; | \; d$      & $n$ divides $d$ \\
  $H \leq G$         & $H$ is a subgroup of $G$ (used sparsely in this notebook) \\
  $H \triangleleft G$& $H$ is a normal subgroup of $G$ \\
  $\faktor{G}{H}$    & quotient group of $G$ by $H \triangleleft G$ \\
  $\ker \alpha$      & kernel of $\alpha$ \\
  $\img \alpha$      & image of $\alpha$ \\
  $G^{(m)}$          & group of elements of $G$ with order $m$
\end{tabular}

% chapter List of Symbols (end)


\end{document}
