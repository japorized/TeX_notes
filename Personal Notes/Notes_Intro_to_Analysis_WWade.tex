% Document Head
\documentclass[11pt, oneside]{book}
\usepackage{geometry}
\geometry{letterpaper}
\usepackage[parfill]{parskip}
\usepackage{graphicx}

% Essential Packages
\usepackage{ragged2e}
\usepackage{amssymb}
\usepackage{amsmath}
\usepackage{mathrsfs}
\usepackage[utf8]{inputenc}
\usepackage[english]{babel}
\usepackage[hyperref]{ntheorem}

% Theorem Style Customization
\setlength\theorempreskipamount{2ex}
\setlength\theorempostskipamount{3ex}

% hyperref Package Settings
\usepackage{hyperref}
\hypersetup{
	colorlinks = true,
	linkcolor = magenta
}

% ntheorem Declarations
\theoremstyle{break}
\newtheorem{thm}{Theorem}[section]
\newtheorem*{proof}{Proof}
\newtheorem{crly}{Corollary}[thm]
\newtheorem{lemma}[thm]{Lemma}
\newtheorem{propo}{Proposition}[section]
\newtheorem*{remark}{Remark}
\newtheorem*{note}{Note}
\newtheorem{defn}{Definition}[section]
\newtheorem{eg}{Example}[section]

% ntheorem listtheorem style
\makeatletter
\def\thm@@thmline@name#1#2#3#4{%
        \@dottedtocline{-2}{0em}{2.3em}%
                   {\makebox[\widesttheorem][l]{#1 \protect\numberline{#2}}#3}%
                   {#4}}
\@ifpackageloaded{hyperref}{
\def\thm@@thmline@name#1#2#3#4#5{%
    \ifx\\#5\\%
        \@dottedtocline{-2}{0em}{2.3em}%
            {\makebox[\widesttheorem][l]{#1 \protect\numberline{#2}}#3}%
            {#4}
    \else
        \ifHy@linktocpage\relax\relax
            \@dottedtocline{-2}{0em}{2.3em}%
                {\makebox[\widesttheorem][l]{#1 \protect\numberline{#2}}#3}%
                {\hyper@linkstart{link}{#5}{#4}\hyper@linkend}%
        \else
            \@dottedtocline{-2}{0em}{2.3em}%
                {\hyper@linkstart{link}{#5}%
                  {\makebox[\widesttheorem][l]{#1 \protect\numberline{#2}}#3}\hyper@linkend}%
                    {#4}%
        \fi
    \fi}
}
\makeatother
\newlength\widesttheorem
\AtBeginDocument{
  \settowidth{\widesttheorem}{Proposition A.1.1.1\quad}
}

\theoremlisttype{allname}

% Shortcuts
\newcommand{\bb}[1]{\mathbb{#1}}			% using bb instead of mathbb
\newcommand{\floor}[1]{\lfloor #1 \rfloor}	% simplifying the writing of a floor function
\newcommand{\ceiling}[1]{\lceil #1 \rceil}	% simplifying the writing of a ceiling function
\newcommand{\dotp}{\, \cdotp}				% dot product to distinguish from \cdot

% Main Body
\title{Personal Notes for An Introduction to Analysis William R. Wade}
\author{Johnson Ng}

\begin{document}
\maketitle

\tableofcontents

\chapter{Differentiability on \texorpdfstring{$\bb{R}$}{R}}

\section{The Derivative}

\begin{defn}[Differentiable]
    A real function f is said to be differentiable at a point $a \in \bb{R}$ if and only if f is defined on some open interval I containing a and
    \begin{equation}\label{eq:1.1}
        f'(a) := \lim_{h \to \infty} \frac{f(a+h) - f(a)}{h}
    \end{equation}
    exists. In this case, $f'(a)$ is called the derivative of f at a.
\end{defn}

There are two characterizations of diffrentiability which we shall use to study derivatives. The first one which characterizes the derivatives in terms of the "chord function"

\begin{equation}\label{eq:1.2}
    F(x) := \frac{f(x) - f(a)}{x-a} \quad x \neq a,
\end{equation}

will be used to establish the Chain Rule.

\begin{thm}
    A real function f is differentiable at some point $a \in \bb{R} \iff \exists$ an open interval I and a function $F: I \to \bb{R}$ such that $a \in I$, f is defined on I, F is continuous on a, and
    \begin{equation}\label{eq:1.3}
        f(x) = F(x)(x-a) + f(a)
    \end{equation}
    holds for all $x \in I$, in which case $F(a) = f'(a)$.
\end{thm}

\begin{proof}
    Notice that $\forall x \in I \setminus \{a\}$, \ref{eq:1.2} and \ref{eq:1.3} are equivalent. SPS f is differentiable at a. Then by definition, f is defined on some open interval I that contains a, and the limit in \ref{eq:1.1} exists. Define F on I by \ref{eq:1.2} if $x \neq a$ and $F(a) := f'(a)$. Then \ref{eq:1.3} holds $\forall x \in I$ and F is continuous on a by \ref{eq:1.2} since $f'(a)$ exists.

    Conversely, SPS \ref{eq:1.3} holds. Then \ref{eq:1.2} holds $\forall x \in I \setminus \{a\}$. As $x \to a$, since F is continuous on a, we have that $F(a) = f'(a)$. Thus by definition, f is differentiable on a.
\end{proof}

\end{document}