% Document Head
\documentclass[11pt, oneside]{book}
\usepackage{geometry}
\geometry{letterpaper}
\usepackage[parfill]{parskip}
\usepackage{graphicx}

% Essential Packages
\usepackage{ragged2e}
\usepackage{amssymb}
\usepackage{amsmath}
\usepackage{mathrsfs}
\usepackage[utf8]{inputenc}
\usepackage[english]{babel}
\usepackage[hyperref]{ntheorem}

% Theorem Style Customization
\setlength\theorempreskipamount{2ex}
\setlength\theorempostskipamount{3ex}

% hyperref Package Settings
\usepackage{hyperref}
\hypersetup{
	colorlinks = true,
	linkcolor = magenta
}

% ntheorem Declarations
\theoremstyle{break}
\newtheorem{thm}{Theorem}[section]
\newtheorem*{proof}{Proof}
\newtheorem{crly}{Corollary}[section]
\newtheorem{lemma}{Lemma}[section]
\newtheorem{propo}{Proposition}[section]
\newtheorem{claim}{Claim}[section]
\newtheorem*{remark}{Remark}
\newtheorem*{note}{Note}
\newtheorem{defn}{Definition}[section]
\newtheorem{eg}{Example}[section]
\newtheorem{ex}{Exercise}[chapter]

\providecommand{\defnautorefname}{Definition}
\providecommand{\thmautorefname}{Theorem}

% ntheorem listtheorem style
\makeatother
\newlength\widesttheorem
\AtBeginDocument{
  \settowidth{\widesttheorem}{Proposition A.1.1.1\quad}
}

\makeatletter
\def\thm@@thmline@name#1#2#3#4{%
        \@dottedtocline{-2}{0em}{2.3em}%
                   {\makebox[\widesttheorem][l]{#1 \protect\numberline{#2}}#3}%
                   {#4}}
\@ifpackageloaded{hyperref}{
\def\thm@@thmline@name#1#2#3#4#5{%
    \ifx\\#5\\%
        \@dottedtocline{-2}{0em}{2.3em}%
            {\makebox[\widesttheorem][l]{#1 \protect\numberline{#2}}#3}%
            {#4}
    \else
        \ifHy@linktocpage\relax\relax
            \@dottedtocline{-2}{0em}{2.3em}%
                {\makebox[\widesttheorem][l]{#1 \protect\numberline{#2}}#3}%
                {\hyper@linkstart{link}{#5}{#4}\hyper@linkend}%
        \else
            \@dottedtocline{-2}{0em}{2.3em}%
                {\hyper@linkstart{link}{#5}%
                  {\makebox[\widesttheorem][l]{#1 \protect\numberline{#2}}#3}\hyper@linkend}%
                    {#4}%
        \fi
    \fi}
}

\makeatletter
\def\thm@@thmline@noname#1#2#3#4{%
        \@dottedtocline{-2}{0em}{5em}%
                   {{\protect\numberline{#2}}#3}%
                   {#4}}
\@ifpackageloaded{hyperref}{
\def\thm@@thmline@noname#1#2#3#4#5{%
    \ifx\#5\%
        \@dottedtocline{-2}{0em}{5em}%
            {{\protect\numberline{#2}}#3}%
            {#4}
    \else
        \ifHy@linktocpage\relax\relax
            \@dottedtocline{-2}{0em}{5em}%
                {{\protect\numberline{#2}}#3}%
                {\hyper@linkstart{link}{#5}{#4}\hyper@linkend}%
        \else
            \@dottedtocline{-2}{0em}{5em}%
                {\hyper@linkstart{link}{#5}%
                  {{\protect\numberline{#2}}#3}\hyper@linkend}%
                    {#4}%
        \fi
    \fi}
}

% Shortcuts
\newcommand{\bb}[1]{\mathbb{#1}}			% using bb instead of mathbb
\newcommand{\floor}[1]{\lfloor #1 \rfloor}	% simplifying the writing of a floor function
\newcommand{\ceiling}[1]{\lceil #1 \rceil}	% simplifying the writing of a ceiling function
\newcommand{\dotp}{\, \cdotp}				% dot product to distinguish from \cdot

% Main Body
\title{Personal Notes for An Introduction to Analysis William R. Wade}
\author{Johnson Ng}

\begin{document}
\hypersetup{pageanchor=false}
\maketitle
\hypersetup{pageanchor=true}

\tableofcontents

\chapter*{List of Definitions}
\theoremlisttype{all}
\listtheorems{defn}

\chapter*{List of Theorems}
\theoremlisttype{allname}
\listtheorems{axiom,lemma,thm,crly,propo}

\chapter{Information from Earlier Chapters}\label{chp:important theorems and definitions}

\section{Real Number System}\label{sect:Real Number System}

\begin{defn}[Open and Closed Intervals]\label{defn:Open and Closed Intervals}
    Let $a$ and $b$ be real numbers. A \textbf{closed interval} is a set of the form
    \begin{gather*}
        [a, b] := \{ x \in \bb{R} : a \leq x \leq b \}, \quad [a, \infty) := \{ x \in \bb{R} : a \leq x \}, \\
        (-\infty, b] := \{ x \in \bb{R} : x \leq b \}, \quad \text{or } (-\infty, \infty) := \bb{R}
    \end{gather*}
    and an \textbf{open interval} is a set of the form
    \begin{gather*}
        (a, b) := \{ x \in \bb{R} : a < x < b \}, \quad (a, \infty) := \{ x \in \bb{R} : a < x \}, \\
        (-\infty, b) := \{ x \in \bb{R} : x < b \}, \quad \text{or } (-\infty, \infty) := \bb{R}
    \end{gather*}
\end{defn}

\begin{defn}[Degenerate and Non-Degenerate Intervals]\label{defn:Degenerate and Non-Degenerate}
    Given \hyperref[defn:Open and Closed Intervals]{the above definition}, an interval $I$ with endpoints $a, b$ is called \textbf{degenerate} if $a = b$ and \textbf{non-degenerate} if $a < b$.

    A degenerate open interval is the empty set, and a non-degenerate closed interval is a point $a = b$.
\end{defn}

\section{Continuity}\label{sect:Continuity}

\begin{defn}[Continuity]\label{defn:Continuity}
    Let $\emptyset \neq E \subseteq \bb{R}$ and $f: E \to R$.
    \begin{enumerate}
        \item $f$ is said to be \textbf{continuous} at a point $a \in E$ if and only if given $\epsilon > 0, \; \exists \delta > 0$ (which in general depends on $\epsilon, f$ and $a$) such that
            \begin{equation}
                |x - a| < \delta \; \land \; x \in E \implies |f(x) - f(a)| < \epsilon
            \end{equation}
        \item $f$ is said to be \textbf{continuous} on $E$ (notation: $f: E \to \bb{R}$ is continuous) if and only if $f$ is continuous at every $x \in E$.
    \end{enumerate}
\end{defn}

\chapter{Differentiability on \texorpdfstring{$\bb{R}$}{R}}

\section{The Derivative}

\begin{defn}[Differentiable]\label{defn:Differentiable}
    A real function f is said to be differentiable at a point $a \in \bb{R}$ if and only if f is defined on some open interval I containing a and
    \begin{equation}\label{eq:differentation defn}
        f'(a) := \lim_{h \to \infty} \frac{f(a+h) - f(a)}{h}
    \end{equation}
    exists. In this case, $f'(a)$ is called the derivative of f at a.
\end{defn}

There are two characterizations of diffrentiability which we shall use to study derivatives. The first one which characterizes the derivatives in terms of the "chord function"

\begin{equation}\label{eq:equiv differentiation defn}
    F(x) := \frac{f(x) - f(a)}{x-a} \quad x \neq a,
\end{equation}

will be used to establish the Chain Rule.

\begin{thm}[Differentiability and Continuity]\label{thm:Differentiability and Continuity}
    A real function $f$ is differentiable at some point $a \in \bb{R}$ if and only if there exists an open interval $I$ and a function $F : I \to \bb{R}$ such that $a \in I, f$ is defined on $I$, $F$ is continuous at $a$, and
    \begin{equation}\label{eq:2.3}
        f(x) = F(x)(x - a) + f(a)
    \end{equation}
    holds for all $x \in I$, in which case $F(a) = f'(a)$.
\end{thm}

\begin{proof}
    Note that for $x \in I \setminus \{a\}$, \autoref{eq:equiv differentiation defn} and \autoref{eq:2.3} are equivalent. Suppose $f$ is differentiable at $a \in \bb{R}$. By \autoref{defn:Differentiable}, $f$ is defined on some \textbf{open interval $I$ containing $a$} and the limit in \autoref{eq:differentation defn} exists. Define
    \begin{equation*}
        F(x) = \begin{cases}
            \frac{f(x) - f(a)}{x - a}   &   x \neq a \\
            f'(a)                       &   x = a
        \end{cases}
    \end{equation*}
    Then \autoref{eq:2.3} holds for all $x \in I$, $F$ is continuous on $a$ by \autoref{eq:equiv differentiation defn} since $f'(a)$ exists.

    Conversely, if \autoref{eq:2.3} holds, then \autoref{eq:equiv differentiation defn} holds for $x \in I \setminus \{a\}$. Taking the limit of \autoref{eq:equiv differentiation defn} as $x \to a$, and since $F$ is continuous on $a$, $F(a) = f'(a)$. Thus by \hyperref[defn:Differentiable]{Definition of Differentiability}, $f$ is continuous on $a$.
\end{proof}

\begin{thm}
    A real function $f$ is differentiable at $a$ if and only if $\exists T(x) := mx$ which is a function, such that
    \begin{equation}\label{eq:2.4}
        \lim_{h \to 0} \frac{f(a + h) - f(a) - T(h)}{h} = 0
    \end{equation}
\end{thm}

\begin{proof}
    Suppose $f$ is differentiable at $a$, and let $m := f'(a)$, then
    \begin{equation*}
        \frac{f(a + h) - f(a) - T(h)}{h} = \frac{f(a + h) - f(a)}{h} - f'(a) \to 0
    \end{equation*}
    as $h \to 0$.

    Conversely, suppose \autoref{eq:2.4} holds for $T(x) := mx$ and $h \neq 0$. Then
    \begin{align}
        \frac{f(a + h) - f(a)}{h} &= m + \frac{f(a + h) - f(a) - mh}{h} \nonumber \\
                &= m + \frac{f(a + h) - f(a) - T(h)}{h} \label{eq:2.4.1}
    \end{align}
    By \autoref{eq:2.4}, the limit of \autoref{eq:2.4.1} is $m$. Thus it follows that $\left( f(a + h) - f(a) \right) / h \to m$ as $h \to 0$; i.e. that $f'(a)$ exists and equals $m$ by \hyperref[defn:Differentiable]{Definition of Differentiability}, and thus $f$ is differentiable at $a$.
\end{proof}

With \autoref{thm:Differentiability and Continuity}, we will answer a rather interesting question: Are differentiability and continuity related? If so, how?

\begin{thm}[Differentiability $\implies$ Continuity]\label{thm:differentiability implies continuity}
    $f$ is differentiable at $a \implies f$ is continuous at $a$.
\end{thm}

\begin{proof}
    Suppose that $f$ is differentiable at $a$. By \autoref{thm:Differentiability and Continuity}, there is an open interval $I$ and a function $F$, that is continuous at $a$, such that
    \begin{equation}\label{eq:2.4.2}
        \forall x \in I \enspace f(x) = F(x)(x - a) + f(a).
    \end{equation}
    So taking the limit of \autoref{eq:2.4.2} as $x \to a$, we observe that
    \begin{equation*}
        \lim_{x \to a} f(x) = F(a) \cdot 0 + f(a) = f(a).
    \end{equation*}
    In particular, $f(x) \to f(a)$ as $x \to a$, which by \hyperref[defn:Continuity]{Definition of Continuity}, $f$ is continuous at $a$.
\end{proof}



\end{document}
% Document END