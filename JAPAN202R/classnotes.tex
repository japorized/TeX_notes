% !TEX TS-program = lualatex
\documentclass[notoc,notitlepage]{tufte-book}
% \nonstopmode % uncomment to enable nonstopmode

\usepackage{classnotetitle}

\title{JAPAN202R S19 --- Second Year Japanese 2}
\author{Johnson Ng (ジョンソン)}
\subtitle{ノート}
\credentials{BMath (Hons), Pure Mathematics major, Actuarial Science Minor}
\institution{University of Waterloo}

\usepackage{longtable,multirow}
\usepackage{titlesec}
\usepackage{luatexja}
\usepackage{luatexja-fontspec}
\usepackage{pxrubrica}
\setcounter{secnumdepth}{3}
\setcounter{tocdepth}{3}

\renewcommand{\baselinestretch}{1.1}
\usepackage{geometry}
\geometry{letterpaper}
\usepackage[parfill]{parskip}
\usepackage{graphicx}

% Essential Packages
\usepackage{makeidx}
\makeindex
\usepackage{enumitem}
\usepackage[T1]{fontenc}
\usepackage{natbib}
\bibliographystyle{apalike}
\usepackage{ragged2e}
\usepackage{etoolbox}
\usepackage{amssymb}
\usepackage{fontawesome}
\usepackage{amsmath}
\usepackage{mathrsfs}
\usepackage{mathtools}
\usepackage{xparse}
\usepackage{tkz-euclide}
\usetkzobj{all}
\usepackage[utf8]{inputenc}
\usepackage{csquotes}
\usepackage[english]{babel}
\usepackage{marvosym}
\usepackage{pgf,tikz}
\usepackage{pgfplots}
\usepackage{fancyhdr}
\usepackage{array}
\usepackage{faktor}
\usepackage{float}
\usepackage{xcolor}
\usepackage{centernot}
\usepackage{silence}
  \WarningFilter*{latex}{Marginpar on page \thepage\space moved}
\usepackage{tcolorbox}
\tcbuselibrary{skins,breakable}
\usepackage{longtable}
\usepackage[amsmath,hyperref]{ntheorem}
\usepackage{hyperref}
\usepackage[noabbrev,capitalize,nameinlink]{cleveref}

% xcolor (scheme: base16 eighties)
\definecolor{base16-eighties-dark}{HTML}{2D2D2D}
\definecolor{base16-eighties-light}{HTML}{D3D0C8}
\definecolor{base16-eighties-magenta}{HTML}{CD98CD}
\definecolor{base16-eighties-red}{HTML}{F47678}
\definecolor{base16-eighties-yellow}{HTML}{E2B552}
\definecolor{base16-eighties-green}{HTML}{98CD97}
\definecolor{base16-eighties-lightblue}{HTML}{61CCCD}
\definecolor{base16-eighties-blue}{HTML}{6498CE}
\definecolor{base16-eighties-brown}{HTML}{D47B4E}
\definecolor{base16-eighties-gray}{HTML}{747369}

% hyperref Package Settings
\hypersetup{
    bookmarks=true,         % show bookmarks bar?
    unicode=true,          % non-Latin characters in Acrobat’s bookmarks
    pdftoolbar=false,        % show Acrobat’s toolbar?
    pdfmenubar=false,        % show Acrobat’s menu?
    pdffitwindow=true,     % window fit to page when opened
    colorlinks=true,
    allcolors=base16-eighties-magenta,
}

% tikz
\usepgfplotslibrary{polar}
\usepgflibrary{shapes.geometric}
\usetikzlibrary{angles,patterns,calc,decorations.markings}
\tikzset{midarrow/.style 2 args={
        decoration={markings,
            mark= at position #2 with {\arrow{#1}} ,
        },
        postaction={decorate}
    },
    midarrow/.default={latex}{0.5}
}
\def\centerarc[#1](#2)(#3:#4:#5)% Syntax: [draw options] (center) (initial angle:final angle:radius)
    { \draw[#1] ($(#2)+({#5*cos(#3)},{#5*sin(#3)})$) arc (#3:#4:#5); }

% enumitems
\newlist{inlinelist}{enumerate*}{1}
\setlist*[inlinelist,1]{%
  label=(\roman*),
}

% Theorem Style Customization
\setlength\theorempreskipamount{2ex}
\setlength\theorempostskipamount{3ex}

\makeatletter
\let\nobreakitem\item
\let\@nobreakitem\@item
\patchcmd{\nobreakitem}{\@item}{\@nobreakitem}{}{}
\patchcmd{\nobreakitem}{\@item}{\@nobreakitem}{}{}
\patchcmd{\@nobreakitem}{\@itempenalty}{\@M}{}{}
\patchcmd{\@xthm}{\ignorespaces}{\nobreak\ignorespaces}{}{}
\patchcmd{\@ythm}{\ignorespaces}{\nobreak\ignorespaces}{}{}

\renewtheoremstyle{break}%
  {\item{\theorem@headerfont
          ##1\ ##2\theorem@separator}\hskip\labelsep\relax\nobreakitem}%
  {\item{\theorem@headerfont
          ##1\ ##2\ (##3)\theorem@separator}\hskip\labelsep\relax\nobreakitem}
\makeatother

% ntheorem + framed
\makeatletter

% ntheorem Declarations
\theorempreskip{10pt}
\theorempostskip{5pt}
\theoremstyle{break}

\newtheorem*{solution}{\faPencil $\enspace$ Solution}
\newtheorem*{remark}{Remark}
\newtheorem{eg}{Example}[section]
\newtheorem{ex}{Exercise}[section]

    % definition env
\theoremprework{\textcolor{base16-eighties-blue}{\hrule height 2pt}}
\theoremheaderfont{\color{base16-eighties-blue}\normalfont\bfseries}
\theorempostwork{\textcolor{base16-eighties-blue}{\hrule height 2pt}}
\theoremindent10pt
\newtheorem{defn}{\faBook \enspace Definition}

    % definition env no num
\theoremprework{\textcolor{base16-eighties-blue}{\hrule height 2pt}}
\theoremheaderfont{\color{base16-eighties-blue}\normalfont\bfseries}
\theorempostwork{\textcolor{base16-eighties-blue}{\hrule height 2pt}}
\theoremindent10pt
\newtheorem*{defnnonum}{\faBook \enspace Definition}

    % theorem envs
\theoremprework{\textcolor{base16-eighties-magenta}{\hrule height 2pt}}
\theoremheaderfont{\color{base16-eighties-magenta}\normalfont\bfseries}
\theorempostwork{\textcolor{base16-eighties-magenta}{\hrule height 2pt}}
\theoremindent10pt
\newtheorem{thm}{\faCoffee \enspace Theorem}

\theoremprework{\textcolor{base16-eighties-magenta}{\hrule height 2pt}}
\theorempostwork{\textcolor{base16-eighties-magenta}{\hrule height 2pt}}
\theoremindent10pt
\newtheorem{propo}[thm]{\faTint \enspace Proposition}

\theoremprework{\textcolor{base16-eighties-magenta}{\hrule height 2pt}}
\theorempostwork{\textcolor{base16-eighties-magenta}{\hrule height 2pt}}
\theoremindent10pt
\newtheorem{crly}[thm]{\faSpaceShuttle \enspace Corollary}

\theoremprework{\textcolor{base16-eighties-magenta}{\hrule height 2pt}}
\theorempostwork{\textcolor{base16-eighties-magenta}{\hrule height 2pt}}
\theoremindent10pt
\newtheorem{lemma}[thm]{\faTree \enspace Lemma}

\theoremprework{\textcolor{base16-eighties-magenta}{\hrule height 2pt}}
\theorempostwork{\textcolor{base16-eighties-magenta}{\hrule height 2pt}}
\theoremindent10pt
\newtheorem{axiom}[thm]{\faShield \enspace Axiom}

    % theorem envs without counter
\theoremprework{\textcolor{base16-eighties-magenta}{\hrule height 2pt}}
\theoremheaderfont{\color{base16-eighties-magenta}\normalfont\bfseries}
\theorempostwork{\textcolor{base16-eighties-magenta}{\hrule height 2pt}}
\theoremindent10pt
\newtheorem*{thmnonum}{\faCoffee \enspace Theorem}

\theoremprework{\textcolor{base16-eighties-magenta}{\hrule height 2pt}}
\theorempostwork{\textcolor{base16-eighties-magenta}{\hrule height 2pt}}
\theoremindent10pt
\newtheorem*{propononum}{\faTint \enspace Proposition}

\theoremprework{\textcolor{base16-eighties-magenta}{\hrule height 2pt}}
\theorempostwork{\textcolor{base16-eighties-magenta}{\hrule height 2pt}}
\theoremindent10pt
\newtheorem*{crlynonum}{\faSpaceShuttle \enspace Corollary}

\theoremprework{\textcolor{base16-eighties-magenta}{\hrule height 2pt}}
\theorempostwork{\textcolor{base16-eighties-magenta}{\hrule height 2pt}}
\theoremindent10pt
\newtheorem*{lemmanonum}{\faTree \enspace Lemma}

\theoremprework{\textcolor{base16-eighties-magenta}{\hrule height 2pt}}
\theorempostwork{\textcolor{base16-eighties-magenta}{\hrule height 2pt}}
\theoremindent10pt
\newtheorem*{axiomnonum}{\faShield \enspace Axiom}

    % proof env
\theoremprework{\textcolor{base16-eighties-brown}{\hrule height 2pt}}
\theoremheaderfont{\color{base16-eighties-brown}\normalfont\bfseries}
\theorempostwork{\textcolor{base16-eighties-brown}{\hrule height 2pt}}
\newtheorem*{proof}{\faPencil \enspace Proof}

    % note and notation env
\theoremprework{\textcolor{base16-eighties-yellow}{\hrule height 2pt}}
\theoremheaderfont{\color{base16-eighties-yellow}\normalfont\bfseries}
\theorempostwork{\textcolor{base16-eighties-yellow}{\hrule height 2pt}}
\newtheorem*{note}{\faQuoteLeft \enspace Note}

\theoremprework{\textcolor{base16-eighties-yellow}{\hrule height 2pt}}
\theorempostwork{\textcolor{base16-eighties-yellow}{\hrule height 2pt}}
\newtheorem*{notation}{\faPaw \enspace Notation}

    % warning env
\theoremprework{\textcolor{base16-eighties-red}{\hrule height 2pt}}
\theoremheaderfont{\color{base16-eighties-red}\normalfont\bfseries}
\theorempostwork{\textcolor{base16-eighties-red}{\hrule height 2pt}}
\theoremindent10pt
\newtheorem*{warning}{\faBug \enspace Warning}

% more environments
\newtcolorbox{redquote}{
  blanker,enhanced,breakable,standard jigsaw,
  opacityback=0,
  coltext=base16-eighties-light,
  left=5mm,right=5mm,top=2mm,bottom=2mm,
  colframe=base16-eighties-red,
  boxrule=0pt,leftrule=3pt,
  fontupper=\itshape
}
\newtcolorbox{bluequote}{
  blanker,enhanced,breakable,standard jigsaw,
  opacityback=0,
  coltext=base16-eighties-light,
  left=5mm,right=5mm,top=2mm,bottom=2mm,
  colframe=base16-eighties-blue,
  boxrule=0pt,leftrule=3pt,
  fontupper=\itshape
}
\newtcolorbox{greenquote}{
  blanker,enhanced,breakable,standard jigsaw,
  opacityback=0,
  coltext=base16-eighties-light,
  left=5mm,right=5mm,top=2mm,bottom=2mm,
  colframe=base16-eighties-green,
  boxrule=0pt,leftrule=3pt,
  fontupper=\itshape
}
\newtcolorbox{yellowquote}{
  blanker,enhanced,breakable,standard jigsaw,
  opacityback=0,
  coltext=base16-eighties-light,
  left=5mm,right=5mm,top=2mm,bottom=2mm,
  colframe=base16-eighties-yellow,
  boxrule=0pt,leftrule=3pt,
  fontupper=\itshape
}
\newtcolorbox{magentaquote}{
  blanker,enhanced,breakable,standard jigsaw,
  opacityback=0,
  coltext=base16-eighties-light,
  left=5mm,right=5mm,top=2mm,bottom=2mm,
  colframe=base16-eighties-magenta,
  boxrule=0pt,leftrule=3pt,
  fontupper=\itshape
}

% ntheorem listtheorem style
\makeatother
\newlength\widesttheorem
\AtBeginDocument{
  \settowidth{\widesttheorem}{Proposition A.1.1.1\quad}
}

\makeatletter
\def\thm@@thmline@name#1#2#3#4{%
        \@dottedtocline{-2}{0em}{2.3em}%
                   {\makebox[\widesttheorem][l]{#1 \protect\numberline{#2}}#3}%
                   {#4}}
\@ifpackageloaded{hyperref}{
\def\thm@@thmline@name#1#2#3#4#5{%
    \ifx\#5\%
        \@dottedtocline{-2}{0em}{2.3em}%
            {\makebox[\widesttheorem][l]{#1 \protect\numberline{#2}}#3}%
            {#4}
    \else
        \ifHy@linktocpage\relax\relax
            \@dottedtocline{-2}{0em}{2.3em}%
                {\makebox[\widesttheorem][l]{#1 \protect\numberline{#2}}#3}%
                {\hyper@linkstart{link}{#5}{#4}\hyper@linkend}%
        \else
            \@dottedtocline{-2}{0em}{2.3em}%
                {\hyper@linkstart{link}{#5}%
                  {\makebox[\widesttheorem][l]{#1 \protect\numberline{#2}}#3}\hyper@linkend}%
                    {#4}%
        \fi
    \fi}
}

\makeatletter
\def\thm@@thmline@noname#1#2#3#4{%
        \@dottedtocline{-2}{0em}{5em}%
                   {{\protect\numberline{#2}}#3}%
                   {#4}}
\@ifpackageloaded{hyperref}{
\def\thm@@thmline@noname#1#2#3#4#5{%
    \ifx\#5\%
        \@dottedtocline{-2}{0em}{5em}%
            {{\protect\numberline{#2}}#3}%
            {#4}
    \else
        \ifHy@linktocpage\relax\relax
            \@dottedtocline{-2}{0em}{5em}%
                {{\protect\numberline{#2}}#3}%
                {\hyper@linkstart{link}{#5}{#4}\hyper@linkend}%
        \else
            \@dottedtocline{-2}{0em}{5em}%
                {\hyper@linkstart{link}{#5}%
                  {{\protect\numberline{#2}}#3}\hyper@linkend}%
                    {#4}%
        \fi
    \fi}
}

\theoremlisttype{allname}

\AtBeginDocument{\renewcommand\contentsname{Table of Contents}}

% Heading formattings
% chapter format
\titleformat{\chapter}%
  {\huge\rmfamily\itshape\color{base16-eighties-magenta}}% format applied to label+text
  {\llap{\colorbox{base16-eighties-magenta}{\parbox{1.5cm}{\hfill\itshape\huge\textcolor{base16-eighties-dark}{\thechapter}}}}}% label
  {5pt}% horizontal separation between label and title body
  {}% before the title body
  []% after the title body

% section format
\titleformat{\section}%
  {\normalfont\Large\rmfamily\itshape\color{base16-eighties-blue}}% format applied to label+text
  {\llap{\colorbox{base16-eighties-blue}{\parbox{1.5cm}{\hfill\itshape\textcolor{base16-eighties-dark}{\thesection}}}}}% label
  {5pt}% horizontal separation between label and title body
  {}% before the title body
  []% after the title body

% subsection format
\titleformat{\subsection}%
  {\normalfont\large\itshape\color{base16-eighties-green}}% format applied to label+text
  {\llap{\colorbox{base16-eighties-green}{\parbox{1.5cm}{\hfill\textcolor{base16-eighties-dark}{\thesubsection}}}}}% label
  {1em}% horizontal separation between label and title body
  {}% before the title body
  []% after the title body

% Sidenote enhancements
\def\mathmarginnote#1{%
  \tag*{\rlap{\hspace\marginparsep\smash{\parbox[t]{\marginparwidth}{%
  \footnotesize#1}}}}
}

% Custom table columning
\newcolumntype{L}[1]{>{\raggedright\let\newline\\\arraybackslash\hspace{0pt}}m{#1}}
\newcolumntype{C}[1]{>{\centering\let\newline\\\arraybackslash\hspace{0pt}}m{#1}}
\newcolumntype{R}[1]{>{\raggedleft\let\newline\\\arraybackslash\hspace{0pt}}m{#1}}

% Custom math operator
% \DeclareMathOperator{\rem}{rem}
\DeclareMathOperator*{\argmax}{arg\,max}
\DeclareMathOperator*{\argmin}{arg\,min}
\DeclareMathOperator{\re}{Re}
\DeclareMathOperator{\im}{Im}
\DeclareMathOperator{\caparg}{Arg}
\DeclareMathOperator{\Ind}{Ind}
\DeclareMathOperator{\Res}{Res}

% Graph styles
\pgfplotsset{compat=1.15}
\usepgfplotslibrary{fillbetween}
\pgfplotsset{four quads/.append style={axis x line=middle, axis y line=
middle, xlabel={$x$}, ylabel={$y$}, axis equal }}
\pgfplotsset{four quad complex/.append style={axis x line=middle, axis y line=
middle, xlabel={$\re$}, ylabel={$\im$}, axis equal }}

% Shortcuts
\newcommand{\floor}[1]{\lfloor #1 \rfloor}      % simplifying the writing of a floor function
\newcommand{\ceiling}[1]{\lceil #1 \rceil}      % simplifying the writing of a ceiling function
\newcommand{\dotp}{\, \cdotp}			        % dot product to distinguish from \cdot
\newcommand{\qed}{\hfill\ensuremath{\square}}   % Q.E.D sign
\newcommand{\abs}[1]{\left|#1\right|}						% absolute value
\newcommand{\lra}[1]{\langle \; #1 \; \rangle}
\newcommand{\at}[2]{\Big|_{#1}^{#2}}
\newcommand{\Arg}[1]{\caparg #1}
\renewcommand{\bar}[1]{\mkern 1.5mu \overline{\mkern -1.5mu #1 \mkern -1.5mu} \mkern 1.5mu}
\newcommand{\quotient}[2]{\faktor{#1}{#2}}
\newcommand{\cyclic}[1]{\left\langle #1 \right\rangle}
	% highlighting shortcuts
\newcommand{\hlimpo}[1]{\textcolor{base16-eighties-red}{\textbf{#1}}}
\newcommand{\hlwarn}[1]{\textcolor{base16-eighties-yellow}{\textbf{#1}}}
\newcommand{\hldefn}[1]{\textcolor{base16-eighties-blue}{\index{#1}\textbf{#1}}}
\newcommand{\hlnotea}[1]{\textcolor{base16-eighties-green}{\textbf{#1}}}
\newcommand{\hlnoteb}[1]{\textcolor{base16-eighties-lightblue}{\textbf{#1}}}
\newcommand{\hlnotec}[1]{\textcolor{base16-eighties-brown}{\textbf{#1}}}
\newcommand{\WTP}{\textcolor{base16-eighties-brown}{WTP} }
\newcommand{\WTS}{\textcolor{base16-eighties-brown}{WTS} }
\newcommand{\ind}[2]{\Ind_{#2}\left( #1 \right)}
\newcommand{\notimply}{\centernot\implies}
\newcommand{\res}[2]{\underset{#2}{\Res} #1 }
\newcommand{\tworow}[3]{\begin{tabular}{@{}#1@{}} #2 \\ #3 \end{tabular}}
\renewcommand{\epsilon}{\varepsilon}
\newcommand{\lrarrow}{\leftrightarrow}
\newcommand{\larrow}{\leftarrow}
\newcommand{\rarrow}{\rightarrow}
\renewcommand{\atop}[2]{\genfrac{}{}{0pt}{}{#1}{#2}}
\newcommand*\dif{\mathop{}\!d}

  % inspiration from: https://tex.stackexchange.com/questions/8720/overbrace-underbrace-but-with-an-arrow-instead#37758
\newcommand{\overarrow}[2]{
  \overset{\makebox[0pt]{\begin{tabular}{@{}c@{}}#2\\[0pt]\ensuremath{\uparrow}\end{tabular}}}{#1}
}
\newcommand{\underarrow}[2]{
  \underset{\makebox[0pt]{\begin{tabular}{@{}c@{}}\downarrow\\[0pt]\ensuremath{#2}\end{tabular}}}{#1}
}

% Document header formatting
\renewcommand{\chaptermark}[1]{\markboth{#1}{}}
\renewcommand{\sectionmark}[1]{\markright{#1}}
\makeatletter
\pagestyle{fancy}
\fancyhead{}
\fancyhead[RO]{\textsl{\@title} \enspace \thepage}
\fancyhead[LE]{\thepage \enspace \textsl{\leftmark \enspace - \enspace \rightmark}}
\makeatother

% Comment the two lines below if you want to print the document
\pagecolor{base16-eighties-dark}
\color{base16-eighties-light}


\setmainfont[Ligatures=TeX]{P052}
\setmainjfont{M+ 1mn}

\begin{document}
% \hypersetup{pageanchor=false}
\maketitle
\hypersetup{pageanchor=true}
\begin{fullwidth}
\tableofcontents
\end{fullwidth}

\newpage
\begin{fullwidth}
  \renewcommand{\listtheoremname}{\faBook\ \slshape List of Definitions}
  \listoftheorems[ignoreall,show={defn}]
  \addcontentsline{toc}{chapter}{List of Definitions}
\end{fullwidth}

\newpage 
\begin{fullwidth}
  \renewcommand{\listtheoremname}{\faCoffee\ \slshape List of Theorems}
  \listoftheorems[ignoreall,
    show={axiom,lemma,thm,crly,propo,marginthm,marginpropo,marginlemma,marginaxiom,margincrly}
  ]
  \addcontentsline{toc}{chapter}{List of Theorems}
\end{fullwidth}

\makeatletter
\fancyhead[LE]{\thepage \enspace \textsl{\leftmark}}
\makeatother

\hypersetup{pageanchor=false}
\maketitle
\hypersetup{pageanchor=true}
\begin{fullwidth}
\chaptermark{Table of Contents}
\tableofcontents
\end{fullwidth}

\makeatletter
\fancyhead[LE]{\thepage \enspace \textsl{\leftmark \enspace \rightmark}}
\makeatother

\chapter*{Preface}%
\label{chp:preface}
\addcontentsline{toc}{chapter}{Preface}
% chapter preface

This set of notes will follow almost precisely the presentation in
\citealt{banno1999}.

The notes is also meant for readers that are already somewhat familiar
with Japanese culture and speaking patterns, and so it shall be more
focused on the linguistics, rather than bringing the reader
to a level where they can understand, read and write some Japanese.
As such, the GENKI series is probably not the ideal material,
but this is also partly just to study for the exam on campus.

% chapter preface (end)

\chapter{第18課}%
\label{chp:dai_18_ka}
% chapter dai_18_ka

\section{単語(18課)}%
\label{sec:tango_c18}
% section tango_c18

\begin{longtable}{r l l l}
\multicolumn{4}{l}{\hlnotea{名詞}} \\
* & あと           & 後         & the rest \\
  & エアコン       &            & air conditioner \\
  & カーテン       &            & curtain \\
  & クッション     &            & cushion \\
  & シャンプー     &            & shampoo \\
* & しょうゆ       & 醤油       & soy sauce \\
* & スイッチ       &            & switch \\
  & スープ         &            & soup \\
* & スカート       &            & skirt \\
* & そと           & 外         & outside \\
  & ソファ         &            & sofa \\
* & タオル         &            & towel \\
  & にっき         & 日記       & diary \\
  & バナナ         &            & banana \\
  & ポップコーン   &            & popcorn \\
  & むし           & 虫         & insect \\
  & やちん         & 家賃       & rent \\
* & ゆうがた       & 夕方       & evening \\
  & るすばんでんわ & 留守番電話 & answering machine \\
* & れいぞうこ     & 冷蔵庫     & refrigerator \\
  & ろうそく       & 蝋燭       & candle \\
\multicolumn{4}{l}{\hlnotea{イ --- 形容詞}} \\
  & あかるい         & 明るい     & bright \\
  & きぶんがわるいう & 気分が悪い & to feel sick \\
  & くらい           & 暗い       & dark \\
  & はずかしい       & 恥ずかしい & embarrassing; \\
  &                  &            & to feel embarrassed \\
\multicolumn{4}{l}{\hlnotea{ウ --- 動詞}} \\
  & あく       & 開く       & (something) opens \\
  & あやまる   & 謝る       & to apologize \\
* & おす       & 押す       & to press; to push \\
* & おとす     & 落とす     & to drop (something) \\
  & おゆがわく & お湯が沸く & to boil water \\
  & ころぶ     & 転ぶ       & to fall down \\
  & こわす     & 壊す       & to break (something) \\
  & さく       & 咲く       & to bloom \\
  & しまる     & 閉まる     & to close (something) \\
* & たすかる   & 助かる     & to be saved; to be helped \\
* & たのむ     & 頼む       & to ask a favor \\
* & つく       &            & to turn (appliance) on \\
  & よごす     & 汚す       & to make dirty \\
\multicolumn{4}{l}{\hlnotea{ル --- 動詞}} \\
  & おちる     & 落ちる   & to drop (something) \\
* & かたづける & 片付ける & to tidy up \\
  & かんがえる & 考える   & to think (about); to consider \\
  & きえる     & 消える   & (something) goes off \\
  & こわれる   & 壊れる   & (something) breaks \\
* & よごれる   & 汚れる   & to become dirty \\
\multicolumn{4}{l}{\hlnotea{特別動詞}} \\
  & ちゅうもんする & 注文する & to place an order \\
\multicolumn{4}{l}{\hlnotea{他の}} \\
* & いますぐ               & 今すぐ             & right away \\
* & おかげで               &                    & thanks to ... \\
* & おさきにしつれいします & お先に失礼します   & I shall be leaving first \\
* & おつかれさま(でした) & お疲れ様(でした) & That must be tiring for you \\
* & 〜(ん)だろう         &                    & short for 〜(ん)でしょう \\
* & どうしよう             &                    & What should I/we do? \\
* & ほんとうに             & 本当に             & really \\
* & まず                   &                    & first of all \\
  & 〜までに               &                    & by (time/date)
\end{longtable}

% section tango_c18 (end)

\section{自動詞と他動詞}%
\label{sec:jidoushi_to_tadoushi}
% section jidoushi_to_tadoushi

This is a direct equivalence of \hlnotea{transitivity} in English, where
\begin{itemize}
  \item 自動詞 --- intransitive verbs
  \item 他動詞 --- transitive verbs
\end{itemize}

\begin{defn}[自動詞]\index{自動詞}\label{defn:jidoushi}
  \hlnoteb{自動詞} (intransitive verbs) are verbs that describe the changes or
  the state that things or people undergo.
\end{defn}

\begin{defn}[他動詞]\index{他動詞}\label{defn:tadoushi}
  \hlnoteb{他動詞} (transitive verbs) are verbs that describe
  situations in which human beings act on things.
\end{defn}

Most verbs do not have a counterpart of the opposite transitivity.
\cref{table:transitive_intransitive_verb_pair} shows some examples.

\begin{table}[ht]
  \centering
  \caption{自動詞と他動詞ペア}
  \label{table:transitive_intransitive_verb_pair}
  \begin{tabular}{l | l}
    他動詞 & 自動詞 \\
    \hline
    開ける & 開く \\
    閉める & 閉まる \\
    入れる & 入る \\
    出す   & 出る \\
    つける & つく \\
    消す   & 消える \\
    壊す   & 壊れる \\
    汚す   & 汚れる \\
    落とす & 落ちる \\
    沸かす & 沸く
  \end{tabular}
\end{table}

\begin{note}
  \begin{enumerate}
    \item 他動詞の場合はよく物や人に使うから、よく「を」と使います。 \\
      例:たけしさんが電気\underline{を}つけました。
    \item 自動詞は物を形容するみたいなことなので、よく「が」と
      使っています。 \\
      例:電気\underline{が}付きました。
  \end{enumerate}
\end{note}

There is also a subtle note that should be noted about the tenses
of transitive and intransitive verbs.

\begin{figure*}[ht]
  \centering
  \begin{tikzpicture}
    \draw[->] (-2, 0) -- (2, 0) node[right] {time};
    \draw[thick,color=green]
      (-2, -1) -- (-1, -1) node[color=foreground,midway,below] {will happen} --
      (-0.5, 1) -- (0.5, 1) node[color=foreground,midway,above] {happening} --
      (1, -1) -- (2, -1) node[color=foreground,midway,below] {happened};
    \node[below=15pt,color=green] at (-1.5, -1) {壊します};
    \node[above=15pt,color=green] at (0, 1) {壊しています};
    \node[below=15pt,color=green] at (1.5, -1) {壊しました};

    \draw[->] (4, 0) -- (8, 0) node[right] {time};
    \draw[thick,color=green]
      (4, -1) -- (5.5, -1) node[color=foreground,midway,below] {old state} --
      (6.5, 1) -- (8, 1) node[color=foreground,midway,above] {new state};
    \node[circle,fill=green,inner sep=1.5pt,
      pin={[pin edge={latex'-}, pin distance=10pt]315:{state change}}]
      at (6, 0) {};
    \node[below=15pt,color=green] at (4.75, -1) {壊れます};
    \node[below=5pt,color=green] at (7.5, -0.5) {壊れました};
    \node[above=15pt,color=green] at (7.5, 1) {壊れています};
  \end{tikzpicture}
  \caption{Tenses Subtleties with Transitive and Intransitive Verbs}
  \label{fig:tenses_subtleties_with_transitive_and_intransitive_verbs}
\end{figure*}

% section jidoushi_to_tadoushi (end)

\section{〜てしまう}%
\label{sec:_teshimau}
% section _teshimau

\begin{defn}[〜てしまう]\index{〜てしまう}\label{defn:teshimau}
  \hlnoteb{〜てしまう} has 2 uses:
  \begin{enumerate}
    \item to say that one ``carries out with determination''
      a plan described by the verb;
    \item to say that there is a ``lack of premeditation
      or control over how things have turned out''.
      要するに、「〜てしまう」を使うときは、
      自分がコントロールできないことや
      起こりたくないことを起こったなど
      の無力の感情を表します。
  \end{enumerate}
\end{defn}

\begin{eg}
  To show the use of 〜しまう, we have the following examples:
  \begin{enumerate}
    \item 本を読んでしまいました。 I ended up reading the book.
    \item 電車の中にかばんを忘れてしまいました。 I left my bag in the train.
    \item 宿題を忘れたので、先生は怒ってしまいました。 I was reprimanded
      for forgetting about my homework.
  \end{enumerate}
\end{eg}

\begin{note}
  In speech, we often hear 〜てしまう and 〜でしまう contracted as
  〜ちゃう and 〜じゃう, respectively.
\end{note}

\begin{warning}
  Note that しまう goes with the verbal テ form,
  which is affirmative, and so we cannot use 〜てしまう
  to express negated ideas or actions, such as, "unfortunately,
  I did not do $x$."
\end{warning}

% section _teshimau (end)

\section{〜と}%
\label{sec:_to}
% section _to

\begin{defn}[〜と]\index{〜と}\label{defn:to}
  Usually used as
  \begin{center}
    A とB,
  \end{center}
  where A is a situation, and B is the consequent effect.
  〜と is one of the conditional conjugates.
\end{defn}

\begin{note}
  The use of と here is not to be confused with its other use as an equivalence
  of ``and'' in English.
\end{note}

\begin{eg}
  \begin{enumerate}
    \item 私はその人と話すと元気になる。
    \item メアリーさんが国に帰ると寂しくなります。
    \item 秋になると木が赤くなります。
  \end{enumerate}
\end{eg}

\begin{warning}
  The event described by the second clause must be a rather clear consequence
  of the first situation.
  For instance, we do not say
  \begin{center}
    私はその人と話すと\ruby[j]{喫茶店}{きっ|さ|てん}に行きます。
  \end{center}
\end{warning}

% section _to (end)

\section{〜ながら}%
\label{sec:_nagara}
% section _nagara

\begin{defn}[〜ながら]\index{〜ながら}\label{defn:nagara}
  We write
  \begin{center}
    A \hlnoteb{ながら} B をします
  \end{center}
  to express that we do B while A is either being done or ongoing.
\end{defn}

\begin{note}
  It is important to note that the two verbs that flank ながら
  must be two actions performed by the same person.

  For the case where person 2 does something
  while person 1 is doing something else, we shall use
  \nameref{sec:aidani}.
\end{note}

\begin{eg}[Basic examples for 〜ながら]
  \begin{enumerate}
    \item 私はいつもラジオ番組を聞きながら
      勉強します。
    \item たけしさんは歌を歌いながら
      掃除します。
    \item アルバイトをしながら
      学校に行くのは大変です。
  \end{enumerate}
\end{eg}

% section _nagara (end)

\section{〜ばよかったです}%
\label{sec:_bayokatsutadesu}
% section _bayokatsutadesu

\begin{defn}[〜ばよかったです]\index{〜ばよかったです}\label{defn:_bayokatsutadesu}
  We say
  \begin{center}
    X すれ\hlnoteb{ばよかったです}
  \end{center}
  to indicate that \hlnotef{we hope we had did X}.
  It is used to describe an alternative course of actions that we
  regret not taking.
\end{defn}

\begin{note}
  For the rules of conjugating for 〜ば, we shall refer the reader to
  \nameref{sec:ba}, which gives a broader treatment to the form.
\end{note}

\begin{eg}
  \begin{enumerate}
    \item あの時、「愛している」と言えばよかったです。
    \item 彼女と別れなければよかったです。
  \end{enumerate}
\end{eg}

% section _bayokatsutadesu (end)

% chapter dai_18_ka (end)

\chapter{第19課}%
\label{chp:dai_19_ka}
% chapter dai_19_ka

\section{単語(19課)}%
\label{sec:tango_c19}
% section tango_c19

\begin{longtable}{r l l l}
\multicolumn{4}{l}{\hlnotea{名詞}} \\
* & おくさま       & 奥様           & (your/his) wife (polite) \\
  & おこさん       & お子さん       & (your/their) child (polite) \\
  & おれい         & お礼           & expression of gratitude \\
  & けいご         & 敬語           & honorific language \\
* & こちら         &                & this way (polite) \\
* & しゅっちょう   & 出張           & business trip \\
  & しゅるい       & 種類           & a kind; a sort \\
  & せいかく       & 性格           & personality; temperament \\
  & ちゅうがくせい & 中学生         & junior high school student \\
  & どちら         &                & where (polite) \\
  & なまけもの     & 怠け者         & lazy person \\
  & なやみ         & 悩み           & worry \\
  & はずかしがりや & 恥ずかしがりや & shy person \\
* & はなし         & 話             & chat; talk \\
* & ぶちょう       & 部長           & department manager; club president \\
  & ぶんか         & 文化           & culture \\
  & まちがい       & 間違い         & mistake \\
\multicolumn{4}{l}{\hlnotea{イ --- 形容詞}} \\
  & なかがいい & 仲がいい & to be on good terms; to get along well \\
\multicolumn{4}{l}{\hlnotea{ナ --- 形容詞}} \\
  & まじめな & 真面目な & serious; sober; diligent \\
\multicolumn{4}{l}{\hlnotea{ウ --- 動詞}} \\
  & いらっしゃる     &              & honorific form for いく, くる and いる \\
* & おくる           & 送る         & to walk/drive (someone) \\
  & おこる           & 怒る         & to gete angry \\
  & おっしゃる       &              & honorific form for 言う \\
* & おやすみになる   & お休みになる & honorific form for 寝る \\
* & きまる * 決まる  &              & to make a decision \\
  & くださる         & 下さる       & honorific form for くれる \\
  & ごらんいなる     & ご覧になる   & honorific form for 見る \\
  & しりあう         & 知り合う     & to get acquainted with \\
  & 〜ていらっしゃる &              & honorific form for 〜ている \\
  & なさる           &              & honorific form for する \\
  & ひっこす         & 引っ越す     & to move (to another place to live) \\
* & めしあがる       & 召し上がる   & honorific form for 食べる and 飲む \\
* & よぶ             & 呼ぶ         & to call (someone's name); to invite (person) \\
* & よる             & 寄る         & to stop by \\
\multicolumn{4}{l}{\hlnotea{ル --- 動詞}} \\
* & おくれる & 遅れる & to become late (for something) \\
  & かける   &        & to sit down (\textit{seat} に) \\
  & はれる   & 晴れる & to become sunny \\
  & もてる   &        & to be popular (in terms of romantic interest) \\
\multicolumn{4}{l}{\hlnotea{特別動詞}} \\
* & えんりょする & 遠慮する & to hold back for the time being; \\
  & & & to refrain from \\
* & ごちそうする & & to treat/invite (someone) to a meal \\
  & しょうたいする & 招待する & to invite someone (to an event/place) \\
  & ちゅういする & 注意する & to watch out; to give warning \\
* & はなしをする & 話をする & to have a talk \\
\multicolumn{4}{l}{\hlnotea{他の}} \\
  & おととい         &                & the day before yesterday \\
  & それで           &                & then; therefore \\
  & なぜ             &                & why (=どうして) \\
* & ほんとうは       & 本当は         & in fact; originally so \\
  & まいあさ         & 毎朝           & every morning \\
* & まだ             &                & still \\
* & 〜めいさま       & 〜名様         & party of ... people \\
  & ようこそ         &                & Welcome \\
* & よろしくおつたえ & よろしくお伝え & Please give my regards (to... ) \\
  & ください         & ください       &
\end{longtable}

% section tango_c19 (end)

\section{尊敬語 Honorific Verbs}%
\label{sec:sonkeigo_honorific_verbs}
% section sonkeigo_honorific_verbs

We use honorific verbs to show respect to the person
of which the action pertains to.

The following are some of the verbs converted to their
honorific forms, along with their \hlnotec{irregular conjugations}.

\begin{table}[ht]
  \centering
  \caption{Table of Basic Verbs in their Honorific Form and Their Irregular Conjugation}
  \label{table:table_of_basic_verbs_in_their_honorific_form_and_their_irregular_conjugation}
  \begin{tabular}{l | l l}
    Basic Verb & Honorific Verbs             & Irregular Conjugation \\
    \hline
    いる       &                             & \\
    行く       & いらっしゃる                & いらしゃ\underline{い}ます \\
    来る       &                             & \\
    \hline
    見る       & ご覧になる                  & \\
    \hline
    言う       & おっしゃる                  & おっしゃ\underline{い}ます \\
    \hline
    する       & なさる                      & なさ\underline{い}ます \\
    \hline
    食べる     & \multirow{2}{*}{召し上がる} & \\
    飲む       &                             & \\
    \hline
    くれる     & くださる                    & くださ\underline{い}ます \\
    \hline
    寝る       & お休みになる                & \\
    \hline
    〜ている   & 〜ていらっしゃる            & 〜ていらっしゃ\underline{い}ます
  \end{tabular}
\end{table}

\begin{warning}
  Honorific verbs are directed towards describing the action
  of the person that we speak of.
  We never use honorific verbs to describe our own actions.
\end{warning}

\begin{eg}
  \begin{enumerate}
    \item 先生は今日学校にいらっしゃいません。
    \item 何が召し上がりますか。
    \item 田中さんのお母さんがこの本をくださいました。
    \item 先生は自分で料理なさるそうです。
  \end{enumerate}
\end{eg}

\begin{note}
  For verbs that lack an honorific counterpart,
  we add the respect factor as follows:
  \begin{enumerate}
    \item Using ていらっしゃいます instead of ています,
      if the sentence has the helping verb ている. \\
      例:先生が電話で話していらっしゃいます。\\
      例:先生は疲れていらっしゃるみたいです。
    \item Flank the verb stem with お and になる,
      in most cases. \\
      例:先生はもうお帰りになりました。 \\
      例:この雑誌をお読みになったことがありますか。
  \end{enumerate}
\end{note}

\begin{note}
  As shown in the above examples,
  we can turn most combinations of a verb and a post-predicate expression
  into the honorific style simply by turning the verb into the honorific form.
  However, post-predicate expressions such as
  ことがあります and ください remain unchanged.
  This also applies to expressions like てもいい and てはいけない,
  to the potential verbs.

  It is noteworthy that it is note in good taste to talk about what
  the respected person can or cannot do, may or must not do.

  ている is exceptional in being a post-predicate that regularly undergoes the
  honorific style change.
  Special honorific verbs generally \hlnoted{take priority} over
  ていらっしゃる, such as in
  \begin{center}
    先生はテレビを\hlnoted{ご覧になっています}。
  \end{center}
  However, forms like 見ていらっしゃいます are also acceptable.
\end{note}

% section sonkeigo_honorific_verbs (end)

\section{Giving Respectful Advice}%
\label{sec:giving_respectful_advice}
% section giving_respectful_advice

We usually hear the form
\begin{center}
  お + verb stem + ください
\end{center}
in public address announcements and in the speech of store attendants.

\begin{eg}
  \begin{enumerate}
    \item 切符をお取りください。
    \item 説明をお読みください。
  \end{enumerate}
\end{eg}

While the sentences end with ください,
it is best to consider them as courteously phrased commands,
rather than a request.
In particular, you are `advised' to perform said action for your own sake.
Thus, it is inappropriate to say
\begin{center}
  $\times$ \ruby[j]{塩}{しお}をお取りください。
\end{center}
to mean that you wish for someone to pass the salt to you.

\begin{eg}
  Some further examples of giving respectful advice are the following:
  \begin{enumerate}
    \item ご注意ください。
    \item ご覧ください。
    \item お召し上がりください。
    \item お休みください。
  \end{enumerate}
  Notice the use of ご instead of お with certain compound verbs.
\end{eg}

% section giving_respectful_advice (end)

\section{〜てくれてありがとう}%
\label{sec:_tekuretearigatou}
% section _tekuretearigatou

\begin{defn}[〜てくれてありがとう]\index{〜てくれてありがとう}\label{defn:_tekuretearigatou}
  When we want to \hlnotef{express gratitude to someone for a specific action},
  we can say
  \begin{center}
    verb テ form + くれてありがとう.
  \end{center}
\end{defn}

\begin{eg}
  \begin{enumerate}
    \item 手伝ってくれてありがとう。
    \item 推薦状を書いてくださってありがとうございました。
    \item いい友達でいてくれてありがとう。
  \end{enumerate}
\end{eg}

% section _tekuretearigatou (end)

\section{〜てよかったです}%
\label{sec:_teyokatsutadesu}
% section _teyokatsutadesu

\begin{defn}[〜てよかった]\index{〜てよかった}\label{defn:_teyokatsuta}
  We say
  \begin{center}
    verb テ form てよかったです
  \end{center}
  to express that we are \hlnotef{glad that a certain action (or inaction) is done}.
\end{defn}

\begin{eg}
  \begin{enumerate}
    \item 日本語を勉強してよかったです。
    \item メアリーさんが元気になってよかったです。
    \item きのうのパーティーに行かなくてよかったです。
  \end{enumerate}
\end{eg}

% section _teyokatsutadesu (end)

\section{〜はずです}%
\label{sec:_hazudesu}
% section _hazudesu

\begin{defn}[〜はずです]\index{〜はずです}\label{defn:_hazudesu}
  We say
  \begin{center}
    verb stem + はずです
  \end{center}
  to show that an action was \hlnotef{supposed to be the case}.

  We may also use 〜はずです with adjectives and nouns,
  almost without explicit modification but in their modal forms.
\end{defn}

\begin{eg}
  \begin{enumerate}
    \item 今日は日曜日だから、銀行は閉まっているはずです。
    \item きのうメアリーさんはどこにも行かなかったはずです。
    \item この試験は難しいはずです。
    \item そんなはずじゃなかったです。
    \item 日本人のはずです。
  \end{enumerate}
\end{eg}

% section _hazudesu (end)

% chapter dai_19_ka (end)

\chapter{第20課}%
\label{chp:dai_20_ka}
% chapter dai_20_ka

\section{単語(20課)}%
\label{sec:tango_c20}
% section tango_c20

\begin{longtable}{r l l l}
\multicolumn{4}{l}{\hlnotea{名詞}} \\
  & あちら       &          & that way (polite) \\
  & アニメ       &          & animation \\
  & うちゅうじん & 宇宙人   & space alien \\
* & おと         & 音       & sound \\
  & おにぎり     &          & rice ball \\
* & かかりのもの & 係の者   & (our) person in charge \\
* & かど         & 角       & corner \\
  & くうこう     & 空港     & airport \\
  & じ           & 字       & letter; character \\
  & してん       & 支店     & branch office \\
  & しゅみ       & 趣味     & hobby; pastime \\
  & しょうせつ   & 小説     & novel \\
  & しんご       & 信号     & traffice light \\
  & スニーカー   &          & sneakers \\
* & せんす       & 扇子     & paper fan \\
  & つき         & 月       & moon \\
* & でんしじしょ & 電子辞書 & electronic dictionary \\
  & ドイツ       &          & Germany \\
  & ハイヒール   &          & high heels \\
* & 〜や         & 〜屋     & .. shop \\
\multicolumn{4}{l}{\hlnotea{イ --- 形容詞}} \\
* & おもい & 重い & heavy; serious (illness, topic) \\
  & かるい & 軽い & light \\
\multicolumn{4}{l}{\hlnotea{ウ --- 動詞}} \\
* & いたす     & 致す   & extra-modest form for する \\
* & いただく   & 頂く   & extra-modest form for 食べる and 飲む; \\
  &            &        & humble form for もらう \\
  & うかがう   & 伺う   & to humbly visit; to humbly ask \\
  & おる       &        & extra-modest form for いる \\
* & ござる     &        & extra-modest form for ある \\
  & 〜ておる   &        & extra-modest form for 〜ている \\
  & 〜てござる &        & extra-modest form for です \\
* & まいる     & 参る   & extra-modest form for いく and くる \\
* & まがる     & 曲がる & to run at a corner \\
* & もうす     & 申す   & extra-modest form for 言う \\
  & もどる     & 戻る   & to come back \\
\multicolumn{4}{l}{\hlnotea{ル --- 動詞}} \\
* & きこえる   & 聞こえる   & to be audible \\
  & さしあげる & 差し上げる & humble form for あげる \\
  & つたえる   & 伝える     & to convey (a message) \\
* & またせる   & 待たせる   & to keep (someone) waiting \\
\multicolumn{4}{l}{\hlnotea{特別動詞}} \\
* & こうかんする & 交換する & to exchange \\
  & せいかつする & 生活する & to lead a life \\
* & へんぴんする & 返品する & to return (merchandise) \\
\multicolumn{4}{l}{\hlnotea{他の}} \\
* & おや?           &                  & Oh? \\
  & 〜かい           & 〜階             & ...th floor \\
* & かしこまりました &                  & Certainly. \\
  & さあ             &                  & I am not very sure, ... \\
* & しつれいしました & 失礼いたしました & I am very sorry. \\
* & しょうしょう     & 少々             & a few seconds \\
  & それでは         &                  & if that is the case, ... \\
* & できれば         &                  & if possible \\
* & まことに         & 誠に             & really; sincerely (very polite) \\
  & また             &                  & again \\
  & 〜みたいな       &                  & such as... \\
* & もうしわけ       & 申し訳           & You have my apologies. \\
  & ありません       & ありません       & \\
* & よろしかったら   &                  & if it is okay (polite)
\end{longtable}

% section tango_c20 (end)

\section{丁重語 Extra-modest Expressions}%
\label{sec:teichougo_extra_modest_expressions}
% section teichougo_extra_modest_expressions

To speak modestly of our own actions, we shall use what is called
\ruby[j]{丁重語}{てい|ちょう|ご}.
\cref{table:basic_extra_modest_expressions} shows some of the basic
extra-modest expressions.

\begin{table}[ht]
  \centering
  \caption{Basic Extra-modest Expressions}
  \label{table:basic_extra_modest_expressions}
  \begin{tabular}{l l l}
             & \multicolumn{2}{l}{extra-modest expression} \\
    いる     & おります                      & (おる) \\
    行く     & \multirow{2}{*}{参ります}     & \multirow{2}{*}{(参る)} \\
    来る     &                               & \\
    言う     & 申します                      & (申す) \\
    する     & いたします                    & (いたす) \\
    食べる   & \multirow{2}{*}{いただきます} & \multirow{2}{*}{(いただく)} \\
    飲む     &                               & \\
    ある     & ございます                    & (ござる) \\
    〜ている & 〜ております                  & (〜ておる) \\
    〜です   & 〜でございます                & (〜でござる)
  \end{tabular}
\end{table}

\begin{eg}
  \begin{enumerate}
    \item 私は来年も日本におります。 cf. います
    \item 私は今年の六月に大学を卒業いたしました。 cf. 卒業します
    \item 電車が参ります。
    \item 父が家におります。
  \end{enumerate}
\end{eg}

% section teichougo_extra_modest_expressions (end)

\section{謙虚語 Humble Expressions}%
\label{sec:kenkyogo_humble_expressions}
% section kenkyogo_humble_expressions

When doing something out of respect for somebody,
we can describe our actions using a verb in the humble pattern
\begin{center}
  お + stem + する.
\end{center}
Note that certain する compound verbs do not follow this pattern,
but instead takes on the form
\begin{center}
  ご + base compound verb + する.
\end{center}

\begin{eg}
  \begin{enumerate}
    \item 私は昨日先生にお会いしました。
    \item 私は先生の本をお貸しするつもりです。
    \item 部長にお電話します。
    \item お客様をご案内します。
  \end{enumerate}
\end{eg}

\begin{warning}
  It has been debated that we should avoid using 差し上げる
  to say that ``we do something for somebody''.
  The argument is that the idea of doing something for somebody
  as a service is but an insolent belief, and having to speak
  of it in a humble way is an unconvincing facade.
  Instead, we make use of the お + stem + する form in this cases.
\end{warning}

% section kenkyogo_humble_expressions (end)

\section{〜ないで}%
\label{sec:_naide}
% section _naide

\begin{defn}[〜ないで]\index{〜ないで}\label{defn:_naide}
  When doing something without doing something else,
  we say
  \begin{center}
    X し\hlnoteb{ないで} Y します
  \end{center}
  to mean that we do Y without doing X.
\end{defn}

\begin{note}
  We use 〜ない for both present and past actions.
  One may think of the speaker say that they did not do X
  while thinking of themselves as being at that time when
  the action was expected to happen.
\end{note}

\begin{eg}
  \begin{enumerate}
    \item 昨日の夜は、寝ないで、勉強しました。
    \item 辞書を使わないで、新聞を読みます。
  \end{enumerate}
\end{eg}

% section _naide (end)

\section{Questions within Larger Sentences}%
\label{sec:questions_within_larger_sentences}
% section questions_within_larger_sentences

One can include a question as part of a longer sentence
and express ideas such as ``I don't know when the test is''
and ``I don't remember whether Mary came to the party.''

\begin{eg}
  We shall highlight the quoted question clauses in the following
  examples:
  \begin{enumerate}
    \item 山下先生は\hlnoted{昨日何を食べたか}覚えていません。
    \item \hlnoted{メアリーさんは来たかどうか}知っていますか。 
  \end{enumerate}
\end{eg}

% section questions_within_larger_sentences (end)

\section{name という item}%
\label{sec:name_toiu_item}
% section name_toiu_item

\begin{defn}[name という item]\index{name という item}\label{defn:name_toiu_item}
  When we talk about something that we do not think our
  listener may not be familiar with, we can use the pattern
  \begin{center}
    name という category,
  \end{center}
  where the category tries to describe the item in the name
  usually in a generic sense.
\end{defn}

\begin{eg}
  \begin{enumerate}
    \item ポチという犬
    \item 「花」という歌
  \end{enumerate}
\end{eg}

% section name_toiu_item (end)

\section{〜やすい / 〜にくい}%
\label{sec:_yasui_nikui}
% section _yasui_nikui

When describing that something is ``easy to do'' or
``difficult to do'', we use the conjugatation
〜やすい and 〜にくい, respectively.

\begin{eg}
  \begin{enumerate}
    \item この電子辞書は使いやすいです。 cf 使う
    \item 骨が多いので、魚は食べにくいです。 cf 食べる
  \end{enumerate}
\end{eg}

% section _yasui_nikui (end)

% chapter dai_20_ka (end)

\chapter{第21課}%
\label{chp:dai_21_ka}
% chapter dai_21_ka

\section{単語(21課)}%
\label{sec:tango_c21}
% section tango_c21

\begin{longtable}{r l l l}
\multicolumn{4}{l}{\hlnotea{名詞}} \\
  & あかちゃん & 赤ちゃん & baby \\
  & か         & 蚊       & mosquito \\
  & かいぎ     & 会議     & business meeting; conference \\
  & ガソリン   &          & gasoline \\
  & かんきょう & 環境     & environment \\
* & けいさつ   & 警察     & police \\
  & こうじょう & 工場     & factory \\
* & こと       & 事       & things; matters \\
* & しゅうでん & 終電     & last train \\
  & じゅんび   & 準備     & preparation \\
  & スピーチ   &          & speech \\
  & せいふ     & 政府     & government \\
  & ちかん     &          & sexual offender; prevent \\
  & どうりょう & 同僚     & colleague \\
* & どろぼう   & 泥棒     & thief; burglar \\
* & バイト     &          & short for アルバイト \\
* & はんにん   & 犯人     & criminal \\
  & ポスター   &          & poster \\
  & むかし     & 昔       & old days; past \\
  & もんく     & 文句     & complaint \\
* & るす       & 留守     & absence; not at home \\
\multicolumn{4}{l}{\hlnotea{イ --- 形容詞}} \\
  & とおい & 遠い & far away \\
  & ひどい & 酷い & awful; cruel \\
\multicolumn{4}{l}{\hlnotea{ナ --- 形容詞}} \\
  & あんぜん     & 安全 & safe \\
  & たいせつ     & 大切 & precious; valuable \\
* & めちゃくちゃ &      & messy; disorganized \\
\multicolumn{4}{l}{\hlnotea{ウ --- 動詞}} \\
  & おく         & 置く       & to put; to lay; to place \\
* & きがつく     & 気が付く   & to notice \\
  & ける         &            & to kick \\
  & さす         & 刺す       & to sting; to poke \\
  & さわる       & 触る       & to touch \\
* & つかまる     & 捕まる     & to be arrested; to be caught \\
  & つつむ       & 包む       & to wrap; to cover \\
  & なぐる       & 殴る       & to strike; to hit; to punch \\
  & ぬすむ       & 盗む       & to steal; to rob \\
  & はる         & 貼る       & to post; to stick \\
  & ふむ         & 踏む       & to step on \\
  & ふる         &            & to turn down; to reject; to jilt \\
  & もんくをいう & 文句を言う & to complain \\
  & やく         & 焼く       & to bake; to burn; to grill \\
  & やる         &            & to give (to pets, plants, younger siblings, etc.) \\
\multicolumn{4}{l}{\hlnotea{ル --- 動詞}} \\
  & いじめる   &          & to bully \\
  & きがえる   & 着替える & to change clothes \\
* & ためる     &          & to save money \\
  & つづける   & 続ける   & to continue \\
  & ほめる     & 褒める   & to praise; to say nice things \\
  & まちがえる & 間違える & to make a mistake \\
  & みつける   & 見つける & to find \\
\multicolumn{4}{l}{\hlnotea{特別動詞}} \\
  & ばかにする   &            & to insult; to make a fool of... \\
* & びっくりする &            & to be surprised \\
  & ひるねをする & 昼寝をする & to take a nap \\
* & れんらくする & 連絡する   & to contact \\
\multicolumn{4}{l}{\hlnotea{他の}} \\
* & 〜あいだに & 〜間に & while ... \\
* & ころ       & 頃     & time of ...; when ... \\
  & すこし     & 少し   & a little \\
* & とにかく   &        & anyway; anyhow
\end{longtable}

% section tango_c21 (end)

\section{受動文 Passive Sentences}%
\label{sec:jyudoubun_passive_sentences}
% section jyudoubun_passive_sentences

We can change verbs into their passive forms, and use them as we do in English.

\begin{table}[ht]
  \centering
  \caption{Rules of converting from base form to passive form}
  \label{table:rules_of_converting_from_base_form_to_passive_form}
  \begin{tabular}{l c l c l c l}
  \multicolumn{7}{l}{ル --- 動詞: 「る」は「られる」になる} \\
  食べる & $\to$ & 食べられる &  &  &  &  \\
  $  $ \\
  \multicolumn{7}{l}{う --- 動詞: 最後の「う」の音は「あれる」になる} \\
  行く & $\to$ & 行かれる & & 話す & $\to$ & 話される \\
  待つ & $\to$ & 待たれる & & 死ぬ & $\to$ & 死なれる \\
  読む & $\to$ & 読まれる & & 取る & $\to$ & 取られる \\
  泳ぐ & $\to$ & 泳がれる & & 遊ぶ & $\to$ & 遊ばれる \\
  買う & $\to$ & 買われる \\
  $  $ \\
  \multicolumn{7}{l}{特別動詞} \\
  来る & $\to$ & 来られる & & する & $\to$ & される
  \end{tabular}
\end{table}

\begin{note}
  One may notice that the passive forms for ル動詞 and 特別動詞
  are the same as their potential forms, while
  ウ動詞 are different from their potential counterparts.
\end{note}

% section jyudoubun_passive_sentences (end)

\section{〜てある}%
\label{sec:tearu}
% section tearu



% section tearu (end)

\section{〜間に}%
\label{sec:aidani}
% section aidani



% section aidani (end)

\section{形容詞+する}%
\label{sec:keiyoushi_suru}
% section keiyoushi_suru



% section keiyoushi_suru (end)

\section{〜てほしい}%
\label{sec:tehoshii}
% section tehoshii



% section tehoshii (end)

% chapter dai_21_ka (end)

\chapter{第22課}%
\label{chp:dai_22_ka}
% chapter dai_22_ka

\section{単語(22課)}%
\label{sec:tango_c22}
% section tango_c22



% section tango_c22 (end)

\section{使役動詞 Causative Verbs}%
\label{sec:shiekidoushi_causative_verbs}
% section shiekidoushi_causative_verbs

The verb derivation that shall be introduced here is called the
\hldefn{causative form}. We use causative forms when we want to describe
who makes someone do something, or who lets someone do something.

\begin{table}[ht]
  \centering
  \caption{Rules of converting from base form to causative form}
  \label{table:rules_of_converting_from_base_form_to_causative_form}
  \begin{tabular}{l c l c l c l}
  \multicolumn{7}{l}{ル --- 動詞: 「る」は「させる」になる} \\
  食べる & $\to$ & 食べさせる &  &  &  &  \\
  $  $ \\
  \multicolumn{7}{l}{う --- 動詞: 最後の「う」の音は「あせる」になる} \\
  行く & $\to$ & 行かせる & & 話す & $\to$ & 話させる \\
  待つ & $\to$ & 待たせる & & 死ぬ & $\to$ & 死なせる \\
  読む & $\to$ & 読ませる & & 取る & $\to$ & 取らせる \\
  泳ぐ & $\to$ & 泳がせる & & 遊ぶ & $\to$ & 遊ばせる \\
  買う & $\to$ & 買わせる \\
  $  $ \\
  \multicolumn{7}{l}{特別動詞} \\
  来る & $\to$ & 来させる & & する & $\to$ & させる
  \end{tabular}
\end{table}

% section shiekidoushi_causative_verbs (end)

\section{動詞+なさい}%
\label{sec:doushi_nasai}
% section doushi_nasai



% section doushi_nasai (end)

\section{〜ば}%
\label{sec:ba}
% section ba

\begin{defn}[〜ば]\index{〜ば}\label{defn:_ba}
  We use \hlnoteb{〜ば} to express a condition at which an event that follows
  occur. We write
  \begin{center}
    A ば B
  \end{center}
  to say ``if A then B''.
\end{defn}

\begin{table*}[ht]
  \centering
  \caption{Rules of conjugation for 〜ば}
  \label{table:rules_of_conjugation_for_ba}
  \begin{tabular}{l c l c l c l c l c l}
  \multicolumn{11}{l}{Verbs in the affirmative:} \\
  \multicolumn{11}{l}{「う」の音を捨てる、「えば」になる} \\
  食べる & $\to$ & 食べれば \\
  行く   & $\to$ & 行けば &  & 待つ & $\to$ & 待てば &  & 買う & $\to$ & 買えば \\
  する   & $\to$ & すれば \\
  来る   & $\to$ & 来れば \\
  $  $ \\
  \multicolumn{11}{l}{Verbs in the negative} \\
  \multicolumn{11}{l}{「い」を捨てる、「ければ」になる} \\
  行かない & $\to$ & 行かなければ
  \end{tabular}
\end{table*}

% section ba (end)

\section{〜のに}%
\label{sec:noni}
% section noni



% section noni (end)

\section{〜のような / 〜のように}%
\label{sec:noyouna_noyouni}
% section noyouna_noyouni



% section noyouna_noyouni (end)

% chapter dai_22_ka (end)

\chapter{第23課}%
\label{chp:dai_23_ka}
% chapter dai_23_ka

\section{単語(23課)}%
\label{sec:tango_c23}
% section tango_c23



% section tango_c23 (end)

\section{使役動詞の受動態 Causative-Passive Forms}%
\label{sec:shiekidoushi_no_jyudoudai_causative_passive_forms}
% section shiekidoushi_no_jyudoudai_causative_passive_forms



% section shiekidoushi_no_jyudoudai_causative_passive_forms (end)

\section{〜ても}%
\label{sec:temo}
% section temo



% section temo (end)

\section{〜ことにする}%
\label{sec:kotonisuru}
% section kotonisuru



% section kotonisuru (end)

\section{〜まで}%
\label{sec:made}
% section made



% section made (end)

\section{〜方}%
\label{sec:kata}
% section kata



% section kata (end)

% chapter dai_23_ka (end)

\appendix

\backmatter

\fancyhead[LE]{\thepage \enspace \textsl{\leftmark}}

% \nobibliography*
\bibliography{references}

\printindex

\end{document}
% vim:tw=80:fdm=syntax

