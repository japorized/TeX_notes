\documentclass{tufte-handout}
% \nonstopmode % uncomment to enable nonstopmode

\title{How-to: Use my notes}
\author{Johnson Ng}

\renewcommand{\baselinestretch}{1.1}
\usepackage{geometry}
\geometry{letterpaper}
\usepackage[parfill]{parskip}
\usepackage{graphicx}
\usepackage{enumitem}
\usepackage{ragged2e}
\usepackage{fontawesome}
\usepackage{amsmath}
\usepackage{xcolor}
\usepackage{mathrsfs}
\usepackage{stmaryrd}
\usepackage{tcolorbox}
\tcbuselibrary{skins,breakable}
\usepackage{hyperref}

\hypersetup{
    unicode=true,          % non-Latin characters in Acrobat’s bookmarks
    pdftoolbar=false,        % show Acrobat’s toolbar?
    pdfmenubar=false,        % show Acrobat’s menu?
    pdffitwindow=true,     % window fit to page when opened
    colorlinks=true,
    allcolors=be-magenta,
}

\definecolor{dark}{HTML}{2D2D2D}
\definecolor{light}{HTML}{D3D0C8}
\definecolor{be-red}{HTML}{F47678}
\definecolor{be-green}{HTML}{98CD97}
\definecolor{be-yellow}{HTML}{E2B552}
\definecolor{be-blue}{HTML}{6498CE}
\definecolor{be-magenta}{HTML}{CD98CD}
\definecolor{be-cyan}{HTML}{61CCCD}
\definecolor{be-gray}{HTML}{747369}
\definecolor{be-brown}{HTML}{D47B4E}

% section format
\titleformat{\section}%
  {\normalfont\Large\rmfamily\itshape\color{be-blue}}% format applied to label+text
  {\llap{\colorbox{be-blue}{\parbox{1.5cm}{\hfill\itshape\textcolor{dark}{\thesection}}}}}% label
  {5pt}% horizontal separation between label and title body
  {}% before the title body
  []% after the title body


\newcommand{\hlimpo}[1]{\textcolor{be-red}{\textbf{#1}}}
\newcommand{\hlwarn}[1]{\textcolor{be-yellow}{\textbf{#1}}}
\newcommand{\hldefn}[1]{\textcolor{be-blue}{\textbf{#1}}}
\newcommand{\hlnotea}[1]{\textcolor{be-green}{\textbf{#1}}}
\newcommand{\hlnoteb}[1]{\textcolor{be-cyan}{\textbf{#1}}}
\newcommand{\hlnotec}[1]{\textcolor{be-brown}{\textbf{#1}}}
\newcommand{\hlb}[2]{\colorbox{#1!30!dark}{\textbf{#2}}}
\newcommand{\hlbnotea}[1]{\hlb{be-green}{#1}}
\newcommand{\hlbnoteb}[1]{\hlb{be-cyan}{#1}}
\newcommand{\hlbnotec}[1]{\hlb{be-brown}{#1}}
\newcommand{\hlbnoted}[1]{\hlb{be-magenta}{#1}}
\newcommand{\hlbnotee}[1]{\hlb{be-red}{#1}}

\pagecolor{dark}
\color{light}

\begin{document}

\pagenumbering{false}
\section{Usage}%

Notes are presented in two columns: main notes on the left, and sidenotes on the right.
Main notes will have a larger margin.
The following is the color code for the notes: \\
\begin{table}[ht]
  \caption{Text color coding}
  \begin{tabular}{l l}
    \hldefn{Blue}                     & Definitions \\
    \hlimpo{Red}                      & Important points \\
    \hlwarn{Yellow}                   & Points to watch out for \\
                                      & / comment for incompletion \\
    \hlnotea{Green}                   & External definitions, theorems, etc. \\
    \hlnoteb{Light Blue}              & Regular highlighting \\
    \hlnotec{Brown}                   & Secondary highlighting \\
    \hlbnotea{Highlighted Green}      & Usually used in cases in proofs \\
    \hlbnoted{Highlighted Magenta}    & Also used for cases in proofs; \\
                                      & secondary highlighting \\
    \hlbnoteb{Highlighted Light Blue} & Tertiary highlighting \\
    \hlbnotec{Highlighted Brown}      & Quaternary highlighting
  \end{tabular}
\end{table}

\noindent
The following is the color code for boxes, that begin and end with a line of the same color: \\
\begin{table}[ht]
  \caption{Environment color coding}
  \begin{tabular}{l l}
    \hldefn{Blue}                   & Definitions \\
    \hlimpo{Red}                    & Warning \\
    \hlwarn{Yellow}                 & Notes, remarks, etc. \\
    \hlnotec{Brown}                 & Proofs \\
    \textcolor{be-magenta}{Magenta} & Theorems, Propositions, Lemmas, etc.
  \end{tabular}
\end{table}

\noindent
Hyperlinks are underlined in \textcolor{be-magenta}{magenta}. If your PDF reader supports it,
you can follow the links to either be redirected to an external website, or a theorem, definition,
etc., in the same document. Note that this is only reliable if you have the full set of notes as a
single document, which you can find on: \\ 
\url{https://japorized.github.io/TeX_notes} or \\
\url{https://tex.japorized.ink}

\nobibliography*
\bibliography{references}

\printindex

\end{document}

