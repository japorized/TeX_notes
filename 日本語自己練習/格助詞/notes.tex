% !TEX TS-program = lualatex
\documentclass{tufte-handout}
% \nonstopmode % uncomment to enable nonstopmode

\usepackage[japanese]{babel}
\usepackage{natbib}
\bibliographystyle{apalike}
\usepackage{titlesec}
\usepackage{luatexja}
\usepackage{luatexja-fontspec}
\usepackage{pxrubrica}
\usepackage{fancyhdr}
\usepackage[fixed]{fontawesome5}

\newcommand{\personalcolor}{false}
\ifthenelse{\equal{\personalcolor}{true}}{
  \usepackage{colorscheme-chaos}
}{
  \usepackage{colorscheme-student}
}

\pagecolor{background}
\color{foreground}

\setmainjfont{IPAMincho}

\makeatletter
\pagestyle{fancy}
\fancyhead{}
\fancyhead[R]{\textsl{\@title} \enspace \thepage}
\makeatother

\titleformat{\section}%
  {\LARGE\rmfamily\itshape\color{magenta}}% format applied to label+text
  {}% label
  {}% horizontal separation between label and title body
  {}% before the title body
  []% after the title body

\titleformat{\subsection}%
  {\Large\rmfamily\itshape\color{cyan}}% format applied to label+text
  {}% label
  {}% horizontal separation between label and title body
  {}% before the title body
  []% after the title body

\title{格助詞 簡略まとめと練習}
\author{Japorized}

\makeatletter
\renewcommand{\maketitle}{%
\thispagestyle{empty}
\noindent
{\Huge\@title\unskip\strut\par}
\noindent
{\small\@author\unskip\strut\par}

\vspace{0.5cm}
\hrule
\vspace{1.5cm}
}
\makeatother

\begin{document}
\maketitle
\nocite{etsuko}
\nocite{tkforeignlang}

\section{格助詞の覚え方}%
\label{sec:kakujoshinooboekata}
% section kakujoshinooboekata

「鬼が戸よりで、空の部屋」

\noindent
「を・に・が・と・より・で・から・の・へ・や」

% section kakujoshinooboekata (end)

\section{各格助詞の使い方}%
\label{sec:kakukakujoshinotsukaikata}
% section kakukakujoshinotsukaikata

\subsection{「を」の使い方}%
\label{sub:wo_notsukaikata}
% subsection wo_notsukaikata

\begin{itemize}
  \item \ruby[j]{動作}{どう|さ}の\ruby[j]{対象}{たい|しょ}を
    \ruby[j]{表}{あらわ}す
  \item \ruby[j]{通過}{つう|か}、移動する場所を表す
  \item \ruby[j]{離}{はな}れる場所、\ruby[j]{起点}{き|てん}を表す
  \item 「〜をしている」の\ruby[j]{形}{かたち}で、物や人の
    \ruby[j]{形状}{けい|じょう}を表す
\end{itemize}

% subsection wo_notsukaikata (end)

\subsection{「に」の使い方}%
\label{sub:ni_notsukaikata}
% subsection ni_notsukaikata

\begin{itemize}
  \item 物や\ruby[j]{生物}{せい|ぶつ}が\ruby[j]{存在}{そん|ざい}
    する場所、\ruby[j]{状態}{じょう|たい}が表れている場所を
    あらわす
  \item 動作の対象の\ruby[j]{到着点}{とう|ちゃく|てん}、動作の
    \ruby[j]{主体}{しゅ|たい}の到着点を表す
  \item 動作が\ruby[j]{及}{およ}ぶ対象を表す
  \item \ruby[j]{行為}{こう|い}を表す名詞や、動詞の
    \ruby[j]{連用形}{れん|よう|かた}について、行為の目的を
    表す
  \item \ruby[j]{時点}{じ|てん}を表す
  \item \ruby[j]{比較}{ひ|かく}の\ruby[j]{基準}{き|じゅん}や
    \ruby[j]{適用}{てき|よう}の\ruby[j]{範囲}{はん|い}を
    表す(例:〜に似ている)
\end{itemize}

% subsection ni_notsukaikata (end)

\subsection{「が」の使い方}%
\label{sub:ga_notsukaikata}
% subsection ga_notsukaikata

\begin{itemize}
  \item 動作、\ruby[j]{出来事}{で|き|ごと}の
    \ruby[j]{主体}{しゅ|たい}(\ruby[j]{疑問詞}{ぎ|もん|し}の場合
    もある)を\ruby[j]{示}{しめ}す。目の前のことを言うときや、
    他のものでないと言いたいときによく使われる (rule out
    options)
  \item 「〜は〜が...」の形の文で、話題の一部分や、能力、感
    覚、感情の対象を示す
\end{itemize}

% subsection ga_notsukaikata (end)

\subsection{「と」の使い方}%
\label{sub:to_notsukaikata}
% subsection to_notsukaikata

\begin{itemize}
  \item \ruby[j]{比}{く}べるものを表す
  \item 行為を一緒にする相手を表す
  \item 対する相手、比べる基準を表す
  \item 名前や発言、考えなどの内容を表す
\end{itemize}

% subsection to_notsukaikata (end)

\subsection{「より」の使い方}%
\label{sub:yori_notsukaikata}
% subsection yori_notsukaikata

\begin{itemize}
  \item 比較の基準を表す
  \item 出発点や時の起点を表すのが本来の用法
\end{itemize}

% subsection yori_notsukaikata (end)

\subsection{「で」の使い方}%
\label{sub:de_notsukaikata}
% subsection de_notsukaikata

\begin{itemize}
  \item 行為、行事が行われる場所、出来事が起こる場所を表
    す
  \item 手段、材料を表す
  \item 範囲を表す
  \item 原因、理由を表す
\end{itemize}

% subsection de_notsukaikata (end)

\subsection{「から」の使い方}%
\label{sub:kara_notsukaikata}
% subsection kara_notsukaikata

\begin{itemize}
  \item 範囲の起点を表す
  \item 原料を表す
  \item 出発点を表す用法からの拡張で、「が」の代わりにも
    のや情報の与え手としての主語を表示することもある
  \item 「に」の代わりに、\ruby[j]{使役}{つ|えき}や
    \ruby[j]{受身}{うけ|み}の\ruby[j]{動作主}{どう|さ|しゅ}を表す
\end{itemize}

% subsection kara_notsukaikata (end)

\subsection{「の」の使い方}%
\label{sub:no_notsukaikata}
% subsection no_notsukaikata

\begin{itemize}
  \item 名詞について、他の名詞との間に何かの関係があること
    を表す
  \item 「が」に代わって、名詞を修飾する節の主語を表わす
    ことがある
\end{itemize}

% subsection no_notsukaikata (end)

\subsection{「へ」の使い方}%
\label{sub:e_notesukaikata}
% subsection e_notesukaikata

\begin{itemize}
  \item 移動の\ruby[j]{方向}{ほう|こう}や着点を表す
\end{itemize}

% subsection e_notesukaikata (end)

\subsection{「や」の使い方}%
\label{sub:ya_notsukaekata}
% subsection ya_notsukaekata

\begin{itemize}
  \item 二つ以上の名詞を\ruby[j]{並}{な}べ立てるときに使う
\end{itemize}

% subsection ya_notsukaekata (end)

\newpage

\section{練習}%
\label{sec:renshuu}
% section renshuu

\subsection{練習1}%
\label{sub:renshuu_1}
% subsection renshuu_1

(  )のなかに「を」が「と」を書いてください。そして、
選択の理由を簡単に説明してください。

\begin{enumerate}
  \item サラ「あれ、マリさん、どうしてここにいるの?」 \\
    マリ「リサさん(  )待っているの。彼女(  )映
    画を見に行くの。」
  \item 9時に家(  )出ました。
  \item サラの「や」の書き方はぼくの書き方(  )違う
    ね。
  \item 先生「一人でこの教室(  )掃除したの?」 \\
    学生「いえ、トムさんやサラさん(  )一緒にしました
    。」
  \item 空(  )飛んで、南の国へ行きたい。
  \item 駅で友達(  )別れて家に帰りました。
  \item プレゼントをくれた人にお礼(  )言います。
\end{enumerate}

% subsection renshuu_1 (end)

\subsection{練習2}%
\label{sub:renshuu_2}
% subsection renshuu_2

aかbかいいほうを選んでください。そして、選択の理由を簡
単に説明してください。

\begin{enumerate}
  \item 2時に東京駅を(a 降りました  b 出発しました)。
  \item 昨日は会社を(a 休みました  b 働きました)。
  \item こちらの山道を(a 休むところがありますよ  b 歩き
    ましょう)。
  \item 廊下を(a 走らないでください  b 遊ばないでくださ
    い)。
  \item ぼくはマリさんと(a 電話をかけます  b 結婚し
    ます)。
  \item 会議室は、七階でエレベーターを(a 乗って  b
    降りて)、すぐ目の前ですよ。
  \item 駅でサラさんと(a 見て  b 会って)、一緒に公
    園へ行きます。
  \item 【タクシーの中】\\
    運転手さん、次の角を(a 降ります  b 曲がってくださ
    い)。
  \item この漢字は何と(a 読みますか  b 使いますか)。
\end{enumerate}

% subsection renshuu_2 (end)

\subsection{練習3}%
\label{sub:renshuu_3}
% subsection renshuu_3

(  )の中に「で」か「に」を書いてください。そして、
選択の理由を簡単に説明してください。

\begin{enumerate}
  \item 子供の部屋(  )大きい窓を作りました。部屋が明
    るくなりました。
  \item 子供の部屋(  )一緒に紙の飛行機を作りました。
  \item 事故(  )電車が止まっています。
  \item 東京駅(  )地下鉄(  )乗り換えます。
  \item 風(  )外の洗濯物が飛びそうです。中(  )入
    れてください。
  \item 明日ホール(  )お茶の会はあります。3時(
      )ホール(  )集めってください。
  \item むこう(  )高い山が見えるでしょう?今日はあ
    の山(  )登ります。
  \item この紙(  )黒いペン(  )名前を書いてくださ
    い。
\end{enumerate}

% subsection renshuu_3 (end)

\subsection{練習4}%
\label{sub:renshuu_4}
% subsection renshuu_4

a か b かいいほうを選んでください。そして、選択の理由を
簡単に説明してください。

\begin{enumerate}
  \item 公園で(a \ruby[j]{池}{いけ}が  b コンサートが)あります。
  \item ここに(a 荷物をおいて  b 料理を作って)ください。
  \item 台風で(a 橋が壊れました  b 橋を\ruby[j]{渡}{わた}り
    ました)。
  \item 図書館に本を(a 返します  b 探します)。
  \item このホテルに(a 一番いい部屋はどこですか  b とて
    もいい部屋があります)。
  \item 駅の前に(a 花屋ができました  b 花を買いました)。
  \item いつも(a 八時間に  b 八時に)寝ます。
  \item 動物園にいろいろな(a 動物がいます  b 動物を見ま
    しょう)。
  \item 世界で(a いろいろな国があります b 一番広い国は
    どこですか)。
\end{enumerate}

% subsection renshuu_4 (end)

\subsection{練習5}%
\label{sub:renshuu_5}
% subsection renshuu_5

\marginnote{「は」は格助詞じゃないが、「が」と勘違いする場
所が多いから、この練習をノートに入りました。}

(  )の中に「は」か「が」を書いてください。そして、
選択の理由を簡単に説明してください。

\begin{enumerate}
  \item どの問題(  )難しですか。
  \item 一時間(  )60分です。
  \item A「見てください。桜(  )たくさん咲いていますよ 。」 \\
    B「ええ、桜(  )本当にきれいな花ですね。」
  \item ああ、冷たいビール(  )飲みたい。
  \item A「どれ(  )あなたの傘ですか。」\\
    B「これです。これ(  )もう10年もつかっています。」
  \item リサ「あれ、ジョンさん。どうしたんですか。目
    (  )赤いですよ。」 \\
    ジョン「ごみ(  )入ったようです。」
\end{enumerate}

% subsection renshuu_5 (end)

% section renshuu (end)

% section kakukakujoshinotsukaikata (end)

% \nobibliography*
\bibliography{references}

\end{document}
% vim:tw=80:fdm=syntax

