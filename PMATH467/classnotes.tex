\documentclass[notoc,notitlepage]{tufte-book}
% \nonstopmode % uncomment to enable nonstopmode

\usepackage{classnotetitle}

\title{PMATH467 --- Algebraic Geometry}
\author{Johnson Ng}
\subtitle{Classnotes for Winter 2019}
\credentials{BMath (Hons), Pure Mathematics major, Actuarial Science Minor}
\institution{University of Waterloo}

\setcounter{secnumdepth}{3}
\setcounter{tocdepth}{3}

\renewcommand{\baselinestretch}{1.2}
\usepackage{geometry}
\geometry{letterpaper}
\usepackage[parfill]{parskip}
\usepackage{graphicx}

% Essential Packages
\usepackage{makeidx}
\makeindex
\usepackage{enumitem}
\usepackage[T1]{fontenc}
\usepackage{natbib}
\bibliographystyle{apalike}
\usepackage{ragged2e}
\usepackage{etoolbox}
\usepackage{amssymb}
\usepackage{eso-pic}
\usepackage[fixed]{fontawesome5}
\usepackage{todonotes}
\usepackage{apptools, chngcntr}
\usepackage{amsmath}
\usepackage{mathrsfs}
\usepackage{stmaryrd}
\usepackage{mathtools}
\usepackage{tocloft}
\usepackage{tocbibind}
\usepackage{xparse}
\usepackage{tkz-euclide}
\usetkzobj{all}
\usepackage[utf8]{inputenc}
\usepackage{csquotes}
\usepackage[english]{babel}
\usepackage{marvosym}
\usepackage{pgf,tikz}
\usepackage{tikz-cd}
\usepackage{ifthen}
\usepackage{pgfplots}
\usepackage{fancyhdr}
\usepackage{array}
\usepackage{float}
\usepackage{xcolor}
\usepackage{soul}
\usepackage{centernot}
\usepackage{silence}
  \WarningFilter*{latex}{Marginpar on page \thepage\space moved}
\usepackage{tcolorbox}
\tcbuselibrary{skins,breakable}
\usepackage{longtable,booktabs}
\usepackage[amsmath,hyperref,thmmarks]{ntheorem}
\usepackage{thmtools}
\usepackage{hyperref}
\usepackage[noabbrev,capitalize,nameinlink]{cleveref}

\newcommand{\personalcolor}{false}
\ifthenelse{\equal{\personalcolor}{true}}{
  \usepackage{colorscheme-chaos}
}{
  \usepackage{colorscheme-student}
}

% hyperref Package Settings
\hypersetup{
    unicode=true,          % non-Latin characters in Acrobat’s bookmarks
    pdftoolbar=false,        % show Acrobat’s toolbar?
    pdfmenubar=false,        % show Acrobat’s menu?
    pdffitwindow=true,     % window fit to page when opened
    colorlinks=true,
    allcolors=magenta,
}

% tikz
\usepgfplotslibrary{polar}
\usepgflibrary{shapes.geometric}
\usetikzlibrary{angles,patterns,calc,decorations.markings,arrows.meta,tikzmark,bending}
\tikzset{midarrow/.style 2 args={
        decoration={markings,
            mark= at position #2 with {\arrow{#1}} ,
        },
        postaction={decorate}
    },
    midarrow/.default={latex}{0.5}
}
\def\centerarc[#1](#2)(#3:#4:#5)% Syntax: [draw options] (center) (initial angle:final angle:radius)
    { \draw[#1] ($(#2)+({#5*cos(#3)},{#5*sin(#3)})$) arc (#3:#4:#5); }
% from https://tex.stackexchange.com/questions/67573/tikz-shift-and-rotate-in-3d
\newcommand{\rotateRPY}[4][0/0/0]% point to be saved to \savedxyz, roll, pitch, yaw
{   \pgfmathsetmacro{\rollangle}{#2}
    \pgfmathsetmacro{\pitchangle}{#3}
    \pgfmathsetmacro{\yawangle}{#4}

    % to what vector is the x unit vector transformed, and which 2D vector is this?
    \pgfmathsetmacro{\newxx}{cos(\yawangle)*cos(\pitchangle)}% a
    \pgfmathsetmacro{\newxy}{sin(\yawangle)*cos(\pitchangle)}% d
    \pgfmathsetmacro{\newxz}{-sin(\pitchangle)}% g
    \path (\newxx,\newxy,\newxz);
    \pgfgetlastxy{\nxx}{\nxy};

    % to what vector is the y unit vector transformed, and which 2D vector is this?
    \pgfmathsetmacro{\newyx}{cos(\yawangle)*sin(\pitchangle)*sin(\rollangle)-sin(\yawangle)*cos(\rollangle)}% b
    \pgfmathsetmacro{\newyy}{sin(\yawangle)*sin(\pitchangle)*sin(\rollangle)+ cos(\yawangle)*cos(\rollangle)}% e
    \pgfmathsetmacro{\newyz}{cos(\pitchangle)*sin(\rollangle)}% h
    \path (\newyx,\newyy,\newyz);
    \pgfgetlastxy{\nyx}{\nyy};

    % to what vector is the z unit vector transformed, and which 2D vector is this?
    \pgfmathsetmacro{\newzx}{cos(\yawangle)*sin(\pitchangle)*cos(\rollangle)+ sin(\yawangle)*sin(\rollangle)}
    \pgfmathsetmacro{\newzy}{sin(\yawangle)*sin(\pitchangle)*cos(\rollangle)-cos(\yawangle)*sin(\rollangle)}
    \pgfmathsetmacro{\newzz}{cos(\pitchangle)*cos(\rollangle)}
    \path (\newzx,\newzy,\newzz);
    \pgfgetlastxy{\nzx}{\nzy};

    % transform the point given by #1
    \foreach \x/\y/\z in {#1}
    {   \pgfmathsetmacro{\transformedx}{\x*\newxx+\y*\newyx+\z*\newzx}
        \pgfmathsetmacro{\transformedy}{\x*\newxy+\y*\newyy+\z*\newzy}
        \pgfmathsetmacro{\transformedz}{\x*\newxz+\y*\newyz+\z*\newzz}
        \xdef\savedx{\transformedx}
        \xdef\savedy{\transformedy}
        \xdef\savedz{\transformedz}     
    }
}
\tikzset{RPY/.style={x={(\nxx,\nxy)},y={(\nyx,\nyy)},z={(\nzx,\nzy)}}}
\newcommand{\AxisRotator}[1][rotate=0]{%
    \tikz [x=0.25cm,y=0.60cm,line width=.2ex,-stealth,#1] \draw (0,0) arc (-150:150:1 and 1);%
  }

% enumitems
\newlist{inlinelist}{enumerate*}{1}
\setlist*[inlinelist,1]{%
  label=(\roman*),
}

% Theorem Style Customization
\setlength\theorempreskipamount{2ex}
\setlength\theorempostskipamount{3ex}

\makeatletter
\let\nobreakitem\item
\let\@nobreakitem\@item
\patchcmd{\nobreakitem}{\@item}{\@nobreakitem}{}{}
\patchcmd{\nobreakitem}{\@item}{\@nobreakitem}{}{}
\patchcmd{\@nobreakitem}{\@itempenalty}{\@M}{}{}
\patchcmd{\@xthm}{\ignorespaces}{\nobreak\ignorespaces}{}{}
\patchcmd{\@ythm}{\ignorespaces}{\nobreak\ignorespaces}{}{}

\renewtheoremstyle{break}%
  {\item{\theorem@headerfont
          ##1\ ##2\theorem@separator}\hskip\labelsep\relax\nobreakitem}%
  {\item{\theorem@headerfont
          ##1\ ##2\ (##3)\theorem@separator}\hskip\labelsep\relax\nobreakitem}
\makeatother

% ntheorem Declarations
\theorempreskip{10pt}
\theorempostskip{5pt}
\theoremstyle{break}

\theoremsymbol{\faComment}
\newtheorem{remark}{Remark}[section]
\theoremsymbol{}
\newtheorem*{strategy}{\faPaperPlane Strategy}
\newtheorem*{procedure}{\faCodeBranch\ }
\newtheorem{ex}{Exercise}[section]
\theorembodyfont{\normalfont}
\newtheorem*{solution}{\faPencil* Solution}
\theoremsymbol{\faGavel}
\newtheorem{eg}{Example}[section]
\theoremsymbol{}
\theorembodyfont{\it}

    % definition env
\theoremprework{\textcolor{blue}{\hrule height 2pt width \textwidth}}
\theoremheaderfont{\color{blue}\normalfont\bfseries}
\theorempostwork{\textcolor{blue}{\hrule height 2pt width \textwidth}}
\theoremindent10pt
\newtheorem{defn}{\faBook Definition}

    % definition env no num
\theoremprework{\textcolor{blue}{\hrule height 2pt width \textwidth}}
\theoremheaderfont{\color{blue}\normalfont\bfseries}
\theorempostwork{\textcolor{blue}{\hrule height 2pt width \textwidth}}
\theoremindent10pt
\newtheorem*{defnnonum}{\faBook Definition}

\theoremprework{\textcolor{blue}{\hrule height 2pt width \marginparwidth}}
\theoremheaderfont{\color{blue}\normalfont\bfseries}
\theorempostwork{\textcolor{blue}{\hrule height 2pt width \marginparwidth}}
\theoremindent10pt
\newtheorem{margindefn}[defn]{\faBook Definition}

\theoremprework{\textcolor{blue}{\hrule height 2pt width \textwidth}}
\theoremheaderfont{\color{blue}\normalfont\bfseries}
\theorempostwork{\textcolor{blue}{\hrule height 2pt width \textwidth}}
\theoremindent10pt
\newtheorem*{margindefnnonum}{\faBook Definition}

    % theorem envs
\theoremprework{\textcolor{magenta}{\hrule height 2pt width \textwidth}}
\theoremheaderfont{\color{magenta}\normalfont\bfseries}
\theorempostwork{\textcolor{magenta}{\hrule height 2pt width \textwidth}}
\theoremindent10pt
\newtheorem{thm}{\faCoffee Theorem}

\theoremprework{\textcolor{magenta}{\hrule height 2pt width \textwidth}}
\theorempostwork{\textcolor{magenta}{\hrule height 2pt width \textwidth}}
\theoremindent10pt
\newtheorem{propo}[thm]{\faTint Proposition}

\theoremprework{\textcolor{magenta}{\hrule height 2pt width \textwidth}}
\theorempostwork{\textcolor{magenta}{\hrule height 2pt width \textwidth}}
\theoremindent10pt
\newtheorem{crly}[thm]{\faSpaceShuttle Corollary}

\theoremprework{\textcolor{magenta}{\hrule height 2pt width \textwidth}}
\theorempostwork{\textcolor{magenta}{\hrule height 2pt width \textwidth}}
\theoremindent10pt
\newtheorem{lemma}[thm]{\faTree Lemma}

\theoremprework{\textcolor{magenta}{\hrule height 2pt width \textwidth}}
\theorempostwork{\textcolor{magenta}{\hrule height 2pt width \textwidth}}
\theoremindent10pt
\newtheorem{axiom}[thm]{\faShield Axiom}

    % theorem envs without counter
\theoremprework{\textcolor{magenta}{\hrule height 2pt width \textwidth}}
\theoremheaderfont{\color{magenta}\normalfont\bfseries}
\theorempostwork{\textcolor{magenta}{\hrule height 2pt width \textwidth}}
\theoremindent10pt
\newtheorem*{thmnonum}{\faCoffee Theorem}

\theoremprework{\textcolor{magenta}{\hrule height 2pt width \textwidth}}
\theorempostwork{\textcolor{magenta}{\hrule height 2pt width \textwidth}}
\theoremindent10pt
\newtheorem*{propononum}{\faTint Proposition}

\theoremprework{\textcolor{magenta}{\hrule height 2pt width \textwidth}}
\theorempostwork{\textcolor{magenta}{\hrule height 2pt width \textwidth}}
\theoremindent10pt
\newtheorem*{crlynonum}{\faSpaceShuttle Corollary}

\theoremprework{\textcolor{magenta}{\hrule height 2pt width \textwidth}}
\theorempostwork{\textcolor{magenta}{\hrule height 2pt width \textwidth}}
\theoremindent10pt
\newtheorem*{lemmanonum}{\faTree Lemma}

\theoremprework{\textcolor{magenta}{\hrule height 2pt width \textwidth}}
\theorempostwork{\textcolor{magenta}{\hrule height 2pt width \textwidth}}
\theoremindent10pt
\newtheorem*{axiomnonum}{\faShield Axiom}

    % envs on margins
\theoremprework{\textcolor{magenta}{\hrule height 2pt width \marginparwidth}}
\theoremheaderfont{\color{magenta}\normalfont\bfseries}
\theorempostwork{\textcolor{magenta}{\hrule height 2pt width \marginparwidth}}
\theoremindent10pt
\newtheorem{marginthm}[thm]{\faCoffee Theorem}

\theoremprework{\textcolor{magenta}{\hrule height 2pt width \marginparwidth}}
\theorempostwork{\textcolor{magenta}{\hrule height 2pt width \marginparwidth}}
\theoremindent10pt
\newtheorem{marginpropo}[thm]{\faTint Proposition}

\theoremprework{\textcolor{magenta}{\hrule height 2pt width \marginparwidth}}
\theorempostwork{\textcolor{magenta}{\hrule height 2pt width \marginparwidth}}
\theoremindent10pt
\newtheorem{margincrly}[thm]{\faSpaceShuttle Corollary}

\theoremprework{\textcolor{magenta}{\hrule height 2pt width \marginparwidth}}
\theorempostwork{\textcolor{magenta}{\hrule height 2pt width \marginparwidth}}
\theoremindent10pt
\newtheorem{marginlemma}[thm]{\faTree Lemma}

\theoremprework{\textcolor{magenta}{\hrule height 2pt width \marginparwidth}}
\theorempostwork{\textcolor{magenta}{\hrule height 2pt width \marginparwidth}}
\theoremindent10pt
\newtheorem{marginaxiom}[thm]{\faShield Axiom}

    % envs on margins without counter
\theoremprework{\textcolor{magenta}{\hrule height 2pt width \marginparwidth}}
\theoremheaderfont{\color{magenta}\normalfont\bfseries}
\theorempostwork{\textcolor{magenta}{\hrule height 2pt width \marginparwidth}}
\theoremindent10pt
\newtheorem*{marginthmnonum}{\faCoffee Theorem}

\theoremprework{\textcolor{magenta}{\hrule height 2pt width \marginparwidth}}
\theorempostwork{\textcolor{magenta}{\hrule height 2pt width \marginparwidth}}
\theoremindent10pt
\newtheorem*{marginpropononum}{\faTint Proposition}

\theoremprework{\textcolor{magenta}{\hrule height 2pt width \marginparwidth}}
\theorempostwork{\textcolor{magenta}{\hrule height 2pt width \marginparwidth}}
\theoremindent10pt
\newtheorem*{margincrlynonum}{\faSpaceShuttle Corollary}

\theoremprework{\textcolor{magenta}{\hrule height 2pt width \marginparwidth}}
\theorempostwork{\textcolor{magenta}{\hrule height 2pt width \marginparwidth}}
\theoremindent10pt
\newtheorem*{marginlemmanonum}{\faTree Lemma}

\theoremprework{\textcolor{magenta}{\hrule height 2pt width \marginparwidth}}
\theorempostwork{\textcolor{magenta}{\hrule height 2pt width \marginparwidth}}
\theoremindent10pt
\newtheorem*{marginaxiomnonum}{\faShield Axiom}

    % proof env
\theoremprework{\textcolor{green}{\hrule height 2pt width \textwidth}}
\theorembodyfont{\normalfont}
\theoremheaderfont{\color{green}\normalfont\bfseries}
\theorempostwork{\textcolor{green}{\hrule height 2pt width \textwidth}}
\theoremsymbol{\ensuremath{_\square}}
\newtheorem*{proof}{\faPencil* Proof}
\theoremsymbol{}

\theoremprework{\textcolor{green}{\hrule height 2pt width \marginparwidth}}
\theorembodyfont{\normalfont}
\theoremheaderfont{\color{green}\normalfont\bfseries}
\theorempostwork{\textcolor{green}{\hrule height 2pt width \marginparwidth}}
\theoremsymbol{\ensuremath{_\square}}
\newtheorem*{mproof}{\faPencil* Proof}
\theoremsymbol{}

    % note and notation env
\theorembodyfont{\it}

\theoremprework{\textcolor{yellow}{\hrule height 2pt width \textwidth}}
\theoremheaderfont{\color{yellow}\normalfont\bfseries}
\theorempostwork{\textcolor{yellow}{\hrule height 2pt width \textwidth}}
\newtheorem{note}{\faQuoteLeft Note}[section]

\theoremprework{\textcolor{yellow}{\hrule height 2pt width \marginparwidth}}
\theoremheaderfont{\color{yellow}\normalfont\bfseries}
\theorempostwork{\textcolor{yellow}{\hrule height 2pt width \marginparwidth}}
\newtheorem{mnote}[note]{\faQuoteLeft Note}

\theoremprework{\textcolor{yellow}{\hrule height 2pt width \textwidth}}
\theorempostwork{\textcolor{yellow}{\hrule height 2pt width \textwidth}}
\newtheorem*{notation}{\faPaw Notation}

    % warning env
\theoremprework{\textcolor{red}{\hrule height 2pt width \textwidth}}
\theoremheaderfont{\color{red}\normalfont\bfseries}
\theorempostwork{\textcolor{red}{\hrule height 2pt width \textwidth}}
\theoremindent10pt
\newtheorem*{warning}{\faBug Warning}

\theoremprework{\textcolor{red}{\hrule height 2pt width \marginparwidth}}
\theoremheaderfont{\color{red}\normalfont\bfseries}
\theorempostwork{\textcolor{red}{\hrule height 2pt width \marginparwidth}}
\theoremindent10pt
\newtheorem*{marginwarning}{\faBug Warning}

% rule for appendices
\AtAppendix{\counterwithin{defn}{chapter}}
\AtAppendix{\counterwithin{thm}{chapter}}
\AtAppendix{\counterwithin{propo}{chapter}}
\AtAppendix{\counterwithin{lemma}{chapter}}
\AtAppendix{\counterwithin{crly}{chapter}}
\AtAppendix{\counterwithin{axiom}{chapter}}

% more environments
\newtcolorbox{quotebox}[2]{
  blanker,enhanced,breakable,standard jigsaw,
  opacityback=0,
  coltext=\ifblank{#2}{black}{#2},
  left=5mm,right=5mm,top=2mm,bottom=2mm,
  colframe=\ifblank{#1}{bblack}{#1},
  boxrule=0pt,leftrule=3pt,
  fontupper=\itshape
}

\providecommand{\parthook}{}
\patchcmd{\part}{\thispagestyle}{\parthook\thispagestyle}{}{}
\newcommand{\partimage}[2][]{% \parthook[<options>]{<image>}
  \renewcommand{\parthook}{% Update \parthook
    \AddToShipoutPictureBG*{% Add picture to background of THIS page only
      \AtPageLowerLeft{\includegraphics[width=\paperwidth,height=\paperheight,#1]{#2}}}% Insert image
    \renewcommand{\parthook}{}}}% Restore \parthook

\AtBeginDocument{\renewcommand\contentsname{\slshape Table of Contents\normalfont}}
\cftpagenumbersoff{part}

\newcommand{\tuftepart}[1]{\newgeometry{}\part{#1}\restoregeometry}

% Heading formattings
% chapter format
\titleformat{\chapter}%
  {\huge\rmfamily\itshape\color{magenta}}% format applied to label+text
  {\llap{\colorbox{magenta}{\parbox[c][1cm]{3cm}{\hfill\itshape\Huge\textcolor{background}{\thechapter}}}}}% label
  {5pt}% horizontal separation between label and title body
  {\faLeaf}% before the title body
  []% after the title body

% section format
\titleformat{\section}%
  {\normalfont\Large\rmfamily\itshape\color{blue}}% format applied to label+text
  {\llap{\colorbox{blue}{\parbox{3cm}{\hfill\itshape\textcolor{background}{\thesection}}}}}% label
  {5pt}% horizontal separation between label and title body
  {}% before the title body
  []% after the title body

% subsection format
\titleformat{\subsection}%
  {\normalfont\large\itshape\color{green}}% format applied to label+text
  {\llap{\colorbox{green}{\parbox{3cm}{\hfill\textcolor{background}{\thesubsection}}}}}% label
  {1em}% horizontal separation between label and title body
  {}% before the title body
  []% after the title body

% subsubsection format
\titleclass{\subsubsection}{straight}
\titleformat{\subsubsection}%
  {\normalfont\large\itshape\color{yellow}}% format applied to label+text
  {\llap{\colorbox{yellow}{\parbox{3cm}{\hfill\textcolor{background}{\thesubsubsection}}}}}% label
  {1em}% horizontal separation between label and title body
  {}% before the title body
  []% after the title body

% Sidenote enhancements
\def\mathmarginnote#1{%
  \tag*{\rlap{\hspace\marginparsep\smash{\parbox[t]{\marginparwidth}{%
  \footnotesize#1}}}}
}

% Custom table columning
\newcolumntype{L}[1]{>{\raggedright\let\newline\\\arraybackslash\hspace{0pt}}m{#1}}
\newcolumntype{C}[1]{>{\centering\let\newline\\\arraybackslash\hspace{0pt}}m{#1}}
\newcolumntype{R}[1]{>{\raggedleft\let\newline\\\arraybackslash\hspace{0pt}}m{#1}}

% Graph styles
\pgfplotsset{compat=1.15}
\usepgfplotslibrary{fillbetween}
\pgfplotsset{four quads/.append style={axis x line=middle, axis y line=
middle, xlabel={$x$}, ylabel={$y$}, axis equal }}
\pgfplotsset{four quad complex/.append style={axis x line=middle, axis y line=
middle, xlabel={$\re$}, ylabel={$\im$}, axis equal }}
\def\axisdefaultwidth{360pt}
\pgfplotsset{
  tufteaxis/.append style = {thick},tick style = {thick,black},
  %
  % #1 = x, y, or z
  % #2 = the shift value
  /tikz/normal shift/.code 2 args = {%
    \pgftransformshift{%
        \pgfpointscale{#2}{\pgfplotspointouternormalvectorofticklabelaxis{#1}}%
    }%
  },%
  %
  range3frame/.style = {
    tick align        = outside,
    scaled ticks      = false,
    enlargelimits     = false,
    ticklabel shift   = {10pt},
    axis lines*       = left,
    line cap          = round,
    clip              = false,
    xtick style       = {normal shift={x}{10pt}},
    ytick style       = {normal shift={y}{10pt}},
    ztick style       = {normal shift={z}{10pt}},
    x axis line style = {normal shift={x}{10pt}},
    y axis line style = {normal shift={y}{10pt}},
    z axis line style = {normal shift={z}{10pt}},
  }
}

% Shortcuts
\DeclareMathOperator{\id}{id}
\DeclareMathOperator{\Img}{Img}
\DeclareMathOperator{\Res}{Res}
\DeclareMathOperator*{\argmax}{arg\,max}
\DeclareMathOperator*{\argmin}{arg\,min}
\DeclareMathOperator{\re}{Re}
\DeclareMathOperator{\im}{Im}
\DeclareMathOperator{\caparg}{Arg}
\DeclareMathOperator{\Char}{Char}
\DeclareMathOperator{\sgn}{sgn}
\DeclareMathOperator{\Range}{range}

\newcommand{\floor}[1]{\lfloor #1 \rfloor}      % simplifying the writing of a floor function
\newcommand{\ceiling}[1]{\lceil #1 \rceil}      % simplifying the writing of a ceiling function
\newcommand{\dotp}{\, \cdotp}			        % dot product to distinguish from \cdot
\newcommand{\abs}[1]{\left|#1\right|}						% absolute value
\newcommand{\lra}[1]{\left\langle \; #1 \; \right\rangle}
\newcommand{\at}[2]{\Big|_{#1}^{#2}}
\newcommand{\Arg}[1]{\caparg #1}
\renewcommand{\bar}[1]{\mkern 1.5mu \overline{\mkern -1.5mu #1 \mkern -1.5mu} \mkern 1.5mu}
\newcommand{\faktor}[2]{{\raisebox{.2em}{$#1$}\left/\raisebox{-.2em}{$#2$}\right.}}
\newcommand{\quotient}[2]{\faktor{#1}{#2}}
\newcommand{\cyclic}[1]{\left\langle #1 \right\rangle}
\newcommand{\ind}[2]{\Ind_{#2}\left( #1 \right)}
\newcommand{\notimply}{\centernot\implies}
\newcommand{\res}[2]{\underset{#2}{\Res} #1 }
\newcommand{\tworow}[3]{\begin{tabular}{@{}#1@{}} #2 \\ #3 \end{tabular}}
\renewcommand{\epsilon}{\varepsilon}
\renewcommand{\phi}{\varphi}
\newcommand{\lrarrow}{\leftrightarrow}
\newcommand{\larrow}{\leftarrow}
\newcommand{\rarrow}{\rightarrow}
\renewcommand{\atop}[2]{\genfrac{}{}{0pt}{}{#1}{#2}}
\newcommand*\dif{\mathop{}\!d}
\newcommand{\mmid}{\; \middle| \;}
\newcommand{\coprime}{\; \bot \;}
\newcommand{\norm}[1]{\left\| #1 \right\|}
\newenvironment{spmatrix}
  {\left(\begin{smallmatrix}}
  {\end{smallmatrix}\right)}

  % inspiration from: https://tex.stackexchange.com/questions/8720/overbrace-underbrace-but-with-an-arrow-instead#37758
\newcommand{\overarrow}[2]{
  \overset{\makebox[0pt]{\begin{tabular}{@{}c@{}}#2\\[0pt]\ensuremath{\uparrow}\end{tabular}}}{\ensuremath{#1}}
}
\newcommand{\underarrow}[2]{
  \underset{\makebox[0pt]{\begin{tabular}{@{}c@{}}\downarrow\\[0pt]\ensuremath{#2}\end{tabular}}}{\ensuremath{#1}}
}


	% highlighting shortcuts
\newcommand{\hlimpo}[1]{\textcolor{red}{\textbf{#1}}}
\newcommand{\hlwarn}[1]{\textcolor{yellow}{\textbf{#1}}}
\newcommand{\hldefn}[1]{\textcolor{blue}{\index{#1}\textbf{#1}}}
\newcommand{\hlnotea}[1]{\textcolor{green}{\textbf{#1}}}
\newcommand{\hlnoteb}[1]{\textcolor{cyan}{\textbf{#1}}}
\newcommand{\hlb}[2]{\colorbox{#1!30!background}{#2}}
\newcommand{\hlbnotea}[1]{\hlb{green}{#1}}
\newcommand{\hlbnoteb}[1]{\hlb{cyan}{#1}}
\newcommand{\hlbnotec}[1]{\hlb{yellow}{#1}}
\newcommand{\hlbnoted}[1]{\hlb{magenta}{#1}}
\newcommand{\hlbnotee}[1]{\hlb{red}{#1}}
\newcommand{\WTP}{\textcolor{bwhite}{WTP} }
\newcommand{\WTS}{\textcolor{bwhite}{WTS} }

  % stars on important stuff
\newcommand{\imponote}{\faStar}
\newcommand{\vimponote}{\faStar\faStar}
\newcommand{\vvimponote}{\faStar\faStar\faStar}

% Document header formatting
\makeatletter
\pagestyle{fancy}
\fancyhead{}
\fancyhead[RO]{\textsl{\@title} \enspace \thepage}
\fancyhead[LE]{\thepage \enspace \textsl{\leftmark \enspace \rightmark}}
\makeatother
\renewcommand{\chaptermark}[1]{\markboth{#1}{}}
\renewcommand{\sectionmark}[1]{\markright{#1}}

% Comment the two lines below if you want to print the document
\pagecolor{background}
\color{foreground}


\newcommand{\norm}[1]{\left\| #1 \right\|}
\newcommand{\Int}[1]{\overset{\circ}{#1}}

\begin{document}
\hypersetup{pageanchor=false}
\maketitle
\hypersetup{pageanchor=true}
\begin{fullwidth}
\tableofcontents
\end{fullwidth}

\newpage
\begin{fullwidth}
  \renewcommand{\listtheoremname}{\faBook\ \slshape List of Definitions}
  \listoftheorems[ignoreall,show={defn}]
\end{fullwidth}

\newpage 
\begin{fullwidth}
  \renewcommand{\listtheoremname}{\faCoffee\ \slshape List of Theorems}
  \listoftheorems[ignoreall,
    show={axiom,lemma,thm,crly,propo,marginthm,marginpropo,marginlemma,marginaxiom,margincrly}
  ]
\end{fullwidth}

\newpage
\begin{fullwidth}
  \renewcommand{\listtheoremname}{\faCodeBranch\ \slshape List of Procedures}
  \listoftheorems[ignoreall, show={procedure}]
\end{fullwidth}


\chapter*{Preface}%
\addcontentsline{toc}{chapter}{Preface}
\label{chp:preface}
% chapter preface

The basic goal of the course is to be able to find \hlnotea{algebraic invariants}, which we shall use to classify topological spaces up to homeomorphism.

Other questions that we shall also look into include a uniqueness problem about manifolds; in particular, how many manifolds exist for a given invariant up to homeomorphism? We shall see that for a \hlnotea{2-manifold}, the only such manifold is the 2-dimensional sphere $S^2$. For a $4$-manifold, it is the $4$-dimensional sphere $S^4$. In fact, for any other $n$-manifold for $n > 4$, the unique manifold is the respective $n$-sphere. The problem is trickier with the $3$-manifold, and it is known as the Poincar\'{e} Conjecture, solved in 2003 by Russian Mathematician \href{https://en.wikipedia.org/wiki/Grigori_Perelman}{Grigori Perelman}. Indeed, the said manifold is homeomorphic to the $3$-sphere.

For this course, you are expected to be familiar with notions from \hlnotea{real analysis}, such as topology, and concepts from \hlnotea{group theory}.

The following topics shall be covered:
\begin{enumerate}
  \item Point-Set Topology
  \item Introduction to Topological Manifolds
  \item Simplicial complexes \& Introduction to Homology
  \item Fundamental Groups \& Covering Spaces
  \item Classification of Surfaces
\end{enumerate}

\section*{Basic Logistics for the Course}%
% section basic_logistics_for_the_course

I shall leave this here for my own notes, in case something happens to my hard copy.

\begin{itemize}
  \item OH: (Tue) 1630 - 1800, (Fri) 1245 - 1320
  \item OR: MC 6457
  \item EM: aaleyasin
\end{itemize}

% section basic_logistics_for_the_course (end)

% chapter preface (end)

\newgeometry{letterspace}
\part{Point-Set Topology}
\restoregeometry

\chapter{Lecture 1 Jan 07th}%
\label{chp:lecture_1_jan_07th}
% chapter lecture_1_jan_07th

\section{Euclidean Space}%
\label{sec:euclidean_space}
% section euclidean_space

For any $(x_1, \ldots, x_m) \in \mathbb{R}^m$, we can measure its distance from the origin $0$ using either
\begin{itemize}
  \item $\norm{x}_\infty = \max \{ \abs{ x_i } \}$ (the supremum-norm);
  \item $\norm{x}_2 = \sqrt{ \sum (x_j)^2 }$ (the $2$-norm); or
  \item $\norm{x}_p = \left( \sum \abs{ x_j }^p \right)^\frac{1}{p}$ (the $p$-norm),
\end{itemize}
where we may define a ``distance'' by
\begin{equation*}
  d_p(x, y) = \norm{ x - y }_p.
\end{equation*}

\begin{defn}[Metric]\index{Metric}\label{defn:metric}
  Let $X$ be an arbitrary space. A function $d : X \times X \to \mathbb{R}$ is called a \hlnoteb{metric} if it satisfies
  \begin{enumerate}
    \item (symmetry) $d(x, y) = d(y, x)$ for any $x, y \in X$;
    \item (positive definiteness) $d(x, y) \geq 0$ for any $x, y \in X$, and $d(x, y) = 0 \iff x = y$; and
    \item (triangle inequality) $\forall x, y, z \in X$
      \begin{equation*}
        d(x, y) \leq d(x, z) + d(y, z).
      \end{equation*}
  \end{enumerate}
\end{defn}

\begin{defn}[Open and Closed Sets]\index{Open sets}\index{Closed sets}\label{defn:open_and_closed_sets}
  Given a space $X$ with a metric $d$, and $r > 0$, the set
  \begin{equation*}
    B(x, r) := \{ w \in X \mid d(x, w) < r \}
  \end{equation*}
  is called the \hlnoteb{open ball} of radius $r$ centered at $x$. An \hlnoteb{open set} $A$ is such that $\forall a \in A$, $\exists r > 0$ such that
  \begin{equation*}
    B(a, r) \subseteq A.
  \end{equation*}
  We say that a set is \hlnoteb{closed} if its complement is open.
\end{defn}

\begin{defn}[Continuous Map]\index{Continuous Map}\label{defn:continuous_map}
  A function
  \begin{equation*}
    f : (X, d_1) \to (Y, d_2)
  \end{equation*}
  is said to be \hlnoteb{continuous} if the preimage of an open set in $Y$ is open in $X$.\marginnote{See \href{https://tex.japorized.ink/PMATH351F18/classnotes.pdf\#thm.45}{notes} on Real Analysis for why we defined a continuous map in such a way.}
\end{defn}

\begin{warning}
  This definition does not imply that a continuous map $f$ maps open sets to open sets.
\end{warning}

\begin{ex}
  Contruct a function on $[0, 1]$ which assumes all values between its maximum and minimum, but is not continuous.
\end{ex}

\begin{solution}
  Consider the piecewise function
  \begin{equation*}
    f(x) = \begin{cases}
      x & 0 \leq x < \frac{1}{2} \\
      x - \frac{1}{2} & x \geq \frac{1}{2}.
    \end{cases}
  \end{equation*}
  It is clear that the maximum and minimum are $\frac{1}{2}$ and $0$ respectively, and $f$ assumes all values between $0$ and $\frac{1}{2}$. However, a piecewise function is not continuous.
\end{solution}

\begin{defn}[Homeomorphism]\index{Homeomorphism}\label{defn:homeomorphism}
  A function $f$ is a \hlnoteb{homeomorphism} if it is a bijection and both $f$ and $f^{-1}$ are continuous.
\end{defn}

\begin{eg}
  The function
  \begin{equation*}
    g : [ 0, 2\pi ) \to \mathbb{R}^2 \text{ given by } \theta \mapsto (\cos \theta, \sin \theta)
  \end{equation*}
  is not homeomorphic, since if we consider an alternating series that converges to $0$ on the unit circle on $\mathbb{R}^2$, we have that the preimage of the series does not converge and $f^{-1}$ is in fact discontinuous.
\end{eg}

\newthought{Now}, we want to talk about topologies without referring to a metric.

\begin{defn}[Topology]\index{Topology}\label{defn:topology}
  Let $X$ be a space. We say that the set $\mathcal{T} \subseteq \mathcal{P}(X)$ is a \hlnoteb{topology} if
  \begin{enumerate}
    \item $X, \emptyset \in \mathcal{T}$;
    \item if $\left\{ x_\alpha \right\}_{\alpha \in A} \subseteq \mathcal{T}$ for an arbitrary index set $A$, then
      \begin{equation*}
        \bigcup_{\alpha \in A} x_\alpha \in \mathcal{T}; \text{ and }
      \end{equation*}
    \item If $\{ x_\beta \}_{\beta \in B} \subset \mathcal{T}$ for some finite index set $B$, then
      \begin{equation*}
        \bigcap_{\beta \in B} x_\beta \in \mathcal{T}.
      \end{equation*}
  \end{enumerate}
\end{defn}

% section euclidean_space (end)

% chapter lecture_1_jan_07th (end)

\chapter{Lecture 2 Jan 09th}%
\label{chp:lecture_2_jan_09th}
% chapter lecture_2_jan_09th

\section{Euclidean Space (Continued)}%
\label{sec:euclidean_space_continued}
% section euclidean_space_continued

In the last lecture, from metric topology, 
we generalized the notion to a more abstract one
that is based solely on open sets.

\begin{eg}
  Let $X$ be a set. The following two are
  uninteresting examples of topologies:
  \begin{enumerate}
    \item The \hldefn{trivial topology} $\mathcal{T} = \left\{ \emptyset, X \right\}$.
    \item The \hldefn{discrete topology} $\mathcal{T} = \mathcal{P}(X)$.
  \end{enumerate}
\end{eg}

\newthought{We shall now} continue with looking at
more concepts that we shall need down the road.

\begin{defn}[Closure of a Set]\index{Closure}\label{defn:closure_of_a_set}
  Let $A$ be a set. Its \hlnoteb{closure},
  denoted as $\bar{A}$, is defined as
  \begin{equation*}
    \bar{A} = \bigcap_{C \supset A}^{C: \text{ closed }} C. 
  \end{equation*}
  It is the smallest closed set that contains $A$.
\end{defn}

\begin{note}
  In metric topology, one typically defines the closure of a set by taking the union of $A$ and its limit points.
\end{note}

\begin{defn}[Interior of a Set]\index{Interior}\label{defn:interior_of_a_set}
  Let $A$ be a set. Its \hlnoteb{interior}, 
  denoted either as $\text{Int }(A)$, $A^\circ$ or $\Int{A}$,
  is defined as
  \begin{equation*}
    \Int{A} = \bigcup_{G \subseteq A}^{G: \text{ open }} G.
  \end{equation*}
\end{defn}

\begin{defn}[Boundary of a Set]\index{Boundary}\label{defn:boundary_of_a_set}
  Let $A$ be a set. Its \hlnoteb{boundary},
  denoted as $\partial A$, is defined as
  \begin{equation*}
    \partial A = \bar{A} \setminus \Int{A}.
  \end{equation*}
\end{defn}

\begin{ex}
  Let $A$ be a set. Prove that $\partial A$ is closed.
\end{ex}

\begin{proof}
  Notice that
  \begin{equation*}
    ( \partial A )^C = ( \bar{A} \setminus \Int{A} )^C = X \setminus \bar{A} \cup \Int{A} = X \cap \bar{A}^C \cup \Int{A}
  \end{equation*}
  which is open.\qed\
\end{proof}

\begin{ex}
  Let $A$ be a set. Show that
  \begin{equation*}
    \partial ( \partial A ) = \partial A.
  \end{equation*}
\end{ex}

\begin{proof}
  First, notice that $\Int{\partial A} = \emptyset$.
  Since $\partial A$ is closed, $\bar{\parital A} = \partial A$.
  Then
  \begin{equation*}
    \partial ( \partial A ) = \bar{\partial A} \setminus \Int{\partial A}
      = \partial A \setminus \emptyset = \partial A
  \end{equation*}
\end{proof}

\begin{eg}
  We know that $\mathbb{Q} \subseteq \mathbb{R}$,
  and $\bar{\mathbb{Q}} = \mathbb{R}$.
  We say that $\mathbb{Q}$ is dense in $\mathbb{R}$.
\end{eg}

\begin{defn}[Dense]\index{Dense}\label{defn:dense}
  We say that a subset $A$ of a set $X$ is dense if
  \begin{equation*}
    \bar{A} = X.
  \end{equation*}
\end{defn}

\begin{eg}
  From the last example, we have that $\Int{\mathbb{Q}} = \emptyset$.
\end{eg}

\begin{defn}[Limit Point]\index{Limit Point}\label{defn:limit_point}
  We say that $p \in X \supseteq A$ is a limit point of $A$
  if any neighbourhood of $p$ has a nontrivial intersection
  with $A$.
\end{defn}

\begin{eg}[A Topologist's Circle]\label{eg:a_topologists_circle}
  Consider the function 
  \begin{equation*}
    f(x) = \begin{cases}
      \sin \frac{1}{x} & x \neq 0 \\
      0                & x = 0
    \end{cases}
  \end{equation*}
  on the interval $\left[ -\frac{1}{2 \pi}, \frac{1}{2 \pi} \right]$.
  Extend the function on both ends such that we obtain \cref{fig:a_topologists_circle}
  (See also: \href{https://www.desmos.com/calculator/figksl5wmv}{Desmos}).
  \begin{figure}[ht]
    \centering
    \includegraphics[width=0.7\linewidth]{topologists-circle.png}
    \caption{A Topologist's Circle}
    \label{fig:a_topologists_circle}
  \end{figure}

  The limit points of the graph includes all the points
  on the straight line from $(0, -1)$ to $(0, 1)$, including the endpoints.
  This is the case because for any of the points on this line, for any
  neighbourhood around the point, the neighbourhood intersects the
  graph $f$ infinitely many times.
\end{eg}

\newthought{Going back to continuity}, given a function $f$,
how do we know if $f^{-1}$ maps an open set to an open set?

We can actually reduce the problem to only looking at open balls.
But why are we allowed to do that?

\begin{defn}[Basis of a Topology]\index{Basis}\label{defn:basis_of_a_topology}
  \marginnote{Note that while the definition is similar to that of a cover, we are now ``covering'' over sets and not points.}
  Given a topology $\mathcal{T}$, we say that
  $\mathcal{B} = \left\{ B_{\alpha} \right\}_{\alpha \in I}$
  is a \hlnoteb{basis} if $\forall T \in \mathcal{T}$, there
  exists $J \subset I$ such that
  \begin{equation*}
    T = \bigcup_{\alpha \in J} B_\alpha.
  \end{equation*}
\end{defn}

\begin{eg}
  Let $\mathcal{T}$ be the Euclidean topology on $\mathbb{R}$.
  Then we can take
  \begin{equation*}
    \mathcal{B} = \left\{ (a, b) \mmid a, b \in \mathbb{R}, \, a \leq b \right\}.
  \end{equation*}
  Note that $\mathcal{B}$ is \hlnotea{uncountable}.
  We can, in fact, have
  \sidenote{Recall from
  \href{https://tex.japorized.ink/PMATH351F18/classnotes.pdf}{PMATH 351}
  that we can write $\mathbb{R}$ as a disjoint union 
  of open intervals with rational endpoints.}
  \begin{equation*}
    \mathcal{B}_1 = \left\{ (a, b) \mmid a, b \in \mathbb{Q}, a \leq b \right\},
  \end{equation*}
  which is countable, as a basis for $\mathbb{R}$.
  Furthermore, we can consider the set
  \begin{equation*}
    \mathcal{B}_2 = \left\{ (a, b) \mmid a \leq b, \, a = \frac{m}{2^p}, \, b = \frac{n}{2^q}, \, m, n, p, q \in \mathbb{Z} \right\},
  \end{equation*}
  which is also a countable basis for $\mathbb{R}$.
  Notice that
  \begin{equation*}
    \mathcal{B}_2 \subseteq \mathcal{B}_1 \subseteq \mathcal{B}.
  \end{equation*}
\end{eg}

\begin{eg}
  In $\mathbb{R}^2$, we can do a similar construction of $\mathcal{B}$,
  $\mathcal{B}_1$, and $\mathcal{B}_2$ as in the last example and use
  them as a basis for $\mathbb{R}^2$. In particular, we would have
  \begin{equation*}
    \mathcal{B} = \left\{ (a_1, b_1) \times (a_2, b_2) \mmid a_1, a_2, b_1, b_2 \in \mathbb{R} \right\}.
  \end{equation*}
  This is called a \hldefn{dyadic partitioning} of $\mathbb{R}^2$.
\end{eg}

\begin{eg}
  Let $(X_1, \mathcal{T}_1)$ and $(X_2, \mathcal{T}_2)$ be two topological spaces.
  Then the Cartesian product $X_1 \times X_2$ has topology induced
  from $\mathcal{T}_1$ and $\mathcal{T}_2$ by taking the set
  \begin{equation*}
    \mathcal{B} = \left\{ \beta_1 \times \beta_2 \mmid \beta_1 \in \mathcal{T}_1, \, \beta_2 \in \mathcal{T}_2 \right\}
  \end{equation*}
  as the basis.
\end{eg}

\begin{ex}
  Prove that
  \begin{enumerate}
    \item $\beta_1$ and $\beta_2$ can be taken to be elements of
      bases $\mathcal{B}_1 \subset \mathcal{T}_1$ and
      $\mathcal{B}_2 \subset \mathcal{T}_2$, respectively.
    \item the product topology on $\mathbb{R}^2$ is the same 
      as the Euclidean topology.
  \end{enumerate}
\end{ex}

% section euclidean_space_continued (end)

% chapter lecture_2_jan_09th (end)

\appendix

\backmatter

\pagestyle{plain}

\nobibliography*
\bibliography{references}

\printindex

\end{document}

