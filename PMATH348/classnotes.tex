% !TEX TS-program = lualatex
\documentclass[notoc,notitlepage]{tufte-book}
% \nonstopmode % uncomment to enable nonstopmode

\usepackage{classnotetitle}

\title{PMATH348 --- Fields and Galois Theory}
\author{Johnson Ng}
\subtitle{Classnotes for Winter 2019}
\credentials{BMath (Hons), Pure Mathematics major, Actuarial Science Minor}
\institution{University of Waterloo}

\usepackage{fontspec-luatex}
\setcounter{secnumdepth}{3}
\setcounter{tocdepth}{3}

\renewcommand{\baselinestretch}{1.1}
\usepackage{geometry}
\geometry{letterpaper}
\usepackage[parfill]{parskip}
\usepackage{graphicx}

% Essential Packages
\usepackage{makeidx}
\makeindex
\usepackage{enumitem}
\usepackage[T1]{fontenc}
\usepackage{natbib}
\bibliographystyle{apalike}
\usepackage{ragged2e}
\usepackage{etoolbox}
\usepackage{amssymb}
\usepackage{fontawesome}
\usepackage{amsmath}
\usepackage{mathrsfs}
\usepackage{mathtools}
\usepackage{xparse}
\usepackage{tkz-euclide}
\usetkzobj{all}
\usepackage[utf8]{inputenc}
\usepackage{csquotes}
\usepackage[english]{babel}
\usepackage{marvosym}
\usepackage{pgf,tikz}
\usepackage{pgfplots}
\usepackage{fancyhdr}
\usepackage{array}
\usepackage{faktor}
\usepackage{float}
\usepackage{xcolor}
\usepackage{centernot}
\usepackage{silence}
  \WarningFilter*{latex}{Marginpar on page \thepage\space moved}
\usepackage{tcolorbox}
\tcbuselibrary{skins,breakable}
\usepackage{longtable}
\usepackage[amsmath,hyperref]{ntheorem}
\usepackage{hyperref}
\usepackage[noabbrev,capitalize,nameinlink]{cleveref}

% xcolor (scheme: base16 eighties)
\definecolor{base16-eighties-dark}{HTML}{2D2D2D}
\definecolor{base16-eighties-light}{HTML}{D3D0C8}
\definecolor{base16-eighties-magenta}{HTML}{CD98CD}
\definecolor{base16-eighties-red}{HTML}{F47678}
\definecolor{base16-eighties-yellow}{HTML}{E2B552}
\definecolor{base16-eighties-green}{HTML}{98CD97}
\definecolor{base16-eighties-lightblue}{HTML}{61CCCD}
\definecolor{base16-eighties-blue}{HTML}{6498CE}
\definecolor{base16-eighties-brown}{HTML}{D47B4E}
\definecolor{base16-eighties-gray}{HTML}{747369}

% hyperref Package Settings
\hypersetup{
    bookmarks=true,         % show bookmarks bar?
    unicode=true,          % non-Latin characters in Acrobat’s bookmarks
    pdftoolbar=false,        % show Acrobat’s toolbar?
    pdfmenubar=false,        % show Acrobat’s menu?
    pdffitwindow=true,     % window fit to page when opened
    colorlinks=true,
    allcolors=base16-eighties-magenta,
}

% tikz
\usepgfplotslibrary{polar}
\usepgflibrary{shapes.geometric}
\usetikzlibrary{angles,patterns,calc,decorations.markings}
\tikzset{midarrow/.style 2 args={
        decoration={markings,
            mark= at position #2 with {\arrow{#1}} ,
        },
        postaction={decorate}
    },
    midarrow/.default={latex}{0.5}
}
\def\centerarc[#1](#2)(#3:#4:#5)% Syntax: [draw options] (center) (initial angle:final angle:radius)
    { \draw[#1] ($(#2)+({#5*cos(#3)},{#5*sin(#3)})$) arc (#3:#4:#5); }

% enumitems
\newlist{inlinelist}{enumerate*}{1}
\setlist*[inlinelist,1]{%
  label=(\roman*),
}

% Theorem Style Customization
\setlength\theorempreskipamount{2ex}
\setlength\theorempostskipamount{3ex}

\makeatletter
\let\nobreakitem\item
\let\@nobreakitem\@item
\patchcmd{\nobreakitem}{\@item}{\@nobreakitem}{}{}
\patchcmd{\nobreakitem}{\@item}{\@nobreakitem}{}{}
\patchcmd{\@nobreakitem}{\@itempenalty}{\@M}{}{}
\patchcmd{\@xthm}{\ignorespaces}{\nobreak\ignorespaces}{}{}
\patchcmd{\@ythm}{\ignorespaces}{\nobreak\ignorespaces}{}{}

\renewtheoremstyle{break}%
  {\item{\theorem@headerfont
          ##1\ ##2\theorem@separator}\hskip\labelsep\relax\nobreakitem}%
  {\item{\theorem@headerfont
          ##1\ ##2\ (##3)\theorem@separator}\hskip\labelsep\relax\nobreakitem}
\makeatother

% ntheorem + framed
\makeatletter

% ntheorem Declarations
\theorempreskip{10pt}
\theorempostskip{5pt}
\theoremstyle{break}

\newtheorem*{solution}{\faPencil $\enspace$ Solution}
\newtheorem*{remark}{Remark}
\newtheorem{eg}{Example}[section]
\newtheorem{ex}{Exercise}[section]

    % definition env
\theoremprework{\textcolor{base16-eighties-blue}{\hrule height 2pt}}
\theoremheaderfont{\color{base16-eighties-blue}\normalfont\bfseries}
\theorempostwork{\textcolor{base16-eighties-blue}{\hrule height 2pt}}
\theoremindent10pt
\newtheorem{defn}{\faBook \enspace Definition}

    % definition env no num
\theoremprework{\textcolor{base16-eighties-blue}{\hrule height 2pt}}
\theoremheaderfont{\color{base16-eighties-blue}\normalfont\bfseries}
\theorempostwork{\textcolor{base16-eighties-blue}{\hrule height 2pt}}
\theoremindent10pt
\newtheorem*{defnnonum}{\faBook \enspace Definition}

    % theorem envs
\theoremprework{\textcolor{base16-eighties-magenta}{\hrule height 2pt}}
\theoremheaderfont{\color{base16-eighties-magenta}\normalfont\bfseries}
\theorempostwork{\textcolor{base16-eighties-magenta}{\hrule height 2pt}}
\theoremindent10pt
\newtheorem{thm}{\faCoffee \enspace Theorem}

\theoremprework{\textcolor{base16-eighties-magenta}{\hrule height 2pt}}
\theorempostwork{\textcolor{base16-eighties-magenta}{\hrule height 2pt}}
\theoremindent10pt
\newtheorem{propo}[thm]{\faTint \enspace Proposition}

\theoremprework{\textcolor{base16-eighties-magenta}{\hrule height 2pt}}
\theorempostwork{\textcolor{base16-eighties-magenta}{\hrule height 2pt}}
\theoremindent10pt
\newtheorem{crly}[thm]{\faSpaceShuttle \enspace Corollary}

\theoremprework{\textcolor{base16-eighties-magenta}{\hrule height 2pt}}
\theorempostwork{\textcolor{base16-eighties-magenta}{\hrule height 2pt}}
\theoremindent10pt
\newtheorem{lemma}[thm]{\faTree \enspace Lemma}

\theoremprework{\textcolor{base16-eighties-magenta}{\hrule height 2pt}}
\theorempostwork{\textcolor{base16-eighties-magenta}{\hrule height 2pt}}
\theoremindent10pt
\newtheorem{axiom}[thm]{\faShield \enspace Axiom}

    % theorem envs without counter
\theoremprework{\textcolor{base16-eighties-magenta}{\hrule height 2pt}}
\theoremheaderfont{\color{base16-eighties-magenta}\normalfont\bfseries}
\theorempostwork{\textcolor{base16-eighties-magenta}{\hrule height 2pt}}
\theoremindent10pt
\newtheorem*{thmnonum}{\faCoffee \enspace Theorem}

\theoremprework{\textcolor{base16-eighties-magenta}{\hrule height 2pt}}
\theorempostwork{\textcolor{base16-eighties-magenta}{\hrule height 2pt}}
\theoremindent10pt
\newtheorem*{propononum}{\faTint \enspace Proposition}

\theoremprework{\textcolor{base16-eighties-magenta}{\hrule height 2pt}}
\theorempostwork{\textcolor{base16-eighties-magenta}{\hrule height 2pt}}
\theoremindent10pt
\newtheorem*{crlynonum}{\faSpaceShuttle \enspace Corollary}

\theoremprework{\textcolor{base16-eighties-magenta}{\hrule height 2pt}}
\theorempostwork{\textcolor{base16-eighties-magenta}{\hrule height 2pt}}
\theoremindent10pt
\newtheorem*{lemmanonum}{\faTree \enspace Lemma}

\theoremprework{\textcolor{base16-eighties-magenta}{\hrule height 2pt}}
\theorempostwork{\textcolor{base16-eighties-magenta}{\hrule height 2pt}}
\theoremindent10pt
\newtheorem*{axiomnonum}{\faShield \enspace Axiom}

    % proof env
\theoremprework{\textcolor{base16-eighties-brown}{\hrule height 2pt}}
\theoremheaderfont{\color{base16-eighties-brown}\normalfont\bfseries}
\theorempostwork{\textcolor{base16-eighties-brown}{\hrule height 2pt}}
\newtheorem*{proof}{\faPencil \enspace Proof}

    % note and notation env
\theoremprework{\textcolor{base16-eighties-yellow}{\hrule height 2pt}}
\theoremheaderfont{\color{base16-eighties-yellow}\normalfont\bfseries}
\theorempostwork{\textcolor{base16-eighties-yellow}{\hrule height 2pt}}
\newtheorem*{note}{\faQuoteLeft \enspace Note}

\theoremprework{\textcolor{base16-eighties-yellow}{\hrule height 2pt}}
\theorempostwork{\textcolor{base16-eighties-yellow}{\hrule height 2pt}}
\newtheorem*{notation}{\faPaw \enspace Notation}

    % warning env
\theoremprework{\textcolor{base16-eighties-red}{\hrule height 2pt}}
\theoremheaderfont{\color{base16-eighties-red}\normalfont\bfseries}
\theorempostwork{\textcolor{base16-eighties-red}{\hrule height 2pt}}
\theoremindent10pt
\newtheorem*{warning}{\faBug \enspace Warning}

% more environments
\newtcolorbox{redquote}{
  blanker,enhanced,breakable,standard jigsaw,
  opacityback=0,
  coltext=base16-eighties-light,
  left=5mm,right=5mm,top=2mm,bottom=2mm,
  colframe=base16-eighties-red,
  boxrule=0pt,leftrule=3pt,
  fontupper=\itshape
}
\newtcolorbox{bluequote}{
  blanker,enhanced,breakable,standard jigsaw,
  opacityback=0,
  coltext=base16-eighties-light,
  left=5mm,right=5mm,top=2mm,bottom=2mm,
  colframe=base16-eighties-blue,
  boxrule=0pt,leftrule=3pt,
  fontupper=\itshape
}
\newtcolorbox{greenquote}{
  blanker,enhanced,breakable,standard jigsaw,
  opacityback=0,
  coltext=base16-eighties-light,
  left=5mm,right=5mm,top=2mm,bottom=2mm,
  colframe=base16-eighties-green,
  boxrule=0pt,leftrule=3pt,
  fontupper=\itshape
}
\newtcolorbox{yellowquote}{
  blanker,enhanced,breakable,standard jigsaw,
  opacityback=0,
  coltext=base16-eighties-light,
  left=5mm,right=5mm,top=2mm,bottom=2mm,
  colframe=base16-eighties-yellow,
  boxrule=0pt,leftrule=3pt,
  fontupper=\itshape
}
\newtcolorbox{magentaquote}{
  blanker,enhanced,breakable,standard jigsaw,
  opacityback=0,
  coltext=base16-eighties-light,
  left=5mm,right=5mm,top=2mm,bottom=2mm,
  colframe=base16-eighties-magenta,
  boxrule=0pt,leftrule=3pt,
  fontupper=\itshape
}

% ntheorem listtheorem style
\makeatother
\newlength\widesttheorem
\AtBeginDocument{
  \settowidth{\widesttheorem}{Proposition A.1.1.1\quad}
}

\makeatletter
\def\thm@@thmline@name#1#2#3#4{%
        \@dottedtocline{-2}{0em}{2.3em}%
                   {\makebox[\widesttheorem][l]{#1 \protect\numberline{#2}}#3}%
                   {#4}}
\@ifpackageloaded{hyperref}{
\def\thm@@thmline@name#1#2#3#4#5{%
    \ifx\#5\%
        \@dottedtocline{-2}{0em}{2.3em}%
            {\makebox[\widesttheorem][l]{#1 \protect\numberline{#2}}#3}%
            {#4}
    \else
        \ifHy@linktocpage\relax\relax
            \@dottedtocline{-2}{0em}{2.3em}%
                {\makebox[\widesttheorem][l]{#1 \protect\numberline{#2}}#3}%
                {\hyper@linkstart{link}{#5}{#4}\hyper@linkend}%
        \else
            \@dottedtocline{-2}{0em}{2.3em}%
                {\hyper@linkstart{link}{#5}%
                  {\makebox[\widesttheorem][l]{#1 \protect\numberline{#2}}#3}\hyper@linkend}%
                    {#4}%
        \fi
    \fi}
}

\makeatletter
\def\thm@@thmline@noname#1#2#3#4{%
        \@dottedtocline{-2}{0em}{5em}%
                   {{\protect\numberline{#2}}#3}%
                   {#4}}
\@ifpackageloaded{hyperref}{
\def\thm@@thmline@noname#1#2#3#4#5{%
    \ifx\#5\%
        \@dottedtocline{-2}{0em}{5em}%
            {{\protect\numberline{#2}}#3}%
            {#4}
    \else
        \ifHy@linktocpage\relax\relax
            \@dottedtocline{-2}{0em}{5em}%
                {{\protect\numberline{#2}}#3}%
                {\hyper@linkstart{link}{#5}{#4}\hyper@linkend}%
        \else
            \@dottedtocline{-2}{0em}{5em}%
                {\hyper@linkstart{link}{#5}%
                  {{\protect\numberline{#2}}#3}\hyper@linkend}%
                    {#4}%
        \fi
    \fi}
}

\theoremlisttype{allname}

\AtBeginDocument{\renewcommand\contentsname{Table of Contents}}

% Heading formattings
% chapter format
\titleformat{\chapter}%
  {\huge\rmfamily\itshape\color{base16-eighties-magenta}}% format applied to label+text
  {\llap{\colorbox{base16-eighties-magenta}{\parbox{1.5cm}{\hfill\itshape\huge\textcolor{base16-eighties-dark}{\thechapter}}}}}% label
  {5pt}% horizontal separation between label and title body
  {}% before the title body
  []% after the title body

% section format
\titleformat{\section}%
  {\normalfont\Large\rmfamily\itshape\color{base16-eighties-blue}}% format applied to label+text
  {\llap{\colorbox{base16-eighties-blue}{\parbox{1.5cm}{\hfill\itshape\textcolor{base16-eighties-dark}{\thesection}}}}}% label
  {5pt}% horizontal separation between label and title body
  {}% before the title body
  []% after the title body

% subsection format
\titleformat{\subsection}%
  {\normalfont\large\itshape\color{base16-eighties-green}}% format applied to label+text
  {\llap{\colorbox{base16-eighties-green}{\parbox{1.5cm}{\hfill\textcolor{base16-eighties-dark}{\thesubsection}}}}}% label
  {1em}% horizontal separation between label and title body
  {}% before the title body
  []% after the title body

% Sidenote enhancements
\def\mathmarginnote#1{%
  \tag*{\rlap{\hspace\marginparsep\smash{\parbox[t]{\marginparwidth}{%
  \footnotesize#1}}}}
}

% Custom table columning
\newcolumntype{L}[1]{>{\raggedright\let\newline\\\arraybackslash\hspace{0pt}}m{#1}}
\newcolumntype{C}[1]{>{\centering\let\newline\\\arraybackslash\hspace{0pt}}m{#1}}
\newcolumntype{R}[1]{>{\raggedleft\let\newline\\\arraybackslash\hspace{0pt}}m{#1}}

% Custom math operator
% \DeclareMathOperator{\rem}{rem}
\DeclareMathOperator*{\argmax}{arg\,max}
\DeclareMathOperator*{\argmin}{arg\,min}
\DeclareMathOperator{\re}{Re}
\DeclareMathOperator{\im}{Im}
\DeclareMathOperator{\caparg}{Arg}
\DeclareMathOperator{\Ind}{Ind}
\DeclareMathOperator{\Res}{Res}

% Graph styles
\pgfplotsset{compat=1.15}
\usepgfplotslibrary{fillbetween}
\pgfplotsset{four quads/.append style={axis x line=middle, axis y line=
middle, xlabel={$x$}, ylabel={$y$}, axis equal }}
\pgfplotsset{four quad complex/.append style={axis x line=middle, axis y line=
middle, xlabel={$\re$}, ylabel={$\im$}, axis equal }}

% Shortcuts
\newcommand{\floor}[1]{\lfloor #1 \rfloor}      % simplifying the writing of a floor function
\newcommand{\ceiling}[1]{\lceil #1 \rceil}      % simplifying the writing of a ceiling function
\newcommand{\dotp}{\, \cdotp}			        % dot product to distinguish from \cdot
\newcommand{\qed}{\hfill\ensuremath{\square}}   % Q.E.D sign
\newcommand{\abs}[1]{\left|#1\right|}						% absolute value
\newcommand{\lra}[1]{\langle \; #1 \; \rangle}
\newcommand{\at}[2]{\Big|_{#1}^{#2}}
\newcommand{\Arg}[1]{\caparg #1}
\renewcommand{\bar}[1]{\mkern 1.5mu \overline{\mkern -1.5mu #1 \mkern -1.5mu} \mkern 1.5mu}
\newcommand{\quotient}[2]{\faktor{#1}{#2}}
\newcommand{\cyclic}[1]{\left\langle #1 \right\rangle}
	% highlighting shortcuts
\newcommand{\hlimpo}[1]{\textcolor{base16-eighties-red}{\textbf{#1}}}
\newcommand{\hlwarn}[1]{\textcolor{base16-eighties-yellow}{\textbf{#1}}}
\newcommand{\hldefn}[1]{\textcolor{base16-eighties-blue}{\index{#1}\textbf{#1}}}
\newcommand{\hlnotea}[1]{\textcolor{base16-eighties-green}{\textbf{#1}}}
\newcommand{\hlnoteb}[1]{\textcolor{base16-eighties-lightblue}{\textbf{#1}}}
\newcommand{\hlnotec}[1]{\textcolor{base16-eighties-brown}{\textbf{#1}}}
\newcommand{\WTP}{\textcolor{base16-eighties-brown}{WTP} }
\newcommand{\WTS}{\textcolor{base16-eighties-brown}{WTS} }
\newcommand{\ind}[2]{\Ind_{#2}\left( #1 \right)}
\newcommand{\notimply}{\centernot\implies}
\newcommand{\res}[2]{\underset{#2}{\Res} #1 }
\newcommand{\tworow}[3]{\begin{tabular}{@{}#1@{}} #2 \\ #3 \end{tabular}}
\renewcommand{\epsilon}{\varepsilon}
\newcommand{\lrarrow}{\leftrightarrow}
\newcommand{\larrow}{\leftarrow}
\newcommand{\rarrow}{\rightarrow}
\renewcommand{\atop}[2]{\genfrac{}{}{0pt}{}{#1}{#2}}
\newcommand*\dif{\mathop{}\!d}

  % inspiration from: https://tex.stackexchange.com/questions/8720/overbrace-underbrace-but-with-an-arrow-instead#37758
\newcommand{\overarrow}[2]{
  \overset{\makebox[0pt]{\begin{tabular}{@{}c@{}}#2\\[0pt]\ensuremath{\uparrow}\end{tabular}}}{#1}
}
\newcommand{\underarrow}[2]{
  \underset{\makebox[0pt]{\begin{tabular}{@{}c@{}}\downarrow\\[0pt]\ensuremath{#2}\end{tabular}}}{#1}
}

% Document header formatting
\renewcommand{\chaptermark}[1]{\markboth{#1}{}}
\renewcommand{\sectionmark}[1]{\markright{#1}}
\makeatletter
\pagestyle{fancy}
\fancyhead{}
\fancyhead[RO]{\textsl{\@title} \enspace \thepage}
\fancyhead[LE]{\thepage \enspace \textsl{\leftmark \enspace - \enspace \rightmark}}
\makeatother

% Comment the two lines below if you want to print the document
\pagecolor{base16-eighties-dark}
\color{base16-eighties-light}

\setmainfont[Ligatures=TeX]{Times Newer Roman}
\setsansfont[Ligatures=TeX]{Helvetica Neue LT Std}

\DeclareMathOperator{\Aut}{Aut}
\DeclareMathOperator{\Gal}{Gal}
\DeclareMathOperator{\stab}{stab}
\DeclareMathOperator{\orb}{orb}
\DeclareMathOperator{\ch}{ch}
\DeclareMathOperator{\Span}{span}
\DeclareMathOperator{\id}{id}
\DeclareMathOperator{\lcm}{lcm}
\DeclareMathOperator{\Char}{char}
\makeatletter
\providerobustcmd*{\bigcupdot}{%
   \mathop{%
     \mathpalette\bigop@dot\bigcup
   }%
}
\newrobustcmd*{\bigop@dot}[2]{%
   \setbox0=\hbox{$\m@th#1#2$}%
   \vbox{%
     \lineskiplimit=\maxdimen
     \lineskip=-0.7\dimexpr\ht0+\dp0\relax
     \ialign{%
       \hfil##\hfil\cr
       $\m@th\cdot$\cr
       \box0\cr
     }%
   }%
}
\makeatother

\begin{document}
\hypersetup{pageanchor=false}
\maketitle
\hypersetup{pageanchor=true}
\begin{fullwidth}
\tableofcontents
\end{fullwidth}

\newpage
\begin{fullwidth}
  \renewcommand{\listtheoremname}{\faBook\ \slshape List of Definitions}
  \listoftheorems[ignoreall,show={defn}]
  \addcontentsline{toc}{chapter}{List of Definitions}
\end{fullwidth}

\newpage 
\begin{fullwidth}
  \renewcommand{\listtheoremname}{\faCoffee\ \slshape List of Theorems}
  \listoftheorems[ignoreall,
    show={axiom,lemma,thm,crly,propo,marginthm,marginpropo,marginlemma,marginaxiom,margincrly}
  ]
  \addcontentsline{toc}{chapter}{List of Theorems}
\end{fullwidth}


\chapter*{Preface}%
\addcontentsline{toc}{chapter}{Preface}
\label{chp:preface}
% chapter preface

This is a 3 part course; it is separated into
\begin{enumerate}
  \item \textbf{Sylow's Theorem}

    which is a leftover from group theory (\href{https://tex.japorized.ink/PMATH347S18/classnotes.pdf}{PMATH 347}). It has little to do with the rest of the course, but PMATH 347 was a course that is already content-rich to a point where Sylow's Theorem gets pushed into the later course that is this course.

  \item \textbf{Field Theory}

    is a somewhat understood concept from ring theory, where we learned that it is a special case of a ring where all of its elements have an inverse.

  \item \textbf{Galois Theory}

    is the beautiful theory from the French mathematican Évariste Galois that ties field theory back to group theory. This allows us to reduce certain field theory problems into group theory, which, in some sense, is easier and better understood.
\end{enumerate}

% chapter preface (end)

\tuftepart{Sylow's Theorem}

\chapter{Lecture 1 Jan 07th}%
\label{chp:lecture_1_jan_07th}
% chapter lecture_1_jan_07th

\section{Cauchy's Theorem}%
\label{sec:cauchy_s_theorem}
% section cauchy_s_theorem

Recall Lagrange's Theorem.

\begin{thm}[Lagrange's Theorem]\index{Lagrange's Theorem}\label{thm:lagrange_s_theorem}
  If $G$ is a finite group and $H$ is a subgroup of $G$ \sidenote{I shall write this as $H \leq G$ from hereon.}, then $\abs{H} \mid \abs{G}$ \sidenote{This just means $\abs{H}$ divides $\abs{G}$.}.
\end{thm}

The full converse is not true.

\begin{eg}
  Let $G = A_4$, the \hlnotea{alternating group} of $4$ elements. Then $\abs{G} = 12$ \sidenote{Recall that the symmetric group of $4$ elements $S_4$ has order $4! = 24$, and an alternating group has half of its elements.}. We have that $6 \mid 12$. We shall show that $G$ has no subgroup of order $6$.

  Suppose to the contrary that $H \leq G$ such that $\abs{H} = 6$. Let $a \in G$ such that $\abs{a} = 3$ \sidenote{i.e. the order of $a$ is $3$. This is a \hlimpo{trick}.} There are $8$ such elements in $G$ \sidenote{This shall be left as an exercise.
  \begin{ex}
    Prove that there are $8$ elements in $G$ that have order $3$.
  \end{ex}}. Note that the \hlnotea{index}\sidenote{The index of a subgroup is the number of unique cosets generated by $H$.} of $H$, $\abs{G : H}$, is $\frac{\abs{G}}{\abs{H}} = 2$.

  Now consider the \hlnotea{cosets} $H$, $aH$ and $a^2 H$. Since $\abs{G : H} = 2$, we must have either
  \begin{itemize}
    \item $aH = H \implies a \in H$;
    \item $aH = a^H \overset{\text{`multiply' } a^{-1}}{\implies} H = aH \implies a \in H$; or
    \item $a^2H = H \overset{\text{`multiply' } a}{\implies} H = aH \implies a \in H$.
  \end{itemize}
  Thus all $8$ elements of order $3$ are in $H$ but $\abs{H} = 6$, a contradiction. Therefore, no such subgroup (of order $6$) exists.
\end{eg}

Our goal now is to establish a partial converse of Lagrange's Theorem. To that end, we shall first lay down some definitions.

\begin{defn}[$p$-Group]\index{$p$-Group}\label{defn:_p_group}
  Let $p$ be prime. We say that a group $G$ is a \hlnoteb{$p$-group} if $\abs{G} = p^k$ for some $k \in \mathbb{N}$. For $H \leq G$, we say that $H$ is a $p$-subgroup of $G$ if $H$ is a $p$-group.
\end{defn}

\begin{defn}[Sylow $p$-Subgroup]\index{Sylow $p$-Subgroup}\label{defn:sylow_p_subgroup}
  Let $G$ be a group such that $\abs{G} = p^n m$ for some $n, m \in \mathbb{N}$, such that $p \nmid m$. If $H \leq G$ with order $p^n$, we call $H$ a \hlnoteb{Sylow $p$-subgroup}.
\end{defn}

Recall Cauchy's Theorem for abelian groups\sidenote{In the course I was in, we were introduced only to the full theorem and actually went through this entire part. See notes on \href{https://tex.japorized.ink/PMATH347S18/classnotes.pdf}{PMATH 347}.}.

\begin{thm}[Cauchy's Theorem for Abelian Groups]\index{Cauchy's Theorem for Abelian Groups}\label{thm:cauchy_s_theorem_for_abelian_groups}
  If $G$ is a finite abelian group, and $p$ is prime such that $p \mid \abs{G}$, then $\abs{G}$ has an element of order $p$.
\end{thm}

\begin{defn}[Stabilizers and Orbits]\index{Stabilizers}\index{Orbits}\label{defn:stabilizers_and_orbits}
  Let $G$ be a finite group which acts on a finite set $X$ \sidenote{Recall that a group action is a function $\cdot : G \times X \to X$ such that
  \begin{enumerate}
    \item $g(hx) = (gh)x$; and
    \item $ex = x$.
  \end{enumerate}}. For $x \in X$, the \hlnoteb{stabilizers} of $x$ is the set
  \begin{equation*}
    \stab(x) := \{ g \in G : g x = x \} \leq G.
  \end{equation*}
  The orbits of $x$ is a set
  \begin{equation*}
    \orb(x) := \{ gx : g \in G \}.
  \end{equation*}
\end{defn}

\begin{note}
  One can verify that the function $G / \stab(x) \to \orb(x)$ such that
  \begin{equation*}
    g \stab(x) \mapsto gx
  \end{equation*}
  is a bijection.
\end{note}

\begin{thm}[Orbit-Stabilizer Theorem]\index{Orbit-Stabilizer Theorem}\label{thm:orbit_stabilizer_theorem}
  Let $G$ be a group acting on a set $X$, and for each $x \in X$, $\stab(x)$ and $\orb(x)$ are the stabilizers and orbits of $x$, respectively. Then
  \begin{equation*}
    \abs{G} = \abs{\stab(x)} \cdot \abs{\orb(x)}.
  \end{equation*}
  Moreover, if $x, y \in X$, then either $\orb(x) \cap \orb(y) = \emptyset$ or $\orb(x) = \orb(y)$.
\end{thm}

The theorem is actually equivalent to \href{https://tex.japorized.ink/PMATH347S18/classnotes.pdf#thm.45}{Proposition 45} in the notes for PMATH 347. However, feel free to...

\begin{ex}
  prove \cref{thm:orbit_stabilizer_theorem} as an exercise.
\end{ex}

Consequently, we have that
\begin{equation*}
  \abs{X} = \sum \abs{\orb(a_i)},
\end{equation*}
where $a_i$ are the distinct orbit representatives. Letting
\begin{equation*}
  X_G := \{ x \in X : gx = x, g \in G \},
\end{equation*}
we have...

\begin{thm}[Orbit Decomposition Theorem]\index{Orbit Decomposition Theorem}\label{thm:orbit_decomposition_theorem}
  \begin{equation*}
    \abs{X} = \abs{X_G} + \sum_{a_i \notin X_G} \abs{\orb(a_i)}.
  \end{equation*}
\end{thm}

% section cauchy_s_theorem (end)

% chapter lecture_1_jan_07th (end)

\chapter{Lecture 2 Jan 09th}%
\label{chp:lecture_2_jan_09th}
% chapter lecture_2_jan_09th

\section{Sylow Theory}%
\label{sec:sylow_theory}
% section sylow_theory

From the \hyperref[thm:orbit_decomposition_theorem]{Orbit Decomposition Theorem}, one special case is when $G$ acts on $X = G$ by conjugation.

\begin{crly}[Class Equation]\index{Class Equation}\label{crly:class_equation}
  From \cref{thm:orbit_decomposition_theorem}, if $X = G$, we have
  \begin{align*}
    \abs{G} &= \abs{Z(G)} + \sum \overarrow{\abs{ \orb(a_i) }}{\text{non-central}} \\
            &= \abs{Z(G)} + \sum [ G : \stab(a_i) ] \text{ by } \hyperref[thm:orbit_stabilizer_theorem]{Orbit-Stabilizer} \\
            &= \abs{Z(G)} + \sum [ G : C(a_i) ],
  \end{align*}
  where $C(a_i)$ is called the \hldefn{centralizers} of $G$. 
\end{crly}

\begin{thm}[First Sylow Theorem]\index{First Sylow Theorem}\label{thm:first_sylow_theorem}
  Let $G$ be a finite group, and let $p \mid \abs{G}$ such that $p$ is prime.
  Then $G$ contains a Sylow $p$-subgroup.
\end{thm}

\begin{proof}
  We proceed by induction on the size of $G$.
  If $\abs{G} = 2$, then $p = 2$, and so $G$ is its own Sylow $p$-subgroup
  \sidenote{A $2$-cycle is a Sylow $p$-group.}.
  
  Consider a finite group $G$ with $\abs{G} \geq 2$.
  Let $p$ be a prime that divides $\abs{G}$, and assume
  that the desired result holds for smaller groups.
  
  Let $\abs{G} = p^n m$, where $n, m \in \mathbb{N}$, and $p \nmid m$.

  \noindent
  \hlbnotea{Case 1: $p \mid \abs{Z(G)}$}
  By \cref{thm:cauchy_s_theorem_for_abelian_groups}, $\exists a \in Z(G)$
  such that $\abs{a} = p$.
  Since $\langle a \rangle \subsetneq Z(G)$, we have that
  \begin{equation*}
    \langle a \rangle \triangleleft G \text{ and } \abs{ \langle a \rangle } = p.
  \end{equation*}
  \sidenote{This feels like a struck of genius.
  Let's break it down and find some way that makes it easier to remember.
  We want to find $H \leq G$ such that $\abs{H} = p^n$.
  We have $\abs{ \langle a \rangle } = p$.
  We want to be able to use the \hlimpo{Correspondence Theorem},
  so we should adjust our materials to fit that mold:
  since $\abs{\langle a \rangle} = p$, notice that
  \begin{equation*}
    \frac{\abs{G}}{\abs{\langle a \rangle}} = p^{n - 1} m.
  \end{equation*}
  This is a smaller group than $G$, and so IH tells us that it
  has a Sylow $p$-subgroup, say $\bar{H}$.
  By the Correspondence Theorem, we may retrieve $H$.}
  Notice that the group $G / \langle a \rangle$ is a group
  that has a lower order than $G$, and so by IH,
  $\exists \bar{H} \leq G / \langle a \rangle$ such that $\bar{H}$
  is a Sylow $p$-subgroup of $G / \langle a \rangle$.
  Note that if $n = 1$. then $\langle a \rangle$ itself is the
  Sylow $p$-subgroup.
  WMA $n > 1$. We have that $\abs{H} = p^{n - 1}$.
  By \hyperref[thm:correspondence_theorem]{correspondence},
  \begin{equation*}
    \bar{H} = H / \langle a \rangle,
  \end{equation*}
  where $H \leq G$. By comparing the orders, we have
  \begin{equation*}
    p^{n - 1} = \frac{\abs{H}}{p} \implies \abs{H} = p^n.
  \end{equation*}
  Therefore $H$ is a Sylow $p$-subgroup of $G$.

  \noindent
  \hlbnotea{Case 2: $p \nmid Z(G)$}
  By the \hyperref[crly:class_equation]{class equation},
  notice that
  \begin{equation}\label{eq:first_sylow_theorem_eq1}
    p^n m = \abs{G} = \abs{Z(G)} + \sum [ G : C(a_i) ],
  \end{equation}
  and the summation cannot be $0$ or $p$ would otherwise
  divide $Z(G)$.
  Since $p$ divides the LHS of \cref{eq:first_sylow_theorem_eq1} and not
  $\abs{Z(G)}$, and the sum is nonzero, we must have that $\exists a_i \in G$
  such that $p \nmid [ G : C(a_i) ]$, since only then would $p \mid \abs{G}$
  \sidenote{This is after having this term `neutralizing' $\abs{G}$ so that the
    entire RHS is also divisible by $p$. If $p$ already divides everything, and
    does not divide $\abs{Z(G)}$, then $p$ would not divide $\abs{Z(G)}$.}.
  Since $p \mid \abs{G}$ but not $\abs{ G : C(a_i)}$, it must be that $p^n \mid
  \abs{C(a_i)}$ by Lagrange
  \sidenote{Having $p^n \mid \abs{C(a_i)}$ would cancel
    out all the $p$'s in $\abs{G}$, thus rendering $p$ unable to divide $\abs{G
    : C(a_i)}$.}.

  Note that we have $\abs{C(a_i)} \leq \abs{G}$.  Thus by IH, $C(a_i)$ has a
  Sylow $p$-subgroup, which is also a Sylow $p$-subgroup of $G$.
\end{proof}

\begin{crly}[Cauchy's Theorem]\index{Cauchy's Theorem}\label{crly:cauchy_s_theorem}
  If $p$ is prime and $p \mid \abs{G}$, then $G$ has an element of order $p$.
\end{crly}

\begin{proof}
  WLOG, WMA $\abs{G} = p^n m$, where $n, m \in \mathbb{N}$ and
  $p \nmid m$. By \cref{thm:first_sylow_theorem}, $\exists H \leq G$
  such that $H$ is a Sylow $p$-subgroup.
  Take $a \in H \setminus \left\{ e \right\}$. Then $\abs{a} = p^k$
  for some $k \leq n$.

  Let $b = a^{p^{k - 1}}$. Notice that $b \neq e$, or it would
  contradict the definition of an order (for $a$).
  Then $b^p = \left( a^{p^{k - 1}} \right)^p = a^p = e$.
  Therefore $\abs{b} = p$ and $b \in G$.
\end{proof}

\begin{defn}[Normalizer]\index{Normalizer}\label{defn:normalizer}
  Let $G$ be a group, and $H \leq G$. The set
  \begin{equation*}
    N_G(H) = \left\{ g \in G \mmid gHg^{-1} = H \right\}
  \end{equation*}
  is called the \hlnoteb{normalizer} of $H$ in $G$.
\end{defn}

\begin{ex}
  Verify that $N_G(H)$ is the largest subgroup of $G$
  that contains $H$ as a normal subgroup.
\end{ex}

\begin{proof}
  It is clear by definition of a normalizer that $H \triangleleft N_G(H)$.

  Suppose there exists $N_G(H) < \tilde{H} \leq G$ such that
  $H \triangleleft \tilde{H}$. Let $h \in \tilde{H} \setminus N_G(H)$.
  But since $H \triangleleft \tilde{H}$, we have
  \begin{equation*}
    hHh^{-1} = H,
  \end{equation*}
  which implies that $h \in N_G(H)$, a contradiction.
  Therefore $N_G(H)$ is the largest subgroup that contains $H$ as a
  normal subgroup.
\end{proof}

Before proceeding with the Sylow's next theorem, we require two lemmas.

\marginnote{\cref{lemma:intersection_of_a_sylow_p_subgroup_with_any_other_p_subgroups}
tells us that if we can find a $p$-subgroup $Q$ of $G$, then the elements in $Q$
that serves as the stabilizers of $P$ are precisely the elements that $Q$ shares
with $P$. This is uninteresting if $P$ is either abelian or normal, but it would
highlight what $Z(P)$ is.}
\begin{lemma}[Intersection of a Sylow $p$-subgroup with any other $p$-subgroups]\label{lemma:intersection_of_a_sylow_p_subgroup_with_any_other_p_subgroups}
  Let $G$ be a finite group and $p$ a prime such that $p \mid \abs{G}$.
  Let $P, Q \leq G$ be a Sylow $p$-subgroup and a (regular) $p$-subgroup,
  respectively. Then
  \begin{equation}\label{eq:intersection_of_a_sylow_p_subgroup_with_any_other_p_subgroups}
    Q \cap N_G(P) = Q \cap P.
  \end{equation}
\end{lemma}

\begin{proof}
  Since $P \subseteq N_G(P)$, $\subseteq$ of \cref{eq:intersection_of_a_sylow_p_subgroup_with_any_other_p_subgroups} is done.

  Let $N = N_G(P)$, and let $H = Q \cap N$. WTS $H \subseteq Q \cap P$.
  Since $H = Q \cap N \subseteq Q$, it suffices to show that $H \subseteq P$.
  Since $P$ is a Sylow $p$-subgroup, let $\abs{P} = p^n$. By Lagrange,
  we have that $\abs{H} = p^m$ for some $m \leq n$. Since $P \triangleleft N$,
  we have that $HP \leq N$ 
  \sidenote{See \href{https://tex.japorized.ink/PMATH347S18/classnotes.pdf\#thm.30}{PMATH 347}}.
  Moreover, we have that
  \begin{equation*}
    \abs{HP} = \frac{\abs{H}\abs{P}}{\abs{H \cap P}} = p^k
  \end{equation*}
  for some $k \leq n$. Also, $P \subset HP$, and so $n \leq k$,
  implying that $k = n$. Thus $P = HP$, and thus
  \begin{equation*}
    H \subseteq HP = P,
  \end{equation*}
  as required.
\end{proof}

% section sylow_theory (end)

% chapter lecture_2_jan_09th (end)

\chapter{Lecture 3 Jan 11th}%
\label{chp:lecture_3_jan_11th}
% chapter lecture_3_jan_11th

\section{Sylow Theory (Continued)}%
\label{sec:sylow_theory_continued}
% section sylow_theory_continued

\begin{lemma}[Counting The Conjugates of a Sylow $p$-Subgroup]\label{lemma:counting_the_conjugates_of_a_sylow_p_subgroup}
  Let $G$ be a finite group, and $p$ a prime such that $p \mid \abs{G}$. Let
  \begin{itemize}
    \item $P$ be a Sylow $p$-subgroup;
    \item $Q$ be a $p$-subgroup;
    \item $K = \left\{ gPg^{-1} \mmid g \in G \right\}$;
    \item $Q$ act on $K$ by conjugation; and
    \item $P = P_1, P_2, \ldots, P_r$ be the distinct orbit representatives
      from the action of $Q$ on $K$.
  \end{itemize}
  Then
  \begin{equation*}
    \abs{K} = \sum_{i=1}^{r} \left[ Q : Q \cap P_i \right].
  \end{equation*}
\end{lemma}

\begin{proof}
  From the definition of $K$, and the fact that $Q$ acts on $K$,
  we have
  \begin{align*}
    \abs{K} &= \sum_{i=1}^{r} \abs{ \orb(P_i) } \\
            &= \sum_{i=1}^{r} \abs{ Q } / \abs{ \stab(P_i) } \quad
              \hyperref[thm:orbit_stabilizer_theorem]{\text{orbit-stabilizer}}
              \\
            &= \sum_{i=1}^{r} \abs{ Q } / \abs{ N_G(P_i) \cap Q } \quad \text{
              by the action } \\
            &= \sum_{i=1}^{r} [ Q : N_G(P_i) \cap Q ] \quad \text{ by definition
              } \\
            &= \sum_{i=1}^{r} [ Q : Q \cap P_i ] \quad
            \hyperref[lemma:intersection_of_a_sylow_p_subgroup_with_any_other_p_subgroups]{\text{the
            last lemma}}.
  \end{align*}
  % TODO : Question for prof
  \sidenote{Why can we use
  \cref{lemma:intersection_of_a_sylow_p_subgroup_with_any_other_p_subgroups}?
  Are the $P_i$'s Sylow $p$-subgroups?}
\end{proof}

\begin{thm}[Second Sylow Theorem]\index{Second Sylow Theorem}\label{thm:second_sylow_theorem}
  If $P$ and $Q$ are Sylow $p$-subgroups of $G$, then
  $\exists g \in G$ such that $P = gQg^{-1}$.
\end{thm}

\begin{proof}
  Let $K = \left\{ qPq^{-1} \mmid q \in G \right\}$. WTS $Q \in K$.
  We shall also note that $\abs{P} = p^k$ for some $k \in \mathbb{N}$.

  Let $P$ act on $K$ by conjugation. Let the orbit representatives be
  \begin{equation*}
    P = P_1, P_2, \ldots, P_r.
  \end{equation*}
  By \cref{lemma:counting_the_conjugates_of_a_sylow_p_subgroup}, we have
  \begin{equation*}
    \abs{K} = \sum_{i=1}^{r} [ P : P \cap P_i ] 
            = [ P : P ] + \sum_{i=2}^{r} [ P : P \cap P_i ]
            = 1 + \sum_{i=2}^{r} [ P : P \cap P_i ].
  \end{equation*}
  Thus
  \begin{equation*}
    \abs{K} \equiv 1 \mod p.
  \end{equation*}

  Now let $Q$ act on $K$ by conjugation. Reordering if necessary, the
  orbit representatives are
  \begin{equation*}
    P = P_1, P_2, \ldots, P_s,
  \end{equation*}
  where $s$ is not necessarily $r$. From here, it suffices to show that
  $Q = P_i$ for some $i \in \left\{ 1, 2, \ldots, s \right\}$. Suppose
  not. Then by \cref{lemma:counting_the_conjugates_of_a_sylow_p_subgroup},
  \begin{equation*}
    \abs{K} = \sum_{i=1}^{s} [ Q : P_i \cap Q ].
  \end{equation*}
  Note that it must be the case that $[ Q : P_i \cap Q ] > 1$, for some if
  not all $i$, for otherwise it would imply that $Q \cap P_i$ and that would
  be a contradiction. Then by Lagrange,
  \begin{equation*}
    \abs{K} \equiv 0 \mod p.
  \end{equation*}
  This contradicts the fact that $\abs{K} \equiv 1 \mod p$.

  This shows that $Q = P_i$ for some $i \in \left\{ 1, 2, \ldots, s \right\}$,
  and so $Q$ is a conjugate of $P$.
\end{proof}

\begin{note}[Notation]
  We shall denote $n_p$ as the number of Sylow $p$-subgroups in $G$.
\end{note}

\begin{thm}[Third Sylow Theorem]\index{Third Sylow Theorem}\label{thm:third_sylow_theorem}
  Let $p$ be a prime, and that it divides $\abs{G}$, where $G$ is a group.
  Suppose $\abs{G} = p^n m$, where $n, m \in \mathbb{N}$ and $p \nmid m$.
  Then
  \begin{enumerate}
    \item $n_p \equiv 1 \mod p$; and
    \item $n_p \mid m$.
  \end{enumerate}
\end{thm}

\begin{proof}
  Let $P$ be a Sylow $p$-subgroup of $G$, and let
  \begin{equation*}
    K = \left\{ gPg^{-1} \mmid g \in G \right\}.
  \end{equation*}
  By \hyperref[thm:second_sylow_theorem]{Sylow's second theorem},
  $n_p = \abs{K}$ as all the conjugates are exactly the Sylow
  $p$-subgroups. And by our last proof, we saw that $n_p \equiv 1 \mod p$.

  Let $G$ act on $K$ by conjugation. Then by the \hyperref[thm:orbit_stabilizer_theorem]{Orbit-Stabilizer Theorem},
  \begin{equation*}
    \abs{G} = \abs{\stab(P)}\abs{\orb(P)}.
  \end{equation*}
  Thus
  \begin{equation}\label{eq:third_sylow_theorem_eq1}
    p^n m = \abs{ N_G(P) } n_p.
  \end{equation}
  Thus $n_p \mid p^n m$. Since $n_p \equiv 1 \not\equiv 0 \mod p$,
  we must have $n_p \mid m$.
\end{proof}

\begin{remark}
  \begin{enumerate}
    \item From \cref{eq:third_sylow_theorem_eq1}, we have that
      \begin{equation*}
        n_p = [ G : N_G(P) ].
      \end{equation*}
    \item \imponote\ Note that
      \begin{equation*}
        n_p = 1 \iff \forall g \in G \; gPg^{-1} = P \iff P \triangleleft G.
      \end{equation*}
      However, note that $P$ \hlimpo{may be trivial}! This means that if $G$
      is simple, it does not imply that $n_p = 1$.
  \end{enumerate}
\end{remark}

\begin{defn}[Simple Group]\index{Simple Group}\label{defn:simple_group}
  A group is said to be \hlnoteb{simple} if it has no non-trivial\sidenote{By
  non-trivial, we mean that the normal subgroup is not the group with only the
  identity element.} normal subgroups.
\end{defn}

\marginnote{
  \begin{procedures}[No simple subgroup of order $n$]\label{procedures:no_simple_subgroup_of_order_n}
    The approach to showing that there are no simple groups of a certain
    order is as follows:
    \begin{itemize}
      \item we make use of the fact that each group has a Sylow subgroup, and
        there are usually not many such subgroups;
      \item using each of the possiblities as cases, we find out if a group of
        the given order will have a normal subgroup.
    \end{itemize}   
  \end{procedures}
}
\begin{eg}
  Prove that there is no simple group of order $56$.
\end{eg}

\begin{proof}
  Let $G$ be a group.
  Note that $56 = 2^3 \cdot 7$. Then $n_7 \equiv 1 \mod 7$ and
  $n_7 \mid 8 = 2^3$. Thus
  \begin{equation*}
    n_7 = 1 \text{ or } n_7 = 8.
  \end{equation*}

  \hlbnotea{$n_7 = 1$} By the remark above, $G$ has a normal Sylow
  $7$-subgroup. Thus $G$ is not simple.

  \hlbnotea{$n_7 = 8$} By Lagrange, since $7$ is prime \sidenote{This makes use
  of the fact that the Sylow $7$-subgroup has a prime order, not just because
  $7$ itself is prime. We say this here because if the order of the Sylow
  $p$-subgroup is prime, then by \hyperref[thm:lagrange_s_theorem]{Lagrange},
  $\abs{P \cap Q}$, where $P$ and $Q$ are distinct Sylow $p$-subgroups, is a
  subgroup of $P$ (and $Q$), and must hence either be $1$ or $p$. But this
  intersection cannot have order $p$, since $P$ and $Q$ are distinct. Thus
  $\abs{P \cap Q} = 1$.

  It is also important to note that this is only true if the order of the Sylow
  $p$-subgroups are prime, i.e. simply $p$ itself. If their orders are $p^n$ for
  some $n > 1$, this is not necessarily true.
  }, the distinct Sylow $7$-subgroups of $G$ intersect trivially. Therefore,
  there are $8 \times 6 = 48$ elements of order $7$ in $G$. But this implies
  that $56 - 48 = 8$ elements that are not of order $7$.  One of them is the
  identity, thus the remaining $7$ elements must have order $2$ 
  \sidenote{They cannot be of any other order as that would create a cyclic
    group that is not of order $2$ or $7$, which is impossible.}.
  This implies that
  \begin{equation*}
    n_2 = 7 \equiv 1 \mod 2,
  \end{equation*}
  which by our remark means that $G$ has a normal Sylow $2$-subgroup.
  Thus $G$ is not simple by both accounts.
\end{proof}

% section sylow_theory_continued (end)

% chapter lecture_3_jan_11th (end)

\chapter{Lecture 4 Jan 14th}%
\label{chp:lecture_4_jan_14th}
% chapter lecture_4_jan_14th

\section{Sylow Theory (Continued 2)}%
\label{sec:sylow_theory_continued_2}
% section sylow_theory_continued_2

\begin{remark}
  \begin{enumerate}
    \item Let $p \neq q$ both be primes, and $p, q \mid \abs{ G }$. Let
      $H_p$ and $H_q$ be a Sylow $p$-subgroup and a Sylow $q$-subgroup of $G$,
      respectively. By Lagrange's Theorem, we must have that
      $H_p \cap H_q = \left\{ e \right\}$. Then
      \begin{equation*}
        \abs{ H_p \cup H_q } = \abs{ H_p } + \abs{ H_q } - 1.
      \end{equation*}

    \item Let $\abs{ G } = pm$ and $p \nmid m$, where $p$ is prime. If $H, K$ are
      Sylow $p$-subgroups of $G$ with $H \neq K$, then $H \cap K = \left\{ e \right\}$.
  \end{enumerate}
\end{remark}

\begin{eg}
  Let $G = D_6$. Notice that
  \begin{equation*}
    H = \langle 1, s \rangle, \quad K = \langle 1, rs \rangle
  \end{equation*}
  are both Sylow $2$-subgroups of $D_6$ and $H \neq K$, and their intersection is
  trivial.
\end{eg}

\begin{eg}
  Let $\abs{ G } = pq$ where $p, q$ are primes with $p < q$ and $p \nmid q - 1$.
  Then $\abs{G}$ is cyclic.
\end{eg}

\begin{proof}
  By the Third Sylow Theorem, $n_p \equiv 1 \mod p$ and $n_p \mid q$. Notice that $n_p = 1$,
  since if $n_p = q$, then $n_p \equiv 1 \mod p \implies p \mid q - 1$, contradicting our
  assumption. By our remark last lecture, $G$ has a normal Sylow $p$-subgroup, which we shall
  call $H_p$.

  On the other hand, $n_q \equiv 1 \mod q$ and $n_q \mid p$. Since $p < q$, $q \nmid p - 1$,
  and so the same argument as before holds. Hence $n_q = 1$, and so $G$ has a normal Sylow
  $q$-subgroup.

  Since $H_p \triangleleft G$, we know that $H_p H_q \leq G$, and we notice that
  \begin{equation*}
    \abs{ H_p H_q } = \frac{\abs{ H_p } \abs{ H_q }}{\abs{ H_p \cap H_q }} = pq = \abs{ G }.
  \end{equation*}
  Thus $G = H_p H_q$. Let $a, b \in G$. If $a, b$ is either both in $H_p$ or both in $H_q$,
  then $ab = ba$ \sidenote{Note: $H_p$ and $H_q$ are normal subgroups.}.
  WMA $a \in H_p$ and $b \in H_q$. By our first remark today, note that 
  $H_p \cap H_q = \left\{ e \right\}$. Then, observe that
  \begin{equation*}
    \underbrace{aba^{-1}}_{H_q} \tikzmark{a}b^{-1} \in H_q 
    \qquad \tikzmark{b}a \underbrace{ba^{-1}b^{-1}}_{H_p} \in H_p
  \end{equation*}
  \begin{tikzpicture}[remember picture,overlay]
    \draw[latex'-]
      ([shift={(1pt,-2pt)}]pic cs:a) |- ([shift={(1pt,-10pt)}]pic cs:a)
      node[anchor=north] {$H_q$};
    \draw[latex'-]
      ([shift={(1pt,-2pt)}]pic cs:b) |- ([shift={(1pt,-10pt)}]pic cs:b)
      node[anchor=north] {$H_p$};
  \end{tikzpicture}
  Thus $aba^{-1}b^{-1} = e \implies ab = ba$. So $G$ is abelian. By the
  Fundamental Theorem of Finite Abelian Groups
  \begin{equation*}
    G \simeq \mathbb{Z}_p \times \mathbb{Z}_q \simeq \mathbb{Z}_{pq},
  \end{equation*}
  which is cyclic.
\end{proof}

\begin{eg}
  By the Fundamental Theorem of Finite Abelian Groups
  \begin{equation*}
    S_3 \simeq \mathbb{Z}_2 \times \mathbb{Z}_3,
  \end{equation*}
  and $\abs{ S_3 } = 6 = 2 \cdot 3$, is not cyclic. Notice that $S_3$ does not
  fulfill the requirements for the last example since $2 \mid 3 - 1 = 2$.
\end{eg}

\begin{eg}
  If $\abs{ G } = 30$, then $G$ has a subgroup isomorphic to $\mathbb{Z}_{15}$.
  Note that $\abs{ G } = 2 \cdot 3 \cdot 5$. By the Third Sylow Theorem,
  \begin{equation*}
    n_5 \equiv 1 \mod 5 \text{ and } n_5 \mid 6 \implies n_5 = 1 \text{ or } 6
  \end{equation*}
  and
  \begin{equation*}
    n_3 \equiv 1 \mod 3 \text{ and } n_3 \mid 10 \implies n_3 = 1 \text{ or } 10.
  \end{equation*}
  Suppose $n_5 = 6$ and $n_3 = 10$. Since the Sylow $3$-subgroups and Sylow 
  $5$-subgroups intersect trivially, this accounts for 
  $(6 \times 4) + ( 10 \times 2 ) = 44$ elements but $\abs{G} = 30 < 44$.
  Thus we must have $n_5 = 1$ or $n_3 = 1$. Thus $G$ is not simple.

  Let $H_3$ and $H_5$ be Sylow $3$- and $5$-subgroups, respectively. WLOG, suppose
  $H_3 \triangleleft G$. Then $H_3 H_5 \leq G$, and notice that $\abs{H_3 H_5} = 15$.
  Since $15 = 3 \cdot 5$ and $3 \nmid 4 = 5 - 1$, we know that $H_3 H_5 \simeq \mathbb{Z}_{15}$
  by an earlier example.
\end{eg}

\begin{eg}\label{eg:g_60_is_simple}
  Let $\abs{ G } = 60$ with $n_5 > 1$. Then $G$ is simple.
\end{eg}

This is an important example for it is with this that we can prove the following:

\begin{crly}[$A_5$ is Simple]\label{crly:_a_5_is_simple}
  $A_5$ is simple.
\end{crly}

\begin{proof}
  Note that $\abs{ A_5 } = \frac{5!}{2} = 60$, and
  \begin{equation*}
    \lra{\begin{pmatrix} 1 & 2 & 3 & 4 & 5 \end{pmatrix}} 
    \text{ and } \lra{\begin{pmatrix} 1 & 3 & 2 & 4 & 5 \end{pmatrix}}
  \end{equation*}
  are both Sylow $5$-subgroups that are distinct (one has odd \hlnotea{parity} while the
  other has even).
\end{proof}

\begin{proof}[For \cref{eg:g_60_is_simple}]
  Suppose $n_5 > 1$. Notice that $60 = 2^2 \cdot 3 \cdot 5$. By \cref{thm:third_sylow_theorem},
  $n_5 \equiv 1 \mod 5$ and $n_5 \mid 12$, and thus $n_5 = 6$. This accounts for
  $6 \times 4 + 1 = 25$ elements. Now suppose $H \triangleleft G$ is proper and non-trivial.

  If $5 \mid \abs{ H }$, then $H$ contains a Sylow $5$-subgroup of $G$. Since
  $H \triangleleft G$, $H$ contains all the conjugates of this Sylow $5$-subgroup. Thus by
  our argument above, we have that $\abs{ H } \geq 25$ \sidenote{These are the $25$ elements
  that were found in the last paragraph.}. Also, $H \mid 60$. Thus it must be that
  $\abs{ H } = 30$. But then by the last example, $n_5 = 1$, a contradiction.

  So $5 \nmid \abs{ H }$. By Lagrange, it remains that
  \begin{equation*}
    \abs{ H } = 2, \, 3, \, 4, \, 6\, \text{ or } 12.
  \end{equation*}

  \hlbnoted{Case A} $\abs{ H } = 12 = 2^2 \cdot 3$.\sidenote{
  \begin{ex}
    Prove that either $n_2 = 1$ or $n_3 = 1$.
  \end{ex}}
  So $H$ contains a normal Sylow $2$- or $3$-subgroup that is normal in $G$.
\end{proof}

The proof shall be continued next lecture.

% section sylow_theory_continued_2 (end)

% chapter lecture_4_jan_14th (end)

\tuftepart{Fields}

\chapter{Lecture 5 Jan 14th}%
\label{chp:lecture_5_jan_14th}
% chapter lecture_5_jan_14th

\section{Sylow Theory (Continued 3)}%
\label{sec:sylow_theory_continued_3}
% section sylow_theory_continued_3

We shall continue with the last proof from where we left off.

\begin{proof}[\cref{eg:g_60_is_simple} continued]
  \hlbnoted{Case A} $\abs{H} = 12$. WLOG, let $K$ be a normal Sylow $3$-subgroup
  of $H$, which is also normal in $G$ \sidenote{In Sylow Theory, normality is
  transitive:
  \begin{mproof}
    If $P$ is a normal Sylow $p$-subgroup of $G$, and $Q$ is a normal subgroup
    of $P$, then $\forall q \in Q$, we have $q \in P$ and so $gqg^{-1} = q$ by
    normality of $P$. It follows that $gQg^{-1} = Q$ and so $Q$ is also normal
    in $G$.
  \end{mproof}
  }.

  \noindent
  \hlbnoted{Case B} $\abs{H} = 6$. $H$ would then have a normal Sylow $3$-subgroup,
  which is normal in $G$. We shall also call this subgroup $K$.

  \noindent
  By replacing $H$ with $K$ if necessary, wma $\abs{ H } \in \left\{ 2, 3, 4 \right\}$.
  Consider $\bar{G} = G / H$. Then $\abs{ \bar{G} } \in \left\{ 15, 20, 30 \right\}$.
  \sidenote{
  \begin{ex}
    Prove that $\bar{G}$ has a normal Sylow $5$-subgroup in all the three possible
    orders of $\bar{G}$.
  \end{ex}}
  In any case, $\bar{G}$ has a normal Sylow $5$-subgroup. Call this normal subgroup 
  $\bar{P}$. By \hyperref[thm:correspondence_theorem]{correspondence}, $\bar{P} = P / H$
  where $P$ is a normal subgroup of $G$ \sidenote{Note: correspondence works for the
  normal case as well.}. Thus $P$ is a proper non-trivial normal subgroup of $G$.
  Also,
  \begin{equation*}
    \abs{ P } = \abs{ \bar{P} } \cdot \abs{ H } = 5 \cdot \abs{ H }.
  \end{equation*}
  Thus $5 \mid \abs{ P }$, putting us back to the case where $5 \mid \abs{ H }$.
  Thus $G$ does not have a non-trivial normal subgroup, i.e. $G$ is simple.
\end{proof}

% section sylow_theory_continued_3 (end)

\section{Review of Ring Theory}%
\label{sec:review_of_ring_theory}
% section review_of_ring_theory

Let $F$ be a \hlnotea{field}, and $I$ be an \hlnotea{ideal} of $F[x]$, its
\hlnotea{polynomial ring}. Since $F[x]$ is a PID, we have $I = \langle p(x) \rangle$
for some $p(x) \in F[x]$.

Moreover, $I$ is \hlnotea{maximal} iff $p(x)$ is \hlnotea{irreducible}.

Thus we observe that
\begin{center}
  $F[x]/I$ is a field iff $I = \langle p(x) \rangle$ is maximal iff $p(x) \in F[x]$ is irreducible.
\end{center}

Therefore, to talk about fields, we need to understand irreducibles.

% section review_of_ring_theory (end)

\section{Irreducibles}%
\label{sec:irreducibles}
% section irreducibles

\begin{defn}[Irreducible]\index{Irreducible}\label{defn:irreducible}
  Let $R$ be an integral domain (ID) \sidenote{\hldefn{Integral domains} are commutative
  rings that has no zero divisors.}. We say that $f(x) \in R[x]$ is \hlnoteb{irreducible}
  (over $R$) if
  \begin{enumerate}
    \item $f(x) \neq 0$;
    \item $f(x) \notin R^\times$, where $R^\times$ is the set of units of $R$;
    \item whenever $f(x) = g(x) h(x)$, where $g(x), h(x) \in R[x]$, then either 
      $g(x) \in R^\times$ or $h(x) \in R^\times$.
  \end{enumerate}
  If $f(x) \neq 0$, $f(x) \notin R^\times$ and $f(x)$ is not irreducible, we say that
  $f(x)$ is \hldefn{reducible} (over $R$).
\end{defn}

\begin{eg}
  $f(x) = x^2 - 2$ is irreducible over $\mathbb{Q}$ but reducible over $\mathbb{R}$ as
  \begin{equation*}
    f(x) = \left(x - \sqrt{2}\right)\left(x + \sqrt{2}\right).
  \end{equation*}
\end{eg}

Let $F$ be a field, $f(x) \in F[x]$ and $a \in F$. By the \hlnotea{Division Algorithm},
we can write
\begin{equation*}
  f(x) = (x - a) q(x) + r(x),
\end{equation*}
where $q(x), r(x) \in F[x]$. Note that we either have $r(x) = 0$ or 
$\deg r < \deg (x - a) = 1$. In the latter case, $r \in F$, and so
\begin{equation*}
  f(x) = (x - a) q(x) + r.
\end{equation*}
Then $f(a) = 0 + r = r$, and so $f(x) = (x - a) q(x) + f(a)$.
\begin{equation*}
  \therefore (x - a) \mid f(x) \iff f(a) = 0.
\end{equation*}

\begin{propo}[Polynomials with Roots are Reducible]\label{propo:polynomials_with_roots_are_reducible}
  Let $F$ be a field. If $f(x) \in F[x]$ with $\deg f > 1$, and $f$ has a root in $F$,
  then $f$ is reducible (over $F$).
\end{propo}

\begin{eg}
  Let $f(x) = x^6 + x^3 + x^4 + x^3 + 3 \in \mathbb{Z}_7[x]$. Then $f(1) = 0$.
  Therefore
  \begin{equation*}
    f(x) = (x - 1) g(x) \text{ where } g(x) \in \mathbb{Z}_7[x].
  \end{equation*}
  Thus $f(x)$ is reducible over $\mathbb{Z}_7$.
\end{eg}

\begin{propo}[Irreducible Rootless Polynomials]\label{propo:irreducible_rootless_polynomials}
  Let $F$ be a field\sidenote{Note that this does not work in an ID. For example, $2x^2 + 2$.
  }. If $f(x) \in F[x]$ with $\deg f \in \{ 2, 3 \}$, then $f(x)$ is irreducible over $F$
  iff $f(x)$ has no roots in $F$.
\end{propo}

\begin{warning}
  $(x^2 + 1)^2 \in \mathbb{R}[x]$ is reducible but has no root in $\mathbb{R}$. Note that
  the degree of the polynomial is $4$.
\end{warning}

\begin{eg}
  Let $f(x) = x^3 + x + 1 \in \mathbb{Z}_2[x]$. Note that $f(0) = 1$ and 
  $f(1) = 3 \equiv 1 \mod 2$. Since $\deg f = 3$ and $f$ has no roots in $\mathbb{Z}_2$,
  $f(x)$ is irreducible over $\mathbb{Z}_2$.
\end{eg}

\begin{thm}[Gauss' Lemma]\index{Gauss' Lemma}\label{thm:gauss_lemma}
  Let $R$ be a Unique Factorization Domain (UFD), with field of fractions $F$.
  Let $p(x) \in R[x]$. If
  \begin{equation*}
    p(x) = A(x) B(x),
  \end{equation*}
  where $A(x), B(x)$ are non-constant in $F[x]$, then 
  $\exists r, s \in F^\times$ non-zero such that
  \begin{equation*}
    p(x) = a(x) b(x),
  \end{equation*}
  where $a(x) = rA(x)$ and $b(x) = sB(x)$.
\end{thm}

\begin{note}
  If $p(x) \in R[x]$ is reducible over $F$, then $p(x)$ is reducible over $R$.
\end{note}

\begin{note}
  If $R = \mathbb{Z}$ and $F = \mathbb{Q}$, then $p(x)$ is irreducible over
  $\mathbb{Z}$, then $p(x)$ is irreducible over $\mathbb{Q}$.
\end{note}

% section irreducibles (end)

% chapter lecture_5_jan_14th (end)

\chapter{Lecture 6 Jan 18th}%
\label{chp:lecture_6_jan_18th}
% chapter lecture_6_jan_18th

\section{Irreducibles (Continued)}%
\label{sec:irreducibles_continued}
% section irreducibles_continued

Our goal in this section is to develop methods to test for the irreducibility
of polynomials.

\begin{warning}
  Note that $f(x) = 2x + 4 = 2(x + 2)$ is reducible ovver $\mathbb{Z}$
  \sidenote{This is interesting over $\mathbb{Z}$, since
  $2 \notin \mathbb{Z}^\times$.} but
  irreducible over $\mathbb{Q}$.
\end{warning}

\begin{propo}[Mod-$p$ Irreducibility Test]\index{Mod-$p$ Irreducibility Test}\label{propo:mod_p_irreducibility_test}
  Let $f(x) \in \mathbb{Z}[x]$ with $\deg f \geq 1$. Let $p \in \mathbb{Z}$ be prime.
  If $\bar{f}(x)$ is the corresponding polynomial in $\mathbb{Z}_p[x]$ such that
  \begin{itemize}
    \item the coefficients of $\bar{f}(x)$ are coefficients of $f(x)$ in mod $p$,
    \item $\deg f = \deg \bar{f}$ \sidenote{This means that the leading
      coefficient of $f$ is not killed off.}, and
    \item $\bar{f}$ is irreducible over $\mathbb{Z}_p$,
  \end{itemize}
  then $f(x)$ is irreducible over $\mathbb{Q}$.
\end{propo}

\begin{proof}
  Suppose $\deg f = \deg \bar{f}$, and $\bar{f}(x) \in \mathbb{Z}_p$ is 
  irreducible over $\mathbb{Z}_p$. Suppose to the contrary that $f(x)$ is
  reducible over $\mathbb{Q}$. Then for some $g(x), h(x) \in \mathbb{Q}[x]$
  with $\deg g, \deg h < \deg f$, we have
  \begin{equation*}
    f(x) = g(x) h(x).
  \end{equation*}
  By \hyperref[thm:gauss_lemma]{Gauss' Lemma}, wma $g(x), h(x) \in \mathbb{Z}[x]$.
  Then we have
  \begin{equation*}
    \bar{f}(x) = \bar{g}(x) \bar{h}(x) \in \mathbb{Z}_p[x].
  \end{equation*}
  By assumption, $\bar{f}$ is irreducible over $\mathbb{Z}_p$, either
  \begin{equation*}
    \deg \bar{g} = 0 \text{ or } \deg \bar{h} = 0.
  \end{equation*}
  Wlog, $\deg \bar{g} = 0$. Then
  \begin{equation*}
    \deg h \leq \deg f = \deg \bar{f} = \deg \bar{h} \leq \deg h,
  \end{equation*}
  which implies that $\deg f = \deg h$ but $\deg h < \deg f$. Thus $f$ is irreducible
  over $\mathbb{Q}$.
\end{proof}

\begin{eg}
  Consider the polynomial
  \begin{equation*}
    f(x) = 3x^3 + 22x^2 + 17x + 471.
  \end{equation*}
  Then consider
  \begin{equation*}
    \bar{f}(x) = x^3 + x + 1 \in \mathbb{Z}_2[x].
  \end{equation*}
  Since $\bar{f}(0) \neq 0$ and $\bar{f}(1) \neq 0$, and $\deg f = 3$, by
  \cref{propo:irreducible_rootless_polynomials}, $\bar{f}(x)$ is irreducible over
  $\mathbb{Z}_2$. Since $\deg f = \deg \bar{f}$, $f$ is irreducible over $\mathbb{Q}$
  by the \hyperref[propo:mod_p_irreducibility_test]{Mod-$2$ irreducible test}.
\end{eg}

\begin{warning}
  Consider $f(x) = 2x^2 + x \in \mathbb{Q}[x]$, which is reducible over $\mathbb{Q}$.
  However, $\bar{f}(x) = x \in \mathbb{Z}_2[x]$ is \hlimpo{reducible} over $\mathbb{Z}_2$.
  Notice here that $\deg \bar{f} \neq \deg f$.
\end{warning}

More generally so...

\begin{propo}[Polynomials that Cannot be Factored Over the Ideals is Irreducible]\label{propo:polynomials_that_cannot_be_factored_over_the_ideals_is_irreducible}
  Let $I$ be a proper ideal of an ID $R$. Let $p(x) \in R[x]$ be monic and non-const.
  If $p(x)$ cannot be factored in $\left( R / I \right)[x]$ \sidenote{Note that
  $\left( R / I \right)$ may not be an ID even if $R$ is one.} into polynomials of
  lesser degree, then $p(x)$ is irreducible over $R$.
\end{propo}

\begin{proof}
  Sps to the contrary that $p(x)$ is reducible over $R$. Then
  \begin{equation*}
    p(x) = f(x) g(x)
  \end{equation*}
  for some $f(x), g(x) \notin R^\times$. Since $p(x)$ is monic, and $\deg f, \deg g
  < \deg p$, wma $f(x)$ and $g(x)$ are also monic. Then
  \begin{equation*}
    \bar{p}(x) = \bar{f}(x) \bar{g}(x) \in \left( R/I \right)[x].
  \end{equation*}
  Since $I \subsetneq R$, we have that $1 \notin I$, and so
  \begin{equation*}
    \deg \bar{f}, \deg \bar{g} < \deg \bar{p}
  \end{equation*}
  but that implies that $p(x)$ can be factored in $\left( R/I \right)[x]$.
\end{proof}

\begin{propo}[Eisenstein's Criterion]\index{Eisenstein's Criterion}\label{propo:eisenstein_s_criterion}
  Let $R$ be an ID. Let $P$ be a prime ideal of $R$. Let
  \begin{equation*}
    f(x) = x^n + a_{n - 1} x^{n - 1} + \hdots + a_1 x + a_0 \in R[x]
  \end{equation*}
  with $n \geq 1$. Note that $f$ is monic. Now if
  \begin{equation*}
    a_{n - 1}, a_{n - 2}, \ldots, a_1, a_0 \in P \text{ and } a_0 \notin P^2,
  \end{equation*}
  then $f$ is irreducible over $R$.
\end{propo}

\begin{proof}
  Sps to the contrary that $f$ is reducible over $R$. Since $f(x)$ is monic,
  \begin{equation*}
    f(x) = g(x) h(x)
  \end{equation*}
  where $g(x), h(x) \in R[x]$ and $\deg g, \deg h < \deg f$. Then
  \begin{equation*}
    \bar{f}(x) = \bar{g}(x) \bar{h}(x) = x^n \in (R/P)[x]
  \end{equation*}
  since $a_{n - 1}, a_{n - 2}, \ldots, a_1, a_0 \in P$.  Since $P$ is prime,
  $R/P$ is an ID, we have that either $\bar{g}(0) = 0$ or $\bar{h}(0) = 0$.
  Wlog, $\bar{g}(0) = 0 \in P$. But that implies that 
  $a_0 = \bar{g}(0)\bar{h}(0) = 0 \in P^2$, a contradiction.
\end{proof}

% section irreducibles_continued (end)

% chapter lecture_6_jan_18th (end)

\chapter{Lecture 7 Jan 21st}%
\label{chp:lecture_7_jan_21st}
% chapter lecture_7_jan_21st

\section{Irreducibles (Continued 2)}%
\label{sec:irreducibles_continued_2}
% section irreducibles_continued_2

\begin{eg}
  Prove that $f(x, y) = x^2 + y^2 - 1$ is irreducible in $\mathbb{Q}[x, y] = (\mathbb{Q}[x])[y]$.
\end{eg}

\begin{proof}
  Let $g(y) = y^2 + (x^2 + 1)$. Since $x + 1$ is irreducible, let $P = \langle x + 1 \rangle$,
  which is therefore a prime ideal of $\mathbb{Q}[x]$. Moreover, notice that
  \begin{equation*}
    x^2 - 1 = ( x + 1 )( x - 1 ) \in P.
  \end{equation*}
  Since $(x + 1)^2 \nmid \left(x^2 - 1\right)$, we have that $x^2 - 1 \notin P^2$. Then by
  \hyperref[propo:eisenstein_s_criterion]{Eisenstein}, we have that $f(x, y)$ is irreducible.
\end{proof}

\begin{crly}[Eisenstein + Gauss]\label{crly:eisenstein_gauss}
  Let $p \in \mathbb{Z}$ be a prime, and let
  \begin{equation*}
    f(x) = x^n + a_{n - 1} x^{n - 1} + \hdots + a_1 x + a_0
  \end{equation*}
  be non-const in $\mathbb{Z}[x]$. If $p \mid a_i$ for all $i \in \{0, \ldots, n - 1\}$, and
  $p^2 \nmid a_0$, then $f$ is irreducible over $\mathbb{Q}$.
\end{crly}

\marginnote{Recall that the prime ideals of $\mathbb{Z}$ are $\mathbb{Z}_p$ where $p$ is prime.}

\begin{proof}
  Let $P = \langle p \rangle$. It follows from Eisenstein that $f$ is irreducible over $\mathbb{Z}$,
  and then from Gauss that $f$ is irreducible over $\mathbb{Q}$.
\end{proof}

\begin{eg}
  Let $f(x) = x^n - d \in \mathbb{Z}[x]$ where $\exists p \in \mathbb{Z}$ prime such that
  $p^2 \nmid d$ and $p \mid d$. Let $P = \langle p \rangle$ and so by \cref{crly:eisenstein_gauss},
  $f$ is irreducible over $\mathbb{Q}$.
\end{eg}

\begin{note}
  The above example is noteworthy since it will appear rather often throughout this course. Notice
  that if we have polynomials of the above form, then we immediately have that the polynomial is
  irreducible.
\end{note}

\begin{eg}\label{eg:examples_of_irreducible_polynomials}
  Are the following irreducible over $\mathbb{Q}$?
  \begin{enumerate}
    \item $f(x) = x^7 + 21 x^5 + 15x^2 + 9x + 6$

      Yes. Notice that all the non-leading coefficients have a factor of $3$, and so if we let
      $p = 3$, since $3^2 = 9 \nmid 6$, it follows from Eisenstein that $f$ is irreducible over
      $\mathbb{Q}$.

    \item $f(x) = x^3 + 2x + 16$

      Eisenstein can't help us here since $\gcd(2, 16) = 2$ and $2^2 = 4 \mid 16$.
      Consider $\bar{f}(x) = x^3 + 2x + 1 \in \mathbb{Z}_3[x]$.  Notice that
      $\bar{f}(0) = 1 = \bar{f}(2)$ and $\bar{f}(1) = 4$. Since $\deg \bar{f} = 3$, it follows
      from \cref{propo:irreducible_rootless_polynomials} that $\bar{f}$ is irreducible over
      $\mathbb{Z}_3$. Since $\deg f = \deg \bar{f}$, it follows from the
      \hyperref[propo:mod_p_irreducibility_test]{Mod-$3$ irreducible test} that $f$ is irreducible
      over $\mathbb{Q}$.

    \item $f(x) = x^4 + 5x^3 + 6x^2 - 1$

      Again, Eisenstein can't help us here, since $5 \coprime 6 \coprime 1$ \sidenote{$\coprime$
      is a common notation for coprimeness.}. Consider
      \begin{equation*}
        \bar{f}(x) = x^4 + x^3 + 1 \in \mathbb{Z}_2[x].
      \end{equation*}
      We know that $\bar{f}(0) = 1 = \bar{f}(1)$, and so $\bar{f}$ has no roots in $\mathbb{Z}_2$.
      \sidenote{Note that we cannot use \cref{propo:irreducible_rootless_polynomials} here as
      $\deg \bar{f} = 4 > 3$.} Consider the quadratics\sidenote{\hlwarn{Why did we only check for
      the quadratics and not others?} We did so as we have already checked for the linear factors
      by checking for roots, which also checks for the cubic factors, since if we can factor out
      a linear factor, we are left with a cubic factor. Ruling out linear factors in turn rules out
      cubic factors.} of $\mathbb{Z}_2[x]$: we have
      \begin{equation*}
        x^2, \quad x^2 + x, \quad x^2 + 1, \quad x^2 + x + 1,
      \end{equation*}
      all, but the last, of which are reducible. However, notice that
      \begin{equation*}
        \left( x^2 + x + 1 \right)^2 = x^4 + x^2 + 1 \neq \bar{f}(x)
      \end{equation*}
      (by the Freshman's Dream). Thus $\bar{f}$ is irreducible in $\mathbb{Z}_2$. Since $\deg f =
      \deg \bar{f}$, by \hyperref[propo:mod_p_irreducibility_test]{Mod-$2$ irreducible test}.

    \item \label{eg:polynomial_with_prime_minus_1_degree}\imponote Let $p$ be a prime, and let
      \begin{equation*}
        f(x) = x^{p - 1} + x^{p - 2} + \hdots + x^2 + x + 1.
      \end{equation*}
      Note that $f(x)(x - 1) = x^p - 1$, and so $f(x) = \frac{x^p - 1}{x - 1}$.
      Furthermore, notice that
      \begin{align*}
        f(x + 1) &= \frac{(x + 1)^p - 1}{x} = \sum_{k = 0}^{p} \binom{p}{k} x^{p - k} - \frac{1}{x} \\
                 &= x^{p - 1} + \binom{p}{p - 1} x^{p - 2} + \hdots + \binom{p}{2} x + \binom{p}{1}.
      \end{align*}
      By setting $P = \langle p \rangle$, we have that $f(x + 1)$ is irreducible by Eisenstein.
      It follows from A3Q2 that $f(x)$ is also irreducible.
  \end{enumerate}
\end{eg}

% section irreducibles_continued_2 (end)

\section{Field Extensions}%
\label{sec:field_extensions}
% section field_extensions

Let $K$ be a field. Recall that a non-empty subset $F \subseteq K$ is called a \hldefn{subfield} of $K$
if $F$ is a field under the same operations.

\begin{eg}
  $\mathbb{Q}(\sqrt{2}) := \left\{ a + b\sqrt{2} \mmid a, b \in \mathbb{Q} \right\}$ is a subfield of
  $\mathbb{C}$. We call this field $\mathbb{Q}$ `\hldefn{adjoin}' $\sqrt{2}$.
\end{eg}

\begin{note}
  We did not actually show that $\mathbb{Q}(\sqrt{2})$ is indeed a field but note the following:
  let $a + b \sqrt{2} \neq 0 \in \mathbb{Q}(\sqrt{2})$. Then
  \begin{equation*}
    \frac{1}{a + b\sqrt{2}} \cdot \frac{(a - b\sqrt{2})}{(a - b \sqrt{2})} = \frac{a - b \sqrt{2}}{a^2 - 2b^2} \in \mathbb{Q}(\sqrt{2}),
  \end{equation*}
  and note that
  \begin{equation*}
    a^2 - 2b^2 \neq 0 \iff \frac{a}{b} = \sqrt{2},
  \end{equation*}
  which does not happen in $\mathbb{Q}$ itself.
\end{note}

\begin{defn}[Field Extension]\index{Field Extension}\label{defn:field_extension}
  Let $F$ be a field. A \hlnoteb{field extension} (or an \hlnoteb{extension}) of $F$ is a field $K$
  which contains an \hlimpo{isomorphic} copy of $F$ as a subfield. We denote this notion of $K/F$.
\end{defn}

\begin{eg}
  \begin{itemize}
    \item We have that $\mathbb{C} / \mathbb{R}$ and $\mathbb{Q}(\sqrt{2}) / \mathbb{Q}$.
    \item For a prime $p$, if
      \begin{equation*}
        \mathbb{Z}_p =(x) \left\{ \frac{f(x)}{g(x)} \mmid f(x), g(x) \in \mathbb{Z}_p[x], g \neq 0 \right\},
      \end{equation*}
      then $\mathbb{Z}_p(x) / \mathbb{Z}_p$.
    \item Let $F$ be a field, and $f(x) \in F[x]$ be irreducible. Then let $K = F[x]/\langle f(x) \rangle$.
      Then $K / F$.
  \end{itemize}
\end{eg}

\begin{note}
  Note that in the last example, $K$ is not a `direct' extension of $F$, but it contains an isomorphic
  copy of $F$. This allows us to have more flexibility in what we can do.
\end{note}

\begin{warning}
  If given $\mathbb{Z}_p = \{ 0, 1, 2, \ldots, p - 1 \}$, then $\mathbb{Q}$ is not an extension of
  $\mathbb{Z}_p$ since the two use different operations.
\end{warning}

% section field_extensions (end)

% chapter lecture_7_jan_21st (end)

\chapter{Lecture 8 Jan 23rd}%
\label{chp:lecture_8_jan_23rd}
% chapter lecture_8_jan_23rd

\section{Field Extensions (Continued)}%
\label{sec:field_extensions_continued}
% section field_extensions_continued

\begin{eg}
  Let $F$ be a field.
  \begin{itemize}
    \item If the characteristic $ch(F) = p > 0$ is a prime, then 
      \begin{equation*}
        F \supset \{ 0, 1, 2, \ldots, p - 1 \} \simeq \mathbb{Z}_p.
      \end{equation*}
      Thus $F / \mathbb{Z}_p$.
    \item If $\ch(F) = 0$, then $F / \mathbb{Q}$.
  \end{itemize}
  In either of these cases, we call $\mathbb{Z}_p$ and/or $\mathbb{Q}$ the \hldefn{prime subfield} of
  $F$.
\end{eg}

\begin{defn}[Generated Field Extension]\index{Generated Field Extension}\label{defn:generated_field_extension}
  Let $K / F$, and $\alpha_1, \ldots, \alpha_n \in K$. The \hlnoteb{field extension of $F$ generated
  by $\{a_i\}_{i = 1}^{n}$} is
  \begin{equation*}
    F(\alpha_1, \ldots, \alpha_n) 
    := \left\{ \frac{f(\alpha_1, \ldots, \alpha_n)}{g(\alpha_1, \ldots, \alpha_n)} \mmid f, g \in F[x_1, \ldots, x_n], g \neq 0 \right\},
  \end{equation*}
  of which we call as $F$ \hldefn{adjoin} $\alpha_1, \ldots, \alpha_n$.
\end{defn}

\begin{note}
  We have that $F(\alpha_1, \ldots, \alpha_n) / F$, and in turn $K / F(\alpha_1, \ldots, \alpha_n)$.
\end{note}

\begin{remark}[Minimality]\label{remark:minimality_of_an_extension}
  Let $K / F$, and $\alpha_1, \ldots, \alpha_n \in K$. If we have $E / F$ such that $K / E$ and
  $\alpha_i \in E$ for all $i$, then
  \begin{equation*}
    F(\alpha_1, \ldots, \alpha_n) \subseteq E,
  \end{equation*}
  i.e. $F(\alpha_1, \ldots, \alpha_n)$ is the smallest extension of $F$ that contains the $\alpha_i$'s.
\end{remark}

\begin{eg}[A classical example of field extensions]
  Show that $\mathbb{Q}(\sqrt{2}, \sqrt{3}) = \mathbb{Q}(\sqrt{2} + \sqrt{3})$.
\end{eg}

\begin{proof}
  Since $\sqrt{2}, \sqrt{3} \in \mathbb{Q}(\sqrt{2}, \sqrt{3})$, by closure, we have that
  $\sqrt{2} + \sqrt{3} \in \mathbb{Q}(\sqrt{2}, \sqrt{3})$, and so $\mathbb{Q}(\sqrt{2} + \sqrt{3}) 
  \subseteq \mathbb{Q}(\sqrt{2}, \sqrt{3})$.

  For the other direction, we have that $\sqrt{2} + \sqrt{3} \in \mathbb{Q}(\sqrt{2} + \sqrt{3})$.
  Then in particular $\frac{1}{\sqrt{2} + \sqrt{3}} \in \mathbb{Q}(\sqrt{2} + \sqrt{3})$. Notice that
  \begin{equation*}
    \frac{1}{\sqrt{2} + \sqrt{3}} \cdot \frac{\sqrt{2} - \sqrt{3}}{\sqrt{2} - \sqrt{3}} 
      = \sqrt{3} - \sqrt{2} \in \mathbb{Q}(\sqrt{2} + \sqrt{3}).
  \end{equation*}
  So $2 \sqrt{3}, 2 \sqrt{2} \in \mathbb{Q}(\sqrt{2} + \sqrt{3})$ \sidenote{$2 \sqrt{2}$ follows from
  a similar argument by using $1 = \frac{\sqrt{3} - \sqrt{2}}{\sqrt{3} - \sqrt{2}}$.}, and in turn
  $\sqrt{2}, \sqrt{3} \in \mathbb{Q}(\sqrt{2}, \sqrt{3})$. Then by minimality, 
  $\mathbb{Q}(\sqrt{2}, \sqrt{3}) \subseteq \mathbb{Q}(\sqrt{2} + \sqrt{3})$.
\end{proof}

\begin{remark}
  Notice that $F(\alpha, \beta) = \left[ F(\alpha) \right](\beta)$.

  We have that $F(\alpha) \subseteq F(\alpha, \beta), \beta \in F(\alpha, \beta)$, which implies that
  $F(\alpha)(\beta) \subseteq F(\alpha, \beta)$ by minimality.

  Also, since $F \subseteq F(\alpha, \beta)$, and $\alpha, \beta \in F(\alpha, \beta)$, we have, by
  minimality (again), that $F(\alpha, \beta) \subseteq F(\alpha)(\beta)$.
\end{remark}

\begin{propo}[Span of the Extension]\label{propo:span_of_the_extension}
  Let $K / F$ and $\alpha \in K$. If $\alpha$ is a root of some non-zero $f(x) \in F[x]$ irreducible
  over $F$, then $F(\alpha) \simeq F[x] / \langle f(x) \rangle$. Moreover, if $\deg f = n$, then
  \begin{equation*}
    F(\alpha) = \Span_F \{ 1, \alpha, \ldots, \alpha^{n - 1} \}.
  \end{equation*}
\end{propo}

\begin{proof}
  Sps $\alpha \in K$ is a root of an irreducible $f(x) \in F[x]$ over $F$. Let $\deg f = n \in \mathbb{N}$.
  Define $\phi : F[x] \to F(\alpha)$ by $\phi(g(x)) = g(\alpha)$. Note that this is a ring homomorphism.
  Let
  \begin{equation*}
    I = \{ g(x) \in F[x] \mid g(\alpha) = 0 \} = \ker \phi,
  \end{equation*}
  which is an ideal. Since $F[x]$ is a PID \sidenote{See 
  \href{https://tex.japorized.ink/PMATH347S18/classnotes.tex}{PMATH347}.}, $\exists g(x) \in F[x]$ such that
  $I = \langle g(x) \rangle$. Since $\alpha$ is a root of $f(x)$, $f(x) \in I$, and so $f(x) = g(x) h(x)$
  for some $h(x) \in F[x]$. Since $I \neq F[x]$ and $f$ is irreducible, $h(x) \in F^\times$. Thus
  $\langle g(x) \rangle = \langle g(x) \rangle$. Then by the \hlnotea{First Isomorphism Theorem},
  \begin{equation*}
    F[x] / \langle f(x) \rangle \simeq \phi(F[x]).
  \end{equation*}
  By construction, $\phi(F[x]) \subseteq F(\alpha)$. Since $\phi(F[x])$ is a field (by isomorphism)
  which contains $\alpha = \phi(x)$ and $F$, and so by minimality $F(\alpha) \subseteq \phi(F[x])$.
  Therefore
  \begin{equation*}
    F[x]/ \langle f(x) \rangle \simeq F(\alpha),
  \end{equation*}
  as required.

  Through the isomorphism, for any $h(x) \in F[x]$, we have
  \begin{equation*}
    h(x) + \langle f(x) \rangle \mapsto h(\alpha).
  \end{equation*}
  So
  \begin{equation*}
    F[x] / \langle f(x) \rangle 
      = \left\{ c_{n - 1} x^{n - 1} + \hdots + c_1 x + c_0 + \langle f(x) \rangle \mmid c_i \in F \right\}
  \end{equation*}
  and thus
  \begin{align*}
    F(\alpha)
      &= \left\{ c_{n - 1} \alpha^{n - 1} + \hdots + c_1 \alpha + c_0 + \mmid c_i \in F \right\} \\
      &= \Span_F \left\{ 1, \alpha, \ldots, \alpha^{n - 1} \right\},
  \end{align*}
  as claimed.
\end{proof}

% section field_extensions_continued (end)

% chapter lecture_8_jan_23rd (end)

\chapter{Lecture 9 Jan 25th}%
\label{chp:lecture_9_jan_25th}
% chapter lecture_9_jan_25th

\section{Field Extensions (Continued 2)}%
\label{sec:field_extensions_continued_2}
% section field_extensions_continued_2

Let $K / F$, and $0 \neq g(x) \in F[x]$, and $\alpha \in K$ such that $g(\alpha) = 0$. Since $F[x]$ is an
ID, $g(x)$ must have an irreducible factor $f(x) \in F[x]$ such that $f(\alpha) = 0$. By the proof of
\cref{propo:span_of_the_extension},
\begin{equation*}
  \langle f(x) \rangle = \ker \phi = I = \{ h(x) \in F[x] \mid h(\alpha) = 0 \}.
\end{equation*}
In particular,
\begin{itemize}
  \item If $h(x) \in F[x]$ such that $h(\alpha) = 0$, then $h(x) \in \langle f(x) \rangle$. In particular,
    $f(x) \mid h(x)$.
  \item $\langle f(x) \rangle$ contains a unique, monic, irreducible polynomial: for any $g(x) \in 
    \langle f(x) \rangle$ that is irreducible, we know that $g(x) = uf(x)$, where $0 \neq u \in F^\times$,
    and so we can just divide the polynomial $g$ by $u$ to make it monic.
\end{itemize}

\begin{defn}[Minimal Polynomial]\index{Minimal Polynomial}\label{defn:minimal_polynomial}
  Let $K / F$, and $\alpha \in K$ be a root of a non-zero polynomial in $F[x]$. Then there exists a
  unique irreducible monic polynomial $f(x) \in F[x]$ such that $f(\alpha) = 0$. We call this $f(x)$
  the \hlnoteb{minimal polynomial} for $\alpha$ over $F$. If $\deg f = n$, we call $n$ the \hldefn{degree}
  of $\alpha$ over $F$, denoted $\deg_F(\alpha)$.
\end{defn}

\begin{note}
  For an $\alpha \in K$, its minimal polynomial is unique, but a minimal polynomial need not have only
  one root.
\end{note}

\begin{propo}[Span of an Extension if Linearly Independent]\label{propo:span_of_an_extension_if_linearly_independent}
  Let $K / F$, and $\alpha \in K$ with minimal polynomial $f(x) \in F[x]$, with $\deg_F(\alpha) = n$. Then
  the span $F(\alpha) = \Span_F \{ 1, \alpha, \ldots, \alpha^{n - 1} \}$ is linearly independent over $F$.
\end{propo}

\begin{proof}
  Sps to the contrary that
  \begin{equation*}
    c_{n - 1} \alpha^{n - 1} + c_{n - 2} \alpha^{n - 2} + \hdots + c_1 \alpha + c_0 = 0, \; c_i \in F,
  \end{equation*}
  has a non-trivial solution, i.e. not all $c_i$'s are $0$ (i.e. we assume that the $\alpha$'s are linearly
  dependent). Consider
  \begin{equation*}
    g(x) = c_{n - 1} x^{n - 1} + \hdots + c_1 x + c_0,
  \end{equation*}
  and so $g \neq 0$. However, $g(\alpha) = 0$, so $g(x) \in \langle f(x) \rangle$, i.e. $f(x) \mid g(x)$.
  However, that contradicts the fact that $\deg f = n > n - 1 \geq \deg g$.
\end{proof}

\begin{eg}
  Consider $K / F$, and $\alpha \in K$. Then
  \begin{equation*}
    \deg_F(\alpha) = 1 \iff \text{ min. polym } f(x) = x - \alpha \in F[x] \iff \alpha \in F.
  \end{equation*}
\end{eg}

\begin{eg}
  Consider $\mathbb{Q}(\sqrt{2}) / \mathbb{Q}$. Let $\alpha = \sqrt{2}$. Note that $f(\alpha) = 0$ for
  $f(x) = x^2 - 2$, which is irreducible by Eisenstein by $P = \langle 2 \rangle$. Thus 
  $\deg_F(\alpha) = 2$, and so
  \begin{equation*}
    \mathbb{Q}(\sqrt{2}) = \Span_{\mathbb{Q}} \{ 1, \alpha \} = \{ a + b \sqrt{2} \mid a, b \in \mathbb{Q} \}.
  \end{equation*}
\end{eg}

\begin{eg}
  Let $\alpha = \sqrt{1 + \sqrt{3}}$. Notice that $\alpha^2 = 1 + \sqrt{3}$, and so $(\alpha^2 - 1)^2 = 3$.
  Thus
  \begin{equation*}
    \alpha^4 - 2 \alpha^2 + 1 - 3 = 0.
  \end{equation*}
  Let $f(x) = x^4 - 2x^2 - x \in \mathbb{Q}[x]$. Note that $f$ is monic and $f(\alpha) = 0$. By Eisenstein,
  $f$ is irreducible if we pick $P = \langle 2 \rangle$. Thus $f$ is a minimal polynomial for $\alpha$. We
  have that
  \begin{equation*}
    \deg_{\mathbb{Q}}(\alpha) = \deg f = 4.
  \end{equation*}
\end{eg}

\begin{eg}
  Let $f(x) = x^3 + x + 1 \in \mathbb{Z}_2[x]$. Let $\alpha$ be a root of $f(x)$ in some extension of
  $\mathbb{Z}_2$. Compute the size of $\mathbb{Z}_2(\alpha)$.
\end{eg}

\begin{solution}
  We showed in one of our previous examples that such an $f$ is irreducible in $\mathbb{Z}_2$. Thus
  $\deg_{\mathbb{Z}_2}(\alpha) = 3$. Then
  \begin{equation*}
    \mathbb{Z}_2(\alpha) = \Span_{\mathbb{Z}_2} \{ 1, \alpha, \alpha^2 \}, 
  \end{equation*}
  where $\{ 1, \alpha, \alpha^2 \}$ is linearly independent over $\mathbb{Z}_2$. Thus
  \begin{equation*}
    \abs{\mathbb{Z}_2(\alpha)} = 2 \times 2 \times 2 = 8.
  \end{equation*}
\end{solution}

\begin{note}
  Notice that there is no guarantee that such a root exists, but it does, which is a theorem that we shall
  prove later. (Link will be provided later) % TODO : include link to Kronecker's theorem
\end{note}

\begin{crly}[Isomorphism between Extensions]\label{crly:isomorphism_between_extensions}
  Let $K / F$ and $\alpha, \beta \in K$ have the same minimal polynomial $f(x) \in F[x]$. Then
  $F(\alpha) \simeq F(\beta)$.
\end{crly}

\begin{proof}
  From \cref{propo:span_of_the_extension}, we have that
  \begin{equation*}
    F(\alpha) \simeq F[x] / \langle f(x) \rangle \simeq F(\beta).
  \end{equation*}
\end{proof}

% section field_extensions_continued_2 (end)

% chapter lecture_9_jan_25th (end)

\chapter{Lecture 10 Jan 28th}%
\label{chp:lecture_10_jan_28th}
% chapter lecture_10_jan_28th 

\section{Field Extensions (Continued 3)}%
\label{sec:field_extensions_continued_3}
% section field_extensions_continued_3

How can we work with field extensions algebraically?

\subsection{Linear Algebra on Field Extensions}%
\label{sub:linear_algebra_on_field_extensions}
% subsection linear_algebra_on_field_extensions

We can look at $K / F$ as $K$ being an $F$-vector space.

\begin{defn}[Finite Extension]\index{Finite Extension}\label{defn:finite_extension}
  We say $K / F$ is a \hlnoteb{finite extension} if $K$ is a finite dimensional $F$-vector space.
  We call the dimension, $\dim_F K$, the \hldefn{degree} of $K / F$, and denote this dimension as
  \begin{equation*}
    [ K : F ].
  \end{equation*}
\end{defn}

\begin{eg}
  We have $[ \mathbb{C} : \mathbb{R} ] = \abs{ \{ 1, i \} } = 2$.
\end{eg}

\begin{eg}
  $[ \mathbb{R} : \mathbb{Q} ] = \infty$.
\end{eg}

\begin{eg}
  Let $K / F$ and $\alpha \in K$ with the minimal polynomial $f(x) \in F[x]$.
  Then $[ F(\alpha) : F ] = \abs{ \{ 1, \alpha, \ldots, \alpha^{n - 1} \} } = n$,
  where $n = \deg f = \deg_F(\alpha)$.\sidenote{This is why we call the dimension of $K/F$ as
  a degree.}
\end{eg}

\begin{defn}[Tower of Fields]\index{Tower of Fields}\label{defn:tower_of_fields}
  We say $F_1 / F_2 / F_3 / \hdots / F_n$ is a \hlnoteb{tower of fields} if each $F_i / F_{i + 1}$
  is a field extension.
\end{defn}

\begin{thm}[Tower Theorem]\index{Tower Theorem}\label{thm:tower_theorem}
  If $K / E$ and $E / F$ are finite extensions, then
  \begin{equation*}
    [K : F] = [K : E] [E : F].
  \end{equation*}
\end{thm}

\begin{proof}
  Let $\mathcal{B}_v = \left\{ v_1, \ldots, v_n \right\}$ be a basis for $K / E$ and
  $\mathcal{B}_w = \left\{ w_1, \ldots, w_m \right\}$ be a basis for $E / F$.

  \noindent
  \hlbnoted{Claim} The set $\left\{ v_i w_j : :1 \leq i \leq n, 1 \leq j \leq m \right\}$
  is a basis for $K / F$.

  \noindent
  \hlbnotea{Linear Independence} Assume
  \begin{equation}\label{eq:tower_of_fields_eq1}
    \sum_{i, j} c_{i, j} w_j v_i = 0.
  \end{equation}
  Notice that we may write \cref{eq:tower_of_fields_eq1} as
  \begin{equation*}
    \sum_{i} \left( \sum_{j} c_{i, j} w_j \right) v_i = 0.
  \end{equation*}
  Since $\mathcal{B}_v$ is a basis of $K / E$, for each $i$, we have
  \begin{equation*}
    \sum_{j} c_{i, j} w_j = 0.
  \end{equation*}
  Since $\mathcal{B}_w$ is a basis for $E / F$, for each $j$, we have
  \begin{equation*}
    c_{i, j} = 0.
  \end{equation*}
  It follows that the $w_j v_i$'s are linearly independent of each other.

  \noindent
  \hlbnotea{Span} Let $u \in K$. Then
  \begin{equation*}
    u = \sum_{i=1}^{n} c_i v_i,
  \end{equation*}
  where $c_i \in E$ is given by
  \begin{equation*}
    c_i = \sum_{j=1}^{m} d_{i, j} w_j.
  \end{equation*}
  Then
  \begin{equation*}
    u = \sum_{i, j} d_{i, j} w_j v_i.
  \end{equation*}
  Thus $\{ v_i, w_j \}$ is a basis for $K / F$.
\end{proof}

\begin{eg}
  Compute $[\mathbb{Q}(\sqrt[3]{5}, i) : \mathbb{Q}]$.
\end{eg}

\begin{solution}
  By the \hyperref[thm:tower_theorem]{Tower Theorem}, we have that
  \begin{equation*}
    [\mathbb{Q}(\sqrt[3]{5}, i) : \mathbb{Q}] 
      = [\mathbb{Q}(\sqrt[3]{5})(i) : \mathbb{Q}(\sqrt[3]{5})] \cdot [\mathbb{Q}(\sqrt[3]{5}) : \mathbb{Q}].
  \end{equation*}
  Notice that
  \begin{equation*}
    [\mathbb{Q}(\sqrt[3]{5}) : \mathbb{Q}] = \deg ( x^3 - 5 ) = 3.
  \end{equation*}
  For $[\mathbb{Q}(\sqrt[3]{5})(i) : \mathbb{Q}(\sqrt[3]{5})]$, let $p(x)$ be the minimal polynomial for $i$
  over $\mathbb{Q}(\sqrt[3]{5})$. Since $i^2 + 1 = 0$, we know that $i$ is a root of $x^2 + 1 = 0$. Then in
  particular, we must have $p(x) \mid x^2 + 1$. So $\deg p \in \{ 1, 2 \}$.

  Now since $\mathbb{Q}(\sqrt[3]{5}) \subseteq \mathbb{R}$ and $i \notin \mathbb{Q}(\sqrt[3]{5})$, we observe
  that $\deg p \neq 1$. Thus $\deg p = 2$. It follows that
  \begin{equation*}
    [\mathbb{Q}(\sqrt[3]{5})(i) : \mathbb{Q}(\sqrt[3]{5})] = 2.
  \end{equation*}
  Therefore
  \begin{equation*}
    [\mathbb{Q}(\sqrt[3]{5}, i) : \mathbb{Q}]  = 2 \cdot 3 = 6.
  \end{equation*}
\end{solution}

% subsection linear_algebra_on_field_extensions (end)

\subsection{Polynomials on Field Extensions}%
\label{sub:polynomials_on_field_extensions}
% subsection polynomials_on_field_extensions

\begin{defn}[Algebraic and Transcendental]\index{Algebraic}\index{Transcendental}\label{defn:algebraic_and_transcendental}
  Let $K / F$. We say that $\alpha \in K$ is \hlnoteb{algebraic} over $F$ if $\exists 0 \neq f(x) \in F[x]$
  such that $f(\alpha) = 0$. Otherwise, we say that $\alpha$ is \hlnoteb{transcendental} over $F$; that is,
  there is no non-zero polynomial over $F$ such that $\alpha$ is a root.

  We say that $K / F$ is algebraic if every $\alpha \in K$ is \hlnoteb{ algebraic } over $F$. Otherwise, we
  say that $K / F$ is \hlnoteb{transcendental}.
\end{defn}

\begin{eg}
  $\pi$ is transcendental over $\mathbb{Q}$ \sidenote{The proof of this statement is beyond our power at this
  point.}. However, $\pi$ is algebraic over $\mathbb{R}$ (note that $x - \pi \in \mathbb{R}[x]$.).
\end{eg}

\begin{eg}
  As a direct consequence of the above example, we have that $\mathbb{R}/\mathbb{Q}$ is transcendental.
\end{eg}

\begin{eg}
  As we have seen numerous times, $\mathbb{Q}(\sqrt{2}) / \mathbb{Q}$ is algebraic.
\end{eg}

\begin{remark}
  If $\alpha \in K$ is algebraic over $F$, then $\alpha$ has a minimal polynomial in $F[x]$.
\end{remark}

\begin{thm}[Finite Extensions are Algebraic]\label{thm:finite_extensions_are_algebraic}
  If $K / F$ is finite, then $K / F$ is algebraic.
\end{thm}

\begin{proof}
  \marginnote{\faLightbulb The idea is to make use of the fact that the extension will at least have
  the algebraic number as a span up to some degree $n$, and instead of working with the spanning set,
  we work with one $\alpha$ away. There will be two cases, each of which can be dealt with at relative
  ease.}
  Suppose $[K : F] = n < \infty$. Let $\alpha \in K$. Consider
  \begin{equation*}
    \alpha, \alpha^2, \ldots, \alpha^n, \alpha^{n + 1}.
  \end{equation*}

  \noindent
  \hlbnotea{Case 1} Suppose $\alpha^i = \alpha^j$ for some $i \neq j \in \{ 1, \ldots, n + 1\}$. Then
  $\alpha$ is certainly a root of $f(x) = x^i - x^j$.

  \noindent
  \hlbnotea{Case 2} Suppose $\alpha^i \neq \alpha^j$ for all $i \neq j$. Then we must have that
  \begin{equation*}
    \alpha, \alpha^2, \ldots, \alpha^n, \alpha^{n + 1}
  \end{equation*}
  is linearly dependent over $F$. Thus we may have
  \begin{equation*}
    c_1 \alpha + c_2 \alpha^2 + \hdots + c_{n + 1} \alpha^{n + 1} = 0
  \end{equation*}
  where not all $c_i$'s are $0$. Then $\alpha$ is a root of
  \begin{equation*}
    f(x) = c_{ n + 1 } x^{n + 1} \hdots + c_1 x,
  \end{equation*}
  which is a non-zero polynomial.

  In either case, we observe that $\alpha$ is algebraic over $F$. Therefore $K / F$ is algebraic.
\end{proof}

% subsection polynomials_on_field_extensions (end)

% section field_extensions_continued_3 (end)

% chapter lecture_10_jan_28th  (end)

\chapter{Lecture 11 Jan 30th}%
\label{chp:lecture_11_jan_30th}
% chapter lecture_11_jan_30th

\section{Field Extensions (Continued 4)}%
\label{sec:field_extensions_continued_4}
% section field_extensions_continued_4

\subsection{Polynomials on Field Extensions (Continued)}%
\label{sub:polynomials_on_field_extensions_continued}
% subsection polynomials_on_field_extensions_continued

\begin{note}
  Recall that given $K / F$,
  \begin{itemize}
    \item Finite (defn): $\dim_F K = [ K : F ] < \infty$
    \item Algebraic (defn) :$\forall \alpha \in K, \exists 0 \neq f \in F[x]$, such that
      $f(\alpha) = 0$
    \item Finite $\implies$ Algebraic
  \end{itemize}
\end{note}

\begin{defn}[Finitely Generated Extension]\index{Finitely Generated Extension}\label{defn:finitely_generated_extension}
  We say $K$ is a \hlnoteb{finitely generated extension} of $F$ if $\exists \alpha_1,
  \alpha_2, \ldots, \alpha_n \in K$ such that $K = F(\alpha_1, \ldots, a_n)$.
\end{defn}

\begin{propo}[Finitely Generated Algebraic Extensions are Finite]\label{propo:finitely_generated_algebraic_extensions_are_finite}
  If $K$ is a finitely generated algebraic extension of $F$, then $K / F$ is finite.\sidenote{This
  proposition is actually an \hlnotea{iff} statemnt in disguise.}
\end{propo}

\begin{proof}
  Sps $K / F$ is algebraic, where $K = F(\alpha_1, \ldots, \alpha_n)$. We shall proceed by performing
  induction on $n$. If $n = 1$, then $[F(\alpha_1) : F] = \deg_F(\alpha_1) < \infty$.

  Now suppose that the result holds for $n$. Consider 
  \begin{equation*}
    K = F(\alpha_1, \ldots, \alpha_n, \alpha_{n + 1}).
  \end{equation*}
  Then by the \hyperref[thm:tower_theorem]{Tower Theorem},
  \begin{align*}
    &[F(\alpha_1, \ldots, \alpha_n, \alpha_{n + 1}) : F] \\
    &= [F(\alpha_1, \ldots, \alpha_{n})(\alpha_{n + 1}) : F(\alpha_1, \ldots, \alpha_n)]
        \cdot [F(\alpha_1, \ldots, \alpha_n) : F].
  \end{align*}
  It follows from the base case and the induction hypothesis that
  \begin{equation*}
    [F(\alpha_1, \ldots, \alpha_{n +1}):F]
  \end{equation*}
  is finite.
\end{proof}

\begin{note}
  Finite extensions are, therefore, finitely generated.
\end{note}

\begin{eg}
  The field $\mathbb{Q}(\sqrt{2}, \sqrt{3}, \sqrt{4}, \ldots)$ is an algebraic extension of $\mathbb{Q}$
  but it is not a finite extension.
\end{eg}

\begin{propo}[Greater Algebraic Extensions]\label{propo:greater_algebraic_extensions}
  If $K / E$ and $E / F$ are algebraic extensions, then $K / F$ is an algebraic extension.
\end{propo}

\begin{proof}
  Let $\alpha \in K$. Since $K / E$ is algebraic, $\alpha$ has a minimal polynomial in $E[x]$, say it is
  \begin{equation*}
    p(x) = x^n + c_{n - 1} x^{n - 1} + \hdots + c_1 x + c_0.
  \end{equation*}
  Then $\alpha$ is algebraic over $F(c_{n - 1}, \ldots, c_1, c_0)$. By the Tower Theorem,
  \begin{equation*}
    [F(c_{n - 1}, \ldots, c_1, c_0, \alpha) : F(c_{n - 1}, \ldots, c_0)] < \infty,
  \end{equation*}
  and so $F(c_{n - 1}, \ldots, c_1, c_0, \alpha) \subseteq E$.
  
  Now $F(c_{n - 1}, \ldots, c_0) / F$ is algebraic and finitely generated. So it follows from the Tower
  Theorem that
  \begin{equation*}
    [F(c_{n - 1}, \ldots, c_0, \alpha) : F] < \infty.
  \end{equation*}
  Thus $\alpha$ is algebraic over $F$ and so $K / F$ is algebraic.
\end{proof}

\begin{propo}[Algebraic Numbers Form a Subfield]\label{propo:algebraic_numbers_form_a_subfield}
  Let $K / F$. The set of elements of $K$ algebraic over $F$ form a subfield of $K$.
\end{propo}

\begin{proof}[Sketch proof]
  Let $L = \{ \alpha \in K : \alpha \text{ is alg. over } F \}$. Let $\alpha, \beta \in L$ and $\beta \neq 0$.
  Then
  \begin{equation*}
    \alpha, \beta, \alpha + \beta, \alpha \beta, \beta^{-1} \in F(\alpha, \beta).
  \end{equation*}
  Then $[F(\alpha, \beta) : F] < \infty$ implies that $L$ is finitely generated, which is thus
  algebraic, and is hence a subfield of $K$.
\end{proof}

% subsection polynomials_on_field_extensions_continued (end)

% section field_extensions_continued_4 (end)

\section{Splitting Fields}%
\label{sec:splitting_fields}
% section splitting_fields

From various examples in the past, we notice that many of the roots that we have come across
live in $\mathbb{C}$. We shall see why later on, but we can ask ourselves if we can generalize
this notion and make use of properties from this notion.

\begin{defn}[Splits]\index{Splits}\label{defn:splits}
  Let $f(x) \in F[x]$ be non-constant. We say $f(x)$ \hlnoteb{splits} in an extension $K/F$ if
  there exists $\exists u \in F$, and $\exists \alpha_1, \ldots, \alpha_n \in K$ such that
  \begin{equation*}
    f(x) = u(x - \alpha_1) \hdots (x - \alpha_n).
  \end{equation*}
\end{defn}

\begin{eg}
  Every non-constant polynomial in $\mathbb{R}[x]$ splits in $\mathbb{C}$.
\end{eg}

\begin{thm}[Kronecker's Theorem]\index{Kronecker's Theorem}\label{thm:kronecker_s_theorem}
  Let $f(x) \in F[x]$ be non-constant. There exists an extension $K / F$ such that $f(x)$ has
  a root in $K$.
\end{thm}

\begin{proof}
  Let $f(x) \in F[x]$ be non-constant. Then let $p(x) \in F[x]$ be an irreducible factor of
  $f(x)$. Then consider $K = F[t] / <p(t)>$. which we know is a field. Then
  \begin{equation*}
    \bar{t} = t + p(t) \in K
  \end{equation*}
  is a root of $p(x)$, which means that $\bar{t}$ is also a root for $f(x)$.
\end{proof}

\begin{thm}[Repeated Kronecker's Theorem]\label{thm:repeated_kronecker_s_theorem}
  Let $f(x) \in F[x]$ be non-constant. Then there exists an extension $K / F$ such that
  $f(x)$ splits over $K$.
\end{thm}

\begin{proof}
  By the Fundamental Theorem of Algebra, if we suppose that $\deg f = n < \infty$, then
  $f$ has $n$ roots. Consequently, we need only to apply \cref{thm:kronecker_s_theorem}
  for at most $n$-many times to get to an extension where $f(x)$ splits.
\end{proof}

% section splitting_fields (end)

% chapter lecture_11_jan_30th (end)

\chapter{Lecture 12 Feb 01st}%
\label{chp:lecture_12_feb_01st}
% chapter lecture_12_feb_01st

\section{Splitting Fields (Continued)}%
\label{sec:splitting_fields_continued}
% section splitting_fields_continued

\begin{defn}[Splitting Field]\index{Splitting Field}\label{defn:splitting_field}
  Let $f(x) \in F[x]$ be non-constant. A \hlnotea{minimal extension} $K$ of $F$ with the
  property that $f(x)$ splits over $K$ is called a \hlnoteb{splitting field} for $f(x)$
  over $F$.
\end{defn}

The following result is a direct consequence of \cref{thm:repeated_kronecker_s_theorem}.

\begin{propo}[A Splitting Field is Generated]\label{propo:a_splitting_field_is_generated}
  Let $f(x) \in F[x]$ be non-constant, and let $K / F$ be such that $f(x)$ splits over $K$.
  Suppose
  \begin{equation*}
    f(x) = u(x - \alpha_1) \hdots (x - \alpha_n),
  \end{equation*}
  where $u \in F$ and $\alpha_1, \ldots, \alpha_n \in K$. Then a splitting field for $f(x)$
  over $F$ is $F(\alpha_1, \ldots, \alpha_n)$.
\end{propo}

\begin{eg}
  Find a splitting field for
  \begin{equation*}
    f(x) = x^4 + x^2 - 6
  \end{equation*}
  over $\mathbb{Q}$.
\end{eg}

\begin{solution}
  Notice that
  \begin{equation*}
    f(x) = (x^2 + 3)(x^2 - 2) = (x + \sqrt{3} i)(x - \sqrt{3}i)(x - \sqrt{2})(x + \sqrt{2})
  \end{equation*}
  in $\mathbb{C}[x$. Then a splitting field of $f(x)$ over $\mathbb{Q}$ is 
  $\mathbb{Q}(\sqrt{2}, \sqrt{3}i)$.
\end{solution}

\newthought{Now what if} we had two differing extensions at which $f(x)$ splits, say $K$ and
$E$, and $K$ and $E$ are not the same field extension? In particular, $K$ and $E$ would contain
some subfield, say $F(\alpha_1, \ldots, \alpha_n)$ and $F(\beta_1, \ldots, \beta_n)$
respectively, which may not be the same spliting field. How are these splitting fields related?

\begin{marginfigure}
  \centering
  \resizebox{\marginparwidth}{!}{
  \begin{tikzpicture}
    \node[label={270:{$f(x) \in F[x]$}}] at (0, 0) {};
    \node[label={90:{$f(x)$ splits in $K$}}] at (-2, 2) {};
    \node[label={90:{$f(x)$ splits in $E$}}] at (2, 2) {};
    \draw (-1, 0) -- (-2, 2) node[midway,fill=black] {$F(\alpha_1, \ldots, \alpha_n)$};
    \draw (1, 0) -- (2, 2) node[midway,fill=black] {$F(\beta_1, \ldots, \beta_n)$};
    \draw[latex-latex] (-0.5, 1.3) to[bend left] (0.5, 1.3) node[midway,above=1.5] {relation?};
  \end{tikzpicture}}
  \caption{Differing Splitting Fields}
  \label{fig:differing_splitting_fields}
\end{marginfigure}

\begin{lemma}[Isomorphic Fields have Isomorphic Polynomial Rings]\label{lemma:isomorphic_fields_have_isomorphic_polynomial_rings}
  Let $F$ ad $F'$ be fields. If $\phi : F \to F'$ is an isomorphism, there exists a map
  $\tilde{\phi} : F[x] \to F'[x]$ that is also an isomorphism.
\end{lemma}

\begin{proof}
  The map $\tilde{\phi} : F[x] \to F'[x]$ given by
  \begin{equation*}
    \tilde{\phi}(\alpha_n x^n + \hdots + \alpha_1 x + \alpha_0) = \tilde{\alpha}_n x^n \hdots + \tilde{\alpha}_1 x + \tilde{\alpha}_0
  \end{equation*}
  is clearly an isomorphism between $F[x]$ and $F'[x]$.
\end{proof}

\begin{note}
  Since there is no difference between talking about $\phi$ and $\tilde{\phi}$, we shall
  freely write $\tilde{\phi}$ as $\phi$ without remorse.
\end{note}

\begin{lemma}[Isomorphism Extension Lemma]\index{Isomorphism Extension Lemma}\label{lemma:isomorphism_extension_lemma}
  Let $F$ and $F'$ be fields, $\phi : F[x] \to F'[x]$ be an isomorphism, $f(x) \in F[x]$ be
  irreducible, $\alpha$ be a root of $f(x)$ in an extension of $F$, and $\beta$ be a root of
  $f(x)$ be a root of $f(x)$ in an extension of $F'$. Then there exists an isomorphism
  $\psi : F(\alpha) \to F'(\beta)$ such that $\psi \restriction_F = \phi$. Moreover,
  $\psi(\alpha) = \beta$.
\end{lemma}

\begin{proof}[Sketch]
  Using the \hlnotea{First Isomorphism Theorem} to find $\rho_1$ and $\rho_2$, we have
  \begin{equation*}
    F(\alpha) \overset{\rho_1}{\to} F[x] \Big/ \langle f(x) \rangle \overset{\sigma}{\to}
    F'[x] \Big/ \langle \phi(f(x)) \rangle \overset{\rho_2}{\to} F'(\beta),
  \end{equation*}
  \sidenote{
  \begin{ex}
    Prove that $\phi(f(x))$ is irreducible.
  \end{ex}
  } where $\sigma(\bar{g(x)}) = \bar{\phi(g(x))}$. \sidenote{
  \begin{ex}
    Prove that $\sigma$ is an isomorphism.
  \end{ex}
  }
  
  Then $\psi = \rho_2 \circ \sigma \rho_1 : F(\alpha) \to F'(\beta)$ is an isomorphism.

  Let $a \in F$. Then
  \begin{equation*}
    \psi(a) = \rho_2 \circ \sigma \circ \rho_1 (a) = \rho_2 \circ \sigma(\bar{a}) 
      = \rho_2(\bar{\phi(a)}) = \phi(a).
  \end{equation*}
  Also,
  \begin{equation*}
    \psi(\alpha) = \rho_2 \circ \sigma \circ \rho_1(\alpha) = \rho_2 \circ \sigma(\bar{x}) 
      = \rho_2(\bar{\phi(x)}) = \rho_2(\bar{x}) = \beta.
  \end{equation*}
\end{proof}

It follows from induction that

\begin{lemma}[Extended Isomorphism Extension Lemma]\label{lemma:extended_isomorphism_extension_lemma}
  Let $F$ be a field, $f(x) \in F[x]$ non-constant, $K$ a splitting field for $f(x)$ over $F$,
  $F'$ a field, $\phi : F \to F'$ an isomorphism, and $K'$ a splitting field for $\phi(f(x))$
  over $F'$. Then there is an isomorphism $\psi : K \to K'$ such that $\psi \restriction_F = \phi$.
\end{lemma}

\begin{crly}[Splitting Fields are Unique up to Isomorphism]\label{crly:splitting_fields_are_unique_up_to_isomorphism}
  Let $f(x) \in F[x]$ be non-constant. If $K$ and $K'$ are splitting fields for $f(x)$ over $F$,
  then $K \cong K'$.
\end{crly}

\begin{proof}
  Consider $\phi = \id$ and use \cref{lemma:extended_isomorphism_extension_lemma}.
\end{proof}

% section splitting_fields_continued (end)

\section{Algebraic Closures}%
\label{sec:algebraic_closures}
% section algebraic_closures

We talked about algebraicity, and it makes sense asking about where exactly `upstairs' that
we will be able to find all of the algebraic numbers over our given field. A lot of the
machinery has been taken care of with the introduction of 
\hyperref[sec:splitting_fields]{splitting fields}.

\begin{defn}[Algebraic Closures]\index{Algebraic Closures}\label{defn:algebraic_closures}
  A field $\bar{F}$ is an \hlnoteb{algebraic closure} of a field $F$ if
  \begin{enumerate}
    \item $\bar{F} / F$ is algebraic; and
    \item every non-constant $f(x) \in F[x]$ splits over $\bar{F}$.
  \end{enumerate}
\end{defn}

\begin{eg}
  $\mathbb{C}$ is an algebraic closure for $\mathbb{R}$.
\end{eg}

\begin{eg}
  $\mathbb{C}$ is \hlimpo{not} an algebraic closure for $\mathbb{Q}$.\sidenote{Why?}
\end{eg}

\begin{defn}[Algebraically Closed]\index{Algebraically Closed}\label{defn:algebraically_closed}
  A field $F$ is \hlnoteb{algebraically closed} if every non-constant $f(x) \in F[x]$
  has a root in $F$.
\end{defn}

\begin{remark}
  If $F$ is algebraically closed, then every non-constant $f(x) \in F[x]$ splits over $F$.
\end{remark}

\begin{eg}
  $\mathbb{C}$ is algebraically closed.
\end{eg}

% section algebraic_closures (end)

% chapter lecture_12_feb_01st (end)

\chapter{Lecture 13 Feb 04th}%
\label{chp:lecture_13_feb_04th}
% chapter lecture_13_feb_04th

\section{Algebraic Closures (Continued)}%
\label{sec:algebraic_closures_continued}
% section algebraic_closures_continued

This may seem obvious from the names (closure, closed?), but it is actually not
immediately clear that algebraic closures are algebraically closed.

\begin{propo}[Algebraic Closures are Algebraically Closed]\label{propo:algebraic_closures_are_algebraically_closed}
  If $\bar{F}$ is an algebraic closure for $F$, then $\bar{F}$ is algebraically closed.
\end{propo}

\begin{proof}
  Let $f(x) \in \bar{F}[x]$ be non-constant. Then by
  \hyperref[thm:kronecker_s_theorem]{Kronecker's Theorem}, $f(x)$ has a root $\alpha$ in
  some extension of $\bar{F}$. Since $\bar{F}(\alpha) / \bar{F}$ is algebraic and $\bar{F}
  / F$ is also algebraic, we have that $\bar{F}(\alpha) / F$ is algebraic. Thus $\alpha$
  is a root of some $p(x) \in F[x]$. Since $\bar{F}$ is the algebraic closure of
  $\bar{F}$, $p(x)$ splits over $\bar{F}[x]$, and so it follows that $\alpha \in \bar{F}$.
  Therefore, $\bar{F}$ is algebraically closed.
\end{proof}

\marginnote{
\begin{mnote}
  \cref{thm:every_field_has_an_algebraic_closure} is an exercise in A5.
\end{mnote}
}
\begin{thm}[Every Field has an Algebraic Closure]\label{thm:every_field_has_an_algebraic_closure}
  For every field $F$, there exists an algebraically closed field that contains $F$.
\end{thm}

\begin{thm}[Smallest Algebraic Closure]\label{thm:smallest_algebraic_closure}
  Let $K$ be an algebraically closed field that contains $F$. The collection of elements
  in $K$ which are algebraic over $F$ is an algebraic closure of $F$.
\end{thm}

\begin{proof}
  Let
  \begin{equation*}
    L := \left\{ \alpha \in K \mmid \alpha \text{ is algebraic over } F \right\}.
  \end{equation*}
  As given in the statement, we want to show that $L$ is an algebraic closure of $F$.

  It is clear that $L / F$, since every $\beta \in F$ is algebraic over $F$ and is hence
  in $L$. Let $f(x) \in F[x]$ with $\deg f \geq 1$. Since $f(x)$ splits over $K$, we have
  \begin{equation*}
    f(x) = u(x - \alpha_1)(x - \alpha_2) \hdots (x - \alpha_n),
  \end{equation*}
  where $u \in F^\times$ and $\alpha_i \in K$ for $i \in \{ 1, \ldots, n \}$. Then since
  $f(\alpha_i) = 0$ for all $i$, it follows that each of the $\alpha_i \in L$. In other
  words, $f(x)$ splits over $L$.
\end{proof}

% section algebraic_closures_continued (end)

\section{Cyclotimic Extensions}%
\label{sec:cyclotimic_extensions}
% section cyclotimic_extensions

We look into a specific class of field extensions, which is rather important to us.
Consider the following question:

\begin{quotebox}{green}{white}
  What is the splitting field of the polynomial $f(x) = x^n - 1$ over $\mathbb{Q}$?
\end{quotebox}

The following definition should remind one of MATH 135.

\begin{defn}[$n$\textsuperscript{th} Roots of Unity]\index{$n$\textsuperscript{th} Roots of Unity}\label{defn:_n_th_roots_of_unity}
  We call the roots of $x^n - 1$ (over $\mathbb{C}$) the \hlnoteb{$n$\textsuperscript{th}
  roots of unity}.
\end{defn}

\begin{eg}
  We can obtain all the $n$\textsuperscript{th} roots of unity using Euler's identity
  \begin{equation*}
    \xi_n = \cos \left( \frac{2\pi}{n} \right) + i \sin \left( \frac{2 \pi}{n} \right),
  \end{equation*}
  which we label these roots as $1 = \xi_n^1, \xi_n^2, \xi_n^3, \ldots, \xi_n^{n - 1}$.
\end{eg}

Following the various results that we have proven in the last few lectures, we know that
the splitting field of $x^n - 1$ over $\mathbb{Q}$ is therefore $\mathbb{Q}(\xi_n)$.

\newthought{We can then ask} ourselves what is the degree of $\mathbb{Q}(\xi_n)$ over
$\mathbb{Q}$, i.e. what is $[ \mathbb{Q}(\xi_n) : \mathbb{Q} ]$?

If $n = p$ where $p$ is prime, then since we may write
\begin{equation*}
  x^p - 1 = (x - 1)(x^{p - 1} + x^{p - 2} + \hdots + x + 1),
\end{equation*}
by \cref{eg:polynomial_with_prime_minus_1_degree} in
\cref{eg:examples_of_irreducible_polynomials}, we know that
\begin{equation*}
  \Phi_p(x) = x^{p - 1} + \hdots + x + 1
\end{equation*}
is irreducible over $\mathbb{Q}$. So $\Phi_p(x)$ is the minimal polynomial for $\xi_n$
over $\mathbb{Q}$.

It thus follows that $[ \mathbb{Q}(\xi_p) : \mathbb{Q} ] = p - 1$.

\begin{eg}
  We shall calculate $[ \mathbb{Q}(\xi_6) : \mathbb{Q} ]$. Note that
  \begin{equation*}
    \xi_6 = \cos \left( \frac{2 \pi}{6} \right) + i \sin \left( \frac{2 \pi}{6} \right) =
    \frac{1}{2} + i \frac{\sqrt{3}}{2}.
  \end{equation*}
  Since $1, 2 \in \mathbb{Q}$, we have that $\mathbb{Q}(\xi_6) = \mathbb{Q}(i \sqrt{3})$.
  By \hyperref[propo:eisenstein_s_criterion]{$3$-Eisenstein}, the polynomial $x^2 + 3$ is
  irreducible and is a polynomial where $i \sqrt{3}$ is a root. Thus
  \begin{equation*}
    [ \mathbb{Q}(\xi_6) : \mathbb{Q} ] = [ \mathbb{Q}(i \sqrt{3}) : \mathbb{Q} ] = \deg
      (x^2 + 3) = 2.
  \end{equation*}
\end{eg}

\begin{remark}
  The $n$\textsuperscript{th} roots of unity form a cyclic group. A generator of this
  group is called a \hldefn{primitive $n$\textsuperscript{th} root of unity}.

  In other words, $\xi_n^k$ is an primitive $n$\textsuperscript{th} root of unity iff
  $(\xi_n^k)^m \neq 1$ for $m = 1, 2, \ldots, n - 1$.

  From Group Theory, $\xi_n^k$ is a primitive $n$\textsuperscript{th} root of unity iff
  $\gcd(n, k) = 1$. Thus, there are
  \begin{equation*}
    \phi(n) = \abs{ \{ 1 \leq k \leq n : \gcd(k, n) = 1 \} }
  \end{equation*}
  primitive $n$\textsuperscript{th} root of unity\sidenote{Explanation required.}.
\end{remark}

\begin{defn}[$n$\textsuperscript{th} Cyclotomic Polynomial]\index{$n$\textsuperscript{th} Cyclotomic Polynomial}\label{defn:_n_th_cyclotomic_polynomial}
  For $n \geq 1$, the \hlnoteb{$n$\textsuperscript{th} cyclotomic polynomial} is
  \begin{equation*}
    \Phi_n(x) = \prod_{\substack{1 \leq k \leq n \\ \gcd(k, n) = 1}} \left( x - e^{2 \pi i
    \frac{k}{n}} \right) = (x - \alpha_1, \ldots)(x - \alpha_n) \hdots (x - \alpha_{\phi(n)}),
  \end{equation*}
  where the $\alpha_i$'s are the primitive $n$\textsuperscript{th} roots of unity.
\end{defn}

\begin{remark}
  Since $\Phi_n(x)$ has rational coefficients, we know that $\Phi_n(x) \in \mathbb{C}[x]$.
\end{remark}

In fact, $\Phi_n(x)$ is the minimal polynomial for $\xi_n$ over $\mathbb{Q}$, which then
gives us that $[ \mathbb{Q}(\xi_n) : \mathbb{Q} ] = \phi(n)$. However, we are not yet
ready to show this.

\begin{eg}
  The following are $n$\textsuperscript{th} cyclotomic polynomials, where $n = 1, 2, 3$
  and $4$:
  \marginnote{See the first 30 cyclotomic polynomials on
  \href{https://en.wikipedia.org/wiki/Cyclotomic_polynomial\#Examples}{Wikipedia}.}
  \begin{itemize}
    \item $\Phi_1(x) = x - 1$
    \item $\Phi_2(x) = \left(x - e^{2 \pi i \frac{1}{2}}\right) = (x + 1)$
    \item $\Phi_3(x) = \left(x + e^{2 \pi i \frac{1}{3}}\right)\left(x - e^{2 \pi i
      \frac{2}{3}}\right) =  x^2 + x + 1$
    \item $\Phi_4(x) = (x + i)(x - i) = x^2 + 1$
  \end{itemize}
\end{eg}

\begin{eg}
  Let $n = p$ be prime. Then the $p$\textsuperscript{th} roots of unity are
  \begin{equation*}
    1, \xi_p^2, \xi_p^3, \ldots, \xi_p^{p - 1}
  \end{equation*}
  and the primitives are
  \begin{equation*}
    \xi_p^2, \xi_p^3, \ldots, \xi_p^{p - 1}.
  \end{equation*}
  Thus
  \begin{equation*}
    x^p - 1 = (x - 1)(x^{p - 1} + x^{p - 2} + \hdots + x^2 + x + 1) = (x - 1) \Phi_p(x).
  \end{equation*}
\end{eg}

A good question to ask here is:

\begin{quotebox}{green}{white}
  Is there an easier way to compute $\Phi_n(x)$ for all $n$?
\end{quotebox}

% section cyclotimic_extensions (end)

% chapter lecture_13_feb_04th (end)

\chapter{Lecture 14 Feb 08th}%
\label{chp:lecture_14_feb_08th}
% chapter lecture_14_feb_08th

\section{Cyclotomic Extensions (Continued)}%
\label{sec:cyclotomic_extensions_continued}
% section cyclotomic_extensions_continued

\begin{remark}
  Note that $Z := \{ z \in \mathbb{C} : z^n = 1 \}$ is a group. We may write
  \begin{equation*}
    \bigcupdot_{d \mid n} \left\{ \text{ primitive } d\text{\textsuperscript{th} roots of
    unity } \right\}.
  \end{equation*}
\end{remark}

\begin{lemma}[$x^n - 1 = \prod_{d \mid n} \Phi_d(x)$]\label{lemma:_x_n_1_d mid n_phi_d_x_}
  We have
  \begin{equation*}
    x^n - 1 = \prod_{d \mid n} \Phi_d(x).
  \end{equation*}
\end{lemma}

\begin{eg}
  \begin{align*}
    \Phi_6(x) &= \frac{x^6 - 1}{\Phi_1(x) \Phi_2(x) \Phi_3(x)} \\
              &= \frac{x^6 - 1}{(x - 1)(x + 1)(x^2 + x + 1)} = x^2 - x + 1.
  \end{align*}
\end{eg}

\begin{propo}[Cyclotomic Polynomials have Integer Coefficients]\label{propo:cyclotomic_polynomials_have_integer_coefficients}
  For every $n \geq 1$, $\Phi_n(x) \in \mathbb{Z}[x]$.
\end{propo}

\marginnote{
\begin{strategy}
  We shall use strong induction here.
\end{strategy}
}
\begin{proof}
  We proceed by induction on $n$. If $n = 1$, then $\Phi_1(x) = x - 1 \in \mathbb{Z}[x]$.

  Suppose the results holds for all $l < n$. By \cref{lemma:_x_n_1_d mid n_phi_d_x_}, we
  have
  \begin{equation*}
    x^n - 1 = f(x) \Phi_n(x)
  \end{equation*}
  where
  \begin{equation*}
    f(x) = \prod_{\substack{d \mid n \\ d < n}} \Phi_d(x).
  \end{equation*}
  By the induction hypothesis, $f(x) \in \mathbb{Z}[x]$. Let $F = \mathbb{Q}(\xi_n)$ so
  that $\Phi_n(x) \in F[x]$. By the division algorithm, $\exists ! q(x), r(x) \in F[x]$
  such that
  \begin{equation*}
    x^n - 1 = f(x) q(x) + r(x).
  \end{equation*}
  Similarly, $\exists ! \tilde{q}(x), \tilde{r}(x) \in \mathbb{Q}[x] \supset
  \mathbb{Z}[x]$ such that
  \begin{equation*}
    x^n - 1 f(x) \tilde{q}(x) + \tilde{r}(x).
  \end{equation*}
  Since $\mathbb{Q} \subseteq F \subseteq \mathbb{C}$, by uniqueness\sidenote{This part
  should be thought of in the following way: we know that there is some $q(x) \in F[x]$,
  which is an extension of $\mathbb{Q}[x]$, and we also found that there is some
  $\tilde{q}(x) \in \mathbb{Q}[x]$, and so uniqueness tells us that the two must be the
  same.},
  \begin{equation*}
    \Phi_n(x) = q(x) = \tilde{q}(x) \in \mathbb{Q}[x].
  \end{equation*}
  It follows by \hyperref[thm:gauss_lemma]{Gauss' Lemnma} that $\Phi_n(x) \in
  \mathbb{Z}[x]$.
\end{proof}

\marginnote{The proof for \cref{thm:cyclotomic_polynomials_are_irreducible_over_q_} is
provided over two separate lectures, in particular it is provided at the end of this
lecture and the beginning of Lecture 16. For sanity, the entire proof will be provided
here.}
\begin{thm}[Cyclotomic Polynomials are Irreducible over $\mathbb{Q}$]\label{thm:cyclotomic_polynomials_are_irreducible_over_q_}
  For $n \geq 1$, $\Phi_n(x)$ is irreducible over $\mathbb{Q}$.
\end{thm}

\marginnote{
\begin{strategy}
  We will show that $\Phi_n(x)$ is a minimal polynomial. If $g(x) \in \mathbb{Q}[x]$ is a
  minimal polynomial for $\xi_n$, then since $\xi_n$ is also a root of $\Phi_n(x)$, we
  must have $g(x) \mid \Phi_n(x)$. So to show that $g(x)$ is actually $\Phi_n(x)$, it
  suffices to show that $\Phi_n(x) \mid g(x)$.
\end{strategy}
}
\begin{proof}
  Let $g(x) \in \mathbb{Q}[x]$ be a minimal polynomial for $\xi_n$. It suffices for us to
  show that $\Phi_n(x) \mid g(x)$. To that end, we can show that every root of $\Phi_n(x)$
  is a root of $g(x)$ (in $\mathbb{C}$).

  Let $\alpha$ be a root of $\Phi_n(x)$. Then by \cref{defn:_n_th_cyclotomic_polynomial},
  $\alpha = \xi_n^k$ for some $k \in \{ 1, \ldots, n - 1 \}$ such that $\gcd(k, n) = 1$.
  Then let $k = p_1 p_2 \hdots p_N$, where each $p_i$ is a prime and $p_i \nmid n$
  \sidenote{Note that this must be the case since $\gcd(k, n) = 1$.}.
  
  Thus, the statement which we wish to prove becomes the following: $\xi_n^{p_1},
  \xi_n^{p_1 p_2}, \ldots, \xi_n^{p_1 p_2 \hdots p_N} = \alpha$ are roots of $g(x)$.

  To prove the above, it suffices for us to show that if $\xi \in \mathbb{C}$ is a root of
  $g(x)$, then $\xi^p$, where $p$ is prime and $p \nmid n$, is also a root of $g(x)$.

  Suppose not \faDizzy, i.e. that $g(\xi) = 0$ but $g\left(\xi^p\right) \neq 0$, where $p$
  is prime and $p \nmid n$. Now since $g(x) \mid \Phi_n(x)$, we have $\Phi_n(\xi) = 0$.
  Since $p \nmid n$, it follows that $\xi^p$ is also a primitive $n$\textsuperscript{th}
  root of unity, i.e. $\Phi_n(\xi_n^p) = 0$. Now since $g(x) \mid \Phi_n(x)$, $\exists
  h(x) \in \mathbb{Q}[x]$ such that $\Phi_n(x) =g(x) h(x)$. By
  \hyperref[thm:gauss_lemma]{Gauss}, WMA $h(x) \in \mathbb{Z}[x]$. Since $\mathbb{Z}[x$ is
  an integral domain, $\Phi_n(\xi^p) = 0$ and $g(\xi^p) \neq 0 \implies h(\xi^p) = 0$.

  Let $f(x) = h(x^p) \in \mathbb{Z}[x]$. Then $f(\xi) = 0$. Moreover, we have $g(x) \mid
  f(x)$ in $\mathbb{Q}[x]$. Thus $f(x) = g(x) k(x)$ for some $k(x) \in \mathbb{Z}[x]$
  (again, through Gauss).

  Suppose $h(x) = \sum b_j x^j$, which then implies that $f(x) = \sum b_j x^{pj}$.
  Consider $\bar{f}(x) \in \mathbb{Z}_p[x]$, i.e.
  \begin{equation*}
    \bar{f}(x) = \sum \bar{b}_j x^{pj}, \quad \bar{b}_j \equiv b_j \mod p.
  \end{equation*}
  Then
  \begin{align*}
    \bar{f}(x) &= \sum \bar{b}_j^p x^{pj} \quad \because \text{ \hlnotea{Fermat's Little
                Theorem} } \\
               &= \left( \sum \bar{b}_j x^j \right)^p \quad \because \text{
                \hlnotea{Freshman's Dream} } \\
               &= \left( \bar{h}(x) \right)^p.
  \end{align*}
  It follows that
  \begin{equation*}
    \left( \bar{h}(x) \right)^p = \bar{f}(x) = \bar{g}(x) \bar{k}(x) \in \mathbb{Z}_p[x].
  \end{equation*}

  Now let $\bar{l}(x)$ be an irreducible factor of $\bar{g}(x)$ over $\mathbb{Z}_p[x]$
  \sidenote{Note that this $\bar{l}(x)$ may be $\bar{g}(x)$ itself if $\bar{g}(x)$ is
  still irreducible over $\mathbb{Z}_p[x]$.}. Since $\bar{l}(x) \mid \bar{h}(x)^p$, we
  have that $\bar{l}(x) \mid \bar{h}(x)$ \sidenote{\hlwarn{Why?}}. % TODO : Find out why

  On the other hand, in $\mathbb{Z}_p[x]$, we have that $\bar{\Phi}_n(x) = \bar{g}(x)
  \bar{h}(x)$. It follows that $\bar{l}(x)^2 \mid \bar{\Phi}_n(x)$
  \sidenote{\hlwarn{Why?}}. % TODO : Find out why
  Since $\bar{\Phi}_n(x) \mid x^n - 1$, we have that
  \begin{equation*}
    x^n - 1 = \bar{l}(x)^2 \bar{q}(x) \in \mathbb{Z}_p[x].
  \end{equation*}
  By \hlnotea{taking derivatives} on both sides, we have
  \begin{align*}
    \bar{n} x^{n - 1} &= 2 \bar{l}(x) \bar{l}'(x) \bar{q}(x) + \bar{l}(x)^2 \bar{q}'(x) \\
                      &= \bar{l}(x) [ \faEllipsisH ] \in \mathbb{Z}_p[x],
  \end{align*}
  where $\faEllipsisH$ is an irrelevant factor. Since $\bar{n} \neq 0$, we have that the
  only root of LHS is $\bar{0}$, and so the only root of $\bar{l}(x)$ is some extension of
  $\mathbb{Z}_p$ is $\bar{0}$. Since $\bar{l}(x) \mid x^n - \bar{1}$, we have that
  $\bar{0}^n - \bar{1} = 0$ but that mean $0 = 1 \in \mathbb{Z}_p$, a contradiction.

  Tracing back our long convoluted line of thought, we have that \faDizzy is not true, and
  so we must have $g(\xi^p) = 0$, which
  \begin{itemize}
    \item[$\implies$] all the $\xi_n^{p_1}, \xi_n^{p_1 p_2}, \ldots, \alpha$ are all roots
      of $g(x)$;
    \item[$\implies$] $\Phi_n(x) \mid g(x)$;
    \item[$\implies$] $\Phi_n(x) = g(x)$,
  \end{itemize}
  which is what we want to show.
\end{proof}

\begin{crly}[Cyclotomic Polynomials are Minimal Polynomials of Its Roots over $\mathbb{Q}$]\label{crly:cyclotomic_polynomials_are_minimal_polynomials_of_its_roots_over_q_}
  $\Phi_n(x)$ is the minimal polynomial for $\xi_n$ over $\mathbb{Q}$. In particular, $[
  \mathbb{Q}(\xi_n) : \mathbb{Q} ] = \phi(n)$.
\end{crly}

\begin{eg}
  Let $f(x) = x^5 - 3$. Describe the splitting field of $f(x)$ over $\mathbb{Q}$. We shall
  find a basis for this splitting field over $\mathbb{Q}$.

  The roots of $f(x)$ are
  \begin{equation*}
    \sqrt[5]{3}, \, \xi_5 \sqrt[5]{3}, \xi_5^2 \sqrt[5]{3}, \xi_5^3 \sqrt[5]{3}, \xi_5^4
    \sqrt[5]{3}.
  \end{equation*}
  It follows that the splitting field for $f$ is $F = \mathbb{Q}(\sqrt[5]{3}, \xi_5)$.
  Note that since
  \begin{equation*}
    \deg_{\mathbb{Q}}(\xi_5) = \phi(5) = 4 \text{ and } \deg_{\mathbb{Q}}(\sqrt[5]{3}) =
    5,
  \end{equation*}
  it follows from A4Q2 that
  \begin{equation*}
    [ \mathbb{Q}(\sqrt[5]{3}, \xi_5) : \mathbb{Q} ] = [ \mathbb{Q}(\sqrt[5]{3}) :
    \mathbb{Q} ][ \mathbb{Q}(\xi_5) : \mathbb{Q} ] = 4 \cdot 5 = 20.
  \end{equation*}

  Now a \hlbnotea{basis for $\mathbb{Q}(\xi_5)(\sqrt[5]{3}) / \mathbb{Q}(\xi_5)$} is
  \begin{equation*}
    \left\{ 1, \sqrt[5]{3}, \left( \sqrt[5]{3} \right)^2, \left( \sqrt[5]{3} \right)^3,
    \left( \sqrt[5]{3} \right)^4 \right\},
  \end{equation*}
  while a \hlbnotea{basis for $\mathbb{Q}(\xi_5) / \mathbb{Q}$} is
  \begin{equation*}
    \left\{ 1, \xi_5, \xi_5^2, \xi_5^3 \right\}.
  \end{equation*}
  Following the \hyperref[thm:tower_theorem]{Tower Theorem}, a basis for the splitting
  field $F$ is
  \begin{equation*}
    \left\{ \left( \sqrt[5]{3} \right)^i \left( \xi_5 \right)^j \mmid 0 \leq i \leq 4, \,
    0 \leq j \leq 3 \right\}.
  \end{equation*}
\end{eg}

% section cyclotomic_extensions_continued (end)

% chapter lecture_14_feb_08th (end)

\chapter{Lecture 15 Feb 11th}%
\label{chp:lecture_15_feb_11th}
% chapter lecture_15_feb_11th

\section{Finite Fields}%
\label{sec:finite_fields}
% section finite_fields

Finite fields are very easy to work with a grasp. The nice thing about finite fields is
that, up to isomorphism, there is only one field that has order prime to some power, which
we shall show in this section.

\begin{lemma}[Units of a Finite Field Form a Finite Cyclic Group]\label{lemma:units_of_a_finite_field_form_a_finite_cyclic_group}
  Let $F$ be a finite gield. Then $G = F^\times$ is a finite cyclic group.
\end{lemma}

\begin{proof}
  Since $G$ is the set of units of $F$, we know that $G$ is an abelian group by its
  construction, and it is finite since $F$ is finite. Then, by the \hlnotea{Finite Abelian
  Group Structure}, $\exists n_1, \ldots, n_m \in \mathbb{Z}$ such that
  \begin{equation}\label{eq:lemma_units_of_finite_fields_eq1}
    G \simeq \mathbb{Z}_{n_1} \times \mathbb{Z}_{n_2} \times \hdots \times
    \mathbb{Z}_{n_m},
  \end{equation}
  and each $n_i$ is a prime power. Let
  \begin{align*}
    N &:= n_1 n_2 \hdots n_m \text{ and } \\
    M &:= \lcm(n_1, \ldots, n_m).
  \end{align*}
  By construction, $M \leq N$. Now $\forall a \in G$, we have that $a$ is a root of $x^M -
  1 \in F[x]$ due to \cref{eq:lemma_units_of_finite_fields_eq1}\sidenote{$a$ is of one of
  the orders $n_1, n_2, \ldots. n_m$, so it is a root of $x^M - 1$.}.
  
  Note that $N = \abs{G}$, and the polynomial $x^M - 1$ has at most $M$ roots. Therefore,
  $N \leq M$. Thus we must have $N = M$, thus forcing the $n_i$'s to be coprimes, and so
  we have
  \begin{equation*}
    G \simeq \mathbb{Z}_{n_1} \times \mathbb{Z}_{n_2} \times \hdots \times
    \mathbb{Z}_{n_m} = \mathbb{Z}_N.
  \end{equation*}
\end{proof}

\begin{propo}[Order of Finite Fields are Powers of Its Primal Characteristic]\label{propo:order_of_finite_fields_are_powers_of_its_primal_characteristic}
  Let $F$ be a finite field. Then
  \begin{enumerate}
    \item $\abs{F} = p^n$, where $p$ is the \hlnotea{characteristic}\sidenote{Recall from
      PMATH 347 that the definition of the characteristic is the order of $1$ under
      addition. We shall use $\Char(F)$ to mean the characteristic of the field $F$.} of
      $F$ and $n = [F : \mathbb{Z}_p]$.
    \item $F = \mathbb{Z}_p(\alpha)$ for some $\alpha$ such that
      $\deg_{\mathbb{Z}_p}(\alpha) = n$.
  \end{enumerate}
\end{propo}

\begin{proof}
  Let $F$ be a finite field with characteristic $p$. Then $\mathbb{Z}_p$ is a prime
  subfield of $F$, and in particular $F / \mathbb{Z}_p$. Let $n = [ F : \mathbb{Z}_p ]$.
  By \cref{lemma:units_of_a_finite_field_form_a_finite_cyclic_group}, let $\alpha \in G =
  F^\times$ be such that $G = \langle \alpha \rangle$. By adding a unit of $F$ to
  $\mathbb{Z}_p$, since $\mathbb{Z}_p$ is a prime subfield, we have that
  $\mathbb{Z}_p(\alpha) = F$.

  Now since $n = [F : \mathbb{Z}_p]$, we have that
  \begin{equation*}
    F = \Span_{\mathbb{Z}_p} \{ 1, \alpha, \alpha^2, \ldots, \alpha^{n - 1} \}.
  \end{equation*}
  It follows that $\abs{F} = p^n$.
\end{proof}

\marginnote{\cref{thm:finite_fields_as_splitting_fields} is the important theorem that
tells us that there is only one finite field for every $p^n$ up to isomorphism, and this
follows from the \hyperref[crly:splitting_fields_are_unique_up_to_isomorphism]{uniqueness
of splitting fields}.}
\begin{thm}[Finite Fields as Splitting Fields]\label{thm:finite_fields_as_splitting_fields}
  Let $p$ be a prime and $n \in \mathbb{N}$. Then $F$ is a finite field of order $p^n$ iff
  $F$ is the splitting field of $f(x) = x^{p^n} - x$ over $\mathbb{Z}_p[x]$.
\end{thm}

\begin{proof}
  Suppose $\abs{F} = p^n$. By \hyperref[thm:lagrange_s_theorem]{Lagrange}\sidenote{Is it
  really Lagrange?}, $a^{p^n - 1} - 1 = 0$ for every $a \in F^\times$. Then in particular,
  \begin{equation*}
    a(a^{p^n} - 1) = a^{p^n} - a = 0.
  \end{equation*}
  It follows that every $a \in F$ is a root of $x^{p^n} - x$.

  Since $x^{p^n} - x$ has at most $p^n$ roots, $F$ must thus contain al roots of $x^{p^n}
  - x$, and so $x^{p^n} - x$ splits over $F[x]$. Any proper subfield of $F$ would not have
  enough elements to be a splitting field for $x^{p^n} - x$. Thus $F$ is a splitting field
  of $x^{p^n} - x$.

  For the \hlbnoted{$\impliedby$} direction, let $F$ be the splitting field of $f(x) =
  x^{p^n} - x$. Let
  \begin{equation*}
    K = \{ \alpha \in F : f(\alpha) = 0 \}.
  \end{equation*}
  \begin{ex}
    $K$ is a field.
  \end{ex}
  \marginnote{
  \begin{solution}[to the ex. in the proof]
    For $\alpha, \beta \in K$, we have that 
    \begin{equation*}
      \alpha^{p^n} - \alpha = 0 \text{ and } \beta^{p^n} - \beta = 0.
    \end{equation*}
    It then follows by the Freshman's Dream that
    \begin{gather*}
    \left( \alpha^{p^n} + \beta^{p^n} \right) - \alpha - \beta = 0 \\
    \implies \\
    (\alpha + \beta)^{p^n} - (\alpha + \beta) = 0.
   \end{gather*} 
  \end{solution}}

  Then $K \leq F$. However, we also have that $F \leq K$, since all roots of $f$ are in
  $F$ since $F$ is a splitting field, and $f$ also splits over $K$.

  Also, note that $f'(x) = -1$ since $\Char F = p$, and so $f$ has no repeated roots since
  it is a decreasing function.
\end{proof}

% section finite_fields (end)

% chapter lecture_15_feb_11th (end)

\chapter{Lecture 16 Feb 13th}%
\label{chp:lecture_16_feb_13th}
% chapter lecture_16_feb_13th

\section{Finite Fields (Continued)}%
\label{sec:finite_fields_continued}
% section finite_fields_continued

\marginnote{Since I moved the `second half' of the proof of
\cref{thm:cyclotomic_polynomials_are_irreducible_over_q_} over to
\cref{chp:lecture_14_feb_08th}, not too much content is left here.}
By \cref{lemma:units_of_a_finite_field_form_a_finite_cyclic_group},
\cref{propo:order_of_finite_fields_are_powers_of_its_primal_characteristic} and
\cref{thm:finite_fields_as_splitting_fields}, we have the following result.

\begin{thm}[Classification of Finite Fields]\label{thm:classification_of_finite_fields}
  For any prime $p$ and $n \in \mathbb{N}$, we have
  \begin{itemize}
    \item there exists a field $F$ such that $\abs{F} = p^n$; and
    \item any 2 fields of order $p^n$ are isomorphic to one another.
  \end{itemize}
\end{thm}

\begin{note}[Notation]
  We denote the field of order $p^n$ by $\mathbb{F}_{p^n}$, i.e.
  \begin{equation*}
    \mathbb{F}_{p^n} := \left\{ x \mmid f(x) = x^{p^n} - x = 0 \right\}.
  \end{equation*}
\end{note}

In the next lecture, we shall prove the following theorem.

\begin{thmnonum}[Subfields of Finite Fields]\label{thmnonum:subfields_of_finite_fields}
  If $E$ is a subfield of $\mathbb{F}_{p^n}$, then $E \simeq \mathbb{F}_{p^r}$, where $r
  \mid n$. Moreover, if $r \mid n$, then $\mathbb{F}_{p^n}$ has a unique\sidenote{This is
  truly unique, not unique up to isomorphism, which is \hlimpo{rare}.} subfield of order
  $p^r$.
\end{thmnonum}

The above theorem gives us the following example.

\begin{eg}
  Given the finite field $\mathbb{F}_{2^{12}}$, we know that the divisors of $12$ are
  \begin{equation*}
    1, \, 2, \, 3, \, 4, \, 6, \, 12.
  \end{equation*}
  By the above theorem, we have the following \hlnotea{lattice structure}. 
  \begin{figure}[ht]
    \centering
    \begin{tikzcd}
                                         & \mathbb{F}_{2^{12}} \arrow[rd, no head] \arrow[ld, no head] &                                                          &                                      \\
    \mathbb{F}_{2^4} \arrow[rd, no head] &                                                             & \mathbb{F}_{2^6} \arrow[ld, no head] \arrow[rd, no head] &                                      \\
                                         & \mathbb{F}_{2^2} \arrow[rd, no head]                        &                                                          & \mathbb{F}_{2^3} \arrow[ld, no head] \\
                                         &                                                             & \mathbb{F}_{2^1}                                         &                                     
    \end{tikzcd}
    \caption{Lattice of $\mathbb{F}_{2^12}$}
    \label{fig:lattice_of_f__2_12_}
  \end{figure}
\end{eg}

% section finite_fields_continued (end)

% chapter lecture_16_feb_13th (end)

\tuftepart{Galois Theory}

\chapter{Lecture 17 Feb 15th}%
\label{chp:lecture_17_feb_15th}
% chapter lecture_17_feb_15th

\section{Finite Fields (Continued 2)}%
\label{sec:finite_fields_continued_2}
% section finite_fields_continued_2

We shall now prove the last theorem that we stated.

\begin{thm}[Subfields of Finite Fields]\label{thm:subfields_of_finite_fields}
  If $E$ is a subfield of $\mathbb{F}_{p^n}$, then $E \simeq \mathbb{F}_{p^r}$,
  where $r \mid n$. Moreover, if $r \mid n$, then $\mathbb{F}_{p^n}$ has a
  unique\sidenote{This is truly unique, not unique up to isomorphism, which is
  \hlimpo{rare}.} subfield of order $p^r$.
\end{thm}

\begin{proof}
  \hlbnoted{Part 1} Let $E < \mathbb{F}_{p^n}$. By
  \cref{propo:order_of_finite_fields_are_powers_of_its_primal_characteristic}
  and \hyperref[thm:tower_theorem]{the Tower Theorem}, we have
  \begin{equation*}
    n = [ \mathbb{F}_{p^n} : \mathbb{F}_p ] = [ \mathbb{F}_{p^n} : E ][ E :
    \mathbb{F}_p ].
  \end{equation*}
  Then by letting $r = [ E : \mathbb{F}_p ]$, we have that $r \mid n$ and
  $\abs{E} = p^r$.

  \noindent
  \hlbnoted{Part 2} Suppose $r \mid n$, i.e. $\exists k \in \mathbb{Z}$ such
  that $n = rk$. Consider
  \begin{equation*}
    \mathbb{F}_{p^n} = \left\{ \alpha \in \bar{\mathbb{F}}_p \mmid
    \alpha^{p^{rk}} - \alpha = 0 \right\},
  \end{equation*}
  \sidenote{Note that we consider the closure just so that we contain all the
  roots. \hlwarn{Should $\mathbb{F}_{p^n}$ not already have everything?}}which
  we see is the splitting field of $x^{p^n} - x$, i.e. it is the set of roots of
  $x^{p^n} - x$. Since $r \mid n$, we have
  \begin{equation*}
    p^n - 1 = (p^r - 1)(p^{n - r} + p^{n - 2r} + \hdots + p^r + 1).
  \end{equation*}
  Then, let
  \begin{align*}
    E &:= \left\{ \alpha \in \bar{\mathbb{F}}_p \mmid \alpha^{p^r} - \alpha = 0
    \right\} \\
      &= \left\{ \alpha \in \bar{\mathbb{F}}_p \mmid \alpha^{p^r - 1} - 1 = 0
      \right\} \cup \{ 0 \} \\
      &\subseteq \bar{\mathbb{F}}_{p^n}.
  \end{align*}
  Moreover, we have that $\abs{E} = p^r$.

  For uniqueness, suppose if there exists $K < \mathbb{F}_{p^n}$ with order
  $p^r$. Then $\forall \alpha \in K$,
  \begin{equation*}
    \alpha^{p^r} - \alpha = 0 \implies \alpha \in E.
  \end{equation*}
  Thus $K = E$.
\end{proof}

% section finite_fields_continued_2 (end)

\section{Introduction to Galois Theory}%
\label{sec:introduction_to_galois_theory}
% section introduction_to_galois_theory

Let $f(x) \in F[x]$ be non-constant, and $\alpha_1, \ldots, \alpha_n$ be the
roots of $f(x)$ in its splitting field $K$. Our goal is to study these roots by
permuting them under automorphisms of the splitting field $K$.

\begin{defn}[Galois Group]\index{Galois Group}\label{defn:galois_group}
  Let $K / F$. We define the \hlnoteb{Galois Group} of $K / F$, by
  \begin{equation*}
    \Gal(K / F) := \left\{ \phi \in \Aut(K) \mmid \phi \restriction_{F} = \id
    \right\} \leq \Aut(K),
  \end{equation*}
  where $\Aut(K)$ is the \hlnotea{group of automorphisms} of $K$.
\end{defn}

\begin{lemma}[The Galois Group permutes roots]\label{lemma:the_galois_group_permutes_roots}
  Let $K / F$. If $\alpha \in K$ is a root of $f(x) \in F[x]$ and $\phi \in
  \Gal(K / F)$, then $\phi(\alpha)$ is also a root of $f(x)$.
\end{lemma}

\begin{proof}
  Let $f(x) \in F[x]$. Then $f(x) = \sum a_i x^i$. Since $\alpha$ is a root, we
  have $f(\alpha) = \sum a_i \alpha^i = 0$. Since $\phi$ is an automorphism, we
  must therefore have $0 = \phi(0)$. Since $\phi \in \Gal(K / F)$, we have that
  \begin{align*}
    0 = \phi(0) = \phi ( \sum a_i \alpha^i ) = \sum \phi(a_i) \phi(\alpha^i)
    \overset{(*)}{=} \sum a_i \phi(\alpha)^i = f(\phi(\alpha)),
  \end{align*}
  where $(*)$ is since $\phi$ fixes $F$.
\end{proof}

\begin{crly}[Elements of the Galois Group permutes roots of the same minimal
  polynomial]\label{crly:elements_of_the_galois_group_permutes_roots_of_the_same_minimal_polynomial}
  Let $K / F$. If $\alpha \in K$ is algebraic over $F$, and $\phi \in \Gal(K /
  F)$, then $\phi(\alpha)$ is algebraic over $F$, and $\alpha$ and
  $\phi(\alpha)$ has the same minimal polynomial in $F[x]$.
\end{crly}

\begin{eg}
  Let $F = \mathbb{Q}$ and $K = F(\sqrt{2})$. Then $\Gal( \mathbb{Q}(\sqrt{2}) /
  \mathbb{Q}) = \Aut \mathbb{Q}(\sqrt{2})$ \sidenote{\hlwarn{Why}? Is it cause
  there is very little room for us to wiggle around $\phi \restriction_F =
  \id$?}.
  Note that the minimal polynomial of $\sqrt{2}$ is $x^2 - 2 = (x - \sqrt{2})(x
  + \sqrt{2}) \in K[x]$. Thus if $\phi \in
  \Gal(\mathbb{Q}(\sqrt{2})/\mathbb{Q})$, then $\phi(\sqrt{2}) = \sqrt{2}$ or
  $-\sqrt{2}$ \sidenote{Note that we must fix everything else, by definition of
  a Galois group.}. It follows that the only two maps in $\Gal(K / F)$ are
  \begin{gather*}
    \phi_1 : a + b \sqrt{2} \mapsto a + b \sqrt{2} \\
    \phi_2 : a + b \sqrt{2} \mapsto a - b \sqrt{2}.
  \end{gather*}
  Thus $\Gal(K / F) = \{ \phi_1, \phi_2 \} \simeq \mathbb{Z}_2$.
\end{eg}

% section introduction_to_galois_theory (end)

% chapter lecture_17_feb_15th (end)

\chapter{Lecture 18 Feb 25th}%
\label{chp:lecture_18_feb_25th}
% chapter lecture_18_feb_25th

\section{Introduction to Galois Theory (Continued)}%
\label{sec:introduction_to_galois_theory_continued}
% section introduction_to_galois_theory_continued

\begin{eg}
  Consider the Galois group $\Gal( \mathbb{Q}(\sqrt{2}, \sqrt{3}) / \mathbb{Q}
  )$. Now the minimal polynomial for $\sqrt{2}$ and $\sqrt{3}$ are
  \begin{equation*}
    x^2 - 2, \text{ and } x^2 - 3,
  \end{equation*}
  respectively. Then we can only have $\phi(\sqrt{2}) = \pm \sqrt{2}$ and
  $\phi(\sqrt{3}) = \pm \sqrt{3}$, i.e.
  \begin{table}[ht]
    \centering
    \caption{All possible elements of $\Gal(\mathbb{Q}(\sqrt{2}, \sqrt{3}) /
    \mathbb{Q})$}
    \label{table:all_possible_elements_of_gal_q_sqrt_2_sqrt_3_q}
    \begin{tabular}{c | c c}
               & $\sqrt{2}$  & $\sqrt{3}$ \\
       \hline
      $\phi_1$ & $\sqrt{2}$  & $\sqrt{3}$ \\
      $\phi_2$ & $\sqrt{2}$  & $-\sqrt{3}$ \\
      $\phi_3$ & $-\sqrt{2}$ & $\sqrt{3}$ \\
      $\phi_4$ & $-\sqrt{2}$ & $-\sqrt{3}$
    \end{tabular}
  \end{table}
  So $\Gal( \mathbb{Q}(\sqrt{2}, \sqrt{3}) / \mathbb{Q} ) = \{ \phi_i : i = 1,
  2, 3, 4 \}$. Note that $\abs{\phi_i} = 2$ for $i = 2, 3, 4$. It follows that
  $\Gal( \mathbb{Q}(\sqrt{2}, \sqrt{3}) / \mathbb{Q} )$ is abelian and has order
  $4$. Therefore
  \begin{equation*}
    \Gal \left( \faktor{\mathbb{Q}(\sqrt{2}, \sqrt{3})}{\mathbb{Q}} \right)
    \simeq \mathbb{Z} \times \mathbb{Z}.
  \end{equation*}
\end{eg}

\marginnote{
Notice that in
\cref{eg:galois_elements_can_only_permute_roots_in_the_same_field}, the field
where the roots lie in is important; we see that the Galois group ended up being
the trivial group because the other roots of the minimal polynomial of
$\sqrt[3]{2}$ live in a higher extension.
}

\begin{eg}\label{eg:galois_elements_can_only_permute_roots_in_the_same_field}
  Consider $G = \Gal(\mathbb{Q}(\sqrt[3]{2}) / \mathbb{Q})$. Let $\phi \in G$.
  Since $\phi(\sqrt[3]{2})$ is a root of $x^3 - 2$, we must have that
  \begin{equation*}
    \phi(\sqrt[3]{2}) \in \left\{ \sqrt[3]{2}, \, \sqrt[3]{2} \zeta_3, \,
    \sqrt[3]{2} \zeta_3^2 \right\}.
  \end{equation*}
  However, $\sqrt[3]{2} \zeta_3, \, \sqrt[3]{2} \zeta_3^2 \notin
  \mathbb{Q}(\sqrt[3]{2})$. Therefore we must have $\phi(\sqrt[3]{2}) =
  \sqrt[3]{2}$, i.e. $\phi = \id$. It follows that $G = \{ 1 \}$.
\end{eg} 

% section introduction_to_galois_theory_continued (end)

\section{The Galois Group as a Permutation Group}%
\label{sec:the_galois_group_as_a_permutation_group}
% section the_galois_group_as_a_permutation_group

Let $F$ be a field, $f(x) \in F[x]$, $\deg f = n \geq 1$, and $K$ a splitting
field of $f(x)$ over $F$. Let $\alpha_1, \ldots, \alpha_n \in K$ be the roots of
$f(x)$, and let $G = \Gal(K / F)$. From the last few examples, we notice that
for any $\phi \in G$, $\phi(\alpha_i) = \alpha_j$.

In this section, we will show that $G$ is actually a \hlnotea{permutation group}
of the roots, as a subgroup of $S_n$ in the case of permuting the roots of
$f(x)$, the degree $n$ polynomial.

In fact, more is true, but we shall see that down the road.

\newthought{From} the last two examples, one cannot help but notice a possible
problem:
\begin{quotebox}{green}{white}
  what if there are repeated roots?
\end{quotebox}
If there are, indeed, repeated roots, say $\alpha_1 = \alpha_2$ among the roots
$\alpha_1, \alpha_2, \alpha_3, \alpha_4$, where $\alpha_4 \neq \alpha_3 \neq
\alpha_1 \neq \alpha_4$, then the identity element would be indistinguishable
from $\phi$ that is defined as
\begin{equation*}
  \phi(\alpha_1) = \alpha_2, \, \phi(\alpha_2) = \alpha_1, \, \phi(\alpha_3) =
  \alpha_3, \, \phi(\alpha_4) = \alpha_4.
\end{equation*}

So it suffices for us to consider for the case where $f(x)$ does not have
multiple roots of the same value, i.e. the \hlnotea{multiplicity} of all roots
is $1$.  Such polynomials are called \hlnotea{separable} polynomials.

\begin{defn}[Separable Polynomials]\index{Separable Polynomials}\label{defn:separable_polynomials}
  A polynomial $f(x) \in F[x]$ is said to be \hlnoteb{separable} if all of its
  roots have multiplicity $1$.
\end{defn}

\newthought{Let} $f(x) \in F[x]$ be separable with $\deg f = n \geq 1$, and
suppose $K$ is the splitting field of $f(x)$ over $F$. Let $\alpha_1, \ldots,
\alpha_n$ be the roots of $f$ in $K$. From our discussion above, we want to show
that $\Gal(K / F) \simeq P \leq S_n$. In other words, we want to see that given
$\phi \in \Gal(K / F)$, $\exists \pi \in P \leq S_n$ such that $\phi(\alpha_i) =
\alpha_{\pi(i)}$.

\begin{notation}
  Given $f(x) \in F[x]$, and $K$ the splitting field of $f(x)$, we sometimes
  write
  \begin{equation*}
    \Gal(f(x)) := \Gal(K / F).
  \end{equation*}
  In other words, when we write $\Gal(f(x))$, we are talking about the Galois
  group over the splitting field of $f(x)$ over $F$.
\end{notation}

\begin{eg}
  Recall an earlier example of ours where $f(x) = (x^2 - 2)(x^2 - 3) \in
  \mathbb{Q}[x]$, where we showed that the Galois group $\Gal(f(x)) =
  \mathbb{Z}_2 \times \mathbb{Z}_2$. Let
  \begin{equation*}
    \alpha_1 = \sqrt{2}, \, \alpha_2 = - \sqrt{2}, \, \alpha_3 = \sqrt{3}, \,
    \alpha_4 = - \sqrt{3}.
  \end{equation*}
  Then
  \begin{equation*}
    \Gal(f(x)) \simeq \{ \epsilon, \, (3 \; 4), \, (1 \; 2), \, (1 \; 2)(3 \; 4)
    \}.
  \end{equation*}
\end{eg}

\begin{eg}
  Let $x^2 + 1 \in \mathbb{Q}[x]$. Then \sidenote{The adjoined elements are $\pm
  i$.}
  \begin{equation*}
    \Gal(x^2 + 1) \simeq \mathbb{Z}_2.
  \end{equation*}
  However, if we consider $x^2 + 1 \in \mathbb{Z}_2[x]$, then
  \begin{equation*}
    \Gal(x^2 + 1) = \Gal((x + 1)^2) = \{ 1 \}.
  \end{equation*}
\end{eg}

The following is a quick corollary of from our discussion and observation.

\begin{crly}[The Galois Group completely captures all permutation of the roots]\label{crly:the_galois_group_completely_captures_all_permutation_of_the_roots}
  Let $F$ be a field, $f(x) \in F[x]$ a non-constant and irreducible, $K$ a
  splitting field of $f(x)$ over $F$. Then $\forall \alpha, \, \beta \in K$ such
  that $f(\alpha) = 0 = f(\beta)$, $\exists \phi \in \Gal(K / F) = \Gal(f(x))$
  such that $\phi(\alpha) = \beta$.
\end{crly}

\begin{proof}
  We shall use the \hyperref[lemma:isomorphism_extension_lemma]{Isomorphism
  Extension Lemma} to prove this.
  \begin{marginfigure}
    \centering
    \begin{tikzcd}
      K \arrow[rr, "\phi"]                   &                                        & K \arrow[dd, no head]        \\
                                             &                                        &                              \\
      F(\alpha) \arrow[uu, no head]          & \overset{\alpha \mapsto \beta}{\simeq} & F(\beta) \arrow[dd, no head] \\
                                             &                                        &                              \\
      F \arrow[rr, "id"] \arrow[uu, no head] &                                        & F                           
    \end{tikzcd}
    \caption{Constructing elements of the Galois Group}\label{fig:constructing_elements_of_the_galois_group}
  \end{marginfigure}
  Consider the identity as our isomorphism $\id : F \to F$. The Isomorphism
  Extension Lemma gives us the isomorphism that goes from $F(\alpha)$ to
  $F(\beta)$ by mapping $\alpha$ to $\beta$. We may thus define $\phi$ such that
  $\phi$ fixes $F$ and $\phi(\alpha) = \beta$, in $K$.
\end{proof}

\newthought{Permutation groups} that allows one to traverse all around the
indices, such as the Galois group, have a special name.

\begin{defn}[Transitive Subgroup]\index{Transitive Subgroup}\label{defn:transitive_subgroup}
  A subgroup $H \leq S_n$ is \hlnoteb{transitive} if $\forall i, j \in \{ 1,
  \ldots, n \}$, $\exists \pi \in H$ such that $\pi(i) = j$.
\end{defn}

\begin{crly}[The Galois Group of a Separable, Irreducible Polynomial is Transitive]\label{crly:the_galois_group_of_a_separable_irreducible_polynomial_is_transitive}
  Let $f(x) \in F[x]$, with $\deg f = n \geq 1$, be separable and irreducible.
  Then $\Gal(f(x)) \simeq H \leq S_n$ that is transitive.
\end{crly}

\begin{eg}
  Consider $G = \Gal(x^3 - 2)$ over $\mathbb{Q}[x]$.

  Since $f(x) = x^3 - 2$ is irreducible (by
  \hyperref[propo:eisenstein_s_criterion]{$2$-Eisenstein}) and $\ch \mathbb{Q} =
  0$, $f(x)$ is separable \sidenote{See A5Q3(d).}. It follows from
  \cref{crly:the_galois_group_of_a_separable_irreducible_polynomial_is_transitive}
  that $G \simeq H \leq S_3$ transitive.

  Let $\alpha_1, \, \alpha_2, \, \alpha_3$ be the roots of $f(x)$. Let $X = \{
  \alpha_1, \, \alpha_2, \, \alpha_3 \}$, and $G$ act on $X$ via $\phi \cdot
  \alpha_i = \phi(\alpha_i)$. By the
  \hyperref[thm:orbit_stabilizer_theorem]{Orbit-Stabilizer Theorem}, we have
  \begin{equation*}
    \abs{ G } = \abs{ \orb(\alpha_1) } \cdot \abs{ \stab(\alpha_1) } = 3 \cdot
    \abs{ \stab(\alpha_1) },
  \end{equation*}
  where we note that $\abs{ \orb(\alpha_1) }$ since all the orbits of $\alpha_1$
  are exactly elements of $X$. It follows that $3 \mid \abs{G}$. Since the only
  subgroups of $S_3$ that are divisible by $3$ are $A_3$ and $S_3$, we either
  have
  \begin{equation*}
    G \simeq A_3 \text{ or } G \simeq S_3.
  \end{equation*}

\end{eg}

We shall finish the rest of this example in the next lecture.

% section the_galois_group_as_a_permutation_group (end)

% chapter lecture_18_feb_25th (end)

\chapter{Lecture 19 Feb 27th}%
\label{chp:lecture_19_feb_27th}
% chapter lecture_19_feb_27th

\section{The Galois Group as a Permutation Group (Continued)}%
\label{sec:the_galois_group_as_a_permutation_group_continued}
% section the_galois_group_as_a_permutation_group_continued

We shall continue with the last example of the last lecture.

\begin{eg}
  We considered $G = \Gal(x^3 - 2)$ over $\mathbb{Q}[x]$, and showed that we
  either have
  \begin{equation*}
    G \simeq A_3 \text{ or } G \simeq S_3.
  \end{equation*}
  Recall that the roots of $f(x) = x^3 - 2$ are
  \begin{equation*}
    \alpha_1 = \sqrt[3]{2}, \, \alpha_2 = \alpha_1 \zeta_3, \, \alpha_3 =
    \alpha_1 \zeta_3^2.
  \end{equation*}
  Note that $f(x)$ is irreducible over $\mathbb{Q}(\zeta_3)$ \sidenote{Well,
  none of the roots of $f$ are in $\mathbb{Q}(\zeta_3)$, after all. Also, note
  that since $\alpha_1 \notin \mathbb{Q}(\zeta_3)$, $f$ remains the minimal
  polynomial of $\alpha_1$ over $\mathbb{Q}(\zeta_3)$, and so
  $\deg_{\mathbb{Q}(\zeta_3)} (\alpha_1) = 3$, but $[ \mathbb{Q}(\zeta_3) :
  \mathbb{Q} ] = 2$.}. By the
  \cref{crly:the_galois_group_completely_captures_all_permutation_of_the_roots},
  $\exists \phi \in G$ such that we have the relation as shown in
  \cref{fig:crly_galois_group_has_all_permutations_in_action}.
  \begin{figure}[ht]
    \centering
    \begin{tikzcd}
      \mathbb{Q}(\sqrt[3]{2}, \zeta_3) \arrow[rr, "\phi"]                            &  & \mathbb{Q}(\sqrt[3]{2}, \zeta_3) \arrow[dd, no head]        \\
                                                                                     &  & \\
      \mathbb{Q}(\zeta_3) \arrow[rr, "\zeta_3 \mapsto \zeta_3^2"] \arrow[uu,no head] &  & \mathbb{Q}(\zeta_3) \arrow[dd, no head]        \\
                                                                                     &  & \\
      \mathbb{Q} \arrow[rr, "id"] \arrow[uu, no head]                                &  & \mathbb{Q}
    \end{tikzcd}
    \caption[][20pt]{\cref{crly:the_galois_group_completely_captures_all_permutation_of_the_roots} in action}
    \label{fig:crly_galois_group_has_all_permutations_in_action}
  \end{figure}

  \sidenote{Note that $\zeta_3 \mapsto \zeta_3^2$ is a valid isomorphism,
  especially since they have the same minimal polynomial.} Note that
  \begin{gather*}
    \phi(\alpha_1) = \alpha_1 \\
    \phi(\alpha_2) = \phi(\alpha_1 \zeta_3) = \alpha_1 \zeta_3^2 = \alpha_3 \\
    \phi(\alpha_3) = \phi(\alpha_1 \zeta_3^2) = \alpha_1 \zeta_3 = \alpha_2
  \end{gather*}
  It thus follows that $\phi \sim (2 \; 3)$, a $2$-cycle, in $G$. Thus $\phi$ is
  an element of order $2$, which is an element that $A_3$ does not have. Thus $G
  \simeq S_3$.
\end{eg}

From the above example, we notice the following helpful observation.

\begin{remark}
  When computing $G = \Gal(K / F)$, it is often helpful to first know $\abs{G}$.
\end{remark}

Fortunately, in the finite dimensional world, $\abs{G}$ has an upper bound.

\begin{defn}[$F$-map]\index{$F$-map}\label{defn:_f_map}
  Let $K / F$ and $E / F$. Any \hlnotea{homomorphism} $\phi : K \to E$ which
  fixes $F$, i.e. $\phi \restriction_F = \id_F$, is called an
  \hlnoteb{$F$-map}.
\end{defn}

\begin{remark}
  Suppose $K / F$ and $E / F$, and $\phi : K \to E$ an $F$-map.
  \begin{enumerate}
    \item Since $\ker \phi \neq K$, we have $\ker \Phi = 0$ \sidenote{Note that
      in finite fields, $\ker \phi \in \{ \{ 0 \}, K \}$.}. Thus $\phi$ is
      \hlnotea{injective}.
    \item For any $\alpha \in F$, $v \in K$, $\phi(av) = \phi(a) \phi(v) = a
      \phi(v)$ since $\phi$ is a homomorphism. It follows that $\phi$ is a
      \hlnotea{linear transformation}.
    \item Let $\phi : K \to K$ be an $F$-map, and suppose $K$ is a
      finite-dimensional $F$-vector space with $[ K : F ] < \infty$. Then $\phi
      $ is surjective.

      It follows that $\phi : K \to K$ ($[K : F] < \infty$) is an $F$-map $\iff
      \phi \in \Gal(K / F)$.
  \end{enumerate}
\end{remark}

\begin{lemma}[Number of Distinct $F$-maps]\label{lemma:number_of_distinct_f_maps}
  Let $K / F$ and $E / F$, and suppose $K / F$ is a finite extension. The number
  of distinct $F$-maps from $K$ to $E$ is at most $[K : F]$.
\end{lemma}

\begin{proof}
  We shall do induction on the number of generators of $K / F$, which is also
  $[ K : F ] = n$, which is what we can iterate on. When $n = 1$, we have $K =
  F(\alpha_1)$ and $\phi : K \to E$ an $F$-map. Then the roots $\alpha_1$ and
  $\phi(\alpha_1)$ have the same minimal polynomial \sidenote{\hlwarn{Why?}}
  over $F$. Thus, there are at most $[ K : F ]$-many choices for
  $\phi(\alpha_1)$, meaning that there are at most $[ K : F ]$-many such
  $F$-maps.

  \marginnote{\WTS that the number of $F$-maps is at most $[K : F] = [ K : L ][
  L : F]$. We can get $[ L : F ]$ from the \hlnotea{induction hypothesis} and
  $[ K : L ]$ from an argument similar to the base case.}
  Continuing with this inductive line of thought, suppose that the statement is
  true for $K = F(\alpha_1, \ldots, \alpha_n)$ for some $n > 1$. Now let
  \begin{equation*}
    L = F(\alpha_1, \ldots, \alpha_{n - 1}), \text{ so that } K = L(\alpha_n).
  \end{equation*}
  Let $\phi : K \to E$ be an $F$-map. Note that $\phi \restriction_L : F \to E$
  is still an $F$-map. By the induction hypothesis, the number of choices for
  $\phi \restriction_L$ is at most $[ L : F ]$. Since $\phi$ is completely
  determined by $\phi \restriction_L$ and $\phi(\alpha_n)$, there are,
  therefore, at most
  \begin{equation*}
    [L : F][ L(\alpha_n) : L ] = [ K : F ]\text{-many}
  \end{equation*}
  choices for $\phi$, following the Tower Theorem.
\end{proof}

The following corollary follows immediately from the realization that $F$-maps
going from $K \to K$ are exactly the elements of the Galois group $\Gal(K / F)$.

\begin{crly}[Upper Bound for the Galois Group of Finite Extensions]\label{crly:upper_bound_for_the_galois_group_of_finite_extensions}
  If $K / F$ is finite, then
  \begin{equation*}
    \abs{ \Gal(K / F) } \leq [ K : F ].
  \end{equation*}
\end{crly}

\begin{warning}
  There are extensions $K$ of a field $F$ such that $\Gal(K / F) < [ K : F ]$.
  \begin{enumerate}
    \item We saw in an earlier example that $G = \Gal( \mathbb{Q}(\sqrt[3]{2}) :
      \mathbb{Q} ) = \{ 1 \}$, but $[ \mathbb{Q}(\sqrt[3]{2}) : \mathbb{Q} ] =
      3$, and $\sqrt[3]{2} \zeta_3, \, \sqrt[3]{2} \zeta_3^2 \notin
      \mathbb{Q}(\sqrt[3]{2})$. In this case, the Galois group is too tiny.
    \item Consider $G = \Gal( \mathbb{Z}_2(x) / \mathbb{Z}_2(t^2) )$. Note that
      $[\mathbb{Z}_2(x) : \mathbb{Z}_2(t^2)] = 2$, since the minimal polynomial
      of $t$ in $\mathbb{Z}_2(t^2)[x]$ is
      \begin{equation*}
        x^2 - t^2 = (x - t)^2 \in \mathbb{Z}_2(t)[x].
      \end{equation*}
      Thus if $\phi \in G$, then it is necesssary that $\phi(t) = t$, implying
      that $G = \{ 1 \}$.

      In this case, it is because $t$ is a root with multiplicity $> 1$.
  \end{enumerate}
\end{warning}

% section the_galois_group_as_a_permutation_group_continued (end)

% chapter lecture_19_feb_27th (end)

\chapter{Lecture 20 Mar 01st}%
\label{chp:lecture_20_mar_01st}
% chapter lecture_20_mar_01st

\section{Galois Group of Separable Fields}%
\label{sec:galois_group_of_separable_fields}
% section galois_group_of_separable_fields

\begin{quotebox}{green}{foreground}
  So when exactly does $\abs{ \Gal(K / F) } = [ K : F ]$?
\end{quotebox}

\begin{defn}[Separable Elements and Separable Extensions]\index{Separable Elements}\index{Separable Extensions}\label{defn:separable_elements_and_separable_extensions}
  Let $K / F$ \sidenote{This need not be a finite extension.}. We say that
  $\alpha \in K$ is \hlnoteb{separable} if $\alpha$ is \hlnotea{algebraic} over
  $F$ and its minimal polynomial is
  \hyperref[defn:separable_polynomials]{separable} (over $F$) \sidenote{This
  also means that the root is unique.}.

  We say that the extension $K / F$ is \hlnoteb{separable} if $K / F$ is
  \hyperref[defn:algebraic_and_transcendental]{algebraic} and $\forall \alpha
  \in K$, $\alpha$ is separable over $F$.
\end{defn}

\marginnote{
\begin{remark}
  \cref{defn:perfect_fields} means that all polynomials over the field are
  separable, i.e. they do not have repeated roots.
\end{remark}}
\begin{defn}[Perfect Fields]\index{Perfect Fields}\label{defn:perfect_fields}
  We say that a field $F$ is \hlnoteb{perfect} if every algebraic extension of
  $F$ is separable.
\end{defn}

\begin{note}
  Recall from A5, we showed that given an irreducible $f(x) \in F[x]$,
  \begin{equation*}
    f(x) \text{ is separable } \iff f'(x) \neq 0.
  \end{equation*}
\end{note}

\begin{propo}[Separability and the Characteristic of a Field]\label{propo:separability_and_the_characteristic_of_a_field}
  Let $f(x) \in F[x]$ be irreducible.
  \begin{enumerate}
    \item If $\ch F = 0$, then $f(x)$ is separable. \sidenote{This is proven in
      A5.}
    \item If $\ch F = p$ prime, then $f(x)$ is not separable iff $f(x) = g(x^p)$
      for some $g(x) \in F[x]$.
  \end{enumerate}
\end{propo}

\begin{proof}
  \begin{enumerate}
    \setcounter{enumi}{1}
    \item Let
      \begin{equation*}
        f(x) \in a_n x^n + a_{n - 1} x^{n - 1} + \hdots + a_1 x + a_0
      \end{equation*}
      Then $f(x)$ is not separable \\
      $\iff$ $f'(x) = 0$ \\
      $\iff$ $na_n x^{n - 1} + (n - 1)a_{n - 1} x^{n - 2} + \hdots + a_1 = 0$
        \\
      $\iff$ $ka_k = 0$ for $k \in \{ 1, \, \ldots, \, n \}$ \\
      $\iff$ $ka_k = pm_k a_k$ where $m_k \in \mathbb{N}$, $k \in \{ 1, \ldots,
        n \}$ since either $p \mid k$ or $a_k = 0$ \\
      $\iff$ $f(x) = a_n x^{m_n p} + a_{n - 1}x^{m_{n - 1} p} + \hdots + a_1
        x^{m_1 p} + a_0$ \\
      $\iff$ $f(x) = g(x^p)$ where
      \begin{equation*}
        g(x) = a_n x^{m_n} + a_{n - 1} x^{m_{n - 1}} + \hdots + a_1 x^{m_1} +
        a_0.
      \end{equation*}
  \end{enumerate}
\end{proof}

\begin{crly}[Fields of Characteristic Zero are Perfect]\label{crly:fields_of_characteristic_zero_are_perfect}
  If $\ch F = 0$, then $F$ is perfect.
\end{crly}

\begin{eg}
  Note that in $\mathbb{Z}_2(t) / \mathbb{Z}_2(t^2)$, we have that
  \begin{equation*}
    x^2 - t^2 = (x - t)^2,
  \end{equation*}
  i.e. $t$ is a root with multiplicity $2$. Thus $\mathbb{Z}_2(t^2)$ is not
  perfect.
\end{eg}

\begin{crly}[Every Finite Field is Perfect]\label{crly:every_finite_field_is_perfect}
  Every finite field $F$ is perfect.
\end{crly}

\marginnote{
\begin{strategy}
  Of course, we want to use
  \cref{propo:separability_and_the_characteristic_of_a_field}. We can do so by
  supposing that $f(x)$ is irreducible but not separable, which then forces $f(x) =
  g(x^p)$. The important point here is to notice that in a finite field, all
  elements of the field will eventually cycle back as we add or multiply them.
  Then, by using the fact that $\ch F = p$ is prime, in particular by the
  \hlnotea{Freshman's Dream}, we can use \hldefn{Frobenius's Homomorphism}
  $\phi(a) = a^p$, and we end up showing that every element in $F$ is some other
  element of $F$ with power $p$. This will cause $f(x)$ to become reducible due
  to the Freshman's Dream.
\end{strategy}}

\begin{proof}
  Let $F$ be finite with $\ch F = p > 0$ \sidenote{Note that fields of
  characteristic $0$ must be infinite, so this is a valid assumption.}. Suppose
  to the contrary that $\exists f(x) \in F[x]$ such that $f(x)$ is irreducible
  but not separable. Then $\exists g(x) \in F[x]$ such that $f(x) = g(x^p)$. In
  particular, we have
  \begin{equation*}
    f(x) = a_n x^{p m_n} + a_{n - 1} x^{p m_{n - 1}} + \hdots + a_1 x^{p m_1} +
    a_0.
  \end{equation*}

  Now consider $\phi : F \to F$ given by $\phi(a) = a^p$. By the
  \hlnotea{Freshman's Dream}, $\phi$ is a homomorphism. It is clear that it is
  injective since if $a \neq b$, then $a^p \neq b^p$, for otherwise
  \begin{equation*}
    0 = a^p - b^p = (a - b)^p \iff 0 = a - b \iff a = b.
  \end{equation*}
  Also, since $F$ is finite, injectivity of $\phi$ guarantees that it is
  surjective. This means that $\forall a_k \in F$, $\exists b_k \in F$ such that
  \begin{equation*}
    a_k = b_k^p = \phi(b_k).
  \end{equation*}
  Then we have
  \begin{align*}
    f(x) &= b_n^p x^{p m_n} + b_{n - 1}^p x^{p m_{n - 1}} + \hdots + b_1^p x^{p
          m_1} + b_0^p \\
         &= ( b_n x^{m_n} + b_{n - 1} x^{m_{n - 1}} + \hdots + b_1 x^{m_1} + b_0
          )^p,
  \end{align*}
  again, by the \hlnotea{Freshman's Dream}. Therefore $f(x)$ is irreducible,
  contradicting our assumption.
\end{proof}

\begin{thm}[Galois Group of a Splitting Field of a Separable Polynomial has Order the Degree of the Extension]\label{thm:galois_group_of_a_splitting_field_of_a_separable_polynomial_has_Order_the_degree_of_the_extension}
  Let $f(x) \in F[x]$ be non-constant and separable. Let $K$ be the splitting
  field of $f(x)$ over $F$. Then
  \begin{equation*}
    \abs{ \Gal(K / F) } = \abs{ \Gal(f(x)) } = [ K : F ].
  \end{equation*}
\end{thm}

\begin{proof}
  We shall perform induction on $[ K : F ] = n$.

  \noindent
  \hlbnoted{$n = 1$} We have
  \begin{equation*}
    1 \leq \abs{ \Gal(K / F) } \leq [ K : F ] \leq 1,
  \end{equation*}
  since we always have $\epsilon \in \Gal(K / F)$.
  
  Proceeding inductively...
  
  \noindent
  \hlbnoted{$n = k + 1$} Let $p(x) \in F[x]$ be an irreducible factor of $f(x)$
  \sidenote{Note that it suffices for us to show for irreducible polynomials,
  since we can always factor a polynomial into irreducible terms.}. Note that
  $p(x)$ is also separable over $F$. Let
  \begin{equation*}
    \alpha_1, \ldots, \alpha_m \in K
  \end{equation*}
  be the roots of $p(x)$, where $m = \deg p(x)$, and we note that $\alpha_i \neq
  \alpha_j$ for all $i \neq j$ since $p(x)$ is separable. Now since $[ K : F ] >
  1$, wma $\alpha_1 \notin F$. Then consider $E = F(\alpha_1)$. Since $p(x)$ is
  irreducible in $F[x]$, it follows that $[ E : F ] = m$. Thus by the
  \hyperref[thm:tower_theorem]{Tower Theorem}, we have
  \begin{equation*}
    [ K : E ] = \frac{[ K : F ]}{[ E : F ]} = \frac{n}{m} < n.
  \end{equation*}
  Note that we still have $K$ as the splitting field of $f(x)$ over $E$. It
  follows from induction that
  \begin{equation}\label{eq:galois_group_of_splitting_field_of_separable_polym_1}
    \abs{ \Gal( K / E ) } = [ K : E ] = \frac{n}{m}.
  \end{equation}
  Since $p(x)$ is irreducible, by the
  \hyperref[lemma:isomorphism_extension_lemma]{Isomorphism Extension Lemma},
  $\forall j$, $\exists \phi_j \in \Gal(K / F)$ such that $\phi_j(\alpha_1) =
  \alpha_j$. Since the roots are distinct, it follows that each of the
  $\phi_j$'s are distinct in $\Gal (K / F)$, and there are $m$-many such
  automorphisms.

  Furthermore, we have that $\phi_j^{-1} \phi_i (\alpha_1) \neq \alpha_1 \in E$,
  and so $\phi_j^{-1} \phi_i \notin \Gal(K / E)$. This means that
  \begin{equation*}
    \phi_j \Gal(K / E) \neq \phi_i \Gal(K / E),
  \end{equation*}
  and so we have that there must be
  \begin{equation*}
    \abs{ \Gal(K / F) / \Gal(K / E) } \geq m.
  \end{equation*}
  By \hyperref[thm:lagrange_s_theorem]{Lagrange}, we have from
  \cref{eq:galois_group_of_splitting_field_of_separable_polym_1} that
  \begin{equation*}
    \abs{ \Gal(K / F) } \geq m \cdot \abs{ \Gal(K / E) } = m \cdot \frac{n}{m} =
    n,
  \end{equation*}
  as desired.
\end{proof}

% section galois_group_of_separable_fields (end)

% chapter lecture_20_mar_01st (end)

\appendix

\chapter{Asides and Prior Knowledge}%
\label{chp:asides_and_prior_knowledge}
% chapter asides_and_prior_knowledge

\section{Correspondence Theorem}%
\label{sec:correspondence_theorem}
% section correspondence_theorem

\hlnotea{The Correspondence Theorem} is somewhat widely known 
as the Fourth Isomorphism Theorem, although some authors associates
the name with a proposition known as 
\href{https://en.wikipedia.org/wiki/Zassenhaus_lemma}{Zaessenhaus Lemma}.

\begin{thm}[Correspondence Theorem]\index{Correspondence Theorem}\label{thm:correspondence_theorem}
  Let $G$ be a group, and $N \triangleleft G$
  \sidenote{Recall that this symbol means that $N$ is a normal
  subgroup of $G$.}. Then there exists a bijection between
  the set of all subgroups $A \leq G$ such that $A \supseteq N$
  and the set of subgroups $A / N$ of $G / N$.
\end{thm}

\begin{proof}
  % TODO: prove the Correspondence Theorem
\end{proof}

% section correspondence_theorem (end)

% chapter asides_and_prior_knowledge (end)

\backmatter

\pagestyle{plain}

\nobibliography*
\bibliography{references}

\printindex

\end{document}
% vim: tw=80
