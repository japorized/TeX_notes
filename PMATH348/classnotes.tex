% !TEX TS-program = lualatex
\documentclass[notoc,notitlepage]{tufte-book}
% \nonstopmode % uncomment to enable nonstopmode

\usepackage{classnotetitle}

\title{PMATH348 --- Fields and Galois Theory}
\author{Johnson Ng}
\subtitle{Classnotes for Winter 2019}
\credentials{BMath (Hons), Pure Mathematics major, Actuarial Science Minor}
\institution{University of Waterloo}

\usepackage{fontspec-luatex}
\setcounter{secnumdepth}{3}
\setcounter{tocdepth}{3}

\renewcommand{\baselinestretch}{1.1}
\usepackage{geometry}
\geometry{letterpaper}
\usepackage[parfill]{parskip}
\usepackage{graphicx}

% Essential Packages
\usepackage{makeidx}
\makeindex
\usepackage{enumitem}
\usepackage[T1]{fontenc}
\usepackage{natbib}
\bibliographystyle{apalike}
\usepackage{ragged2e}
\usepackage{etoolbox}
\usepackage{amssymb}
\usepackage{fontawesome}
\usepackage{amsmath}
\usepackage{mathrsfs}
\usepackage{mathtools}
\usepackage{xparse}
\usepackage{tkz-euclide}
\usetkzobj{all}
\usepackage[utf8]{inputenc}
\usepackage{csquotes}
\usepackage[english]{babel}
\usepackage{marvosym}
\usepackage{pgf,tikz}
\usepackage{pgfplots}
\usepackage{fancyhdr}
\usepackage{array}
\usepackage{faktor}
\usepackage{float}
\usepackage{xcolor}
\usepackage{centernot}
\usepackage{silence}
  \WarningFilter*{latex}{Marginpar on page \thepage\space moved}
\usepackage{tcolorbox}
\tcbuselibrary{skins,breakable}
\usepackage{longtable}
\usepackage[amsmath,hyperref]{ntheorem}
\usepackage{hyperref}
\usepackage[noabbrev,capitalize,nameinlink]{cleveref}

% xcolor (scheme: base16 eighties)
\definecolor{base16-eighties-dark}{HTML}{2D2D2D}
\definecolor{base16-eighties-light}{HTML}{D3D0C8}
\definecolor{base16-eighties-magenta}{HTML}{CD98CD}
\definecolor{base16-eighties-red}{HTML}{F47678}
\definecolor{base16-eighties-yellow}{HTML}{E2B552}
\definecolor{base16-eighties-green}{HTML}{98CD97}
\definecolor{base16-eighties-lightblue}{HTML}{61CCCD}
\definecolor{base16-eighties-blue}{HTML}{6498CE}
\definecolor{base16-eighties-brown}{HTML}{D47B4E}
\definecolor{base16-eighties-gray}{HTML}{747369}

% hyperref Package Settings
\hypersetup{
    bookmarks=true,         % show bookmarks bar?
    unicode=true,          % non-Latin characters in Acrobat’s bookmarks
    pdftoolbar=false,        % show Acrobat’s toolbar?
    pdfmenubar=false,        % show Acrobat’s menu?
    pdffitwindow=true,     % window fit to page when opened
    colorlinks=true,
    allcolors=base16-eighties-magenta,
}

% tikz
\usepgfplotslibrary{polar}
\usepgflibrary{shapes.geometric}
\usetikzlibrary{angles,patterns,calc,decorations.markings}
\tikzset{midarrow/.style 2 args={
        decoration={markings,
            mark= at position #2 with {\arrow{#1}} ,
        },
        postaction={decorate}
    },
    midarrow/.default={latex}{0.5}
}
\def\centerarc[#1](#2)(#3:#4:#5)% Syntax: [draw options] (center) (initial angle:final angle:radius)
    { \draw[#1] ($(#2)+({#5*cos(#3)},{#5*sin(#3)})$) arc (#3:#4:#5); }

% enumitems
\newlist{inlinelist}{enumerate*}{1}
\setlist*[inlinelist,1]{%
  label=(\roman*),
}

% Theorem Style Customization
\setlength\theorempreskipamount{2ex}
\setlength\theorempostskipamount{3ex}

\makeatletter
\let\nobreakitem\item
\let\@nobreakitem\@item
\patchcmd{\nobreakitem}{\@item}{\@nobreakitem}{}{}
\patchcmd{\nobreakitem}{\@item}{\@nobreakitem}{}{}
\patchcmd{\@nobreakitem}{\@itempenalty}{\@M}{}{}
\patchcmd{\@xthm}{\ignorespaces}{\nobreak\ignorespaces}{}{}
\patchcmd{\@ythm}{\ignorespaces}{\nobreak\ignorespaces}{}{}

\renewtheoremstyle{break}%
  {\item{\theorem@headerfont
          ##1\ ##2\theorem@separator}\hskip\labelsep\relax\nobreakitem}%
  {\item{\theorem@headerfont
          ##1\ ##2\ (##3)\theorem@separator}\hskip\labelsep\relax\nobreakitem}
\makeatother

% ntheorem + framed
\makeatletter

% ntheorem Declarations
\theorempreskip{10pt}
\theorempostskip{5pt}
\theoremstyle{break}

\newtheorem*{solution}{\faPencil $\enspace$ Solution}
\newtheorem*{remark}{Remark}
\newtheorem{eg}{Example}[section]
\newtheorem{ex}{Exercise}[section]

    % definition env
\theoremprework{\textcolor{base16-eighties-blue}{\hrule height 2pt}}
\theoremheaderfont{\color{base16-eighties-blue}\normalfont\bfseries}
\theorempostwork{\textcolor{base16-eighties-blue}{\hrule height 2pt}}
\theoremindent10pt
\newtheorem{defn}{\faBook \enspace Definition}

    % definition env no num
\theoremprework{\textcolor{base16-eighties-blue}{\hrule height 2pt}}
\theoremheaderfont{\color{base16-eighties-blue}\normalfont\bfseries}
\theorempostwork{\textcolor{base16-eighties-blue}{\hrule height 2pt}}
\theoremindent10pt
\newtheorem*{defnnonum}{\faBook \enspace Definition}

    % theorem envs
\theoremprework{\textcolor{base16-eighties-magenta}{\hrule height 2pt}}
\theoremheaderfont{\color{base16-eighties-magenta}\normalfont\bfseries}
\theorempostwork{\textcolor{base16-eighties-magenta}{\hrule height 2pt}}
\theoremindent10pt
\newtheorem{thm}{\faCoffee \enspace Theorem}

\theoremprework{\textcolor{base16-eighties-magenta}{\hrule height 2pt}}
\theorempostwork{\textcolor{base16-eighties-magenta}{\hrule height 2pt}}
\theoremindent10pt
\newtheorem{propo}[thm]{\faTint \enspace Proposition}

\theoremprework{\textcolor{base16-eighties-magenta}{\hrule height 2pt}}
\theorempostwork{\textcolor{base16-eighties-magenta}{\hrule height 2pt}}
\theoremindent10pt
\newtheorem{crly}[thm]{\faSpaceShuttle \enspace Corollary}

\theoremprework{\textcolor{base16-eighties-magenta}{\hrule height 2pt}}
\theorempostwork{\textcolor{base16-eighties-magenta}{\hrule height 2pt}}
\theoremindent10pt
\newtheorem{lemma}[thm]{\faTree \enspace Lemma}

\theoremprework{\textcolor{base16-eighties-magenta}{\hrule height 2pt}}
\theorempostwork{\textcolor{base16-eighties-magenta}{\hrule height 2pt}}
\theoremindent10pt
\newtheorem{axiom}[thm]{\faShield \enspace Axiom}

    % theorem envs without counter
\theoremprework{\textcolor{base16-eighties-magenta}{\hrule height 2pt}}
\theoremheaderfont{\color{base16-eighties-magenta}\normalfont\bfseries}
\theorempostwork{\textcolor{base16-eighties-magenta}{\hrule height 2pt}}
\theoremindent10pt
\newtheorem*{thmnonum}{\faCoffee \enspace Theorem}

\theoremprework{\textcolor{base16-eighties-magenta}{\hrule height 2pt}}
\theorempostwork{\textcolor{base16-eighties-magenta}{\hrule height 2pt}}
\theoremindent10pt
\newtheorem*{propononum}{\faTint \enspace Proposition}

\theoremprework{\textcolor{base16-eighties-magenta}{\hrule height 2pt}}
\theorempostwork{\textcolor{base16-eighties-magenta}{\hrule height 2pt}}
\theoremindent10pt
\newtheorem*{crlynonum}{\faSpaceShuttle \enspace Corollary}

\theoremprework{\textcolor{base16-eighties-magenta}{\hrule height 2pt}}
\theorempostwork{\textcolor{base16-eighties-magenta}{\hrule height 2pt}}
\theoremindent10pt
\newtheorem*{lemmanonum}{\faTree \enspace Lemma}

\theoremprework{\textcolor{base16-eighties-magenta}{\hrule height 2pt}}
\theorempostwork{\textcolor{base16-eighties-magenta}{\hrule height 2pt}}
\theoremindent10pt
\newtheorem*{axiomnonum}{\faShield \enspace Axiom}

    % proof env
\theoremprework{\textcolor{base16-eighties-brown}{\hrule height 2pt}}
\theoremheaderfont{\color{base16-eighties-brown}\normalfont\bfseries}
\theorempostwork{\textcolor{base16-eighties-brown}{\hrule height 2pt}}
\newtheorem*{proof}{\faPencil \enspace Proof}

    % note and notation env
\theoremprework{\textcolor{base16-eighties-yellow}{\hrule height 2pt}}
\theoremheaderfont{\color{base16-eighties-yellow}\normalfont\bfseries}
\theorempostwork{\textcolor{base16-eighties-yellow}{\hrule height 2pt}}
\newtheorem*{note}{\faQuoteLeft \enspace Note}

\theoremprework{\textcolor{base16-eighties-yellow}{\hrule height 2pt}}
\theorempostwork{\textcolor{base16-eighties-yellow}{\hrule height 2pt}}
\newtheorem*{notation}{\faPaw \enspace Notation}

    % warning env
\theoremprework{\textcolor{base16-eighties-red}{\hrule height 2pt}}
\theoremheaderfont{\color{base16-eighties-red}\normalfont\bfseries}
\theorempostwork{\textcolor{base16-eighties-red}{\hrule height 2pt}}
\theoremindent10pt
\newtheorem*{warning}{\faBug \enspace Warning}

% more environments
\newtcolorbox{redquote}{
  blanker,enhanced,breakable,standard jigsaw,
  opacityback=0,
  coltext=base16-eighties-light,
  left=5mm,right=5mm,top=2mm,bottom=2mm,
  colframe=base16-eighties-red,
  boxrule=0pt,leftrule=3pt,
  fontupper=\itshape
}
\newtcolorbox{bluequote}{
  blanker,enhanced,breakable,standard jigsaw,
  opacityback=0,
  coltext=base16-eighties-light,
  left=5mm,right=5mm,top=2mm,bottom=2mm,
  colframe=base16-eighties-blue,
  boxrule=0pt,leftrule=3pt,
  fontupper=\itshape
}
\newtcolorbox{greenquote}{
  blanker,enhanced,breakable,standard jigsaw,
  opacityback=0,
  coltext=base16-eighties-light,
  left=5mm,right=5mm,top=2mm,bottom=2mm,
  colframe=base16-eighties-green,
  boxrule=0pt,leftrule=3pt,
  fontupper=\itshape
}
\newtcolorbox{yellowquote}{
  blanker,enhanced,breakable,standard jigsaw,
  opacityback=0,
  coltext=base16-eighties-light,
  left=5mm,right=5mm,top=2mm,bottom=2mm,
  colframe=base16-eighties-yellow,
  boxrule=0pt,leftrule=3pt,
  fontupper=\itshape
}
\newtcolorbox{magentaquote}{
  blanker,enhanced,breakable,standard jigsaw,
  opacityback=0,
  coltext=base16-eighties-light,
  left=5mm,right=5mm,top=2mm,bottom=2mm,
  colframe=base16-eighties-magenta,
  boxrule=0pt,leftrule=3pt,
  fontupper=\itshape
}

% ntheorem listtheorem style
\makeatother
\newlength\widesttheorem
\AtBeginDocument{
  \settowidth{\widesttheorem}{Proposition A.1.1.1\quad}
}

\makeatletter
\def\thm@@thmline@name#1#2#3#4{%
        \@dottedtocline{-2}{0em}{2.3em}%
                   {\makebox[\widesttheorem][l]{#1 \protect\numberline{#2}}#3}%
                   {#4}}
\@ifpackageloaded{hyperref}{
\def\thm@@thmline@name#1#2#3#4#5{%
    \ifx\#5\%
        \@dottedtocline{-2}{0em}{2.3em}%
            {\makebox[\widesttheorem][l]{#1 \protect\numberline{#2}}#3}%
            {#4}
    \else
        \ifHy@linktocpage\relax\relax
            \@dottedtocline{-2}{0em}{2.3em}%
                {\makebox[\widesttheorem][l]{#1 \protect\numberline{#2}}#3}%
                {\hyper@linkstart{link}{#5}{#4}\hyper@linkend}%
        \else
            \@dottedtocline{-2}{0em}{2.3em}%
                {\hyper@linkstart{link}{#5}%
                  {\makebox[\widesttheorem][l]{#1 \protect\numberline{#2}}#3}\hyper@linkend}%
                    {#4}%
        \fi
    \fi}
}

\makeatletter
\def\thm@@thmline@noname#1#2#3#4{%
        \@dottedtocline{-2}{0em}{5em}%
                   {{\protect\numberline{#2}}#3}%
                   {#4}}
\@ifpackageloaded{hyperref}{
\def\thm@@thmline@noname#1#2#3#4#5{%
    \ifx\#5\%
        \@dottedtocline{-2}{0em}{5em}%
            {{\protect\numberline{#2}}#3}%
            {#4}
    \else
        \ifHy@linktocpage\relax\relax
            \@dottedtocline{-2}{0em}{5em}%
                {{\protect\numberline{#2}}#3}%
                {\hyper@linkstart{link}{#5}{#4}\hyper@linkend}%
        \else
            \@dottedtocline{-2}{0em}{5em}%
                {\hyper@linkstart{link}{#5}%
                  {{\protect\numberline{#2}}#3}\hyper@linkend}%
                    {#4}%
        \fi
    \fi}
}

\theoremlisttype{allname}

\AtBeginDocument{\renewcommand\contentsname{Table of Contents}}

% Heading formattings
% chapter format
\titleformat{\chapter}%
  {\huge\rmfamily\itshape\color{base16-eighties-magenta}}% format applied to label+text
  {\llap{\colorbox{base16-eighties-magenta}{\parbox{1.5cm}{\hfill\itshape\huge\textcolor{base16-eighties-dark}{\thechapter}}}}}% label
  {5pt}% horizontal separation between label and title body
  {}% before the title body
  []% after the title body

% section format
\titleformat{\section}%
  {\normalfont\Large\rmfamily\itshape\color{base16-eighties-blue}}% format applied to label+text
  {\llap{\colorbox{base16-eighties-blue}{\parbox{1.5cm}{\hfill\itshape\textcolor{base16-eighties-dark}{\thesection}}}}}% label
  {5pt}% horizontal separation between label and title body
  {}% before the title body
  []% after the title body

% subsection format
\titleformat{\subsection}%
  {\normalfont\large\itshape\color{base16-eighties-green}}% format applied to label+text
  {\llap{\colorbox{base16-eighties-green}{\parbox{1.5cm}{\hfill\textcolor{base16-eighties-dark}{\thesubsection}}}}}% label
  {1em}% horizontal separation between label and title body
  {}% before the title body
  []% after the title body

% Sidenote enhancements
\def\mathmarginnote#1{%
  \tag*{\rlap{\hspace\marginparsep\smash{\parbox[t]{\marginparwidth}{%
  \footnotesize#1}}}}
}

% Custom table columning
\newcolumntype{L}[1]{>{\raggedright\let\newline\\\arraybackslash\hspace{0pt}}m{#1}}
\newcolumntype{C}[1]{>{\centering\let\newline\\\arraybackslash\hspace{0pt}}m{#1}}
\newcolumntype{R}[1]{>{\raggedleft\let\newline\\\arraybackslash\hspace{0pt}}m{#1}}

% Custom math operator
% \DeclareMathOperator{\rem}{rem}
\DeclareMathOperator*{\argmax}{arg\,max}
\DeclareMathOperator*{\argmin}{arg\,min}
\DeclareMathOperator{\re}{Re}
\DeclareMathOperator{\im}{Im}
\DeclareMathOperator{\caparg}{Arg}
\DeclareMathOperator{\Ind}{Ind}
\DeclareMathOperator{\Res}{Res}

% Graph styles
\pgfplotsset{compat=1.15}
\usepgfplotslibrary{fillbetween}
\pgfplotsset{four quads/.append style={axis x line=middle, axis y line=
middle, xlabel={$x$}, ylabel={$y$}, axis equal }}
\pgfplotsset{four quad complex/.append style={axis x line=middle, axis y line=
middle, xlabel={$\re$}, ylabel={$\im$}, axis equal }}

% Shortcuts
\newcommand{\floor}[1]{\lfloor #1 \rfloor}      % simplifying the writing of a floor function
\newcommand{\ceiling}[1]{\lceil #1 \rceil}      % simplifying the writing of a ceiling function
\newcommand{\dotp}{\, \cdotp}			        % dot product to distinguish from \cdot
\newcommand{\qed}{\hfill\ensuremath{\square}}   % Q.E.D sign
\newcommand{\abs}[1]{\left|#1\right|}						% absolute value
\newcommand{\lra}[1]{\langle \; #1 \; \rangle}
\newcommand{\at}[2]{\Big|_{#1}^{#2}}
\newcommand{\Arg}[1]{\caparg #1}
\renewcommand{\bar}[1]{\mkern 1.5mu \overline{\mkern -1.5mu #1 \mkern -1.5mu} \mkern 1.5mu}
\newcommand{\quotient}[2]{\faktor{#1}{#2}}
\newcommand{\cyclic}[1]{\left\langle #1 \right\rangle}
	% highlighting shortcuts
\newcommand{\hlimpo}[1]{\textcolor{base16-eighties-red}{\textbf{#1}}}
\newcommand{\hlwarn}[1]{\textcolor{base16-eighties-yellow}{\textbf{#1}}}
\newcommand{\hldefn}[1]{\textcolor{base16-eighties-blue}{\index{#1}\textbf{#1}}}
\newcommand{\hlnotea}[1]{\textcolor{base16-eighties-green}{\textbf{#1}}}
\newcommand{\hlnoteb}[1]{\textcolor{base16-eighties-lightblue}{\textbf{#1}}}
\newcommand{\hlnotec}[1]{\textcolor{base16-eighties-brown}{\textbf{#1}}}
\newcommand{\WTP}{\textcolor{base16-eighties-brown}{WTP} }
\newcommand{\WTS}{\textcolor{base16-eighties-brown}{WTS} }
\newcommand{\ind}[2]{\Ind_{#2}\left( #1 \right)}
\newcommand{\notimply}{\centernot\implies}
\newcommand{\res}[2]{\underset{#2}{\Res} #1 }
\newcommand{\tworow}[3]{\begin{tabular}{@{}#1@{}} #2 \\ #3 \end{tabular}}
\renewcommand{\epsilon}{\varepsilon}
\newcommand{\lrarrow}{\leftrightarrow}
\newcommand{\larrow}{\leftarrow}
\newcommand{\rarrow}{\rightarrow}
\renewcommand{\atop}[2]{\genfrac{}{}{0pt}{}{#1}{#2}}
\newcommand*\dif{\mathop{}\!d}

  % inspiration from: https://tex.stackexchange.com/questions/8720/overbrace-underbrace-but-with-an-arrow-instead#37758
\newcommand{\overarrow}[2]{
  \overset{\makebox[0pt]{\begin{tabular}{@{}c@{}}#2\\[0pt]\ensuremath{\uparrow}\end{tabular}}}{#1}
}
\newcommand{\underarrow}[2]{
  \underset{\makebox[0pt]{\begin{tabular}{@{}c@{}}\downarrow\\[0pt]\ensuremath{#2}\end{tabular}}}{#1}
}

% Document header formatting
\renewcommand{\chaptermark}[1]{\markboth{#1}{}}
\renewcommand{\sectionmark}[1]{\markright{#1}}
\makeatletter
\pagestyle{fancy}
\fancyhead{}
\fancyhead[RO]{\textsl{\@title} \enspace \thepage}
\fancyhead[LE]{\thepage \enspace \textsl{\leftmark \enspace - \enspace \rightmark}}
\makeatother

% Comment the two lines below if you want to print the document
\pagecolor{base16-eighties-dark}
\color{base16-eighties-light}

\setmainfont[Ligatures=TeX]{Times Newer Roman}
\setsansfont[Ligatures=TeX]{Helvetica Neue LT Std}

\DeclareMathOperator{\stab}{stab}
\DeclareMathOperator{\orb}{orb}
\DeclareMathOperator{\ch}{ch}
\DeclareMathOperator{\Span}{span}

\begin{document}
\hypersetup{pageanchor=false}
\maketitle
\hypersetup{pageanchor=true}
\begin{fullwidth}
\tableofcontents
\end{fullwidth}

\newpage
\begin{fullwidth}
  \renewcommand{\listtheoremname}{\faBook\ \slshape List of Definitions}
  \listoftheorems[ignoreall,show={defn}]
  \addcontentsline{toc}{chapter}{List of Definitions}
\end{fullwidth}

\newpage 
\begin{fullwidth}
  \renewcommand{\listtheoremname}{\faCoffee\ \slshape List of Theorems}
  \listoftheorems[ignoreall,
    show={axiom,lemma,thm,crly,propo,marginthm,marginpropo,marginlemma,marginaxiom,margincrly}
  ]
  \addcontentsline{toc}{chapter}{List of Theorems}
\end{fullwidth}


\chapter*{Preface}%
\addcontentsline{toc}{chapter}{Preface}
\label{chp:preface}
% chapter preface

This is a 3 part course; it is separated into
\begin{enumerate}
  \item \textbf{Sylow's Theorem}

    which is a leftover from group theory (\href{https://tex.japorized.ink/PMATH347S18/classnotes.pdf}{PMATH 347}). It has little to do with the rest of the course, but PMATH 347 was a course that is already content-rich to a point where Sylow's Theorem gets pushed into the later course that is this course.

  \item \textbf{Field Theory}

    is a somewhat understood concept from ring theory, where we learned that it is a special case of a ring where all of its elements have an inverse.

  \item \textbf{Galois Theory}

    is the beautiful theory from the French mathematican Évariste Galois that ties field theory back to group theory. This allows us to reduce certain field theory problems into group theory, which, in some sense, is easier and better understood.
\end{enumerate}

% chapter preface (end)

\tuftepart{Sylow's Theorem}

\chapter{Lecture 1 Jan 07th}%
\label{chp:lecture_1_jan_07th}
% chapter lecture_1_jan_07th

\section{Cauchy's Theorem}%
\label{sec:cauchy_s_theorem}
% section cauchy_s_theorem

Recall Lagrange's Theorem.

\begin{thm}[Lagrange's Theorem]\index{Lagrange's Theorem}\label{thm:lagrange_s_theorem}
  If $G$ is a finite group and $H$ is a subgroup of $G$ \sidenote{I shall write this as $H \leq G$ from hereon.}, then $\abs{H} \mid \abs{G}$ \sidenote{This just means $\abs{H}$ divides $\abs{G}$.}.
\end{thm}

The full converse is not true.

\begin{eg}
  Let $G = A_4$, the \hlnotea{alternating group} of $4$ elements. Then $\abs{G} = 12$ \sidenote{Recall that the symmetric group of $4$ elements $S_4$ has order $4! = 24$, and an alternating group has half of its elements.}. We have that $6 \mid 12$. We shall show that $G$ has no subgroup of order $6$.

  Suppose to the contrary that $H \leq G$ such that $\abs{H} = 6$. Let $a \in G$ such that $\abs{a} = 3$ \sidenote{i.e. the order of $a$ is $3$. This is a \hlimpo{trick}.} There are $8$ such elements in $G$ \sidenote{This shall be left as an exercise.
  \begin{ex}
    Prove that there are $8$ elements in $G$ that have order $3$.
  \end{ex}}. Note that the \hlnotea{index}\sidenote{The index of a subgroup is the number of unique cosets generated by $H$.} of $H$, $\abs{G : H}$, is $\frac{\abs{G}}{\abs{H}} = 2$.

  Now consider the \hlnotea{cosets} $H$, $aH$ and $a^2 H$. Since $\abs{G : H} = 2$, we must have either
  \begin{itemize}
    \item $aH = H \implies a \in H$;
    \item $aH = a^H \overset{\text{`multiply' } a^{-1}}{\implies} H = aH \implies a \in H$; or
    \item $a^2H = H \overset{\text{`multiply' } a}{\implies} H = aH \implies a \in H$.
  \end{itemize}
  Thus all $8$ elements of order $3$ are in $H$ but $\abs{H} = 6$, a contradiction. Therefore, no such subgroup (of order $6$) exists.
\end{eg}

Our goal now is to establish a partial converse of Lagrange's Theorem. To that end, we shall first lay down some definitions.

\begin{defn}[$p$-Group]\index{$p$-Group}\label{defn:_p_group}
  Let $p$ be prime. We say that a group $G$ is a \hlnoteb{$p$-group} if $\abs{G} = p^k$ for some $k \in \mathbb{N}$. For $H \leq G$, we say that $H$ is a $p$-subgroup of $G$ if $H$ is a $p$-group.
\end{defn}

\begin{defn}[Sylow $p$-Subgroup]\index{Sylow $p$-Subgroup}\label{defn:sylow_p_subgroup}
  Let $G$ be a group such that $\abs{G} = p^n m$ for some $n, m \in \mathbb{N}$, such that $p \nmid m$. If $H \leq G$ with order $p^n$, we call $H$ a \hlnoteb{Sylow $p$-subgroup}.
\end{defn}

Recall Cauchy's Theorem for abelian groups\sidenote{In the course I was in, we were introduced only to the full theorem and actually went through this entire part. See notes on \href{https://tex.japorized.ink/PMATH347S18/classnotes.pdf}{PMATH 347}.}.

\begin{thm}[Cauchy's Theorem for Abelian Groups]\index{Cauchy's Theorem for Abelian Groups}\label{thm:cauchy_s_theorem_for_abelian_groups}
  If $G$ is a finite abelian group, and $p$ is prime such that $p \mid \abs{G}$, then $\abs{G}$ has an element of order $p$.
\end{thm}

\begin{defn}[Stabilizers and Orbits]\index{Stabilizers}\index{Orbits}\label{defn:stabilizers_and_orbits}
  Let $G$ be a finite group which acts on a finite set $X$ \sidenote{Recall that a group action is a function $\cdot : G \times X \to X$ such that
  \begin{enumerate}
    \item $g(hx) = (gh)x$; and
    \item $ex = x$.
  \end{enumerate}}. For $x \in X$, the \hlnoteb{stabilizers} of $x$ is the set
  \begin{equation*}
    \stab(x) := \{ g \in G : g x = x \} \leq G.
  \end{equation*}
  The orbits of $x$ is a set
  \begin{equation*}
    \orb(x) := \{ gx : g \in G \}.
  \end{equation*}
\end{defn}

\begin{note}
  One can verify that the function $G / \stab(x) \to \orb(x)$ such that
  \begin{equation*}
    g \stab(x) \mapsto gx
  \end{equation*}
  is a bijection.
\end{note}

\begin{thm}[Orbit-Stabilizer Theorem]\index{Orbit-Stabilizer Theorem}\label{thm:orbit_stabilizer_theorem}
  Let $G$ be a group acting on a set $X$, and for each $x \in X$, $\stab(x)$ and $\orb(x)$ are the stabilizers and orbits of $x$, respectively. Then
  \begin{equation*}
    \abs{G} = \abs{\stab(x)} \cdot \abs{\orb(x)}.
  \end{equation*}
  Moreover, if $x, y \in X$, then either $\orb(x) \cap \orb(y) = \emptyset$ or $\orb(x) = \orb(y)$.
\end{thm}

The theorem is actually equivalent to \href{https://tex.japorized.ink/PMATH347S18/classnotes.pdf#thm.45}{Proposition 45} in the notes for PMATH 347. However, feel free to...

\begin{ex}
  prove \cref{thm:orbit_stabilizer_theorem} as an exercise.
\end{ex}

Consequently, we have that
\begin{equation*}
  \abs{X} = \sum \abs{\orb(a_i)},
\end{equation*}
where $a_i$ are the distinct orbit representatives. Letting
\begin{equation*}
  X_G := \{ x \in X : gx = x, g \in G \},
\end{equation*}
we have...

\begin{thm}[Orbit Decomposition Theorem]\index{Orbit Decomposition Theorem}\label{thm:orbit_decomposition_theorem}
  \begin{equation*}
    \abs{X} = \abs{X_G} + \sum_{a_i \notin X_G} \abs{\orb(a_i)}.
  \end{equation*}
\end{thm}

% section cauchy_s_theorem (end)

% chapter lecture_1_jan_07th (end)

\chapter{Lecture 2 Jan 09th}%
\label{chp:lecture_2_jan_09th}
% chapter lecture_2_jan_09th

\section{Sylow Theory}%
\label{sec:sylow_theory}
% section sylow_theory

From the \hyperref[thm:orbit_decomposition_theorem]{Orbit Decomposition Theorem}, one special case is when $G$ acts on $X = G$ by conjugation.

\begin{crly}[Class Equation]\index{Class Equation}\label{crly:class_equation}
  From \cref{thm:orbit_decomposition_theorem}, if $X = G$, we have
  \begin{align*}
    \abs{G} &= \abs{Z(G)} + \sum \overarrow{\abs{ \orb(a_i) }}{\text{non-central}} \\
            &= \abs{Z(G)} + \sum [ G : \stab(a_i) ] \text{ by } \hyperref[thm:orbit_stabilizer_theorem]{Orbit-Stabilizer} \\
            &= \abs{Z(G)} + \sum [ G : C(a_i) ],
  \end{align*}
  where $C(a_i)$ is called the \hldefn{centralizers} of $G$. 
\end{crly}

\begin{thm}[First Sylow Theorem]\index{First Sylow Theorem}\label{thm:first_sylow_theorem}
  Let $G$ be a finite group, and let $p \mid \abs{G}$ such that $p$ is prime.
  Then $G$ contains a Sylow $p$-subgroup.
\end{thm}

\begin{proof}
  We proceed by induction on the size of $G$.
  If $\abs{G} = 2$, then $p = 2$, and so $G$ is its own Sylow $p$-subgroup
  \sidenote{A $2$-cycle is a Sylow $p$-group.}.
  
  Consider a finite group $G$ with $\abs{G} \geq 2$.
  Let $p$ be a prime that divides $\abs{G}$, and assume
  that the desired result holds for smaller groups.
  
  Let $\abs{G} = p^n m$, where $n, m \in \mathbb{N}$, and $p \nmid m$.

  \noindent
  \hlbnotea{Case 1: $p \mid \abs{Z(G)}$}
  By \cref{thm:cauchy_s_theorem_for_abelian_groups}, $\exists a \in Z(G)$
  such that $\abs{a} = p$.
  Since $\langle a \rangle \subsetneq Z(G)$, we have that
  \begin{equation*}
    \langle a \rangle \triangleleft G \text{ and } \abs{ \langle a \rangle } = p.
  \end{equation*}
  \sidenote{This feels like a struck of genius.
  Let's break it down and find some way that makes it easier to remember.
  We want to find $H \leq G$ such that $\abs{H} = p^n$.
  We have $\abs{ \langle a \rangle } = p$.
  We want to be able to use the \hlimpo{Correspondence Theorem},
  so we should adjust our materials to fit that mold:
  since $\abs{\langle a \rangle} = p$, notice that
  \begin{equation*}
    \frac{\abs{G}}{\abs{\langle a \rangle}} = p^{n - 1} m.
  \end{equation*}
  This is a smaller group than $G$, and so IH tells us that it
  has a Sylow $p$-subgroup, say $\bar{H}$.
  By the Correspondence Theorem, we may retrieve $H$.}
  Notice that the group $G / \langle a \rangle$ is a group
  that has a lower order than $G$, and so by IH,
  $\exists \bar{H} \leq G / \langle a \rangle$ such that $\bar{H}$
  is a Sylow $p$-subgroup of $G / \langle a \rangle$.
  Note that if $n = 1$. then $\langle a \rangle$ itself is the
  Sylow $p$-subgroup.
  WMA $n > 1$. We have that $\abs{H} = p^{n - 1}$.
  By \hyperref[thm:correspondence_theorem]{correspondence},
  \begin{equation*}
    \bar{H} = H / \langle a \rangle,
  \end{equation*}
  where $H \leq G$. By comparing the orders, we have
  \begin{equation*}
    p^{n - 1} = \frac{\abs{H}}{p} \implies \abs{H} = p^n.
  \end{equation*}
  Therefore $H$ is a Sylow $p$-subgroup of $G$.

  \noindent
  \hlbnotea{Case 2: $p \nmid Z(G)$}
  By the \hyperref[crly:class_equation]{class equation},
  notice that
  \begin{equation}\label{eq:first_sylow_theorem_eq1}
    p^n m = \abs{G} = \abs{Z(G)} + \sum [ G : C(a_i) ],
  \end{equation}
  and the summation cannot be $0$ or $p$ would otherwise
  divide $Z(G)$.
  \marginnote{This highlighted part requires clarification.}
  % TODO: this needs the prof's intervention
  \begin{quotebox}{be-red}{light}
  Since $p$ divides the LHS of \cref{eq:first_sylow_theorem_eq1}
  and not $\abs{Z(G)}$, and the sum is nonzero, we must 
  have that $\exists a_i \in G$ such that $p \nmid [ G : C(a_i) ]$.
  This implies that $p^n \mid \abs{ C(a_i) }$.
  \end{quotebox}
  Since $a_i \notin Z(G)$, we have $\abs{C(a_i)} \leq \abs{G}$.
  Thus by IH, $C(a_i)$ has a Sylow $p$-subgroup, which is also a
  Sylow $p$-subgroup of $G$.\qed\
\end{proof}

\begin{crly}[Cauchy's Theorem]\index{Cauchy's Theorem}\label{crly:cauchy_s_theorem}
  If $p$ is prime and $p \mid \abs{G}$, then $G$ has an element of order $p$.
\end{crly}

\begin{proof}
  WLOG, WMA $\abs{G} = p^n m$, where $n, m \in \mathbb{N}$ and
  $p \nmid m$. By \cref{thm:first_sylow_theorem}, $\exists H \leq G$
  such that $H$ is a Sylow $p$-subgroup.
  Take $a \in H \setminus \left\{ e \right\}$. Then $\abs{a} = p^k$
  for some $k \leq n$.

  Let $b = a^{p^{k - 1}}$. Notice that $b \neq e$, or it would
  contradict the definition of an order (for $a$).
  Then $b^p = \left( a^{p^{k - 1}} \right)^p = a^p = e$.
  Therefore $\abs{b} = p$ and $b \in G$.\qed\
\end{proof}

\begin{defn}[Normalizer]\index{Normalizer}\label{defn:normalizer}
  Let $G$ be a group, and $H \leq G$. The set
  \begin{equation*}
    N_G(H) = \left\{ g \in G \mmid gHg^{-1} = H \right\}
  \end{equation*}
  is called the \hlnoteb{normalizer} of $H$ in $G$.
\end{defn}

\begin{ex}
  Verify that $N_G(H)$ is the largest subgroup of $G$
  that contains $H$ as a normal subgroup.
\end{ex}

\begin{proof}
  It is clear by definition of a normalizer that $H \triangleleft N_G(H)$.

  Suppose there exists $N_G(H) < \tilde{H} \leq G$ such that
  $H \triangleleft \tilde{H}$. Let $h \in \tilde{H} \setminus N_G(H)$.
  But since $H \triangleleft \tilde{H}$, we have
  \begin{equation*}
    hHh^{-1} = H,
  \end{equation*}
  which implies that $h \in N_G(H)$, a contradiction.
  Therefore $N_G(H)$ is the largest subgroup that contains $H$ as a
  normal subgroup.
\end{proof}

Before proceeding with the Sylow's next theorem, we require two lemmas.

\begin{lemma}[Intersection of a Sylow $p$-subgroup with any other $p$-subgroups]\label{lemma:intersection_of_a_sylow_p_subgroup_with_any_other_p_subgroups}
  Let $G$ be a finite group and $p$ a prime such that $p \mid \abs{G}$.
  Let $P, Q \leq G$ be a Sylow $p$-subgroup and a (regular) $p$-subgroup,
  respectively. Then
  \begin{equation}\label{eq:intersection_of_a_sylow_p_subgroup_with_any_other_p_subgroups}
    Q \cap N_G(P) = Q \cap P.
  \end{equation}
\end{lemma}

\begin{proof}
  Since $P \subseteq N_G(P)$, $\subseteq$ of \cref{eq:intersection_of_a_sylow_p_subgroup_with_any_other_p_subgroups} is done.

  Let $N = N_G(P)$, and let $H = Q \cap N$. WTS $H \subseteq Q \cap P$.
  Since $H = Q \cap N \subseteq Q$, it suffices to show that $H \subseteq P$.
  Since $P$ is a Sylow $p$-subgroup, let $\abs{P} = p^n$. By Lagrange,
  we have that $\abs{H} = p^m$ for some $m \leq n$. Since $P \triangleleft N$,
  we have that $HP \leq N$ 
  \sidenote{See \href{https://tex.japorized.ink/PMATH347S18/classnotes.pdf\#thm.30}{PMATH 347}}.
  Moreover, we have that
  \begin{equation*}
    \abs{HP} = \frac{\abs{H}\abs{P}}{\abs{H \cap P}} = p^k
  \end{equation*}
  for some $k \leq n$. Also, $P \subset HP$, and so $n \leq k$,
  implying that $k = n$. Thus $P = HP$, and thus
  \begin{equation*}
    H \subseteq HP = P,
  \end{equation*}
  as required.\qed\
\end{proof}

% section sylow_theory (end)

% chapter lecture_2_jan_09th (end)

\chapter{Lecture 3 Jan 11th}%
\label{chp:lecture_3_jan_11th}
% chapter lecture_3_jan_11th

\section{Sylow Theory (Continued)}%
\label{sec:sylow_theory_continued}
% section sylow_theory_continued

\begin{lemma}[Counting The Conjugates of a Sylow $p$-Subgroup]\label{lemma:counting_the_conjugates_of_a_sylow_p_subgroup}
  Let $G$ be a finite group, and $p$ a prime such that $p \mid \abs{G}$. Let
  \begin{itemize}
    \item $P$ be a Sylow $p$-subgroup;
    \item $Q$ be a $p$-subgroup;
    \item $K = \left\{ gPg^{-1} \mmid g \in G \right\}$;
    \item $Q$ act on $K$ by conjugation; and
    \item $P = P_1, P_2, \ldots, P_r$ be the distinct orbit representatives
      from the action of $Q$ on $K$.
  \end{itemize}
  Then
  \begin{equation*}
    \abs{K} = \sum_{i=1}^{r} \left[ Q : Q \cap P_i \right].
  \end{equation*}
\end{lemma}

\begin{proof}
  From the definition of $K$, and the fact that $Q$ acts on $K$,
  we have
  \begin{align*}
    \abs{K} &= \sum_{i=1}^{r} \abs{ \orb(P_i) } \\
            &= \sum_{i=1}^{r} \abs{ Q } / \abs{ \stab(P_i) } \quad \hyperref[thm:orbit_stabilizer_theorem]{\text{orbit-stabilizer}} \\
            &= \sum_{i=1}^{r} \abs{ Q } / \abs{ N_G(P_i) \cap Q } \quad \text{ by the action } \\
            &= \sum_{i=1}^{r} [ Q : N_G(P_i) \cap Q ] \quad \text{ by definition } \\
            &= \sum_{i=1}^{r} [ Q : Q \cap P_i ] \quad \hyperref[lemma:intersection_of_a_sylow_p_subgroup_with_any_other_p_subgroups]{\text{the last lemma}}.
  \end{align*}\qed\
\end{proof}

\begin{thm}[Second Sylow Theorem]\index{Second Sylow Theorem}\label{thm:second_sylow_theorem}
  If $P$ and $Q$ are Sylow $p$-subgroups of $G$, then
  $\exists g \in G$ such that $P = gQg^{-1}$.
\end{thm}

\begin{proof}
  Let $K = \left\{ qPq^{-1} \mmid q \in G \right\}$. WTS $Q \in K$.
  We shall also note that $\abs{P} = p^k$ for some $k \in \mathbb{N}$.

  Let $P$ act on $K$ by conjugation. Let the orbit representatives be
  \begin{equation*}
    P = P_1, P_2, \ldots, P_r.
  \end{equation*}
  By \cref{lemma:counting_the_conjugates_of_a_sylow_p_subgroup}, we have
  \begin{equation*}
    \abs{K} = \sum_{i=1}^{r} [ P : P \cap P_i ] 
            = [ P : P ] + \sum_{i=2}^{r} [ P : P \cap P_i ]
            = 1 + \sum_{i=2}^{r} [ P : P \cap P_i ].
  \end{equation*}
  Thus
  \begin{equation*}
    \abs{K} \equiv 1 \mod p.
  \end{equation*}

  Now let $Q$ act on $K$ by conjugation. Reordering if necessary, the
  orbit representatives are
  \begin{equation*}
    P = P_1, P_2, \ldots, P_s,
  \end{equation*}
  where $s$ is not necessarily $r$. From here, it suffices to show that
  $Q = P_i$ for some $i \in \left\{ 1, 2, \ldots, s \right\}$. Suppose
  not. Then by \cref{lemma:counting_the_conjugates_of_a_sylow_p_subgroup},
  \begin{equation*}
    \abs{K} = \sum_{i=1}^{s} [ Q : P_i \cap Q ].
  \end{equation*}
  Note that it must be the case that $[ Q : P_i \cap Q ] > 1$, for some if
  not all $i$, for otherwise it would imply that $Q \cap P_i$ and that would
  be a contradiction. Then by Lagrange,
  \begin{equation*}
    \abs{K} \equiv 0 \mod p.
  \end{equation*}
  This contradicts the fact that $\abs{K} \equiv 1 \mod p$.

  This shows that $Q = P_i$ for some $i \in \left\{ 1, 2, \ldots, s \right\}$,
  and so $Q$ is a conjugate of $P$.\qed\
\end{proof}

\begin{note}[Notation]
  We shall denote $n_p$ as the number of Sylow $p$-subgroups in $G$.
\end{note}

\begin{thm}[Third Sylow Theorem]\index{Third Sylow Theorem}\label{thm:third_sylow_theorem}
  Let $p$ be a prime, and that it divides $\abs{G}$, where $G$ is a group.
  Suppose $\abs{G} = p^n m$, where $n, m \in \mathbb{N}$ and $p \nmid m$.
  Then
  \begin{enumerate}
    \item $n_p \equiv 1 \mod p$; and
    \item $n_p \mid m$.
  \end{enumerate}
\end{thm}

\begin{proof}
  Let $P$ be a Sylow $p$-subgroup of $G$, and let
  \begin{equation*}
    K = \left\{ gPg^{-1} \mmid g \in G \right\}.
  \end{equation*}
  By \hyperref[thm:second_sylow_theorem]{Sylow's second theorem},
  $n_p = \abs{K}$ as all the conjugates are exactly the Sylow
  $p$-subgroups. And by our last proof, we saw that $n_p \equiv 1 \mod p$.

  Let $G$ act on $K$ by conjugation. Then by the \hyperref[thm:orbit_stabilizer_theorem]{Orbit-Stabilizer Theorem},
  \begin{equation*}
    \abs{G} = \abs{\stab(P)}\abs{\orb(P)}.
  \end{equation*}
  Thus
  \begin{equation}\label{eq:third_sylow_theorem_eq1}
    p^n m = \abs{ N_G(P) } n_p.
  \end{equation}
  Thus $n_p \mid p^n m$. Since $n_p \equiv 1 \not\equiv 0 \mod p$,
  we must have $n_p \mid m$.\qed\
\end{proof}

\begin{remark}
  \begin{enumerate}
    \item From \cref{eq:third_sylow_theorem_eq1}, we have that
      \begin{equation*}
        n_p = [ G : N_G(P) ].
      \end{equation*}
    \item \imponote\ Note that
      \begin{equation*}
        n_p = 1 \iff \forall g \in G \; gPg^{-1} = P \iff P \triangleleft G.
      \end{equation*}
      However, note that $P$ \hlimpo{may be trivial}! This means that if $G$
      is simple, it does not imply that $n_p = 1$.
  \end{enumerate}
\end{remark}

\begin{defn}[Simple Group]\index{Simple Group}\label{defn:simple_group}
  A group is said to be \hlnoteb{simple} if it has no non-trivial normal subgroups.
\end{defn}

\begin{eg}
  Prove that there is no simple group of order $56$.
\end{eg}

\begin{proof}
  Let $G$ be a group.
  Note that $56 = 2^3 \cdot 7$. Then $n_7 \equiv 1 \mod 7$ and
  $n_7 \mid 8 = 2^3$. Thus
  \begin{equation*}
    n_7 = 1 \text{ or } n_7 = 8.
  \end{equation*}

  \hlbnotea{$n_7 = 1$} By the remark above, $G$ has a normal Sylow
  $7$-subgroup. Thus $G$ is not simple.

  \hlbnotea{$n_7 = 8$} By Lagrange, since $7$ is prime, by the Finite
  Abelian group structure, the distinct Sylow $8$-subgroups of $G$
  intersect trivially. Therefore, there are $8 \times 6 = 48$ elements 
  of order $7$ in $G$. But this implies that $56 - 48 = 8$ elements that
  are not of order $7$.  One of them is the identity,
  thus the remaining $7$ elements must have order $2$ 
  \sidenote{They cannot be of any other order as that would
  create a cyclic group that is not of order $2$ or $7$, which is
  impossible.}. This implies that
  \begin{equation*}
    n_2 = 7 \equiv 1 \mod 2,
  \end{equation*}
  which by our remark means that $G$ has a normal Sylow $2$-subgroup.
  Thus $G$ is not simple by both accounts.
\end{proof}

% section sylow_theory_continued (end)

% chapter lecture_3_jan_11th (end)

\chapter{Lecture 4 Jan 14th}%
\label{chp:lecture_4_jan_14th}
% chapter lecture_4_jan_14th

\section{Sylow Theory (Continued 2)}%
\label{sec:sylow_theory_continued_2}
% section sylow_theory_continued_2

\begin{remark}
  \begin{enumerate}
    \item Let $p \neq q$ both be primes, and $p, q \mid \abs{ G }$. Let
      $H_p$ and $H_q$ be a Sylow $p$-subgroup and a Sylow $q$-subgroup of $G$,
      respectively. By Lagrange's Theorem, we must have that
      $H_p \cap H_q = \left\{ e \right\}$. Then
      \begin{equation*}
        \abs{ H_p \cup H_q } = \abs{ H_p } + \abs{ H_q } - 1.
      \end{equation*}

    \item Let $\abs{ G } = pm$ and $p \nmid m$, where $p$ is prime. If $H, K$ are
      Sylow $p$-subgroups of $G$ with $H \neq K$, then $H \cap K = \left\{ e \right\}$.
  \end{enumerate}
\end{remark}

\begin{eg}
  % TODO : Wait what!?? Isn't r^3 = 1? This was given as a warning. See handwritten notes
  Note that the second remark is not true if $G = D_6$. Notice that
  \begin{equation*}
    H = \langle 1, s \rangle, \quad K = \langle 1, rs \rangle
  \end{equation*}
  are both Sylow $2$-subgroups of $D_6$ and $H \neq K$, and their intersection is
  trivial.
\end{eg}

\begin{eg}
  Let $\abs{ G } = pq$ where $p, q$ are primes with $p < q$ and $p \nmid q - 1$.
  Then $\abs{G}$ is cyclic.
\end{eg}

\begin{proof}
  By the Third Sylow Theorem, $n_p \equiv 1 \mod p$ and $n_p \mid q$. Notice that $n_p = 1$,
  since if $n_p = q$, then $n_p \equiv 1 \mod p \implies p \mid q - 1$, contradicting our
  assumption. By our remark last lecture, $G$ has a normal Sylow $p$-subgroup, which we shall
  call $H_p$.

  On the other hand, $n_q \equiv 1 \mod q$ and $n_q \mid p$. Since $p < q$, $q \nmid p - 1$,
  and so the same argument as before holds. Hence $n_q = 1$, and so $G$ has a normal Sylow
  $q$-subgroup.

  Since $H_p \triangleleft G$, we know that $H_p H_q \leq G$, and we notice that
  \begin{equation*}
    \abs{ H_p H_q } = \frac{\abs{ H_p } \abs{ H_q }}{\abs{ H_p \cap H_q }} = pq = \abs{ G }.
  \end{equation*}
  Thus $G = H_p H_q$. Let $a, b \in G$. If $a, b$ is either both in $H_p$ or both in $H_q$,
  then $ab = ba$ \sidenote{Crap, I don't remember why... 
    % TODO: figure out why
  }.
  WMA $a \in H_p$ and $b \in H_q$. By our first remark today, note that 
  $H_p \cap H_q = \left\{ e \right\}$. Then, observe that
  \begin{equation*}
    \underbrace{aba^{-1}}_{H_q} \tikzmark{a}b^{-1} \in H_q 
    \qquad \tikzmark{b}a \underbrace{ba^{-1}b^{-1}}_{H_p} \in H_p
  \end{equation*}
  \begin{tikzpicture}[remember picture,overlay]
    \draw[latex'-]
      ([shift={(1pt,-2pt)}]pic cs:a) |- ([shift={(1pt,-10pt)}]pic cs:a)
      node[anchor=north] {$H_q$};
    \draw[latex'-]
      ([shift={(1pt,-2pt)}]pic cs:b) |- ([shift={(1pt,-10pt)}]pic cs:b)
      node[anchor=north] {$H_p$};
  \end{tikzpicture}
  Thus $aba^{-1}b^{-1} = e \implies ab = ba$. So $G$ is abelian. By the Fundamental Theorem
  of Finite Abelian Groups
  \begin{equation*}
    G \simeq \mathbb{Z}_p \times \mathbb{Z}_q \simeq \mathbb{Z}_{pq},
  \end{equation*}
  which is cyclic.\qed\
\end{proof}

\begin{eg}
  By the Fundamental Theorem of Finite Abelian Groups
  \begin{equation*}
    S_3 \simeq \mathbb{Z}_2 \times \mathbb{Z}_3,
  \end{equation*}
  and $\abs{ S_3 } = 6 = 2 \cdot 3$, is not cyclic.
\end{eg}

\begin{eg}
  If $\abs{ G } = 30$, then $G$ has a subgroup isomorphic to $\mathbb{Z}_{15}$.
  Note that $\abs{ G } = 2 \cdot 3 \cdot 5$. By the Third Sylow Theorem,
  \begin{equation*}
    n_5 \equiv 1 \mod 5 \text{ and } n_5 \mid 6 \implies n_5 = 1 \text{ or } 6
  \end{equation*}
  and
  \begin{equation*}
    n_3 \equiv 1 \mod 3 \text{ and } n_3 \mid 10 \implies n_3 = 1 \text{ or } 10.
  \end{equation*}
  Suppose $n_5 = 6$ and $n_3 = 10$. Since the Sylow $3$-subgroups and Sylow 
  $5$-subgroups intersect trivially, this accounts for 
  $(6 \times 4) + ( 10 \times 2 ) = 44$ elements but $\abs{G} = 30 < 44$.
  Thus we must have $n_5 = 1$ or $n_3 = 1$. Thus $G$ is not simple.

  Let $H_3$ and $H_5$ be Sylow $3$- and $5$-subgroups, respectively. WLOG, suppose
  $H_3 \triangleleft G$. Then $H_3 H_5 \leq G$, and notice that $\abs{H_3 H_5} = 15$.
  Since $15 = 3 \cdot 5$ and $3 \nmid 4 = 5 - 1$, we know that $H_3 H_5 \simeq \mathbb{Z}_{15}$
  by an earlier example.
\end{eg}

\begin{eg}\label{eg:g_60_is_simple}
  Let $\abs{ G } = 60$ with $n_5 > 1$. Then $G$ is simple.
\end{eg}

This is an important example for it is with this that we can prove the following:

\begin{crly}[$A_5$ is Simple]\label{crly:_a_5_is_simple}
  $A_5$ is simple.
\end{crly}

\begin{proof}
  Note that $\abs{ A_5 } = \frac{5!}{2} = 60$, and
  \begin{equation*}
    \lra{\begin{pmatrix} 1 & 2 & 3 & 4 & 5 \end{pmatrix}} 
    \text{ and } \lra{\begin{pmatrix} 1 & 3 & 2 & 4 & 5 \end{pmatrix}}
  \end{equation*}
  are both Sylow $5$-subgroups that are distinct (one has odd \hlnotea{parity} while the
  other has even).
\end{proof}

\begin{proof}[For \cref{eg:g_60_is_simple}]
  Suppose $n_5 > 1$. Notice that $60 = 2^2 \cdot 3 \cdot 5$. By \cref{thm:third_sylow_theorem},
  $n_5 \equiv 1 \mod 5$ and $n_5 \mid 12$, and thus $n_5 = 6$. This accounts for
  $6 \times 4 + 1 = 25$ elements. Now suppose $H \triangleleft G$ is proper and non-trivial.

  If $5 \mid \abs{ H }$, then $H$ contains a Sylow $5$-subgroup of $G$. Since
  $H \triangleleft G$, $H$ contains all the conjugates of this Sylow $5$-subgroup. Thus by
  our argument above, we have that $\abs{ H } \geq 25$ \sidenote{These are the $25$ elements
  that were found in the last paragraph.}. Also, $H \mid 60$. Thus it must be that
  $\abs{ H } = 30$. But then by the last example, $n_5 = 1$, a contradiction.

  So $5 \nmid \abs{ H }$. By Lagrange, it remains that
  \begin{equation*}
    \abs{ H } = 2, \, 3, \, 4, \, 6\, \text{ or } 12.
  \end{equation*}

  \hlbnoted{Case A} $\abs{ H } = 12 = 2^2 \cdot 3$.\sidenote{
  \begin{ex}
    Prove that either $n_2 = 1$ or $n_3 = 1$.
  \end{ex}}
  So $H$ contains a normal Sylow $2$- or $3$-subgroup that is normal in $G$.
\end{proof}

The proof shall be continued next lecture.

% section sylow_theory_continued_2 (end)

% chapter lecture_4_jan_14th (end)

\tuftepart{Fields}

\chapter{Lecture 5 Jan 14th}%
\label{chp:lecture_5_jan_14th}
% chapter lecture_5_jan_14th

\section{Sylow Theory (Continued 3)}%
\label{sec:sylow_theory_continued_3}
% section sylow_theory_continued_3

We shall continue with the last proof from where we left off.

\begin{proof}[\cref{eg:g_60_is_simple} continued]
  \hlbnoted{Case A} $\abs{H} = 12$. WLOG, let $K$ be a normal Sylow $3$-subgroup
  of $H$, which is also normal in $G$ \sidenote{In Sylow Theory, normality is
  transitive.}.

  \noindent
  \hlbnoted{Case B} $\abs{H} = 6$. $H$ would then have a normal Sylow $3$-subgroup,
  which is normal in $G$. We shall also call this subgroup $K$.

  \noindent
  By replacing $H$ with $K$ if necessary, wma $\abs{ H } \in \left\{ 2, 3, 4 \right\}$.
  Consider $\bar{G} = G / H$. Then $\abs{ \bar{G} } \in \left\{ 15, 20, 30 \right\}$.
  \sidenote{
  \begin{ex}
    Prove that $\bar{G}$ has a normal Sylow $5$-subgroup in all the three possible
    orders of $\bar{G}$.
  \end{ex}}
  In any case, $\bar{G}$ has a normal Sylow $5$-subgroup. Call this normal subgroup 
  $\bar{P}$. By \hyperref[thm:correspondence_theorem]{correspondence}, $\bar{P} = P / H$
  where $P$ is a normal subgroup of $G$ \sidenote{Note: correspondence works for the
  normal case as well.}. Thus $P$ is a proper non-trivial normal subgroup of $G$.
  Also,
  \begin{equation*}
    \abs{ P } = \abs{ \bar{P} } \cdot \abs{ H } = 5 \cdot \abs{ H }.
  \end{equation*}
  Thus $5 \mid \abs{ P }$, putting us back to the case where $5 \mid \abs{ H }$.
  Thus $G$ does not have a non-trivial normal subgroup, i.e. $G$ is simple.\qed\
\end{proof}

% section sylow_theory_continued_3 (end)

\section{Review of Ring Theory}%
\label{sec:review_of_ring_theory}
% section review_of_ring_theory

Let $F$ be a \hlnotea{field}, and $I$ be an \hlnotea{ideal} of $F[x]$, its
\hlnotea{polynomial ring}. Since $F[x]$ is a PID, we have $I = \langle p(x) \rangle$
for some $p(x) \in F[x]$.

Moreover, $I$ is \hlnotea{maximal} iff $p(x)$ is \hlnotea{irreducible}.

Thus we observe that
\begin{center}
  $F[x]/I$ is a field iff $I = \langle p(x) \rangle$ is maximal iff $p(x) \in F[x]$ is irreducible.
\end{center}

Therefore, to talk about fields, we need to understand irreducibles.

% section review_of_ring_theory (end)

\section{Irreducibles}%
\label{sec:irreducibles}
% section irreducibles

\begin{defn}[Irreducible]\index{Irreducible}\label{defn:irreducible}
  Let $R$ be an integral domain (ID) \sidenote{\hldefn{Integral domains} are commutative
  rings that has no zero divisors.}. We say that $f(x) \in R[x]$ is \hlnoteb{irreducible}
  (over $R$) if
  \begin{enumerate}
    \item $f(x) \neq 0$;
    \item $f(x) \notin R^\times$, where $R^\times$ is the set of units of $R$;
    \item whenever $f(x) = g(x) h(x)$, where $g(x), h(x) \in R[x]$, then either 
      $g(x) \in R^\times$ or $h(x) \in R^\times$.
  \end{enumerate}
  If $f(x) \neq 0$, $f(x) \notin R^\times$ and $f(x)$ is not irreducible, we say that
  $f(x)$ is \hldefn{reducible} (over $R$).
\end{defn}

\begin{eg}
  $f(x) = x^2 - 2$ is irreducible over $\mathbb{Q}$ but reducible over $\mathbb{R}$ as
  \begin{equation*}
    f(x) = \left(x - \sqrt{2}\right)\left(x + \sqrt{2}\right).
  \end{equation*}
\end{eg}

Let $F$ be a field, $f(x) \in F[x]$ and $a \in F$. By the \hlnotea{Division Algorithm},
we can write
\begin{equation*}
  f(x) = (x - a) q(x) + r(x),
\end{equation*}
where $q(x), r(x) \in F[x]$. Note that we either have $r(x) = 0$ or 
$\deg r < \deg (x - a) = 1$. In the latter case, $r \in F$, and so
\begin{equation*}
  f(x) = (x - a) q(x) + r.
\end{equation*}
Then $f(a) = 0 + r = r$, and so $f(x) = (x - a) q(x) + f(a)$.
\begin{equation*}
  \therefore (x - a) \mid f(x) \iff f(a) = 0.
\end{equation*}

\begin{propo}[Polynomials with Roots are Reducible]\label{propo:polynomials_with_roots_are_reducible}
  Let $F$ be a field. If $f(x) \in F[x]$ with $\deg f > 1$, and $f$ has a root in $F$,
  then $f$ is reducible (over $F$).
\end{propo}

\begin{eg}
  Let $f(x) = x^6 + x^3 + x^4 + x^3 + 3 \in \mathbb{Z}_7[x]$. Then $f(1) = 0$.
  Therefore
  \begin{equation*}
    f(x) = (x - 1) g(x) \text{ where } g(x) \in \mathbb{Z}_7[x].
  \end{equation*}
  Thus $f(x)$ is reducible over $\mathbb{Z}_7$.
\end{eg}

\begin{propo}[Irreducible Rootless Polynomials]\label{propo:irreducible_rootless_polynomials}
  Let $F$ be a field\sidenote{Note that this does not work in an ID. For example, $2x^2 + 2$.
  }. If $f(x) \in F[x]$ with $\deg f \in \{ 2, 3 \}$, then $f(x)$ is irreducible over $F$
  iff $f(x)$ has no roots in $F$.
\end{propo}

\begin{warning}
  $(x^2 + 1)^2 \in \mathbb{R}[x]$ is reducible but has no root in $\mathbb{R}$. Note that
  the degree of the polynomial is $4$.
\end{warning}

\begin{eg}
  Let $f9x) = x^3 + x + 1 \in \mathbb{Z}_2[x]$. Note that $f(0) = 1$ and 
  $f(1) = 3 \equiv 1 \mod 2$. Since $\deg f = 3$ and $f$ has no roots in $\mathbb{Z}_2$,
  $f(x)$ is irreducible over $\mathbb{Z}_2$.
\end{eg}

\begin{thm}[Gauss' Lemma]\index{Gauss' Lemma}\label{thm:gauss_lemma}
  Let $R$ be a Unique Factorization Domain (UFD), with field of fractions $F$.
  Let $p(x) \in R[x]$. If
  \begin{equation*}
    p(x) = A(x) B(x),
  \end{equation*}
  where $A(x), B(x)$ are non-constant in $F[x]$, then 
  $\exists r, s \in F^\times$ non-zero such that
  \begin{equation*}
    p(x) = a(x) b(x),
  \end{equation*}
  where $a(x) = rA(x)$ and $b(x) = sB(x)$.
\end{thm}

\begin{note}
  If $p(x) \in R[x]$ is reducible over $F$, then $p(x)$ is reducible over $R$.
\end{note}

\begin{note}
  If $R = \mathbb{Z}$ and $F = \mathbb{Q}$, then $p(x)$ is irreducible over
  $\mathbb{Z}$, then $p(x)$ is irreducible over $\mathbb{Q}$.
\end{note}

% section irreducibles (end)

% chapter lecture_5_jan_14th (end)

\chapter{Lecture 6 Jan 18th}%
\label{chp:lecture_6_jan_18th}
% chapter lecture_6_jan_18th

\section{Irreducibles (Continued)}%
\label{sec:irreducibles_continued}
% section irreducibles_continued

Our goal in this section is to develop methods to test for the irreducibility
of polynomials.

\begin{warning}
  Note that $f(x) = 2x + 4 = 2(x + 2)$ is reducible ovver $\mathbb{Z}$
  \sidenote{This is interesting over $\mathbb{Z}$, since
  $2 \notin \mathbb{Z}^\times$.} but
  irreducible over $\mathbb{Q}$.
\end{warning}

\begin{propo}[Mod-$p$ Irreducibility Test]\index{Mod-$p$ Irreducibility Test}\label{propo:mod_p_irreducibility_test}
  Let $f(x) \in \mathbb{Z}[x]$ with $\deg f \geq 1$. Let $p \in \mathbb{Z}$ be prime.
  If $\bar{f}(x)$ is the corresponding polynomial in $\mathbb{Z}_p[x]$ such that
  \begin{itemize}
    \item the coefficients of $\bar{f}(x)$ are coefficients of $f(x)$ in mod $p$,
    \item $\deg f = \deg \bar{f}$ \sidenote{This means that the leading
      coefficient of $f$ is not killed off.}, and
    \item $\bar{f}$ is irreducible over $\mathbb{Z}_p$,
  \end{itemize}
  then $f(x)$ is irreducible over $\mathbb{Q}$.
\end{propo}

\begin{proof}
  Suppose $\deg f = \deg \bar{f}$, and $\bar{f}(x) \in \mathbb{Z}_p$ is 
  irreducible over $\mathbb{Z}_p$. Suppose to the contrary that $f(x)$ is
  reducible over $\mathbb{Q}$. Then for some $g(x), h(x) \in \mathbb{Q}[x]$
  with $\deg g, \deg h < \deg f$, we have
  \begin{equation*}
    f(x) = g(x) h(x).
  \end{equation*}
  By \hyperref[thm:gauss_lemma]{Gauss' Lemma}, wma $g(x), h(x) \in \mathbb{Z}[x]$.
  Then we have
  \begin{equation*}
    \bar{f}(x) = \bar{g}(x) \bar{h}(x) \in \mathbb{Z}_p[x].
  \end{equation*}
  By assumption, $\bar{f}$ is irreducible over $\mathbb{Z}_p$, either
  \begin{equation*}
    \deg \bar{g} = 0 \text{ or } \deg \bar{h} = 0.
  \end{equation*}
  Wlog, $\deg \bar{g} = 0$. Then
  \begin{equation*}
    \deg h \leq \deg f = \deg \bar{f} = \deg \bar{h} \leq \deg h,
  \end{equation*}
  which implies that $\deg f = \deg h$ but $\deg h < \deg f$. Thus $f$ is irreducible
  over $\mathbb{Q}$.\qed\
\end{proof}

\begin{eg}
  Consider the polynomial
  \begin{equation*}
    f(x) = 3x^3 + 22x^2 + 17x + 471.
  \end{equation*}
  Then consider
  \begin{equation*}
    \bar{f}(x) = x^3 + x + 1 \in \mathbb{Z}_2[x].
  \end{equation*}
  Since $\bar{f}(0) \neq 0$ and $\bar{f}(1) \neq 0$, and $\deg f = 3$, by
  \cref{propo:irreducible_rootless_polynomials}, $\bar{f}(x)$ is irreducible over
  $\mathbb{Z}_2$. Since $\deg f = \deg \bar{f}$, $f$ is irreducible over $\mathbb{Q}$
  by the \hyperref[propo:mod_p_irreducibility_test]{Mod-$2$ irreducible test}.
\end{eg}

\begin{warning}
  Consider $f(x) = 2x^2 + x \in \mathbb{Q}[x]$, which is reducible over $\mathbb{Q}$.
  However, $\bar{f}(x) = x \in \mathbb{Z}_2[x]$ is \hlimpo{reducible} over $\mathbb{Z}_2$.
  Notice here that $\deg \bar{f} \neq \deg f$.
\end{warning}

More generally so...

\begin{propo}[Polynomials that Cannot be Factored Over the Ideals is Irreducible]\label{propo:polynomials_that_cannot_be_factored_over_the_ideals_is_irreducible}
  Let $I$ be a proper ideal of an ID $R$. Let $p(x) \in R[x]$ be monic and non-const.
  If $p(x)$ cannot be factored in $\left( R / I \right)[x]$ \sidenote{Note that
  $\left( R / I \right)$ may not be an ID even if $R$ is one.} into polynomials of
  lesser degree, then $p(x)$ is irreducible over $R$.
\end{propo}

\begin{proof}
  Sps to the contrary that $p(x)$ is reducible over $R$. Then
  \begin{equation*}
    p(x) = f(x) g(x)
  \end{equation*}
  for some $f(x), g(x) \notin R^\times$. Since $p(x)$ is monic, and $\deg f, \deg g
  < \deg p$, wma $f(x)$ and $g(x)$ are also monic. Then
  \begin{equation*}
    \bar{p}(x) = \bar{f}(x) \bar{g}(x) \in \left( R/I \right)[x].
  \end{equation*}
  Since $I \subsetneq R$, we have that $1 \notin I$, and so
  \begin{equation*}
    \deg \bar{f}, \deg \bar{g} < \deg \bar{p}
  \end{equation*}
  but that implies that $p(x)$ can be factored in $\left( R/I \right)[x]$.\qed\
\end{proof}

\begin{propo}[Eisenstein's Criterion]\index{Eisenstein's Criterion}\label{propo:eisenstein_s_criterion}
  Let $R$ be an ID. Let $P$ be a prime ideal of $R$. Let
  \begin{equation*}
    f(x) = x^n + a_{n - 1} x^{n - 1} + \hdots + a_1 x + a_0 \in R[x]
  \end{equation*}
  with $n \geq 1$. Note that $f$ is monic. Now if
  \begin{equation*}
    a_{n - 1}, a_{n - 2}, \ldots, a_1, a_0 \in P \text{ and } a_0 \notin P^2,
  \end{equation*}
  then $f$ is irreducible over $R$.
\end{propo}

\begin{proof}
  Sps to the contrary that $f$ is reducible over $R$. Since $f(x)$ is monic,
  \begin{equation*}
    f(x) = g(x) h(x)
  \end{equation*}
  where $g(x), h(x) \in R[x]$ and $\deg g, \deg h < \deg f$. Then
  \begin{equation*}
    \bar{f}(x) = \bar{g}(x) \bar{h}(x) = x^n \in (R/P)[x]
  \end{equation*}
  since $a_{n - 1}, a_{n - 2}, \ldots, a_1, a_0 \in P$.  Since $P$ is prime,
  $R/P$ is an ID, we have that either $\bar{g}(0) = 0$ or $\bar{h}(0) = 0$.
  Wlog, $\bar{g}(0) = 0 \in P$. But that implies that 
  $a_0 = \bar{g}(0)\bar{h}(0) = 0 \in P^2$, a contradiction.\qed\
\end{proof}

% section irreducibles_continued (end)

% chapter lecture_6_jan_18th (end)

\chapter{Lecture 7 Jan 21st}%
\label{chp:lecture_7_jan_21st}
% chapter lecture_7_jan_21st

\section{Irreducibles (Continued 2)}%
\label{sec:irreducibles_continued_2}
% section irreducibles_continued_2

\begin{eg}
  Prove that $f(x, y) = x^2 + y^2 - 1$ is irreducible in $\mathbb{Q}[x, y] = (\mathbb{Q}[x])[y]$.
\end{eg}

\begin{proof}
  Let $g(y) = y^2 + (x^2 + 1)$. Since $x + 1$ is irreducible, let $P = \langle x + 1 \rangle$,
  which is therefore a prime ideal of $\mathbb{Q}[x]$. Moreover, notice that
  \begin{equation*}
    x^2 - 1 = ( x + 1 )( x - 1 ) \in P.
  \end{equation*}
  Since $(x + 1)^2 \nmid \left(x^2 - 1\right)$, we have that $x^2 - 1 \notin P^2$. Then by
  \hyperref[propo:eisenstein_s_criterion]{Eisenstein}, we have that $f(x, y)$ is irreducible.
\end{proof}

\begin{crly}[Eisenstein + Gauss]\label{crly:eisenstein_gauss}
  Let $p \in \mathbb{Z}$ be a prime, and let
  \begin{equation*}
    f(x) = x^n + a_{n - 1} x^{n - 1} + \hdots + a_1 x + a_0
  \end{equation*}
  be non-const in $\mathbb{Z}[x]$. If $p \mid a_i$ for all $i \in \{0, \ldots, n - 1\}$, and
  $p^2 \nmid a_0$, then $f$ is irreducible over $\mathbb{Q}$.
\end{crly}

\marginnote{Recall that the prime ideals of $\mathbb{Z}$ are $\mathbb{Z}_p$ where $p$ is prime.}

\begin{proof}
  Let $P = \langle p \rangle$. It follows from Eisenstein that $f$ is irreducible over $\mathbb{Z}$,
  and then from Gauss that $f$ is irreducible over $\mathbb{Q}$.\qed\
\end{proof}

\begin{eg}
  Let $f(x) = x^n - d \in \mathbb{Z}[x]$ where $\exists p \in \mathbb{Z}$ prime such that
  $p^2 \nmid d$ and $p \mid d$. Let $P = \langle p \rangle$ and so by \cref{crly:eisenstein_gauss},
  $f$ is irreducible over $\mathbb{Q}$.
\end{eg}

\begin{note}
  The above example is noteworthy since it will appear rather often throughout this course. Notice
  that if we have polynomials of the above form, then we immediately have that the polynomial is
  irreducible.
\end{note}

\begin{eg}
  Are the following irreducible over $\mathbb{Q}$?
  \begin{enumerate}
    \item $f(x) = x^7 + 21 x^5 + 15x^2 + 9x + 6$

      Yes. Notice that all the non-leading coefficients have a factor of $3$, and so if we let
      $p = 3$, since $3^2 = 9 \nmid 6$, it follows from Eisenstein that $f$ is irreducible over
      $\mathbb{Q}$.

    \item $f(x) = x^3 + 2x + 16$

      Eisenstein can't help us here since $\gcd(2, 16) = 2$ and $2^2 = 4 \mid 16$.
      Consider $\bar{f}(x) = x^3 + 2x + 1 \in \mathbb{Z}_3[x]$.  Notice that
      $\bar{f}(0) = 1 = \bar{f}(2)$ and $\bar{f}(1) = 4$. Since $\deg \bar{f} = 3$, it follows
      from \cref{propo:irreducible_rootless_polynomials} that $\bar{f}$ is irreducible over
      $\mathbb{Z}_3$. Since $\deg f = \deg \bar{f}$, it follows from the
      \hyperref[propo:mod_p_irreducibility_test]{Mod-$3$ irreducible test} that $f$ is irreducible
      over $\mathbb{Q}$.

    \item $f(x) = x^4 + 5x^3 + 6x^2 - 1$

      Again, Eisenstein can't help us here, since $5 \coprime 6 \coprime 1$ \sidenote{$\coprime$
      is a common notation for coprimeness.}. Consider
      \begin{equation*}
        \bar{f}(x) = x^4 + x^3 + 1 \in \mathbb{Z}_2[x].
      \end{equation*}
      We know that $\bar{f}(0) = 1 = \bar{f}(1)$, and so $\bar{f}$ has no roots in $\mathbb{Z}_2$.
      \sidenote{Note that we cannot use \cref{propo:irreducible_rootless_polynomials} here as
      $\deg \bar{f} = 4 > 3$.} Consider the quadratics\sidenote{\hlwarn{Why did we only check for
      the quadratics and not others?} We did so as we have already checked for the linear factors
      by checking for roots, which also checks for the cubic factors, since if we can factor out
      a linear factor, we are left with a cubic factor. Ruling out linear factors in turn rules out
      cubic factors.} of $\mathbb{Z}_2[x]$: we have
      \begin{equation*}
        x^2, \quad x^2 + x, \quad x^2 + 1, \quad x^2 + x + 1,
      \end{equation*}
      all, but the last, of which are reducible. However, notice that
      \begin{equation*}
        \left( x^2 + x + 1 \right)^2 = x^4 + x^2 + 1 \neq \bar{f}(x)
      \end{equation*}
      (by the Freshman's Dream). Thus $\bar{f}$ is irreducible in $\mathbb{Z}_2$. Since $\deg f =
      \deg \bar{f}$, by \hyperref[propo:mod_p_irreducibility_test]{Mod-$2$ irreducible test}.

    \item \imponote Let $p$ be a prime, and let
      \begin{equation*}
        f(x) = x^{p - 1} + x^{p - 2} + \hdots + x^2 + x + 1.
      \end{equation*}
      Note that $f(x)(x - 1) = x^p - 1$, and so $f(x) = \frac{x^p - 1}{x - 1}$.
      Furthermore, notice that
      \begin{align*}
        f(x + 1) &= \frac{(x + 1)^p - 1}{x} = \sum_{k = 0}^{p} \binom{p}{k} x^{p - k} - \frac{1}{x} \\
                 &= x^{p - 1} + \binom{p}{p - 1} x^{p - 2} + \hdots + \binom{p}{2} x + \binom{p}{1}.
      \end{align*}
      By setting $P = \langle p \rangle$, we have that $f(x + 1)$ is irreducible by Eisenstein.
      It follows from A3Q2 that $f(x)$ is also irreducible.
  \end{enumerate}
\end{eg}

% section irreducibles_continued_2 (end)

\section{Field Extensions}%
\label{sec:field_extensions}
% section field_extensions

Let $K$ be a field. Recall that a non-empty subset $F \subseteq K$ is called a \hldefn{subfield} of $K$
if $F$ is a field under the same operations.

\begin{eg}
  $\mathbb{Q}(\sqrt{2}) := \left\{ a + b\sqrt{2} \mmid a, b \in \mathbb{Q} \right\}$ is a subfield of
  $\mathbb{C}$. We call this field $\mathbb{Q}$ `\hldefn{adjoin}' $\sqrt{2}$.
\end{eg}

\begin{note}
  We did not actually show that $\mathbb{Q}(\sqrt{2})$ is indeed a field but note the following:
  let $a + b \sqrt{2} \neq 0 \in \mathbb{Q}(\sqrt{2})$. Then
  \begin{equation*}
    \frac{1}{a + b\sqrt{2}} \cdot \frac{(a - b\sqrt{2})}{(a - b \sqrt{2})} = \frac{a - b \sqrt{2}}{a^2 - 2b^2} \in \mathbb{Q}(\sqrt{2}),
  \end{equation*}
  and note that
  \begin{equation*}
    a^2 - 2b^2 \neq 0 \iff \frac{a}{b} = \sqrt{2},
  \end{equation*}
  which does not happen in $\mathbb{Q}$ itself.
\end{note}

\begin{defn}[Field Extension]\index{Field Extension}\label{defn:field_extension}
  Let $F$ be a field. A \hlnoteb{field extension} (or an \hlnoteb{extension}) of $F$ is a field $K$
  which contains an \hlimpo{isomorphic} copy of $F$ as a subfield. We denote this notion of $K/F$.
\end{defn}

\begin{eg}
  \begin{itemize}
    \item We have that $\mathbb{C} / \mathbb{R}$ and $\mathbb{Q}(\sqrt{2}) / \mathbb{Q}$.
    \item For a prime $p$, if
      \begin{equation*}
        \mathbb{Z}_p =(x) \left\{ \frac{f(x)}{g(x)} \mmid f(x), g(x) \in \mathbb{Z}_p[x], g \neq 0 \right\},
      \end{equation*}
      then $\mathbb{Z}_p(x) / \mathbb{Z}_p$.
    \item Let $F$ be a field, and $f(x) \in F[x]$ be irreducible. Then let $K = F[x]/\langle f(x) \rangle$.
      Then $K / F$.
  \end{itemize}
\end{eg}

\begin{note}
  Note that in the last example, $K$ is not a `direct' extension of $F$, but it contains an isomorphic
  copy of $F$. This allows us to have more flexibility in what we can do.
\end{note}

\begin{warning}
  If given $\mathbb{Z}_p = \{ 0, 1, 2, \ldots, p - 1 \}$, then $\mathbb{Q}$ is not an extension of
  $\mathbb{Z}_p$ since the two use different operations.
\end{warning}

% section field_extensions (end)

% chapter lecture_7_jan_21st (end)

\chapter{Lecture 8 Jan 23rd}%
\label{chp:lecture_8_jan_23rd}
% chapter lecture_8_jan_23rd

\section{Field Extensions (Continued)}%
\label{sec:field_extensions_continued}
% section field_extensions_continued

\begin{eg}
  Let $F$ be a field.
  \begin{itemize}
    \item If the characteristic $ch(F) = p > 0$ is a prime, then 
      $F \supset \{ 0, 1, 2, \ldots, p - 1 \} \simeq \mathbb{Z}_p$.
      Thus $F / \mathbb{Z}_p$.
    \item If $\ch(F) = 0$, then $F / \mathbb{Q}$.
  \end{itemize}
  In either of these cases, we call $\mathbb{Z}_p$ and/or $\mathbb{Q}$ the \hldefn{prime subfield} of
  $F$.
\end{eg}

\begin{defn}[Generated Field Extension]\index{Generated Field Extension}\label{defn:generated_field_extension}
  Let $K / F$, and $\alpha_1, \ldots, \alpha_n \in K$. The \hlnoteb{field extension of $F$ generated
  by $\{a_i\}_{i = 1}^{n}$} is
  \begin{equation*}
    F(\alpha_1, \ldots, \alpha_n) 
    := \left\{ \frac{f(\alpha_1, \ldots, \alpha_n)}{g(\alpha_1, \ldots, \alpha_n)} \mmid f, g \in F[x_1, \ldots, x_n], g \neq 0 \right\},
  \end{equation*}
  of which we call as $F$ \hldefn{adjoin} $\alpha_1, \ldots, \alpha_n$.
\end{defn}

\begin{note}
  We have that $F(\alpha_1, \ldots, \alpha_n) / F$, and in turn $K / F(\alpha_1, \ldots, \alpha_n)$.
\end{note}

\begin{remark}[Minimality]\label{remark:minimality_of_an_extension}
  Let $K / F$, and $\alpha_1, \ldots, \alpha_n \in K$. If we have $E / F$ such that $K / E$ and
  $\alpha_i \in E$ for all $i$, then
  \begin{equation*}
    F(\alpha_1, \ldots, \alpha_n) \subseteq E,
  \end{equation*}
  i.e. $F(\alpha_1, \ldots, \alpha_n)$ is the smallest extension of $F$ that contains the $\alpha_i$'s.
\end{remark}

\begin{eg}[A classical example of field extensions]
  Show that $\mathbb{Q}(\sqrt{2}, \sqrt{3}) = \mathbb{Q}(\sqrt{2} + \sqrt{3})$.
\end{eg}

\begin{proof}
  Since $\sqrt{2}, \sqrt{3} \in \mathbb{Q}(\sqrt{2}, \sqrt{3})$, by closure, we have that
  $\sqrt{2} + \sqrt{3} \in \mathbb{Q}(\sqrt{2}, \sqrt{3})$, and so $\mathbb{Q}(\sqrt{2} + \sqrt{3}) 
  \subseteq \mathbb{Q}(\sqrt{2}, \sqrt{3})$.

  For the other direction, we have that $\sqrt{2} + \sqrt{3} \in \mathbb{Q}(\sqrt{2} + \sqrt{3})$.
  Then in particular $\frac{1}{\sqrt{2} + \sqrt{3}} \in \mathbb{Q}(\sqrt{2} + \sqrt{3})$. Notice that
  \begin{equation*}
    \frac{1}{\sqrt{2} + \sqrt{3}} \cdot \frac{\sqrt{2} - \sqrt{3}}{\sqrt{2} - \sqrt{3}} 
      = \sqrt{3} - \sqrt{2} \in \mathbb{Q}(\sqrt{2} + \sqrt{3}).
  \end{equation*}
  So $2 \sqrt{3}, 2 \sqrt{2} \in \mathbb{Q}(\sqrt{2} + \sqrt{3})$ \sidenote{$2 \sqrt{2}$ follows from
  a similar argument by using $1 = \frac{\sqrt{3} - \sqrt{2}}{\sqrt{3} - \sqrt{2}}$.}, and in turn
  $\sqrt{2}, \sqrt{3} \in \mathbb{Q}(\sqrt{2}, \sqrt{3})$. Then by minimality, 
  $\mathbb{Q}(\sqrt{2}, \sqrt{3}) \subseteq \mathbb{Q}(\sqrt{2} + \sqrt{3})$.\qed\
\end{proof}

\begin{remark}
  Notice that $F(\alpha, \beta) = \left[ F(\alpha) \right](\beta)$.

  We have that $F(\alpha) \subseteq F(\alpha, \beta), \beta \in F(\alpha, \beta)$, which implies that
  $F(\alpha)(\beta) \subseteq F(\alpha, \beta)$ by minimality.

  Also, since $F \subseteq F(\alpha, \beta)$, and $\alpha, \beta \in F(\alpha, \beta)$, we have, by
  minimality (again), that $F(\alpha, \beta) \subseteq F(\alpha)(\beta)$.
\end{remark}

\begin{propo}[Span of the Extension]\label{propo:span_of_the_extension}
  Let $K / F$ and $\alpha \in K$. If $\alpha$ is a root of some non-zero $f(x) \in F[x]$ irreducible
  over $F$, then $F(\alpha) \simeq F[x] / \langle f(x) \rangle$. Moreover, if $\deg f = n$, then
  \begin{equation*}
    F(\alpha) = \Span_F \{ 1, \alpha, \ldots, \alpha^{n - 1} \}.
  \end{equation*}
\end{propo}

\begin{proof}
  Sps $\alpha \in K$ is a root of an irreducible $f(x) \in F[x]$ over $F$. Let $\deg f = n \in \mathbb{N}$.
  Define $\phi : F[x] \to F(\alpha)$ by $\phi(g(x)) = g(\alpha)$. Note that this is a ring homomorphism.
  Let
  \begin{equation*}
    I = \{ g(x) \in F[x] \mmid g(\alpha) = 0 \} = \ker \phi,
  \end{equation*}
  which is an ideal. Since $F[x]$ is a PID \sidenote{See 
  \href{https://tex.japorized.ink/PMATH347S18/classnotes.tex}{PMATH347}.}, $\exists g(x) \in F[x]$ such that
  $I = \langle g(x) \rangle$. Since $\alpha$ is a root of $f(x)$, $f(x) \in I$, and so $f(x) = g(x) h(x)$
  for some $h(x) \in F[x]$. Since $I \neq F[x]$ and $f$ is irreducible, $h(x) \in F^\times$. Thus
  $\langle g(x) \rangle = \langle g(x) \rangle$. Then by the \hlnotea{First Isomorphism Theorem},
  \begin{equation*}
    F[x] / \langle f(x) \rangle \simeq \phi(F[x]).
  \end{equation*}
  By construction, $\phi(F[x]) \subseteq F(\alpha)$. Since $\phi(F[x])$ is a field (by isomorphism)
  which contains $\alpha = \phi(x)$ and $F$, and so by minimality $F(\alpha) \subseteq \phi(F[x])$.
  Therefore
  \begin{equation*}
    F[x]/ \langle f(x) \rangle \simeq F(\alpha),
  \end{equation*}
  as required.

  Through the isomorphism, for any $h(x) \in F[x]$, we have
  \begin{equation*}
    h(x) + \langle f(x) \rangle \mapsto h(\alpha).
  \end{equation*}
  So
  \begin{equation*}
    F[x] / \langle f(x) \rangle 
      = \left\{ c_{n - 1} x^{n - 1} + \hdots + c_1 x + c_0 + \langle f(x) \rangle \mmid c_i \in F \right\} \\
  \end{equation*}
  and thus
  \begin{align*}
    F(\alpha)
      &= \left\{ c_{n - 1} \alpha^{n - 1} + \hdots + c_1 \alpha + c_0 + \mmid c_i \in F \right\} \\
      &= \Span_F \left\{ 1, \alpha, \ldots, \alpha^{n - 1} \right\},
  \end{align*}
  as claimed.\qed\
\end{proof}

% section field_extensions_continued (end)

% chapter lecture_8_jan_23rd (end)

\chapter{Lecture 9 Jan 25th}%
\label{chp:lecture_9_jan_25th}
% chapter lecture_9_jan_25th

\section{Field Extensions (Continued 2)}%
\label{sec:field_extensions_continued_2}
% section field_extensions_continued_2

Let $K / F$, and $0 \neq g(x) \in F[x]$, and $\alpha \in K$ such that $g(\alpha) = 0$. Since $F[x]$ is an
ID, $g(x)$ must have an irreducible factor $f(x) \in F[x]$ such that $f(\alpha) = 0$. By the proof of
\cref{propo:span_of_the_extension},
\begin{equation*}
  \langle f(x) \rangle = \ker \phi = I = \{ h(x) \in F[x] \mmid h(\alpha) = 0 \}.
\end{equation*}
In particular,
\begin{itemize}
  \item If $h(x) \in F[x]$ such that $h(\alpha) = 0$, then $h(x) \in \langle f(x) \rangle$. In particular,
    $f(x) \mid h(x)$.
  \item $\langle f(x) \rangle$ contains a unique, monic, irreducible polynomial: for any $g(x) \in 
    \langle f(x) \rangle$ that is irreducible, we know that $g(x) = uf(x)$, where $0 \neq u \in F^\times$,
    and so we can just divide the polynomial $g$ by $u$ to make it monic.
\end{itemize}

\begin{defn}[Minimal Polynomial]\index{Minimal Polynomial}\label{defn:minimal_polynomial}
  Let $K / F$, and $\alpha \in K$ be a root of a non-zero polynomial in $F[x]$. Then there exists a
  unique irreducible monic polynomial $f(x) \in F[x]$ such that $f(\alpha) = 0$. We call this $f(x)$
  the \hlnoteb{minimal polynomial} for $\alpha$ over $F$. If $\deg f = n$, we call $n$ the \hldefn{degree}
  of $\alpha$ over $F$, denoted $\deg_F(\alpha)$.
\end{defn}

\begin{note}
  For an $\alpha \in K$, its minimal polynomial is unique, but a minimal polynomial need not have only
  one root.
\end{note}

\begin{propo}[Span of an Extension if Linearly Independent]\label{propo:span_of_an_extension_if_linearly_independent}
  Let $K / F$, and $\alpha \in K$ with minimal polynomial $f(x) \in F[x]$, with $\deg_F(\alpha) = n$. Then
  the span $F(\alpha) = \Span_F \{ 1, \alpha, \ldots, \alpha^{n - 1} \}$ is linearly independent over $F$.
\end{propo}

\begin{proof}
  Sps to the contrary that
  \begin{equation*}
    c_{n - 1} \alpha^{n - 1} + c_{n - 2} \alpha^{n - 2} + \hdots + c_1 \alpha + c_0 = 0, \; c_i \in F,
  \end{equation*}
  has a non-trivial solution, i.e. not all $c_i$'s are $0$ (i.e. we assume that the $\alpha$'s are linearly
  dependent). Consider
  \begin{equation*}
    g(x) = c_{n - 1} x^{n - 1} + \hdots + c_1 x + c_0,
  \end{equation*}
  and so $g \neq 0$. However, $g(\alpha) = 0$, so $g(x) \in \langle f(x) \rangle$, i.e. $f(x) \mid g(x)$.
  However, that contradicts the fact that $\deg f = n > n - 1 \geq \deg g$.\qed\
\end{proof}

\begin{eg}
  Consider $K / F$, and $\alpha \in K$. Then
  \begin{equation*}
    \deg_F(\alpha) = 1 \iff \text{ min. polym } f(x) = x - \alpha \in F[x] \iff \alpha \in F.
  \end{equation*}
\end{eg}

\begin{eg}
  Consider $\mathbb{Q}(\sqrt{2}) / \mathbb{Q}$. Let $\alpha = \sqrt{2}$. Note that $f(\alpha) = 0$ for
  $f(x) = x^2 - 2$, which is irreducible by Eisenstein by $P = \langle 2 \rangle$. Thus 
  $\deg_F(\alpha) = 2$, and so
  \begin{equation*}
    \mathbb{Q}(\sqrt{2}) = \Span_{\mathbb{Q}} \{ 1, \alpha \} = \{ a + b \sqrt{2} \mmid a, b \in \mathbb{Q} \}.
  \end{equation*}
\end{eg}

\begin{eg}
  Let $\alpha = \sqrt{1 + \sqrt{3}}$. Notice that $\alpha^2 = 1 + \sqrt{3}$, and so $(\alpha^2 - 1)^2 = 3$.
  Thus
  \begin{equation*}
    \alpha^4 - 2 \alpha^2 + 1 - 3 = 0.
  \end{equation*}
  Let $f(x) = x^4 - 2x^2 - x \in \mathbb{Q}[x]$. Note that $f$ is monic and $f(\alpha) = 0$. By Eisenstein,
  $f$ is irreducible if we pick $P = \langle 2 \rangle$. Thus $f$ is a minimal polynomial for $\alpha$. We
  have that
  \begin{equation*}
    \deg_{\mathbb{Q}}(\alpha) = \deg f = 4.
  \end{equation*}
\end{eg}

\begin{eg}
  Let $f(x) = x^3 + x + 1 \in \mathbb{Z}_2[x]$. Let $\alpha$ be a root of $f(x)$ in some extension of
  $\mathbb{Z}_2$. Compute the size of $\mathbb{Z}_2(\alpha)$.
\end{eg}

\begin{solution}
  We showed in one of our previous examples that such an $f$ is irreducible in $\mathbb{Z}_2$. Thus
  $\deg_{\mathbb{Z}_2}(\alpha) = 3$. Then
  \begin{equation*}
    \mathbb{Z}_2(\alpha) = \Span_{\mathbb{Z}_2} \{ 1, \alpha, \alpha^2 \}, 
  \end{equation*}
  where $\{ 1, \alpha, \alpha^2 \}$ is linearly independent over $\mathbb{Z}_2$. Thus
  \begin{equation*}
    \abs{\mathbb{Z}_2(\alpha)} = 2 \times 2 \times 2 = 8.
  \end{equation*}
\end{solution}

\begin{note}
  Notice that there is no guarantee that such a root exists, but it does, which is a theorem that we shall
  prove later. (Link will be provided later) % TODO : include link to Kronecker's theorem
\end{note}

\begin{crly}[Isomorphism between Extensions]\label{crly:isomorphism_between_extensions}
  Let $K / F$ and $\alpha, \beta \in K$ have the same minimal polynomial $f(x) \in F[x]$. Then
  $F(\alpha) \simeq F(\beta)$.
\end{crly}

\begin{proof}
  From \cref{propo:span_of_the_extension}, we have that
  \begin{equation*}
    F(\alpha) \simeq F[x] / \langle f(x) \rangle \simeq F(\beta).
  \end{equation*}\qed\
\end{proof}

% section field_extensions_continued_2 (end)

% chapter lecture_9_jan_25th (end)

\chapter{Lecture 10 Jan 28th}%
\label{chp:lecture_10_jan_28th}
% chapter lecture_10_jan_28th 

\section{Field Extensions (Continued 3)}%
\label{sec:field_extensions_continued_3}
% section field_extensions_continued_3

How can we work with field extensions algebraically?

\subsection{Linear Algebra on Field Extensions}%
\label{sub:linear_algebra_on_field_extensions}
% subsection linear_algebra_on_field_extensions

We can look at $K / F$ as $K$ being an $F$-vector space.

\begin{defn}[Finite Extension]\index{Finite Extension}\label{defn:finite_extension}
  We say $K / F$ is a \hlnoteb{finite extension} if $K$ is a finite dimensional $F$-vector space.
  We call the dimension, $\dim_F K$, the \hldefn{degree} of $K / F$, and denote this dimension as
  \begin{equation*}
    [ K : F ].
  \end{equation*}
\end{defn}

\begin{eg}
  We have $[ \mathbb{C} : \mathbb{R} ] = \abs{ \{ 1, i \} } = 2$.
\end{eg}

\begin{eg}
  $[ \mathbb{R} : \mathbb{Q} ] = \infty$.
\end{eg}

\begin{eg}
  Let $K / F$ and $\alpha \in K$ with the minimal polynomial $f(x) \in F[x]$.
  Then $[ F(\alpha) : F ] = \abs{ \{ 1, \alpha, \ldots, \alpha^{n - 1} \} } = n$,
  where $n = \deg f = \deg_F(\alpha)$.\sidenote{This is why we call the dimension of $K/F$ as
  a degree.}
\end{eg}

\begin{defn}[Tower of Fields]\index{Tower of Fields}\label{defn:tower_of_fields}
  We say $F_1 / F_2 / F_3 / \hdots / F_n$ is a \hlnoteb{tower of fields} if each $F_i / F_{i + 1}$
  is a field extension.
\end{defn}

\begin{thm}[Tower Theorem]\index{Tower Theorem}\label{thm:tower_theorem}
  If $K / E$ and $E / F$ are finite extensions, then
  \begin{equation*}
    [K : F] = [K : E] [E : F].
  \end{equation*}
\end{thm}

\begin{proof}
  Let $\mathcal{B}_v = \left\{ v_1, \ldots, v_n \right\}$ be a basis for $K / E$ and
  $\mathcal{B}_w = \left\{ w_1, \ldots, w_m \right\}$ be a basis for $E / F$.

  \noindent
  \hlbnoted{Claim} The set $\left\{ v_i w_j : :1 \leq i \leq n, 1 \leq j \leq m \right\}$
  is a basis for $K / F$.

  \noindent
  \hlbnotea{Linear Independence} Assume
  \begin{equation}\label{eq:tower_of_fields_eq1}
    \sum_{i, j} c_{i, j} w_j v_i = 0.
  \end{equation}
  Notice that we may write \cref{eq:tower_of_fields_eq1} as
  \begin{equation*}
    \sum_{i} \left( \sum_{j} c_{i, j} w_j \right) v_i = 0.
  \end{equation*}
  Since $\mathcal{B}_v$ is a basis of $K / E$, for each $i$, we have
  \begin{equation*}
    \sum_{j} c_{i, j} w_j = 0.
  \end{equation*}
  Since $\mathcal{B}_w$ is a basis for $E / F$, for each $j$, we have
  \begin{equation*}
    c_{i, j} = 0.
  \end{equation*}
  It follows that the $w_j v_i$'s are linearly independent of each other.

  \noindent
  \hlbnotea{Span} Let $u \in K$. Then
  \begin{equation*}
    u = \sum_{i=1}^{n} c_i v_i,
  \end{equation*}
  where $c_i \in E$ is given by
  \begin{equation*}
    c_i = \sum_{j=1}^{m} d_{i, j} w_j.
  \end{equation*}
  Then
  \begin{equation*}
    u = \sum_{i, j} d_{i, j} w_j v_i.
  \end{equation*}
  Thus $\{ v_i, w_j \}$ is a basis for $K / F$.\qed\
\end{proof}

\begin{eg}
  Compute $[\mathbb{Q}(\sqrt[3]{5}, i) : \mathbb{Q}]$.
\end{eg}

\begin{solution}
  By the \hyperref[thm:tower_theorem]{Tower Theorem}, we have that
  \begin{equation*}
    [\mathbb{Q}(\sqrt[3]{5}, i) : \mathbb{Q}] 
      = [\mathbb{Q}(\sqrt[3]{5})(i) : \mathbb{Q}(\sqrt[3]{5})] \cdot [\mathbb{Q}(\sqrt[3]{5}) : \mathbb{Q}].
  \end{equation*}
  Notice that
  \begin{equation*}
    [\mathbb{Q}(\sqrt[3]{5}) : \mathbb{Q}] = \deg ( x^3 - 5 ) = 3.
  \end{equation*}
  For $[\mathbb{Q}(\sqrt[3]{5})(i) : \mathbb{Q}(\sqrt[3]{5})]$, let $p(x)$ be the minimal polynomial for $i$
  over $\mathbb{Q}(\sqrt[3]{5})$. Since $i^2 + 1 = 0$, we know that $i$ is a root of $x^2 + 1 = 0$. Then in
  particular, we must have $p(x) \mid x^2 + 1$. So $\deg p \in \{ 1, 2 \}$.

  Now since $\mathbb{Q}(\sqrt[3]{5}) \subseteq \mathbb{R}$ and $i \notin \mathbb{Q}(\sqrt[3]{5})$, we observe
  that $\deg p \neq 1$. Thus $\deg p = 2$. It follows that
  \begin{equation*}
    [\mathbb{Q}(\sqrt[3]{5})(i) : \mathbb{Q}(\sqrt[3]{5})] = 2.
  \end{equation*}
  Therefore
  \begin{equation*}
    [\mathbb{Q}(\sqrt[3]{5}, i) : \mathbb{Q}]  = 2 \cdot 3 = 6.
  \end{equation*}
\end{solution}

% subsection linear_algebra_on_field_extensions (end)

\subsection{Polynomials on Field Extensions}%
\label{sub:polynomials_on_field_extensions}
% subsection polynomials_on_field_extensions

\begin{defn}[Algebraic and Transcendental]\index{Algebraic}\index{Transcendental}\label{defn:algebraic_and_transcendental}
  Let $K / F$. We say that $\alpha \in K$ is \hlnoteb{algebraic} over $F$ if $\exists 0 \neq f(x) \in F[x]$
  such that $f(\alpha) = 0$. Otherwise, we say that $\alpha$ is \hlnoteb{transcendental} over $F$; that is,
  there is no non-zero polynomial over $F$ such that $\alpha$ is a root.

  We say that $K / F$ is algebraic if every $\alpha \in K$ is \hlnoteb{ algebraic } over $F$. Otherwise, we
  say that $K / F$ is \hlnoteb{transcendental}.
\end{defn}

\begin{eg}
  $\pi$ is transcendental over $\mathbb{Q}$ \sidenote{The proof of this statement is beyond our power at this
  point.}. However, $\pi$ is algebraic over $\mathbb{R}$ (note that $x - \pi \in \mathbb{R}[x]$.).
\end{eg}

\begin{eg}
  As a direct consequence of the above example, we have that $\mathbb{R}/\mathbb{Q}$ is transcendental.
\end{eg}

\begin{eg}
  As we have seen numerous times, $\mathbb{Q}(\sqrt{2}) / \mathbb{Q}$ is algebraic.
\end{eg}

\begin{remark}
  If $\alpha \in K$ is algebraic over $F$, then $\alpha$ has a minimal polynomial in $F[x]$.
\end{remark}

\begin{thm}[Finite Extensions are Algebraic]\label{thm:finite_extensions_are_algebraic}
  If $K / F$ is finite, then $K / F$ is algebraic.
\end{thm}

\begin{proof}
  \marginnote{\faLightbulbO The idea is to make use of the fact that the extension will at least have
  the algebraic number as a span up to some degree $n$, and instead of working with the spanning set,
  we work with one $\alpha$ away. There will be two cases, each of which can be dealt with at relative
  ease.}
  Suppose $[K : F] = n < \infty$. Let $\alpha \in K$. Consider
  \begin{equation*}
    \alpha, \alpha^2, \ldots, \alpha^n, \alpha^{n + 1}.
  \end{equation*}

  \noindent
  \hlbnotea{Case 1} Suppose $\alpha^i = \alpha^j$ for some $i \neq j \in \{ 1, \ldots, n + 1\}$. Then
  $\alpha$ is certainly a root of $f(x) = x^i - x^j$.

  \noindent
  \hlbnotea{Case 2} Suppose $\alpha^i \neq \alpha^j$ for all $i \neq j$. Then we must have that
  \begin{equation*}
    \alpha, \alpha^2, \ldots, \alpha^n, \alpha^{n + 1}
  \end{equation*}
  is linearly dependent over $F$. Thus we may have
  \begin{equation*}
    c_1 \alpha + c_2 \alpha^2 + \hdots + c_{n + 1} \alpha^{n + 1} = 0
  \end{equation*}
  where not all $c_i$'s are $0$. Then $\alpha$ is a root of
  \begin{equation*}
    f(x) = c_{ n + 1 } x^{n + 1} \hdots + c_1 x,
  \end{equation*}
  which is a non-zero polynomial.

  In either case, we observe that $\alpha$ is algebraic over $F$. Therefore $K / F$ is algebraic.\qed\
\end{proof}

% subsection polynomials_on_field_extensions (end)

% section field_extensions_continued_3 (end)

% chapter lecture_10_jan_28th  (end)

\chapter{Lecture 11 Jan 30th}%
\label{chp:lecture_11_jan_30th}
% chapter lecture_11_jan_30th

\section{Field Extensions (Continued 4)}%
\label{sec:field_extensions_continued_4}
% section field_extensions_continued_4

\subsection{Polynomials on Field Extensions (Continued)}%
\label{sub:polynomials_on_field_extensions_continued}
% subsection polynomials_on_field_extensions_continued

\begin{note}
  Recall that given $K / F$,
  \begin{itemize}
    \item Finite (defn): $\dim_F K = [ K : F ] < \infty$
    \item Algebraic (defn) :$\forall \alpha \in K, \exists 0 \neq f \in F[x]$, such that
      $f(\alpha) = 0$
    \item Finite $\implies$ Algebraic
  \end{itemize}
\end{note}

\begin{defn}[Finitely Generated Extension]\index{Finitely Generated Extension}\label{defn:finitely_generated_extension}
  We say $K$ is a \hlnoteb{finitely generated extension} of $F$ if $\exists \alpha_1,
  \alpha_2, \ldots, \alpha_n \in K$ such that $K = F(\alpha_1, \ldots, a_n)$.
\end{defn}

\begin{propo}[Finitely Generated Algebraic Extensions are Finite]\label{propo:finitely_generated_algebraic_extensions_are_finite}
  If $K$ is a finitely generated algebraic extension of $F$, then $K / F$ is finite.\sidenote{This
  proposition is actually an \hlnotea{iff} statemnt in disguise.}
\end{propo}

\begin{proof}
  Sps $K / F$ is algebraic, where $K = F(\alpha_1, \ldots, \alpha_n)$. We shall proceed by performing
  induction on $n$. If $n = 1$, then $[F(\alpha_1) : F] = \deg_F(\alpha_1) < \infty$.

  Now suppose that the result holds for $n$. Consider $K = F(\alpha_1, \ldots, \alpha_n, \alpha_{n + 1})$.
  Then by the \hyperref[thm:tower_theorem]{Tower Theorem},
  \begin{align*}
    &[F(\alpha_1, \ldots, \alpha_n, \alpha_{n + 1}) : F] \\
    &= [F(\alpha_1, \ldots, \alpha_{n})(\alpha_{n + 1}) : F(\alpha_1, \ldots, \alpha_n)]
        \cdot [F(\alpha_1, \ldots, \alpha_n) : F].
  \end{align*}
  It follows from the base case and the induction hypothesis that $[F(\alpha_1, \ldots, \alpha_{n +1}):F]$
  is finite.\qed\
\end{proof}

\begin{note}
  Finite extensions are, therefore, finitely generated.
\end{note}

\begin{eg}
  The field $\mathbb{Q}(\sqrt{2}, \sqrt{3}, \sqrt{4}, \ldots)$ is an algebraic extension of $\mathbb{Q}$
  but it is not a finite extension.
\end{eg}

\begin{propo}[Greater Algebraic Extensions]\label{propo:greater_algebraic_extensions}
  If $K / E$ and $E / F$ are algebraic extensions, then $K / F$ is an algebraic extension.
\end{propo}

\begin{proof}
  Let $\alpha \in K$. Since $K / E$ is algebraic, $\alpha$ has a minimal polynomial in $E[x]$, say it is
  \begin{equation*}
    p(x) = x^n + c_{n - 1} x^{n - 1} + \hdots + c_1 x + c_0.
  \end{equation*}
  Then $\alpha$ is algebraic over $F(c_{n - 1}, \ldots, c_1, c_0)$. By the Tower Theorem,
  \begin{equation*}
    [F(c_{n - 1}, \ldots, c_1, c_0, \alpha) : F(c_{n - 1}, \ldots, c_0)] < \infty,
  \end{equation*}
  and so $F(c_{n - 1}, \ldots, c_1, c_0, \alpha) \subseteq E$.
  
  Now $F(c_{n - 1}, \ldots, c_0) / F$ is algebraic and finitely generated. So it follows from the Tower
  Theorem that
  \begin{equation*}
    [F(c_{n - 1}, \ldots, c_0, \alpha) : F] < \infty.
  \end{equation*}
  Thus $\alpha$ is algebraic over $F$ and so $K / F$ is algebraic.\qed\
\end{proof}

\begin{propo}[Algebraic Numbers Form a Subfield]\label{propo:algebraic_numbers_form_a_subfield}
  Let $K / F$. The set of elements of $K$ algebraic over $F$ form a subfield of $K$.
\end{propo}

\begin{proof}[Sketch proof]
  Let $L = \{ \alpha \in K : \alpha \text{ is alg. over } F \}$. Let $\alpha, \beta \in L$ and $\beta \neq 0$.
  Then
  \begin{equation*}
    \alpha, \beta, \alpha + \beta, \alpha \beta, \beta^{-1} \in F(\alpha, \beta).
  \end{equation*}
  Then $[F(\alpha, \beta) : F] < \infty$ implies that $L$ is finitely generated, which is thus algebraic,
  and is hence a subfield of $K$.\qed\
\end{proof}

% subsection polynomials_on_field_extensions_continued (end)

% section field_extensions_continued_4 (end)

\section{Splitting Fields}%
\label{sec:splitting_fields}
% section splitting_fields



% section splitting_fields (end)

% chapter lecture_11_jan_30th (end)

\appendix

\chapter{Asides and Prior Knowledge}%
\label{chp:asides_and_prior_knowledge}
% chapter asides_and_prior_knowledge

\section{Correspondence Theorem}%
\label{sec:correspondence_theorem}
% section correspondence_theorem

\hlnotea{The Correspondence Theorem} is somewhat widely known 
as the Fourth Isomorphism Theorem, although some authors associates
the name with a proposition known as 
\href{https://en.wikipedia.org/wiki/Zassenhaus_lemma}{Zaessenhaus Lemma}.

\begin{thm}[Correspondence Theorem]\index{Correspondence Theorem}\label{thm:correspondence_theorem}
  Let $G$ be a group, and $N \triangleleft G$
  \sidenote{Recall that this symbol means that $N$ is a normal
  subgroup of $G$.}. Then there exists a bijection between
  the set of all subgroups $A \leq G$ such that $A \supseteq N$
  and the set of subgroups $A / N$ of $G / N$.
\end{thm}

\begin{proof}
  % TODO: prove the Correspondence Theorem
\end{proof}

% section correspondence_theorem (end)

% chapter asides_and_prior_knowledge (end)

\backmatter

\pagestyle{plain}

\nobibliography*
\bibliography{references}

\printindex

\end{document}

