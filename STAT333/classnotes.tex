\documentclass[notoc,notitlepage]{tufte-book}
% \nonstopmode % uncomment to enable nonstopmode

\usepackage{classnotetitle}

\title{STAT333 - Applied Probability}
\author{Johnson Ng}
\subtitle{Classnotes for Winter 2017}
\credentials{BMath (Hons), Pure Mathematics major, Actuarial Science Minor}
\institution{University of Waterloo}

\setcounter{secnumdepth}{3}
\setcounter{tocdepth}{3}

\renewcommand{\baselinestretch}{1.2}
\usepackage{geometry}
\geometry{letterpaper}
\usepackage[parfill]{parskip}
\usepackage{graphicx}

% Essential Packages
\usepackage{makeidx}
\makeindex
\usepackage{enumitem}
\usepackage[T1]{fontenc}
\usepackage{natbib}
\bibliographystyle{apalike}
\usepackage{ragged2e}
\usepackage{etoolbox}
\usepackage{amssymb}
\usepackage{eso-pic}
\usepackage[fixed]{fontawesome5}
\usepackage{todonotes}
\usepackage{apptools, chngcntr}
\usepackage{amsmath}
\usepackage{mathrsfs}
\usepackage{stmaryrd}
\usepackage{mathtools}
\usepackage{tocloft}
\usepackage{tocbibind}
\usepackage{xparse}
\usepackage{tkz-euclide}
\usetkzobj{all}
\usepackage[utf8]{inputenc}
\usepackage{csquotes}
\usepackage[english]{babel}
\usepackage{marvosym}
\usepackage{pgf,tikz}
\usepackage{tikz-cd}
\usepackage{ifthen}
\usepackage{pgfplots}
\usepackage{fancyhdr}
\usepackage{array}
\usepackage{float}
\usepackage{xcolor}
\usepackage{soul}
\usepackage{centernot}
\usepackage{silence}
  \WarningFilter*{latex}{Marginpar on page \thepage\space moved}
\usepackage{tcolorbox}
\tcbuselibrary{skins,breakable}
\usepackage{longtable,booktabs}
\usepackage[amsmath,hyperref,thmmarks]{ntheorem}
\usepackage{thmtools}
\usepackage{hyperref}
\usepackage[noabbrev,capitalize,nameinlink]{cleveref}

\newcommand{\personalcolor}{false}
\ifthenelse{\equal{\personalcolor}{true}}{
  \usepackage{colorscheme-chaos}
}{
  \usepackage{colorscheme-student}
}

% hyperref Package Settings
\hypersetup{
    unicode=true,          % non-Latin characters in Acrobat’s bookmarks
    pdftoolbar=false,        % show Acrobat’s toolbar?
    pdfmenubar=false,        % show Acrobat’s menu?
    pdffitwindow=true,     % window fit to page when opened
    colorlinks=true,
    allcolors=magenta,
}

% tikz
\usepgfplotslibrary{polar}
\usepgflibrary{shapes.geometric}
\usetikzlibrary{angles,patterns,calc,decorations.markings,arrows.meta,tikzmark,bending}
\tikzset{midarrow/.style 2 args={
        decoration={markings,
            mark= at position #2 with {\arrow{#1}} ,
        },
        postaction={decorate}
    },
    midarrow/.default={latex}{0.5}
}
\def\centerarc[#1](#2)(#3:#4:#5)% Syntax: [draw options] (center) (initial angle:final angle:radius)
    { \draw[#1] ($(#2)+({#5*cos(#3)},{#5*sin(#3)})$) arc (#3:#4:#5); }
% from https://tex.stackexchange.com/questions/67573/tikz-shift-and-rotate-in-3d
\newcommand{\rotateRPY}[4][0/0/0]% point to be saved to \savedxyz, roll, pitch, yaw
{   \pgfmathsetmacro{\rollangle}{#2}
    \pgfmathsetmacro{\pitchangle}{#3}
    \pgfmathsetmacro{\yawangle}{#4}

    % to what vector is the x unit vector transformed, and which 2D vector is this?
    \pgfmathsetmacro{\newxx}{cos(\yawangle)*cos(\pitchangle)}% a
    \pgfmathsetmacro{\newxy}{sin(\yawangle)*cos(\pitchangle)}% d
    \pgfmathsetmacro{\newxz}{-sin(\pitchangle)}% g
    \path (\newxx,\newxy,\newxz);
    \pgfgetlastxy{\nxx}{\nxy};

    % to what vector is the y unit vector transformed, and which 2D vector is this?
    \pgfmathsetmacro{\newyx}{cos(\yawangle)*sin(\pitchangle)*sin(\rollangle)-sin(\yawangle)*cos(\rollangle)}% b
    \pgfmathsetmacro{\newyy}{sin(\yawangle)*sin(\pitchangle)*sin(\rollangle)+ cos(\yawangle)*cos(\rollangle)}% e
    \pgfmathsetmacro{\newyz}{cos(\pitchangle)*sin(\rollangle)}% h
    \path (\newyx,\newyy,\newyz);
    \pgfgetlastxy{\nyx}{\nyy};

    % to what vector is the z unit vector transformed, and which 2D vector is this?
    \pgfmathsetmacro{\newzx}{cos(\yawangle)*sin(\pitchangle)*cos(\rollangle)+ sin(\yawangle)*sin(\rollangle)}
    \pgfmathsetmacro{\newzy}{sin(\yawangle)*sin(\pitchangle)*cos(\rollangle)-cos(\yawangle)*sin(\rollangle)}
    \pgfmathsetmacro{\newzz}{cos(\pitchangle)*cos(\rollangle)}
    \path (\newzx,\newzy,\newzz);
    \pgfgetlastxy{\nzx}{\nzy};

    % transform the point given by #1
    \foreach \x/\y/\z in {#1}
    {   \pgfmathsetmacro{\transformedx}{\x*\newxx+\y*\newyx+\z*\newzx}
        \pgfmathsetmacro{\transformedy}{\x*\newxy+\y*\newyy+\z*\newzy}
        \pgfmathsetmacro{\transformedz}{\x*\newxz+\y*\newyz+\z*\newzz}
        \xdef\savedx{\transformedx}
        \xdef\savedy{\transformedy}
        \xdef\savedz{\transformedz}     
    }
}
\tikzset{RPY/.style={x={(\nxx,\nxy)},y={(\nyx,\nyy)},z={(\nzx,\nzy)}}}
\newcommand{\AxisRotator}[1][rotate=0]{%
    \tikz [x=0.25cm,y=0.60cm,line width=.2ex,-stealth,#1] \draw (0,0) arc (-150:150:1 and 1);%
  }

% enumitems
\newlist{inlinelist}{enumerate*}{1}
\setlist*[inlinelist,1]{%
  label=(\roman*),
}

% Theorem Style Customization
\setlength\theorempreskipamount{2ex}
\setlength\theorempostskipamount{3ex}

\makeatletter
\let\nobreakitem\item
\let\@nobreakitem\@item
\patchcmd{\nobreakitem}{\@item}{\@nobreakitem}{}{}
\patchcmd{\nobreakitem}{\@item}{\@nobreakitem}{}{}
\patchcmd{\@nobreakitem}{\@itempenalty}{\@M}{}{}
\patchcmd{\@xthm}{\ignorespaces}{\nobreak\ignorespaces}{}{}
\patchcmd{\@ythm}{\ignorespaces}{\nobreak\ignorespaces}{}{}

\renewtheoremstyle{break}%
  {\item{\theorem@headerfont
          ##1\ ##2\theorem@separator}\hskip\labelsep\relax\nobreakitem}%
  {\item{\theorem@headerfont
          ##1\ ##2\ (##3)\theorem@separator}\hskip\labelsep\relax\nobreakitem}
\makeatother

% ntheorem Declarations
\theorempreskip{10pt}
\theorempostskip{5pt}
\theoremstyle{break}

\theoremsymbol{\faComment}
\newtheorem{remark}{Remark}[section]
\theoremsymbol{}
\newtheorem*{strategy}{\faPaperPlane Strategy}
\newtheorem*{procedure}{\faCodeBranch\ }
\newtheorem{ex}{Exercise}[section]
\theorembodyfont{\normalfont}
\newtheorem*{solution}{\faPencil* Solution}
\theoremsymbol{\faGavel}
\newtheorem{eg}{Example}[section]
\theoremsymbol{}
\theorembodyfont{\it}

    % definition env
\theoremprework{\textcolor{blue}{\hrule height 2pt width \textwidth}}
\theoremheaderfont{\color{blue}\normalfont\bfseries}
\theorempostwork{\textcolor{blue}{\hrule height 2pt width \textwidth}}
\theoremindent10pt
\newtheorem{defn}{\faBook Definition}

    % definition env no num
\theoremprework{\textcolor{blue}{\hrule height 2pt width \textwidth}}
\theoremheaderfont{\color{blue}\normalfont\bfseries}
\theorempostwork{\textcolor{blue}{\hrule height 2pt width \textwidth}}
\theoremindent10pt
\newtheorem*{defnnonum}{\faBook Definition}

\theoremprework{\textcolor{blue}{\hrule height 2pt width \marginparwidth}}
\theoremheaderfont{\color{blue}\normalfont\bfseries}
\theorempostwork{\textcolor{blue}{\hrule height 2pt width \marginparwidth}}
\theoremindent10pt
\newtheorem{margindefn}[defn]{\faBook Definition}

\theoremprework{\textcolor{blue}{\hrule height 2pt width \textwidth}}
\theoremheaderfont{\color{blue}\normalfont\bfseries}
\theorempostwork{\textcolor{blue}{\hrule height 2pt width \textwidth}}
\theoremindent10pt
\newtheorem*{margindefnnonum}{\faBook Definition}

    % theorem envs
\theoremprework{\textcolor{magenta}{\hrule height 2pt width \textwidth}}
\theoremheaderfont{\color{magenta}\normalfont\bfseries}
\theorempostwork{\textcolor{magenta}{\hrule height 2pt width \textwidth}}
\theoremindent10pt
\newtheorem{thm}{\faCoffee Theorem}

\theoremprework{\textcolor{magenta}{\hrule height 2pt width \textwidth}}
\theorempostwork{\textcolor{magenta}{\hrule height 2pt width \textwidth}}
\theoremindent10pt
\newtheorem{propo}[thm]{\faTint Proposition}

\theoremprework{\textcolor{magenta}{\hrule height 2pt width \textwidth}}
\theorempostwork{\textcolor{magenta}{\hrule height 2pt width \textwidth}}
\theoremindent10pt
\newtheorem{crly}[thm]{\faSpaceShuttle Corollary}

\theoremprework{\textcolor{magenta}{\hrule height 2pt width \textwidth}}
\theorempostwork{\textcolor{magenta}{\hrule height 2pt width \textwidth}}
\theoremindent10pt
\newtheorem{lemma}[thm]{\faTree Lemma}

\theoremprework{\textcolor{magenta}{\hrule height 2pt width \textwidth}}
\theorempostwork{\textcolor{magenta}{\hrule height 2pt width \textwidth}}
\theoremindent10pt
\newtheorem{axiom}[thm]{\faShield Axiom}

    % theorem envs without counter
\theoremprework{\textcolor{magenta}{\hrule height 2pt width \textwidth}}
\theoremheaderfont{\color{magenta}\normalfont\bfseries}
\theorempostwork{\textcolor{magenta}{\hrule height 2pt width \textwidth}}
\theoremindent10pt
\newtheorem*{thmnonum}{\faCoffee Theorem}

\theoremprework{\textcolor{magenta}{\hrule height 2pt width \textwidth}}
\theorempostwork{\textcolor{magenta}{\hrule height 2pt width \textwidth}}
\theoremindent10pt
\newtheorem*{propononum}{\faTint Proposition}

\theoremprework{\textcolor{magenta}{\hrule height 2pt width \textwidth}}
\theorempostwork{\textcolor{magenta}{\hrule height 2pt width \textwidth}}
\theoremindent10pt
\newtheorem*{crlynonum}{\faSpaceShuttle Corollary}

\theoremprework{\textcolor{magenta}{\hrule height 2pt width \textwidth}}
\theorempostwork{\textcolor{magenta}{\hrule height 2pt width \textwidth}}
\theoremindent10pt
\newtheorem*{lemmanonum}{\faTree Lemma}

\theoremprework{\textcolor{magenta}{\hrule height 2pt width \textwidth}}
\theorempostwork{\textcolor{magenta}{\hrule height 2pt width \textwidth}}
\theoremindent10pt
\newtheorem*{axiomnonum}{\faShield Axiom}

    % envs on margins
\theoremprework{\textcolor{magenta}{\hrule height 2pt width \marginparwidth}}
\theoremheaderfont{\color{magenta}\normalfont\bfseries}
\theorempostwork{\textcolor{magenta}{\hrule height 2pt width \marginparwidth}}
\theoremindent10pt
\newtheorem{marginthm}[thm]{\faCoffee Theorem}

\theoremprework{\textcolor{magenta}{\hrule height 2pt width \marginparwidth}}
\theorempostwork{\textcolor{magenta}{\hrule height 2pt width \marginparwidth}}
\theoremindent10pt
\newtheorem{marginpropo}[thm]{\faTint Proposition}

\theoremprework{\textcolor{magenta}{\hrule height 2pt width \marginparwidth}}
\theorempostwork{\textcolor{magenta}{\hrule height 2pt width \marginparwidth}}
\theoremindent10pt
\newtheorem{margincrly}[thm]{\faSpaceShuttle Corollary}

\theoremprework{\textcolor{magenta}{\hrule height 2pt width \marginparwidth}}
\theorempostwork{\textcolor{magenta}{\hrule height 2pt width \marginparwidth}}
\theoremindent10pt
\newtheorem{marginlemma}[thm]{\faTree Lemma}

\theoremprework{\textcolor{magenta}{\hrule height 2pt width \marginparwidth}}
\theorempostwork{\textcolor{magenta}{\hrule height 2pt width \marginparwidth}}
\theoremindent10pt
\newtheorem{marginaxiom}[thm]{\faShield Axiom}

    % envs on margins without counter
\theoremprework{\textcolor{magenta}{\hrule height 2pt width \marginparwidth}}
\theoremheaderfont{\color{magenta}\normalfont\bfseries}
\theorempostwork{\textcolor{magenta}{\hrule height 2pt width \marginparwidth}}
\theoremindent10pt
\newtheorem*{marginthmnonum}{\faCoffee Theorem}

\theoremprework{\textcolor{magenta}{\hrule height 2pt width \marginparwidth}}
\theorempostwork{\textcolor{magenta}{\hrule height 2pt width \marginparwidth}}
\theoremindent10pt
\newtheorem*{marginpropononum}{\faTint Proposition}

\theoremprework{\textcolor{magenta}{\hrule height 2pt width \marginparwidth}}
\theorempostwork{\textcolor{magenta}{\hrule height 2pt width \marginparwidth}}
\theoremindent10pt
\newtheorem*{margincrlynonum}{\faSpaceShuttle Corollary}

\theoremprework{\textcolor{magenta}{\hrule height 2pt width \marginparwidth}}
\theorempostwork{\textcolor{magenta}{\hrule height 2pt width \marginparwidth}}
\theoremindent10pt
\newtheorem*{marginlemmanonum}{\faTree Lemma}

\theoremprework{\textcolor{magenta}{\hrule height 2pt width \marginparwidth}}
\theorempostwork{\textcolor{magenta}{\hrule height 2pt width \marginparwidth}}
\theoremindent10pt
\newtheorem*{marginaxiomnonum}{\faShield Axiom}

    % proof env
\theoremprework{\textcolor{green}{\hrule height 2pt width \textwidth}}
\theorembodyfont{\normalfont}
\theoremheaderfont{\color{green}\normalfont\bfseries}
\theorempostwork{\textcolor{green}{\hrule height 2pt width \textwidth}}
\theoremsymbol{\ensuremath{_\square}}
\newtheorem*{proof}{\faPencil* Proof}
\theoremsymbol{}

\theoremprework{\textcolor{green}{\hrule height 2pt width \marginparwidth}}
\theorembodyfont{\normalfont}
\theoremheaderfont{\color{green}\normalfont\bfseries}
\theorempostwork{\textcolor{green}{\hrule height 2pt width \marginparwidth}}
\theoremsymbol{\ensuremath{_\square}}
\newtheorem*{mproof}{\faPencil* Proof}
\theoremsymbol{}

    % note and notation env
\theorembodyfont{\it}

\theoremprework{\textcolor{yellow}{\hrule height 2pt width \textwidth}}
\theoremheaderfont{\color{yellow}\normalfont\bfseries}
\theorempostwork{\textcolor{yellow}{\hrule height 2pt width \textwidth}}
\newtheorem{note}{\faQuoteLeft Note}[section]

\theoremprework{\textcolor{yellow}{\hrule height 2pt width \marginparwidth}}
\theoremheaderfont{\color{yellow}\normalfont\bfseries}
\theorempostwork{\textcolor{yellow}{\hrule height 2pt width \marginparwidth}}
\newtheorem{mnote}[note]{\faQuoteLeft Note}

\theoremprework{\textcolor{yellow}{\hrule height 2pt width \textwidth}}
\theorempostwork{\textcolor{yellow}{\hrule height 2pt width \textwidth}}
\newtheorem*{notation}{\faPaw Notation}

    % warning env
\theoremprework{\textcolor{red}{\hrule height 2pt width \textwidth}}
\theoremheaderfont{\color{red}\normalfont\bfseries}
\theorempostwork{\textcolor{red}{\hrule height 2pt width \textwidth}}
\theoremindent10pt
\newtheorem*{warning}{\faBug Warning}

\theoremprework{\textcolor{red}{\hrule height 2pt width \marginparwidth}}
\theoremheaderfont{\color{red}\normalfont\bfseries}
\theorempostwork{\textcolor{red}{\hrule height 2pt width \marginparwidth}}
\theoremindent10pt
\newtheorem*{marginwarning}{\faBug Warning}

% rule for appendices
\AtAppendix{\counterwithin{defn}{chapter}}
\AtAppendix{\counterwithin{thm}{chapter}}
\AtAppendix{\counterwithin{propo}{chapter}}
\AtAppendix{\counterwithin{lemma}{chapter}}
\AtAppendix{\counterwithin{crly}{chapter}}
\AtAppendix{\counterwithin{axiom}{chapter}}

% more environments
\newtcolorbox{quotebox}[2]{
  blanker,enhanced,breakable,standard jigsaw,
  opacityback=0,
  coltext=\ifblank{#2}{black}{#2},
  left=5mm,right=5mm,top=2mm,bottom=2mm,
  colframe=\ifblank{#1}{bblack}{#1},
  boxrule=0pt,leftrule=3pt,
  fontupper=\itshape
}

\providecommand{\parthook}{}
\patchcmd{\part}{\thispagestyle}{\parthook\thispagestyle}{}{}
\newcommand{\partimage}[2][]{% \parthook[<options>]{<image>}
  \renewcommand{\parthook}{% Update \parthook
    \AddToShipoutPictureBG*{% Add picture to background of THIS page only
      \AtPageLowerLeft{\includegraphics[width=\paperwidth,height=\paperheight,#1]{#2}}}% Insert image
    \renewcommand{\parthook}{}}}% Restore \parthook

\AtBeginDocument{\renewcommand\contentsname{\slshape Table of Contents\normalfont}}
\cftpagenumbersoff{part}

\newcommand{\tuftepart}[1]{\newgeometry{}\part{#1}\restoregeometry}

% Heading formattings
% chapter format
\titleformat{\chapter}%
  {\huge\rmfamily\itshape\color{magenta}}% format applied to label+text
  {\llap{\colorbox{magenta}{\parbox[c][1cm]{3cm}{\hfill\itshape\Huge\textcolor{background}{\thechapter}}}}}% label
  {5pt}% horizontal separation between label and title body
  {\faLeaf}% before the title body
  []% after the title body

% section format
\titleformat{\section}%
  {\normalfont\Large\rmfamily\itshape\color{blue}}% format applied to label+text
  {\llap{\colorbox{blue}{\parbox{3cm}{\hfill\itshape\textcolor{background}{\thesection}}}}}% label
  {5pt}% horizontal separation between label and title body
  {}% before the title body
  []% after the title body

% subsection format
\titleformat{\subsection}%
  {\normalfont\large\itshape\color{green}}% format applied to label+text
  {\llap{\colorbox{green}{\parbox{3cm}{\hfill\textcolor{background}{\thesubsection}}}}}% label
  {1em}% horizontal separation between label and title body
  {}% before the title body
  []% after the title body

% subsubsection format
\titleclass{\subsubsection}{straight}
\titleformat{\subsubsection}%
  {\normalfont\large\itshape\color{yellow}}% format applied to label+text
  {\llap{\colorbox{yellow}{\parbox{3cm}{\hfill\textcolor{background}{\thesubsubsection}}}}}% label
  {1em}% horizontal separation between label and title body
  {}% before the title body
  []% after the title body

% Sidenote enhancements
\def\mathmarginnote#1{%
  \tag*{\rlap{\hspace\marginparsep\smash{\parbox[t]{\marginparwidth}{%
  \footnotesize#1}}}}
}

% Custom table columning
\newcolumntype{L}[1]{>{\raggedright\let\newline\\\arraybackslash\hspace{0pt}}m{#1}}
\newcolumntype{C}[1]{>{\centering\let\newline\\\arraybackslash\hspace{0pt}}m{#1}}
\newcolumntype{R}[1]{>{\raggedleft\let\newline\\\arraybackslash\hspace{0pt}}m{#1}}

% Graph styles
\pgfplotsset{compat=1.15}
\usepgfplotslibrary{fillbetween}
\pgfplotsset{four quads/.append style={axis x line=middle, axis y line=
middle, xlabel={$x$}, ylabel={$y$}, axis equal }}
\pgfplotsset{four quad complex/.append style={axis x line=middle, axis y line=
middle, xlabel={$\re$}, ylabel={$\im$}, axis equal }}
\def\axisdefaultwidth{360pt}
\pgfplotsset{
  tufteaxis/.append style = {thick},tick style = {thick,black},
  %
  % #1 = x, y, or z
  % #2 = the shift value
  /tikz/normal shift/.code 2 args = {%
    \pgftransformshift{%
        \pgfpointscale{#2}{\pgfplotspointouternormalvectorofticklabelaxis{#1}}%
    }%
  },%
  %
  range3frame/.style = {
    tick align        = outside,
    scaled ticks      = false,
    enlargelimits     = false,
    ticklabel shift   = {10pt},
    axis lines*       = left,
    line cap          = round,
    clip              = false,
    xtick style       = {normal shift={x}{10pt}},
    ytick style       = {normal shift={y}{10pt}},
    ztick style       = {normal shift={z}{10pt}},
    x axis line style = {normal shift={x}{10pt}},
    y axis line style = {normal shift={y}{10pt}},
    z axis line style = {normal shift={z}{10pt}},
  }
}

% Shortcuts
\DeclareMathOperator{\id}{id}
\DeclareMathOperator{\Img}{Img}
\DeclareMathOperator{\Res}{Res}
\DeclareMathOperator*{\argmax}{arg\,max}
\DeclareMathOperator*{\argmin}{arg\,min}
\DeclareMathOperator{\re}{Re}
\DeclareMathOperator{\im}{Im}
\DeclareMathOperator{\caparg}{Arg}
\DeclareMathOperator{\Char}{Char}
\DeclareMathOperator{\sgn}{sgn}
\DeclareMathOperator{\Range}{range}

\newcommand{\floor}[1]{\lfloor #1 \rfloor}      % simplifying the writing of a floor function
\newcommand{\ceiling}[1]{\lceil #1 \rceil}      % simplifying the writing of a ceiling function
\newcommand{\dotp}{\, \cdotp}			        % dot product to distinguish from \cdot
\newcommand{\abs}[1]{\left|#1\right|}						% absolute value
\newcommand{\lra}[1]{\left\langle \; #1 \; \right\rangle}
\newcommand{\at}[2]{\Big|_{#1}^{#2}}
\newcommand{\Arg}[1]{\caparg #1}
\renewcommand{\bar}[1]{\mkern 1.5mu \overline{\mkern -1.5mu #1 \mkern -1.5mu} \mkern 1.5mu}
\newcommand{\faktor}[2]{{\raisebox{.2em}{$#1$}\left/\raisebox{-.2em}{$#2$}\right.}}
\newcommand{\quotient}[2]{\faktor{#1}{#2}}
\newcommand{\cyclic}[1]{\left\langle #1 \right\rangle}
\newcommand{\ind}[2]{\Ind_{#2}\left( #1 \right)}
\newcommand{\notimply}{\centernot\implies}
\newcommand{\res}[2]{\underset{#2}{\Res} #1 }
\newcommand{\tworow}[3]{\begin{tabular}{@{}#1@{}} #2 \\ #3 \end{tabular}}
\renewcommand{\epsilon}{\varepsilon}
\renewcommand{\phi}{\varphi}
\newcommand{\lrarrow}{\leftrightarrow}
\newcommand{\larrow}{\leftarrow}
\newcommand{\rarrow}{\rightarrow}
\renewcommand{\atop}[2]{\genfrac{}{}{0pt}{}{#1}{#2}}
\newcommand*\dif{\mathop{}\!d}
\newcommand{\mmid}{\; \middle| \;}
\newcommand{\coprime}{\; \bot \;}
\newcommand{\norm}[1]{\left\| #1 \right\|}
\newenvironment{spmatrix}
  {\left(\begin{smallmatrix}}
  {\end{smallmatrix}\right)}

  % inspiration from: https://tex.stackexchange.com/questions/8720/overbrace-underbrace-but-with-an-arrow-instead#37758
\newcommand{\overarrow}[2]{
  \overset{\makebox[0pt]{\begin{tabular}{@{}c@{}}#2\\[0pt]\ensuremath{\uparrow}\end{tabular}}}{\ensuremath{#1}}
}
\newcommand{\underarrow}[2]{
  \underset{\makebox[0pt]{\begin{tabular}{@{}c@{}}\downarrow\\[0pt]\ensuremath{#2}\end{tabular}}}{\ensuremath{#1}}
}


	% highlighting shortcuts
\newcommand{\hlimpo}[1]{\textcolor{red}{\textbf{#1}}}
\newcommand{\hlwarn}[1]{\textcolor{yellow}{\textbf{#1}}}
\newcommand{\hldefn}[1]{\textcolor{blue}{\index{#1}\textbf{#1}}}
\newcommand{\hlnotea}[1]{\textcolor{green}{\textbf{#1}}}
\newcommand{\hlnoteb}[1]{\textcolor{cyan}{\textbf{#1}}}
\newcommand{\hlb}[2]{\colorbox{#1!30!background}{#2}}
\newcommand{\hlbnotea}[1]{\hlb{green}{#1}}
\newcommand{\hlbnoteb}[1]{\hlb{cyan}{#1}}
\newcommand{\hlbnotec}[1]{\hlb{yellow}{#1}}
\newcommand{\hlbnoted}[1]{\hlb{magenta}{#1}}
\newcommand{\hlbnotee}[1]{\hlb{red}{#1}}
\newcommand{\WTP}{\textcolor{bwhite}{WTP} }
\newcommand{\WTS}{\textcolor{bwhite}{WTS} }

  % stars on important stuff
\newcommand{\imponote}{\faStar}
\newcommand{\vimponote}{\faStar\faStar}
\newcommand{\vvimponote}{\faStar\faStar\faStar}

% Document header formatting
\makeatletter
\pagestyle{fancy}
\fancyhead{}
\fancyhead[RO]{\textsl{\@title} \enspace \thepage}
\fancyhead[LE]{\thepage \enspace \textsl{\leftmark \enspace \rightmark}}
\makeatother
\renewcommand{\chaptermark}[1]{\markboth{#1}{}}
\renewcommand{\sectionmark}[1]{\markright{#1}}

% Comment the two lines below if you want to print the document
\pagecolor{background}
\color{foreground}


\DeclareMathOperator{\Bernoulli}{Bernoulli }
\DeclareMathOperator{\Bin}{Bin }
\DeclareMathOperator{\Geo}{Geo }
\DeclareMathOperator{\Poi}{Poi }
\DeclareMathOperator{\NB}{NB }
\DeclareMathOperator{\Exp}{Exp }
\DeclareMathOperator{\Unif}{Unif }
\DeclareMathOperator{\Nor}{N }
\DeclareMathOperator{\Gau}{G }
\DeclareMathOperator{\HG}{HG }
\DeclareMathOperator{\Gam}{Gam }
\DeclareMathOperator{\BetaDist}{Beta }
\DeclareMathOperator{\Mult}{Mult }
\DeclareMathOperator{\BVN}{BVN }
\DeclareMathOperator{\Wei}{Wei }
\DeclareMathOperator{\Dom}{Dom }
\DeclareMathOperator{\Var}{Var }
\DeclareMathOperator{\sd}{sd }
\DeclareMathOperator{\Cov}{Cov }
\DeclareMathOperator{\Corr}{Corr }

\begin{document}
\hypersetup{pageanchor=false}
\maketitle
\hypersetup{pageanchor=true}
\tableofcontents

\chapter*{\faBook \enspace List of Definitions}
\addcontentsline{toc}{chapter}{List of Definitions}
\theoremlisttype{all}
\listtheorems{defn}

\chapter*{\faCoffee \enspace List of Theorems}
\addcontentsline{toc}{chapter}{List of Theorems}
\theoremlisttype{allname}
\listtheorems{axiom,lemma,thm,crly,propo}

\chapter*{Foreword}%
\label{chp:foreword}
% chapter foreword

\begin{fullwidth}
  I am transcribing this set of notes from my handwritten ones in Winter 2017, back at a time which I have yet to organize my notes by lecture. However, I will try my best to organize them by chapters and topics as presented in class.

  I will try to be as rigourous as possible while transcribing my notes. However, given the nature of the course and the presentation, this will not always be possible, and I am mostly keeping these notes for ``legacy purposes'', and so I will not put too much effort into making the notes as complete as my newer ones.

  For this course, you are expected to have basic knowledge of probability in order to be able to understand the material.
\end{fullwidth}

% chapter foreword (end)

\chapter{Elementary Probability Review}%
\label{chp:elementary_probability_review}
% chapter elementary_probability_review

\section{Introductions}%
\label{sec:introductions}
% section introductions

\begin{defn}[Fundamental Definition of a Probability Function]
\label{defn:fundamental_definition_of_a_probability_function}
  For each event $A$ of a sample space $S$, $P(A)$ is defined as the ``\hlnoteb{probability of the event $A$}'', satisfying these 3 conditions:
  \begin{enumerate}
    \item $0 \leq P(A) \leq 1$
    \item $P(S) = 1$ \sidenote{This can also be stated as $P(\emptyset) = 0$, where $\emptyset$ is the null event.} \label{item:funddefn_probfn_item2}
    \item $P\left( \bigcup\limits_{i = 1}^{n} A_i \right) = \sum\limits_{i=1}^{n} P(A_i)$, where $A_i \cap A_j = A_i A_j = \emptyset$ for all $i \neq j$ \sidenote{We can also say that the sequence $\{A_i\}_{i = 1}^{n}$ has mutually exclusive elements.} \label{item:funddefn_probfn_item3}
  \end{enumerate}
\end{defn}

\begin{note}
  By \cref{item:funddefn_probfn_item2} and \cref{item:funddefn_probfn_item3}, we have
  \begin{equation*}
    1 = P(S) = P(A \cup A^C) = P(A) + P(A^C)
  \end{equation*}
  which implies that
  \begin{equation*}
    P(A^C) = 1 - P(A).
  \end{equation*}
\end{note}

\begin{defn}[Conditional Probability]\index{Conditional Probability}
\label{defn:conditional_probability}
  Given events $A$ and $B$ in a sample space $S$, the \hlnoteb{conditional probability of $A$ given $B$} is given by
  \begin{equation}\label{eq:conditional_probability}
    P(A \mid B) = \frac{P(A \cap B)}{P(B)} \enspace \text{ where } P(B) > 0.
  \end{equation}
\end{defn}

\begin{note}
  When $B = S$, \cref{eq:conditional_probability} becomes
  \begin{equation*}
    P(A \mid S) = \frac{P(A \cap S)}{P(S)} = \frac{P(A)}{1} = P(A).
  \end{equation*}
  Also, we have, from \cref{eq:conditional_probability}, that
  \begin{equation*}
    P(A \cap B) = P(A \mid B) \cdot P(B).
  \end{equation*}
\end{note}

\begin{thm}[Law of Total Probability]
\index{Law of Total Probability}
\label{thm:law_of_total_probability}
  Let $S$ be a sample space. Let $\{ B_i \}_{i = 1}^{n}$ be a sequence of mutually exclusive events such that
  \begin{equation*}
    S = \bigcup_{i=1}^{n} B_i.
  \end{equation*}
  We say that the sequence $\{ B_i \}_{i = 1}^{n}$ is a \hldefn{partition} of $S$. Let $A \subseteq S$ be an event. Then
  \begin{equation*}
    P(A) = \sum_{i=1}^{n} P(A \mid B_i) \cdot P(B_i)
  \end{equation*}
\end{thm}

\begin{proof}
  Observe that
  \begin{align*}
    P(A) &= P(A \cap S) = P\left( A \cap \left\{ \bigcup_{i=1}^{n} B_i  \right\} \right) \\
         &= P\left( \bigcup_{i=1}^{n} \left\{ A \cap B_i \right\} \right) = \sum_{i=1}^{n} P(A \cap B_i) \\
         &= \sum_{i=1}^{n} P(A \mid B_i) P(B_i)
  \end{align*}
  where the second last step is by \cref{item:funddefn_probfn_item3}, and the last step is by \cref{defn:conditional_probability}.\qed
\end{proof}

Consequently, we have the following:

\begin{crly}[Bayes' Formula/Rule]
  \index{Bayes' Formula}\index{Bayes' Rule}
\label{crly:bayes_formula}
  Let $\{ B_i \}_{i = 1}^{n}$ be a partition of a sample space $S$. Then for any event $A$, we have
  \begin{equation*}
    P(B_j \mid A) = \frac{P(A \mid B_j) P(B_j)}{\sum_{i=1}^{n} P(A \mid B_i) \cdot P(B_i)}.
  \end{equation*}
\end{crly}

% section introductions (end)

\section{Random Variables}%
\label{sec:random_variables}
% section random_variables

\subsection{Discrete Random Variables}%
\label{sub:discrete_random_variables}
% subsection discrete_random_variables

\hlwarn{No formal definition of a discrete rv is given in class.}

A discrete rv $X$:
\begin{itemize}
  \item takes on either finite or countable number of possible values;
  \item has a \hldefn{probability mass function} (pmf) expressed as
    \begin{equation*}
      p(a) = P(X = a);
    \end{equation*}
  \item has a \hldefn{cumulative distribution function} (cdf) expressed as
    \begin{equation*}
      F(a) = P(X \leq a) = \sum_{x \leq a}^{p(x)} 
    \end{equation*}
\end{itemize}

\begin{note}
  If $X \in \{a_1, a_2, ...\}$ where $a_1 < a_2 < \hdots$ such that $p(a_i) > 0$ for all $i \in \mathbb{N}$, then
  \begin{align*}
    p(a_1) &= F(a_1) \text{ and } \\
    p(a_i) &= F(a_i) - F(a_{i - 1}) \text{ for } i = 2, 3, 4, ....
  \end{align*}
\end{note}

\newthought{The following} are some of the most common discrete distributions.

\paragraph{Binomial Distribution}\index{Binomial Distribution} For an rv $X$ that follows a Binomial Distribution, in which we denote as $X \sim \Bin(n, p)$, where $n \in \mathbb{N}$ and $p \in [0, 1]$, its pmf is
\begin{equation*}
  p(x) = \binom{n}{x} p^x (1 - p)^{n - x}.
\end{equation*}

\paragraph{Bernoulli Distribution}\index{Bernoulli Distribution} Following the above distribution where $n = 1$, we have that $X$ follows what is called a Bernoulli Distribution, denoted as $X \sim \Bernoulli(p)$.

\paragraph{Negative Binomial Distribution}\index{Negative Binomial Distribution} For an rv $X$ that follows a Negative Binomial Distribution, in which we denote as $X \sim \NB(k, p)$, where $k \in \mathbb{N}$ and $p \in [0, 1]$, its pmf is\marginnote{The Negative Binomial Distribution has a model that measures the probability that the $k$th success occurs.}
\begin{equation*}
  p(x) = \binom{x - 1}{k - 1} p^k (1 - p)^{x - k}
\end{equation*}

\paragraph{Geometric Distribution}\index{Geometric Distribution} Following the above distribution where $k = 1$, we have that $X$ folows what is called a Geometric Distribution, denoted as $X \sim \Geo(p)$.

\paragraph{Hypergeometric Distribution}\index{Hypergeometric Distribution} For an rv $X$ that follows a Hypergeometric Distribution, in which we denote as $X \sim \HG(N, r n)$, where $r, n \leq N \in \mathbb{N}$, its pmf is
\begin{equation*}
  p(x) = \frac{\binom{r}{x} \binom{N - r}{n - x}}{\binom{N}{n}}
\end{equation*}

\paragraph{Poisson Distribution}\index{Poisson Distribution} For an rv $X$ that follows a Poisson Distribution, in which we denote as $X \sim \Poi(\lambda)$, where $\lambda > 0$, its pmf is
\begin{equation*}
  p(x) = \frac{e^{- \lambda} \lambda^{x}}{x!}
\end{equation*}

% subsection discrete_random_variables (end)

\subsection{Continuous Random Variables}%
\label{sub:continuous_random_variables}
% subsection continuous_random_variables

\hlwarn{No formal definition of a continuous rv is given in class.}

A continuous rv $X$:
\begin{itemize}
  \item takes on a continuum of possible values
  \item has a \hldefn{probability density function} (pdf) expressed as
    \begin{equation*}
      f(x) = \frac{d}{dx} F(x)
    \end{equation*}
    where $F(x)$ is:
  \item (has a) \hldefn{cumulative distribution function} (cdf) of
    \begin{equation*}
      F(x) = P(X \leq x) = \int_{-\infty}^{x} f(y) \dif{y}
    \end{equation*}
\end{itemize}

\begin{note}
  Note that our convention is that $P(X = x) = 0$ for a continuous rv $X$.
\end{note}

\newthought{The following} are some of the most common continuous distributions.

\paragraph{Uniform Distribution}\index{Uniform Distribution} For an rv $X$ that follows a Uniform Distribution, in which we denote as $X \sim \Unif(a, b)$, where $a, b \in \mathbb{R}$, its pdf is
\begin{equation*}
  f(x) = \frac{1}{b - a}.
\end{equation*}

\paragraph{Gamma Distribution}\index{Gamma Distribution} For an rv $X$ that follows a Gamma Distribution, in which we denote as $X \sim \Gam(n, \lambda)$, where $n \in \mathbb{N}$ and $\lambda > 0$, its pdf is
\begin{equation*}
  f(x) = \frac{\lambda^n x^{n - 1} e^{-\lambda x}}{( n - 1 )!}.
\end{equation*}

\paragraph{Exponential Distribution}\index{Exponential Distribution} Following the above distribution where $n = 1$, we have that $X$ follows what is called an Exponential Distribution, denoted as $X \sim \Exp(\lambda)$, where its pdf is
\begin{equation*}
  f(x) = \lambda e^{- \lambda x}.
\end{equation*}

% subsection continuous_random_variables (end)

% section random_variables (end)

\section{Moments}%
\label{sec:moments}
% section moments

\begin{defn}[Expectation]\index{Expectation}
  \label{defn:expectation}\marginnote{Note that this definition is actually the \href{https://en.wikipedia.org/wiki/Law_of_the_unconscious_statistician}{Law of the Unconscious Statistician}}
  Let $X$ be an rv. Given a function $g$ that is defined over $X$, the \hlnoteb{expectation} of $g(X)$ is given by
  \begin{equation*}
    E[ g(X) ] = \begin{cases}
      \sum_{x} g(x) p(x) & \text{ if } X \text{ is a discrete rv } \\
      \int_{x} g(x) f(x) & \text{ if } X \text{ is a continuous rv }
    \end{cases}.
  \end{equation*}
\end{defn}

Now if $g(X) = X^k$ for some $k \in \mathbb{N}$, we have the following notion:

\begin{defn}[Moment]\index{Moment}
\label{defn:moment}
  Let $X$ be an rv. The $k$th moment of $X$ is defined as $E[ X^k ]$.
\end{defn}

Another notion that is commonly introduced after expectation is the variance.

\begin{defn}[Variance]\index{Variance}
\label{defn:variance}
  Let $X$ be an rv. The \hlnoteb{variance} of $X$ is given by
  \begin{equation*}
    \Var(X) = E[ (X - E[X])^2 ] = E[ X^2 ] - ( E[X] )^2
  \end{equation*}
\end{defn}

In relation to the variance, we have the standard deviation.

\begin{defn}[Standard Deviation]\index{Standard Deviation}
\label{defn:standard_deviation}
  Let $X$ be an rv. The \hlnoteb{standard deviation} (sd) is given by
  \begin{equation*}
    \sd(X) = \sqrt{\Var(X)} = \sqrt{ E[X^2] - (E[X])^2 }.
  \end{equation*}
\end{defn}

We shall state the following properties without providing proof\sidenote{The proofs are very easy, but it serves as a strengthening exercise for the unfamiliar. Therefore,
\begin{ex}
  Proof both \cref{propo:linearity_of_the_expectation} and \cref{propo:linearity_of_the_variance}.
\end{ex}
}:

\begin{propo}[Linearity of the Expectation]
\label{propo:linearity_of_the_expectation}
  Let $X$ be an rv. Let $a, b \in \mathbb{R}$. We have that
  \begin{equation*}
    E[ aX + b ] = aE[x] + b
  \end{equation*}
\end{propo}

\begin{propo}[Linearity of the Variance]
\label{propo:linearity_of_the_variance}
  Let $X$ be an rv. Let $a, b \in \mathbb{R}$. We have that
  \begin{equation*}
    \Var(aX + b) = a^2 \Var(X).
  \end{equation*}
\end{propo}

Referring back to \cref{defn:expectation}, if $g(X) = e^{tX}$, we have ourselves, what is called, the moment generating function.

\begin{defn}[Moment Generating Function]\index{Moment Generating Function}
\label{defn:moment_generating_function}
  Let $X$ be an rv. The \hlnoteb{moment generating function} (mgf) of $X$ is given by
  \begin{equation*}
    \phi_X(t) = E\left[ e^{tX} \right].
  \end{equation*}
\end{defn}

\begin{note}
  \begin{enumerate}
    \item Observe that $\phi_X(0) = E\left[ e^0 \right] = 1$.
    \item The reason such an expression is called a moment generating function is as follows: observe that
      \begin{align*}
        \phi_X(t) &= E\left[ e^{tX} \right] = E\left[ \sum_{i=0}^{\infty} \frac{(tX)^i}{i!} \right] \\
                  &= E\left[ 1 + \frac{tX}{1!} + \frac{(tX)^2}{2!} + \hdots + \frac{(tX)^n}{n!} + \hdots \right] \\
                  &= \frac{t^0}{0!} E[1] + \frac{t}{1!} E[X] + \frac{t^2}{2!} E\left[X^2\right] + \hdots + \frac{t^n}{n!} E\left[X^n\right] + \hdots
      \end{align*}
      by \cref{propo:linearity_of_the_expectation}. If we take the $k$th derivative wrt $t$ and set $t = 0$, we will obtain the $k$th moment of $X$. In other words,
      \begin{equation*}
        E[X^k] = \phi_X^{(n)}(0) = \frac{d}{dt} \phi_X (t) \at{t = 0}{}.
      \end{equation*}
  \end{enumerate}
\end{note}

% section moments (end)

% chapter elementary_probability_review (end)

\appendix

\backmatter

\pagestyle{plain}

\nobibliography*
\bibliography{references}

\printindex

\end{document}

