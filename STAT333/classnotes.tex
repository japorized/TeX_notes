\documentclass[notoc,notitlepage]{tufte-book}
% \nonstopmode % uncomment to enable nonstopmode

\usepackage{classnotetitle}

\title{STAT333 --- Applied Probability}
\author{Johnson Ng}
\subtitle{Classnotes for Winter 2017}
\credentials{BMath (Hons), Pure Mathematics major, Actuarial Science Minor}
\institution{University of Waterloo}

\setcounter{secnumdepth}{3}
\setcounter{tocdepth}{3}

\renewcommand{\baselinestretch}{1.1}
\usepackage{geometry}
\geometry{letterpaper}
\usepackage[parfill]{parskip}
\usepackage{graphicx}

% Essential Packages
\usepackage{makeidx}
\makeindex
\usepackage{enumitem}
\usepackage[T1]{fontenc}
\usepackage{natbib}
\bibliographystyle{apalike}
\usepackage{ragged2e}
\usepackage{etoolbox}
\usepackage{amssymb}
\usepackage{fontawesome}
\usepackage{amsmath}
\usepackage{mathrsfs}
\usepackage{mathtools}
\usepackage{xparse}
\usepackage{tkz-euclide}
\usetkzobj{all}
\usepackage[utf8]{inputenc}
\usepackage{csquotes}
\usepackage[english]{babel}
\usepackage{marvosym}
\usepackage{pgf,tikz}
\usepackage{pgfplots}
\usepackage{fancyhdr}
\usepackage{array}
\usepackage{faktor}
\usepackage{float}
\usepackage{xcolor}
\usepackage{centernot}
\usepackage{silence}
  \WarningFilter*{latex}{Marginpar on page \thepage\space moved}
\usepackage{tcolorbox}
\tcbuselibrary{skins,breakable}
\usepackage{longtable}
\usepackage[amsmath,hyperref]{ntheorem}
\usepackage{hyperref}
\usepackage[noabbrev,capitalize,nameinlink]{cleveref}

% xcolor (scheme: base16 eighties)
\definecolor{base16-eighties-dark}{HTML}{2D2D2D}
\definecolor{base16-eighties-light}{HTML}{D3D0C8}
\definecolor{base16-eighties-magenta}{HTML}{CD98CD}
\definecolor{base16-eighties-red}{HTML}{F47678}
\definecolor{base16-eighties-yellow}{HTML}{E2B552}
\definecolor{base16-eighties-green}{HTML}{98CD97}
\definecolor{base16-eighties-lightblue}{HTML}{61CCCD}
\definecolor{base16-eighties-blue}{HTML}{6498CE}
\definecolor{base16-eighties-brown}{HTML}{D47B4E}
\definecolor{base16-eighties-gray}{HTML}{747369}

% hyperref Package Settings
\hypersetup{
    bookmarks=true,         % show bookmarks bar?
    unicode=true,          % non-Latin characters in Acrobat’s bookmarks
    pdftoolbar=false,        % show Acrobat’s toolbar?
    pdfmenubar=false,        % show Acrobat’s menu?
    pdffitwindow=true,     % window fit to page when opened
    colorlinks=true,
    allcolors=base16-eighties-magenta,
}

% tikz
\usepgfplotslibrary{polar}
\usepgflibrary{shapes.geometric}
\usetikzlibrary{angles,patterns,calc,decorations.markings}
\tikzset{midarrow/.style 2 args={
        decoration={markings,
            mark= at position #2 with {\arrow{#1}} ,
        },
        postaction={decorate}
    },
    midarrow/.default={latex}{0.5}
}
\def\centerarc[#1](#2)(#3:#4:#5)% Syntax: [draw options] (center) (initial angle:final angle:radius)
    { \draw[#1] ($(#2)+({#5*cos(#3)},{#5*sin(#3)})$) arc (#3:#4:#5); }

% enumitems
\newlist{inlinelist}{enumerate*}{1}
\setlist*[inlinelist,1]{%
  label=(\roman*),
}

% Theorem Style Customization
\setlength\theorempreskipamount{2ex}
\setlength\theorempostskipamount{3ex}

\makeatletter
\let\nobreakitem\item
\let\@nobreakitem\@item
\patchcmd{\nobreakitem}{\@item}{\@nobreakitem}{}{}
\patchcmd{\nobreakitem}{\@item}{\@nobreakitem}{}{}
\patchcmd{\@nobreakitem}{\@itempenalty}{\@M}{}{}
\patchcmd{\@xthm}{\ignorespaces}{\nobreak\ignorespaces}{}{}
\patchcmd{\@ythm}{\ignorespaces}{\nobreak\ignorespaces}{}{}

\renewtheoremstyle{break}%
  {\item{\theorem@headerfont
          ##1\ ##2\theorem@separator}\hskip\labelsep\relax\nobreakitem}%
  {\item{\theorem@headerfont
          ##1\ ##2\ (##3)\theorem@separator}\hskip\labelsep\relax\nobreakitem}
\makeatother

% ntheorem + framed
\makeatletter

% ntheorem Declarations
\theorempreskip{10pt}
\theorempostskip{5pt}
\theoremstyle{break}

\newtheorem*{solution}{\faPencil $\enspace$ Solution}
\newtheorem*{remark}{Remark}
\newtheorem{eg}{Example}[section]
\newtheorem{ex}{Exercise}[section]

    % definition env
\theoremprework{\textcolor{base16-eighties-blue}{\hrule height 2pt}}
\theoremheaderfont{\color{base16-eighties-blue}\normalfont\bfseries}
\theorempostwork{\textcolor{base16-eighties-blue}{\hrule height 2pt}}
\theoremindent10pt
\newtheorem{defn}{\faBook \enspace Definition}

    % definition env no num
\theoremprework{\textcolor{base16-eighties-blue}{\hrule height 2pt}}
\theoremheaderfont{\color{base16-eighties-blue}\normalfont\bfseries}
\theorempostwork{\textcolor{base16-eighties-blue}{\hrule height 2pt}}
\theoremindent10pt
\newtheorem*{defnnonum}{\faBook \enspace Definition}

    % theorem envs
\theoremprework{\textcolor{base16-eighties-magenta}{\hrule height 2pt}}
\theoremheaderfont{\color{base16-eighties-magenta}\normalfont\bfseries}
\theorempostwork{\textcolor{base16-eighties-magenta}{\hrule height 2pt}}
\theoremindent10pt
\newtheorem{thm}{\faCoffee \enspace Theorem}

\theoremprework{\textcolor{base16-eighties-magenta}{\hrule height 2pt}}
\theorempostwork{\textcolor{base16-eighties-magenta}{\hrule height 2pt}}
\theoremindent10pt
\newtheorem{propo}[thm]{\faTint \enspace Proposition}

\theoremprework{\textcolor{base16-eighties-magenta}{\hrule height 2pt}}
\theorempostwork{\textcolor{base16-eighties-magenta}{\hrule height 2pt}}
\theoremindent10pt
\newtheorem{crly}[thm]{\faSpaceShuttle \enspace Corollary}

\theoremprework{\textcolor{base16-eighties-magenta}{\hrule height 2pt}}
\theorempostwork{\textcolor{base16-eighties-magenta}{\hrule height 2pt}}
\theoremindent10pt
\newtheorem{lemma}[thm]{\faTree \enspace Lemma}

\theoremprework{\textcolor{base16-eighties-magenta}{\hrule height 2pt}}
\theorempostwork{\textcolor{base16-eighties-magenta}{\hrule height 2pt}}
\theoremindent10pt
\newtheorem{axiom}[thm]{\faShield \enspace Axiom}

    % theorem envs without counter
\theoremprework{\textcolor{base16-eighties-magenta}{\hrule height 2pt}}
\theoremheaderfont{\color{base16-eighties-magenta}\normalfont\bfseries}
\theorempostwork{\textcolor{base16-eighties-magenta}{\hrule height 2pt}}
\theoremindent10pt
\newtheorem*{thmnonum}{\faCoffee \enspace Theorem}

\theoremprework{\textcolor{base16-eighties-magenta}{\hrule height 2pt}}
\theorempostwork{\textcolor{base16-eighties-magenta}{\hrule height 2pt}}
\theoremindent10pt
\newtheorem*{propononum}{\faTint \enspace Proposition}

\theoremprework{\textcolor{base16-eighties-magenta}{\hrule height 2pt}}
\theorempostwork{\textcolor{base16-eighties-magenta}{\hrule height 2pt}}
\theoremindent10pt
\newtheorem*{crlynonum}{\faSpaceShuttle \enspace Corollary}

\theoremprework{\textcolor{base16-eighties-magenta}{\hrule height 2pt}}
\theorempostwork{\textcolor{base16-eighties-magenta}{\hrule height 2pt}}
\theoremindent10pt
\newtheorem*{lemmanonum}{\faTree \enspace Lemma}

\theoremprework{\textcolor{base16-eighties-magenta}{\hrule height 2pt}}
\theorempostwork{\textcolor{base16-eighties-magenta}{\hrule height 2pt}}
\theoremindent10pt
\newtheorem*{axiomnonum}{\faShield \enspace Axiom}

    % proof env
\theoremprework{\textcolor{base16-eighties-brown}{\hrule height 2pt}}
\theoremheaderfont{\color{base16-eighties-brown}\normalfont\bfseries}
\theorempostwork{\textcolor{base16-eighties-brown}{\hrule height 2pt}}
\newtheorem*{proof}{\faPencil \enspace Proof}

    % note and notation env
\theoremprework{\textcolor{base16-eighties-yellow}{\hrule height 2pt}}
\theoremheaderfont{\color{base16-eighties-yellow}\normalfont\bfseries}
\theorempostwork{\textcolor{base16-eighties-yellow}{\hrule height 2pt}}
\newtheorem*{note}{\faQuoteLeft \enspace Note}

\theoremprework{\textcolor{base16-eighties-yellow}{\hrule height 2pt}}
\theorempostwork{\textcolor{base16-eighties-yellow}{\hrule height 2pt}}
\newtheorem*{notation}{\faPaw \enspace Notation}

    % warning env
\theoremprework{\textcolor{base16-eighties-red}{\hrule height 2pt}}
\theoremheaderfont{\color{base16-eighties-red}\normalfont\bfseries}
\theorempostwork{\textcolor{base16-eighties-red}{\hrule height 2pt}}
\theoremindent10pt
\newtheorem*{warning}{\faBug \enspace Warning}

% more environments
\newtcolorbox{redquote}{
  blanker,enhanced,breakable,standard jigsaw,
  opacityback=0,
  coltext=base16-eighties-light,
  left=5mm,right=5mm,top=2mm,bottom=2mm,
  colframe=base16-eighties-red,
  boxrule=0pt,leftrule=3pt,
  fontupper=\itshape
}
\newtcolorbox{bluequote}{
  blanker,enhanced,breakable,standard jigsaw,
  opacityback=0,
  coltext=base16-eighties-light,
  left=5mm,right=5mm,top=2mm,bottom=2mm,
  colframe=base16-eighties-blue,
  boxrule=0pt,leftrule=3pt,
  fontupper=\itshape
}
\newtcolorbox{greenquote}{
  blanker,enhanced,breakable,standard jigsaw,
  opacityback=0,
  coltext=base16-eighties-light,
  left=5mm,right=5mm,top=2mm,bottom=2mm,
  colframe=base16-eighties-green,
  boxrule=0pt,leftrule=3pt,
  fontupper=\itshape
}
\newtcolorbox{yellowquote}{
  blanker,enhanced,breakable,standard jigsaw,
  opacityback=0,
  coltext=base16-eighties-light,
  left=5mm,right=5mm,top=2mm,bottom=2mm,
  colframe=base16-eighties-yellow,
  boxrule=0pt,leftrule=3pt,
  fontupper=\itshape
}
\newtcolorbox{magentaquote}{
  blanker,enhanced,breakable,standard jigsaw,
  opacityback=0,
  coltext=base16-eighties-light,
  left=5mm,right=5mm,top=2mm,bottom=2mm,
  colframe=base16-eighties-magenta,
  boxrule=0pt,leftrule=3pt,
  fontupper=\itshape
}

% ntheorem listtheorem style
\makeatother
\newlength\widesttheorem
\AtBeginDocument{
  \settowidth{\widesttheorem}{Proposition A.1.1.1\quad}
}

\makeatletter
\def\thm@@thmline@name#1#2#3#4{%
        \@dottedtocline{-2}{0em}{2.3em}%
                   {\makebox[\widesttheorem][l]{#1 \protect\numberline{#2}}#3}%
                   {#4}}
\@ifpackageloaded{hyperref}{
\def\thm@@thmline@name#1#2#3#4#5{%
    \ifx\#5\%
        \@dottedtocline{-2}{0em}{2.3em}%
            {\makebox[\widesttheorem][l]{#1 \protect\numberline{#2}}#3}%
            {#4}
    \else
        \ifHy@linktocpage\relax\relax
            \@dottedtocline{-2}{0em}{2.3em}%
                {\makebox[\widesttheorem][l]{#1 \protect\numberline{#2}}#3}%
                {\hyper@linkstart{link}{#5}{#4}\hyper@linkend}%
        \else
            \@dottedtocline{-2}{0em}{2.3em}%
                {\hyper@linkstart{link}{#5}%
                  {\makebox[\widesttheorem][l]{#1 \protect\numberline{#2}}#3}\hyper@linkend}%
                    {#4}%
        \fi
    \fi}
}

\makeatletter
\def\thm@@thmline@noname#1#2#3#4{%
        \@dottedtocline{-2}{0em}{5em}%
                   {{\protect\numberline{#2}}#3}%
                   {#4}}
\@ifpackageloaded{hyperref}{
\def\thm@@thmline@noname#1#2#3#4#5{%
    \ifx\#5\%
        \@dottedtocline{-2}{0em}{5em}%
            {{\protect\numberline{#2}}#3}%
            {#4}
    \else
        \ifHy@linktocpage\relax\relax
            \@dottedtocline{-2}{0em}{5em}%
                {{\protect\numberline{#2}}#3}%
                {\hyper@linkstart{link}{#5}{#4}\hyper@linkend}%
        \else
            \@dottedtocline{-2}{0em}{5em}%
                {\hyper@linkstart{link}{#5}%
                  {{\protect\numberline{#2}}#3}\hyper@linkend}%
                    {#4}%
        \fi
    \fi}
}

\theoremlisttype{allname}

\AtBeginDocument{\renewcommand\contentsname{Table of Contents}}

% Heading formattings
% chapter format
\titleformat{\chapter}%
  {\huge\rmfamily\itshape\color{base16-eighties-magenta}}% format applied to label+text
  {\llap{\colorbox{base16-eighties-magenta}{\parbox{1.5cm}{\hfill\itshape\huge\textcolor{base16-eighties-dark}{\thechapter}}}}}% label
  {5pt}% horizontal separation between label and title body
  {}% before the title body
  []% after the title body

% section format
\titleformat{\section}%
  {\normalfont\Large\rmfamily\itshape\color{base16-eighties-blue}}% format applied to label+text
  {\llap{\colorbox{base16-eighties-blue}{\parbox{1.5cm}{\hfill\itshape\textcolor{base16-eighties-dark}{\thesection}}}}}% label
  {5pt}% horizontal separation between label and title body
  {}% before the title body
  []% after the title body

% subsection format
\titleformat{\subsection}%
  {\normalfont\large\itshape\color{base16-eighties-green}}% format applied to label+text
  {\llap{\colorbox{base16-eighties-green}{\parbox{1.5cm}{\hfill\textcolor{base16-eighties-dark}{\thesubsection}}}}}% label
  {1em}% horizontal separation between label and title body
  {}% before the title body
  []% after the title body

% Sidenote enhancements
\def\mathmarginnote#1{%
  \tag*{\rlap{\hspace\marginparsep\smash{\parbox[t]{\marginparwidth}{%
  \footnotesize#1}}}}
}

% Custom table columning
\newcolumntype{L}[1]{>{\raggedright\let\newline\\\arraybackslash\hspace{0pt}}m{#1}}
\newcolumntype{C}[1]{>{\centering\let\newline\\\arraybackslash\hspace{0pt}}m{#1}}
\newcolumntype{R}[1]{>{\raggedleft\let\newline\\\arraybackslash\hspace{0pt}}m{#1}}

% Custom math operator
% \DeclareMathOperator{\rem}{rem}
\DeclareMathOperator*{\argmax}{arg\,max}
\DeclareMathOperator*{\argmin}{arg\,min}
\DeclareMathOperator{\re}{Re}
\DeclareMathOperator{\im}{Im}
\DeclareMathOperator{\caparg}{Arg}
\DeclareMathOperator{\Ind}{Ind}
\DeclareMathOperator{\Res}{Res}

% Graph styles
\pgfplotsset{compat=1.15}
\usepgfplotslibrary{fillbetween}
\pgfplotsset{four quads/.append style={axis x line=middle, axis y line=
middle, xlabel={$x$}, ylabel={$y$}, axis equal }}
\pgfplotsset{four quad complex/.append style={axis x line=middle, axis y line=
middle, xlabel={$\re$}, ylabel={$\im$}, axis equal }}

% Shortcuts
\newcommand{\floor}[1]{\lfloor #1 \rfloor}      % simplifying the writing of a floor function
\newcommand{\ceiling}[1]{\lceil #1 \rceil}      % simplifying the writing of a ceiling function
\newcommand{\dotp}{\, \cdotp}			        % dot product to distinguish from \cdot
\newcommand{\qed}{\hfill\ensuremath{\square}}   % Q.E.D sign
\newcommand{\abs}[1]{\left|#1\right|}						% absolute value
\newcommand{\lra}[1]{\langle \; #1 \; \rangle}
\newcommand{\at}[2]{\Big|_{#1}^{#2}}
\newcommand{\Arg}[1]{\caparg #1}
\renewcommand{\bar}[1]{\mkern 1.5mu \overline{\mkern -1.5mu #1 \mkern -1.5mu} \mkern 1.5mu}
\newcommand{\quotient}[2]{\faktor{#1}{#2}}
\newcommand{\cyclic}[1]{\left\langle #1 \right\rangle}
	% highlighting shortcuts
\newcommand{\hlimpo}[1]{\textcolor{base16-eighties-red}{\textbf{#1}}}
\newcommand{\hlwarn}[1]{\textcolor{base16-eighties-yellow}{\textbf{#1}}}
\newcommand{\hldefn}[1]{\textcolor{base16-eighties-blue}{\index{#1}\textbf{#1}}}
\newcommand{\hlnotea}[1]{\textcolor{base16-eighties-green}{\textbf{#1}}}
\newcommand{\hlnoteb}[1]{\textcolor{base16-eighties-lightblue}{\textbf{#1}}}
\newcommand{\hlnotec}[1]{\textcolor{base16-eighties-brown}{\textbf{#1}}}
\newcommand{\WTP}{\textcolor{base16-eighties-brown}{WTP} }
\newcommand{\WTS}{\textcolor{base16-eighties-brown}{WTS} }
\newcommand{\ind}[2]{\Ind_{#2}\left( #1 \right)}
\newcommand{\notimply}{\centernot\implies}
\newcommand{\res}[2]{\underset{#2}{\Res} #1 }
\newcommand{\tworow}[3]{\begin{tabular}{@{}#1@{}} #2 \\ #3 \end{tabular}}
\renewcommand{\epsilon}{\varepsilon}
\newcommand{\lrarrow}{\leftrightarrow}
\newcommand{\larrow}{\leftarrow}
\newcommand{\rarrow}{\rightarrow}
\renewcommand{\atop}[2]{\genfrac{}{}{0pt}{}{#1}{#2}}
\newcommand*\dif{\mathop{}\!d}

  % inspiration from: https://tex.stackexchange.com/questions/8720/overbrace-underbrace-but-with-an-arrow-instead#37758
\newcommand{\overarrow}[2]{
  \overset{\makebox[0pt]{\begin{tabular}{@{}c@{}}#2\\[0pt]\ensuremath{\uparrow}\end{tabular}}}{#1}
}
\newcommand{\underarrow}[2]{
  \underset{\makebox[0pt]{\begin{tabular}{@{}c@{}}\downarrow\\[0pt]\ensuremath{#2}\end{tabular}}}{#1}
}

% Document header formatting
\renewcommand{\chaptermark}[1]{\markboth{#1}{}}
\renewcommand{\sectionmark}[1]{\markright{#1}}
\makeatletter
\pagestyle{fancy}
\fancyhead{}
\fancyhead[RO]{\textsl{\@title} \enspace \thepage}
\fancyhead[LE]{\thepage \enspace \textsl{\leftmark \enspace - \enspace \rightmark}}
\makeatother

% Comment the two lines below if you want to print the document
\pagecolor{base16-eighties-dark}
\color{base16-eighties-light}

\DeclareMathOperator{\Bernoulli}{Bernoulli }
\DeclareMathOperator{\Bin}{Bin }
\DeclareMathOperator{\Geo}{Geo }
\DeclareMathOperator{\Poi}{Poi }
\DeclareMathOperator{\NB}{NB }
\DeclareMathOperator{\Exp}{Exp }
\DeclareMathOperator{\Unif}{Unif }
\DeclareMathOperator{\Nor}{N }
\DeclareMathOperator{\Gau}{G }
\DeclareMathOperator{\Gam}{Gam }
\DeclareMathOperator{\BetaDist}{Beta }
\DeclareMathOperator{\Mult}{Mult }
\DeclareMathOperator{\BVN}{BVN }
\DeclareMathOperator{\LogN}{LogN }
\DeclareMathOperator{\Wei}{Wei }
\DeclareMathOperator{\Pareto}{Pareto }
\DeclareMathOperator{\Erlang}{Erlang }
\DeclareMathOperator{\Dom}{Dom }
\DeclareMathOperator{\Var}{Var }
\DeclareMathOperator{\Cov}{Cov }
\DeclareMathOperator{\Corr}{Corr }
\DeclareMathOperator{\supp}{supp }
\DeclareMathOperator{\sd}{sd }
\DeclareMathOperator{\HG}{HG }
\DeclareMathOperator{\mgf}{mgf}
\DeclareMathOperator{\bias}{bias}
\DeclareMathOperator{\MSE}{MSE}


\begin{document}
\hypersetup{pageanchor=false}
\maketitle
\hypersetup{pageanchor=true}
\begin{fullwidth}
\tableofcontents
\end{fullwidth}

\newpage
\begin{fullwidth}
  \renewcommand{\listtheoremname}{\faBook\ \slshape List of Definitions}
  \listoftheorems[ignoreall,show={defn}]
  \addcontentsline{toc}{chapter}{List of Definitions}
\end{fullwidth}

\newpage 
\begin{fullwidth}
  \renewcommand{\listtheoremname}{\faCoffee\ \slshape List of Theorems}
  \listoftheorems[ignoreall,
    show={axiom,lemma,thm,crly,propo,marginthm,marginpropo,marginlemma,marginaxiom,margincrly}
  ]
  \addcontentsline{toc}{chapter}{List of Theorems}
\end{fullwidth}


\chapter*{Foreword}%
\label{chp:foreword}
% chapter foreword

I am transcribing this set of notes from my handwritten ones in Winter 2017,
back at a time which I have yet to organize my notes by lecture. However, I will
try my best to organize them by chapters and topics as presented in class.

I will try to be as rigourous as possible while transcribing my notes. However,
given the nature of the course and the presentation, this will not always be
possible, and I am mostly keeping these notes for ``legacy purposes'', and so I
will not put too much effort into making the notes as complete as my newer
ones.

For this course, you are expected to have basic knowledge of probability in
order to be able to understand the material. You may want to have my
\href{https://tex.japorized.ink/STAT330S18/classnotes-white.pdf}{STAT330} notes
ready and/or reviewed.

% chapter foreword (end)

\chapter{Elementary Probability Review}%
\label{chp:elementary_probability_review}
% chapter elementary_probability_review

\section{Introductions}%
\label{sec:introductions}
% section introductions

\begin{defn}[Fundamental Definition of a Probability Function]
\label{defn:fundamental_definition_of_a_probability_function}
  For each event $A$ of a sample space $S$, $P(A)$ is defined as the ``\hlnoteb{probability of the event $A$}'', satisfying these 3 conditions:
  \begin{enumerate}
    \item $0 \leq P(A) \leq 1$
    \item $P(S) = 1$ \sidenote{This can also be stated as $P(\emptyset) = 0$, where $\emptyset$ is the null event.} \label{item:funddefn_probfn_item2}
    \item $P\left( \bigcup\limits_{i = 1}^{n} A_i \right) = \sum\limits_{i=1}^{n} P(A_i)$, where $A_i \cap A_j = A_i A_j = \emptyset$ for all $i \neq j$ \sidenote{We can also say that the sequence $\{A_i\}_{i = 1}^{n}$ has mutually exclusive elements.} \label{item:funddefn_probfn_item3}
  \end{enumerate}
\end{defn}

\begin{note}
  By \cref{item:funddefn_probfn_item2} and \cref{item:funddefn_probfn_item3}, we
  have
  \begin{equation*}
    1 = P(S) = P(A \cup A^C) = P(A) + P(A^C)
  \end{equation*}
  which implies that
  \begin{equation*}
    P(A^C) = 1 - P(A).
  \end{equation*}
\end{note}

\begin{defn}[Conditional Probability]\index{Conditional Probability}\label{defn:conditional_probability}
  Given events $A$ and $B$ in a sample space $S$, the \hlnoteb{conditional
  probability of $A$ given $B$} is given by
  \begin{equation}\label{eq:conditional_probability}
    P(A \mid B) = \frac{P(A \cap B)}{P(B)} \enspace \text{ where } P(B) > 0.
  \end{equation}
\end{defn}

\begin{note}
  When $B = S$, \cref{eq:conditional_probability} becomes
  \begin{equation*}
    P(A \mid S) = \frac{P(A \cap S)}{P(S)} = \frac{P(A)}{1} = P(A).
  \end{equation*}
  Also, we have, from \cref{eq:conditional_probability}, that
  \begin{equation*}
    P(A \cap B) = P(A \mid B) \cdot P(B).
  \end{equation*}
\end{note}

\begin{thm}[Law of Total Probability]\index{Law of Total Probability}\label{thm:law_of_total_probability}
  Let $S$ be a sample space. Let $\{ B_i \}_{i = 1}^{n}$ be a sequence of
  mutually exclusive events such that
  \begin{equation*}
    S = \bigcup_{i=1}^{n} B_i.
  \end{equation*}
  We say that the sequence $\{ B_i \}_{i = 1}^{n}$ is a \hldefn{partition} of $S$. Let $A \subseteq S$ be an event. Then
  \begin{equation*}
    P(A) = \sum_{i=1}^{n} P(A \mid B_i) \cdot P(B_i)
  \end{equation*}
\end{thm}

\begin{proof}
  Observe that
  \begin{align*}
    P(A) &= P(A \cap S) = P\left( A \cap \left\{ \bigcup_{i=1}^{n} B_i  \right\} \right) \\
         &= P\left( \bigcup_{i=1}^{n} \left\{ A \cap B_i \right\} \right) = \sum_{i=1}^{n} P(A \cap B_i) \\
         &= \sum_{i=1}^{n} P(A \mid B_i) P(B_i)
  \end{align*}
  where the second last step is by \cref{item:funddefn_probfn_item3}, and the
  last step is by \cref{defn:conditional_probability}.
\end{proof}

Consequently, we have the following:

\begin{crly}[Bayes' Formula/Rule]
  \index{Bayes' Formula}\index{Bayes' Rule}
\label{crly:bayes_formula}
  Let $\{ B_i \}_{i = 1}^{n}$ be a partition of a sample space $S$. Then for any
  event $A$, we have
  \begin{equation*}
    P(B_j \mid A) = \frac{P(A \mid B_j) P(B_j)}{\sum_{i=1}^{n} P(A \mid B_i) \cdot P(B_i)}.
  \end{equation*}
\end{crly}

% section introductions (end)

\section{Random Variables}%
\label{sec:random_variables}
% section random_variables

\subsection{Discrete Random Variables}%
\label{sub:discrete_random_variables}
% subsection discrete_random_variables

\hlwarn{No formal definition of a discrete rv is given in class.}

A discrete rv $X$:
\begin{itemize}
  \item takes on either finite or countable number of possible values;
  \item has a \hldefn{probability mass function} (pmf) expressed as
    \begin{equation*}
      p(a) = P(X = a);
    \end{equation*}
  \item has a \hldefn{cumulative distribution function} (cdf) expressed as
    \begin{equation*}
      F(a) = P(X \leq a) = \sum_{x \leq a}^{p(x)} 
    \end{equation*}
\end{itemize}

\begin{note}
  If $X \in \{a_1, a_2, ...\}$ where $a_1 < a_2 < \hdots$ such that $p(a_i) > 0$ for all $i \in \mathbb{N}$, then
  \begin{align*}
    p(a_1) &= F(a_1) \text{ and } \\
    p(a_i) &= F(a_i) - F(a_{i - 1}) \text{ for } i = 2, 3, 4, ....
  \end{align*}
\end{note}

\newthought{The following} are some of the most common discrete distributions.

\paragraph{Binomial Distribution}\index{Binomial Distribution} For an rv $X$
that follows a Binomial Distribution, in which we denote as $X \sim \Bin(n, p)$,
where $n \in \mathbb{N}$ and $p \in [0, 1]$, its pmf is
\begin{equation*}
  p(x) = \binom{n}{x} p^x (1 - p)^{n - x}.
\end{equation*}

\paragraph{Bernoulli Distribution}\index{Bernoulli Distribution} Following the
above distribution where $n = 1$, we have that $X$ follows what is called a
Bernoulli Distribution, denoted as $X \sim \Bernoulli(p)$.

\paragraph{Negative Binomial Distribution}\index{Negative Binomial Distribution}
For an rv $X$ that follows a Negative Binomial Distribution, in which we denote
as $X \sim \NB(k, p)$, where $k \in \mathbb{N}$ and $p \in [0, 1]$, its pmf
is\marginnote{The Negative Binomial Distribution has a model that measures the
probability that the $k$th success occurs.}
\begin{equation*}
  p(x) = \binom{x - 1}{k - 1} p^k (1 - p)^{x - k}
\end{equation*}

\paragraph{Geometric Distribution}\index{Geometric Distribution} Following the
above distribution where $k = 1$, we have that $X$ folows what is called a
Geometric Distribution, denoted as $X \sim \Geo(p)$.

\paragraph{Hypergeometric Distribution}\index{Hypergeometric Distribution} For
an rv $X$ that follows a Hypergeometric Distribution, in which we denote as $X
\sim \HG(N, r n)$, where $r, n \leq N \in \mathbb{N}$, its pmf is
\begin{equation*}
  p(x) = \frac{\binom{r}{x} \binom{N - r}{n - x}}{\binom{N}{n}}
\end{equation*}

\paragraph{Poisson Distribution}\index{Poisson Distribution} For an rv $X$ that
follows a Poisson Distribution, in which we denote as $X \sim \Poi(\lambda)$,
where $\lambda > 0$, its pmf is
\begin{equation*}
  p(x) = \frac{e^{- \lambda} \lambda^{x}}{x!}
\end{equation*}

% subsection discrete_random_variables (end)

\subsection{Continuous Random Variables}%
\label{sub:continuous_random_variables}
% subsection continuous_random_variables

\hlwarn{No formal definition of a continuous rv is given in class.}

A continuous rv $X$:
\begin{itemize}
  \item takes on a continuum of possible values
  \item has a \hldefn{probability density function} (pdf) expressed as
    \begin{equation*}
      f(x) = \frac{d}{dx} F(x)
    \end{equation*}
    where $F(x)$ is:
  \item (has a) \hldefn{cumulative distribution function} (cdf) of
    \begin{equation*}
      F(x) = P(X \leq x) = \int_{-\infty}^{x} f(y) \dif{y}
    \end{equation*}
\end{itemize}

\begin{note}
  Note that our convention is that $P(X = x) = 0$ for a continuous rv $X$.
\end{note}

\newthought{The following} are some of the most common continuous
distributions.

\paragraph{Uniform Distribution}\index{Uniform Distribution} For an rv $X$ that
follows a Uniform Distribution, in which we denote as $X \sim \Unif(a, b)$,
where $a, b \in \mathbb{R}$, its pdf is
\begin{equation*}
  f(x) = \frac{1}{b - a}.
\end{equation*}

\paragraph{Gamma Distribution}\index{Gamma Distribution} For an rv $X$ that
follows a Gamma Distribution, in which we denote as $X \sim \Gam(n, \lambda)$,
where $n \in \mathbb{N}$ and $\lambda > 0$, its pdf is
\begin{equation*}
  f(x) = \frac{\lambda^n x^{n - 1} e^{-\lambda x}}{( n - 1 )!}.
\end{equation*}

\paragraph{Exponential Distribution}\index{Exponential Distribution} Following
the above distribution where $n = 1$, we have that $X$ follows what is called an
Exponential Distribution, denoted as $X \sim \Exp(\lambda)$, where its pdf is
\begin{equation*}
  f(x) = \lambda e^{- \lambda x}.
\end{equation*}

% subsection continuous_random_variables (end)

% section random_variables (end)

\section{Moments}%
\label{sec:moments}
% section moments

\begin{defn}[Expectation]\index{Expectation}
  \label{defn:expectation}\marginnote{Note that this definition is actually the
  \href{https://en.wikipedia.org/wiki/Law_of_the_unconscious_statistician}{Law
  of the Unconscious Statistician}}
  Let $X$ be an rv. Given a function $g$ that is defined over $X$, the
  \hlnoteb{expectation} of $g(X)$ is given by
  \begin{equation*}
    E[ g(X) ] = \begin{cases}
      \sum_{x} g(x) p(x) & \text{ if } X \text{ is a discrete rv } \\
      \int_{x} g(x) f(x) & \text{ if } X \text{ is a continuous rv }
    \end{cases}.
  \end{equation*}
\end{defn}

Now if $g(X) = X^k$ for some $k \in \mathbb{N}$, we have the following notion:

\begin{defn}[Moment]\index{Moment}
\label{defn:moment}
  Let $X$ be an rv. The $k$th moment of $X$ is defined as $E[ X^k ]$.
\end{defn}

Another notion that is commonly introduced after expectation is the variance.

\begin{defn}[Variance]\index{Variance}
\label{defn:variance}
  Let $X$ be an rv. The \hlnoteb{variance} of $X$ is given by
  \begin{equation*}
    \Var(X) = E[ (X - E[X])^2 ] = E[ X^2 ] - ( E[X] )^2
  \end{equation*}
\end{defn}

In relation to the variance, we have the standard deviation.

\begin{defn}[Standard Deviation]\index{Standard Deviation}
\label{defn:standard_deviation}
  Let $X$ be an rv. The \hlnoteb{standard deviation} (sd) is given by
  \begin{equation*}
    \sd(X) = \sqrt{\Var(X)} = \sqrt{ E[X^2] - (E[X])^2 }.
  \end{equation*}
\end{defn}

We shall state the following properties without providing proof\sidenote{The
proofs are very easy, but it serves as a strengthening exercise for the
unfamiliar. Therefore,
\begin{ex}
  Proof both \cref{propo:linearity_of_the_expectation} and \cref{propo:linearity_of_the_variance}.
\end{ex}
}:

\begin{propo}[Linearity of the Expectation]
\label{propo:linearity_of_the_expectation}
  Let $X$ be an rv. Let $a, b \in \mathbb{R}$. We have that
  \begin{equation*}
    E[ aX + b ] = aE[x] + b
  \end{equation*}
\end{propo}

\begin{propo}[Linearity of the Variance]
\label{propo:linearity_of_the_variance}
  Let $X$ be an rv. Let $a, b \in \mathbb{R}$. We have that
  \begin{equation*}
    \Var(aX + b) = a^2 \Var(X).
  \end{equation*}
\end{propo}

Referring back to \cref{defn:expectation}, if $g(X) = e^{tX}$, we have
ourselves, what is called, the moment generating function.

\begin{defn}[Moment Generating Function]\index{Moment Generating Function}
\label{defn:moment_generating_function}
  Let $X$ be an rv. The \hlnoteb{moment generating function} (mgf) of $X$ is given by
  \begin{equation*}
    \phi_X(t) = E\left[ e^{tX} \right].
  \end{equation*}
\end{defn}

\begin{note}
  \begin{enumerate}
    \item Observe that $\phi_X(0) = E\left[ e^0 \right] = 1$.
    \item The reason such an expression is called a moment generating function
      is as follows: observe that
      \begin{align*}
        \phi_X(t) &= E\left[ e^{tX} \right] = E\left[ \sum_{i=0}^{\infty}
                    \frac{(tX)^i}{i!} \right] \\
                  &= E\left[ 1 + \frac{tX}{1!} + \frac{(tX)^2}{2!} + \hdots +
                    \frac{(tX)^n}{n!} + \hdots \right] \\
                  &= \frac{t^0}{0!} E[1] + \frac{t}{1!} E[X] + \frac{t^2}{2!}
                    E\left[X^2\right] + \hdots + \frac{t^n}{n!} E\left[X^n\right]
                    + \hdots
      \end{align*}
      by \cref{propo:linearity_of_the_expectation}. If we take the $k$th
      derivative wrt $t$ and set $t = 0$, we will obtain the $k$th moment of
      $X$. In other words,
      \begin{equation*}
        E[X^k] = \phi_X^{(k)}(0) = \frac{d^k}{dt^k} \phi_X (t) \at{t = 0}{}.
      \end{equation*}
      It is \hlimpo{important} to note here that $t = 0$ can be interpreted in
      the sense of a limit, i.e. $\lim\limits_{t \to 0}$.
  \end{enumerate}
\end{note}

\begin{eg}
  Suppose $X \sim \Bin(n, p)$. Find $\mgf(X)$ and use it to find $E[X]$ and
  $\Var(X)$.

  \begin{solution}
    First, observe the \hlnotea{binomial formula}:
    \begin{equation*}
      (a + b)^m = \sum_{x = 0}^{m} \binom{m}{x} a^x b^m.
    \end{equation*}
    Now
    \begin{align*}
      \phi_X(t) &= E \left[ e^{tX} \right] = \sum_{x=0}^{n} e^{tx} \binom{n}{x}
                  p^x (1-p)^{n-x} \\
                &= \sum_{x=0}^{n} \binom{n}{x} \left( pe^t \right)^x \left( 1 -
                  p \right)^{n - x} \\
                &= \left( pe^t + 1 - p \right)^n.
    \end{align*}
    We observe that this works for all $t \in \mathbb{R}$.

    To find the expectation and variance, we do
    \begin{equation*}
      E[X] = \frac{d}{dt} \phi_X(t) \at{t=0}{} = n \left( pe^t + 1 - p
      \right)^{n-1} \cdot pe^t \at{t=0}{} = np
    \end{equation*}
    and
    \begin{align*}
      E[X^2] &= \frac{d^2}{dt^2} \phi_X(t) \at{t=0}{} \\
             &= npe^t \left( pe^t + 1 - p \right)^{n-1} + (n-1)np^2e^{2t} \left( pe^t
               + 1 - p \right)^{n-2} \at{t=0}{} \\
             &= np(1 + np - p)
    \end{align*}
    and conclude that
    \begin{equation*}
      \Var(X) = np + np(n-1)p - n^2p^2 = np(1 - p).
    \end{equation*}
  \end{solution}
\end{eg}

\begin{ex}
  For $X \sim \Poi(\lambda)$, show that
  \begin{equation*}
    \mgf(X) = \phi_X(t) = e^{\lambda(e^t - 1)}, \text{ for } t \in \mathbb{R}.
  \end{equation*}
  Find $E[X]$ and $\Var(X)$ using the mgf.
\end{ex}

\begin{ex}
  For $X \sim \Exp(\lambda)$, show that
  \begin{equation*}
    \mgf(X) = \phi_X(t) = \frac{\lambda}{\lambda - t}, \text{ for } t < \lambda.
  \end{equation*}
  Find $E[X]$ and $\Var(X)$ using the mgf.
\end{ex}

\begin{note}[Important property of the MGF]\label{note:mgf_uniquely_determines_distribution}
  The mgf is important to us because it \hlimpo{uniquely determines} the
  distribution of an rv.
\end{note}

% section moments (end)

\section{Joint Distributions}%
\label{sec:joint_distributions}
% section joint_distributions

We shall only review \hlnotea{bivariate distributions}. One can easily extend
the bivariate case to a multivariate situation.

\begin{defn}[Joint CDF]\index{Joint CDF}\label{defn:joint_cdf}
  The \hlnoteb{joint cdf} is defined by
  \begin{equation*}
    F(a, b) = P(X \leq a, Y \leq b) = P(\{ X \leq a \} \cap \{ Y \leq b \})
  \end{equation*}
  for all $a, b \in \mathbb{R}$, where $X, Y$ are rvs.
\end{defn}

\begin{defn}[Marginal CDF]\index{Marginal CDF}\label{defn:marginal_cdf}
  Given a joint cdf $F$, we define the \hlnoteb{marginal cdf} as
  \begin{equation*}
    F_X(a) = P(X \leq a) \coloneqq F(a, \infty) = \lim_{b \to \infty} F(a, b).
  \end{equation*}
\end{defn}

\begin{defn}[Joint and Marginal Probability Mass Functions]\index{Joint Probability Mass Function}\index{Marginal Probability Mass Function}\label{defn:joint_pmf}
  Suppose $X, Y$ are discrete rvs. We define the \hlnoteb{joint probability mass
  function (pmf)} of $X$ and $Y$ as
  \begin{equation*}
    p(x, y) \coloneqq P(X = x, Y = y).
  \end{equation*}
  The \hlnoteb{marginal pmf} of $X$ and $Y$ are
  \begin{equation*}
    p_X(x) = P(X = x) = \sum_{y} p(x, y)
  \end{equation*}
  and
  \begin{equation*}
    p_Y(y) = P(Y = y) = \sum_{x} p(x, y),
  \end{equation*}
  respectively.
\end{defn}

We are not equipped with the knowledge to formally define a \hlnotea{probability
density function}, and so we shall only introduce it in a roundabout way.

\begin{defn}[Joint and Marginal Probability Density Function]\index{Joint Probability Density Function}\index{Marginal Probability Density Function}\label{defn:joint_pdf}
  Suppose $X, Y$ are continuous rvs. We define \hlnoteb{joint probability
  density function (pdf)} of $X$ and $Y$ as
  \sidenote{
  \begin{marginwarning}
    It is important to note that
    \begin{equation*}
      f(x, y) \neq P(X = x, Y = x),
    \end{equation*}
    and so we cannot put \cref{defn:joint_pmf} and \cref{defn:joint_pdf} as one
    definition.
  \end{marginwarning}
  }
  \begin{equation*}
    f : \mathbb{R}^2 \to \mathbb{R}.
  \end{equation*}
  The \hlnoteb{marginal pdf} of $X$ and $Y$ are
  \begin{equation*}
    f_X(x) \coloneqq \int_{\forall y} f(x, y) \dif{y}
  \end{equation*}
  and
  \begin{equation*}
    f_Y(y) \coloneqq \int_{\forall x} f(x, y) \dif{x},
  \end{equation*}
  respectively.
\end{defn}

\begin{note}
  Note that
  \begin{equation*}
    f(x, y) = \frac{\partial^2}{\partial x \partial y} F(x, y).
  \end{equation*}
\end{note}

\section{Expectation of Joint Distributions}%
\label{sec:expectation_of_joint_distributions}
% section expectation_of_joint_distributions

\begin{defn}[Expectation of Joint Distributions]\index{Expectation of Joint Distributions}\label{defn:expectation_of_joint_distributions}
  Let $X$ and $Y$ be rvs, and $g(X, Y)$ a function. We define the
  \hlnoteb{expectation of a joint distribution} of $X$ and $Y$ as
  \begin{equation*}
    E[g(X, Y)] = \begin{cases}
      \sum_{x} \sum_{y} g(x, y) p(x, y) & X, Y \text{ are jointly discrete } \\
      \int_{\forall x} \int_{\forall y} g(x, y) f(x, y) & X, Y \text{ are jointly continuous }
    \end{cases}
  \end{equation*}
\end{defn}

The following is a special case of an expectation, given for when
\begin{equation*}
  g(X, Y) = (X - E[X])(Y - E[Y]).
\end{equation*}

\begin{defn}[Covariance of Joint Distributions]\index{Covariance of Joint Distributions}\label{defn:covariance_of_joint_distributions}
  We define the \hlnoteb{covariance of a joint distribution} of $X$ and $Y$ as
  \begin{equation*}
    \Cov(X, Y) = E[(X - E[X])(Y - E[Y])] = E[XY] - E[X]E[Y].
  \end{equation*}
\end{defn}

\begin{note}
  Observe that
  \begin{equation*}
    \Cov(X, X) = \Var(X).
  \end{equation*}
\end{note}

\marginnote{
\begin{ex}
  Prove \cref{propo:linearity_of_the_expectation_over_addition}.
\end{ex}
}
\begin{propo}[Linearity of the Expectation over Addition]\label{propo:linearity_of_the_expectation_over_addition}
  Let $X$ and $Y$ be rvs. Then for $a, b \in \mathbb{R}$,
  \begin{equation*}
    E[aX + bY] = aE[X] + bE[Y].
  \end{equation*}
\end{propo}

\marginnote{
\begin{ex}
  Prove \cref{propo:variance_over_a_linear_combination_of_rvs}.
\end{ex}
}
\begin{propo}[Variance over a Linear Combination of RVs]\label{propo:variance_over_a_linear_combination_of_rvs}
  Let $X$ and $Y$ be rvs. Then for $a, b \in \mathbb{R}$,
  \begin{equation*}
    \Var(aX + bY) = a^2 \Var(X) + b^2 \Var(Y) + 2ab \Cov(X, Y).
  \end{equation*}
\end{propo}

% section expectation_of_joint_distributions (end)

% section joint_distributions (end)

\section{Independence of Random Variables}%
\label{sec:independence_of_random_variables}
% section independence_of_random_variables

\begin{defn}[Independence]\index{Independence}\label{defn:independence}
  Let $X$ and $Y$ be rvs, and $F$ their joint cdf. We say that $X$ and $Y$ are
  \hlnoteb{independent} if
  \begin{equation*}
    F(x, y) = F_X(x) \cdot F_Y(y),
  \end{equation*}
  where $F_X$ and $F_Y$ are the marginal cdfs of $X$ and $Y$, respectively. We
  shall denote this relationship between rvs as $X \bot Y$.
\end{defn}

\begin{note}[Equivalent definition of Independence]
  One may also define independence of $X$ and $Y$ by
  \begin{equation*}
    p(x, y) = p_X(x) \cdot p_Y(y)
  \end{equation*}
  if $X$ and $Y$ are discrete, and
  \begin{equation*}
    f(x, y) = f_X(x) \cdot f_Y(y)
  \end{equation*}
  if $X$ and $Y$ are continuous.
\end{note}

\begin{propo}[Expectation of Independent RVs]\label{propo:expectation_of_independent_rvs}
  Let $X$ and $Y$ be independent rvs, and $g$ a function of $X$ and $h$ a
  function of $Y$. Then
  \begin{equation*}
    E(g(X) h(Y)) = E[g(X)] \cdot E[h(Y)].
  \end{equation*}
\end{propo}

\begin{crly}[MGF of Independent RVs]\label{crly:mgf_of_independent_rvs}
  Suppose $X_1, X_2, \ldots, X_n$ are independent rvs, then consider $T =
  \sum_{i=1}^{n} X_i$. The mgf of $T$ is then given by
  \begin{equation*}
    \phi_T(t) = \prod_{i=1}^{n} \phi_{X_i}(t).
  \end{equation*}
\end{crly}

\begin{proof}
  Suppose $X_1, X_2, \ldots, X_n$ are independent rvs, then consider $T =
  \sum_{i=1}^{n} X_i$. Then the mgf of $T$ is
  \begin{align*}
    \phi_T(t) &= E\left[e^{tT}\right] = E[\exp\{t(X_1 + X_2 + \hdots + X_n)\}] \\
              &= E \left[ e^{tX_1} \cdot \hdots \cdot e^{tX_n} \right] \\
              &= E \left[ e^{tX_1} \right] \hdots E \left[ e^{tX_n} \right]
              \enspace \because \cref{propo:expectation_of_independent_rvs} \\
              &= \phi_{X_1}(t) \cdot \hdots \cdot \phi_{X_n}(t) =
              \prod_{i=1}^{n} \phi_{X_i}(t),
  \end{align*}
  which is what we want.
\end{proof}

\begin{eg}[MGF of Independent Binomial Distributions with the Same Probability]
  Let $X_1, X_2, \ldots, X_m$ be an independent sequence of rvs where $X_i \sim
  \Bin(n_i, p)$. Let $T = \sum_{i=1}^{m} X_i$. By
  \cref{crly:mgf_of_independent_rvs} mgf of $T$ is
  \begin{equation*}
    \phi_T(t) = \prod_{i=1}^{m} \phi_{X_i}(t) = \prod_{i=1}^{m} (pe^t + 1 -
    p)^{n_i} = \left( pelt + 1 - p \right)^{\sum_{i=1}^{m} n_i}.
  \end{equation*}
  We observe that
  \begin{equation*}
    T \sim \Bin \left( \sum_{i=1}^{m} n_i, p \right).
  \end{equation*}
\end{eg}

% section independence_of_random_variables (end)

% chapter elementary_probability_review (end)

\chapter{Conditional Probabilities}%
\label{chp:conditional_probabilities}
% chapter conditional_probabilities

\section{Conditional Probability for Discrete Random Variables}%
\label{sec:conditional_probability_for_discrete_random_variables}
% section conditional_probability_for_discrete_random_variables

\begin{defn}[Conditional Distribution of Discrete Random Variables]\index{Conditional Distribution}\index{Conditional Probability}\label{defn:conditional_distribution_of_discrete_random_variables}
  Let $X_1, X_2$ be discrete rvs with joint pmf $p(x_1, x_2)$ and marginal pmfs
  $p_1(x_1)$ and $p_1(x_2)$, respectively. We call $X_1 \mid X_2 = x_2$ the
  \hlnoteb{conditional distribution} of $X_1$ given $X_2 = x_2$.

  The \hlnoteb{conditional pmf} of $X_1 \mid X_2 = x_2$ is defined as
  \begin{equation*}
    p(x_1 \mid x_2) = P(X_1 = x_2 \mid X_2 = x_2) = \frac{P(X_1 = x_1 \land X_2
    = x_2)}{P(X_2 = x_2)},
  \end{equation*}
  or more succinctly,
  \begin{equation*}
    p(x_1 \mid x_2) = P(X_1 \mid X_2 = x_2) = \frac{p(x_1, x_2)}{p_2(x_2)},
  \end{equation*}
  where we require $p_2(x_2) > 0$.
\end{defn}

\begin{note}
  \begin{itemize}
    \item If $X_1$ and $X_2$ are independent, then
      \begin{equation*}
        p(x_1 \mid x_2) = \frac{p(x_1, x_2)}{p_2(x_2)} = \frac{p_1(x_1)
        p_2(x_2)}{p_2(x_2)} = p_1(x_1).
      \end{equation*}
    \item We may extend the above definition to multivariate cases.
  \end{itemize}
\end{note}

\begin{eg}
  Suppose $X_1, X_2, X_3$ are discrete rvs, with $p_3$ as the pmf of $X_3$. Then
  we define the conditional pmf of $(X_1, X_2) \mid X_3 = x_3$ as
  \begin{equation*}
    p((x_1, x_2) \mid x_3) = \frac{p(x_1, x_2, x_3)}{p_3(x_3)}.
  \end{equation*}
\end{eg}

\begin{defn}[Conditional Expectation for Discrete RVs]\index{Conditional Expectation}\label{defn:conditional_expectation_for_discrete_rvs}
  Given rvs $X_1$ and $X_2$, and a function $g$ on $X_1$, We define the
  conditional expectation of $X_1$ given $X_2 = x_2$ as
  \begin{equation*}
    E[g(X_1) \mid X_2 = x_2] = \sum_{x_1} g(x_1) p(x_1 \mid x_2).
  \end{equation*}
\end{defn}

\begin{note}
  By \cref{propo:linearity_of_the_expectation_over_addition}, we have that given
  rvs $X_1$ an $X_2$, $a, b \in \mathbb{R}$, and functions $g, h$ on $X_1$,
  \begin{align*}
    &E \left[ ag(X_1) + bh(X_1) \mid X_2 = x_2 \right] \\
    &= aE[g(X_1) \mid X_2 = x_2] + aE[h(X_1) \mid X_2 = x_2]
  \end{align*}
\end{note}

\begin{propo}[Conditional Variance of Discrete RVs]\label{propo:conditional_variance_of_discrete_rvs}
  We have that
  \begin{equation*}
    \Var(X_1 \mid X_2 = x_2) = E[X_1^2 \mid X_2 = x_2] - E[X_1 \mid X_2 =
    x_2]^2.
  \end{equation*}
\end{propo}

\begin{proof}
  Consider
  \begin{equation*}
    g(x) = [X_1 - E[X_1 \mid X_2 = x_2]]^2,
  \end{equation*}
  which is the typical definition of a variance. Then
  \begin{align*}
    &\Var(X_1 \mid X_2 = x_2) \\
    &= E[(X_1 - E[X_1 \mid X_2 = x_2])^2 \mid X_2 = x_2] \\
    &= E \Big[X_1^2 - 2X_1 E[X_1 \mid X_2 = x_2] + E[X_1 \mid X_2 = x_2]^2 \;
      \Bigg\mid X_2 = x_2 \Big] \\
    &= E[X_1^2 \mid X_2 = x_2] - 2E[X_1 \mid X_2 = x_2]^2 + E[X_1 \mid X_2 =
      x_2]^2 \\
    &= E[X_1^2 \mid X_2 = x_2] - E[X_1 \mid X_2 = x_2]^2.
  \end{align*}
\end{proof}

Even better, we have the following proposition.

\begin{propo}[Linearity of Conditional Expectation]\label{propo:linearity_of_conditional_expectation}
  Given rvs $X_1, \, X_2$ and $X_3$, we have
  \begin{equation*}
    E[X_1 + X_2 \mid X_3 = x_3] = E[X_1 \mid X_3 = x_3] + E[X_2 \mid X_3 = x_3].
  \end{equation*}
\end{propo}

\begin{proof}
  Let $p_3$ be the pmf of $X_3$. Observe that
  \begin{align*}
    &E[X_1 + X_2 \mid X_3 = x_3] \\
    &= \sum_{x_1} \sum_{x_2} (x_1 + x_2) p(x_1, x_2 \mid x_3) \\
    &= \sum_{x_1} \sum_{x_2} x_1 \frac{p(x_1, x_2, x_3)}{p_3(x_3)} + \sum_{x_1}
      \sum_{x_2} x_2 \frac{p(x_1, x_2, x_3)}{p_3(x_3)} \\
    &= \sum_{x_1} \frac{x_1}{p_3(x_3)} \sum_{x_2} p(x_1, x_2, x_3) + \sum_{x_2}
      \frac{x_2}{p_3(x_3)} \sum_{x_1} p(x_1, x_2, x_3) \\
    &= \sum_{x_1} \frac{x_1}{p_3(x_3)} p(x_1, x_3) + \sum_{x_2}
      \frac{x_2}{p_3(x_3)} p(x_2, x_3) \\
    &= \sum_{x_1} x_1 p(x_1 \mid x_3) + \sum_{x_2} x_2 p(x_2 \mid x_3) \\
    &= E[X_1 \mid X_3 = x_3] + E[X_2 \mid X_3 = x_3].
  \end{align*}
\end{proof}

\begin{crly}[General Linearity of Conditional Expectation]\label{crly:general_linearity_of_conditional_expectation}
  Given an rv $Y$, a sequence of discrete rvs $\{X_i\}_{i=1}^{n}$, and a sequence of
  scalars $\{a_i\}_{i=1}^{n} \subseteq \mathbb{R}$, we have
  \begin{equation*}
    E \left[ \sum_{i=1}^{n} a_i X_i \mid Y = y \right] = \sum_{i=1}^{n} a_i
    E[X_i \mid Y = y].
  \end{equation*}
\end{crly}

\begin{eg}
  Suppose that $X$ and $Y$ are discrete rvs having joint pmf
  \begin{equation*}
    p(x, y) = \begin{cases}
      \frac{1}{5}  & x = 1, \, y = 0 \\
      \frac{2}{15} & x = 0, \, y = 1 \\
      \frac{1}{15} & x = 1, \, y = 2 \\
      \frac{1}{5}  & x = 2, \, y = 0 \\
      \frac{2}{5}  & x = 1, \, y = 1 \\
      0            & \text{ otherwise }
    \end{cases}
  \end{equation*}
  Let us first find the conditional distribution of $X \mid Y = 1$. First,
  consider the following table:
  \begin{table}[ht]
    \centering
    \caption{Tabulating values of $p(x, y)$}
    \label{table:tabulating_values_of_p_x_y}
    \begin{tabular}{c | c c c | c}
      $X \diagdown Y$ & $0$           & $1$            & $2$            & $p_X$ \\
      \hline
      $0$             & $0$           & $\frac{2}{15}$ & $0$            & $\frac{2}{15}$ \\
      $1$             & $\frac{1}{5}$ & $\frac{2}{5}$  & $\frac{1}{15}$ & $\frac{2}{3}$ \\
      $2$             & $\frac{1}{5}$ & $0$            & $0$            & $\frac{1}{5}$ \\
      \hline
      $p_Y$           & $\frac{2}{5}$ & $\frac{8}{15}$ & $\frac{1}{15}$
    \end{tabular}
  \end{table}

  Observe that
  \begin{equation*}
    p(0, 1) = \frac{\frac{2}{15}}{\frac{8}{15}} = \frac{1}{4} \text{ and } p(1,
    1) = \frac{\frac{2}{5}}{\frac{8}{15}} = \frac{3}{4}.
  \end{equation*}
  Thus the conditional distribution of $X \mid Y = 1$ is given as in
  \cref{table:conditional_distribution_of_x_mid_y_1}.
  \begin{table}[ht]
    \centering
    \caption{Conditional distribution of $X \mid Y = 1$}
    \label{table:conditional_distribution_of_x_mid_y_1}
    \begin{tabular}{c | c c}
      $X$           & $0$           & $1$ \\
      \hline
      $p(x \mid 1)$ & $\frac{1}{4}$ & $\frac{3}{4}$
    \end{tabular}
  \end{table}

  With this we can calculate the conditional expectation and conditional
  variance of $X \mid Y = 1$. Note that
  \begin{equation*}
    X \mid Y = 1 \sim \Bernoulli \left( \frac{3}{4} \right).
  \end{equation*}
  Thus
  \begin{equation*}
    E[X \mid Y = 1] = \frac{3}{4} \text{ and } \Var(X \mid Y = 1) = \frac{3}{4}
    \cdot \frac{1}{4} = \frac{3}{16}.
  \end{equation*}
\end{eg}

\begin{eg}
  For $i = 1, 2$ suppose that $X_i \sim \Bin(n_i, p)$, where $X_1 \bot X_2$. We
  want to find the conditional distribution of $X_1$ given $X_1 + X_2 = n$, i.e.
  the conditional distribution of $X_1 \mid X_1 + X_2 = n$.

  First, note that the sum of two \hlnotea{binomial distributions} is also a
  binomial distribution, where the number of trials is the sum of the number of
  trials from each distribution. In particular, we have that $X_1 + X_2 \sim
  \Bin(n_1 + n_2, p)$.

  Let the conditional pmf of $X_1 \mid X_1 + X_2 = n$ be denoted as $p(x_1 \mid
  n)$. Then
  \begin{align*}
    p(x_1 \mid n) &= P(X_1 = x_1 \mid X_1 + X_2 = n) \\
                  &= \frac{P(X_1 = x_1, \, X_1 + X_2 = n)}{P(X_1 + X_2 = n)} \\
                  &= \frac{P(X_1 = x_1, X_2 = n - x_1)}{P(X_1 + X_2 = n)} \\
                  &= \frac{P(X_1 = x_1)P(X_2 = n - x_1)}{P(X_1 + X_2 = n)}
                  \enspace \because X_1 \bot X_2 \\
                  &= \frac{\binom{n_1}{x_1}p^{x_1}(1-p)^{n_1 - x_1}
                    \binom{n_2}{n - x_1}p^{n - x_1}(1-p)^{n_2 - n +
                    x_1}}{\binom{n_1 + n_2}{n}p^{n}(1-p)^{n_1 + n_2 - n}} \\
                  &= \frac{\binom{n_1}{x_1} \binom{n_2}{n - x_1}}{\binom{n_1 +
                    n_2}{n}},
  \end{align*}
  for $0 \leq x_1 \leq n_1$ and $0 \leq n - x_1 \leq n_2$. We observe that $X_1
  \mid X_1 + X_2 = n$ has a \hlnotea{Hypergeometric Distribution}, i.e.
  \begin{equation*}
    X_1 \mid X_1 + X_2 = n \sim \HG(n_1 + n_2, n_1, n).
  \end{equation*}

  Recall that the intuition to understanding the hypergeometric distribution in
  this case is as follows: if we say that $1, 2, \ldots, n_1$ are the indexed
  trials of $X_1$ and $1, 2, \ldots, n_2$ are those of $X_2$, then if we arrange
  the trials as
  \begin{equation*}
    1, 2, \ldots, n_1, 1, 2, \ldots, n_2,
  \end{equation*}
  we may then think that we want to calculate the probability of getting $x_1$ 
  successes given that the rest of the $n - x_1$ are failures.

  Using the formulas for the expectation and variance of a hypergeometric
  distribution, we have that
  \begin{equation*}
    E[X_1 \mid X_1 + X_2 = n] = \frac{n(n_1)}{n_1 + n_2}
  \end{equation*}
  and
  \begin{align*}
    \Var(X_1 \mid X_1 + X_2 = n)
    &= \frac{n(n_1)(n_1 + n_2 - n_1)(n_1 + n_2 - n)}{(n_1 + n_2)^2(n_1 + n_2 -
      1)} \\
    &= \frac{n(n_1)(n_2)(n_1 + n_2 - n)}{(n_1 + n_2)^2(n_1 + n_2 - 1)}.
  \end{align*}
\end{eg}

Let's deal with a more general case of the above, but this time with the
individual rvs following $\Poi(\Lambda_i)$.

\begin{eg}
  Let $\{X_i\}_{i=1}^{m}$ be a sequence of independent rvs where $X_i \sim
  \Poi(\Lambda_i)$, and $i = 1, 2, \ldots, m$. Let $Y = \sum_{i=1}^{m} X_i$. Let
  us try to deduce the conditional distribution of $X_j \mid Y = n$.

  First, note that a sum of independent Poisson rvs is also a Poisson
  distribution, with each of its mean (which is also its parameter) summed up,
  i.e. given any $Z_1 \sim \Poi(\lambda_1), \ldots, Z_k \sim \Poi(\lambda_k)$,
  we have
  \begin{equation*}
    \sum_{i=1}^{k} Z_i \sim \Poi \left( \sum_{i=1}^{k} \lambda_i \right).
  \end{equation*}
  Also a subtle point, note that
  \begin{equation*}
    Z_j \; \bot \; \sum_{\substack{i=1 \\ i \neq j}}^{k} Z_i.
  \end{equation*}
  
  Observe that
  \begin{align*}
    &P(X_j = x_j \mid Y = n) \\
    &= \frac{P(X_j = x_j, Y = n)}{P(Y = n)} \\
    &= \frac{P\left(X_j = x_j, \, \sum\limits_{\substack{i=1 \\ i \neq j}}^{m} X_i = n -
      x_j \right)}{P(Y = n)} \\
    &= \frac{P(X_j = x_j) P\left( \sum\limits_{\substack{i=1 \\ i \neq j}}^{m} X_i = n - x_j \right)}{P(Y = n)} \\
    &= \frac{\exp(-\lambda_j) \lambda_j^{x_j} \left( \frac{1}{x_j!} \right) \cdot
      \exp\left(-\left( \sum\limits_{\substack{i=1 \\ i \neq j}}^{m} \lambda_i \right)\right)
      \left( \sum\limits_{\substack{i=1 \\ i \neq j}}^{m} \lambda_i \right)^{n-x_j}
      \left( \frac{1}{(n - x_j)!} \right)}{\exp\left(-\left(
      \sum\limits_{i=1}^{m} \lambda_i \right)\right) \left( \sum\limits_{i=1}^{m}
      \lambda_i \right)^m \left( \frac{1}{n!} \right)} \\
    &= \binom{n}{x_j} \frac{\lambda_j^{x_j} \left( \sum\limits_{i=1}^{m}
      \lambda_i - \lambda_j \right)^{n - x_j}}{\left( \sum\limits_{i=1}^{m}
      \lambda_i \right)^{x_j} \left( \sum\limits_{i=1}^{m} \lambda_i
      \right)^{n-x_j}} \\
    &= \binom{n}{x_j} \left( \frac{\lambda_j}{\sum_{i=1}^{m} \lambda_i}
      \right)^{x_j} \left( 1 - \frac{\lambda_j}{\sum_{i=1}^{m} \lambda_i}
      \right)^{n-x_j},
  \end{align*}
  where $x_j \in [0, n]$. We see that
  \begin{equation*}
    X_j \mid Y = n \sim \Bin \left( n, \frac{\lambda_j}{\sum_{i=1}^{m}
    \lambda_i} \right).
  \end{equation*}
\end{eg}

\begin{ex}
  It is a straightforward exercise to compute $E[X_j \mid Y = n]$ and $\Var(X_j
  \mid Y = n)$.
\end{ex}

\begin{eg}
  Suppose $X \sim \Poi(\lambda)$ and $Y \mid X = x \sim \Bin(x, p)$. Let us
  compute the conditional distribution of $X \mid Y = y$. We have that
  \begin{align*}
    p(x \mid y)
    &= \frac{P(X = x, Y = y)}{P(Y = y)} \\
    &= \frac{p(y \mid x) \cdot P(X = x)}{\sum_{x} p(y \mid x) P(X = x)} \\
    &= \frac{\exp(-\lambda) \lambda^x \left( \frac{1}{x!} \right) \cdot
        \binom{x}{y} p^y (1-p)^{x-y}}{\sum\limits_{x = y}^{\infty}
        \exp(-\lambda)\lambda^x \left( \frac{1}{x!} \right) \cdot \binom{x}{y}
        p^y (1-p)^{x-y}} \\
    &= \frac{\exp(-\lambda) \lambda^x \cdot \frac{1}{y!(x-y)!} p^y
        (1-p)^{x-y}}{\sum\limits_{x = y}^{\infty} \exp(-\lambda)\lambda^x \cdot
        \frac{1}{y!(x-y)!} p^y (1-p)^{x-y}}
  \end{align*}
  Now notice that we may work out the denominator to be
  \begin{align*}
    & \sum_{x=y}^{\infty} e^{-\lambda} \lambda^x \frac{1}{y!(x-y)!}p^y
      (1-p)^{x-y} \\
    &= \frac{e^{-\lambda}}{y!} p^y \sum_{x=y}^{\infty} \frac{\lambda^x}{(x-y)!}
      (1-p)^{x-y} \\
    &= \frac{e^{-\lambda} p^y \lambda^y}{y!} \sum_{x=y}^{\infty}
      \frac{\lambda^{x-y}(1-p)^{x-y}}{(x-y)!} \\
    &= \frac{e^{-\lambda} p^y \lambda^y}{y!} \sum_{x-y=0}^{\infty}
      \frac{(\lambda(1-p))^{x-y}}{(x-y)!} \\
    &= \frac{e^{-\lambda + \lambda(1 - p)} p^y \lambda^y}{y!} \\
    &= \frac{e^{-\lambda p} (\lambda p)^y}{y!},
  \end{align*}
  where the second last equality follows since
  \begin{equation*}
    e^x = \sum_{n=0}^{\infty} \frac{x^n}{n!}
  \end{equation*}
  by the \hlnotea{Taylor Expansion of the exponential function}. Thus we have
  \begin{align*}
    p(x \mid y)
    &= \frac{\exp(-\lambda) \lambda^x \cdot \frac{1}{y!(x-y)!} p^y
      (1-p)^{x-y}}{\frac{(\lambda p)^y e^{\lambda p}}{y!}} \\
    &= \frac{e^{-\lambda(1 - p)}(\lambda(1-p))^{x-y}}{(x-y)!},
  \end{align*}
  for $x \in [y, \infty)$, which is a Poisson-like pmf.

  We say that
  \begin{equation*}
    X \mid Y = y \sim W + y
  \end{equation*}
  where
  \begin{equation*}
    W \sim \Poi(\lambda(1 - p)).
  \end{equation*}
  This distribution, $W + y$, is called a \hldefn{shifted Poisson}
  distribution.

  Observe that it is relatively easy to compute the expectation and variance due
  to their linearity properties: we have
  \begin{equation*}
    E[X \mid Y = y] = E[W + y] = E[W] + y = \lambda(1-p) + y
  \end{equation*}
  and
  \begin{equation*}
    \Var(X \mid Y = y) = \Var(W + y) = \Var(W) = \lambda(1 - p).
  \end{equation*}
\end{eg}

% section conditional_probability_for_discrete_random_variables (end)

\section{Conditional Probability for Continuous Random Variables}%
\label{sec:conditional_probability_for_continuous_random_variables}
% section conditional_probability_for_continuous_random_variables

In the
\hyperref[sec:conditional_probability_for_discrete_random_variables]{jointly
discrete case}, it was natural to define
\begin{equation*}
  p(x \mid y) = P(X = x \mid Y = y) = \frac{P(X = x, Y = y)}{P(Y = y)}
\end{equation*}
due to \hlnotea{Bayes' Rule}, or simply by realizing that we are trying to
determine the probability of $X = x$ and $Y = y$ given that we already know the
probability of $Y = y$.

This, however, does not make sense in the continuous case, especially since
$f(x, y) \neq P(X = x, Y = y)$, and $f_Y(y) \neq P(Y = y)$.

\sidenote{I am actually not sure how this paragraph inspires our definition,
cause some things don't seem to match up nicely.} We can use a similar
definition of a single variable continuous rv and extend it for 2 rvs: we can
consider
\begin{equation}\label{eq:conditional_prob_for_cts_rvs_inspiration}
  f(x, y) = \lim_{\substack{\dif{x} \to 0 \\ \dif{y} \to 0}} \frac{P(x \leq X
  \leq x + \dif{x}, y \leq Y \leq y + \dif{y})}{\dif{x} \dif{y}}.
\end{equation}
Notice that for small $\dif{x}$ and $\dif{y}$, we have
\begin{align*}
  P(\underbrace{x \leq X \leq x + \dif{x}}_{A} \mid \underbrace{y \leq Y \leq y
  + \dif{y}}_{B})
  &= \frac{P(A \cap B)}{P(B)} \\
  &\approx \frac{f(x,y) \dif{x} \dif{y}}{f_Y(y) \dif{y}} \\
  &= \frac{f(x,y)}{f_Y(y)} \dif{x}.
\end{align*}

\begin{defn}[Conditional Distribution of Continuous Random Variables]\index{Conditional Distribution}\label{defn:conditional_distribution_of_continuous_random_variables}
  Let $X$ and $Y$ be continuous rvs, with joint pdf $f(x, y)$, and $Y$ has the
  marginal pdf $f_Y$. We call $X \mid Y = y$ the \hlnoteb{conditional
  distribution} of $X$ given $Y = y$, whose pdf is defined as
  \begin{equation*}
    f(x \mid y) = \frac{f(x, y)}{f_Y(y)} = \lim_{\substack{\dif{x} \to 0 \\
    \dif{y} \to 0}} \frac{P(x \leq X \leq x + \dif{x}, y \leq Y \leq y +
    \dif{y})}{\dif{x}}.
  \end{equation*}
\end{defn}

\begin{eg}
  Suppose the joint pdf of $X$ and $Y$ is given by
  \begin{equation*}
    f(x, y) = \begin{cases}
      \frac{12}{5} x (2 - x - y) & 0 < x < 1, \, 0 < y < 1 \\
      0                          & \text{ otherwise }
    \end{cases}
  \end{equation*}
  \hlimpo{First}, note that the \hldefn{region of support} for $X$ and $Y$ is
  given as in \cref{fig:region_of_support_as_a_rectangle}.
  \begin{marginfigure}
    \centering
    \begin{tikzpicture}
      \draw[->] (-0.5, 0) -- (2, 0) node[right] {$x$};
      \draw[->] (0, -0.5) -- (0, 2) node[above] {$y$};
      \fill[color=blue!30!background] (0, 0) -- (1, 0) -- (1, 1) -- (0, 1) -- cycle;
      \node[none,circle,fill,inner sep=1pt,label={270:{$1$}}] at (1, 0) {};
      \node[none,circle,fill,inner sep=1pt,label={180:{$1$}}] at (0, 1) {};
    \end{tikzpicture}
    \caption{Region of support as a rectangle}\label{fig:region_of_support_as_a_rectangle}
  \end{marginfigure}
  We compute the conditional pdf of $X \mid Y = y$:
  \begin{align*}
    f(x \mid y) &= \frac{f(x,y)}{f_Y(y)} \\
                &= \frac{\cancel{\frac{12}{5}}x(2-x-y)}{\int_{0}^{1} \cancel{\frac{12}{5}}x(2-x-y)
                  \dif{x}} \\
                &= \frac{2x - x^2 - xy}{x^2 - \frac{1}{3}x^3 - \frac{1}{2}x^2 y
                  \at{x=0}{x=1}} \\
                &= \frac{2x - x^2 - xy}{1 - \frac{1}{3} - \frac{1}{2} y} \\
                &= \frac{2x - xy - x^2}{\frac{2}{3} - \frac{1}{2} y} \\
                &= \frac{12x - 6xy - 6x^2}{4 - 3y}, \text{ for } 0 < x < 1.
  \end{align*}
\end{eg}

% section conditional_probability_for_continuous_random_variables (end)

% chapter conditional_probabilities (end)

\appendix

\backmatter

\fancyhead[LE]{\thepage \enspace \textsl{\leftmark}}

\bibliography{references}

\printindex

\end{document}
% vim:tw=80:fdm=syntax
