% Document Head
\documentclass[11pt, oneside]{book}
\usepackage{geometry}
\geometry{letterpaper}
\usepackage[parfill]{parskip}
\usepackage{graphicx}

% Essential Packages
\usepackage{ragged2e}
\usepackage{amssymb}
\usepackage{amsmath}
\usepackage{mathrsfs}
\usepackage{tkz-euclide}
\usetkzobj{all}
\usepackage[utf8]{inputenc}
\usepackage[english]{babel}
\usepackage{pgf,tikz}
\usepackage{pgfplots}
\usepackage{faktor}
\usepackage{float}
\usepackage[hyperref]{ntheorem}
\usepackage{hyperref}
\usepackage[noabbrev,capitalize,nameinlink]{cleveref}

% hyperref Package Settings
\usepackage{hyperref}
\hypersetup{
	colorlinks = true,
	linkcolor = magenta
}

% tikz Libraries
\usepgfplotslibrary{polar}
\usepgflibrary{shapes.geometric}
\usetikzlibrary{angles,patterns,calc}

% Theorem Style Customization
\setlength\theorempreskipamount{2ex}
\setlength\theorempostskipamount{3ex}

% ntheorem Declarations
\theoremstyle{break}
\newtheorem{thm}{Theorem}[section]
\newtheorem*{proof}{Proof}
\newtheorem*{solution}{Solution}
\newtheorem{crly}{Corollary}[section]
\newtheorem{lemma}{Lemma}[section]
\newtheorem{propo}{Proposition}[section]
\newtheorem*{remark}{Remark}
\newtheorem*{note}{Note}
\newtheorem*{notation}{Notation}
\newtheorem{ex}{Exercise}[section]
\newtheorem{defn}{Definition}[section]
\newtheorem{eg}{Example}[section]
\newtheorem{axiom}{Axiom}[section]

% ntheorem listtheorem style
\makeatother
\newlength\widesttheorem
\AtBeginDocument{
  \settowidth{\widesttheorem}{Proposition A.1.1.1\quad}
}

\makeatletter
\def\thm@@thmline@name#1#2#3#4{%
        \@dottedtocline{-2}{0em}{2.3em}%
                   {\makebox[\widesttheorem][l]{#1 \protect\numberline{#2}}#3}%
                   {#4}}
\@ifpackageloaded{hyperref}{
\def\thm@@thmline@name#1#2#3#4#5{%
    \ifx\#5\%
        \@dottedtocline{-2}{0em}{2.3em}%
            {\makebox[\widesttheorem][l]{#1 \protect\numberline{#2}}#3}%
            {#4}
    \else
        \ifHy@linktocpage\relax\relax
            \@dottedtocline{-2}{0em}{2.3em}%
                {\makebox[\widesttheorem][l]{#1 \protect\numberline{#2}}#3}%
                {\hyper@linkstart{link}{#5}{#4}\hyper@linkend}%
        \else
            \@dottedtocline{-2}{0em}{2.3em}%
                {\hyper@linkstart{link}{#5}%
                  {\makebox[\widesttheorem][l]{#1 \protect\numberline{#2}}#3}\hyper@linkend}%
                    {#4}%
        \fi
    \fi}
}

\makeatletter
\def\thm@@thmline@noname#1#2#3#4{%
        \@dottedtocline{-2}{0em}{5em}%
                   {{\protect\numberline{#2}}#3}%
                   {#4}}
\@ifpackageloaded{hyperref}{
\def\thm@@thmline@noname#1#2#3#4#5{%
    \ifx\#5\%
        \@dottedtocline{-2}{0em}{5em}%
            {{\protect\numberline{#2}}#3}%
            {#4}
    \else
        \ifHy@linktocpage\relax\relax
            \@dottedtocline{-2}{0em}{5em}%
                {{\protect\numberline{#2}}#3}%
                {\hyper@linkstart{link}{#5}{#4}\hyper@linkend}%
        \else
            \@dottedtocline{-2}{0em}{5em}%
                {\hyper@linkstart{link}{#5}%
                  {{\protect\numberline{#2}}#3}\hyper@linkend}%
                    {#4}%
        \fi
    \fi}
}

\theoremlisttype{allname}

% Custom math operator
% \DeclareMathOperator{\rem}{rem}
\DeclareMathOperator{\re}{Re}
\DeclareMathOperator{\im}{Im}
\DeclareMathOperator{\caparg}{Arg}

% Graph styles
\pgfplotsset{compat=1.15}
\usepgfplotslibrary{fillbetween}
\pgfplotsset{four quads/.append style={axis x line=middle, axis y line=
middle, xlabel={$x$}, ylabel={$y$}, axis equal }}
\pgfplotsset{four quad complex/.append style={axis x line=middle, axis y line=
middle, xlabel={$\re$}, ylabel={$\im$}, axis equal }}

% Shortcuts
\newcommand{\floor}[1]{\lfloor #1 \rfloor}      % simplifying the writing of a floor function
\newcommand{\ceiling}[1]{\lceil #1 \rceil}      % simplifying the writing of a ceiling function
\newcommand{\dotp}{\, \cdotp}			        % dot product to distinguish from \cdot
\newcommand{\qed}{\hfill\ensuremath{\square}}   % Q.E.D sign
\newcommand{\abs}[1]{\left|#1\right|}						% absolute value
\newcommand{\Arg}[1]{\caparg #1}
\renewcommand{\bar}[1]{\mkern 1.5mu \overline{\mkern -1.5mu #1 \mkern -1.5mu} \mkern 1.5mu}
\newcommand{\quotient}[2]{\faktor{#1}{#2}}
\newcommand{\cyclic}[1]{\left\langle #1 \right\rangle}

% Main Body
\title{Assignment 1}
\author{Johnson Ng}

\begin{document}

\begin{tabular*}{\textwidth}{l @{\extracolsep{\fill}} l @{\extracolsep{\fill}} l @{\extracolsep{\fill}} l}
	Student name &: Ng Keng Seng	&	UW ID\#		&: 20635817 \\
	Class 		 &: PMATH352		&	Assignment 	&: \#1 \\
	Term 		 &: Winter 2018
\end{tabular*}

\hrule

\begin{enumerate}
	\item A total ordering on $\mathbb{C}$ is a relation $\succ$ between complex numbers that satisfies \textit{all} the following conditions:
	\begin{enumerate}
		\item [(C1)] For any two complex numbers $z, w$, one and only one of the following is true: $z \succ w, w \succ z$ or $z = w$.
		\item [(C2)] For all $z_1, z_2, z_3 \in \mathbb{C}$, the relation $z_1 \succ z_2$ implies $z_1 + z_3 \succ z_1 + z_3$.
		\item [(C3)] For all $z_1, z_2, z_3 \in \mathbb{C}$, if $z_3 \succ 0$, then the relation $z_1 \succ z_2$ implies $z_1 z_3 \succ z_2 z_3$.
	\end{enumerate}
	Show that it is impossible to define a total ordering on $\mathbb{C}$. [Hint: Assume a relation between $i$ and $0$]

	\begin{proof}
		We know that $i \neq 0$.

		Suppose $i \succ 0$. Then by (C2),
		\begin{equation*}
			i \succ 0 \quad \implies \quad i - i = 0 \succ -i = 0 - i
		\end{equation*}
		But by (C3),
		\begin{equation*}
			i \succ 0 \quad \implies \quad i^2 = -1 \succ 0 \quad \implies \quad (-1)i = -i \succ 0.
		\end{equation*}
		Thus $i \not\succ 0$. Suppose $0 \succ i$. By (C2),
		\begin{equation*}
			0 \succ i \quad \implies \quad 0 - i = -i \succ 0 = i - i
		\end{equation*}
		But by (C3),
		\begin{equation*}
			-i \succ 0 \quad \implies \quad (-i)(-i) = -1 \succ 0 \quad \implies \quad (-1)(-i) = i \succ 0.
		\end{equation*}
		Then $0 \not\succ i$. Hence, it is impossible to define a total ordering on $\mathbb{C}$. \qed
	\end{proof}

	\item Let $w$ be a complex number with $0 < \abs{w} < 1$. Show that the set of all $z \in \mathbb{C}$ with $\abs{z - w} < \abs{1 - \bar{w}z}$ is the disc $\{z \in \mathbb{C} : \abs{z} < 1\}$.

	\begin{proof}
		Let $w = u + iv$ and $z = x + iy$ where $u, v, x, y \in \mathbb{R}$. Note that
		\begin{equation*}
			\bar{w}z = (u - iv)(x + iy) = ux + vy + i(uy - vx).
		\end{equation*}
		Thus
		\begin{align*}
			\abs{z - w} &< \abs{1 - \bar{w}z} \\
			(x - u)^2 + (y - v)^2 &< (1 - ux - vy)^2 + (uy - vx)^2 \\
			(x^2 + y^2)[1 - (u^2 + v^2)] + u^2 + v^2 &< 1 \\
			x^2 + y^2 &< 1
		\end{align*}
		where in the second line, we may disregard the square roots since all the terms are squares and are, therefore, positive. This completes the proof. \qed
	\end{proof}

	\item Let $P(z)$ be a polynomial with real coefficients. Show that the complex roots of $P$ appear in conjugate pairs.

	\begin{proof}
		Let
		\begin{equation*}
			P(z) = a_n z^n + a_{n - 1} z^{n - 1} + \hdots + a_1 z + a_0 = \sum_{k = 0}^{n} a_k z^k
		\end{equation*}

		and $w \in \mathbb{C}$ be a complex root of $P$. Thus
		\begin{equation}\label{eq:3_1}
			\sum_{k = 0}^{n} a_k w^k = 0
		\end{equation}

		Now
		\begin{equation*}
			P(\bar{w}) = \sum_{k = 0}^{n} a_k \bar{w}^k = \sum_{k = 0}^{n} a_k \bar{w^k} = \sum_{k=0}^{n} \bar{a_k w^k} = \bar{\sum_{k=0}^{n} a_k w^k}
		\end{equation*}
		by the properties of complex numbers. By \cref{eq:3_1}, we obtain that
		\begin{equation*}
			P(\bar{w}) = \bar{\sum_{k=0}^{n} a_k w^k} = \bar{0} = 0,
		\end{equation*}
		showing that the conjugate of $w$ is also a root of $P$. \qed
	\end{proof}

	\pagebreak

	\item Suppose $f, g : \Omega \subseteq \mathbb{C} \to \mathbb{C}$ are holomorphic at $z_0 \in \Omega$. Show that $fg$ is holomorphic and at $z_0$ and $(fg)' = f'g + fg'$ at $z_0$.

	\begin{proof}
		Consider the limit
		\begin{align*}
				&\lim_{h \to 0} \frac{f(z_0 + h)g(z_0 + h) - f(z_0)g(z_0)}{h} \\
			=	&\lim_{h \to 0} \frac{f(z_0 + h)g(z_0 + h) + f(z_0)g(z_0 + h) - f(z_0)g(z_0 + h) - f(z_0)g(z_0)}{h} \\
			=	&\lim_{h \to 0} \left[ \frac{f(z_0 + h) - f(z_0)}{h} g(z_0 + h) + f(z_0) \frac{g(z_0 + h) - g(z_0)}{h} \right] \\
			=	&f'(z_0)g(z_0) + f(z_0)g'(z_0)
		\end{align*}
		Thus, we have that $fg$ is holomorphic, and that $(fg)' = f'g + fg'$ at $z_0$. \qed
	\end{proof}

	\item A function $f$ is said to be entire if $f$ is holomorphic in the entire complex plane. Consider a polynomial in $z$ of degree $n \geq 1$:
	\begin{equation*}
		P(z) = a_n z^n + a_{n - 1} z^{n - 1} + \hdots + a_1 z + a_0, \quad (a_i \in \mathbb{C}, a_n \neq 0)
	\end{equation*}
	Show that
	\begin{enumerate}
		\item $P(z)$ is an entire function and $P'(z) = na_n z^{n - 1} + (n - 1)a_{n - 1} z^{n - 2} + \hdots + a_1$.
		\item $P$ cannot take only imaginary values.
	\end{enumerate}

	\begin{proof}
		\begin{enumerate}
			\item We will prove this statement inductively. $\forall n \in \mathbb{N} \setminus \{0\}$, let $Q(n)$ be the statement: $\forall z \in \mathbb{C}, P(z)$ is an entire function and $P'(z) = na_n z^{n - 1} + (n - 1)a_{n - 1} z^{n - 2} + \hdots + a_1$.

			When $n = 1, P(z) = a_1z + a_0$, then
			\begin{equation*}
				\lim_{h \to 0} \frac{a_1(z + h) + a_0 - a_1 z - a_0}{h} = \lim_{h \to 0} \frac{a_1 h}{h} = a_1.
			\end{equation*}
			Thus $Q(1)$ is true. Let $k \in \mathbb{N} \setminus \{0\}$, and suppose that $Q(k)$ is true, i.e. we have
			\begin{align}
				P'(z)
					&= \lim_{h \to 0} \frac{\sum_{i=0}^{k} a_i \left(\sum_{j=0}^{i} \binom{i}{j}z^{i - j} h^j \right) - \sum_{i=0}^{k} a_i z^i}{h} \nonumber \\
					&= \lim_{h \to 0} \frac{\sum_{i=0}^{k} a_i \left[\sum_{j=0}^{i} \left( \binom{i}{j} z^{i - j} h^j - z^i \right)\right]}{h} \label{eq:5_1} \\
					&= ka_k z^{k - 1} + (k - 1)a_{k - 1} z^{k - 2} + \hdots + a_1 \nonumber
			\end{align}

			When $n = k + 1$,
			\begin{align*}
					&\lim_{h \to 0} \frac{\sum_{i=0}^{k+1} a_i \left(\sum_{j=0}^{i} \binom{i}{j} z^{i-j} h^j \right) - \sum_{i=0}^{k+1} a_i z^i}{h} \\
				\begin{split}
					=&\lim_{h \to 0} \frac{a_{k+1}\left[\sum_{j=0}^{k+1} \left( \binom{k+1}{j}z^{k+1-j} h^j - z^{k + 1} \right)\right]}{h} \\
					&\quad+ \lim_{h \to 0} \frac{\sum_{i=0}^{k} a_i \left[\sum_{j=0}^{i} \left( \binom{i}{j} z^{i - j} h^j - z^i \right)\right]}{h}
				\end{split} \\
				\begin{split}
					=&\lim_{h \to 0} \frac{a_{k+1} \left[z^{k + 1} + \binom{k+1}{1}z^k h + \binom{k+1}{2} z^{k - 1} h^2 + \hdots + h^{k + 1} - z^{k + 1} \right]}{h} \\
					&\quad+ \cref{eq:5_1}
				\end{split}\\
				=& (k + 1)a_{k + 1}z^k + \cref{eq:5_1}
			\end{align*}
			Thus $Q(k + 1)$ is true, and hence by mathematical induction, $\forall n \in \mathbb{N} \setminus \{0\}, P(z)$ is an entire function.

			\item 
		\end{enumerate}
	\end{proof}
\end{enumerate}

\end{document}