\section{Integration} % (fold)
\label{sec:integration}

\subsection{Curves} % (fold)
\label{sub:curves}
A curve in $\mathbb{C}$ is a cont' fn $\gamma : [a, b] \subseteq \mathbb{R} \to \mathbb{C}$. Image of $\gamma$ is called $\gamma^*$.
% subsection curves (end)

\subsection{Equivalent Parameterization} % (fold)
\label{sub:equivalent_parameterization}
Let $\gamma_1 : [a, b] \subseteq \mathbb{R} \to \mathbb{C} \; \gamma_2 : [c, d] \subseteq \mathbb{R} \to \mathbb{C}$ desc path $\gamma^*$. $\gamma_1, \gamma_2$ are equiv if $\exists h : [a, b] \to [c, d]$, bijective and cont', s.t. $\forall t \in \dom(h) \; \gamma_1 (t) = \gamma_2(h(t))$.
% subsection equivalent_parameterization (end)

\subsection{Smooth Curve} % (fold)
\label{sub:smooth_curve}
$\gamma$ is smooth if $\exists \gamma'$ is cont' on $\dom(\gamma) \; \land \; \forall t \in \dom(\gamma) \; \gamma'(t) \neq 0$.
% subsection smooth_curve (end)

\subsection{Piecewise Smooth Curve} % (fold)
\label{sub:piecewise_smooth_curve}
$\gamma$ is piecewise smooth if $\gamma$ is smooth on $\dom(\gamma)$ except on finitely many pts.
% subsection piecewise_smooth_curve (end)

\subsection{Integral over path} % (fold)
\label{sub:integral_over_path}
Let $\gamma: [a, b] \to \mathbb{C} \; \land \; f : \mathbb{C} \to \mathbb{C}$ con' on $\gamma$. Integral $f$ along $\gamma$ is
\begin{equation*}
	\int_{\gamma} f(z) dz = \int_{a}^{b} f(\gamma(t))\gamma'(t) dt
\end{equation*}
Integral over a curve $\gamma^*$ is independent of the path chosen.
% subsection integral_over_path (end)

\subsection{Integral Properties} % (fold)
\label{sub:integral_properties}
\begin{enumerate}
	\item (Linearity) $\int_{\gamma} (\alpha f + \beta g) = \alpha \int_{\gamma} f + \beta \int_{\gamma} g$
	\item \begin{enumerate}
		\item $\abs{\int_{a}^{b} g} \leq \int_{a}^{b} \abs{g}$
		\item $\abs{\int_{\gamma} f dz} \leq \sup_{z \in \Omega} f(z) \cdot \int_{a}^{b} \abs{\gamma'(t)} dt$
	\end{enumerate}
	\item $\gamma^-$ is in opposite orientation of $\gamma \implies \int_{\gamma^-} f = - \int_{\gamma} f$
\end{enumerate}
% subsection integral_properties (end)

\subsection{Fundamental Theorem of Calculus} % (fold)
\label{sub:fundamental_theorem_of_calculus}
Let $(\gamma : [a, b] \subseteq \mathbb{R} \to \mathbb{C}) \in \Omega \subseteq \mathbb{C}$. $f$ cont' on $\gamma \; \exists F' = f$ holo on $\Omega \implies \int_{\gamma} f = F(\gamma(b)) - F(\gamma(a))$
% subsection fundamental_theorem_of_calculus (end)

\subsection{Corollary of FTC} % (fold)
\label{sub:corollary_of_ftc}
If $F \in H(\Omega), \, \Omega \subseteq \mathbb{C}, \, \gamma \subseteq \Omega$ that is a closed path, then
\begin{equation*}
	\int_{\gamma} F'(z) \dif{z} = 0
\end{equation*}
% subsection corollary_of_ftc (end)

\subsection{Goursat's Theorem} % (fold)
\label{sub:goursat_s_theorem}
Let $\Omega \subseteq \mathbb{C}$ be open. Sps $\Delta \subseteq \Omega$ is a closed triangle, and $\Delta^0 \subseteq \Omega$, and let $f \in H(\Omega)$. Then
\begin{equation*}
	\int_{\Delta} f(z) \dif{z} = 0
\end{equation*}
% subsection goursat_s_theorem (end)

\subsection{Convex Set} % (fold)
\label{sub:convex_set}
A set $S \subseteq \mathbb{C}$ is a convex set if the line segment joining any pair of points in $S$ lies entirely in $S$.
% subsection convex_set (end)

\subsection{Cauchy's Theorem for Convex Set} % (fold)
\label{sub:cauchy_s_theorem_for_convex_set}
Let $\Omega \subseteq \mathbb{C}$ be a convex open set, and $f \in H(\Omega)$. Then
\begin{enumerate}
	\item $f = F'$ for some $F \in H(\Omega)$
	\item $\int_{\gamma} f(z) \dif{z} = 0$ for any closed path $\gamma \in \Omega$
\end{enumerate}
% subsection cauchy_s_theorem_for_convex_set (end)

\subsection{Cauchy's Integral Formula 1} % (fold)
\label{sub:cauchy_s_integral_formula_1}
Let $\Omega \subseteq \mathbb{C}$ be a convex open set, and $C$ be a closed circle path in $\Omega$. If $w \in \Omega \setminus \partial C$, and $f \in H(\Omega)$, then
\begin{equation*}
	f(w) \ind{w}{C} = \frac{1}{2 \pi i} \int_{C} \frac{f(z)}{z - w} \dif{z}
\end{equation*}
where
\begin{equation*}
	\ind{w}{C} = \frac{1}{2 \pi i} \int_{\gamma} \frac{\dif{z}}{z - w}
\end{equation*}
% subsection cauchy_s_integral_formula_1 (end)

\subsection{Holomorphic Functions as Power Series} % (fold)
\label{sub:holomorphic_functions_as_power_series}
Let $\Omega \subseteq \mathbb{C}$ be an open set, $f \in H(\Omega)$. Then $f$ can be expressed as a power series.
% subsection holomorphic_functions_as_power_series (end)

\subsection{Cauchy's Integral Formula 2} % (fold)
\label{sub:cauchy_s_integral_formula_2}
Let $\Omega \subseteq \mathbb{C}$ be open, $f \in H(\Omega)$. Then
\begin{enumerate}
	\item $\forall w \in \Omega, \, f$ has a power series expansion at $w$
	\item $f$ is differentiable infinitely many times in $\Omega$
	\item $\forall C \subseteq \Omega$ that is a closed circle oriented anticlockwise, $\forall w \in C^0$,
	\begin{equation*}
		f^{(n)}(w) = \frac{n!}{2 \pi i} \int_{C} \frac{f(z)}{(z - w)^{n + 1}} \dif{z}
	\end{equation*}
\end{enumerate}
% subsection cauchy_s_integral_formula_2 (end)

\subsection{Taylor Expansion of Entire Functions} % (fold)
\label{sub:taylor_expansion_of_entire_functions}
If $f$ is entire, then $\forall z_0 \in \mathbb{C}$,
\begin{equation*}
	f(z) = \sum_{n=0}^{\infty} \frac{f^{(n)}(z_0)}{n!} (z - z_))^n
\end{equation*}
% subsection taylor_expansion_of_entire_functions (end)

\subsection{Analytic Functions} % (fold)
\label{sub:analytic_functions}
$f$ is analytic in $\Omega$ if $f$ has a power series expansion $\forall z \in \Omega$.
% subsection analytic_functions (end)

\subsection{Analyticity \& Holomorphicity} % (fold)
\label{sub:analyticity_&_holomorphicity}
It is an iff relation
% subsection analyticity_&_holomorphicity (end)

\subsection{Cauchy's Inequality} % (fold)
\label{sub:cauchy_s_inequality}
$\forall z_0 \in \mathbb{C} \; \forall R > 0 \in \mathbb{R} \; \forall f \in H(C = D(z_0, R))$
\begin{equation*}
	f^{(n)}(z_0) \leq \frac{n!}{R^n} \cdot \sup_{z \in C} \abs{f(z)}
\end{equation*}
% subsection cauchy_s_inequality (end)

\subsection{Liouville's Theorem} % (fold)
\label{sub:liouville_s_theorem}
A bounded entire function $f : \mathbb{C} \to \mathbb{C}$ is a constant.
% subsection liouville_s_theorem (end)

\subsection{Parseveal's Theorem} % (fold)
\label{sub:parseveal_s_theorem}
$\Omega \subseteq \mathbb{C}$ be open, $f \in H(\Omega)$, $\bar{D(z_0, R)} \subseteq \Omega \implies$

$\forall z \in \bar{D(z_0, R)}, \, f(z) = \sum_{n=0}^{\infty} c_n (z - z_0)^n \implies$

$\forall z \in \bar{D(z_0, R)} \enspace f(z_0 + re^{i \theta}) = \sum_{n=0}^{\infty} c_n (re^{i \theta})^n$
% subsection parseveal_s_theorem (end)

\subsection{Parseval's Indentity} % (fold)
\label{sub:parseval_s_indentity}
Same setup as above,
\begin{equation*}
	\frac{1}{2 \pi} \int_{0}^{2 \pi} \abs{f(z_0 = re^{i \theta})}^2 \dif{\theta} = \sum_{n=0}^{\infty} \abs{c_n}^2 r^{2n}
\end{equation*}
% subsection parseval_s_indentity (end)

\subsection{Principle of Analytic Continuation} % (fold)
\label{sub:principle_of_analytic_continuation}
$\Omega \subseteq \mathbb{C}$ open \& connected, $f \in H(\Omega)$. $Z(f) := \{a \in \Omega : f(a) = 0 \}$. Then either $Z(f) = \Omega$ or $Z(f)$ has no limit point (i.e. points where $f = 0$ are isolated)
% subsection principle_of_analytic_continuation (end)

\subsection{Maximum Modulus Principle} % (fold)
\label{sub:maximum_modulus_principle}

$\Omega \subseteq \mathbb{C} \, f \in H(\Omega) \, \exists r > 0 \, D_{z_0} = \bar{D(z_0, r)} \subseteq \Omega \implies$

$\abs{f(z_0)} \leq \max_{z \in \partial D_{z_0}} \abs{f(z)}$ and

$\abs{f(z_0)} = \max_{z \in \partial D_{z_0}} \iff f $ is constant on $\Omega$

% subsection maximum_modulus_principle (end)

\subsection{Fundamental Theorem of Algebra} % (fold)
\label{sub:fundamental_theorem_of_algebra}
$\forall P(z) \in \mathbb{C}[z] \, \deg P(z) = n \in \mathbb{N} \, \exists \alpha_1, \alpha_2, ..., \alpha_n \in \mathbb{C} \, \land \exists A \in \mathbb{C}$
\begin{equation*}
	P(z) = A(z - \alpha_1)(z - \alpha_2) \hdots (z - \alpha_n)
\end{equation*}
% subsection fundamental_theorem_of_algebra (end)

\subsection{Uniqueness of a Function} % (fold)
\label{sub:uniqueness_of_a_function}
$\Omega \subseteq \mathbb{C}$ open \& connected, $\forall f, g \in H(\Omega)$

For any $\Omega' \subseteq \Omega$, $\forall z \in \Omega' \; f(z) = g(z) \implies$

$\forall z \in \Omega \; f(z) = g(z)$
% subsection uniqueness_of_a_function (end)

% section integration (end)