% Document Head
\documentclass[11pt, oneside]{book}
\setcounter{secnumdepth}{3}
\setcounter{tocdepth}{3}

\renewcommand{\baselinestretch}{1.1}
\usepackage{geometry}
\geometry{letterpaper}
\usepackage[parfill]{parskip}
\usepackage{graphicx}

% Essential Packages
\usepackage{makeidx}
\makeindex
\usepackage{enumitem}
\usepackage[T1]{fontenc}
\usepackage{natbib}
\bibliographystyle{apalike}
\usepackage{ragged2e}
\usepackage{etoolbox}
\usepackage{amssymb}
\usepackage{fontawesome}
\usepackage{amsmath}
\usepackage{mathrsfs}
\usepackage{mathtools}
\usepackage{xparse}
\usepackage{tkz-euclide}
\usetkzobj{all}
\usepackage[utf8]{inputenc}
\usepackage{csquotes}
\usepackage[english]{babel}
\usepackage{marvosym}
\usepackage{pgf,tikz}
\usepackage{pgfplots}
\usepackage{fancyhdr}
\usepackage{array}
\usepackage{faktor}
\usepackage{float}
\usepackage{xcolor}
\usepackage{centernot}
\usepackage{silence}
  \WarningFilter*{latex}{Marginpar on page \thepage\space moved}
\usepackage{tcolorbox}
\tcbuselibrary{skins,breakable}
\usepackage{longtable}
\usepackage[amsmath,hyperref]{ntheorem}
\usepackage{hyperref}
\usepackage[noabbrev,capitalize,nameinlink]{cleveref}

% xcolor (scheme: base16 eighties)
\definecolor{base16-eighties-dark}{HTML}{2D2D2D}
\definecolor{base16-eighties-light}{HTML}{D3D0C8}
\definecolor{base16-eighties-magenta}{HTML}{CD98CD}
\definecolor{base16-eighties-red}{HTML}{F47678}
\definecolor{base16-eighties-yellow}{HTML}{E2B552}
\definecolor{base16-eighties-green}{HTML}{98CD97}
\definecolor{base16-eighties-lightblue}{HTML}{61CCCD}
\definecolor{base16-eighties-blue}{HTML}{6498CE}
\definecolor{base16-eighties-brown}{HTML}{D47B4E}
\definecolor{base16-eighties-gray}{HTML}{747369}

% hyperref Package Settings
\hypersetup{
    bookmarks=true,         % show bookmarks bar?
    unicode=true,          % non-Latin characters in Acrobat’s bookmarks
    pdftoolbar=false,        % show Acrobat’s toolbar?
    pdfmenubar=false,        % show Acrobat’s menu?
    pdffitwindow=true,     % window fit to page when opened
    colorlinks=true,
    allcolors=base16-eighties-magenta,
}

% tikz
\usepgfplotslibrary{polar}
\usepgflibrary{shapes.geometric}
\usetikzlibrary{angles,patterns,calc,decorations.markings}
\tikzset{midarrow/.style 2 args={
        decoration={markings,
            mark= at position #2 with {\arrow{#1}} ,
        },
        postaction={decorate}
    },
    midarrow/.default={latex}{0.5}
}
\def\centerarc[#1](#2)(#3:#4:#5)% Syntax: [draw options] (center) (initial angle:final angle:radius)
    { \draw[#1] ($(#2)+({#5*cos(#3)},{#5*sin(#3)})$) arc (#3:#4:#5); }

% enumitems
\newlist{inlinelist}{enumerate*}{1}
\setlist*[inlinelist,1]{%
  label=(\roman*),
}

% Theorem Style Customization
\setlength\theorempreskipamount{2ex}
\setlength\theorempostskipamount{3ex}

\makeatletter
\let\nobreakitem\item
\let\@nobreakitem\@item
\patchcmd{\nobreakitem}{\@item}{\@nobreakitem}{}{}
\patchcmd{\nobreakitem}{\@item}{\@nobreakitem}{}{}
\patchcmd{\@nobreakitem}{\@itempenalty}{\@M}{}{}
\patchcmd{\@xthm}{\ignorespaces}{\nobreak\ignorespaces}{}{}
\patchcmd{\@ythm}{\ignorespaces}{\nobreak\ignorespaces}{}{}

\renewtheoremstyle{break}%
  {\item{\theorem@headerfont
          ##1\ ##2\theorem@separator}\hskip\labelsep\relax\nobreakitem}%
  {\item{\theorem@headerfont
          ##1\ ##2\ (##3)\theorem@separator}\hskip\labelsep\relax\nobreakitem}
\makeatother

% ntheorem + framed
\makeatletter

% ntheorem Declarations
\theorempreskip{10pt}
\theorempostskip{5pt}
\theoremstyle{break}

\newtheorem*{solution}{\faPencil $\enspace$ Solution}
\newtheorem*{remark}{Remark}
\newtheorem{eg}{Example}[section]
\newtheorem{ex}{Exercise}[section]

    % definition env
\theoremprework{\textcolor{base16-eighties-blue}{\hrule height 2pt}}
\theoremheaderfont{\color{base16-eighties-blue}\normalfont\bfseries}
\theorempostwork{\textcolor{base16-eighties-blue}{\hrule height 2pt}}
\theoremindent10pt
\newtheorem{defn}{\faBook \enspace Definition}

    % definition env no num
\theoremprework{\textcolor{base16-eighties-blue}{\hrule height 2pt}}
\theoremheaderfont{\color{base16-eighties-blue}\normalfont\bfseries}
\theorempostwork{\textcolor{base16-eighties-blue}{\hrule height 2pt}}
\theoremindent10pt
\newtheorem*{defnnonum}{\faBook \enspace Definition}

    % theorem envs
\theoremprework{\textcolor{base16-eighties-magenta}{\hrule height 2pt}}
\theoremheaderfont{\color{base16-eighties-magenta}\normalfont\bfseries}
\theorempostwork{\textcolor{base16-eighties-magenta}{\hrule height 2pt}}
\theoremindent10pt
\newtheorem{thm}{\faCoffee \enspace Theorem}

\theoremprework{\textcolor{base16-eighties-magenta}{\hrule height 2pt}}
\theorempostwork{\textcolor{base16-eighties-magenta}{\hrule height 2pt}}
\theoremindent10pt
\newtheorem{propo}[thm]{\faTint \enspace Proposition}

\theoremprework{\textcolor{base16-eighties-magenta}{\hrule height 2pt}}
\theorempostwork{\textcolor{base16-eighties-magenta}{\hrule height 2pt}}
\theoremindent10pt
\newtheorem{crly}[thm]{\faSpaceShuttle \enspace Corollary}

\theoremprework{\textcolor{base16-eighties-magenta}{\hrule height 2pt}}
\theorempostwork{\textcolor{base16-eighties-magenta}{\hrule height 2pt}}
\theoremindent10pt
\newtheorem{lemma}[thm]{\faTree \enspace Lemma}

\theoremprework{\textcolor{base16-eighties-magenta}{\hrule height 2pt}}
\theorempostwork{\textcolor{base16-eighties-magenta}{\hrule height 2pt}}
\theoremindent10pt
\newtheorem{axiom}[thm]{\faShield \enspace Axiom}

    % theorem envs without counter
\theoremprework{\textcolor{base16-eighties-magenta}{\hrule height 2pt}}
\theoremheaderfont{\color{base16-eighties-magenta}\normalfont\bfseries}
\theorempostwork{\textcolor{base16-eighties-magenta}{\hrule height 2pt}}
\theoremindent10pt
\newtheorem*{thmnonum}{\faCoffee \enspace Theorem}

\theoremprework{\textcolor{base16-eighties-magenta}{\hrule height 2pt}}
\theorempostwork{\textcolor{base16-eighties-magenta}{\hrule height 2pt}}
\theoremindent10pt
\newtheorem*{propononum}{\faTint \enspace Proposition}

\theoremprework{\textcolor{base16-eighties-magenta}{\hrule height 2pt}}
\theorempostwork{\textcolor{base16-eighties-magenta}{\hrule height 2pt}}
\theoremindent10pt
\newtheorem*{crlynonum}{\faSpaceShuttle \enspace Corollary}

\theoremprework{\textcolor{base16-eighties-magenta}{\hrule height 2pt}}
\theorempostwork{\textcolor{base16-eighties-magenta}{\hrule height 2pt}}
\theoremindent10pt
\newtheorem*{lemmanonum}{\faTree \enspace Lemma}

\theoremprework{\textcolor{base16-eighties-magenta}{\hrule height 2pt}}
\theorempostwork{\textcolor{base16-eighties-magenta}{\hrule height 2pt}}
\theoremindent10pt
\newtheorem*{axiomnonum}{\faShield \enspace Axiom}

    % proof env
\theoremprework{\textcolor{base16-eighties-brown}{\hrule height 2pt}}
\theoremheaderfont{\color{base16-eighties-brown}\normalfont\bfseries}
\theorempostwork{\textcolor{base16-eighties-brown}{\hrule height 2pt}}
\newtheorem*{proof}{\faPencil \enspace Proof}

    % note and notation env
\theoremprework{\textcolor{base16-eighties-yellow}{\hrule height 2pt}}
\theoremheaderfont{\color{base16-eighties-yellow}\normalfont\bfseries}
\theorempostwork{\textcolor{base16-eighties-yellow}{\hrule height 2pt}}
\newtheorem*{note}{\faQuoteLeft \enspace Note}

\theoremprework{\textcolor{base16-eighties-yellow}{\hrule height 2pt}}
\theorempostwork{\textcolor{base16-eighties-yellow}{\hrule height 2pt}}
\newtheorem*{notation}{\faPaw \enspace Notation}

    % warning env
\theoremprework{\textcolor{base16-eighties-red}{\hrule height 2pt}}
\theoremheaderfont{\color{base16-eighties-red}\normalfont\bfseries}
\theorempostwork{\textcolor{base16-eighties-red}{\hrule height 2pt}}
\theoremindent10pt
\newtheorem*{warning}{\faBug \enspace Warning}

% more environments
\newtcolorbox{redquote}{
  blanker,enhanced,breakable,standard jigsaw,
  opacityback=0,
  coltext=base16-eighties-light,
  left=5mm,right=5mm,top=2mm,bottom=2mm,
  colframe=base16-eighties-red,
  boxrule=0pt,leftrule=3pt,
  fontupper=\itshape
}
\newtcolorbox{bluequote}{
  blanker,enhanced,breakable,standard jigsaw,
  opacityback=0,
  coltext=base16-eighties-light,
  left=5mm,right=5mm,top=2mm,bottom=2mm,
  colframe=base16-eighties-blue,
  boxrule=0pt,leftrule=3pt,
  fontupper=\itshape
}
\newtcolorbox{greenquote}{
  blanker,enhanced,breakable,standard jigsaw,
  opacityback=0,
  coltext=base16-eighties-light,
  left=5mm,right=5mm,top=2mm,bottom=2mm,
  colframe=base16-eighties-green,
  boxrule=0pt,leftrule=3pt,
  fontupper=\itshape
}
\newtcolorbox{yellowquote}{
  blanker,enhanced,breakable,standard jigsaw,
  opacityback=0,
  coltext=base16-eighties-light,
  left=5mm,right=5mm,top=2mm,bottom=2mm,
  colframe=base16-eighties-yellow,
  boxrule=0pt,leftrule=3pt,
  fontupper=\itshape
}
\newtcolorbox{magentaquote}{
  blanker,enhanced,breakable,standard jigsaw,
  opacityback=0,
  coltext=base16-eighties-light,
  left=5mm,right=5mm,top=2mm,bottom=2mm,
  colframe=base16-eighties-magenta,
  boxrule=0pt,leftrule=3pt,
  fontupper=\itshape
}

% ntheorem listtheorem style
\makeatother
\newlength\widesttheorem
\AtBeginDocument{
  \settowidth{\widesttheorem}{Proposition A.1.1.1\quad}
}

\makeatletter
\def\thm@@thmline@name#1#2#3#4{%
        \@dottedtocline{-2}{0em}{2.3em}%
                   {\makebox[\widesttheorem][l]{#1 \protect\numberline{#2}}#3}%
                   {#4}}
\@ifpackageloaded{hyperref}{
\def\thm@@thmline@name#1#2#3#4#5{%
    \ifx\#5\%
        \@dottedtocline{-2}{0em}{2.3em}%
            {\makebox[\widesttheorem][l]{#1 \protect\numberline{#2}}#3}%
            {#4}
    \else
        \ifHy@linktocpage\relax\relax
            \@dottedtocline{-2}{0em}{2.3em}%
                {\makebox[\widesttheorem][l]{#1 \protect\numberline{#2}}#3}%
                {\hyper@linkstart{link}{#5}{#4}\hyper@linkend}%
        \else
            \@dottedtocline{-2}{0em}{2.3em}%
                {\hyper@linkstart{link}{#5}%
                  {\makebox[\widesttheorem][l]{#1 \protect\numberline{#2}}#3}\hyper@linkend}%
                    {#4}%
        \fi
    \fi}
}

\makeatletter
\def\thm@@thmline@noname#1#2#3#4{%
        \@dottedtocline{-2}{0em}{5em}%
                   {{\protect\numberline{#2}}#3}%
                   {#4}}
\@ifpackageloaded{hyperref}{
\def\thm@@thmline@noname#1#2#3#4#5{%
    \ifx\#5\%
        \@dottedtocline{-2}{0em}{5em}%
            {{\protect\numberline{#2}}#3}%
            {#4}
    \else
        \ifHy@linktocpage\relax\relax
            \@dottedtocline{-2}{0em}{5em}%
                {{\protect\numberline{#2}}#3}%
                {\hyper@linkstart{link}{#5}{#4}\hyper@linkend}%
        \else
            \@dottedtocline{-2}{0em}{5em}%
                {\hyper@linkstart{link}{#5}%
                  {{\protect\numberline{#2}}#3}\hyper@linkend}%
                    {#4}%
        \fi
    \fi}
}

\theoremlisttype{allname}

\AtBeginDocument{\renewcommand\contentsname{Table of Contents}}

% Heading formattings
% chapter format
\titleformat{\chapter}%
  {\huge\rmfamily\itshape\color{base16-eighties-magenta}}% format applied to label+text
  {\llap{\colorbox{base16-eighties-magenta}{\parbox{1.5cm}{\hfill\itshape\huge\textcolor{base16-eighties-dark}{\thechapter}}}}}% label
  {5pt}% horizontal separation between label and title body
  {}% before the title body
  []% after the title body

% section format
\titleformat{\section}%
  {\normalfont\Large\rmfamily\itshape\color{base16-eighties-blue}}% format applied to label+text
  {\llap{\colorbox{base16-eighties-blue}{\parbox{1.5cm}{\hfill\itshape\textcolor{base16-eighties-dark}{\thesection}}}}}% label
  {5pt}% horizontal separation between label and title body
  {}% before the title body
  []% after the title body

% subsection format
\titleformat{\subsection}%
  {\normalfont\large\itshape\color{base16-eighties-green}}% format applied to label+text
  {\llap{\colorbox{base16-eighties-green}{\parbox{1.5cm}{\hfill\textcolor{base16-eighties-dark}{\thesubsection}}}}}% label
  {1em}% horizontal separation between label and title body
  {}% before the title body
  []% after the title body

% Sidenote enhancements
\def\mathmarginnote#1{%
  \tag*{\rlap{\hspace\marginparsep\smash{\parbox[t]{\marginparwidth}{%
  \footnotesize#1}}}}
}

% Custom table columning
\newcolumntype{L}[1]{>{\raggedright\let\newline\\\arraybackslash\hspace{0pt}}m{#1}}
\newcolumntype{C}[1]{>{\centering\let\newline\\\arraybackslash\hspace{0pt}}m{#1}}
\newcolumntype{R}[1]{>{\raggedleft\let\newline\\\arraybackslash\hspace{0pt}}m{#1}}

% Custom math operator
% \DeclareMathOperator{\rem}{rem}
\DeclareMathOperator*{\argmax}{arg\,max}
\DeclareMathOperator*{\argmin}{arg\,min}
\DeclareMathOperator{\re}{Re}
\DeclareMathOperator{\im}{Im}
\DeclareMathOperator{\caparg}{Arg}
\DeclareMathOperator{\Ind}{Ind}
\DeclareMathOperator{\Res}{Res}

% Graph styles
\pgfplotsset{compat=1.15}
\usepgfplotslibrary{fillbetween}
\pgfplotsset{four quads/.append style={axis x line=middle, axis y line=
middle, xlabel={$x$}, ylabel={$y$}, axis equal }}
\pgfplotsset{four quad complex/.append style={axis x line=middle, axis y line=
middle, xlabel={$\re$}, ylabel={$\im$}, axis equal }}

% Shortcuts
\newcommand{\floor}[1]{\lfloor #1 \rfloor}      % simplifying the writing of a floor function
\newcommand{\ceiling}[1]{\lceil #1 \rceil}      % simplifying the writing of a ceiling function
\newcommand{\dotp}{\, \cdotp}			        % dot product to distinguish from \cdot
\newcommand{\qed}{\hfill\ensuremath{\square}}   % Q.E.D sign
\newcommand{\abs}[1]{\left|#1\right|}						% absolute value
\newcommand{\lra}[1]{\langle \; #1 \; \rangle}
\newcommand{\at}[2]{\Big|_{#1}^{#2}}
\newcommand{\Arg}[1]{\caparg #1}
\renewcommand{\bar}[1]{\mkern 1.5mu \overline{\mkern -1.5mu #1 \mkern -1.5mu} \mkern 1.5mu}
\newcommand{\quotient}[2]{\faktor{#1}{#2}}
\newcommand{\cyclic}[1]{\left\langle #1 \right\rangle}
	% highlighting shortcuts
\newcommand{\hlimpo}[1]{\textcolor{base16-eighties-red}{\textbf{#1}}}
\newcommand{\hlwarn}[1]{\textcolor{base16-eighties-yellow}{\textbf{#1}}}
\newcommand{\hldefn}[1]{\textcolor{base16-eighties-blue}{\index{#1}\textbf{#1}}}
\newcommand{\hlnotea}[1]{\textcolor{base16-eighties-green}{\textbf{#1}}}
\newcommand{\hlnoteb}[1]{\textcolor{base16-eighties-lightblue}{\textbf{#1}}}
\newcommand{\hlnotec}[1]{\textcolor{base16-eighties-brown}{\textbf{#1}}}
\newcommand{\WTP}{\textcolor{base16-eighties-brown}{WTP} }
\newcommand{\WTS}{\textcolor{base16-eighties-brown}{WTS} }
\newcommand{\ind}[2]{\Ind_{#2}\left( #1 \right)}
\newcommand{\notimply}{\centernot\implies}
\newcommand{\res}[2]{\underset{#2}{\Res} #1 }
\newcommand{\tworow}[3]{\begin{tabular}{@{}#1@{}} #2 \\ #3 \end{tabular}}
\renewcommand{\epsilon}{\varepsilon}
\newcommand{\lrarrow}{\leftrightarrow}
\newcommand{\larrow}{\leftarrow}
\newcommand{\rarrow}{\rightarrow}
\renewcommand{\atop}[2]{\genfrac{}{}{0pt}{}{#1}{#2}}
\newcommand*\dif{\mathop{}\!d}

  % inspiration from: https://tex.stackexchange.com/questions/8720/overbrace-underbrace-but-with-an-arrow-instead#37758
\newcommand{\overarrow}[2]{
  \overset{\makebox[0pt]{\begin{tabular}{@{}c@{}}#2\\[0pt]\ensuremath{\uparrow}\end{tabular}}}{#1}
}
\newcommand{\underarrow}[2]{
  \underset{\makebox[0pt]{\begin{tabular}{@{}c@{}}\downarrow\\[0pt]\ensuremath{#2}\end{tabular}}}{#1}
}

% Document header formatting
\renewcommand{\chaptermark}[1]{\markboth{#1}{}}
\renewcommand{\sectionmark}[1]{\markright{#1}}
\makeatletter
\pagestyle{fancy}
\fancyhead{}
\fancyhead[RO]{\textsl{\@title} \enspace \thepage}
\fancyhead[LE]{\thepage \enspace \textsl{\leftmark \enspace - \enspace \rightmark}}
\makeatother

% Comment the two lines below if you want to print the document
\pagecolor{base16-eighties-dark}
\color{base16-eighties-light}


% Main Body
\title{PMATH352W18 - Complex Analysis - Topical Exercises}
\author{Johnson Ng}

\begin{document}
\hypersetup{pageanchor=false}
\maketitle
\hypersetup{pageanchor=true}
\tableofcontents

\chapter{Complex Numbers}
	\label{chapter:complex_numbers}

\section{Basic Algebraic Properties} % (fold)
\label{sec:basic_algebraic_properties}

\begin{enumerate}
	\item Verify that

	\begin{inlinelist}
		\item $(\sqrt{2} - i) - i (1 - \sqrt{2}i) = -2i$;
		\item $(2 - 3i)(-2 + i) = -1 + 8i$;
		\item $(3 + i)(3 - i)(\frac{1}{5} - \frac{1}{10} i) = 2 + i$.
	\end{inlinelist}
	\item Find the complex numbers which are complex conjugates of

	\begin{inlinelist}
		\item Their own squares;
		\item Their own cubes.
	\end{inlinelist}

	\item Caclulate the following quantities:

	\begin{inlinelist}
		\item $\frac{1 + i \tan \theta}{1 - i \tan \theta}$;
		\item $\frac{(1 + 2i)^3 - (1 - i)^3}{(3 + 2i)^3 - (2 + i)^2}$;
		\item $\frac{(1 - i)^5 - 1}{(1 + i)^5 + 1}$;
		\item $\frac{(1 + i)^9}{(1 - i)^7}$. 
	\end{inlinelist}

	\item Find the points $z = x + iy$ such that

	\begin{inlinelist}
		\item $\abs{z} \leq 2$;
		\item $\im z > 0$;
		\item $\re z \leq \frac{1}{2}$;
		\item $\re (z^2) = a$;
		\item $\abs{z^2 - 1} = a$;
		\item $\abs{\frac{z-1}{z + 1}} \leq 1$;
		\item $\abs{\frac{z - \alpha}{z - \beta}} = 1$.
	\end{inlinelist}

	\item Derive the identity
	\begin{equation*}
		\left( \frac{z_1}{z_3} \right)\left( \frac{z_2}{z_4} \right) = \frac{z_1 z_2}{z_3 z_4} \quad (z_3 \neq 0, z_4 \neq 0)
	\end{equation*}

	\item Using the above identity, derive the cancellation law
	\begin{equation*}
		\frac{z_1 z}{z_2 z} = \frac{z_1}{z_2} \quad (z_2 \neq 0, z \neq 0)
	\end{equation*}

	\item Using properties of moduli that has been introduced, show that when $\abs{z_3} \neq \abs{z_4}$,
	\begin{equation*}
		\frac{\re(z_1 + z_2)}{\abs{z_3 + z_4}} \leq \frac{\abs{z_1} + \abs{z_2}}{\abs{\abs{z_3} - \abs{z_4}}} 
	\end{equation*}

	\item Verify that $\sqrt{2}\abs{z} \geq \abs{\re z} + \abs{\im z}$. (Hint: Reduce this inequality to $(\abs{x} - \abs{y})^2 \geq 0$.)
\end{enumerate}

(Jump to \hyperref[sub:basic_algebraic_properties]{solutions})

% section basic_algebraic_properties (end)

\section{Polar Form} % (fold)
\label{sec:polar_form}

\begin{enumerate}
	\item Represent the following complex numbers in polar form:

	\begin{inlinelist}
		\item $1 + i$;
		\item $-1 + i$;
		\item $-1 - i$;
		\item $1 - i$;
		\item $ 1 + \sqrt{3} i$;
		\item $-1 + \sqrt{3} i$;
		\item $-1 - \sqrt{3} i$;
		\item $ 1 - \sqrt{3} i$;
		\item $2 + \sqrt{3} + i$.
	\end{inlinelist}

	\item Generalize the Triangle Inequality.
	\item Prove the identity
	\begin{equation*}
		\abs{z_1 + z_2}^2 + \abs{z_1 - z_2}^2 = 2(\abs{z_1}^2 + \abs{z_2}^2),
	\end{equation*}
	for arbitrary complex numbers $z_1, z_2, ..., z_n$.

	\item When do three points $z_1, z_2, z_3 \in \mathbb{C}$ lie on a straight line in the complex plane?

	\item Let $\sigma$ be the line segment joining two points $z_1$ and $z_2$. Find the point $z$ dividign $\sigma$ in the ratio $\lambda_1 : \lambda_2$.

	\item Four points $z_1, z_2, z_3, z_4$ satisfy the conditions
	\begin{equation*}
		z_1 + z_2 + z_3 + z_4 = 0, \quad \abs{z_1} = \abs{z_2} = \abs{z_3} = \abs{z_4} = 1.
	\end{equation*}
	Show that the points either lie ast the vertices of a square inscribed in the unit circle or else coincide in pairs.

	\item Calculate the following quantities:

	\begin{inlinelist}
		\item $(1 + i)^{25}$;
		\item $\left(\frac{1 + \sqrt{3} i}{1 - i} \right)^{30}$;
		\item $\left( 1 - \frac{\sqrt{3 - i}}{2} \right)^24$;
		\item $\frac{(-1 + \sqrt{3}i)^{15}}{(1 - i)^{30}} + \frac{(-1-\sqrt{3}i)^{15}}{(1 + i)^{20}}$.
	\end{inlinelist}

	\item Use De Moivre's theorem to express $\cos nx$ and $\sim nx$ in terms of powers of $\cos x$ and $\sin x$.

	\item Express $\tan 6x$ in terms of $\tan x$.

	\item Write $\sqrt{1 + i}$ in polar form.
\end{enumerate}

% section polar_form (end)

\section{Roots of a Complex Number} % (fold)
\label{sec:roots_of_a_complex_number}

\begin{enumerate}
	\item Find all the values of the following roots:

	\begin{inlinelist}
		\item $\sqrt[3]{1}$;
		\item $\sqrt[3]{i}$;
		\item $\sqrt[4]{-1}$;
		\item $\sqrt[6]{-8}$;
		\item $\sqrt[8]{1}$;
		\item $\sqrt{3 + 4i}$;
		\item $\sqrt[3]{-2 + 2i}$;
		\item $\sqrt[5]{-4 + 3i}$;
		\item $\sqrt[6]{\frac{1 - i}{\sqrt{3} + i}}$;
		\item $\sqrt[8]{\frac{1 + i}{\sqrt{3} - i}}$;
	\end{inlinelist}

	\item Prove that the sum of all the distinct $n$th roots of unity is zero. What geometric fact does this express?

	\item Let $\epsilon$ be any $n$th root of unity other than 1. prove that
	\begin{equation*}
		1 + 2\epsilon + 3 \epsilon^2 + \hdots + n\epsilon^{n - 1} = \frac{n}{\epsilon - 1} 
	\end{equation*}

	\item Prove that every complex number $\alpha \neq -1$ of unit modulus can be represented in the form
	\begin{equation*}
		\alpha = \frac{1 + it}{1 - it},
	\end{equation*}
	where $t \in \mathbb{R}$.
\end{enumerate}

% section roots_of_a_complex_number (end)

% chapter complex_numbers (end)

\chapter{Complex Functions}
	\label{chapter:complex_functions}

\section{Limits and Continuity} % (fold)
\label{sec:limits_and_continuity}

\begin{enumerate}
	\item Let $f : \Omega \subseteq \mathbb{C} \to \mathbb{C}$, $z_0 \in \Omega$. $\forall z \in \Omega$, prove that
	\begin{equation*}
		(z \to z_0 \implies f(z) \to \infty) \iff (z \to z_0 \implies \psi(z) = \frac{1}{f(z)} \to 0)
	\end{equation*}

	\item The \hlnotea{Cauchy Convergence Criterion for sequences} states that a complex sequence $z_n$ is convergent iff
	\begin{gather*}
		\forall \epsilon > 0 \; \exists N = N(\epsilon) > 0 \; \forall m, n > N \\
		\abs{z_m - z_n} < \epsilon.
	\end{gather*}
	Prove the generalization of the criterion: The function $f(z)$ approaches a limit as $z \to z_0$ iff
	\begin{gather*}
		\forall \epsilon > 0 \; \exists \delta = \delta(\epsilon) > 0 \\
		(0 < \abs{z' - z_0} < \delta \; \land \; 0 < \abs{z'' - z} < \delta) \implies \abs{f(z') - f(z'')} < \epsilon
	\end{gather*}

	\item Let $f(z)$ be a rational function, i.e., a ratio
	\begin{equation}\label{eq:1}
		f(z) = \frac{a_0 + a_1 z + \hdots a_m z^m}{b_0 + b_1z + \hdots + b_n z^n} \enspace (a_m \neq 0, b_n \neq 0)
	\end{equation}
	of two polynomials. Discuss the possible values of $\lim_{z \to \infty} f(z)$.

	\item Where is the function \cref{eq:1} continuous?

	\item Prove that if $f(z)$ is continuous in a region $\Omega$, then so is $\abs{f(z)}$.

	\item Is the function
	\begin{equation*}
		f(zz) = \frac{1}{1 - z}
	\end{equation*}
	continuous in the open disk $\abs{z} < 1$?
\end{enumerate}

% section limits_and_continuity (end)

% chapter complex_functions (end)

\chapter*{Answers}\label{chapter:answers}
\addcontentsline{toc}{chapter}{Answers}

\fontsize{9}{11}\selectfont

\section*{Chapter 1} % (fold)
\label{sec:chapter_1}

\subsection*{Basic Algebraic Properties} % (fold)
\label{sub:basic_algebraic_properties}

(Jump to: \cref{sec:basic_algebraic_properties})

\begin{enumerate}
	\item 
	\item \begin{enumerate}
		\item $(0, 0), (1, 0), (-\frac{1}{2} , \pm \sqrt{\frac{3}{4}})$
		\item $(0, 0), (1, 0), (-1, 0), (0, i), (0, -i)$
	\end{enumerate}
	\item \begin{enumerate}
		\item $(\cos \theta - i \sin \theta)^2$
		\item 
		\item $\frac{9 - 40i}{41}$
		\item $2$
	\end{enumerate}
	\item \begin{enumerate}
		\item $\{(x, y) : x^2 + y^2 \leq 4\}$
		\item $\{(x, y) : y > 0\}$
		\item $\{(x, y) : x \leq \frac{1}{2} \}$
		\item $x^2 - y^2 = a$. Also, refer to this graph on Desmos: \url{https://www.desmos.com/calculator/buwtyobjrn}
		\item Refer to graph on Desmos: \url{https://www.desmos.com/calculator/a3pnbwueja}
	\end{enumerate}
	\item \begin{align*}
		\left( \frac{z_1}{z_3} \right) \left( \frac{z_2}{z_4} \right)
			&= z_1 \left( \frac{\bar{z_3}}{\abs{z_3}^2} \right) z_2 \left( \frac{\bar{z_4}}{\abs{z_4}^2} \right) \quad \text{since } z\bar{z} = \abs{z}^2 \\
			&= z_1 z_2 \left( \frac{\bar{z_3} \bar{z_4}}{\abs{z_3}^2 \abs{z_4}^2} \right) \\
			&= z_1 z_2 \left( \frac{\bar{z_3 z_4}}{\abs{z_3 z_4}^2} \right) \quad \text{since } \bar{z}\bar{w} = \bar{zw} \text{ and } \abs{z}\abs{w} = \abs{zw} \\
			&= \frac{z_1 z_2}{z_3 z_4} \quad \text{since } z\bar{z} = \abs{z}^2
	\end{align*}
	\item $\frac{z_1 z}{z_2 z} = \left( \frac{z_1}{z_2} \right) \left(\frac{z}{z} \right) = \frac{z_1}{z_2}$
	\item By the Triangle Inequality,
	\begin{equation*}
		\abs{z_1} + \abs{z_2} \geq \abs{z_1 + z_2} \geq \sqrt{\re(z_1 + z_2)^2 + \im(z_1 + z_2)^2} \geq \sqrt{\re(z_1 + z_2)} = \re(z_1 + z_2)
	\end{equation*}
	and by the Reversed Triangle Inequality,
	\begin{gather*}
		\abs{z_3 + z_4} \geq \abs{\abs{z_3} - \abs{z_4}} \\
		\implies \frac{1}{\abs{z_3 + z_4}} \leq \frac{1}{\abs{\abs{z_3} - \abs{z_4}}} 
	\end{gather*}
	Thus
	\begin{equation*}
		\frac{\re(z_1 + z_2)}{\abs{z_3 + z_4}} \leq \frac{\abs{z_1} + \abs{z_2}}{\abs{z_3 + z_4}} \leq \frac{\abs{z_1} + \abs{z_2}}{\abs{\abs{z_3} - \abs{z_4}}}
	\end{equation*}
\end{enumerate}

% subsection basic_algebraic_properties (end)

% section chapter_1 (end)

% chapter answers (end)

\end{document}