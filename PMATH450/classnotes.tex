% !TEX TS-program = pdflatex
\documentclass[notoc,notitlepage]{tufte-book}
% \nonstopmode % uncomment to enable nonstopmode

\usepackage{classnotetitle}

\title{PMATH450 --- Lesbesgue Integration and Fourier Analysis}
\author{Johnson Ng}
\subtitle{Classnotes for Spring 2019}
\credentials{BMath (Hons), Pure Mathematics major, Actuarial Science Minor}
\institution{University of Waterloo}

\setcounter{secnumdepth}{3}
\setcounter{tocdepth}{3}

\renewcommand{\baselinestretch}{1.1}
\usepackage{geometry}
\geometry{letterpaper}
\usepackage[parfill]{parskip}
\usepackage{graphicx}

% Essential Packages
\usepackage{makeidx}
\makeindex
\usepackage{enumitem}
\usepackage[T1]{fontenc}
\usepackage{natbib}
\bibliographystyle{apalike}
\usepackage{ragged2e}
\usepackage{etoolbox}
\usepackage{amssymb}
\usepackage{fontawesome}
\usepackage{amsmath}
\usepackage{mathrsfs}
\usepackage{mathtools}
\usepackage{xparse}
\usepackage{tkz-euclide}
\usetkzobj{all}
\usepackage[utf8]{inputenc}
\usepackage{csquotes}
\usepackage[english]{babel}
\usepackage{marvosym}
\usepackage{pgf,tikz}
\usepackage{pgfplots}
\usepackage{fancyhdr}
\usepackage{array}
\usepackage{faktor}
\usepackage{float}
\usepackage{xcolor}
\usepackage{centernot}
\usepackage{silence}
  \WarningFilter*{latex}{Marginpar on page \thepage\space moved}
\usepackage{tcolorbox}
\tcbuselibrary{skins,breakable}
\usepackage{longtable}
\usepackage[amsmath,hyperref]{ntheorem}
\usepackage{hyperref}
\usepackage[noabbrev,capitalize,nameinlink]{cleveref}

% xcolor (scheme: base16 eighties)
\definecolor{base16-eighties-dark}{HTML}{2D2D2D}
\definecolor{base16-eighties-light}{HTML}{D3D0C8}
\definecolor{base16-eighties-magenta}{HTML}{CD98CD}
\definecolor{base16-eighties-red}{HTML}{F47678}
\definecolor{base16-eighties-yellow}{HTML}{E2B552}
\definecolor{base16-eighties-green}{HTML}{98CD97}
\definecolor{base16-eighties-lightblue}{HTML}{61CCCD}
\definecolor{base16-eighties-blue}{HTML}{6498CE}
\definecolor{base16-eighties-brown}{HTML}{D47B4E}
\definecolor{base16-eighties-gray}{HTML}{747369}

% hyperref Package Settings
\hypersetup{
    bookmarks=true,         % show bookmarks bar?
    unicode=true,          % non-Latin characters in Acrobat’s bookmarks
    pdftoolbar=false,        % show Acrobat’s toolbar?
    pdfmenubar=false,        % show Acrobat’s menu?
    pdffitwindow=true,     % window fit to page when opened
    colorlinks=true,
    allcolors=base16-eighties-magenta,
}

% tikz
\usepgfplotslibrary{polar}
\usepgflibrary{shapes.geometric}
\usetikzlibrary{angles,patterns,calc,decorations.markings}
\tikzset{midarrow/.style 2 args={
        decoration={markings,
            mark= at position #2 with {\arrow{#1}} ,
        },
        postaction={decorate}
    },
    midarrow/.default={latex}{0.5}
}
\def\centerarc[#1](#2)(#3:#4:#5)% Syntax: [draw options] (center) (initial angle:final angle:radius)
    { \draw[#1] ($(#2)+({#5*cos(#3)},{#5*sin(#3)})$) arc (#3:#4:#5); }

% enumitems
\newlist{inlinelist}{enumerate*}{1}
\setlist*[inlinelist,1]{%
  label=(\roman*),
}

% Theorem Style Customization
\setlength\theorempreskipamount{2ex}
\setlength\theorempostskipamount{3ex}

\makeatletter
\let\nobreakitem\item
\let\@nobreakitem\@item
\patchcmd{\nobreakitem}{\@item}{\@nobreakitem}{}{}
\patchcmd{\nobreakitem}{\@item}{\@nobreakitem}{}{}
\patchcmd{\@nobreakitem}{\@itempenalty}{\@M}{}{}
\patchcmd{\@xthm}{\ignorespaces}{\nobreak\ignorespaces}{}{}
\patchcmd{\@ythm}{\ignorespaces}{\nobreak\ignorespaces}{}{}

\renewtheoremstyle{break}%
  {\item{\theorem@headerfont
          ##1\ ##2\theorem@separator}\hskip\labelsep\relax\nobreakitem}%
  {\item{\theorem@headerfont
          ##1\ ##2\ (##3)\theorem@separator}\hskip\labelsep\relax\nobreakitem}
\makeatother

% ntheorem + framed
\makeatletter

% ntheorem Declarations
\theorempreskip{10pt}
\theorempostskip{5pt}
\theoremstyle{break}

\newtheorem*{solution}{\faPencil $\enspace$ Solution}
\newtheorem*{remark}{Remark}
\newtheorem{eg}{Example}[section]
\newtheorem{ex}{Exercise}[section]

    % definition env
\theoremprework{\textcolor{base16-eighties-blue}{\hrule height 2pt}}
\theoremheaderfont{\color{base16-eighties-blue}\normalfont\bfseries}
\theorempostwork{\textcolor{base16-eighties-blue}{\hrule height 2pt}}
\theoremindent10pt
\newtheorem{defn}{\faBook \enspace Definition}

    % definition env no num
\theoremprework{\textcolor{base16-eighties-blue}{\hrule height 2pt}}
\theoremheaderfont{\color{base16-eighties-blue}\normalfont\bfseries}
\theorempostwork{\textcolor{base16-eighties-blue}{\hrule height 2pt}}
\theoremindent10pt
\newtheorem*{defnnonum}{\faBook \enspace Definition}

    % theorem envs
\theoremprework{\textcolor{base16-eighties-magenta}{\hrule height 2pt}}
\theoremheaderfont{\color{base16-eighties-magenta}\normalfont\bfseries}
\theorempostwork{\textcolor{base16-eighties-magenta}{\hrule height 2pt}}
\theoremindent10pt
\newtheorem{thm}{\faCoffee \enspace Theorem}

\theoremprework{\textcolor{base16-eighties-magenta}{\hrule height 2pt}}
\theorempostwork{\textcolor{base16-eighties-magenta}{\hrule height 2pt}}
\theoremindent10pt
\newtheorem{propo}[thm]{\faTint \enspace Proposition}

\theoremprework{\textcolor{base16-eighties-magenta}{\hrule height 2pt}}
\theorempostwork{\textcolor{base16-eighties-magenta}{\hrule height 2pt}}
\theoremindent10pt
\newtheorem{crly}[thm]{\faSpaceShuttle \enspace Corollary}

\theoremprework{\textcolor{base16-eighties-magenta}{\hrule height 2pt}}
\theorempostwork{\textcolor{base16-eighties-magenta}{\hrule height 2pt}}
\theoremindent10pt
\newtheorem{lemma}[thm]{\faTree \enspace Lemma}

\theoremprework{\textcolor{base16-eighties-magenta}{\hrule height 2pt}}
\theorempostwork{\textcolor{base16-eighties-magenta}{\hrule height 2pt}}
\theoremindent10pt
\newtheorem{axiom}[thm]{\faShield \enspace Axiom}

    % theorem envs without counter
\theoremprework{\textcolor{base16-eighties-magenta}{\hrule height 2pt}}
\theoremheaderfont{\color{base16-eighties-magenta}\normalfont\bfseries}
\theorempostwork{\textcolor{base16-eighties-magenta}{\hrule height 2pt}}
\theoremindent10pt
\newtheorem*{thmnonum}{\faCoffee \enspace Theorem}

\theoremprework{\textcolor{base16-eighties-magenta}{\hrule height 2pt}}
\theorempostwork{\textcolor{base16-eighties-magenta}{\hrule height 2pt}}
\theoremindent10pt
\newtheorem*{propononum}{\faTint \enspace Proposition}

\theoremprework{\textcolor{base16-eighties-magenta}{\hrule height 2pt}}
\theorempostwork{\textcolor{base16-eighties-magenta}{\hrule height 2pt}}
\theoremindent10pt
\newtheorem*{crlynonum}{\faSpaceShuttle \enspace Corollary}

\theoremprework{\textcolor{base16-eighties-magenta}{\hrule height 2pt}}
\theorempostwork{\textcolor{base16-eighties-magenta}{\hrule height 2pt}}
\theoremindent10pt
\newtheorem*{lemmanonum}{\faTree \enspace Lemma}

\theoremprework{\textcolor{base16-eighties-magenta}{\hrule height 2pt}}
\theorempostwork{\textcolor{base16-eighties-magenta}{\hrule height 2pt}}
\theoremindent10pt
\newtheorem*{axiomnonum}{\faShield \enspace Axiom}

    % proof env
\theoremprework{\textcolor{base16-eighties-brown}{\hrule height 2pt}}
\theoremheaderfont{\color{base16-eighties-brown}\normalfont\bfseries}
\theorempostwork{\textcolor{base16-eighties-brown}{\hrule height 2pt}}
\newtheorem*{proof}{\faPencil \enspace Proof}

    % note and notation env
\theoremprework{\textcolor{base16-eighties-yellow}{\hrule height 2pt}}
\theoremheaderfont{\color{base16-eighties-yellow}\normalfont\bfseries}
\theorempostwork{\textcolor{base16-eighties-yellow}{\hrule height 2pt}}
\newtheorem*{note}{\faQuoteLeft \enspace Note}

\theoremprework{\textcolor{base16-eighties-yellow}{\hrule height 2pt}}
\theorempostwork{\textcolor{base16-eighties-yellow}{\hrule height 2pt}}
\newtheorem*{notation}{\faPaw \enspace Notation}

    % warning env
\theoremprework{\textcolor{base16-eighties-red}{\hrule height 2pt}}
\theoremheaderfont{\color{base16-eighties-red}\normalfont\bfseries}
\theorempostwork{\textcolor{base16-eighties-red}{\hrule height 2pt}}
\theoremindent10pt
\newtheorem*{warning}{\faBug \enspace Warning}

% more environments
\newtcolorbox{redquote}{
  blanker,enhanced,breakable,standard jigsaw,
  opacityback=0,
  coltext=base16-eighties-light,
  left=5mm,right=5mm,top=2mm,bottom=2mm,
  colframe=base16-eighties-red,
  boxrule=0pt,leftrule=3pt,
  fontupper=\itshape
}
\newtcolorbox{bluequote}{
  blanker,enhanced,breakable,standard jigsaw,
  opacityback=0,
  coltext=base16-eighties-light,
  left=5mm,right=5mm,top=2mm,bottom=2mm,
  colframe=base16-eighties-blue,
  boxrule=0pt,leftrule=3pt,
  fontupper=\itshape
}
\newtcolorbox{greenquote}{
  blanker,enhanced,breakable,standard jigsaw,
  opacityback=0,
  coltext=base16-eighties-light,
  left=5mm,right=5mm,top=2mm,bottom=2mm,
  colframe=base16-eighties-green,
  boxrule=0pt,leftrule=3pt,
  fontupper=\itshape
}
\newtcolorbox{yellowquote}{
  blanker,enhanced,breakable,standard jigsaw,
  opacityback=0,
  coltext=base16-eighties-light,
  left=5mm,right=5mm,top=2mm,bottom=2mm,
  colframe=base16-eighties-yellow,
  boxrule=0pt,leftrule=3pt,
  fontupper=\itshape
}
\newtcolorbox{magentaquote}{
  blanker,enhanced,breakable,standard jigsaw,
  opacityback=0,
  coltext=base16-eighties-light,
  left=5mm,right=5mm,top=2mm,bottom=2mm,
  colframe=base16-eighties-magenta,
  boxrule=0pt,leftrule=3pt,
  fontupper=\itshape
}

% ntheorem listtheorem style
\makeatother
\newlength\widesttheorem
\AtBeginDocument{
  \settowidth{\widesttheorem}{Proposition A.1.1.1\quad}
}

\makeatletter
\def\thm@@thmline@name#1#2#3#4{%
        \@dottedtocline{-2}{0em}{2.3em}%
                   {\makebox[\widesttheorem][l]{#1 \protect\numberline{#2}}#3}%
                   {#4}}
\@ifpackageloaded{hyperref}{
\def\thm@@thmline@name#1#2#3#4#5{%
    \ifx\#5\%
        \@dottedtocline{-2}{0em}{2.3em}%
            {\makebox[\widesttheorem][l]{#1 \protect\numberline{#2}}#3}%
            {#4}
    \else
        \ifHy@linktocpage\relax\relax
            \@dottedtocline{-2}{0em}{2.3em}%
                {\makebox[\widesttheorem][l]{#1 \protect\numberline{#2}}#3}%
                {\hyper@linkstart{link}{#5}{#4}\hyper@linkend}%
        \else
            \@dottedtocline{-2}{0em}{2.3em}%
                {\hyper@linkstart{link}{#5}%
                  {\makebox[\widesttheorem][l]{#1 \protect\numberline{#2}}#3}\hyper@linkend}%
                    {#4}%
        \fi
    \fi}
}

\makeatletter
\def\thm@@thmline@noname#1#2#3#4{%
        \@dottedtocline{-2}{0em}{5em}%
                   {{\protect\numberline{#2}}#3}%
                   {#4}}
\@ifpackageloaded{hyperref}{
\def\thm@@thmline@noname#1#2#3#4#5{%
    \ifx\#5\%
        \@dottedtocline{-2}{0em}{5em}%
            {{\protect\numberline{#2}}#3}%
            {#4}
    \else
        \ifHy@linktocpage\relax\relax
            \@dottedtocline{-2}{0em}{5em}%
                {{\protect\numberline{#2}}#3}%
                {\hyper@linkstart{link}{#5}{#4}\hyper@linkend}%
        \else
            \@dottedtocline{-2}{0em}{5em}%
                {\hyper@linkstart{link}{#5}%
                  {{\protect\numberline{#2}}#3}\hyper@linkend}%
                    {#4}%
        \fi
    \fi}
}

\theoremlisttype{allname}

\AtBeginDocument{\renewcommand\contentsname{Table of Contents}}

% Heading formattings
% chapter format
\titleformat{\chapter}%
  {\huge\rmfamily\itshape\color{base16-eighties-magenta}}% format applied to label+text
  {\llap{\colorbox{base16-eighties-magenta}{\parbox{1.5cm}{\hfill\itshape\huge\textcolor{base16-eighties-dark}{\thechapter}}}}}% label
  {5pt}% horizontal separation between label and title body
  {}% before the title body
  []% after the title body

% section format
\titleformat{\section}%
  {\normalfont\Large\rmfamily\itshape\color{base16-eighties-blue}}% format applied to label+text
  {\llap{\colorbox{base16-eighties-blue}{\parbox{1.5cm}{\hfill\itshape\textcolor{base16-eighties-dark}{\thesection}}}}}% label
  {5pt}% horizontal separation between label and title body
  {}% before the title body
  []% after the title body

% subsection format
\titleformat{\subsection}%
  {\normalfont\large\itshape\color{base16-eighties-green}}% format applied to label+text
  {\llap{\colorbox{base16-eighties-green}{\parbox{1.5cm}{\hfill\textcolor{base16-eighties-dark}{\thesubsection}}}}}% label
  {1em}% horizontal separation between label and title body
  {}% before the title body
  []% after the title body

% Sidenote enhancements
\def\mathmarginnote#1{%
  \tag*{\rlap{\hspace\marginparsep\smash{\parbox[t]{\marginparwidth}{%
  \footnotesize#1}}}}
}

% Custom table columning
\newcolumntype{L}[1]{>{\raggedright\let\newline\\\arraybackslash\hspace{0pt}}m{#1}}
\newcolumntype{C}[1]{>{\centering\let\newline\\\arraybackslash\hspace{0pt}}m{#1}}
\newcolumntype{R}[1]{>{\raggedleft\let\newline\\\arraybackslash\hspace{0pt}}m{#1}}

% Custom math operator
% \DeclareMathOperator{\rem}{rem}
\DeclareMathOperator*{\argmax}{arg\,max}
\DeclareMathOperator*{\argmin}{arg\,min}
\DeclareMathOperator{\re}{Re}
\DeclareMathOperator{\im}{Im}
\DeclareMathOperator{\caparg}{Arg}
\DeclareMathOperator{\Ind}{Ind}
\DeclareMathOperator{\Res}{Res}

% Graph styles
\pgfplotsset{compat=1.15}
\usepgfplotslibrary{fillbetween}
\pgfplotsset{four quads/.append style={axis x line=middle, axis y line=
middle, xlabel={$x$}, ylabel={$y$}, axis equal }}
\pgfplotsset{four quad complex/.append style={axis x line=middle, axis y line=
middle, xlabel={$\re$}, ylabel={$\im$}, axis equal }}

% Shortcuts
\newcommand{\floor}[1]{\lfloor #1 \rfloor}      % simplifying the writing of a floor function
\newcommand{\ceiling}[1]{\lceil #1 \rceil}      % simplifying the writing of a ceiling function
\newcommand{\dotp}{\, \cdotp}			        % dot product to distinguish from \cdot
\newcommand{\qed}{\hfill\ensuremath{\square}}   % Q.E.D sign
\newcommand{\abs}[1]{\left|#1\right|}						% absolute value
\newcommand{\lra}[1]{\langle \; #1 \; \rangle}
\newcommand{\at}[2]{\Big|_{#1}^{#2}}
\newcommand{\Arg}[1]{\caparg #1}
\renewcommand{\bar}[1]{\mkern 1.5mu \overline{\mkern -1.5mu #1 \mkern -1.5mu} \mkern 1.5mu}
\newcommand{\quotient}[2]{\faktor{#1}{#2}}
\newcommand{\cyclic}[1]{\left\langle #1 \right\rangle}
	% highlighting shortcuts
\newcommand{\hlimpo}[1]{\textcolor{base16-eighties-red}{\textbf{#1}}}
\newcommand{\hlwarn}[1]{\textcolor{base16-eighties-yellow}{\textbf{#1}}}
\newcommand{\hldefn}[1]{\textcolor{base16-eighties-blue}{\index{#1}\textbf{#1}}}
\newcommand{\hlnotea}[1]{\textcolor{base16-eighties-green}{\textbf{#1}}}
\newcommand{\hlnoteb}[1]{\textcolor{base16-eighties-lightblue}{\textbf{#1}}}
\newcommand{\hlnotec}[1]{\textcolor{base16-eighties-brown}{\textbf{#1}}}
\newcommand{\WTP}{\textcolor{base16-eighties-brown}{WTP} }
\newcommand{\WTS}{\textcolor{base16-eighties-brown}{WTS} }
\newcommand{\ind}[2]{\Ind_{#2}\left( #1 \right)}
\newcommand{\notimply}{\centernot\implies}
\newcommand{\res}[2]{\underset{#2}{\Res} #1 }
\newcommand{\tworow}[3]{\begin{tabular}{@{}#1@{}} #2 \\ #3 \end{tabular}}
\renewcommand{\epsilon}{\varepsilon}
\newcommand{\lrarrow}{\leftrightarrow}
\newcommand{\larrow}{\leftarrow}
\newcommand{\rarrow}{\rightarrow}
\renewcommand{\atop}[2]{\genfrac{}{}{0pt}{}{#1}{#2}}
\newcommand*\dif{\mathop{}\!d}

  % inspiration from: https://tex.stackexchange.com/questions/8720/overbrace-underbrace-but-with-an-arrow-instead#37758
\newcommand{\overarrow}[2]{
  \overset{\makebox[0pt]{\begin{tabular}{@{}c@{}}#2\\[0pt]\ensuremath{\uparrow}\end{tabular}}}{#1}
}
\newcommand{\underarrow}[2]{
  \underset{\makebox[0pt]{\begin{tabular}{@{}c@{}}\downarrow\\[0pt]\ensuremath{#2}\end{tabular}}}{#1}
}

% Document header formatting
\renewcommand{\chaptermark}[1]{\markboth{#1}{}}
\renewcommand{\sectionmark}[1]{\markright{#1}}
\makeatletter
\pagestyle{fancy}
\fancyhead{}
\fancyhead[RO]{\textsl{\@title} \enspace \thepage}
\fancyhead[LE]{\thepage \enspace \textsl{\leftmark \enspace - \enspace \rightmark}}
\makeatother

% Comment the two lines below if you want to print the document
\pagecolor{base16-eighties-dark}
\color{base16-eighties-light}


\theoremprework{\textcolor{cyan}{\hrule height 2pt width \textwidth}}
\theoremheaderfont{\color{cyan}\normalfont\bfseries}
\theorempostwork{\textcolor{cyan}{\hrule height 2pt width \textwidth}}
\theoremindent10pt
\newtheorem*{culture}{Culture}

\begin{document}
\hypersetup{pageanchor=false}
\maketitle
\hypersetup{pageanchor=true}
\begin{fullwidth}
\tableofcontents
\end{fullwidth}

\newpage
\begin{fullwidth}
  \renewcommand{\listtheoremname}{\faBook\ \slshape List of Definitions}
  \listoftheorems[ignoreall,show={defn}]
  \addcontentsline{toc}{chapter}{List of Definitions}
\end{fullwidth}

\newpage 
\begin{fullwidth}
  \renewcommand{\listtheoremname}{\faCoffee\ \slshape List of Theorems}
  \listoftheorems[ignoreall,
    show={axiom,lemma,thm,crly,propo,marginthm,marginpropo,marginlemma,marginaxiom,margincrly}
  ]
  \addcontentsline{toc}{chapter}{List of Theorems}
\end{fullwidth}


\chapter*{Preface}%
\label{chp:preface}
\addcontentsline{toc}{chapter}{Preface}
% chapter preface

The pre-requisite to this course is Real Analysis. We will use a lot of the
concepts introduced in Real Analysis, at times without explicitly stating it.
Refer to \href{https://tex.japorized.ink/PMATH351F18/classnotes.pdf}{notes on
PMATH351}.

This course is spiritually broken into 2 pieces:
\begin{itemize}
  \item Lesbesgue Integration; and
  \item Fourier Analysis,
\end{itemize}
which is as the name of the course.

In this set of notes, we use a special topic environment called
\hlnotea{culture} to discuss interesting contents related to the course, but
will not be throughly studied and not tested on exams.

% chapter preface (end)

\chapter{Lecture 1 May 07th 2019}%
\label{chp:lecture_1_may_07th_2019}
% chapter lecture_1_may_07th_2019

Since many of our results work for both $\mathbb{C}$ and $\mathbb{R}$, we shall
use $\mathbb{K}$ throughout this course to represent either $\mathbb{C}$ or
$\mathbb{R}$.

\section{Riemannian Integration}%
\label{sec:riemannian_integration}
% section riemannian_integration

\begin{defn}[Norm and Semi-Norm]\index{Norm}\index{Semi-Norm}\label{defn:semi_norm}
  Let $V$ be a vector space over $\mathbb{K}$. We define a
  \hlnoteb{semi-norm} on $V$ as a function
  \begin{equation*}
    \nu : V \to \mathbb{R}
  \end{equation*}
  that satisfies
  \begin{enumerate}
    \item (\hlnotea{Positive Semi-Definite}) $v(x) \geq 0$ for all $x \in V$;
      \label{item:cond1_semi_norm}
    \item $\nu(\kappa x) = \abs{\kappa} \nu(x)$ for any $\kappa \in \mathbb{K}$ 
      and $x \in V$; and \label{item:cond2_semi_norm}
    \item (\hlnotea{Triangle Inequality}) $\nu(x + y) \leq \nu(x) + \nu(y)$ for
      all $x, y \in V$. \label{item:cond3_semi_norm}
  \end{enumerate}
  If $\nu(x) = 0 \implies x = 0$, then we say that $\nu$ is a \hlnoteb{norm}. In
  this case, we usually write $\norm{\cdot}$ to denote the norm, instead of
  $\nu$.
\end{defn}

\begin{note}
  \begin{itemize}
    \item We sometimes call a semi-norm a \hldefn{pseudo-length}.
  \end{itemize}
\end{note}

\begin{remark}
  Notice that we wrote $\nu(x) = 0 \implies x = 0$ instead of $\nu(x) = 0 \iff x
  = 0$. This is because if $z = 0 \in V$, then
  \begin{equation*}
    v(z) = v(0 z) = 0.
  \end{equation*}
\end{remark}

\begin{ex}
  Show that if $\nu$ is a semi-norm on a vector space $V$, then $\forall x, y
  \in V$,
  \begin{equation*}
    \abs{\nu(x) - \nu(y)} \leq \nu(x - y).
  \end{equation*}
\end{ex}

\begin{proof}
  Notice that by condition (\ref{item:cond2_semi_norm}) and
  (\ref{item:cond3_semi_norm}), we have
  \begin{equation*}
    \nu(x - y) \leq \nu(x) + \nu(-y) = \nu(x) - \nu(y),
  \end{equation*}
  and
  \begin{equation*}
    \nu(x - y) = -\nu(y - x) \geq - (\nu(y) - \nu(x)) = \nu(x) - \nu(y).
  \end{equation*}
  It follows that indeed
  \begin{equation*}
    \abs{\nu(x) - \nu(y)} \leq \nu(x - y).
  \end{equation*}
\end{proof}

\begin{eg}
  The absolute value $\abs{\cdot}$ is a \hlnotea{norm} on $\mathbb{K}$.
\end{eg}

\begin{eg}[$p$-norms]\label{eg:p_norms}
  Consider $N \geq 1$ an integer. We define a family of norms on 
  \begin{equation*}
    \mathbb{K}^N = \underbrace{K \times K \times \hdots \times K}_{N \text{
    times }}.
  \end{equation*}
  \hlbnoteb{$1$-norm}
  \begin{equation*}
    \norm{(x_n)_{n=1}^{N}}_{1} \coloneqq \sum_{n=1}^{N} \abs{x_n}.
  \end{equation*}
  \hlbnoteb{Infinity-norm, $\infty$-norm}
  \begin{equation*}
    \norm{(x_n)_{n=1}^{N}}_{\infty} \coloneqq \max_{1 \leq n \leq N} \abs{x_n}.
  \end{equation*}
  \hlbnoteb{Euclidean-norm, $2$-norm}
  \begin{equation*}
    \norm{(x_n)_{n=1}^{N}}_2 \coloneqq \left( \sum_{n=1}^{N} \abs{x_n}^2
    \right)^{\frac{1}{2}}
  \end{equation*}
  It is relatively easy to check that the above norms are indeed norms, except
  for the $2$-form. In particular, the \hlnotea{triangle inequality} is not as
  easy to show \sidenote{See
  \href{https://tex.japorized.ink/PMATH351F18/classnotes.pdf\#thm.29}{Minkowski's
  Inequality}.}.

  Less obviously so, but true nonetheless, we can define the following $p$-norms
  on $\mathbb{K}^N$ :
  \begin{equation*}
    \norm{(x_n)_{n=1}^N}_p \coloneqq \left( \sum_{n=1}^{N} \abs{x_n}^p
    \right)^{\frac{1}{p}},
  \end{equation*}
  for $1 \leq p < \infty$.
\end{eg}

\begin{culture}
  Consider $V = \mathbb{M}_n(\mathbb{C})$, \sidenote{Note that
  $\mathbb{M}_n(\mathbb{C})$ is the set of $n \times n$ matrices over
  $\mathbb{C}$.} where $n \in \mathbb{N}$ is fixed.
  For $T \in \mathbb{M}_n(\mathbb{C})$, we define the \hlnotea{singular numbers}
  of $T$ to be
  \begin{equation*}
    s_1(T) \geq s_2(T) \geq \hdots \geq s_n(T) \geq 0,
  \end{equation*}
  where $\sigma(T^* T) = \{ s_1(T)^2, s_2(T)^2, \ldots, s_n(T)^2 \}$, including
  multiplicity. Then we can define
  \begin{equation*}
    \norm{T}_p \coloneqq \left( \sum_{i=1}^{n} s_i(T)^p \right)^{\frac{1}{p}}
  \end{equation*}
  for $1 \leq p < \infty$, which is called the $p$-norm of $T$ on
  $\mathbb{M}_n(\mathbb{C})$.
\end{culture}

\begin{eg}
  Let
  \begin{equation*}
    V = \mathcal{C}([0, 1], \mathbb{K}) = \{ f : [0, 1] \to \mathbb{K} \mid f
    \text{ is continuous } \}.
  \end{equation*}
  Then
  \begin{equation*}
    \norm{f}_{\sup} \coloneqq \sup \{ \abs{f(x)} \mid x \in [0, 1] \}
  \end{equation*}
  \sidenote{Some authors use $\norm{f}_\infty$, but we will have the notation
  $\norm{[f]}_\infty$ later on, and so we shall use $\norm{f}_{\sup}$ for
  clarity.} defines a norm on $\mathcal{C}([0, 1], \mathbb{K})$.

  A sequence $(f_n)_{n=1)^{\infty}}$ in $V$ converges in this norm to some $f
  \in V$, i.e.
  \begin{equation*}
    \lim_{n \to \infty} \norm{f_n - f}_{\sup} = 0,
  \end{equation*}
  which means that $(f_n)_{n=1}^{\infty}$ converges uniformly to $f$ on $[0,
  1]$.
\end{eg}

\begin{defn}[Normed Linear Space]\index{Normed Linear Space}\label{defn:normed_linear_space}
  A \hlnoteb{normed linear space (NLS)} is a pair $(V, \norm{\cdot})$ where $V$
  is a vector space over $\mathbb{K}$ and $\norm\cdot$ is a norm on $V$.
\end{defn}

\begin{defn}[Metric]\index{Metric}\label{defn:metric}
  Given an NLS $(V, \norm{\cdot})$, we can define a \hldefn{metric} $d$ on $V$ (called the
  \hlnotea{metric induced by the norm}) as follows:
  \begin{equation*}
    d : V \times V \to \mathbb{R} \quad d(x, y) = \norm{x - y},
  \end{equation*}
  such that
  \begin{itemize}
    \item $d(x, y) \geq 0$ for all $x, y \in V$ and $d(x, y) = 0 \iff x = y$;
    \item $d(x, y) = d(y, x)$; and
    \item $d(x, y) \leq d(x, z) + d(y, z)$.
  \end{itemize}
\end{defn}

\begin{note}
  Norms are all metrics, and so any space that has a norm will induce a metric
  on the space.
\end{note}

\begin{defn}[Banach Space]\index{Banach Space}\label{defn:banach_space}
  We say that an NLS $(V, \norm{\cdot})$ is \hldefn{complete} or is a
  \hlnoteb{Banach Space} if the corresponding $(V, d)$, where $d$ is the metric
  induced by the norm, is complete \sidenote{Completeness of a metric space is
  such that any of its Cauchy sequences converges in the space.}.
\end{defn}

\begin{eg}
  $(\mathcal{C}([0, 1], \mathbb{K}), \norm{\cdot}_{\sup})$ is a Banach space.
\end{eg}

\begin{eg}
  We can define a $1$-norm $\norm{\cdot}_1$ on $\mathcal{C}([0, 1], \mathbb{K})$ 
  via
  \begin{equation*}
    \norm{f}_1 \coloneqq \int_{0}^{1} \abs{f}.
  \end{equation*}
  Then $(\mathcal{C}([0, 1], \mathbb{K}), \norm{\cdot}_1)$ is an NLS.
\end{eg}

\begin{ex}
  Show that $(\mathcal{C}([0, 1], \mathbb{K}), \norm{\cdot}_1)$ is not
  complete, which will then give us an example of a \hlimpo{normed linear space
  that is not Banach}.
\end{ex}

\begin{proof}
  Consider the sequence $(f_n)_{n=1}^{\infty}$ of continuous functions given by
  \begin{marginfigure}
    \centering
    \begin{tikzpicture}
      \draw[->] (-0.5, 0) -- (4, 0) node[right] {$x$};
      \draw[->] (0, -0.5) -- (0, 2) node[above] {$y$};
      \draw[line width=1.5pt,color=blue] (0, 0) -- (0.5, 0) -- (3.5, 1) -- (4, 1);
      \draw[line width=1.5pt,color=red] (0, 0) -- (0.5, 0) -- (1.5, 1) -- (4, 1);
      \node[circle,fill,inner sep=1pt,label={270:{$\frac{1}{2}$}}] at (0.5, 0) {};
      \node[circle,fill,inner sep=1pt,label={270:{$\frac{1}{2} + \frac{1}{m}$}}]
        at (1.5, 0) {};
      \node[circle,fill,inner sep=1pt,label={270:{$\frac{1}{2} + \frac{1}{n}$}}]
        at (3.5, 0) {};
    \end{tikzpicture}
    \caption{Sequence of functions $(f_n)_{n=1}^{\infty}$. We show for two indices $n < m$.}\label{fig:sequence_of_functions_f_n___n_1_infty_we_show_for_two_indices_n_m_}
  \end{marginfigure}
  \begin{equation*}
    f_n(x) = \begin{cases}
      0 & 0 \leq x < \frac{1}{2} \\
      n \left( x + \frac{1}{2} \right) & \frac{1}{2} \leq x \leq \frac{1}{2} +
      \frac{1}{n} \\
      1 & \text{ otherwise }
    \end{cases}
  \end{equation*}
  Note that the sequence $(f_n)_{n=1}^{\infty}$ is indeed \hlnotea{Cauchy}: let
  $\epsilon > 0$ and $\abs{n - m} < \frac{\epsilon}{\abs{x - \frac{1}{2}}}$, and
  then we have
  \begin{align*}
    \abs{f_n(x) - f_m(x)}
    &= \abs{n \left( x - \frac{1}{2} \right) - m \left( x - \frac{1}{2}\right)}
    \\
    &= \abs{(n - m) \left( x - \frac{1}{2} \right)}
    = \abs{n-m}\abs{x - \frac{1}{2}} < \epsilon.
  \end{align*}
  However, it is clear that the sequence $(f_n)_{n=1}^{\infty}$ converges to the
  piecewise function (in particular, a non-continuous function)
  \begin{equation*}
    f(x) = \begin{cases}
      0 & 0 \leq x < \frac{1}{2} \\
      1 & x \geq \frac{1}{2}
    \end{cases}.
  \end{equation*}
\end{proof}

\begin{eg}
  If $(\mathfrak{X}, \norm{\cdot}_{\mathfrak{X}})$ and $(\mathfrak{Y},
  \norm{\cdot}_{\mathfrak{Y}})$ are NLS's, and if $T : \mathfrak{X} \to
  \mathfrak{Y}$ is a linear map, we define the \hldefn{operator norm} of $T$ to
  be
  \begin{equation*}
    \norm{T} \coloneqq \sup \{ \norm{T(x)}_{\mathfrak{Y}} \mid
    \norm{x}_{\mathfrak{X}} \leq 1 \}.
  \end{equation*}
  We set
  \begin{equation*}
    B(\mathfrak{X}, \mathfrak{Y}) \coloneqq
    \{ T : \mathfrak{X} \to \mathfrak{Y} \mid T \text{ is linear }, \, \norm{T}
    < \infty \}.
  \end{equation*}
  Note that for any such linear map $T$, $\norm{T} < \infty \iff T$ is
  continuous. Thus $B(\mathfrak{X}, \mathfrak{Y})$ is the set of all continuous
  functions from $\mathfrak{X}$ into $\mathfrak{Y}$.

  Then $(B(\mathfrak{X}, \mathfrak{Y}), \norm{\cdot})$ is an NLS.
\end{eg}

\marginnote{It is likely that we have seen this in Real Analysis.}
\begin{ex}
  Show that $(B(\mathfrak{X}, \mathfrak{Y}), \norm{\cdot})$ is complete iff
  $(\mathfrak{Y}, \norm{\cdot}_{\mathfrak{Y}})$ is complete.
\end{ex}

\begin{note}
  One example of the last example is when $(\mathfrak{Y},
  \norm{\cdot}_{\mathfrak{Y}}) = (\mathbb{K}, \abs{\cdot})$. In this case,
  $B(\mathfrak{X}, \mathbb{K})$ is known as the \hlnotea{dual space} of
  $\mathfrak{X}$, or simple the \hlnotea{dual} of $\mathfrak{X}$.
\end{note}

We are interested in integrating over Banach spaces.

\begin{defn}[Partition of a Set]\index{Partition}\label{defn:partition}
  Let $(\mathfrak{X}, \norm{\cdot}_{\mathfrak{X}})$ be a
  \hyperref[defn:banach_space]{Banach space} and $f:[a, b] \to \mathfrak{X}$ a
  function, where $a < b \in \mathbb{R}$. A \hlnoteb{partition} $P$ of $[a, b]$ 
  is a finite set
  \begin{equation*}
    P = \{ a = p_0 < p_1 < \hdots < p_N = b \}
  \end{equation*}
  for some $N \geq 1$. The set of all partitions of $[a, b]$ is denoted by
  $\mathcal{P}[a, b]$.
\end{defn}

\begin{defn}[Test Values]\index{Test Values}\label{defn:test_values}
  Let $(\mathfrak{X}, \norm{\cdot}_{\mathfrak{X}})$ be a
  \hyperref[defn:banach_space]{Banach space} and $f:[a, b] \to \mathfrak{X}$ a
  function, where $a < b \in \mathbb{R}$. Let $P \in \mathcal{P}[a, b]$. A set
  \begin{equation*}
    P^* \coloneqq \{ p_k^* \}_{k = 1}^{N}
  \end{equation*}
  satisfying
  \begin{equation*}
    p_{k-1} \leq p_k^* \leq p_k, \text{ for } 1 \leq k \leq n
  \end{equation*}
  is called a set of \hlnoteb{test values} for $P$.
\end{defn}

\begin{defn}[Riemann Sum]\index{Riemann Sum}\label{defn:riemann_sum}
  Let $(\mathfrak{X}, \norm{\cdot}_{\mathfrak{X}})$ be a
  \hyperref[defn:banach_space]{Banach space} and $f:[a, b] \to \mathfrak{X}$ a
  function, where $a < b \in \mathbb{R}$. Let $P \in \mathcal{P}[a, b]$ and
  $P^*$ its corresponding set of test values. We define the \hlnoteb{Riemann
  sum} as
  \begin{equation*}
    S(f, P, P^*) = \sum_{k=1}^{N} f(p_k^*)(p_k - p_{k-1}).
  \end{equation*}
\end{defn}

\begin{remark}
  \begin{enumerate}
    \item Note that because \cref{defn:partition}, $p_k - p_{k-1} > 0$.
    \item When $(\mathfrak{X}, \norm{\cdot}) = (\mathbb{R}, \abs{\cdot})$, then
      this is the usual Riemann sum from first-year calculus.
    \item In general, note that
      \begin{equation*}
        \frac{1}{b - a} S(f, P, P^*) = \sum_{k=1}^{N} \lambda_k f(p_k^*),
      \end{equation*}
      where $0 < \lambda_k = \frac{p_k - p_{k-1}}{b - a} < 1$ and \sidenote{via
      the fact that the $\lambda_k$'s form a telescoping sum}
      \begin{equation*}
        \sum_{k=1}^{N} \lambda_k = 1.
      \end{equation*}
      So $\frac{1}{b - a} S(f, P, P^*)$ is an \hlnotea{averaging} of $f$ over
      $[a, b]$. We call $\frac{1}{b - a} S(f, P, P^*)$ the \hldefn{convex
      combination} of the $f(p_{k}^*)$'s.
  \end{enumerate}
\end{remark}

\begin{eg}[Silly example]
  Let $(\mathfrak{X} = \mathcal{C}([-\pi, \pi], \mathbb{K}),
  \norm{\cdot}_{\sup})$. Let
  \begin{equation*}
    f : [0, 1] \to \mathfrak{X} \text{ such that } x \mapsto e^{2\pi x} \sin
    7 \theta + \cos x \cos (12 \theta),
  \end{equation*}
  where $\theta \in [-\pi, \pi]$. Now if we consider the partition
  \begin{equation*}
    P = \left\{ - \pi, \frac{1}{10}, \frac{1}{2}, \pi \right\}
  \end{equation*}
  and its corresponding test value
  \begin{equation*}
    P^* = \left\{ 0, \frac{1}{3}, 2 \right\},
  \end{equation*}
  then
  \begin{align*}
    S(f, P, P^*)
    &= f(0) \left( \frac{1}{10} + \pi \right)
      + f \left( \frac{1}{3} \right) \left( \frac{1}{2} - \frac{1}{10} \right)
      + f(2) \left( \pi - \frac{1}{2} \right) \\
    &= (\sin 7 \theta + \cos 12 \theta) \left( \pi + \frac{1}{10} \right) \\
    &\quad + \left( e^{\frac{2\pi}{3}} \sin 7\theta +
      \cos\frac{1}{3} \cos 12\theta \right) \left( \frac{2}{5} \right) \\
    &\quad + ( e^{4 \pi} \sin 7\theta + \cos 2 \cos 12\theta ) \left( \pi -
      \frac{1}{2} \right)
  \end{align*}
\end{eg}

\begin{defn}[Refinement of a Partition]\index{Refinement}\label{defn:refinement_of_a_partition}
  Let $a < b \in \mathbb{R}$, and $P \in \mathcal{P}[a, b]$. We say $Q$ is a
  \hlnoteb{refinement} of $P$ is $Q \in \mathcal{P}[a, b]$ and $P \subseteq Q$.
\end{defn}

\begin{note}
  In simpler words, $Q$ is a ``finer'' partition that is based on $P$.
\end{note}

\begin{defn}[Riemann Integrable]\index{Riemann Integrable}\label{defn:riemann_integrable}
  Let $a < b \in \mathbb{R}$, $(\mathfrak{X}, \norm{\cdot}_{\mathfrak{X}})$ be a
  Banach space and $f : [a, b] \to \mathfrak{X}$ be a function. We say that $f$ 
  is \hlnoteb{Riemann integrable} over $[a, b]$ if $\exists x_0 \in
  \mathfrak{X}$ such that
  \begin{equation*}
    \forall \epsilon > 0 \quad \exists P \in \mathcal{P}[a, b],
  \end{equation*}
  such that if $Q$ is any refinement of $P$, and $Q^*$ is any set of test values
  of $Q$, then
  \begin{equation*}
    \norm{x_0 - S(f, Q, Q^*)}_{\mathfrak{X}} < \epsilon.
  \end{equation*}
  In this case, we write
  \begin{equation*}
    \int_{a}^{b} f = x_0.
  \end{equation*}
\end{defn}

\begin{propo}[Uniqueness of the Riemann Integral]\label{propo:uniqueness_of_the_riemann_integral}
  If $f$ is Riemann integrable over $[a, b]$, then the value of $\int_{a}^{b} f$ 
  is unique.
\end{propo}

\begin{proof}
  Suppose not, i.e.
  \begin{equation*}
    \int_{a}^{b} f = x_0 \text{ and } \int_{a}^{b} f = y_0
  \end{equation*}
  for some $x_0 \neq y_0$. Then, let
  \begin{equation*}
    \epsilon = \frac{\norm{x_0 - y_0}}{2},
  \end{equation*}
  which is $> 0$ since $\norm{x_0 - y_0} > 0$. Let $P_{x_0}, P_{y_0} \in
  \mathcal{P}[a, b]$ be partitions corresponding to $x_0$ and $y_0$ as in the
  definition of Riemann integrability.

  Then, let $R = P_{x_0} \cup P_{y_0}$, so that $R$ is a \hldefn{common
  refinement} of $P_{x_0}$ and $P_{y_0}$. If $Q$ is any refinement of $R$, then
  $Q$ is also a common refinement of $P_{x_0}$ and $P_{y_0}$. Then for any test
  values $Q^*$ of $Q$, we have
  \begin{align*}
    2 \epsilon &= \norm{x_0 - y_0} \\
               &\leq \norm{x_0 - S(f, Q, Q^*)} + \norm{S(f, Q, Q^*) - y_0} <
               \epsilon + \epsilon = 2 \epsilon,
  \end{align*}
  which is a contradiction.

  Thus $x_0 = y_0$ as required.
\end{proof}

\begin{thm}[Cauchy Criterion of Riemann Integrability]\index{Cauchy Criterion of Riemann Integrability}\label{thm:cauchy_criterion_of_riemann_integrability}
  Let $(\mathfrak{X}, \norm{\cdot}_{\mathfrak{X}})$ be a Banach space, $a < b
  \in \mathbb{R}$ and $f : [a, b] \to \mathfrak{X}$ be a function. TFAE:
  \begin{enumerate}
    \item $f$ is Riemann integrable over $[a, b]$;
    \item $\forall \epsilon > 0, \, R \in \mathcal{P}[a, b]$, if $P, Q$ is any
      refinement of $R$, and $P^*$ (respectively $Q^*$) is any test values of
      $P$ (respectively $Q$), then
      \begin{equation*}
        \norm{S(f, P, P^*) - S(f, Q, Q^*)}_{\mathfrak{X}} < \epsilon.
      \end{equation*}
  \end{enumerate}
\end{thm}

\begin{proof}
  \hlbnoted{$\implies$} This is a rather straightforward proof. Suppose $P, Q
  \in \mathcal{P}[a, b]$ is some refinement of the given partition $R \in
  \mathcal{P}[a, b]$, and $P^*, Q^*$ any test values for $P, Q$, respectively.
  Then by assumption and \cref{propo:uniqueness_of_the_riemann_integral},
  $\exists x_0 \in \mathfrak{X}$ such that
  \begin{equation*}
    \norm{x_0 - S(f, P, P^*)}_{\mathfrak{X}} < \frac{\epsilon}{2} \text{ and }
    \norm{x_0 - S(f, Q, Q^*)}_{\mathfrak{X}} < \frac{\epsilon}{2}.
  \end{equation*}
  It follows that
  \begin{align*}
    &\norm{S(f,P,P^*) - S(f,Q,Q^*)}_{\mathfrak{X}} \\
    &\leq \norm{x_0 - S(f,P,P^*)}_{\mathfrak{X}} + \norm{x_0 -
      S(f,Q,Q^*)}_{\mathfrak{X} } \\
    &< \frac{\epsilon}{2} + \frac{\epsilon}{2} = \epsilon.
  \end{align*}

  \noindent
  \hlbnoted{$\impliedby$} By hypothesis, wma $\epsilon = \frac{1}{n}$ for some
  $n \geq 1$, such that if $P, Q$ are any refinements of the partition $R_n \in
  \mathcal{P}[a, b]$, and $P^*, Q^*$ are the respective arbitrary test values,
  then
  \begin{equation*}
    \norm{S(f,P,P^*) - S(f,Q,Q^*)}_{\mathfrak{X}} < \frac{1}{n}
  \end{equation*}
  
  Now for each $n \geq 1$, define
  \begin{equation*}
    W_n \coloneqq \bigcup_{k=1}^{n} R_k \in \mathcal{P}[a, b],
  \end{equation*}
  so that $W_n$ is a common refinement for $R_1, R_2, \ldots, R_n$. For each $n
  \geq 1$, let $W_n^*$ be an arbitrary set of test values for $W_n$. For
  simplicity, let us write
  \begin{equation*}
    x_n = S(f, \, W_n, \, W_n^*), \text{ for each } n \geq 1.
  \end{equation*}
  \sidenote{Note that it would be nice if for the finer and finer partitions
  that we have constructed, i.e. the $W_n$'s, give us a convergent sequence of
  Riemann sums, since it makes sense that this convergence will give us the final
  value that we want.}

  \noindent
  \hlbnotec{Claim: $(x_n)_{n=1}^{\infty}$ is a Cauchy sequence}
  If $n_1 \geq n_2 > N \in \mathbb{N}$, then
  \begin{align*}
    \norm{x_{n_1} - x_{n_2}}_{\mathfrak{X}} &= \norm{S(f,W_{n_1},W_{n_1}^*) -
    S(f,W_{n_2},W_{n_2}^*)} < \frac{1}{N}
  \end{align*}
  by our assumption, since $W_{n_1}, W_{n_2}$ are refinements of $R_N$. Then by
  picking $N = \frac{1}{\epsilon}$ for any $\epsilon > 0$, we have that
  $(x_n)_{n=1}^{\infty}$ is indeed a Cauchy sequence in $\mathfrak{X}$.

  Since $\mathfrak{X}$ is a Banach space, it is complete, and so $\exists x_0
  \coloneqq \lim\limits_{n \to \infty} x_n \in \mathfrak{X}$. It remains to show
  that, indeed,
  \begin{equation*}
    x_0 = \int_{a}^{b} f.
  \end{equation*}

  Let $\epsilon > 0$, and choose $N \geq 1$ such that
  \begin{itemize}
    \item $\frac{1}{N} < \frac{\epsilon}{2}$ ; and
    \item $k \geq N$ implies that $\norm{x_k - x_0} < \frac{\epsilon}{2}$.
  \end{itemize}
  Then suppose that $V$ is any refinement of $W_N$, and $V^*$ is an arbitrary
  set of test values of $V$. Then we have
  \begin{align*}
    \norm{x_0 - S(f,V,V^*)}_{\mathfrak{X}}
    &\leq \norm{x_0 - x_N}_{\mathfrak{X}} + \norm{x_N -
      S(f,V,V^*)}_{\mathfrak{X}} \\
    &< \frac{\epsilon}{2} + \norm{S(f,W_N,W_N^*) - S(f,V,V^*)}_{\mathfrak{X}} \\
    &<\frac{\epsilon}{2} + \frac{1}{N} \leq \frac{\epsilon}{2} +
    \frac{\epsilon}{2} = \epsilon.
  \end{align*}
  It follows that
  \begin{equation*}
    \int_{a}^{b} f = x_0,
  \end{equation*}
  as desired.
\end{proof}

In first-year calculus, all continuous functions over $\mathbb{R}$ are
integrable. A similar result holds in Banach spaces as well.
In the next lecture, we shall prove the following theorem.

\begin{thmnonum}[Continuous Functions are Riemann Integrable]\label{thmnonum:continuous_functions_are_riemann_integrable}
  Let $(\mathfrak{X}, \norm{\cdot})$ be a Banach space and $a < b \in
  \mathbb{R}$. If $f : [a, b] \to \mathfrak{X}$ is continuous, then $f$ is
  Riemann integrable over $[a, b]$.
\end{thmnonum}

% section riemannian_integration (end)

% chapter lecture_1_may_07th_2019 (end)

\appendix

\backmatter

\fancyhead[LE]{\thepage \enspace \textsl{\leftmark}}

% \nobibliography*
\bibliography{references}

\printindex

\end{document}
% vim:tw=80:fdm=syntax

