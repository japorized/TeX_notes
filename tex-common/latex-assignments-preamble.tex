\usepackage{geometry}
\geometry{letterpaper,lmargin=0.7in,rmargin=3in,bmargin=1in}
\usepackage[parfill]{parskip}
\usepackage{graphicx}

% Essential Packages
\usepackage{makeidx}
\makeindex
\usepackage{enumitem}
\usepackage{ragged2e}
\usepackage{amssymb}
\usepackage{amsmath}
\usepackage{mathrsfs}
\usepackage{tkz-euclide}
\usetkzobj{all}
\usepackage[utf8]{inputenc}
\usepackage[english]{babel}
\usepackage{marvosym}
\usepackage{fontawesome}
\usepackage{pgf,tikz}
\usepackage{pgfplots}
\usepackage{fancyhdr}
\usepackage{array}
\usepackage{faktor}
\usepackage{float}
\usepackage{xcolor}
\usepackage[hyperref]{ntheorem}
\usepackage{hyperref}
\usepackage[noabbrev,capitalize,nameinlink]{cleveref}
\usepackage[most]{tcolorbox}

% hyperref Package Settings
\hypersetup{
	colorlinks,
  breaklinks=true,
	allcolors=black
}
\def\UrlBreaks{\do\/\do-}

\definecolor{dark}{HTML}{2D2D2D}
\definecolor{light}{HTML}{D3D0C8}
\definecolor{red}{HTML}{F47678}
\definecolor{green}{HTML}{98CD97}
\definecolor{yellow}{HTML}{E2B552}
\definecolor{blue}{HTML}{6498CE}
\definecolor{magenta}{HTML}{CD98CD}
\definecolor{cyan}{HTML}{61CCCD}
\definecolor{gray}{HTML}{747369}
\definecolor{brown}{HTML}{D47B4E}

% fancyhdr
\makeatletter
\pagestyle{fancy}
\rhead{\textsl{\@title} \enspace \thepage}
\makeatother

% tikz
\usepgfplotslibrary{polar}
\usepgflibrary{shapes.geometric}
\usetikzlibrary{angles,patterns,calc,decorations.markings,arrows.meta,tikzmark,bending}
\tikzset{midarrow/.style 2 args={
        decoration={markings,
            mark= at position #2 with {\arrow{#1}} ,
        },
        postaction={decorate}
    },
    midarrow/.default={latex}{0.5}
}
\def\centerarc[#1](#2)(#3:#4:#5)% Syntax: [draw options] (center) (initial angle:final angle:radius)
    { \draw[#1] ($(#2)+({#5*cos(#3)},{#5*sin(#3)})$) arc (#3:#4:#5); }

% enumitems
\newlist{inlinelist}{enumerate*}{1}
\setlist*[inlinelist,1]{%
  label=(\roman*),
}

% Theorem Style Customization
\setlength\theorempreskipamount{2ex}
\setlength\theorempostskipamount{3ex}

% ntheorem Declarations
\theorembodyfont{\normalfont}
\theoremstyle{break}
\theoremprework{{\rule{\linewidth}{1pt}}}
\theorempostwork{{\rule{\linewidth}{1pt}}}
\newtheorem*{proof}{\faPencil \enspace Proof}

\theoremprework{\rule{\linewidth}{1pt}}
\theorempostwork{\rule{\linewidth}{1pt}}
\newtheorem*{solution}{\faPencil \enspace Solution}
\newtheorem*{remark}{Remark}
\newtheorem*{note}{Note}

% ntheorem listtheorem style
\makeatother
\newlength\widesttheorem
\AtBeginDocument{
  \settowidth{\widesttheorem}{Proposition A.1.1.1\quad}
}

% Custom table columning
\newcolumntype{L}[1]{>{\raggedright\let\newline\\\arraybackslash\hspace{0pt}}m{#1}}
\newcolumntype{C}[1]{>{\centering\let\newline\\\arraybackslash\hspace{0pt}}m{#1}}
\newcolumntype{R}[1]{>{\raggedleft\let\newline\\\arraybackslash\hspace{0pt}}m{#1}}

% Graph styles
\pgfplotsset{compat=1.15}
\usepgfplotslibrary{fillbetween}
\pgfplotsset{four quads/.append style={axis x line=middle, axis y line=
middle, xlabel={$x$}, ylabel={$y$}, axis equal }}
\pgfplotsset{four quad complex/.append style={axis x line=middle, axis y line=
middle, xlabel={$\re$}, ylabel={$\im$}, axis equal }}

% Shortcuts
\newenvironment{spmatrix}
  {\left(\begin{smallmatrix}}
  {\end{smallmatrix}\right)}
\newcommand{\floor}[1]{\lfloor #1 \rfloor}      % simplifying the writing of a floor function
\newcommand{\ceiling}[1]{\lceil #1 \rceil}      % simplifying the writing of a ceiling function
\newcommand{\dotp}{\, \cdotp}			        % dot product to distinguish from \cdot
\newcommand{\qed}{\hfill\ensuremath{\square}}   % Q.E.D sign
\newcommand{\abs}[1]{\left|#1\right|}						% absolute value
\newcommand{\at}[2]{\Big|_{#1}^{#2}}
\renewcommand{\bar}[1]{\mkern 1.5mu \overline{\mkern -1.5mu #1 \mkern -1.5mu} \mkern 1.5mu}
\newcommand{\lra}[1]{\left\langle \; #1 \; \right\rangle}
	% highlighting shortcuts
\renewcommand{\epsilon}{\varepsilon}
\renewcommand{\atop}[2]{\genfrac{}{}{0pt}{}{#1}{#2}}
\newcommand{\mmid}{\; \middle| \;}
\newcommand*\dif{\mathop{}\!d}
\newcommand{\norm}[1]{\left\| #1 \right\|}

\newcommand{\hlb}[2]{\colorbox{#1!10!light}{\textbf{#2}}}
\newcommand{\hlbnotea}[1]{\hlb{green}{#1}}
\newcommand{\hlbnoteb}[1]{\hlb{cyan}{#1}}
\newcommand{\hlbnotec}[1]{\hlb{brown}{#1}}
\newcommand{\hlbnoted}[1]{\hlb{magenta}{#1}}
\newcommand{\hlbnotee}[1]{\hlb{red}{#1}}
