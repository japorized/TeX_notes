\documentclass[notoc,notitlepage]{tufte-book}
% \nonstopmode % uncomment to enable nonstopmode

\usepackage{classnotetitle}

\title{ACTSC 431 - Loss Model I}
\author{Johnson Ng}
\subtitle{Classnotes for Fall 2018}
\credentials{BMath (Hons), Pure Mathematics major, Actuarial Science Minor}
\institution{University of Waterloo}

\setcounter{secnumdepth}{3}
\setcounter{tocdepth}{3}

\renewcommand{\baselinestretch}{1.1}
\usepackage{geometry}
\geometry{letterpaper}
\usepackage[parfill]{parskip}
\usepackage{graphicx}

% Essential Packages
\usepackage{makeidx}
\makeindex
\usepackage{enumitem}
\usepackage[T1]{fontenc}
\usepackage{natbib}
\bibliographystyle{apalike}
\usepackage{ragged2e}
\usepackage{etoolbox}
\usepackage{amssymb}
\usepackage{fontawesome}
\usepackage{amsmath}
\usepackage{mathrsfs}
\usepackage{mathtools}
\usepackage{xparse}
\usepackage{tkz-euclide}
\usetkzobj{all}
\usepackage[utf8]{inputenc}
\usepackage{csquotes}
\usepackage[english]{babel}
\usepackage{marvosym}
\usepackage{pgf,tikz}
\usepackage{pgfplots}
\usepackage{fancyhdr}
\usepackage{array}
\usepackage{faktor}
\usepackage{float}
\usepackage{xcolor}
\usepackage{centernot}
\usepackage{silence}
  \WarningFilter*{latex}{Marginpar on page \thepage\space moved}
\usepackage{tcolorbox}
\tcbuselibrary{skins,breakable}
\usepackage{longtable}
\usepackage[amsmath,hyperref]{ntheorem}
\usepackage{hyperref}
\usepackage[noabbrev,capitalize,nameinlink]{cleveref}

% xcolor (scheme: base16 eighties)
\definecolor{base16-eighties-dark}{HTML}{2D2D2D}
\definecolor{base16-eighties-light}{HTML}{D3D0C8}
\definecolor{base16-eighties-magenta}{HTML}{CD98CD}
\definecolor{base16-eighties-red}{HTML}{F47678}
\definecolor{base16-eighties-yellow}{HTML}{E2B552}
\definecolor{base16-eighties-green}{HTML}{98CD97}
\definecolor{base16-eighties-lightblue}{HTML}{61CCCD}
\definecolor{base16-eighties-blue}{HTML}{6498CE}
\definecolor{base16-eighties-brown}{HTML}{D47B4E}
\definecolor{base16-eighties-gray}{HTML}{747369}

% hyperref Package Settings
\hypersetup{
    bookmarks=true,         % show bookmarks bar?
    unicode=true,          % non-Latin characters in Acrobat’s bookmarks
    pdftoolbar=false,        % show Acrobat’s toolbar?
    pdfmenubar=false,        % show Acrobat’s menu?
    pdffitwindow=true,     % window fit to page when opened
    colorlinks=true,
    allcolors=base16-eighties-magenta,
}

% tikz
\usepgfplotslibrary{polar}
\usepgflibrary{shapes.geometric}
\usetikzlibrary{angles,patterns,calc,decorations.markings}
\tikzset{midarrow/.style 2 args={
        decoration={markings,
            mark= at position #2 with {\arrow{#1}} ,
        },
        postaction={decorate}
    },
    midarrow/.default={latex}{0.5}
}
\def\centerarc[#1](#2)(#3:#4:#5)% Syntax: [draw options] (center) (initial angle:final angle:radius)
    { \draw[#1] ($(#2)+({#5*cos(#3)},{#5*sin(#3)})$) arc (#3:#4:#5); }

% enumitems
\newlist{inlinelist}{enumerate*}{1}
\setlist*[inlinelist,1]{%
  label=(\roman*),
}

% Theorem Style Customization
\setlength\theorempreskipamount{2ex}
\setlength\theorempostskipamount{3ex}

\makeatletter
\let\nobreakitem\item
\let\@nobreakitem\@item
\patchcmd{\nobreakitem}{\@item}{\@nobreakitem}{}{}
\patchcmd{\nobreakitem}{\@item}{\@nobreakitem}{}{}
\patchcmd{\@nobreakitem}{\@itempenalty}{\@M}{}{}
\patchcmd{\@xthm}{\ignorespaces}{\nobreak\ignorespaces}{}{}
\patchcmd{\@ythm}{\ignorespaces}{\nobreak\ignorespaces}{}{}

\renewtheoremstyle{break}%
  {\item{\theorem@headerfont
          ##1\ ##2\theorem@separator}\hskip\labelsep\relax\nobreakitem}%
  {\item{\theorem@headerfont
          ##1\ ##2\ (##3)\theorem@separator}\hskip\labelsep\relax\nobreakitem}
\makeatother

% ntheorem + framed
\makeatletter

% ntheorem Declarations
\theorempreskip{10pt}
\theorempostskip{5pt}
\theoremstyle{break}

\newtheorem*{solution}{\faPencil $\enspace$ Solution}
\newtheorem*{remark}{Remark}
\newtheorem{eg}{Example}[section]
\newtheorem{ex}{Exercise}[section]

    % definition env
\theoremprework{\textcolor{base16-eighties-blue}{\hrule height 2pt}}
\theoremheaderfont{\color{base16-eighties-blue}\normalfont\bfseries}
\theorempostwork{\textcolor{base16-eighties-blue}{\hrule height 2pt}}
\theoremindent10pt
\newtheorem{defn}{\faBook \enspace Definition}

    % definition env no num
\theoremprework{\textcolor{base16-eighties-blue}{\hrule height 2pt}}
\theoremheaderfont{\color{base16-eighties-blue}\normalfont\bfseries}
\theorempostwork{\textcolor{base16-eighties-blue}{\hrule height 2pt}}
\theoremindent10pt
\newtheorem*{defnnonum}{\faBook \enspace Definition}

    % theorem envs
\theoremprework{\textcolor{base16-eighties-magenta}{\hrule height 2pt}}
\theoremheaderfont{\color{base16-eighties-magenta}\normalfont\bfseries}
\theorempostwork{\textcolor{base16-eighties-magenta}{\hrule height 2pt}}
\theoremindent10pt
\newtheorem{thm}{\faCoffee \enspace Theorem}

\theoremprework{\textcolor{base16-eighties-magenta}{\hrule height 2pt}}
\theorempostwork{\textcolor{base16-eighties-magenta}{\hrule height 2pt}}
\theoremindent10pt
\newtheorem{propo}[thm]{\faTint \enspace Proposition}

\theoremprework{\textcolor{base16-eighties-magenta}{\hrule height 2pt}}
\theorempostwork{\textcolor{base16-eighties-magenta}{\hrule height 2pt}}
\theoremindent10pt
\newtheorem{crly}[thm]{\faSpaceShuttle \enspace Corollary}

\theoremprework{\textcolor{base16-eighties-magenta}{\hrule height 2pt}}
\theorempostwork{\textcolor{base16-eighties-magenta}{\hrule height 2pt}}
\theoremindent10pt
\newtheorem{lemma}[thm]{\faTree \enspace Lemma}

\theoremprework{\textcolor{base16-eighties-magenta}{\hrule height 2pt}}
\theorempostwork{\textcolor{base16-eighties-magenta}{\hrule height 2pt}}
\theoremindent10pt
\newtheorem{axiom}[thm]{\faShield \enspace Axiom}

    % theorem envs without counter
\theoremprework{\textcolor{base16-eighties-magenta}{\hrule height 2pt}}
\theoremheaderfont{\color{base16-eighties-magenta}\normalfont\bfseries}
\theorempostwork{\textcolor{base16-eighties-magenta}{\hrule height 2pt}}
\theoremindent10pt
\newtheorem*{thmnonum}{\faCoffee \enspace Theorem}

\theoremprework{\textcolor{base16-eighties-magenta}{\hrule height 2pt}}
\theorempostwork{\textcolor{base16-eighties-magenta}{\hrule height 2pt}}
\theoremindent10pt
\newtheorem*{propononum}{\faTint \enspace Proposition}

\theoremprework{\textcolor{base16-eighties-magenta}{\hrule height 2pt}}
\theorempostwork{\textcolor{base16-eighties-magenta}{\hrule height 2pt}}
\theoremindent10pt
\newtheorem*{crlynonum}{\faSpaceShuttle \enspace Corollary}

\theoremprework{\textcolor{base16-eighties-magenta}{\hrule height 2pt}}
\theorempostwork{\textcolor{base16-eighties-magenta}{\hrule height 2pt}}
\theoremindent10pt
\newtheorem*{lemmanonum}{\faTree \enspace Lemma}

\theoremprework{\textcolor{base16-eighties-magenta}{\hrule height 2pt}}
\theorempostwork{\textcolor{base16-eighties-magenta}{\hrule height 2pt}}
\theoremindent10pt
\newtheorem*{axiomnonum}{\faShield \enspace Axiom}

    % proof env
\theoremprework{\textcolor{base16-eighties-brown}{\hrule height 2pt}}
\theoremheaderfont{\color{base16-eighties-brown}\normalfont\bfseries}
\theorempostwork{\textcolor{base16-eighties-brown}{\hrule height 2pt}}
\newtheorem*{proof}{\faPencil \enspace Proof}

    % note and notation env
\theoremprework{\textcolor{base16-eighties-yellow}{\hrule height 2pt}}
\theoremheaderfont{\color{base16-eighties-yellow}\normalfont\bfseries}
\theorempostwork{\textcolor{base16-eighties-yellow}{\hrule height 2pt}}
\newtheorem*{note}{\faQuoteLeft \enspace Note}

\theoremprework{\textcolor{base16-eighties-yellow}{\hrule height 2pt}}
\theorempostwork{\textcolor{base16-eighties-yellow}{\hrule height 2pt}}
\newtheorem*{notation}{\faPaw \enspace Notation}

    % warning env
\theoremprework{\textcolor{base16-eighties-red}{\hrule height 2pt}}
\theoremheaderfont{\color{base16-eighties-red}\normalfont\bfseries}
\theorempostwork{\textcolor{base16-eighties-red}{\hrule height 2pt}}
\theoremindent10pt
\newtheorem*{warning}{\faBug \enspace Warning}

% more environments
\newtcolorbox{redquote}{
  blanker,enhanced,breakable,standard jigsaw,
  opacityback=0,
  coltext=base16-eighties-light,
  left=5mm,right=5mm,top=2mm,bottom=2mm,
  colframe=base16-eighties-red,
  boxrule=0pt,leftrule=3pt,
  fontupper=\itshape
}
\newtcolorbox{bluequote}{
  blanker,enhanced,breakable,standard jigsaw,
  opacityback=0,
  coltext=base16-eighties-light,
  left=5mm,right=5mm,top=2mm,bottom=2mm,
  colframe=base16-eighties-blue,
  boxrule=0pt,leftrule=3pt,
  fontupper=\itshape
}
\newtcolorbox{greenquote}{
  blanker,enhanced,breakable,standard jigsaw,
  opacityback=0,
  coltext=base16-eighties-light,
  left=5mm,right=5mm,top=2mm,bottom=2mm,
  colframe=base16-eighties-green,
  boxrule=0pt,leftrule=3pt,
  fontupper=\itshape
}
\newtcolorbox{yellowquote}{
  blanker,enhanced,breakable,standard jigsaw,
  opacityback=0,
  coltext=base16-eighties-light,
  left=5mm,right=5mm,top=2mm,bottom=2mm,
  colframe=base16-eighties-yellow,
  boxrule=0pt,leftrule=3pt,
  fontupper=\itshape
}
\newtcolorbox{magentaquote}{
  blanker,enhanced,breakable,standard jigsaw,
  opacityback=0,
  coltext=base16-eighties-light,
  left=5mm,right=5mm,top=2mm,bottom=2mm,
  colframe=base16-eighties-magenta,
  boxrule=0pt,leftrule=3pt,
  fontupper=\itshape
}

% ntheorem listtheorem style
\makeatother
\newlength\widesttheorem
\AtBeginDocument{
  \settowidth{\widesttheorem}{Proposition A.1.1.1\quad}
}

\makeatletter
\def\thm@@thmline@name#1#2#3#4{%
        \@dottedtocline{-2}{0em}{2.3em}%
                   {\makebox[\widesttheorem][l]{#1 \protect\numberline{#2}}#3}%
                   {#4}}
\@ifpackageloaded{hyperref}{
\def\thm@@thmline@name#1#2#3#4#5{%
    \ifx\#5\%
        \@dottedtocline{-2}{0em}{2.3em}%
            {\makebox[\widesttheorem][l]{#1 \protect\numberline{#2}}#3}%
            {#4}
    \else
        \ifHy@linktocpage\relax\relax
            \@dottedtocline{-2}{0em}{2.3em}%
                {\makebox[\widesttheorem][l]{#1 \protect\numberline{#2}}#3}%
                {\hyper@linkstart{link}{#5}{#4}\hyper@linkend}%
        \else
            \@dottedtocline{-2}{0em}{2.3em}%
                {\hyper@linkstart{link}{#5}%
                  {\makebox[\widesttheorem][l]{#1 \protect\numberline{#2}}#3}\hyper@linkend}%
                    {#4}%
        \fi
    \fi}
}

\makeatletter
\def\thm@@thmline@noname#1#2#3#4{%
        \@dottedtocline{-2}{0em}{5em}%
                   {{\protect\numberline{#2}}#3}%
                   {#4}}
\@ifpackageloaded{hyperref}{
\def\thm@@thmline@noname#1#2#3#4#5{%
    \ifx\#5\%
        \@dottedtocline{-2}{0em}{5em}%
            {{\protect\numberline{#2}}#3}%
            {#4}
    \else
        \ifHy@linktocpage\relax\relax
            \@dottedtocline{-2}{0em}{5em}%
                {{\protect\numberline{#2}}#3}%
                {\hyper@linkstart{link}{#5}{#4}\hyper@linkend}%
        \else
            \@dottedtocline{-2}{0em}{5em}%
                {\hyper@linkstart{link}{#5}%
                  {{\protect\numberline{#2}}#3}\hyper@linkend}%
                    {#4}%
        \fi
    \fi}
}

\theoremlisttype{allname}

\AtBeginDocument{\renewcommand\contentsname{Table of Contents}}

% Heading formattings
% chapter format
\titleformat{\chapter}%
  {\huge\rmfamily\itshape\color{base16-eighties-magenta}}% format applied to label+text
  {\llap{\colorbox{base16-eighties-magenta}{\parbox{1.5cm}{\hfill\itshape\huge\textcolor{base16-eighties-dark}{\thechapter}}}}}% label
  {5pt}% horizontal separation between label and title body
  {}% before the title body
  []% after the title body

% section format
\titleformat{\section}%
  {\normalfont\Large\rmfamily\itshape\color{base16-eighties-blue}}% format applied to label+text
  {\llap{\colorbox{base16-eighties-blue}{\parbox{1.5cm}{\hfill\itshape\textcolor{base16-eighties-dark}{\thesection}}}}}% label
  {5pt}% horizontal separation between label and title body
  {}% before the title body
  []% after the title body

% subsection format
\titleformat{\subsection}%
  {\normalfont\large\itshape\color{base16-eighties-green}}% format applied to label+text
  {\llap{\colorbox{base16-eighties-green}{\parbox{1.5cm}{\hfill\textcolor{base16-eighties-dark}{\thesubsection}}}}}% label
  {1em}% horizontal separation between label and title body
  {}% before the title body
  []% after the title body

% Sidenote enhancements
\def\mathmarginnote#1{%
  \tag*{\rlap{\hspace\marginparsep\smash{\parbox[t]{\marginparwidth}{%
  \footnotesize#1}}}}
}

% Custom table columning
\newcolumntype{L}[1]{>{\raggedright\let\newline\\\arraybackslash\hspace{0pt}}m{#1}}
\newcolumntype{C}[1]{>{\centering\let\newline\\\arraybackslash\hspace{0pt}}m{#1}}
\newcolumntype{R}[1]{>{\raggedleft\let\newline\\\arraybackslash\hspace{0pt}}m{#1}}

% Custom math operator
% \DeclareMathOperator{\rem}{rem}
\DeclareMathOperator*{\argmax}{arg\,max}
\DeclareMathOperator*{\argmin}{arg\,min}
\DeclareMathOperator{\re}{Re}
\DeclareMathOperator{\im}{Im}
\DeclareMathOperator{\caparg}{Arg}
\DeclareMathOperator{\Ind}{Ind}
\DeclareMathOperator{\Res}{Res}

% Graph styles
\pgfplotsset{compat=1.15}
\usepgfplotslibrary{fillbetween}
\pgfplotsset{four quads/.append style={axis x line=middle, axis y line=
middle, xlabel={$x$}, ylabel={$y$}, axis equal }}
\pgfplotsset{four quad complex/.append style={axis x line=middle, axis y line=
middle, xlabel={$\re$}, ylabel={$\im$}, axis equal }}

% Shortcuts
\newcommand{\floor}[1]{\lfloor #1 \rfloor}      % simplifying the writing of a floor function
\newcommand{\ceiling}[1]{\lceil #1 \rceil}      % simplifying the writing of a ceiling function
\newcommand{\dotp}{\, \cdotp}			        % dot product to distinguish from \cdot
\newcommand{\qed}{\hfill\ensuremath{\square}}   % Q.E.D sign
\newcommand{\abs}[1]{\left|#1\right|}						% absolute value
\newcommand{\lra}[1]{\langle \; #1 \; \rangle}
\newcommand{\at}[2]{\Big|_{#1}^{#2}}
\newcommand{\Arg}[1]{\caparg #1}
\renewcommand{\bar}[1]{\mkern 1.5mu \overline{\mkern -1.5mu #1 \mkern -1.5mu} \mkern 1.5mu}
\newcommand{\quotient}[2]{\faktor{#1}{#2}}
\newcommand{\cyclic}[1]{\left\langle #1 \right\rangle}
	% highlighting shortcuts
\newcommand{\hlimpo}[1]{\textcolor{base16-eighties-red}{\textbf{#1}}}
\newcommand{\hlwarn}[1]{\textcolor{base16-eighties-yellow}{\textbf{#1}}}
\newcommand{\hldefn}[1]{\textcolor{base16-eighties-blue}{\index{#1}\textbf{#1}}}
\newcommand{\hlnotea}[1]{\textcolor{base16-eighties-green}{\textbf{#1}}}
\newcommand{\hlnoteb}[1]{\textcolor{base16-eighties-lightblue}{\textbf{#1}}}
\newcommand{\hlnotec}[1]{\textcolor{base16-eighties-brown}{\textbf{#1}}}
\newcommand{\WTP}{\textcolor{base16-eighties-brown}{WTP} }
\newcommand{\WTS}{\textcolor{base16-eighties-brown}{WTS} }
\newcommand{\ind}[2]{\Ind_{#2}\left( #1 \right)}
\newcommand{\notimply}{\centernot\implies}
\newcommand{\res}[2]{\underset{#2}{\Res} #1 }
\newcommand{\tworow}[3]{\begin{tabular}{@{}#1@{}} #2 \\ #3 \end{tabular}}
\renewcommand{\epsilon}{\varepsilon}
\newcommand{\lrarrow}{\leftrightarrow}
\newcommand{\larrow}{\leftarrow}
\newcommand{\rarrow}{\rightarrow}
\renewcommand{\atop}[2]{\genfrac{}{}{0pt}{}{#1}{#2}}
\newcommand*\dif{\mathop{}\!d}

  % inspiration from: https://tex.stackexchange.com/questions/8720/overbrace-underbrace-but-with-an-arrow-instead#37758
\newcommand{\overarrow}[2]{
  \overset{\makebox[0pt]{\begin{tabular}{@{}c@{}}#2\\[0pt]\ensuremath{\uparrow}\end{tabular}}}{#1}
}
\newcommand{\underarrow}[2]{
  \underset{\makebox[0pt]{\begin{tabular}{@{}c@{}}\downarrow\\[0pt]\ensuremath{#2}\end{tabular}}}{#1}
}

% Document header formatting
\renewcommand{\chaptermark}[1]{\markboth{#1}{}}
\renewcommand{\sectionmark}[1]{\markright{#1}}
\makeatletter
\pagestyle{fancy}
\fancyhead{}
\fancyhead[RO]{\textsl{\@title} \enspace \thepage}
\fancyhead[LE]{\thepage \enspace \textsl{\leftmark \enspace - \enspace \rightmark}}
\makeatother

% Comment the two lines below if you want to print the document
\pagecolor{base16-eighties-dark}
\color{base16-eighties-light}

\DeclareMathOperator{\Bernoulli}{Bernoulli }
\DeclareMathOperator{\Bin}{Bin }
\DeclareMathOperator{\Geo}{Geo }
\DeclareMathOperator{\Poi}{Poi }
\DeclareMathOperator{\NB}{NB }
\DeclareMathOperator{\Exp}{Exp }
\DeclareMathOperator{\Unif}{Unif }
\DeclareMathOperator{\Nor}{N }
\DeclareMathOperator{\Gau}{G }
\DeclareMathOperator{\Gam}{Gam }
\DeclareMathOperator{\BetaDist}{Beta }
\DeclareMathOperator{\Mult}{Mult }
\DeclareMathOperator{\BVN}{BVN }
\DeclareMathOperator{\LogN}{LogN }
\DeclareMathOperator{\Wei}{Wei }
\DeclareMathOperator{\Pareto}{Pareto }
\DeclareMathOperator{\Erlang}{Erlang }
\DeclareMathOperator{\Dom}{Dom }
\DeclareMathOperator{\Var}{Var }
\DeclareMathOperator{\Cov}{Cov }
\DeclareMathOperator{\Corr}{Corr }
\DeclareMathOperator{\supp}{supp }
\DeclareMathOperator{\sd}{sd }
\DeclareMathOperator{\HG}{HG }
\DeclareMathOperator{\mgf}{mgf}
\DeclareMathOperator{\bias}{bias}
\DeclareMathOperator{\MSE}{MSE}


\setsidenotefont{\color{base16-eighties-light}\footnotesize}
\setmarginnotefont{\color{base16-eighties-light}\footnotesize}

\DeclareMathOperator{\VaR}{VaR}
\DeclareMathOperator{\TVaR}{TVaR}

\begin{document}
\hypersetup{pageanchor=false}
\maketitle
\hypersetup{pageanchor=true}
\tableofcontents

\chapter*{\faBook \enspace List of Definitions}
\addcontentsline{toc}{chapter}{List of Definitions}
\theoremlisttype{all}
\listtheorems{defn}

\chapter*{\faCoffee \enspace List of Theorems}
\addcontentsline{toc}{chapter}{List of Theorems}
\theoremlisttype{allname}
\listtheorems{axiom,lemma,thm,crly,propo}

\chapter{Lecture 1 Sep 06}%
\label{chp:lecture_1_sep_06}
% chapter lecture_1_sep_06

\section{Introduction and Overview}%
\label{sec:introduction_and_overview}
% section introduction_and_overview

\paragraph{Course Objective} In Loss Model I, the focus of our study is to learn the basic methods which are used by insurers to quantify risk from mathematical/statistical models, in order for insurers to make various decisions\sidenote{e.g. setting premiums, control expenses, deciding for reinsurance, etc.}. By quantifying risk, it helps us monitor underlying risks so that not only are we aware of them, but also so that we can take actions or preventive measures against them.

Our main interest of this course is:
\begin{itemize}
  \item to quantify and seek protection against the loss of funds due either to \hlnoteb{too many claims} or \hlnoteb{a few large claims};
  \item to reduce adverse financial impact of random events that prevent the realization of reasonable expectations.
\end{itemize}

\newthought{The main model that shall be the focus} of this course is \hlnoteb{models for liability risk}.

\begin{defn}[Liability Risk]\index{Liability Risk}
\label{defn:liability_risk}
  A \hlnoteb{liability risk} is a risk that insurance companies assume by selling insurance contracts.
\end{defn}

In particular, the liability that we shall focus on is \hlnotea{insurance claims}.\marginnote{Many of the models that we shall see later in the course are also applied for other types of risks, e.g. investment risk, credit risk, liquidity risk, and operational risk.}

\newthought{We are interested} in modelling the total amount of claims, i.e. the \hldefn{aggregate claim amount}, of a group fo insurance policies over a given period of time. In the actuarial literature, there are two main approaches that have been proposed to model the aggrement claim amount of an insurance portfolio, namely:
\begin{itemize}
  \item individual risk model;
  \item collective risk model.
\end{itemize}

\subsection{Individual Risk Model}%
\label{sub:individual_risk_model}
% subsection individual_risk_model

\begin{defn}[Individual Risk Model]\index{Individual Risk Model}
\label{defn:individual_risk_model}
  In an \hlnoteb{individual risk model}, the aggregate claim is modeled by
  \begin{equation*}
    S = \sum_{i=1}^{n} Z_i
  \end{equation*}
  where $n$ is a \hlnotea{deterministic}\sidenote{i.e. fixed} integer that represents the \hlnotec{total number of insurance policies}, and $Z_i$ is a random variable for the \hlnotec{potential loss of the $i$\textsuperscript{th} insurance policy.}
\end{defn}

\begin{note}
  Since a policy may or may not incur a loss\sidenote{Since a claim may or may not be made!}, we have that
  \begin{equation*}
    P(Z_i = 0) > 0.
  \end{equation*}
  Thus, in an individual risk model, we may also express the aggregate claim amount as
  \begin{equation*}
    S = \sum_{i=1}^{n} X_i I_i
  \end{equation*}
  where $I_i$ is the indicator function about the claimant of policy $i$, while $X_i$ represents the size of the claim(s) for the $i$\textsuperscript{th} policy provided that there is a claim.\sidenote{\hlimpo{This is actually incorrect, despite being in the recommended textbook. See \cref{sec:individual_risk_model_an_alternate_view}.}}
\end{note}

However, in an individual risk model, according to Dhaene and Vyncke (2010)\cite{DhaeneVyncke2010},

\begin{quotebox}{base16-eighties-red}
  A third type of error that may arise when computing aggregate claims follows from the fact that the assumption of mutual independency of the individual claim amounts may be violated in practice.
\end{quotebox}

Due to complications such as this, the individual risk model will not be the focus of our studies.

% subsection individual_risk_model (end)

\subsection{Collective Risk Model}%
\label{sub:collective_risk_model}
% subsection collective_risk_model

\begin{defn}[Collective Risk Model]\index{Collective Risk Model}
\label{defn:collective_risk_model}
  In a \hlnoteb{collective risk model}, the aggregate claim is modeled by
  \begin{equation*}
    S = \sum_{i=1}^{N} X_i, \end{equation*}
  where $N$ is a non-negative integer-valued random variable that denotes \hlnotec{the number of claims among a given set of policies}, while $X_i$ denotes the \hlnotec{size of the $i$\textsuperscript{th} policy.}
\end{defn}

\begin{note}
  In a collective risk model, we need to determine:
  \begin{itemize}
    \item the distribution of the total number of claims for the entire portfolio, i.e. the distribution of $N$; and
    \item the distribution of the loss amount per claim, i.e. the distribution of $X_i$.
  \end{itemize}
\end{note}

% subsection collective_risk_model (end)

In this course, the primary focus of our studies will be on \hlnotea{collective risk models}.

\paragraph{Terminologies} To end today's lecture, the following terminologies are introduced:

\begin{defn}[Severity Distribution]\index{Severity Distribution}
\label{defn:severity_distribution}
  The \hlnoteb{severity distribution} is the distribution of the loss amount of the amount paid by the insurer on a given loss/claim.
\end{defn}

\begin{defn}[Frequency Distribution]\index{Frequency Distribution}
\label{defn:frequency_distribution}
  The \hlnoteb{frequency distribution} is the distributino fo the number of losses/claims paid by the insurer over a given period of time.
\end{defn}

\begin{note}
  The frequency distribution is typically a discrete distribution.
\end{note}

\begin{defn}[Aggrement Payment / Loss]\index{Aggrement Payment}\index{Aggregate Loss}
\label{defn:aggrement_payment_loss}
  The \hlnoteb{aggregate payment (loss)} is the total amout of all claim payments (losses) over a given period of time.
\end{defn}

\begin{note}
  There is a distinction between an aggregate payment and an aggregate loss, since an aggregate payment is ``essentially'' an aggregate loss after certain claim adjustments, such as deductibles, limits, and coinsurance.
\end{note}

% section introduction_and_overview (end)

% chapter lecture_1_sep_06 (end)

\chapter{Lecture 2 Sep 11th}%
\label{chp:lecture_2_sep_11th}
% chapter lecture_2_sep_11th

\section{Review of Probability Theory}%
\label{sec:review_of_probability_theory}
% section review_of_probability_theory

Firstly, we shall review the definition of a random variable.

\begin{defn}[Random Variable]\index{Random Variable}
\label{defn:random_variable}
  Let $\Omega$ be a sample space and $\mathcal{F}$ its $\sigma$-algebra\sidenote{For definitions of $\Omega$ and $\mathcal{F}$, see notes on STAT330.}. A \hlnoteb{random variable} (rv) $X : \Omega \to (\Omega, \mathcal{F})$ is a function from a possible set of outcomes to a measurable space $(\Omega, \mathcal{F})$. Within the context of our interest, $X$ is real-valued, i.e. $(\Omega, \mathcal{F}) = \mathbb{R}$.
\end{defn}

\subsection{Discrete Random Variables}%
\label{sub:discrete_random_variables}
% subsection discrete_random_variables

\begin{defn}[Discrete Random Variable]\index{Discrete Random Variable}
\label{defn:discrete_random_variable}
  A \hlnoteb{discrete random variable} (drv) is an rv $X$ that takes only countable (finite) real values.
\end{defn}

\begin{note}
  Let $X$ be a drv.
  \begin{itemize}
    \item The \hlnotea{probability mass function} (pmf) of $X$ is: for $i \in \mathbb{N}$,
      \begin{equation*}
        p(x_i) = P(X = x_i)
      \end{equation*}

    \item The \hlnotea{cumulative distribution function} (cdf) of $X$ is
      \begin{equation*}
        F(x) = P(X \leq x) = \sum_{x_i \leq x} p(x_i).
      \end{equation*}

    \item The $k$th \hldefn{moment} of $X$ is\sidenote{This implicitly uses the \href{https://en.wikipedia.org/wiki/Law_of_the_unconscious_statistician}{Law of the Unconcious Statistician}.}
      \begin{equation*}
        E[X^k] = \sum_{i \in \mathbb{N}} x_i^k p(x_i)
      \end{equation*}
      if $E[X^k]$ is finite.

    \item Some commonly seen/introduced discrete distributions are: Poisson, Binomial, Negative Binomial
  \end{itemize}
\end{note}

\begin{eg}
  Let $X$ take values from $\{x_1, x_2, x_3, x_4\}$, and
  \begin{equation*}
    p(x_i) = P(X = x_i) \text{ for } i = 1, 2, 3, 4.
  \end{equation*}
  The cdf of $X$ is\marginnote{
    It is recommended to visualize the cdf first before putting it down in pencil.
    \resizebox{4.5cm}{!}{
    \begin{tikzpicture}
      % axes
      \draw[->] (0, 0) -- (0, 5) node[above] {$F(x)$};
      \draw[->] (0, 0) -- (5, 0) node[right] {$x$};
      \node[below=1.5mm] at (1, 0) {$x_1$};
      \node[below=1.5mm] at (2, 0) {$x_2$};
      \node[below=1.5mm] at (3, 0) {$x_3$};
      \node[below=1.5mm] at (4, 0) {$x_4$};

      % cdf
      \draw[-,line width=0.5mm] (0, 0) -- (1, 0);
      \draw[-,line width=0.5mm] (1, 1) -- (2, 1);
      \draw[-,line width=0.5mm] (2, 2) -- (3, 2);
      \draw[-,line width=0.5mm] (3, 3) -- (4, 3);
      \draw[->,line width=0.5mm] (4, 4) -- (5, 4);
      \node[circle,inner sep=2pt,draw] at (1, 0) {};
      \node[circle,inner sep=2pt,draw] at (2, 1) {};
      \node[circle,inner sep=2pt,draw] at (3, 2) {};
      \node[circle,inner sep=2pt,draw] at (4, 3) {};
      \node[circle,inner sep=2pt,fill] at (1, 1) {};
      \node[circle,inner sep=2pt,fill] at (2, 2) {};
      \node[circle,inner sep=2pt,fill] at (3, 3) {};
      \node[circle,inner sep=2pt,fill] at (4, 4) {};
      \draw[dotted] (4, 4) -- (0, 4) node[left] {$1$};

      % jumps
      \draw[dotted] (1, 1) -- (1, 0) node[midway,left] {$p(x_1)$};
      \draw[dotted] (2, 2) -- (2, 1) node[midway,left] {$p(x_2)$};
      \draw[dotted] (3, 3) -- (3, 2) node[midway,left] {$p(x_3)$};
      \draw[dotted] (4, 4) -- (4, 3) node[midway,left] {$p(x_4)$};
    \end{tikzpicture}
    }
  }
  \begin{equation*}
    F(x) = \begin{cases}
      0               & x < x_1 \\
      p(x_1)          & x_1 \leq x < x_2 \\
      p(x_1) + p(x_2) & x_2 \leq x < x_3 \\
      1 - p(x_4)      & x_3 \leq x < x_4 \\
      1               & x \geq x_4
    \end{cases}
  \end{equation*}
\end{eg}

\begin{note}
  \begin{itemize}
    \item It is important that we stress the need for showing \hlnotea{right continuity} in the graph.
    \item Note that the cdf always sums to $1$.
    \item The ``\hlnotea{jumps}'' at $x_i$ correspond to $p(x_i)$, for $i = 1, 2 ,3, 4$.
  \end{itemize}
\end{note}

\begin{defn}[Probability Generating Function]\index{Probability Generating Function}
\label{defn:probability_generating_function}
  Suppose a drv $X$ only takes \hlimpo{non-negative integer values}. The \hlnoteb{probability generating function} (pgf) of $X$ is defined as
  \begin{equation*}
    G(z) = E\left[ z^X \right] = \sum_{k=1}^{\infty} z^k p(k)
  \end{equation*}
  where we note that if $\max X = n$, then $p(m) = 0$ for all $m > n$.
\end{defn}

\begin{note}
  \begin{itemize}
    \item The pgf uniquely identifies the distribution of the drv\sidenote{\faHandPaperO This was given as is without proof, and I cannot find any resources that proves this.}.
    \item To get the probability for $k \in \{0, 1, 2, ...\}$, we simply need to do
      \begin{equation*}
        p(k) = \frac{1}{k!} G^{(k)}(x) \at{x = 0}{}.
      \end{equation*}
  \end{itemize}
\end{note}

\begin{eg}[Lecture Slides: Example 1]
  Consider a drv $X$ with pmf
  \begin{equation*}
    p(x) = P(X = x) = \begin{cases}
      0.5 & x = 0 \\
      0.4 & x = 1 \\
      0.1 & x = 2
    \end{cases}
  \end{equation*}
  Its cdf is\marginnote{
    \resizebox{4.5cm}{!}{
    \begin{tikzpicture}
      % axes
      \draw[->] (0, 0) -- (0, 5) node[above] {$F(x)$};
      \draw[->] (0, 0) -- (3.5, 0) node[right] {$x$};
      \node[below=1.5mm] at (1, 0) {$0$};
      \node[below=1.5mm] at (2, 0) {$1$};
      \node[below=1.5mm] at (3, 0) {$2$};

      % cdf
      \draw[-,line width=0.5mm] (0, 2) -- (1, 2);
      \draw[-,line width=0.5mm] (1, 3.6) -- (2, 3.6);
      \draw[->,line width=0.5mm] (2, 4) -- (3.5, 4);
      \node[circle,inner sep=2pt,draw] at (1, 2) {};
      \node[circle,inner sep=2pt,draw] at (2, 3.6) {};
      \node[circle,inner sep=2pt,fill] at (0, 2) {};
      \node[circle,inner sep=2pt,fill] at (1, 3.6) {};
      \node[circle,inner sep=2pt,fill] at (2, 4) {};
      \draw[dotted] (3, 4) -- (0, 4) node[left] {$1$};

      % jumps
      \draw[dotted] (0, 0) -- (0, 2) node[midway,left] {$0.5$};
      \draw[dotted] (1, 2) -- (1, 3.6) node[midway,left] {$0.4$};
      \draw[dotted] (2, 3.6) -- (2, 4) node[midway,left] {$0.1$};
    \end{tikzpicture}
    }
  }
  \begin{equation*}
    F(x) = P(X \leq x) \begin{cases}
      0   & x < 0 \\
      0.5 & 0 \leq x < 1 \\
      0.9 & 1 \leq x < 2 \\
      1   & x \geq 2
    \end{cases}
  \end{equation*}
  and its pgf is
  \begin{equation*}
    G(z) = E\left[ z^X \right] = 0.5 + 0.4z + 0.1z^2.
  \end{equation*}
\end{eg}

% subsection discrete_random_variables (end)

\subsection{Continuous Random Variables}%
\label{sub:continuous_random_variables}
% subsection continuous_random_variables

\begin{defn}[Continuous Random Variable]\index{Continuous Random Variable}
\label{defn:continuous_random_variable}
  A \hlnoteb{continuous random variable} (crv) takes on a continuum of values.
\end{defn}

\begin{note}
  Let $X$ be a crv.
  \begin{itemize}
    \item $\exists f : X \to \mathbb{R}$ called a \hlnotea{probability density function} (pdf) such that its cdf is
      \begin{equation*}
        F(x) = \int_{-\infty}^{x} f(y) \dif{y},
      \end{equation*}
      and consequently by the \hlnotea{Fundamental Theorem of Calculus}, we have
      \begin{equation*}
        f(x) = F'(x).
      \end{equation*}

    \item The $k$th moment of $X$ is
      \begin{equation*}
        E[X^k] = \int_{x} x^k f(x) \dif{x} 
      \end{equation*}
      so long that $E[X^k]$ is defined.
      
    \item Some commonly introduced distributions are: Uniform, Exponential, Gamma, Weibull, and Normal.
  \end{itemize}
\end{note}

\begin{defn}[Moment Generating Function]\index{Moment Generating Function}
\label{defn:moment_generating_function}
Let $X$ be an rv. The \hlnoteb{moment generating function} (mgf)\marginnote{The mgf is also defined for drvs.} of $X$ is, for $t \in \mathbb{R}$ (appropriately so),
  \begin{equation*}
    M_X(t) = E\left[e^{tX}\right] = \int_{x} e^{tx} f(x) \dif{x}
  \end{equation*}
  provided that the integral is well-defined.
\end{defn}

\begin{note}
  \begin{itemize}
    \item The mgf uniquely determines the distribution of its rv\sidenote{\faHandPaperO This shall, also, not be proven in this course.}

    \item With the mgf, we can obtain the $k$th moment of an rv $X$ by
      \begin{equation*}
        E\left[X^k\right] = \frac{d^k}{dt^k} M_X(t) \at{t = 0}{}
      \end{equation*}
  \end{itemize}
\end{note}

\begin{eg}[Lecture Notes: Example 2]
  Consider an exponential rv $X$ with pdf\sidenote{When not explicitly stated, it shall be assumed that domains at which we did not specify $x$ shall have probability $0$.}
  \begin{equation*}
    f(x) = 0.1e^{-0.1x}, \; x > 0.
  \end{equation*}
  Its cdf is
  \begin{equation*}
    F(x) = \int_{-\infty}^{x} f(y) \dif{y} = \begin{cases}
      1 - e^{-0.1 x} & x \geq 0 \\
      0              & \text{otherwise}
    \end{cases}
  \end{equation*}
  and its mgf is
  \begin{align*}
    M_X(t) &= E\left[ e^{tX} \right] = \int_{0}^{\infty} e^{tx} 0.1 e^{-0.1x} \dif{x} \\
           &= 0.1 \int_{0}^{\infty} e^{( t - 0.1 )x} \dif{x} \\
           &= \frac{0.1}{0.1 - t}, \enspace t < 0.1,
  \end{align*}
  where we note that we must have $t < 0.1$, for otherwise the value of the exponent would render the integral undefined.
\end{eg}

\begin{defn}[Hazard Rate Function]\index{Hazard Rate Function}
\label{defn:hazard_rate_function}
  For a crv $X$, the \hlnoteb{hazard rate function} (aka \hldefn{failure rate}) of $X$ is defined as
  \begin{equation*}
    h(x) = \frac{f(x)}{\bar{F}(x)} = - \frac{d}{dx} \ln \bar{F}(x),
  \end{equation*}
  where $\bar{F}(x) = 1 - F(x)$ is the \hlnotea{survival function}\sidenote{You should be familiar with this if you have studied for Exam P.}
\end{defn}

\begin{note}
  \begin{itemize}
    \item We may also express the survival function in terms of the hazard rate by
      \begin{equation*}
        \bar{F}(x) = e^{- \int_{-\infty}^{x} h(y) \dif{y}}.
      \end{equation*}

    \item In terms of limits, we can express the hazard rate function, for small enough $\delta > 0$, as
      \begin{align*}
        h(x) &= \frac{f(x)}{\bar{F}(x)} = \frac{F'(x)}{\bar{F}(x)} \\
             &\approx \frac{F(x + \delta) - F(x)}{\delta \bar{F}(x)} \\
             &= \frac{P(x < X \leq x + \delta)}{\delta F(X > x)} \\
             &= \frac{1}{\delta} P(x < X \leq x + \delta \mid X > x).
      \end{align*}
      We can make sense of this expression by recalling the notion of the probability of survival from Exam MLC\sidenote{This also tells us that the hazard rate gets its name from life insurance.}, where if a life has survived over $x$, the hazard rate is the probability that the life does not survive beyond another $\delta$ \sidenote{From the perspective of life insurance, the greater the probability, the more likely the claim is going to happen.}.
  \end{itemize}
\end{note}

% subsection continuous_random_variables (end)

% section review_of_probability_theory (end)

% chapter lecture_2_sep_11th (end)

\chapter{Lecture 3 Sep 13th}%
\label{chp:lecture_3_sep_13th}
% chapter lecture_3_sep_13th

\section{Review of Probability Theory (Continued)}%
\label{sec:review_of_probability_theory_continued}
% section review_of_probability_theory_continued

\subsection{Continuous Random Variables (Continued)}%
\label{sub:continuous_random_variables_continued}
% subsection continuous_random_variables_continued

\begin{eg}[Lecture Notes: Example 3]
  Suppose $X \sim \Wei(\theta, \tau)$ with pdf
  \begin{equation*}
    f(x) = \frac{\tau \left( \frac{x}{\theta} \right)^\tau e^{-\left( \frac{x}{\theta} \right)^\tau}}{x}, \quad x > 0,
  \end{equation*}
  where $\theta, \tau > 0$. Find its hazard rate function.
\end{eg}

\begin{solution}
  We first require the survival function:
  \begin{align*}
    \bar{F}(x) &= \int_{x}^{\infty} \frac{1}{y} \tau \left( \frac{y}{\theta} \right)^\tau e^{-\left( \frac{y}{\theta} \right)^\tau} \dif{y} \\
               &= \int_{\frac{x}{\theta}}^{\infty} \frac{1}{u} \tau u^\tau e^{-u^\tau} \dif{u} \qquad \text{ where } u = \frac{y}{\theta} \\
               &= \int_{\frac{x}{\theta}}^{\infty} \tau u^{\tau - 1} e^{-u^\tau} \dif{u} \\
               &= -e^{-u^\tau} \at{\frac{x}{\theta}}{\infty} = e^{-\left( \frac{x}{\theta} \right)^\tau}
  \end{align*}
  The hazard rate is therefore
  \begin{equation*}
    h(x) = \frac{f(x)}{\bar{F}(x)} = \frac{\tau}{x} \left( \frac{x}{\theta} \right)^\tau
  \end{equation*}
\end{solution}

% subsection continuous_random_variables_continued (end)

\subsection{Mixed Random Variable}%
\label{sub:mixed_random_variable}
% subsection mixed_random_variable

\begin{defn}[Mixed Random Variable]\index{Mixed Random Variable}
\label{defn:mixed_random_variable}
  We call $X$ a \hlnoteb{mixed random variable} (mixed rv) if it has both discrete and continuous components.
\end{defn}

\begin{note}
  \begin{itemize}
    \item Mixed rvs are important in modeling insurance claims, e.g., the loss amount is usually a continuous random variable with a probability mass at $0$.
  \end{itemize}
\end{note}

The following is a type of mixed random variable:

\begin{defn}[Deductibles]\index{Deductibles}
\label{defn:deductibles}
  Let $X$ be an rv and $d$ be a fixed value.
  \begin{equation*}
    [ X - d ]_+ = \begin{cases}
      X - d & x \geq d \\
      0     & \text{ otherwise }
    \end{cases}
  \end{equation*}
\end{defn}

\begin{note}
  If $X$ be an rv and $d$ a fixed value, the deductible $[X - d]_+$ has a mass point at $0$ since
  \begin{equation*}
    P( [ X - d ]_+ = 0 ) = P(X < d) > 0
  \end{equation*}
\end{note}

\begin{note}
  Let $\{ x_1, x_2, ... \}$ be a sequence of real numbers in an increasing order. Suppose $X$ is a rv that takes on values on the real, and has a \hlnotea{density function} $f$ on each interval $(x_i, x_{i + 1})$, and has \hlnotea{discrete mass points} at the boundaries of these intervals, i.e.\marginnote{In other words, we treat the discrete and continuous part of a mixed rv separately.}
  \begin{equation*}
    P(X = x_i) = p(x_i) > 0 \quad i \in \mathbb{N}.
  \end{equation*}
  Since $X$ is an rv, it must be the case that
  \begin{equation*}
    \sum_{i \in \mathbb{N}} p(x_i) + \sum_{i \in \mathbb{N}} \int_{x_i}^{x_{i + 1}} f(x) \dif{x}  = 1.
  \end{equation*}
  The cdf of a mixed rv $X$ is
  \begin{equation*}
    F(x) = P(X \leq x) = \sum_{i \in \mathbb{N}} p(x_i) \mathbb{1}_{\{x_i \leq x\}} + \sum_{i \in \mathbb{N}} \int_{x_i}^{x_{i + 1}} f(y) \mathbb{1}_{\{ y \leq x \} } \dif{y}.
  \end{equation*}
  The $k$th moment of $X$ is
  \begin{equation*}
    E\left[X^k\right] = \sum_{i \in \mathbb{N}} (x_i)^k p(x_i) + \sum_{i \in \mathbb{N}} \int_{x_i}^{x_{i + 1}} x^k f(x) \dif{x}.
  \end{equation*}
  The mgf of $X$ is
  \begin{equation*}
    M_X(t) = E\left[ e^{tX} \right] = \sum_{i \in \mathbb{N}} e^{tx_i} p(x_i) + \sum_{i \in \mathbb{N}} \int_{x_i}^{x_{i + 1}} e^{tx} f(x) \dif{x}.
  \end{equation*}
\end{note}

\begin{eg}[Lecture Notes: Example 4]
  Assume a claim amount of an insurance policy is modeled by a non-negative rv $X$ which has probability mass of $p$ and $0$, and otherwise continuous with a pdf $f$ over $(0, \infty)$. Find its cdf, $k$th moment, and mgf.
\end{eg}

\begin{solution}
  The cdf of $X$ is
  \begin{equation*}
    F(x) = \begin{cases}
      p + \int_{0}^{x} f(y) \dif{y} & x \geq 0 \\
      0                             & \text{ otherwise }
    \end{cases}
  \end{equation*}
  The $k$th moment of $X$ is
  \begin{equation*}
    E\left[ X^k \right] = \int_{0}^{\infty} x^k f(x) \dif{x}.
  \end{equation*}
  The mgf of $X$ is
  \begin{equation*}
    M_X(t) = p + \int_{0}^{\infty} e^{tx} f(x) \dif{x}.
  \end{equation*}
\end{solution}

% subsection mixed_random_variable (end)

% section review_of_probability_theory_continued (end)

\section{Distributional Quantities and Risk Measures}%
\label{sec:distributional_quantities_and_risk_measures}
% section distributional_quantities_and_risk_measures

\newthought{This chapter} introduces us to some \hlnotea{distributional quantities} for a given rv $X$. These distributional quantities are informative values to describe the characteristics of a risk.

\subsection{Distributional Quantities}%
\label{sub:distributional_quantities}
% subsection distributional_quantities

\begin{defn}[Central Moment]\index{Central Moment}
\label{defn:central_moment}
  The \hlnoteb{$k$th central moment} of an rv $X$ is defined as
  \begin{equation*}
    E\left[ (X - E(X))^k \right].
  \end{equation*}
\end{defn}

\begin{note}
  The second central moment is the \hldefn{variance}. The square root of the variance is the \hldefn{standard deviation}.
\end{note}

\begin{eg}[Lecture Notes: Example 5]
  Consider an rv $Y = \begin{cases} Y_1 & U = 1 \\ Y_2 & U = 2 \end{cases} \; $\sidenote{This notation is just syntatic sugar for saying $Y_1 = Y \mid ( U = 1 )$ and $Y_2 = Y \mid ( U = 2 )$.}, where $Y_1 = 0$, $Y_2 \sim \Exp(10)$, and $P(U = 1) = P(U = 2) = 0.5$.
  \begin{enumerate}
    \item Find the cdf of $Y$.
    \item Find the mean and variance of $Y$.
    \item Let $Z = \frac{1}{2} Y_1 + \frac{1}{2} Y_2$. Does $Z$ have the same distribution as $Y$? Answer this by solving the mean and variance of $Z$.
  \end{enumerate}
\end{eg}

\begin{solution}
  \begin{enumerate}
    \item Note that
      \begin{equation*}
        F(y) = P( Y_1 \leq y \mid U = 1 ) P(U = 1) + P( Y_2 \leq y \mid U = 2 ) P(U = 2).
      \end{equation*}
      Observe that
      \begin{equation*}
        P(Y_1 \leq y \mid U = 1) = \begin{cases}
          1 & y \geq 0 \\
          0 & y < 0
        \end{cases}
      \end{equation*}
      and
      \begin{equation*}
        P(Y_2 \leq y \mid U = 2) = \begin{cases}
          1 - e^{-10y} & y \geq 0 \\
          0            & y < 0
        \end{cases}
      \end{equation*}
      Therefore
      \begin{equation*}
        F(y) = \begin{cases}
          1 - \frac{1}{2} e^{-10 y} & y \geq 0 \\
          0                         & y < 0
        \end{cases}
      \end{equation*}

    \item The mean of $Y$ is
      \begin{equation*}
        E(Y) = E(Y \mid U = 1) P(U = 1) + E(Y \mid U = 2) P(U = 2) = 10 \cdot \frac{1}{2} = 5.
      \end{equation*}
      To calculate the variance of $Y$, we require
      \begin{align*}
        E\left[Y^2\right] &= E\left[Y^2 \mid U = 1\right] P(U = 1) + E\left[Y^2 \mid U = 2\right] P(U = 2) \\
                          &= ( \Var(Y_2) + E(Y_2)^2 ) \cdot \frac{1}{2} = 100.
      \end{align*}
      Therefore
      \begin{equation*}
        \Var(Y) = 100 - 5^2 = 75.
      \end{equation*}

    \item The mean of $Z$ is
      \begin{equation*}
        E[Z] = E[ \frac{1}{2} Y_1 + \frac{1}{2} Y_2 ] = 5.
      \end{equation*}
      The variance of $Z$ is
      \begin{equation*}
        \Var(Z) = \frac{1}{4} \Var(Y_1) + \frac{1}{4} \Var(Y_2) = 25.
      \end{equation*}
      Therefore, $Z$ does not have the same distribution as $Y$.
  \end{enumerate}
\end{solution}

\begin{defn}[Quantiles]\index{Quantiles}
\label{defn:quantiles}
  The \hlnoteb{$100p\%$ quantile} (or \hldefn{percentile}) of an rv $X$ is a set $\pi_p$ such that\marginnote{This definition may also be presented as: any number $\pi_p$ such that
  \begin{equation*}
    P(X < \pi_p) \leq p \leq P(X \leq \pi_p).
  \end{equation*}}
  \begin{equation*}
    \pi_p = \{ x \in X \mid P(X < x) \leq p \leq P(X \leq x) \}.
  \end{equation*}
\end{defn}

\begin{note}
  \begin{itemize}
    \item If $X$ is a continuous random variable, we have that $P(X < \pi_p) = P(X \leq \pi_p)$ and so we have to define the quantile as
      \begin{equation*}
        \pi_p = F^{-1} (p)
      \end{equation*}
      where $F^{-1}$ is the inverse function of $F$, the cdf of $X$.

    \item A quantile \hlimpo{can be a set of numbers}.
    \item $\pi_{0.5}$ is called the \hldefn{median} of $X$.
  \end{itemize}
\end{note}

\marginnote{Graphical method to interpret this notion will be included.}

\begin{eg}[Lecture Notes: Example 1]
  Find the $100p\%$ quantile of the loss distribution $F(x) = 1 - e^{-\frac{x}{\theta}}$, $x > 0$.
\end{eg}

\begin{solution}
  Note that $F$ is the cdf of an exponential distribution, which is a continuous distribution. Therefore,
  \begin{equation*}
    F(\pi_p) = 1 - e^{-\frac{\pi_p}{\theta}} = p \implies \pi_p = - \theta \ln (1 - p).
  \end{equation*}
\end{solution}

\begin{eg}[Lecture Notes: Example 2]
  Find the median $\pi_{0.5}$ for the following cdf
  \begin{equation*}
    F(x) = \begin{cases}
      0                                & x < 0 \\
      0.6 + 0.4 (1 - e^{-\frac{x}{3}}) & x \geq 0
    \end{cases}
  \end{equation*}
\end{eg}

\begin{solution}
  Since $F(0) = 0.6$ and $F$ is an increasing function, we have that $F(x) = 0$ for all $x < 0$. Therefore
  \begin{equation*}
    \pi_{0.5} = 0.
  \end{equation*}
\end{solution}

\begin{eg}[Lecture Notes: Example 3]
  Find the median $\pi_{0.5}$ for a loss $X$ with pmf
  \begin{equation*}
    p(0) = 0.25, \, p(1) = 0.25), \, p(2) = 0.5.
  \end{equation*}
\end{eg}

\begin{solution}
  The cdf of $X$ is
  \begin{equation*}
    F(x) = \begin{cases}
      0    & x < 0 \\
      0.25 & 0 \leq x < 1 \\
      0.5  & 1 \leq x < 2 \\
      1    & x \geq 2
    \end{cases}
  \end{equation*}
  since $F(x) = 0.5$ when $1 \leq x < 2$, we have that
  \begin{equation*}
    \pi_{0.5} = [1, 2].
  \end{equation*}
\end{solution}

% subsection distributional_quantities (end)

% section distributional_quantities_and_risk_measures (end)

% chapter lecture_3_sep_13th (end)

\chapter{Lecture 4 Sep 18th}%
\label{chp:lecture_4_sep_18th}
% chapter lecture_4_sep_18th

\section{Distributional Quantities and Risk Measures (Continued)}%
\label{sec:distributional_quantities_and_risk_measures_continued}
% section distributional_quantities_and_risk_measures_continued

\subsection{Risk Measures}%
\label{sub:risk_measures}
% subsection risk_measures

\begin{defn}[Risk Measure]\index{Risk Measure}
\label{defn:risk_measure}
  A \hlnoteb{risk measure} is a mapping from the loss rv to the real line $\mathbb{R}$.
\end{defn}

Klugman, Panjer \& Wilmot (2012) \cite{KlugmanPanjerWillmot2012} on risk measure:

\begin{quotebox}{base16-eighties-yellow}
  The level of exposure to risk is often described by one number, or at least a small set of numbers. These numbers are necessarily functions of the model and are often called ‘key risk indicators’. Such key risk indicators indicate to risk managers the degree to which the company is subject to particular aspects of risk.
\end{quotebox}

To ensure its solvency, insurers will have to charge on these risks, i.e. we have to \hlnotea{price these exposures to risks}.

\begin{defn}[Premium Principle]\index{Premium Principle}
\label{defn:premium_principle}
  A \hlnoteb{premium principle} (or \hldefn{insurance pricing}) is a rule for assigning a premium to an insurance risk.
\end{defn}

\begin{note}
  The following are some of the common principles used by insurers:
  \begin{itemize}
    \item \hldefn{Expectation Principle}
      \begin{equation*}
        \Pi(X) = ( 1 + \theta ) E(X), \quad \theta > 0
      \end{equation*}
    \item \hldefn{Standard Deviation Principle}
      \begin{equation*}
        \Pi(X) = E(X) + \theta \sqrt{\Var(X)}, \quad \theta > 0
      \end{equation*}
    \item \hldefn{Dutch Principle}
      \begin{equation*}
        \Pi(X) = E(X) + \theta E( [ X - E(X) ]_+ ), \quad \theta > 0
      \end{equation*}
  \end{itemize}
\end{note}

One particular measure is known as the \hlnotea{Value-at-Risk} (VaR).

\subsubsection{Value-At-Risk}
\label{ssub:Value-At-Risk}

\begin{defn}[Value-at-Risk (VaR)]\index{Value-at-Risk}\index{VaR}
\label{defn:value_at_risk}
The \hlnoteb{Value-at-Risk (VaR)} is a \hlnotea{quantile} of the distribution of aggregate losses, i.e. the $VaR$ of a risk $X$ at the $100\%p$ level is defined as\sidenote{I must find out why we define using $\inf$ instead of $\min$ (see following remark), and I will not take ``safe definition'' as an answer without full justification.}
  \begin{align*}
    \pi_p = \VaR_p (X) &= \inf \{ x \in \mathbb{R} : P (X > x) \leq 1 - p \} \\
               &= \inf \{ x \in \mathbb{R} : P (X \leq x) \geq p \}.
  \end{align*}
\end{defn}

\begin{note}
  \begin{itemize}
    \item $\VaR$ is often called a \hldefn{quantile risk measure}.
    \item $\VaR$ is the standard risk measure used to evaluate exposure to risks.
    \item $\VaR$ measures the amount of capital required by the insurer to remain solvent, with high certainty, in the face of large claims.
    \item In practice, $p$ is generally high: $99.95\%$ or as low as $95\%$.
  \end{itemize}
\end{note}

\begin{remark}
  Observe that\marginnote{This remark basically points out that the left endpoint of the interval $B$ is always included, which should be quite clear by right-continuity of $F$.}
  \begin{equation*}
    B = \{ x \in \mathbb{R} \mid F_X(x) \geq p \} = (A, \infty) \text{ or } [A, \infty)
  \end{equation*}
  for some $A \in \mathbb{R}$, since $F$ is an increasing function. Now let $x_0 \in B$ such that
  \begin{equation*}
    F(x_0) = P(X \leq x_0) \geq p \quad \land \quad F(x_0-) = P(X < x_0) \leq p,
  \end{equation*}
  i.e. it is not necessary that $P(X = x_0) = p$ (see the two example graphs on the margin).
  \begin{marginfigure}
    \begin{tikzpicture}
      \draw[->] (0, 0) -- (4, 0) node[right] {$x$};
      \draw[->] (0, 0) -- (0, 4) node[above] {$F(x)$};
      \draw (0, 1) -- (1, 1);
      \draw (1, 2) -- (2, 2);
      \draw[->] (2, 3) -- (4, 3);
      \node[circle,fill,inner sep=1pt] at (1, 2) {};
      \node[circle,fill,inner sep=1pt] at (2, 3) {};
      \node[circle,draw,inner sep=1pt] at (1, 1) {};
      \node[circle,draw,inner sep=1pt] at (2, 2) {};
      \draw[dotted] (4, 1.5) -- (0, 1.5) node[left] {$p$};
      \draw[dotted] (1, 2) -- (1, 0) node[below] {$x_0$};
    \end{tikzpicture}
    \caption{Discrete cdf}
  \end{marginfigure}
  \begin{marginfigure}
    \begin{tikzpicture}
      \draw[->] (0, 0) -- (4, 0) node[right] {$x$};
      \draw[->] (0, 0) -- (0, 4) node[above] {$F(x)$};
      \draw[->,smooth,domain=0:4] plot (\x,{sqrt(\x)});
      \draw[dotted] (4, 1.5) -- (0, 1.5) node[left] {$p$};
      \draw[dotted] (2.25, 2) -- (2.25, 0) node[below] {$x_0$};
      \node[circle,fill,inner sep=1pt] at (2.25,1.5) {};
    \end{tikzpicture}
    \caption{Continuous cdf}
  \end{marginfigure}
  Let $\{ x_n \}_{n \in \mathbb{N}}$ be a decreasing sequence of points on $\mathbb{R}$ such that $x_n \to x_0$ as $n \to \infty$. Since $F$ is right-continuous, we have that $F(x_n) \to F(x_0)$ as $n \to \infty$. Therefore,
  \begin{equation*}
    B = [ x_0 , \infty )
  \end{equation*}\marginnote{The lecturer asserts that we can really define $\VaR$ using $\min$ instead of $\inf$, but even with this, I am not completely satisfied or convinced.}
  This justifies the definition of $\pi_p$.
\end{remark}

\begin{note}
  \begin{itemize}
    \item Note that by definition, we have
      \begin{equation*}
        P(X < \pi_p) \leq p \leq P(X \leq \pi_p)
      \end{equation*}
    \item If $X$ is a crv whose cdf is strictly increasing, i.e. no constant points, then
      \begin{equation*}
        \pi_p = F^{-1}(p)
      \end{equation*}
      since $P(X < \pi_p) = P(X \leq \pi_p)$.
  \end{itemize}
\end{note}

\begin{warning}[Shortcomings of $\VaR$]
  \begin{itemize}
    \item $\VaR$ cannot tell us the size of the potential loss in the $100(1 - p)\%$ cases, making it difficult for us to prepare the right amount in order to safeguard against insolvency.
    \item $\VaR$ actually fails to satisfy properties to be a \hlnotea{coherent risk measure}\sidenote{See \cref{sec:coherent_risk_measure}.}, for example, \hlnotea{subadditivity}.
    \item $\VaR$ is extensively used in financial risk management of trading risk over a fixed (usually short) time period, which are usually normally distributed, and $\VaR$ satisfies all coherency requirements.
    \item In insurance losses, instead of normal distributions, in general, skewed distributions are used, and in this cases, $\VaR$ is flawed as it lacks subadditivity.
  \end{itemize}
\end{warning}

\begin{eg}
  Suppose that $X$ has a Pareto distribution with cdf
  \begin{equation*}
    F(x) = 1 - \left( \frac{\theta}{x + \theta} \right)^\alpha , \quad x > 0
  \end{equation*}
  where $\alpha, \theta > 0$. Find $\VaR_p(X)$.
\end{eg}

\begin{solution}
  Since $F$ is continuous and strictly increasing, we have that
  \begin{equation*}
    \pi_p = F^{-1}(p) = \theta \left[ (1 - p)^{-\frac{1}{\alpha}} - 1 \right]
  \end{equation*}
\end{solution}

\begin{eg}
  Find $\VaR_{0.95}(X)$, $\VaR_{0.5}(X)$, and $\VaR_{0.3}(X)$ for a random loss with pmf
  \begin{equation*}
    p(0) = 0.25, \, p(1) = 0.25, \, \text{ and } p(2) = 0.5.
  \end{equation*}
\end{eg}

\begin{solution}
  Note that the cdf of $X$ is
  \begin{equation*}
    F(x) = \begin{cases}
      0    & x < 0 \\
      0.25 & 0 \leq x < 1 \\
      0.5  & 1 \leq x < 2 \\
      1    & x \geq 2
    \end{cases}.
  \end{equation*}
  Therefore,
  \begin{equation*}
    \VaR_{0.95}(X) = 2, \, \VaR_{0.5}(X) = 1, \, \text{ and } \VaR_{0.3}(X) = 1.
  \end{equation*}
\end{solution}

\subsubsection{Tail-Value-at-Risk}
\label{ssub:Tail-Value-at-Risk}

To compensate for the weakness of $\VaR$ at giving us the size of the the loss $X$ of which we cannot measure, we use the \hlnotea{Tail-Value-at-Risk}.

\begin{defn}[Tail-Value-at-Risk (TVaR)]\index{Tail-Value-at-Risk}
\label{defn:tail_value_at_risk}
  Let $X$ be an rv. The \hlnoteb{Tail-Value-at-Risk (TVaR)} of $X$ at the $100p\%$ level, denoted as $\TVaR_p(X)$, is defined as the average of all $\VaR$ values above the level $p$, and expressed as
  \begin{equation*}
    \TVaR_p(X) = \frac{1}{1 - p} \int_{p}^{1} \VaR_\alpha(X) \dif{\alpha} = \frac{1}{1 - p} \int_{p}^{1} \pi_\alpha \dif{\alpha}
  \end{equation*}
\end{defn}

\begin{remark}
  By considering the average of $\VaR$ from $p$'s going up to $1$, we take into account even the extreme cases of which $\VaR$ fails to account for.
\end{remark}

Perhaps a clearer definition would be the following, although the expression is only sensible if $X$ is a crv:

\begin{defn}[Tail-Value-at-Risk (TVaR)]\index{Tail-Value-at-Risk}
\label{defn:tail_value_at_risk_v2}
  Let $X$ be an rv. The \hlnoteb{Tail-Value-at-Risk (TVAR)} of $X$ at the $100p\%$ level, denoted $\TVaR_p(X)$, is the expected loss given that the loss exceeds the $100p$ percentile (or quantile) of the distribution of $X$, expressible as
  \begin{equation*}
    \TVaR_p(X) = E[ X \mid X > \pi_p ] = \frac{1}{\bar{F}(\pi_p)} \int_{\pi_p}^{\infty} x f(x) \dif{x}.
  \end{equation*}
\end{defn}

Note that the two definitions agree with one another:

\begin{align*}
  \frac{1}{1 - p} \int_{p}^{1} \pi_\alpha \dif{\alpha} &= \frac{1}{1 - F(\pi_p)} \int_{p}^{1} F^{-1}(\alpha) \dif{\alpha} \\
                                                       &= \frac{1}{\bar{F}(\pi_p)} \int_{\pi_p}^{1} x f(x) \dif{x}
\end{align*}
where we let $\alpha = F(x)$ as substitution.

\begin{note}
  While it is not difficult to notice that
  \begin{equation*}
    \TVaR_p(X) \geq \VaR_p(X),
  \end{equation*}
  the proof is also simple:
  \begin{align*}
    \TVaR_p(X) &= \frac{1}{1 - p} \int_{p}^{1} \pi_\alpha \dif{\alpha} \\
               &\geq \frac{1}{1 - p} \pi_p \int_{p}^{1} \dif{\alpha} = \pi_p = \VaR_p(X).
  \end{align*}
\end{note}

\begin{eg}
  Find $\TVaR_p(X)$ for $X \sim \Exp(\theta)$.
\end{eg}

\begin{solution}
  Since $X$ is a crv, and $F(x) = 1 - e^{- \frac{x}{\theta}}$, we have that
  \begin{equation*}
    \pi_p = F^{-1}(p) = - \theta \ln (1 - p).
  \end{equation*}
  Therefore,
  \begin{align*}
    \TVaR_p(X) &= \frac{1}{1 - p} \int_{p}^{1} \pi_\alpha \dif{\alpha} = \frac{- \theta}{1 - p} \int_{p}^{1} \ln (1 - \alpha) \dif{\alpha} \\
               &= \frac{- \theta}{1 - p} \int_{-\infty}^{\ln (1 - p)} ue^u \dif{u} \quad \text{ let } u = \ln ( 1 - \alpha ) \\
               &= \frac{-\theta}{1 - p} \left[ ue^u \at{-\infty}{\ln (1 - p)} - \int_{-\infty}^{\ln(1-p)} e^u \dif{u} \right] \text{ by IBP } \\
               &= \frac{-\theta}{1 - p} \left[ (1 - p) \ln (1 - p) - ( 1 - p ) \right]\\
               &= \theta [ 1 - \ln (1 - p) ]
  \end{align*}
\end{solution}

\begin{note}
  From the last example, by the memoryless property of $\Exp(\theta)$, notice that we may also do
  \begin{align*}
    \TVaR_p(X) &= E[ X \mid X > \pi_p ] = E [ X - \pi_p + \pi_p \mid X > \pi_p ] \\
               &= E[ X - \pi_p \mid X > \pi_p ] + E[ \pi_p \mid X > \pi_p ] \\
               &= E[ X ] + \pi_p
  \end{align*}
\end{note}

% subsection risk_measures (end)

% section distributional_quantities_and_risk_measures_continued (end)

% chapter lecture_4_sep_18th (end)

\appendix

\chapter{Additional Material}%
\label{chp:additional_material}
% chapter additional_material

\section{Individual Risk Model: An Alternate View}%
\label{sec:individual_risk_model_an_alternate_view}
% section individual_risk_model_an_alternate_view

\textit{This appendix serves to explain why our note of $Z_i = I_i X_i$ is wrong with as mush rigour as we can go for now. There may be hand-wavy parts, but those will be indicated.} 

We mentioned, as shown by Klugman, Panjer and Willmot (2012)\cite{KlugmanPanjerWillmot2012}, that for the \hyperref[defn:individual_risk_model]{Individual Risk Model}, the aggregate claim is modeled by
\begin{equation*}
  S = \sum_{i=1}^{n} Z_i
\end{equation*}
where $Z_i$ is a random variable for the potential loss of the $i$\textsuperscript{th} insurance policy, while $n$ is fixed. It is claimed that we can also express each $Z_i$ as
\begin{equation*}
  Z_i = I_i X_i
\end{equation*}
where $I_i$ is an indicator function given by
\begin{equation*}
  I_i(x) = \begin{cases}
    1 & \text{ if a claim occurs } \\
    0 & \text{ if there are no claims }
  \end{cases},
\end{equation*}
while $X_i$ is the size of the claim(s) for the $i$\textsuperscript{th} policy provided that there is a claim.

\newthought{One problem} that arises is: are $X_i$ and $I_i$ independent? They should be if we wish to define $Z_i$ in such a way. In fact, according to \\
\noindent\textcolor{base16-eighties-magenta}{\underline{Klugman et. al. in page 177}},

\begin{quotebox}{base16-eighties-magenta}
  Let $X_j = I_j B_j$, where $I_1, ..., I_n, B_1, ..., B_n$ are independent.
\end{quotebox}

where $X_j$ is our $Z_i$, $I_j$ is our $I_i$, and $B_j$ is our $X_i$.

\paragraph{$\S \; Z_i$ is not well-defined} Let us be explicit about the definitions of $I_i$ and $X_i$; we have
\begin{gather*}
  I_i = \mathbb{1}_{\{ Z_i > 0 \}} \\
  X_i = Z_i \mid Z_i > 0
\end{gather*}
However, we observe that such a defintion of $X_i$ is undefined on $Z_i = 0$. So the equation
\begin{equation*}
  Z_i = I_i X_i
\end{equation*}
is note well-defined.

\paragraph{$\S$ Independence of $I_i$ and $X_i$} We cannot actually tell if $I_i$ and $X_i$ are independent from each other, as it is equivalent to comparing apples with oranges\sidenote{In fact, I think this analogy fits our case perfectly so.}. Recall from our earlier courses, in particular STAT330, of the following notion:

\begin{defnnonum}[Probability Space]
\label{defn:probability_space}
  Let $\Omega$ be a sample space, and $\mathcal{F}$ a $\sigma$-algebra defined on $\Omega$\sidenote{Note that $(\Omega, \mathcal{F})$ is called a \hlnotea{measurable space}.}. A \hlnoteb{probability space} is the measurable space $(\Omega, \mathcal{F})$ with a \textcolor{base16-eighties-blue}{probability measure}, $f: \mathcal{F} \to [0, 1]$, defined on the space. We denote a probability space as $(\Omega, \mathcal{F}, f)$.
\end{defnnonum}

As mentioned in an earlier $\S$, $X_i$ is not defined on $Z_i = 0$, while $I_i$ is defined on $Z_i = 0$ \sidenote{\faHandPaperO \enspace This statement is hand-wavy.}. So the sample space for $X_i$ and $I_i$ are not the same, and so their probability measures are not the same as well. Therefore, \hlimpo{it is meaningless to ask if $X_i$ and $I_i$ are independent}.

Our best attempt at fixing this is probably the following: let
\begin{equation*}
  Z_i = \sum_{i=1}^{I_i} X_i,
\end{equation*}
which we can then have $X_i$ to be independent from $I_i$. However, interestingly so, this is a very similar approach to a \hyperref[defn:collective_risk_model]{Collective Risk Model}.

% section individual_risk_model_an_alternate_view (end)

\section{Coherent Risk Measure}%
\label{sec:coherent_risk_measure}
% section coherent_risk_measure

An excerpt from Klugman et. al. (2012)\cite{KlugmanPanjerWillmot2012}:

\begin{quotebox}{base16-eighties-magenta}
  The study of risk measures and their properties has been carried out by authors such as Wang. Specific desirable properties of risk measures were proposed as axioms in connection with risk pricing by Wang, Young, and Panjer and more generally in risk measurement by Artzer et. al. The Artzner paper introduced the concept of \textbf{coherence} and is considered to be the groundbreaking paper in risk measurement.
\end{quotebox}

Often, we use the function $\rho(X)$ to denote risk measures. One may think of $\rho(X)$ as \hlnotec{the amount of assets required to protect against adverse outcomes of the risk $X$}.

\begin{defn}[Coherent Risk Measure]\index{Coherent Risk Measure}
\label{defn:coherent_risk_measure}
  A \hlnoteb{coherent risk measure} is a risk measure $\rho(X)$ that has the following four properties for any two loss rvs $X$ and $Y$:
  \begin{enumerate}
    \item (\hlnotea{Subadditivity}) $\rho(X + Y) \leq \rho(X) + \rho(Y)$.
    \item (\hlnotea{Monotonicity}) If $X \leq Y$ for all possible outcomes, then $\rho(X) \leq \rho(Y)$.
    \item (\hlnotea{Positive homogeneity}) $\forall c \in \mathbb{R}_{> 0}$, $\rho(cX) = c\rho(X)$.
    \item (\hlnotea{Translation invariance}) $\forall c \in \mathbb{R}_{> 0}$, $\rho(X + c) = \rho(X) + c$
  \end{enumerate}
\end{defn}

\paragraph{Interpretation of the conditions}

\begin{itemize}
  \item \textbf{Subadditivity}
    \begin{itemize}
      \item the risk measure (and in return, the capital required to cover for it) for two risks combined will not be greater than for the risks to be treated separately;
      \item reflects the fact that there shuld be some diversification benefit from combining risks;
      \item this requirement is disputed: e.g. the merger of several small companies into a larger one exposes each of the small companies to the \hlnotea{reputational risks} of the others.
    \end{itemize}

  \item \textbf{Monotonicity}
    \begin{itemize}
      \item if one risk always has greater losses than the other under all circumstances\sidenote{Probabilistically, this means $P(X > Y) = 0$}, then the risk measure of the greater risk should always be greater than the other.
    \end{itemize}

  \item \textbf{Positive homogeneity}
    \begin{itemize}
      \item the risk measure is independent of the currency used to measure it;
      \item doubling the exposure to a particular risk requires double the capital, which is sensible as doubling provides no diversification.
    \end{itemize}

  \item \textbf{Translation invariance}
    \begin{itemize}
      \item there is no additional risk for an additional risk which has no additional uncertainty.
    \end{itemize}
\end{itemize}

% section coherent_risk_measure (end)

% chapter additional_material (end)

\backmatter

\pagestyle{plain}

\nobibliography*
\bibliography{references}

\chapter{List of Symbols and Abbreviations}%
\label{chp:list_of_symbols_and_abbreviations}
% chapter list_of_symbols_and_abbreviations

\begin{longtable}{l l}
  crv         & continuous random variable \\
  DFR         & Decreasing Failure Rate \\
  DMRL        & Decreasing Mean Residual Lifetime \\
  drv         & discrete random variable \\
  $e_X(d)$    & Mean Excess Loss / Mean Residual Lifetime \\
  $G_N(t)$    & probability generating function of random variable $N$ \\
  $h_X$       & hazard rate of random variable $X$ \\
  IFR         & Increasing Failure Rate \\
  IMRL        & Increasing Mean Residual Lifetime \\
  LER         & Loss Elimination Ratio \\
  mgf         & moment generating function \\
  pf          & probability function \\
  pdf         & probability density function \\
  pmf         & probability mass function \\
  pgf         & probability generating function \\
  rv          & random variable \\
  $\bar{F}_X$ & survival function of random variable $X$ \\
  $T_L$       & Amount Paid per Loss \\
  $T_P$       & Amount Paid Per Payment \\
  $\TVaR$     & Tail-Value-at-Risk \\
  $\VaR$      & Value-at-Risk
\end{longtable}

% chapter list_of_symbols_and_abbreviations (end)


\printindex

\end{document}

