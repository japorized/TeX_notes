% !TEX TS-program = lualatex
\documentclass[notoc,notitlepage]{tufte-book}
\usepackage{luatexja}
\usepackage{pxrubrica}
\usepackage{longtable,multirow}
\nonstopmode
\setcounter{secnumdepth}{3}
\setcounter{tocdepth}{5}

\renewcommand{\baselinestretch}{1.1}

\usepackage{CJKutf8}
\setcounter{secnumdepth}{3}
\setcounter{tocdepth}{3}

\renewcommand{\baselinestretch}{1.1}
\usepackage{geometry}
\geometry{letterpaper}
\usepackage[parfill]{parskip}
\usepackage{graphicx}

% Essential Packages
\usepackage{makeidx}
\makeindex
\usepackage{enumitem}
\usepackage[T1]{fontenc}
\usepackage{natbib}
\bibliographystyle{apalike}
\usepackage{ragged2e}
\usepackage{etoolbox}
\usepackage{amssymb}
\usepackage{fontawesome}
\usepackage{amsmath}
\usepackage{mathrsfs}
\usepackage{mathtools}
\usepackage{xparse}
\usepackage{tkz-euclide}
\usetkzobj{all}
\usepackage[utf8]{inputenc}
\usepackage{csquotes}
\usepackage[english]{babel}
\usepackage{marvosym}
\usepackage{pgf,tikz}
\usepackage{pgfplots}
\usepackage{fancyhdr}
\usepackage{array}
\usepackage{faktor}
\usepackage{float}
\usepackage{xcolor}
\usepackage{centernot}
\usepackage{silence}
  \WarningFilter*{latex}{Marginpar on page \thepage\space moved}
\usepackage{tcolorbox}
\tcbuselibrary{skins,breakable}
\usepackage{longtable}
\usepackage[amsmath,hyperref]{ntheorem}
\usepackage{hyperref}
\usepackage[noabbrev,capitalize,nameinlink]{cleveref}

% xcolor (scheme: base16 eighties)
\definecolor{base16-eighties-dark}{HTML}{2D2D2D}
\definecolor{base16-eighties-light}{HTML}{D3D0C8}
\definecolor{base16-eighties-magenta}{HTML}{CD98CD}
\definecolor{base16-eighties-red}{HTML}{F47678}
\definecolor{base16-eighties-yellow}{HTML}{E2B552}
\definecolor{base16-eighties-green}{HTML}{98CD97}
\definecolor{base16-eighties-lightblue}{HTML}{61CCCD}
\definecolor{base16-eighties-blue}{HTML}{6498CE}
\definecolor{base16-eighties-brown}{HTML}{D47B4E}
\definecolor{base16-eighties-gray}{HTML}{747369}

% hyperref Package Settings
\hypersetup{
    bookmarks=true,         % show bookmarks bar?
    unicode=true,          % non-Latin characters in Acrobat’s bookmarks
    pdftoolbar=false,        % show Acrobat’s toolbar?
    pdfmenubar=false,        % show Acrobat’s menu?
    pdffitwindow=true,     % window fit to page when opened
    colorlinks=true,
    allcolors=base16-eighties-magenta,
}

% tikz
\usepgfplotslibrary{polar}
\usepgflibrary{shapes.geometric}
\usetikzlibrary{angles,patterns,calc,decorations.markings}
\tikzset{midarrow/.style 2 args={
        decoration={markings,
            mark= at position #2 with {\arrow{#1}} ,
        },
        postaction={decorate}
    },
    midarrow/.default={latex}{0.5}
}
\def\centerarc[#1](#2)(#3:#4:#5)% Syntax: [draw options] (center) (initial angle:final angle:radius)
    { \draw[#1] ($(#2)+({#5*cos(#3)},{#5*sin(#3)})$) arc (#3:#4:#5); }

% enumitems
\newlist{inlinelist}{enumerate*}{1}
\setlist*[inlinelist,1]{%
  label=(\roman*),
}

% Theorem Style Customization
\setlength\theorempreskipamount{2ex}
\setlength\theorempostskipamount{3ex}

\makeatletter
\let\nobreakitem\item
\let\@nobreakitem\@item
\patchcmd{\nobreakitem}{\@item}{\@nobreakitem}{}{}
\patchcmd{\nobreakitem}{\@item}{\@nobreakitem}{}{}
\patchcmd{\@nobreakitem}{\@itempenalty}{\@M}{}{}
\patchcmd{\@xthm}{\ignorespaces}{\nobreak\ignorespaces}{}{}
\patchcmd{\@ythm}{\ignorespaces}{\nobreak\ignorespaces}{}{}

\renewtheoremstyle{break}%
  {\item{\theorem@headerfont
          ##1\ ##2\theorem@separator}\hskip\labelsep\relax\nobreakitem}%
  {\item{\theorem@headerfont
          ##1\ ##2\ (##3)\theorem@separator}\hskip\labelsep\relax\nobreakitem}
\makeatother

% ntheorem + framed
\makeatletter

% ntheorem Declarations
\theorempreskip{10pt}
\theorempostskip{5pt}
\theoremstyle{break}

\newtheorem*{solution}{\faPencil $\enspace$ Solution}
\newtheorem*{remark}{Remark}
\newtheorem{eg}{Example}[section]
\newtheorem{ex}{Exercise}[section]

    % definition env
\theoremprework{\textcolor{base16-eighties-blue}{\hrule height 2pt}}
\theoremheaderfont{\color{base16-eighties-blue}\normalfont\bfseries}
\theorempostwork{\textcolor{base16-eighties-blue}{\hrule height 2pt}}
\theoremindent10pt
\newtheorem{defn}{\faBook \enspace Definition}

    % definition env no num
\theoremprework{\textcolor{base16-eighties-blue}{\hrule height 2pt}}
\theoremheaderfont{\color{base16-eighties-blue}\normalfont\bfseries}
\theorempostwork{\textcolor{base16-eighties-blue}{\hrule height 2pt}}
\theoremindent10pt
\newtheorem*{defnnonum}{\faBook \enspace Definition}

    % theorem envs
\theoremprework{\textcolor{base16-eighties-magenta}{\hrule height 2pt}}
\theoremheaderfont{\color{base16-eighties-magenta}\normalfont\bfseries}
\theorempostwork{\textcolor{base16-eighties-magenta}{\hrule height 2pt}}
\theoremindent10pt
\newtheorem{thm}{\faCoffee \enspace Theorem}

\theoremprework{\textcolor{base16-eighties-magenta}{\hrule height 2pt}}
\theorempostwork{\textcolor{base16-eighties-magenta}{\hrule height 2pt}}
\theoremindent10pt
\newtheorem{propo}[thm]{\faTint \enspace Proposition}

\theoremprework{\textcolor{base16-eighties-magenta}{\hrule height 2pt}}
\theorempostwork{\textcolor{base16-eighties-magenta}{\hrule height 2pt}}
\theoremindent10pt
\newtheorem{crly}[thm]{\faSpaceShuttle \enspace Corollary}

\theoremprework{\textcolor{base16-eighties-magenta}{\hrule height 2pt}}
\theorempostwork{\textcolor{base16-eighties-magenta}{\hrule height 2pt}}
\theoremindent10pt
\newtheorem{lemma}[thm]{\faTree \enspace Lemma}

\theoremprework{\textcolor{base16-eighties-magenta}{\hrule height 2pt}}
\theorempostwork{\textcolor{base16-eighties-magenta}{\hrule height 2pt}}
\theoremindent10pt
\newtheorem{axiom}[thm]{\faShield \enspace Axiom}

    % theorem envs without counter
\theoremprework{\textcolor{base16-eighties-magenta}{\hrule height 2pt}}
\theoremheaderfont{\color{base16-eighties-magenta}\normalfont\bfseries}
\theorempostwork{\textcolor{base16-eighties-magenta}{\hrule height 2pt}}
\theoremindent10pt
\newtheorem*{thmnonum}{\faCoffee \enspace Theorem}

\theoremprework{\textcolor{base16-eighties-magenta}{\hrule height 2pt}}
\theorempostwork{\textcolor{base16-eighties-magenta}{\hrule height 2pt}}
\theoremindent10pt
\newtheorem*{propononum}{\faTint \enspace Proposition}

\theoremprework{\textcolor{base16-eighties-magenta}{\hrule height 2pt}}
\theorempostwork{\textcolor{base16-eighties-magenta}{\hrule height 2pt}}
\theoremindent10pt
\newtheorem*{crlynonum}{\faSpaceShuttle \enspace Corollary}

\theoremprework{\textcolor{base16-eighties-magenta}{\hrule height 2pt}}
\theorempostwork{\textcolor{base16-eighties-magenta}{\hrule height 2pt}}
\theoremindent10pt
\newtheorem*{lemmanonum}{\faTree \enspace Lemma}

\theoremprework{\textcolor{base16-eighties-magenta}{\hrule height 2pt}}
\theorempostwork{\textcolor{base16-eighties-magenta}{\hrule height 2pt}}
\theoremindent10pt
\newtheorem*{axiomnonum}{\faShield \enspace Axiom}

    % proof env
\theoremprework{\textcolor{base16-eighties-brown}{\hrule height 2pt}}
\theoremheaderfont{\color{base16-eighties-brown}\normalfont\bfseries}
\theorempostwork{\textcolor{base16-eighties-brown}{\hrule height 2pt}}
\newtheorem*{proof}{\faPencil \enspace Proof}

    % note and notation env
\theoremprework{\textcolor{base16-eighties-yellow}{\hrule height 2pt}}
\theoremheaderfont{\color{base16-eighties-yellow}\normalfont\bfseries}
\theorempostwork{\textcolor{base16-eighties-yellow}{\hrule height 2pt}}
\newtheorem*{note}{\faQuoteLeft \enspace Note}

\theoremprework{\textcolor{base16-eighties-yellow}{\hrule height 2pt}}
\theorempostwork{\textcolor{base16-eighties-yellow}{\hrule height 2pt}}
\newtheorem*{notation}{\faPaw \enspace Notation}

    % warning env
\theoremprework{\textcolor{base16-eighties-red}{\hrule height 2pt}}
\theoremheaderfont{\color{base16-eighties-red}\normalfont\bfseries}
\theorempostwork{\textcolor{base16-eighties-red}{\hrule height 2pt}}
\theoremindent10pt
\newtheorem*{warning}{\faBug \enspace Warning}

% more environments
\newtcolorbox{redquote}{
  blanker,enhanced,breakable,standard jigsaw,
  opacityback=0,
  coltext=base16-eighties-light,
  left=5mm,right=5mm,top=2mm,bottom=2mm,
  colframe=base16-eighties-red,
  boxrule=0pt,leftrule=3pt,
  fontupper=\itshape
}
\newtcolorbox{bluequote}{
  blanker,enhanced,breakable,standard jigsaw,
  opacityback=0,
  coltext=base16-eighties-light,
  left=5mm,right=5mm,top=2mm,bottom=2mm,
  colframe=base16-eighties-blue,
  boxrule=0pt,leftrule=3pt,
  fontupper=\itshape
}
\newtcolorbox{greenquote}{
  blanker,enhanced,breakable,standard jigsaw,
  opacityback=0,
  coltext=base16-eighties-light,
  left=5mm,right=5mm,top=2mm,bottom=2mm,
  colframe=base16-eighties-green,
  boxrule=0pt,leftrule=3pt,
  fontupper=\itshape
}
\newtcolorbox{yellowquote}{
  blanker,enhanced,breakable,standard jigsaw,
  opacityback=0,
  coltext=base16-eighties-light,
  left=5mm,right=5mm,top=2mm,bottom=2mm,
  colframe=base16-eighties-yellow,
  boxrule=0pt,leftrule=3pt,
  fontupper=\itshape
}
\newtcolorbox{magentaquote}{
  blanker,enhanced,breakable,standard jigsaw,
  opacityback=0,
  coltext=base16-eighties-light,
  left=5mm,right=5mm,top=2mm,bottom=2mm,
  colframe=base16-eighties-magenta,
  boxrule=0pt,leftrule=3pt,
  fontupper=\itshape
}

% ntheorem listtheorem style
\makeatother
\newlength\widesttheorem
\AtBeginDocument{
  \settowidth{\widesttheorem}{Proposition A.1.1.1\quad}
}

\makeatletter
\def\thm@@thmline@name#1#2#3#4{%
        \@dottedtocline{-2}{0em}{2.3em}%
                   {\makebox[\widesttheorem][l]{#1 \protect\numberline{#2}}#3}%
                   {#4}}
\@ifpackageloaded{hyperref}{
\def\thm@@thmline@name#1#2#3#4#5{%
    \ifx\#5\%
        \@dottedtocline{-2}{0em}{2.3em}%
            {\makebox[\widesttheorem][l]{#1 \protect\numberline{#2}}#3}%
            {#4}
    \else
        \ifHy@linktocpage\relax\relax
            \@dottedtocline{-2}{0em}{2.3em}%
                {\makebox[\widesttheorem][l]{#1 \protect\numberline{#2}}#3}%
                {\hyper@linkstart{link}{#5}{#4}\hyper@linkend}%
        \else
            \@dottedtocline{-2}{0em}{2.3em}%
                {\hyper@linkstart{link}{#5}%
                  {\makebox[\widesttheorem][l]{#1 \protect\numberline{#2}}#3}\hyper@linkend}%
                    {#4}%
        \fi
    \fi}
}

\makeatletter
\def\thm@@thmline@noname#1#2#3#4{%
        \@dottedtocline{-2}{0em}{5em}%
                   {{\protect\numberline{#2}}#3}%
                   {#4}}
\@ifpackageloaded{hyperref}{
\def\thm@@thmline@noname#1#2#3#4#5{%
    \ifx\#5\%
        \@dottedtocline{-2}{0em}{5em}%
            {{\protect\numberline{#2}}#3}%
            {#4}
    \else
        \ifHy@linktocpage\relax\relax
            \@dottedtocline{-2}{0em}{5em}%
                {{\protect\numberline{#2}}#3}%
                {\hyper@linkstart{link}{#5}{#4}\hyper@linkend}%
        \else
            \@dottedtocline{-2}{0em}{5em}%
                {\hyper@linkstart{link}{#5}%
                  {{\protect\numberline{#2}}#3}\hyper@linkend}%
                    {#4}%
        \fi
    \fi}
}

\theoremlisttype{allname}

\AtBeginDocument{\renewcommand\contentsname{Table of Contents}}

% Heading formattings
% chapter format
\titleformat{\chapter}%
  {\huge\rmfamily\itshape\color{base16-eighties-magenta}}% format applied to label+text
  {\llap{\colorbox{base16-eighties-magenta}{\parbox{1.5cm}{\hfill\itshape\huge\textcolor{base16-eighties-dark}{\thechapter}}}}}% label
  {5pt}% horizontal separation between label and title body
  {}% before the title body
  []% after the title body

% section format
\titleformat{\section}%
  {\normalfont\Large\rmfamily\itshape\color{base16-eighties-blue}}% format applied to label+text
  {\llap{\colorbox{base16-eighties-blue}{\parbox{1.5cm}{\hfill\itshape\textcolor{base16-eighties-dark}{\thesection}}}}}% label
  {5pt}% horizontal separation between label and title body
  {}% before the title body
  []% after the title body

% subsection format
\titleformat{\subsection}%
  {\normalfont\large\itshape\color{base16-eighties-green}}% format applied to label+text
  {\llap{\colorbox{base16-eighties-green}{\parbox{1.5cm}{\hfill\textcolor{base16-eighties-dark}{\thesubsection}}}}}% label
  {1em}% horizontal separation between label and title body
  {}% before the title body
  []% after the title body

% Sidenote enhancements
\def\mathmarginnote#1{%
  \tag*{\rlap{\hspace\marginparsep\smash{\parbox[t]{\marginparwidth}{%
  \footnotesize#1}}}}
}

% Custom table columning
\newcolumntype{L}[1]{>{\raggedright\let\newline\\\arraybackslash\hspace{0pt}}m{#1}}
\newcolumntype{C}[1]{>{\centering\let\newline\\\arraybackslash\hspace{0pt}}m{#1}}
\newcolumntype{R}[1]{>{\raggedleft\let\newline\\\arraybackslash\hspace{0pt}}m{#1}}

% Custom math operator
% \DeclareMathOperator{\rem}{rem}
\DeclareMathOperator*{\argmax}{arg\,max}
\DeclareMathOperator*{\argmin}{arg\,min}
\DeclareMathOperator{\re}{Re}
\DeclareMathOperator{\im}{Im}
\DeclareMathOperator{\caparg}{Arg}
\DeclareMathOperator{\Ind}{Ind}
\DeclareMathOperator{\Res}{Res}

% Graph styles
\pgfplotsset{compat=1.15}
\usepgfplotslibrary{fillbetween}
\pgfplotsset{four quads/.append style={axis x line=middle, axis y line=
middle, xlabel={$x$}, ylabel={$y$}, axis equal }}
\pgfplotsset{four quad complex/.append style={axis x line=middle, axis y line=
middle, xlabel={$\re$}, ylabel={$\im$}, axis equal }}

% Shortcuts
\newcommand{\floor}[1]{\lfloor #1 \rfloor}      % simplifying the writing of a floor function
\newcommand{\ceiling}[1]{\lceil #1 \rceil}      % simplifying the writing of a ceiling function
\newcommand{\dotp}{\, \cdotp}			        % dot product to distinguish from \cdot
\newcommand{\qed}{\hfill\ensuremath{\square}}   % Q.E.D sign
\newcommand{\abs}[1]{\left|#1\right|}						% absolute value
\newcommand{\lra}[1]{\langle \; #1 \; \rangle}
\newcommand{\at}[2]{\Big|_{#1}^{#2}}
\newcommand{\Arg}[1]{\caparg #1}
\renewcommand{\bar}[1]{\mkern 1.5mu \overline{\mkern -1.5mu #1 \mkern -1.5mu} \mkern 1.5mu}
\newcommand{\quotient}[2]{\faktor{#1}{#2}}
\newcommand{\cyclic}[1]{\left\langle #1 \right\rangle}
	% highlighting shortcuts
\newcommand{\hlimpo}[1]{\textcolor{base16-eighties-red}{\textbf{#1}}}
\newcommand{\hlwarn}[1]{\textcolor{base16-eighties-yellow}{\textbf{#1}}}
\newcommand{\hldefn}[1]{\textcolor{base16-eighties-blue}{\index{#1}\textbf{#1}}}
\newcommand{\hlnotea}[1]{\textcolor{base16-eighties-green}{\textbf{#1}}}
\newcommand{\hlnoteb}[1]{\textcolor{base16-eighties-lightblue}{\textbf{#1}}}
\newcommand{\hlnotec}[1]{\textcolor{base16-eighties-brown}{\textbf{#1}}}
\newcommand{\WTP}{\textcolor{base16-eighties-brown}{WTP} }
\newcommand{\WTS}{\textcolor{base16-eighties-brown}{WTS} }
\newcommand{\ind}[2]{\Ind_{#2}\left( #1 \right)}
\newcommand{\notimply}{\centernot\implies}
\newcommand{\res}[2]{\underset{#2}{\Res} #1 }
\newcommand{\tworow}[3]{\begin{tabular}{@{}#1@{}} #2 \\ #3 \end{tabular}}
\renewcommand{\epsilon}{\varepsilon}
\newcommand{\lrarrow}{\leftrightarrow}
\newcommand{\larrow}{\leftarrow}
\newcommand{\rarrow}{\rightarrow}
\renewcommand{\atop}[2]{\genfrac{}{}{0pt}{}{#1}{#2}}
\newcommand*\dif{\mathop{}\!d}

  % inspiration from: https://tex.stackexchange.com/questions/8720/overbrace-underbrace-but-with-an-arrow-instead#37758
\newcommand{\overarrow}[2]{
  \overset{\makebox[0pt]{\begin{tabular}{@{}c@{}}#2\\[0pt]\ensuremath{\uparrow}\end{tabular}}}{#1}
}
\newcommand{\underarrow}[2]{
  \underset{\makebox[0pt]{\begin{tabular}{@{}c@{}}\downarrow\\[0pt]\ensuremath{#2}\end{tabular}}}{#1}
}

% Document header formatting
\renewcommand{\chaptermark}[1]{\markboth{#1}{}}
\renewcommand{\sectionmark}[1]{\markright{#1}}
\makeatletter
\pagestyle{fancy}
\fancyhead{}
\fancyhead[RO]{\textsl{\@title} \enspace \thepage}
\fancyhead[LE]{\thepage \enspace \textsl{\leftmark \enspace - \enspace \rightmark}}
\makeatother

% Comment the two lines below if you want to print the document
\pagecolor{base16-eighties-dark}
\color{base16-eighties-light}


\newenvironment{lrcases}
  {\left\lbrace\quad\begin{aligned}}
  {\end{aligned}\quad\right\rbrace}
\newenvironment{rcases}
  {\left.\quad\begin{aligned}}
  {\end{aligned}\quad\right\rbrace}
\renewcommand\rubysize{.6}

\theoremprework{\textcolor{base16-eighties-blue}{\hrule height 2pt}}
\theoremheaderfont{\color{base16-eighties-blue}\normalfont\bfseries}
\theorempostwork{\textcolor{base16-eighties-blue}{\hrule height 2pt}}
\theoremindent10pt
\newtheorem{grammar}{文法}[section]

\title{JAPAN201RS18}
\author{Johnson Ng}

% Header formatting
\renewcommand{\chaptermark}[1]{\markboth{#1}{}}
\renewcommand{\sectionmark}[1]{\markright{#1}}
\makeatletter
\pagestyle{fancy}
\fancyhead{}
\fancyhead[RO]{\textsl{\@title} \enspace \thepage}
\fancyhead[LE]{\thepage \enspace \textsl{\leftmark \enspace - \enspace \rightmark}}
\makeatother

\begin{document}

\hypersetup{pageanchor=false}
\maketitle
\hypersetup{pageanchor=true}
\tableofcontents

\chapter{Lecture 1 五月三日(水曜日)}
  \label{chapter:lecture_1_may_03rd_2018}

\section{Potential Forms} % (fold)
\label{sec:potential_forms}

We have the following rule for changing verbs to their potential forms:
\begin{itemize}
  \item る-verb $\to$ られる 
  \item う-verb $\to$ 〜える 
  \item Irregular verbs: くる $\to$ こられる $\quad$ する $\to$ できる
\end{itemize}

\begin{eg}
\hlnotea{る-verbs}
  \begin{itemize}
    \item 食べる $\to$ 食べられる $\to$ 食べられます
    \item 見る $\to$ 見られる $\to$ 見られます
  \end{itemize}
  
\noindent\hlnotea{う-verbs}
  \begin{itemize}
    \item 行く $\to$ 行ける $\to$ 行けます
    \item 飲む $\to$ 飲める $\to$ 飲められます
    \item 話す $\to$ 話せる $\to$ 話せます
  \end{itemize}
\end{eg}

\begin{ex}[げんきII - 37ページ、パートA]
Convert the base forms to their potential forms: \marginnote{We can drop the ら in 〜られる. This was originally colloquial, but has now become the norm that it is acceptable to do this even in written text.} \\
  \begin{tabular}{l l}
    1. はなす $\to$ はなせる   & 2. する $\to$ できる \\
    3. いく $\to$ いける       & 4. ねる $\to$ ねられる \\
    5. くる $\to$ こられる     & 6. みる $\to$ みられる \\
    7. やめる $\to$ やめられる & 8. かりる $\to$ かりられる \\
    9. のむ $\to$ のめる       & 10. まつ $\to$ まてる \\
    11. およぐ $\to$ およげる  & 12. はたらく $\to$ はたらける \\
    13. あむ $\to$ あめる
  \end{tabular}
\end{ex}

\begin{ex}[げんきII - 37ページ、パートB]
  Describe the things that Mary can do.
  \begin{enumerate}
    \item メアリーさんは日本語曲がう歌えます。
    \item メアリーさんはヴァイオリンが弾けます。
    \item メアリーさんは空手道ができます。
    \item メアリーさんは寿司が食べられます。
    \item メアリーさんは料理ができます。
    \item メアリーさんは日本語で電話がかけられます。
    \item メアリーさんは車を運転ができます。
    \item メアリーさんはセーターが編めます。
    \item メアリーさんは日本語で手紙がかけます。
    \item メアリーさんは朝早くで起きられる。
    \item メアリーさんは温かいお風呂に入られる。
  \end{enumerate}
\end{ex}

% section potential_forms (end)

% chapter lecture_1_may_03rd_2018 (end)

\chapter{Lecture 2 五月七日(月曜日)}
  \label{chapter:lecture_2_may_07th_2018}

\section{Vocabulary 1}
\label{sect:vocabulary_1}

\hlnotea{名詞 Nouns}
\marginnote{* are words that appear in dialogues.}

\begin{tabular}{r l l l}
  *  & ウェイター &        & waiter \\
     & おたく     & お宅   & (someone's) house/home \\
     & おとな     & 大人   & adult \\
     & がいこくご & 外国語 & foreign language \\
     & がっき     & 楽曲   & musical instruments \\
     & からて     & 空手   & karate \\
  *  & カレー     &        & curry \\
     & きもの     & 着物   & kimono; Japanese traditional dress \\
  *  & こうこく   & 広告   & advertisement \\
     & こうちゃ   & 紅茶   & black tea \\
     & ことば     & 言葉   & language \\
     & ゴルフ     &        & golf \\
     & セーター   &        & sweater \\
     & ぞう       & 象     & elephant \\
     & バイオリン &        & violin \\
     & バイク     &        & motorcycle \\
     & ぶっか     & 物価   & (consumer) prices \\
     & ぶんぽう   & 文法   & grammar \\
     & べんごし   & 弁護士 & lawyer \\
  *  & ぼしゅう   & 募集   & recruitment \\
  *  & みせ       & 店     & shop; store \\
     & やくざ     &        & \textit{yakuza}; gangster \\
     & やくそく   & 約束   & promise; appointment \\
     & レポート   &        & (term) paper; report \\
  *  & わたくし   & 私     & I (formal)
\end{tabular}

\hlnotea{い終わるの形容詞 | い - adjectives} \\
\begin{tabular}{r l l l}
   & うれしい & 嬉しい & glad \\
   & かなしい & 悲しい & sad \\
   & からい   & 辛い   & hot and spicy; salty \\
   & きびしい & 厳しい & strict \\
   & すごい   &        & incredible; awesome \\
   & ちかい   & 近い   & close; near
\end{tabular}

\hlnotea{な終わるの形容詞 | な - adjectives} \\
\begin{tabular}{r l l l}
  *  & いろいろ(な) &      & various; different kinds of \\
     & しあわせ(な) & 幸せ & happy (lasting happiness) \\
  *  & だめ(な)     &      & no good
\end{tabular}

\hlnotea{う終わるの動詞 | う - verbs} \\
\begin{tabular}{r l l l}
      & あむ             & 編む       & to knit (〜を) \\
      & かす             & 貸す       & to lend; to rent \\
      &                  &            & (\textit{person} に \textit{thing} を) \\
  *   & がんばる         & 頑張る     & to do one's best; to try hard \\
      & なく             & 泣く       & to cry \\
      & みがく           & 磨く       & to brush (teeth); to polish (〜を) \\
      & やくそくをまもる & 約束を守る & to keep a promise
\end{tabular}

\hlnotea{Irregular Verb} \\
\begin{tabular}{r l l l}
      & かんどうする & 感動する & to be moved/touched (by...) \\
      &              &          & (〜に)
\end{tabular}

\section{〜し}

\begin{grammar}[〜し]
\label{grammar:_shi}
  We can use 〜し to conjugate reasons. Te form for using 〜し is:
  \begin{equation*}
    \text{ short form } + \text{ し }
  \end{equation*}
\end{grammar}

\begin{ex}
  Write the present affirmative and present negative tenses for the following, using 〜し and ending. After that, write down the past tense form of the statements.

  \begin{tabular}{l l l}
                   & present affirmative & present negative \\
    おもしろいです & おいしいし          & おいしくないし \\
    きれいです     & きれいだし          & きれいじゃないし \\
    がくせいです   & がくせいだし        & がくせいじゃないし \\
    かんどうする   & かんどうだし        & かんどうじゃないし \\
    たべたいです   & たべたいし          & たべたくないし \\
    すんでいます   & すんでいるし        & すんでいないし \\
    いけます       & いけるし            & いけないし \\
                   & past affirmative    & past negative \\
    おもしろいです & おもしろかったし    & おもしろじゃなかったし \\
    きれいです     & きれいだったし      & きれいじゃなかったし \\
    がくせいです   & がくせいだったし    & がくせいじゃなかったし \\
    かんどうする   & かんどうだったし    & かんどうじゃなかったし \\
    たべたいです   & たべたかったし      & たべたくなかったし \\
    すんでいます   & すんでいたし        & すんでいなかったし \\
    いけます       & いけたし            & いけなかったし
  \end{tabular}
\end{ex}

\begin{ex}[げんき II - 39 ページ、パートA]
\hlnotea{物価が高いし、人がたくさんいるし}

  Answer the questions using 〜し〜し. Examine the ideas in the cues and decide whether you want to answer in the affirmative or negative form.

  Example:
  \begin{align*}
    Q &: \text{日本に住みたいですか。} \\
    A &: \text{(物価が高いです。人がたくさんいます。} \\
      &\to \text{物価が高いし、人がたくさんいるし、住みたくないです。}
  \end{align*}

  \begin{enumerate}
    \item 今週は忙しですか。\\
      (試験があります。宿題がたくさんあります。)\\
      \textbf{答え:} ええ、忙しです、試験があるし、宿題もたくさんあるし。
    \item 新しいアルバイトはいいですか。\\
      (会社に近いです。静かです。)\\
      \textbf{答え:} 会社に近いし、静かだし、新しいアルバイトはいいですよ。
    \item 経済の授業をとりますか。\\
      (先生は厳しいです。長いレポートを書かなきゃいけませ。)\\
      \textbf{答え:} 経済の授業を取りたくないです、先生は厳しいし、長いレポートを書かなきゃいけないし。
    \item 旅行は楽しかったですか。\\
      (たべものはおいしくなかったです。言葉がわかりませんでした。)\\
      \textbf{答え:} 楽しくなかったです、食べ物が美味しくなかったし、言葉がわかりませんだし。
    \item 今晩、パーテイーにいきますか。\\
      (かぜをひいています。昨日もパーティーに行きました。)\\
      \textbf{答え:} 今晩のパーティーに行きません、かぜをひいているし、昨日もパーティーにいっただし。
    \item 日本語の新聞が読めますか。\\
      (漢字が読めません。文法がわかりません。)\\
      \textbf{答え:} 読めないです、漢字が読めないし、文法もわからないし。
    \item 一人で旅行ができますか。\\
      (日本語が話せます。もう大人です。)\\
      \textbf{答え:} できますよ、日本語が話せるし、もう大人だし。
    \item 田中さんが好きですか。\\
      (うそをつきます。約束を守りません。)\\
      \textbf{答え:} あんまり好きじゃないです、うそをつくし、約束を守れないし。
  \end{enumerate}
\end{ex}

\section{〜そうです}%
\label{sec:_soudesu}
% section _soudesu

\begin{grammar}[〜そうです]
\label{grammar:_soudesu}
  「〜そうです」has the ``looks like'' meaning in English. For example,
  \begin{center}
    美味し\hlnoteb{そうです}。 \\
    \hlnoteb{Looks} delicious.
  \end{center}
  The negative form of 〜そうです is 〜なさそうです.

  Another usage of 〜そうです is as follows:
  \begin{center}
    (形容詞) $+$ そう $+$ \hlnoteb{な} $+$ (名詞)
  \end{center}
  For example,
  \begin{center}
    美味しそう\hlnoteb{な}寿司です。
  \end{center}
\end{grammar}

\begin{ex}
  げんき II、42ページ、(III) C の宿題を練習してください。
\end{ex}

% section _soudesu (end)

\section{漢字}%
\label{sec:kanji}
% section kanji

漢字の歴史と書き方を簡単的に紹介しました。

\begin{note}[Brief History and Information]
  \begin{itemize}
    \item Kanji are Chinese characters.
    \item Kanji were introduced to Japan 1500 years ago, when Japan has yet to have its own writing system.
    \item Both \textit{hiragana} and \textit{katakana} are evolutions of simplified Chinese characters later on.
    \item Kanji represents both meanings and sounds.
    \item (Just as in Chinese,) most Kanji have multiple readings, which can be divided into 2 types:
      \begin{itemize}
        \item \textit{On-yomi} (音読み)(Chinese readings)
          \begin{itemize}
            \item derived from pronunciations in China
            \item some Kanji has more than one \textit{on-yomi} due to temporal and regional variacnes in Chinese pronunciation
          \end{itemize}
        \item \textit{kun-yomi} (訓読み)(Japanese readings)
      \end{itemize}
  \end{itemize}
\end{note}

\begin{note}[Forms of Kanji]
  There are roughly 4 types of Kanji based on their formation:
  \begin{itemize}
    \item \hlnotea{Pictograms} - Kanji created from pictures (e.g. 山)
    \item \hlnotea{Simple ideograms} - Kanji made from dots and lines to represent numbers of abstract concepts (e.g. 三、上)
    \item \hlnotea{Compund ideograms} - Kanji made from two or more kanji characters (e.g. 曜)
    \item \hlnotea{Phonetic-ideographic characters} - Kanji made of two parts: a meaning element and a sound element
  \end{itemize}
\end{note}

% section kanji (end)

\chapter{Lecture 3 五月九日(水曜日)}%
\label{chp:lecture_3_wu_yue_jiu_ri_shui_yao_ri}
% chapter lecture_3_wu_yue_jiu_ri_shui_yao_ri

\section{〜てみます}%
\label{sec:_temimasu}
% section _temimasu

\begin{grammar}[〜てみます]
\label{grammar:_temimasu}
  「〜てみます」has the meaning of ``shall try''. For example,
  \begin{center}
    食べ\hlnotea{てみます}。\\
    (I) \hlnotea{shall try} to eat.
  \end{center}
\end{grammar}

\begin{ex}[げんき II - 43ページ、IV パート A]
  Respond to the following sentences using 〜てみる.

  Example:
  \begin{align*}
    A &: \text{この服はすてきですよ。} \\
    B &: \text{じゃあ、着てみます。}
  \end{align*}

  \begin{enumerate}
    \item 経済の授業はおもしろかったですよ。 \\
      \textbf{答え:} じゃあ、取ってみます。
    \item あの映画を見て泣きました。 \\
      \textbf{答え:} じゃあ、見てみます。
    \item このほんはかんどうしました。 \\
      \textbf{答え:} じゃあ、読んでみます。
    \item このケーキはおいしいですよ。 \\
      \textbf{答え:} じゃあ、食べてみます。
    \item 東京はおもしろかったですよ。 \\
      \textbf{答え:} じゃあ、行ってみます。
    \item このCDはよかったですよ。 \\
      \textbf{答え:} じゃあ、聞いてみます。
    \item この辞書は便利でしたよ。 \\
      \textbf{答え:} じゃあ、使ってみます。/ じゃあ、買ってみます。
  \end{enumerate}
\end{ex}

\begin{ex}[げんき II - 43-44 ページ、IV パート B]
  You are at a shopping center. Ask store attendants whether you can try out the following, using appropriate verbs.

  Example:
  \begin{align*}
    \text{Customer} &: \text{すみません。使ってもいいですか。} \\
    \text{Store attendant} &: \text{どうぞ、使ってみてください。}
  \end{align*}

  \begin{enumerate}
    \item (服)\\
      お客様:すみません。ワンピースを着てみてもいいですか。\\
      店員:どうぞ、着てみてください。
    \item (椅子)\\
      お客様:すみません。椅子に座ってもいいですか。\\
      店員:どうぞ、座ってください。
    \item (ギター)\\
      お客様:すみません。ギターを弾いてみてもいいですか。\\
      店員:どうぞ、弾いてみてください。
    \item (自転車)\\
      お客様:すみません。自転車に乗ってみてもいいですか。\\
      店員:どうぞ、乗ってみてください。
    \item (靴)\\
      お客様:すみません。靴を履いてみてもいいですか。\\
      店員:どうぞ、履いてみてください。
  \end{enumerate}
\end{ex}

\begin{ex}[げんき II - 44ページ、IV パート C]
  Talk about what you want to try in the following places.

  Example: インド (India)
  \begin{align*}
    A &: \text{インドに行ったことがありますか。} \\
    B &: \text{いいえ。ありません。でも、行ってみたいです。} \\
    A &: \text{そうですか。インドで何がしたいですか。} \\
    B &: \text{インドで象を見たり、ヨガ(Yoga)を習ったりしてみたいです。}
  \end{align*}

  \begin{enumerate}
    \item ケンア (Kenya)
      \begin{align*}
        A &: \text{ケニヤに行ったことがありますか。} \\
        B &: \text{いいえ。ありません。でも、行ってみたいです。} \\
        A &: \text{そうですか。ケニャで何がしたいですか。} \\
        B &: \text{ケニヤでサバンナの\ruby[j]{日暮}{ひ|ぐ}れと\ruby[j]{野生動物}{や|せい|どう|ぶつ}を見たり、お\ruby[j]{寺}{てら}を行ったりしてみたいです。}
      \end{align*}
    \item 東京
      \begin{align*}
        A &: \text{東京に行ったことがありますか。} \\
        B &: \text{はい、去年五月に行きました。} \\
        A &: \text{そうですか。何がしたんですか。} \\
        B &: \text{ラーメンとつけ\ruby[j]{麺}{めん}を食べたり、\ruby[j]{周}{まわ}りの観光地に行ったりしていた。}
      \end{align*}
    \item タイ (Thailand)
      \begin{align*}
        A &: \text{タイに行ったことがありますか。} \\
        B &: \text{いいえ、行ったことないです。でも、行ってみたいです。} \\
        A &: \text{そうですか。タイで何がしたいですか。} \\
        B &: \text{タイでメコン川に遊んたり、トムヤムを食べたりしてみたいです。}
      \end{align*}
    \item ブラジル (Brazil)
      \begin{align*}
        A &: \text{ブラジルに行ったことがありますか。} \\
        B &: \text{いいえ、ないです。でも、行ってみたいです。} \\
        A &: \text{そうですか。ブラジルで何がしたいですか。} \\
        B &: \text{ブラジルで\ruby[j]{滝}{たき}を見たり、山を\ruby[j]{乗}{の}ったりしてみたいです。}
      \end{align*}
    \item チベット (Tibet)
      \begin{align*}
        A &: \text{チベットに行ったことがありますか。} \\
        B &: \text{いいえ、ないです。でも、行ってみたいです。} \\
        A &: \text{そうですか。チベットで何がしたいですか。} \\
        B &: \text{チベットでお寺を見たり、\ruby[j]{自然}{しぜん}を近づくたりしてみたいです。}
      \end{align*}
    \item Your own
      \begin{align*}
        A &: \text{マレシアに行ったことがありますか。} \\
        B &: \text{はい、ありますよ。} \\
        A &: \text{そうですか。マレシアで何がしたんですか。} \\
        B &: \text{マレシアで美味しい食べ物を食べたり、海で泳いでたりしていました。}
      \end{align*}
  \end{enumerate}
\end{ex}

% section _temimasu (end)

% chapter lecture_3_wu_yue_jiu_ri_shui_yao_ri (end)

\chapter{Lecture 4 五月十四日(月曜日)}%
\label{chp:lecture_4_wu_yue_shi_si_ri_yue_yao_ri}
% chapter lecture_4_wu_yue_shi_si_ri_yue_yao_ri

\section{〜なら}%
\label{sec:_nara}
% section _nara

\begin{grammar}[なら]
\label{grammar:nara}
  A statement of the form "noun A なら predicate X" says that the predicate X \textit{applies only to} A and is not more generally valid. The main ideas of a なら sentence, in other words, are contrast (as in Situation 1) and limitation (as in Situation 2).
\end{grammar}

\begin{eg}[Situation 1]\label{eg:nara_situation_1}
\noindent    Q : ブラジルに行ったことがありますか。 \\
\noindent \quad\enspace Have you ever been to Brazil? \\
\noindent    A : チリ\underline{なら}行ったことがありますが、ブラジルはいったことがありません。\sidenote{
  \begin{note}
    Optionally, we can keep the particle に before なら in this example. Particles such as に, で, and から may, but do not have to, intervene between the noun and なら, while は, が, and を never go with なら.
  \end{note}
  } \\
\noindent \quad\enspace I've been to Chile, but never to Brazil. 
\end{eg}

\begin{eg}[Situation 2]\label{eg:nara_situation_2}
\noindent Q : 日本語がわかりますか。\\
\noindent \quad\enspace Do you understand Japanese? \\
\noindent A : ひたがな\underline{なら}わかります。\\
\noindent \quad\enspace If it is (written) in hiragana, yes.
\end{eg}

\begin{note}
  \marginnote{The examples illustrate a good way to keep the conversation ball rolling by adding related information to a question that can simply be answered with a no. Q can then use the ``positive'' information received from A to continue the conversation, instead of just ending the conversation or have a difficult time continuing it.}
  ならintroduces a sentence that says something ``positive'' about the item that is contrasted. In the first situation above, なら puts Chile in a positive light, and in constrast with Brazil, which the question was originally about. In the second situation, a smaller part, namely \textit{hiragana}, is brought up and contrasted with a larger area, namely, the language as  whole. 
\end{note}

\begin{ex}[げんき II - 44 ページ、V パート A]
  Answer the questions as in the example.

  \textbf{例:} \\
  \noindent Q: メアリーさんはけさ、コーヒーを飲みますか。\\
  \noindent A: ($\circ$ tea $\times$ coffee) \\
  \noindent \quad\enspace $\to$ \ruby[j]{紅茶}{こう|ちゃ}なら飲みましたが、コーヒーは\sidenote{
    \begin{note}
      It is entirely okay for us to use は here instead of を as we would normally do, since we are trying to make a comparison between コーヒー and 紅茶.
    \end{note}
  }飲みませんでした。

  \begin{enumerate}
    \item メアリーさんはバイクに乗れますか。($\circ$ bicycle $\times$ motorbike) \\
      \textbf{答え:}自転車なら乗れますが、バイクには乗れません。 
    \item メアリーさんはニュージーランドに行ったことありますか。($\circ$ Australia $\times$ New Zealand) \\
      \textbf{答え:}オーストラリアなら行ったことありますが、ニュージーランドはまだ行ったことありません。 
    \item メアリーさんはゴルフをしますか。($\circ$ tennis $\times$ golf) \\
      \textbf{答え:}テニスならしますが、ゴルフはしたことないです。
    \item けんさんは日本の\ruby[j]{経済}{けい|ざい}に\ruby[j]{興味}{きょう|み}がありますか。($\circ$ history $\times$ economics) \\
      \textbf{答え:}歴史なら興味がありますが、経済はあまり興味がありません。 
    \item けんさんは\ruby[j]{彼女}{かのじょ}がいますか。($\circ$ friend $\times$ girlfriend) \\
      \textbf{答え:}友達ならいますが、彼女はいません。 
    \item けんさんは土曜日に出かけられますか。($\circ$ Sunday $\times$ Saturday) \\
      \textbf{答え:}日曜日なら出かけられますが、土曜日はちょっといけません。 
  \end{enumerate}
\end{ex}

\begin{ex}[げんき II - 45 ページ、V パート B]
  Answer the following questions. Use 〜なら whenever possible.

  \textbf{例:}\\
  \noindent Q: スポーツをよく見ますか。\\
  \noindent A: ええ、\ruby[j]{野球}{や|きゅう}なら見ます。/いいえ、見ません。

  \begin{enumerate}
    \item 外国語ができますか。\\
      \textbf{答え:}はい、中国語、英語、マレー語、広東語と台湾語ならできます。 
    \item アルバイトをしたことがありますか。\\
      \textbf{答え:}はい、\ruby[j]{官公庁}{かん|こう|ちょう} にデータ\ruby[j]{分析}{ぶん|せき}の仕事がしたことがあります。
    \item 日本の料理が作れますか。\\
      \textbf{答え:}いいえ、 日本の料理が好きですが、作れません。
    \item 有名人に会ったことがありますか。\\
      \textbf{答え:}いいえ、そういう\ruby[j]{機会}{き|かい}がないです。 
    \item \ruby[j]{楽器}{がっ|き}できますか。\\
      \textbf{答え:}いいえ。でも、ギータなら前に\ruby[j]{学}{まな}びました。 
    \item お金を\ruby[j]{貸}{か}せますか。\\
      \textbf{答え:}お金を貸せませんが、何のお\ruby[j]{手伝}{て|つだ}いをほしいですか。 
  \end{enumerate}
\end{ex}

% section _nara (end)

\section{一週間に三回}%
\label{sec:yi_zhou_jian_nisan_hui_}
% section yi_zhou_jian_nisan_hui_

\begin{grammar}[(period) に (frequency)]
\label{grammar:_period_ni_frequency_}
  We can describe the frequency of events over a period of time using the following framework:
  \begin{center}
    (period) に (frequency) \qquad (frequency) per (period)
  \end{center}
\end{grammar}

\begin{eg}
\noindent 私は\underline{一週間に三回}髪を\ruby[j]{洗}{あら}います。\\
\noindent 私は\underline{一ヶ月に一回}家族に電話をかけます。\\
\noindent 父は\underline{一年に二回}旅行します。
\end{eg}

\begin{ex}[げんき II - 45 ページ、VI パートA]
  Express the following activities and their frequencies as in the given example.

  \textbf{例:} \\
  eat twice a day \\
  一日に二回食べます。

  \begin{enumerate}
    \item brush (teeth) twice a day \\
      $\to$ 一日に三回\ruby[j]{歯}{は}を\ruby[j]{磨}{み}きます。
    \item sleep seven hours a day \\
      $\to$ 一日に七時寝ます。
    \item study three hours a day \\
      $\to$ 一日に三時勉強します。
    \item do house chores once a week \\
      $\to$ 一週間に一回\ruby[j]{家事}{か|じ}をします。
    \item do the laundry twice a week
      $\to$ 一週間に二回洗濯します。
    \item do part-time job three days a week \\
      $\to$ 一週間に三日アルバイトをします。
    \item go to school five days a week \\
      $\to$ 一週間に五日学校に行きます。
    \item go to the movies once a month \\
      $\to$ 一ヶ月に一回\ruby[j]{映画館}{えい|が|かん}に行って、映画を見ます。
  \end{enumerate}
\end{ex}

\begin{ex}[げんき II - 46 ページ、 VI パート B]
  From the last exercise, create questions above the given activities using the following method:
  \begin{equation*}
    \begin{lrcases} 一日 \\ 一週間 \\ 一ヶ月 \end{lrcases} に \begin{lrcases} 何回 \\ 何時間 \\ 何日 \end{lrcases} 〜ますか
  \end{equation*}

  \textbf{例:} \\
\noindent A: Bさんは一日に何回食べますか。 \\
\noindent B: そうですね。たいてい一日に二回食べます。朝食は食べません。
\end{ex}

% section yi_zhou_jian_nisan_hui_ (end)

% chapter lecture_4_wu_yue_shi_si_ri_yue_yao_ri (end)

\chapter{Lecture 5 五月十六日(水曜日)}%
\label{chp:lecture_5_wu_yue_shi_liu_ri_shui_yao_ri_}
% chapter lecture_5_wu_yue_shi_liu_ri_shui_yao_ri_

\begin{ex}[げんき II - 47 ページ、VII パート A]
  Answer the following questions.

  \begin{enumerate}
    \item 子供の時に何ができましたが。何ができませんでしたか。
    \item 百円で何が買えますか。
    \item どこに行ってみたいですか。どうしてですか。
    \item 子供の時、何がしてみたかったですか。
    \item 今、何がしてみたいですか。
    \item 一日に何時間ぐらい勉強しますか。
    \item 一週間に何回レストランに行きますか。
    \item 一ヶ月にいくらぐらい使いますか。
  \end{enumerate}
\end{ex}

% chapter lecture_5_wu_yue_shi_liu_ri_shui_yao_ri_ (end)

\chapter{Lecture 6 五月二十二日(火曜日)}%
\label{chp:lecture_6_wu_yue_er_shi_er_ri_huo_yao_ri_}
% chapter lecture_6_wu_yue_er_shi_er_ri_huo_yao_ri_

\section{ほしい}
\label{sec:hoshii}
% section hoshii

\begin{grammar}[ほしい]
\label{grammar:hoshii}
  ほしい means ``(I) want (something)''. It is an \hlnotea{い - adjective} and conjugates as such. The object of desire is usually follwoed by the particle が. \hlnoteb{In negative sentences}, the particle は is also used.

  いい漢字の辞書が\hlnotec{ほしいです}。\\
  I want a good kanji dictionary. \\
  子供の時、ゴジラのおもちゃが\hlnotec{\ruby[j]{欲}{ほ}しかったです}。\\
  When I was young, I wanted a toy Godzilla. \\
  お金はあまり\hlnotec{欲しくないです}。\\
  I don't have much desire for money.

  \begin{center}
    (私は) Xが ほしい \qquad I want X.
  \end{center}

  \noindent ほしい is similar to たい (I want to do...), in that its use is primarily limited to the first person, the speaker. These words are called ``\hlnoteb{private predicates},'' and they refer to the inner sensations which are known only to the person feeling them. Everyone else needs to rely on observations and guesses when they want to claim that ``person X wnats such and such.'' Japanese grammar, ever demanding that everything be stated in explicit terms, therefore calls for an extra device for sentences with private predicates as applied to the second or third person.\sidenote{Among the words we have learned so far, 悲しい (sad), いれしい (glad), and 痛い (painful) are private predicates. The observations we make about ほしい below apply to these words as well.}

  \noindent You can quote the people who say that they are feeling these sensations.

  ロバートさんはパソコンが\hlnotec{ほしい}\hlnotea{と言っています}。 \\
  Robert says he wants a computer.

  \noindent You can make clear that you are only making a guess.

  京子さんはクラシックの\ruby[j]{CD}{シーディー}が\hlnotec{ほしくない}\hlnotea{でしょう}。\\
  It is likely that Kyoko does not want a CD of classical music.

  \noindent Or you can use the special construction which says that you are making an observation of a person feeling a private-predicate sensation. In an earlier lesson (in JAPAN102R), we were introduced to the verb \hlnotea{たがる}. which replaces たい.

  智子さんは英語を習いたがっています。\\
  (I understand that) Tomoko wants to study English.

  \noindent ほしい too has a special verb counterpart, ほしがる. It conjugates as an \textit{u}-verb and is usually used in the form 欲しがっている, to describe an observation that the speaker currently thinks holds true. Unlike ほしい, the particle after the object of desire is を.

  トムさんは友達を\hlnotec{欲しがっています}。\\
  (I understand that) Tom wants a friend.
\end{grammar}

\begin{eg}
  \begin{enumerate}
    \item みずさ先生はミニファミコンがほしいと言っていました。
    \item みずさ先生はミニファミコンがほしいと思います。
  \end{enumerate}
\end{eg}

\begin{ex}[げんき II - 61 ページ、I パート A]
  Items marked with $\circ$ are what you want, and items marked with $\times$ are what you do not want. Make sentences using ほしい.

  \begin{enumerate}
    \item $\circ$ money \\
      $\to$ お金がほしいです。
    \item $\times$ sweater \\
      $\to$ セーターが欲しくないです。
    \item $\times$ personal computer \\
      $\to$ パソコンは欲しくないです。
    \item $\circ$ motorbike \\
      $\to$ バイクがほしいです。
    \item $\times$ plush toy \\
      $\to$ ぬいぐるみは欲しくないです。
  \end{enumerate}
\end{ex}

\begin{ex}[げんき II - 61 ページ、I パート B]
  Items marked with $\circ$ are what you wanted when you were a child, and items marked with $\times$ are what you did not want. Make sentences using ほしい.

  \begin{enumerate}
    \item $\circ$ プレーステーション \\
      $\to$ 子供の時、プレーステーションが欲しかったです。
    \item $\times$ 指輪 \\
      $\to$ 子供の時、指輪はほしくなかったです。
    \item $\times$ \ruby[j]{腕}{うで}時計 \\
      $\to$ 子供の時、腕時計は欲しくないです。
    \item $\circ$ おもちゃ \\
      $\to$ 子供の時、おもちゃが欲しかったです。
    \item $\times$ \ruby[j]{花束}{はな|たば} \\
      $\to$ 子供の時、花束がほしくなかったです。
  \end{enumerate}
\end{ex}

% section hoshii (end)

\section{〜かもしれません}%
\label{sec:_kamoshiremasen}
% section _kamoshiremasen

\begin{grammar}[〜かもしれません]
\label{grammar:_kamoshiremasen}
  The usage of かもしれません is similar to that of でしょう that was introduced in Lesson 12 (JAPAN102R), which we use for stating possibilities.

  \noindent かもしれません is placed after the short forms of predicates, in the affirmative and in the negative, in the present as well as the past tense.

  あしたは雨が降るかもしれません。\\
  It may rain tomorrow. \\
  田中さんより、鈴木さんのほうが\ruby[j]{背}{せ}が高いかもしれません。\\
  Suzuki is perhaps taller than Tanaka. \\
  あしたは天気が良くないかもしれません。\\
  The weather may not be good tomorrow. \\
  トムさんは、子供の時、いじわるだったかもしれません。\\
  Tom may have been a bully when he was a kid.

  \noindent Just like でしょう, かもしれません goes directly after a nount or a な - adjective in the present tense affirmative sentences. In other words, だ is dropped in these sentences.

  トムさんはカナダ人だ。\\
  $\to$ トムさんはカナダ人かもしれません。 \\
  Tom is a Canadian. \\
  $\to$ Tom might be a Canadian. \\
  山下先生は犬が嫌いだ。 \\
  $\to$ 山下先生は犬が嫌いかもしれません。\\
  Professor Yamashita is not fond of dogs. \\
  $\to$ It is possible that Professor yamashita is not fond of dogs.

  \textbf{Prssent tense, affirmative} \\
  \begin{equation*}
    \begin{rcases}
      \text{verbs:} &\qquad 行く \\
      \text{い - adjectives:} &\qquad 寒い \\
      \text{な - adjectives:} &\qquad 元気 \\
      \text{noun + です} &\qquad 学生
    \end{rcases}\enspace かもしれません
  \end{equation*}
\end{grammar}

\begin{ex}[げんき II - 63 ページ、II パート A]
  Use the hints to change the following sentences using 〜かもしれません.

  \begin{enumerate}
    \item 女の人は会社員です。(maybe) \\
      $\to$ 女の人は会社員かもしれません。
    \item 男の人は先生です。(maybe not) \\
      $\to$ 男の人は先生じゃないかもしれません。
    \item 女の人はテニスが上手です。(maybe) \\
      $\to$ 女の人はテニスが上手かもしれません。
    \item 男の人は背が\ruby[j]{低}{ひく}いです。(maybe not) \\
      $\to$ 男の人は背が低くないかもしれません。
    \item 今、寒いです。(maybe not) \\
      $\to$ 今、寒くないかもしれません。
    \item 女の人は今日テニスをします。(maybe) \\
      $\to$ 女の人は今日テニスをするかもしれません。
    \item 男の人と女の人は、今、駅にいます。(maybe not) \\
      $\to$ 男の人と女の人は、今、駅にいないかもしれません。
    \item 男の人は結婚しています。(maybe) \\
      $\to$ 男の人は結婚しているかもしれません。
    \item 男の人と女の人は夫婦です。(maybe not) \\
      $\to$ 男の人と女の人は夫婦じゃないかもしれません。
    \item 女の人は男の人に興味があります。(maybe) \\
      $\to$ 女の人は男の人に興味があるかもしれません。
    \item 女の人は昨日テニスをしました。(maybe)
      $\to$ 女の人は昨日テニスをしたかもしれません。
  \end{enumerate}
\end{ex}

% section _kamoshiremasen (end)

% chapter lecture_6_wu_yue_er_shi_er_ri_huo_yao_ri_ (end)

\chapter{Lecture 7 五月二十三日(水曜日)}%
\label{chp:lecture_7_wu_yue_er_shi_san_ri_shui_yao_ri_}
% chapter lecture_7_wu_yue_er_shi_san_ri_shui_yao_ri_

\section{Vocabulary 2}%
\label{sec:vocabulary_2}
% section vocabulary_2

\begin{longtable}{r l l l}
\multicolumn{4}{l}{\hlnotea{名詞 Nouns}} \\
 & あに & 兄 & (my) older brother \\
*& おおやさん & 大家さん & landlord; landlady \\
*& おかえし & お返し & return (as a token of gratitude) \\
 & おくさん & 奥さん & (your/his) wife \\
 & おじさん & & uncle; middle-aged man \\
 & おばさん & & aunt; middle-aged woman \\
 & グラス & & tumbler; glass \\
 & クリスマス & & Christmas \\
 & ごしゅじん & ご主人 & (your/her) husband \\
 & さら & 皿 & plate; dish \\
 & じかん & 時間 & time \\
 & チケット & & ticket \\
*& チョコレート & & chocolate \\
 & トレーナー & & sweat shirt \\
 & ぬいぐるみ & & sutffed animal (e.g. teddy bear) \\
 & ネクタイ & & necktie \\
*& バレンタインデー & & St. Valentine's Day \\
 & ヴィデオカメラ & & camcorder \\
 & ふうふ & 夫婦 & married couple; husband and wife \\
*& ホワイトデー & & "White Day" (yet another gift-giving day) \\
 & マフラー & & winter scarf \\
 & まんが & 漫画 & comic book \\
 & マンション & & \tworow{l}{multistory apartment building;}{condominium}\\
 & みかん & & mandarin orange \\
 & みなさん & 皆さん & everyone; all of you \\
 & ゆびわ & 指輪 & ring \\
 & ラジオ & & radio \\
 & りょうしん & 両親 & parents \\
 & りれきしょ & 履歴書 & r\'esum\'e \\
 $ $\\
\multicolumn{4}{l}{\hlnotea{い終わる形容詞 い - adjectives}} \\
*& ほしい & 欲しい & to want (\textit{thing} が) \\
 $ $ \\
\multicolumn{4}{l}{\hlnotea{な終わる形容詞 な - adjectives}} \\
 & けち(な)& & stingy; cheap \\
 $ $ \\
\multicolumn{4}{l}{\hlnotea{う終わる動詞 う-verbs}} \\
 & おくる & 送る & to send (\textit{person} に \textit{thing} を ) \\
*& にあう & 似合う & \tworow{l}{to look good (on somebody)}{(\textit{thing} が)} \\
 $ $ \\
\multicolumn{4}{l}{\hlnotea{る終わる動詞 る-verbs}} \\
 & あきらめる & 諦める & to give up (〜を) \\
*& あげる & & \tworow{l}{to give (to others)}{(\textit{person} に \textit{thing} を)} \\
*& くれる & & to give (me) (\textit{person} に \textit{thing} を) \\
 & できる & & \tworow{l}{to come into existence; to be made}{(〜が)} \\
 $ $ \\
\multicolumn{4}{l}{\hlnotea{Irregular Verbs}} \\
 & そうだんする & 相談する & to consult (\textit{person} に) \\
 & プロポーズする & & to propose marriage (\textit{person} に) \\
 $ $ \\
\multicolumn{4}{l}{\hlnotea{Adverbs and Other Expressions}} \\
*& おなじ & 同じ & same \\
*& 〜くん & 〜君 & Mr./Ms. ... (casual) \\
*& こんな〜 & & ... like this; this kind of... \\
 & 〜たち & 達 & [makes a noun plural] \\
 & わたしたち & 私達 & we \\
*& ちょうど & & exactly \\
 & どうしたらいい & & what should one do \\
 & よく & & well \\
 $ $ \\
\multicolumn{4}{l}{\hlnotea{Counters}} \\
*& 〜こ & 〜個 & [generic counter for smaller items] \\
 & 〜さつ & 〜冊 & [counter for bound volumes] \\
 & 〜だい & 〜台 & [counter for equipment] \\
 & 〜ひき & 〜匹 & [counter for smaller animals] \\
 & 〜ほん & 〜本 & [counter for long objects]
\end{longtable}

% section vocabulary_2 (end)

\section{あげる/くれる/もらう}%
\label{sec:ageru_kureru_morau}
% section ageru_kureru_morau

\begin{grammar}[あげる/くれる/もらう]
\label{grammar:ageru_kureru_morau}
  Japanese has two verbs for giving. The choice between the pair depends on the direction of the transaction. Refer to the following diagram for a reference of what to use:
  \begin{center}
  \begin{tikzpicture}[scale=1, transform shape]
    \draw[->,line width=2pt] (-3, 2.5) -- (3,2.5) node[middle,above] {あげる};
    \draw[->,line width=2pt] (3, -2.5) -- (-3,-2.5) node[middle,below] {くれる};
    \draw[-,fill=base16-eighties-light,fill opacity=0.25] (-3,0.5) -- (-3,-0.5) -- (-2,-0.5) -- (-2,0.5) -- (-3,0.5);
    \draw[-,fill=base16-eighties-light,fill opacity=0.25] (-3,1) -- (-3,-1) -- (0,-1) -- (0,1) -- (-3,1);
    \draw[-] (-3,1.5) -- (-3,-1.5) -- (3,-1.5) -- (3,1.5) -- (-3,1.5);
    \node at (-2.5,0) {\text{I}};
    \node at (-1,0) {\text{You}};
    \node at (1.5,0) {\text{Others}};
  \end{tikzpicture}
  \end{center}
  With both あげる and くれる, the giver is the subject of the sentence, and is accompanied by the particle は or が. The recipient is accompanied by the particle に.
  \begin{center}
    (giver) は / が (recipient) に \begin{lrcases}あげる \\ くれる \end{lrcases} \\
    (giver) gives to (recipient)
  \end{center}
  Transactions that are described iwth the verb くれる can also be described in the terms of ``receiving'' or もらう. With もらう, it is the recipient that is the subject of the sentence, with は or が, and the giver is accompanied by the particle に or から.
  \begin{center}
    (recipient) は / が (giver) に / から もらう \\
    (recipient) receives from (giver)
  \end{center}
\end{grammar}

\begin{eg}
Below are some example sentences.
  \begin{enumerate}
    \item わたし\underline{は}その女の人\underline{に}花をあげます。
    \item その女の人\underline{は}男の人\underline{に}時計をあげました。
    \item 両親\underline{が}(私\underline{に})新しい車をくれるかもしれません。
    \item 私は姉\underline{に}/姉\underline{から}古い辞書をもらえました。
    \item 姉\underline{が}私\underline{に}古い辞書をくれました。
  \end{enumerate}
\end{eg}

\begin{note}[くれる is only possible when ``I'' am involved]
  When a transaction takes place between two people other than yourself, the verb to use is normally あげる. くれる is possible only in limited contexts in which you think you yourself have benefited because somebody very close to you has received something. It should be relatively easy for you to identify yourself with a member of your immediate family or a very good friend, for example,
  \begin{center}
    \ruby[j]{大統領}{だい|とう|りょう}が妹に手紙をくれました。
  \end{center}
\end{note}

\begin{note}[もらう and くれる]
  もらう is like くれる and implies that you identify yourself more closely with the recipient than with the giver. Thus it is wrong to use もらう if \textbf{you} receive from \textbf{me}, for example\sidenote{This is one indication that nobody can be completely detached from their ego.} 
  \begin{center}
    $\times$ (あなたは)私から手紙をもらいましたが。
  \end{center}
  You can use もらう for third-party transactions if you can assume the perspective of the recipient.
  \begin{center}
    妹は大統領に手紙をもらいました。
  \end{center}
\end{note}

% section ageru_kureru_morau (end)

% chapter lecture_7_wu_yue_er_shi_san_ri_shui_yao_ri_ (end)

\chapter{Lecture 8 五月二十八日(月曜日)}%
\label{chp:lecture_8_wu_yue_er_shi_ba_ri_yue_yao_ri_}
% chapter lecture_8_wu_yue_er_shi_ba_ri_yue_yao_ri_

\section{あげる/くれる/もらう (Continued)}%
\label{sec:ageru_kureru_morau_continued}
% section ageru_kureru_morau_continued

\begin{ex}[げんき II - 66ページ、VパートD]
  Refer to the following diagram and describe who gave what to whom using あげる/くれる/もらう.

  \begin{center}
  \begin{tikzpicture}
  \tikzset{
    giveto/.style={midarrow={latex}{0.8},line width=1pt},
    item/.style={midway, rectangle, draw=none, fill=base16-eighties-dark},
    name/.style={rectangle, draw, fill=base16-eighties-dark}
  }
    \draw[giveto] (0,0) -- (-6.5,3) node[item] {チョコレート};
    \draw[giveto] (-2,3) -- (-6,3) node[item] {トレーナー};
    \draw[giveto] (-2.5,3) -- (0,0) node[item] {ぬいぐるみ};
    \draw[giveto] (0.5,3) -- (0,0) node[item] {手袋};
    \draw[giveto] (0.7,3) -- (3.5,2.3) node[item] {靴};
    \draw[giveto] (2.6,1.7) -- (0,3) node[item] {花束};
    \draw[giveto] (0,0) -- (3,2) node[item] {漫画};
    \draw[giveto] (0,-3) -- (3,-2) node[item] {本};
    \draw[giveto] (0,-3) -- (0,0) node[item] {セーター};
    \draw[giveto] (-5,-3) -- (-3,0) node[item] {ネクタイ};
    \draw[giveto] (-3,0) -- (0,0) node[item] {みかん};
    \node[name] at (0,0) {私};
    \node[name] at (-6,3) {ディエゴ};
    \node[name] at (-2,3) {きょうこ};
    \node[name] at (0,3) {メアリー};
    \node[name] at (3,2) {たけし};
    \node[name] at (3,-2) {スー};
    \node[name] at (0,-3) {ロバート};
    \node[name] at (-3,0) {けん};
    \node[name] at (-5,-3) {ナオミ};
  \end{tikzpicture}
  \end{center}

  \begin{enumerate}
    \item \textbf{トレーナー} \\
      \textbf{あげる} :きょうこさんはディエゴさんにトレーナーをあげました。\\
      \textbf{もらう} :ディエゴさんはきょうこさんにトレーナーをもらいました。

    \item \textbf{チョコレート} \\
      \textbf{あげる} :ディエゴにチョコレートをあげました。

    \item \textbf{ぬいぐるみ} \\
      \textbf{くれる} :きょうこさんがぬいぐるみをくれました。\\
      \textbf{もらう} :私はきょうこさんからぬいぐるみをもらえました。

    \item \textbf{手袋} \\
      \textbf{くれる} :メアリーさんが手袋をくれました。\\
      \textbf{もらう} :私はメアリーさんから手袋をもらえました。

    \item \textbf{花束} \\
      \textbf{あげる} :たけしさんはメアリーさんに花束をあげました。\\
      \textbf{もらう} :メアリーさんはたけしさんに花束をもらえました。
    
    \item \textbf{靴} \\
      \textbf{あげる} :メアリーさんはたけしさんに靴をあげました。\\
      \textbf{もらう} :たけしさんはメアリーさんに靴をもらえました。

    \item \textbf{漫画} \\
      \textbf{あげる} :たけしさんに漫画を上げました。\\

    \item \textbf{本} \\
      \textbf{あげる} :ロバートさんはスーさんに本をあげました。\\
      \textbf{もらう} :スーさんはロバートさんに本をもらえました。

    \item \textbf{セーター} \\
      \textbf{くれる} :ロバートさんがセーターをくれました。\\
      \textbf{もらう} :ロバートさんからセーターをもらえました。

    \item \textbf{ネクタイ} \\
      \textbf{あげる} :ナオミさんはけんさんにネクタイをあげました。\\
      \textbf{もらう} :けんさんはナオミさんからネクタイをもらえました。

    \item \textbf{みかん} \\
      \textbf{くれる} :けんさんがみかんをくれました。\\
      \textbf{もらう} :けんさんにみかんをもらえました。
  \end{enumerate}
\end{ex}

% section ageru_kureru_morau_continued (end)

\section{〜とらどうですか}%
\label{sec:_toradoudesuka}
% section _toradoudesuka

\begin{grammar}[〜たらどうですか]
\label{grammar:_taradoudesuka}
  たらどうですか after a verb conveys \hlnoteb{advice or recommendation}. The initial た in たらどうですか stands for the \hlnoteb{same ending as in the past tence short form} of a verb in the affirmative. In casual speech, たらでどうですか can be shortened to たらどう or たら.

  たらどうですか may sometimes have a \hlwarn{critical tone}, criticizing the person for not having performed the activity already. It is, therefore, safer not to use it unless you have been taped for consulation.

  Also, the pattern is \hlwarn{not used for extending invitations}. If, for example, you want to invite your friend to come visit, you do not want to use たらどうですか, but should use ません.
  \begin{center}
    うちに来ませんか。\\
    $\times$ うちに来たらどうですか。
  \end{center}
\end{grammar}

\begin{eg}
  \begin{enumerate}
    \item もっと勉強したらどうですか。
    \item \ruby[j]{薬}{くすり}をのんだらどうですか。
  \end{enumerate}
\end{eg}

\begin{ex}[げんき II - 68ページ、IVパートA]
  Give advice to the people below.
  \begin{itemize}
    \item メアリー:日本で仕事がしたいんです。
      \begin{enumerate}
        \item check newspaper
        \item consult with the teacher
        \item send r\'esum\'e to company
      \end{enumerate}
    \item ジョン:彼女がほしいです。でも、できないんです。
      \begin{enumerate}
        \setcounter{enumi}{3}
        \item go to a party
        \item join a club
        \item give up
      \end{enumerate}
    \item けん:彼女と結婚したいんです。
      \begin{enumerate}
        \setcounter{enumi}{6}
        \item propose marriage
        \item give her a ring
        \item meet her parents
      \end{enumerate}
  \end{itemize}
  
  \begin{solution}
    \begin{enumerate}
      \item 新聞を見たらどうですか。
      \item 先生に相談してたらどうですか。
      \item 履歴書を会社に送ったらどうですか。
      \item パーティーに行ったらどうですか。
      \item サークルを入ったらどうですか。
      \item 諦めたらどうですか。
      \item プロポーズしたらどうですか。
      \item 指輪をあげたらどうですか。
      \item 彼女の両親を会いに行ったらどうですか。
    \end{enumerate}
  \end{solution}
\end{ex}

\begin{ex}[げんき II - 68ページ、IVパートB]
  Give the following people the right suggestion using 〜たらどうですか.
  \begin{enumerate}
    \item 美味しいケーキは食べたいんです。\\
      \textbf{答え:} ケーキ屋に行ったらどうですか。
    \item 安いカメラがほしんです。\\
      \textbf{答え:} ネットで\ruby[j]{調}{しら}べたらどうですか。
    \item ちょっと\ruby[j]{太}{ふと}ったんです。\\
      \textbf{答え:} 運動したらどうですか。
    \item この\ruby[j]{頃}{ごろ}疲れてるんです。\\
      \textbf{答え:} 少し\ruby[j]{休憩}{きゅ|けい}したらどうですか。
    \item 勉強が\ruby[j]{大嫌}{だい|きら}いなんです。\\
      \textbf{答え:} 勉強をゲームとしたらどうですか。
    \item よく寝られないんです。\\
      \textbf{答え:} 本を読んだらどうですか。
    \item 友達ができないんです。\\
      \textbf{答え:} サークルを入ったらどうですか。
    \item お金がないんです。\\
      \textbf{答え:} アルバイトをしたらどうですか。
    \item 彼/彼女がけちで、何もくれないんです。\\
      \textbf{答え:} 彼/彼女と話したらどうですか。
  \end{enumerate}
\end{ex}

% section _toradoudesuka (end)

\section{数え方}%
\label{sec:shu_efang_}
% section shu_efang_

\hlnotea{\ruby[j]{数}{かぞ}え\ruby[j]{方}{かた}}\\
\begin{longtable}{c | c | c | c | c | c | c}
                        & こ(個)                    & さつ(冊)             & ひき(匹)                       & ほん(本)                       & だい(台) & まい(枚) \\
                        & small items                 & bound volumes          & small animal                     & long objects                     & equipment  & flat objects \\
                        \hline
1                       & \underline{いっ}こ          & \underline{いっ}さつ   & \underline{いっ}\hlnotec{ぴき}   & \underline{いっ}\hlnotec{ぽん}   & いちだい   & いちまい \\
2                       & にこ                        & にさつ                 & にひき                           & にほん                           & にだい     & にまい \\
3                       & さんこ                      & さんさつ               & さん\hlnotec{びき}               & さん\hlnotec{ぼん}               & さんだい   & さんまい \\
4                       & よんこ                      & こんさつ               & よんひき                         & よんほん                         & よんだい   & よんまい \\
5                       & ごこ                        & ごさつ                 & ごひき                           & ごほん                           & ごだい     & ごまい \\
6                       & \underline{ろっ}こ          & ろくさつ               & \underline{ろっ}\hlnotec{ぴき}   & \underline{ろっ}\hlnotec{ぽん}   & ろくだい   & ろくまい \\
7                       & ななこ                      & ななさつ               & ななひき                         & ななほん                         & ななだい   & ななまい \\
8                       & \underline{はっ}こ          & \underline{はっ}さつ   & \underline{はっ}\hlnotec{ぴき}   & \underline{はっ}\hlnotec{ぽん}   & はちだい   & はちまい \\
9                       & きゅうこ                    & きゅうさつ             & きゅうひき                       & きゅうほん                       & きゅうだい & きゅうまい \\
\multirow{2}{*}{10}     & \underline{じゅっ}こ        & \underline{じゅっ}さつ & \underline{じゅっ}\hlnotec{ぴき} & \underline{じゅっ}\hlnotec{ぽん} & じゅうだい & じゅうまい \\
                        & \underline{じっ}こ          & \underline{じっ}さつ   & \underline{じっ}\hlnotec{ぴき}   & \underline{じっ}\hlnotec{ぽん}   &            & \\
\hline
\tworow{c}{How}{many}   & \tworow{c}{なんこ/}{いくつ} & なんさつ               & なん\hlnotec{びき}               & なん\hlnotec{ぼん}               & なんだい   & なんまい \\
\hline
Eg.                     & candy                       & book                   & cat                              & pencil                           & computer   & paper \\
                        & tomato                      & magazine               & dog                              & umbrella                         & TV         & plate \\
                        & eraser                      & dictionary             & snake                            & movie                            & car        & T-shirt \\
                        &                             &                        & bottle                           & bicycle
\end{longtable}

% section shu_efang_ (end)

% chapter lecture_8_wu_yue_er_shi_ba_ri_yue_yao_ri_ (end)

\chapter{Lecture 9 五月三十日(水曜日)}%
\label{chp:lecture_9_wu_yue_san_shi_ri_shui_yao_ri_}
% chapter lecture_9_wu_yue_san_shi_ri_shui_yao_ri_

\section{number + も / number + しか + negative}%
\label{sec:number_mo_number_shika_negative}
% section number_mo_number_shika_negative

\begin{grammar}[も/しか]
\label{grammar:mo_shika}
  Recall the basic structure of expressing numbers in Japanese.
  \begin{center}
    noun \begin{lrcases} が \\ を \end{lrcases} + number
  \end{center}
  You can add も to the number word when you want to say ``as many as''.

  You can add しか to the number word, and turn the predicate into the negative when you want to say ``as few as'' or ``only''.
\end{grammar}

\begin{eg}
  \begin{enumerate}
    \item 私の母は猫を三匹も\ruby[j]{飼}{か}っています。
    \item 昨日のパーティーには学生が二十人も来ました。
    \item 私は日本語の辞書を一冊しか持っていません。
    \item この会社にはパソコンが二台しかありません。
  \end{enumerate}
\end{eg}

\begin{ex}[げんき II - 69ページ、VパートB]
  Describe the activities carried out by メアリーさん and ジョンさん as in the table below using 〜も and 〜しか.
  \begin{center}
  \begin{tabular}{c | c | c | c | c | c}
             & ハムバッカー & 本         & ディーヴィディー & ジュース   & 時間 \\
             \hline
             & 食べました   & 読みました & 持っています     & 飲みました & 寝ます \\
             \hline
    メアリー & 1            & 1          & 50               & 3          & 11 \\
    ジョン   & 4            & 6          & 2                & 1          & 5
  \end{tabular}
  \end{center}
  \begin{enumerate}
    \item メアリーさんはハムバッカーを一個しか食べません。\\
      ジョンさんはハムバッカーを四個も食べます。

    \item メアリーさんは本一冊しか読みません。\\
      ジョンさんは本を六冊も読みます。

    \item メアリーさんはディーヴィディーを五十枚を持っています。\\
      ジョンさんはディーヴィディを二枚しか持っていません。

    \item メアリーさんはジュースを三本も飲みます。\\
      ジョンさんはジュースを一本しか飲みません。

    \item メアリーさんは十一時間も寝ます。\\
      ジョンさんは五時間しか寝ません。
  \end{enumerate}
\end{ex}

% section number_mo_number_shika_negative (end)

% chapter lecture_9_wu_yue_san_shi_ri_shui_yao_ri_ (end)

\chapter{Lecture 10 六月四日(月曜日)}%
\label{chp:lecture_10_liu_yue_si_ri_yue_yao_ri_}
% chapter lecture_10_liu_yue_si_ri_yue_yao_ri_

\section{Volitional Form}%
\label{sec:volitional_form}
% section volitional_form

\begin{grammar}[Volitional Form]
\label{grammar:volitional_form}
  The \hlnoteb{volitional form} of a verb is a less formal, more casual equivalent of ましよう. You can use it to suggest a plan to a close friend, for example.

  The rule around volitional form is as follows:

  \noindent 「る」動詞: Drop the final 〜る and add 〜よう
  \begin{center}
    食べる $\to$ 食べよう
  \end{center}

  \noindent 「う」動詞: Change the final 〜う sound to 〜おう \\
  \begin{aligned}
    行く \quad &$\to$ \quad 行こう & 待つ \quad &$\to$ \quad 待とう \\
    話す \quad &$\to$ \quad 話そう & 読む \quad &$\to$ \quad 読もう \\
    買う \quad &$\to$ \quad 買おう & 死ぬ \quad &$\to$ \quad 死のう \\
    泳ぐ \quad &$\to$ \quad 泳ごう & 取る \quad &$\to$ \quad 取ろう \\
    遊ぶ \quad &$\to$ \quad 遊ぼう & &
  \end{aligned}

  \noindent irregular verbs: \\
  \begin{aligned}
    くる \quad &$\to$ \quad こよう & する \quad &$\to$ \quad しよう
  \end{aligned}

  You can also use volitional plus the question particle が to ask for an opinion in your offer suggestion.
\end{grammar}

\begin{eg}[Basic examples]
  \begin{enumerate}
    \item 明日は授業がないから、今晩、どこがに食べに行こう。
    \item 結婚しよう。
  \end{enumerate}
\end{eg}

\begin{eg}[Using が to ask for opinion for an offer]
  \begin{enumerate}
    \item 手伝おうか。
    \item 友達が面白いと言っていたから、この映画を見ようか。
    \item 今度、いつ会おうか。
  \end{enumerate}
\end{eg}

\begin{grammar}[Volitional Form + と思っています]
\label{grammar:volitional_form_tosi_tsuteimasu}
  We can use volitional form + と思っています to talk about our determinations.

  We can also we the volitional + と思います, which suggests that the decision to perform the activity is being made \hlnotec{on the spot} at the time of speaking. と思っています, in contrast, tends to suggest what you have \hlnotec{already decided} to do something.
\end{grammar}

\begin{eg}[Volitional Form + と思います]
  \begin{aligned}
    Q&: 一万円あげましょう。何に使いますか。\\
    A&: 漢字の辞書を買おうとお見ます。
  \end{aligned}
\end{eg}

\begin{eg}[Volitional Form + と思っています]
  \begin{aligneg}
    Q&: 両親から一万円もらったんですか。何に使うんですか。\\
    A&: 漢字の辞書を買おうと思っています。
  \end{aligneg}
\end{eg}

\begin{note}
  The verbs in volitional forms and verbs in the present tense convery different ideas when they are used with と思います or と思っています. When you use volitionals, you are talking about your intention. When you use the present tense, you are talking about your prediction.
\end{note}

\begin{eg}[Difference between present and volitional when appended with と思います]
  \begin{aligned}
    &\text{Volitional:} \\
    &日本の会社で働こうとと思います。\\
    &\text{I will/intend to work for a Japanese company.} \\
    &$ $\\
    &\text{Present:} \\
    &日本の会社で働くと思います。\\
    &\text{I think they/I will be working for a Japanese company.}
  \end{aligned}
\end{eg}

\begin{ex}[げんき II - 85ページ、IパートB]
  Create sentences for the following actions using volitional form.
  \begin{enumerate}
    \item drink coffee at a coffee shop \\
      喫茶店でコーヒーを飲もうか。
    \item read magazines in the library \\
      図書館で雑誌を読もうか。
    \item see a movie in town \\
      町で映画を見ようか。
    \item take pictures at school \\
      学校で写真を撮ろうか。
    \item swim in pool \\
      プールに泳ごうか。
    \item buy hamburgers at McDonald's \\
      マックドナルドでハムバッカーを買おうか。
    \item dance at a club \\
      クラブで踊ろうか。
    \item climb a mountain in Nagano \\
      長野に山を登ろうか。
    \item have a barbecue at a park \\
      公園でバビキュにしようか。
  \end{enumerate}
\end{ex}

% section volitional_form (end)

% chapter lecture_10_liu_yue_si_ri_yue_yao_ri_ (end)

\chapter{Lecture 11 六月六日(水曜日)}%
\label{chp:lecture_11_liu_yue_liu_ri_shui_yao_ri_}
% chapter lecture_11_liu_yue_liu_ri_shui_yao_ri_

\section{Volitional Form (Continued)}%
\label{sec:volitional_form_continued}
% section volitional_form_continued

\begin{ex}
  Describe what each of the following person is planning to do using volitional forms.

  \begin{enumerate}
    \item きょうこ (do physical exercise) \\
      きょうこさんは体育運動をしようと思っています。
    \item 山下先生 (quit smoking) \\
      山下先生はタバコをやめようと思っています。
    \item ともこ (go on a diet) \\
      ともこさんはダイエットをしようと思っています。
    \item ジョン (get up early in the morning) \\
      ジョンさんは早く起きようと思っています。
    \item ロバート (practise Japanese all day) \\
      ロバートさんは一日中に日本語を練習しようと思っています。
    \item たけし (eat more vegetables) \\
      たけしさんはもっと野菜を食べようと思っています。
    \item スー (make lots of Japanese friends) \\
      スーさんは日本人の友達をたくさん作ろうと思っています。
    \item けん (look for a job) \\
      けんさんは仕事を探そうと思っています。
  \end{enumerate}
\end{ex}

% section volitional_form_continued (end)

% chapter lecture_11_liu_yue_liu_ri_shui_yao_ri_ (end)

\chapter{Lecture 12 六月十一日(月曜日)}%
\label{chp:lecture_12_liu_yue_shi_yi_ri_yue_yao_ri_}
% chapter lecture_12_liu_yue_shi_yi_ri_yue_yao_ri_

\section{〜ておく}%
\label{sec:_teoku}
% section _teoku

\begin{grammar}[〜ておく]
\label{grammar:_teoku}
The て -form of a verb plus the helping verb おく describes an action performed \hlnoteb{in preparation for something}. ておく is often shortened to とく in speech.
\end{grammar}

\begin{eg}
  \begin{enumerate}
    \item 明日試験があるので、今晩勉強しておくます。
    \item 友達が来るから、部屋を掃除しておかなきゃいけません。
    \item ホテルを予約しとくね。
  \end{enumerate}
\end{eg}

\begin{ex}[げんき II - 87 ぺージ、III パート A]
  Describe what the people below will do in advance.

  \begin{enumerate}
    \item メアリー (水と食べ物を買う) \\
      メアリーさんは水と食べ物を買っておきます。
    \item スー (お金を下ろす)\\
      スーさんはお金を下ろしておきます。
    \item ロバート (お金を借りる)\\
      ロバートさんはお金を借りておきます。
    \item 山下先生 (うちを売る) \\
      山下先生はうちを売っておきます。
    \item たけしのお母さん (保険に入る)\\
      たけしのお母さんは保険に入っておきます。
    \item ともこ (大きい家具を捨てる)\\
      ともこさんは大きい家具を捨てておきます。
    \item たけし (たくさん食べる)\\
      たけしさんはたくさん食べておきます。
  \end{enumerate}
\end{ex}

\begin{ex}[げんき II - 88 ページ、III パート B]
  What do you need to prepare for the following situations? Create sentences to describe your preparations using 〜おく.

  \begin{enumerate}
    \item 来週ハワイに行きます。\\
      フライトをもう一回確認しておきます。\\
      行きたい場所をインターネットで調べておきます
    \item 両親が来ます。
      家を掃除しておきます。
    \item デートをします。\\
      面白い恋愛スポットを調べておきます。\\
      適当な服を準備しておきます。
    \item パーティーをします。
      友達が招待しておきます。\\
      パーティー内のイベントを考えておきます。
  \end{enumerate}
\end{ex}

% section _teoku (end)

\section{Using Sentences to Qualify Nouns}%
\label{sec:using_sentences_to_qualify_nouns}
% section using_sentences_to_qualify_nouns

\begin{grammar}[Using Sentences to Qualify Nouns]
\label{grammar:using_sentences_to_qualify_nouns}
  In the phrase 面白い本, the い -adjective 面白い qualifies the noun 本 and tells us what kind of book it is. We can also use sentences to qualify nouns.

  We can use a noun with a qualifier sentence just like any other noun. In other words, a ``qualifier sentence + noun'' combination is just like one big noun phrase. We can put it anywhere in a sentence that has a noun.
\end{grammar}

\begin{eg}[Basic Examples]
  \begin{itemize}
    \item \underline{昨日買った}本
    \item \underline{彼がくれた}本
    \item \underline{机の上にある}本
    \item \underline{日本で買えない}本
  \end{itemize}
\end{eg}

\begin{note}
  Qualifier sentences in the examples above tell us what kind of book we are talking about, just like adjectives. The verbs used in such qualifier sentences are in their short forms, either in the present (as in the 3rd and 4th example) or the past tence (as in 1st and 2nd), and either in the affirmative (1st - 3rd) or in the ngative (4th).

  \hlimpo{Note that} when the subject of the verb, i.e. the person performing the activity, appears inside a qualifier sentence, as in the 2nd example, it is accompanied by the particle が instead of は.
\end{note}

\begin{eg}[Using the qualifier sentence]
  \begin{itemize}
    \item これは\underline{去年の誕生日に彼女はくれた}本です。
    \item 父が\underline{村上春樹が書いた}本をくれました。
    \item \underline{私が一番感動した}映画は「生きる」です。
  \end{itemize}
\end{eg}

\begin{ex}[げんき II - 89 ページ、 IV パート B]
  You are a collector of items associated with world-famous figures. Introduce your collection to your guests.

  \begin{enumerate}
    \item a nunchaku Bruce Lee used \\
      これはブルース・りーが使ったヌンチャクです。
    \item a picture Picasso drew \\
      これはピカソが描いた絵です。
    \item a piano Beethoven played \\
      これはベートーベンが弾いたピアノです。
    \item a jacket Michael Jackson wore \\
      これはマイケル・ジャクソンが着たジャケットです。
    \item a bike Che Guevara rode \\
      これはチェ・ゲバラが乗ったバイクです。
    \item a letter Ghandi wrote \\
      これはガンジーが書いた手紙です。
    \item a movie Kurosawa made \\
      これは黒沢監督作った映画です。
    \item a telephone Bell made \\
      これはベルが作った電話です。
    \item a cap Mao Tse-tung wore \\
      これは毛沢東がかぶった帽子です。
  \end{enumerate}
\end{ex}

% section using_sentences_to_qualify_nouns (end)

% chapter lecture_12_liu_yue_shi_yi_ri_yue_yao_ri_ (end)

\chapter*{Appendix}%
\label{chp:appendix}
% chapter appendix

\hlnotec{日の読み方}

\begin{fullwidth}
\begin{tabular}{c c c c c c c}
日曜日                          & 月曜日                        & 火曜日                          & 水曜日                            & 木曜日                        & 金曜日                          & 土曜日 \\
                                & 1                             & 2                               & 3                                 & 4                             & 5                               & 6 \\
                                & ついたち                      & ふつか                          & みっか                            & よっか                        & いつか                          & むいか \\
7                               & 8                             & 9                               & 10                                & 11                            & 12                              & 13 \\
なのか                          & ようか                        & ここのか                        & とうか                            & \scriptsize{ じゅういちにち } & \scriptsize{ じゅうににち }     & \scriptsize{ じゅうさんにち } \\
14                              & 15                            & 16                              & 17                                & 18                            & 19                              & 20 \\
\scriptsize{ じゅうよっか }     & \scriptsize{ じゅうごにち }   & \scriptsize{ じゅうろくにち }   & \scriptsize{ じゅうしちにち }     & \scriptsize{ じゅうはちにち } & \scriptsize{ じゅうくにち }     & はつか \\
21                              & 22                            & 23                              & 24                                & 25                            & 26                              & 27 \\
\scriptsize{ にじゅういちにち } & \scriptsize{ にじゅうににち } & \scriptsize{ にじゅうさんいち } & \scriptsize{ にじゅうよっか }     & \scriptsize{ にじゅうごにち } & \scriptsize{ にじゅうろくにち } & \scriptsize{ にじゅうしちにち } \\
28                              & 29                            & 30                              & 31 \\
\scriptsize{ にじゅうはちにち } & \scriptsize{ にじゅうくにち } & \scriptsize{ さんじゅうにち }   & \scriptsize{ さんじゅういちにち }
\end{tabular}
\end{fullwidth}

% chapter appendix (end)

\nobibliography*
\bibliography{bibliography}

\printindex
\end{document}
