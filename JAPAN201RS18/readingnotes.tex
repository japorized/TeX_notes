% !TEX TS-program = lualatex
\documentclass[notoc,notitlepage]{tufte-book}
\usepackage{luatexja}
\usepackage{pxrubrica}
\nonstopmode
\setcounter{secnumdepth}{3}
\setcounter{tocdepth}{5}

\renewcommand{\baselinestretch}{1.1}

\usepackage{CJKutf8}
\setcounter{secnumdepth}{3}
\setcounter{tocdepth}{3}

\renewcommand{\baselinestretch}{1.1}
\usepackage{geometry}
\geometry{letterpaper}
\usepackage[parfill]{parskip}
\usepackage{graphicx}

% Essential Packages
\usepackage{makeidx}
\makeindex
\usepackage{enumitem}
\usepackage[T1]{fontenc}
\usepackage{natbib}
\bibliographystyle{apalike}
\usepackage{ragged2e}
\usepackage{etoolbox}
\usepackage{amssymb}
\usepackage{fontawesome}
\usepackage{amsmath}
\usepackage{mathrsfs}
\usepackage{mathtools}
\usepackage{xparse}
\usepackage{tkz-euclide}
\usetkzobj{all}
\usepackage[utf8]{inputenc}
\usepackage{csquotes}
\usepackage[english]{babel}
\usepackage{marvosym}
\usepackage{pgf,tikz}
\usepackage{pgfplots}
\usepackage{fancyhdr}
\usepackage{array}
\usepackage{faktor}
\usepackage{float}
\usepackage{xcolor}
\usepackage{centernot}
\usepackage{silence}
  \WarningFilter*{latex}{Marginpar on page \thepage\space moved}
\usepackage{tcolorbox}
\tcbuselibrary{skins,breakable}
\usepackage{longtable}
\usepackage[amsmath,hyperref]{ntheorem}
\usepackage{hyperref}
\usepackage[noabbrev,capitalize,nameinlink]{cleveref}

% xcolor (scheme: base16 eighties)
\definecolor{base16-eighties-dark}{HTML}{2D2D2D}
\definecolor{base16-eighties-light}{HTML}{D3D0C8}
\definecolor{base16-eighties-magenta}{HTML}{CD98CD}
\definecolor{base16-eighties-red}{HTML}{F47678}
\definecolor{base16-eighties-yellow}{HTML}{E2B552}
\definecolor{base16-eighties-green}{HTML}{98CD97}
\definecolor{base16-eighties-lightblue}{HTML}{61CCCD}
\definecolor{base16-eighties-blue}{HTML}{6498CE}
\definecolor{base16-eighties-brown}{HTML}{D47B4E}
\definecolor{base16-eighties-gray}{HTML}{747369}

% hyperref Package Settings
\hypersetup{
    bookmarks=true,         % show bookmarks bar?
    unicode=true,          % non-Latin characters in Acrobat’s bookmarks
    pdftoolbar=false,        % show Acrobat’s toolbar?
    pdfmenubar=false,        % show Acrobat’s menu?
    pdffitwindow=true,     % window fit to page when opened
    colorlinks=true,
    allcolors=base16-eighties-magenta,
}

% tikz
\usepgfplotslibrary{polar}
\usepgflibrary{shapes.geometric}
\usetikzlibrary{angles,patterns,calc,decorations.markings}
\tikzset{midarrow/.style 2 args={
        decoration={markings,
            mark= at position #2 with {\arrow{#1}} ,
        },
        postaction={decorate}
    },
    midarrow/.default={latex}{0.5}
}
\def\centerarc[#1](#2)(#3:#4:#5)% Syntax: [draw options] (center) (initial angle:final angle:radius)
    { \draw[#1] ($(#2)+({#5*cos(#3)},{#5*sin(#3)})$) arc (#3:#4:#5); }

% enumitems
\newlist{inlinelist}{enumerate*}{1}
\setlist*[inlinelist,1]{%
  label=(\roman*),
}

% Theorem Style Customization
\setlength\theorempreskipamount{2ex}
\setlength\theorempostskipamount{3ex}

\makeatletter
\let\nobreakitem\item
\let\@nobreakitem\@item
\patchcmd{\nobreakitem}{\@item}{\@nobreakitem}{}{}
\patchcmd{\nobreakitem}{\@item}{\@nobreakitem}{}{}
\patchcmd{\@nobreakitem}{\@itempenalty}{\@M}{}{}
\patchcmd{\@xthm}{\ignorespaces}{\nobreak\ignorespaces}{}{}
\patchcmd{\@ythm}{\ignorespaces}{\nobreak\ignorespaces}{}{}

\renewtheoremstyle{break}%
  {\item{\theorem@headerfont
          ##1\ ##2\theorem@separator}\hskip\labelsep\relax\nobreakitem}%
  {\item{\theorem@headerfont
          ##1\ ##2\ (##3)\theorem@separator}\hskip\labelsep\relax\nobreakitem}
\makeatother

% ntheorem + framed
\makeatletter

% ntheorem Declarations
\theorempreskip{10pt}
\theorempostskip{5pt}
\theoremstyle{break}

\newtheorem*{solution}{\faPencil $\enspace$ Solution}
\newtheorem*{remark}{Remark}
\newtheorem{eg}{Example}[section]
\newtheorem{ex}{Exercise}[section]

    % definition env
\theoremprework{\textcolor{base16-eighties-blue}{\hrule height 2pt}}
\theoremheaderfont{\color{base16-eighties-blue}\normalfont\bfseries}
\theorempostwork{\textcolor{base16-eighties-blue}{\hrule height 2pt}}
\theoremindent10pt
\newtheorem{defn}{\faBook \enspace Definition}

    % definition env no num
\theoremprework{\textcolor{base16-eighties-blue}{\hrule height 2pt}}
\theoremheaderfont{\color{base16-eighties-blue}\normalfont\bfseries}
\theorempostwork{\textcolor{base16-eighties-blue}{\hrule height 2pt}}
\theoremindent10pt
\newtheorem*{defnnonum}{\faBook \enspace Definition}

    % theorem envs
\theoremprework{\textcolor{base16-eighties-magenta}{\hrule height 2pt}}
\theoremheaderfont{\color{base16-eighties-magenta}\normalfont\bfseries}
\theorempostwork{\textcolor{base16-eighties-magenta}{\hrule height 2pt}}
\theoremindent10pt
\newtheorem{thm}{\faCoffee \enspace Theorem}

\theoremprework{\textcolor{base16-eighties-magenta}{\hrule height 2pt}}
\theorempostwork{\textcolor{base16-eighties-magenta}{\hrule height 2pt}}
\theoremindent10pt
\newtheorem{propo}[thm]{\faTint \enspace Proposition}

\theoremprework{\textcolor{base16-eighties-magenta}{\hrule height 2pt}}
\theorempostwork{\textcolor{base16-eighties-magenta}{\hrule height 2pt}}
\theoremindent10pt
\newtheorem{crly}[thm]{\faSpaceShuttle \enspace Corollary}

\theoremprework{\textcolor{base16-eighties-magenta}{\hrule height 2pt}}
\theorempostwork{\textcolor{base16-eighties-magenta}{\hrule height 2pt}}
\theoremindent10pt
\newtheorem{lemma}[thm]{\faTree \enspace Lemma}

\theoremprework{\textcolor{base16-eighties-magenta}{\hrule height 2pt}}
\theorempostwork{\textcolor{base16-eighties-magenta}{\hrule height 2pt}}
\theoremindent10pt
\newtheorem{axiom}[thm]{\faShield \enspace Axiom}

    % theorem envs without counter
\theoremprework{\textcolor{base16-eighties-magenta}{\hrule height 2pt}}
\theoremheaderfont{\color{base16-eighties-magenta}\normalfont\bfseries}
\theorempostwork{\textcolor{base16-eighties-magenta}{\hrule height 2pt}}
\theoremindent10pt
\newtheorem*{thmnonum}{\faCoffee \enspace Theorem}

\theoremprework{\textcolor{base16-eighties-magenta}{\hrule height 2pt}}
\theorempostwork{\textcolor{base16-eighties-magenta}{\hrule height 2pt}}
\theoremindent10pt
\newtheorem*{propononum}{\faTint \enspace Proposition}

\theoremprework{\textcolor{base16-eighties-magenta}{\hrule height 2pt}}
\theorempostwork{\textcolor{base16-eighties-magenta}{\hrule height 2pt}}
\theoremindent10pt
\newtheorem*{crlynonum}{\faSpaceShuttle \enspace Corollary}

\theoremprework{\textcolor{base16-eighties-magenta}{\hrule height 2pt}}
\theorempostwork{\textcolor{base16-eighties-magenta}{\hrule height 2pt}}
\theoremindent10pt
\newtheorem*{lemmanonum}{\faTree \enspace Lemma}

\theoremprework{\textcolor{base16-eighties-magenta}{\hrule height 2pt}}
\theorempostwork{\textcolor{base16-eighties-magenta}{\hrule height 2pt}}
\theoremindent10pt
\newtheorem*{axiomnonum}{\faShield \enspace Axiom}

    % proof env
\theoremprework{\textcolor{base16-eighties-brown}{\hrule height 2pt}}
\theoremheaderfont{\color{base16-eighties-brown}\normalfont\bfseries}
\theorempostwork{\textcolor{base16-eighties-brown}{\hrule height 2pt}}
\newtheorem*{proof}{\faPencil \enspace Proof}

    % note and notation env
\theoremprework{\textcolor{base16-eighties-yellow}{\hrule height 2pt}}
\theoremheaderfont{\color{base16-eighties-yellow}\normalfont\bfseries}
\theorempostwork{\textcolor{base16-eighties-yellow}{\hrule height 2pt}}
\newtheorem*{note}{\faQuoteLeft \enspace Note}

\theoremprework{\textcolor{base16-eighties-yellow}{\hrule height 2pt}}
\theorempostwork{\textcolor{base16-eighties-yellow}{\hrule height 2pt}}
\newtheorem*{notation}{\faPaw \enspace Notation}

    % warning env
\theoremprework{\textcolor{base16-eighties-red}{\hrule height 2pt}}
\theoremheaderfont{\color{base16-eighties-red}\normalfont\bfseries}
\theorempostwork{\textcolor{base16-eighties-red}{\hrule height 2pt}}
\theoremindent10pt
\newtheorem*{warning}{\faBug \enspace Warning}

% more environments
\newtcolorbox{redquote}{
  blanker,enhanced,breakable,standard jigsaw,
  opacityback=0,
  coltext=base16-eighties-light,
  left=5mm,right=5mm,top=2mm,bottom=2mm,
  colframe=base16-eighties-red,
  boxrule=0pt,leftrule=3pt,
  fontupper=\itshape
}
\newtcolorbox{bluequote}{
  blanker,enhanced,breakable,standard jigsaw,
  opacityback=0,
  coltext=base16-eighties-light,
  left=5mm,right=5mm,top=2mm,bottom=2mm,
  colframe=base16-eighties-blue,
  boxrule=0pt,leftrule=3pt,
  fontupper=\itshape
}
\newtcolorbox{greenquote}{
  blanker,enhanced,breakable,standard jigsaw,
  opacityback=0,
  coltext=base16-eighties-light,
  left=5mm,right=5mm,top=2mm,bottom=2mm,
  colframe=base16-eighties-green,
  boxrule=0pt,leftrule=3pt,
  fontupper=\itshape
}
\newtcolorbox{yellowquote}{
  blanker,enhanced,breakable,standard jigsaw,
  opacityback=0,
  coltext=base16-eighties-light,
  left=5mm,right=5mm,top=2mm,bottom=2mm,
  colframe=base16-eighties-yellow,
  boxrule=0pt,leftrule=3pt,
  fontupper=\itshape
}
\newtcolorbox{magentaquote}{
  blanker,enhanced,breakable,standard jigsaw,
  opacityback=0,
  coltext=base16-eighties-light,
  left=5mm,right=5mm,top=2mm,bottom=2mm,
  colframe=base16-eighties-magenta,
  boxrule=0pt,leftrule=3pt,
  fontupper=\itshape
}

% ntheorem listtheorem style
\makeatother
\newlength\widesttheorem
\AtBeginDocument{
  \settowidth{\widesttheorem}{Proposition A.1.1.1\quad}
}

\makeatletter
\def\thm@@thmline@name#1#2#3#4{%
        \@dottedtocline{-2}{0em}{2.3em}%
                   {\makebox[\widesttheorem][l]{#1 \protect\numberline{#2}}#3}%
                   {#4}}
\@ifpackageloaded{hyperref}{
\def\thm@@thmline@name#1#2#3#4#5{%
    \ifx\#5\%
        \@dottedtocline{-2}{0em}{2.3em}%
            {\makebox[\widesttheorem][l]{#1 \protect\numberline{#2}}#3}%
            {#4}
    \else
        \ifHy@linktocpage\relax\relax
            \@dottedtocline{-2}{0em}{2.3em}%
                {\makebox[\widesttheorem][l]{#1 \protect\numberline{#2}}#3}%
                {\hyper@linkstart{link}{#5}{#4}\hyper@linkend}%
        \else
            \@dottedtocline{-2}{0em}{2.3em}%
                {\hyper@linkstart{link}{#5}%
                  {\makebox[\widesttheorem][l]{#1 \protect\numberline{#2}}#3}\hyper@linkend}%
                    {#4}%
        \fi
    \fi}
}

\makeatletter
\def\thm@@thmline@noname#1#2#3#4{%
        \@dottedtocline{-2}{0em}{5em}%
                   {{\protect\numberline{#2}}#3}%
                   {#4}}
\@ifpackageloaded{hyperref}{
\def\thm@@thmline@noname#1#2#3#4#5{%
    \ifx\#5\%
        \@dottedtocline{-2}{0em}{5em}%
            {{\protect\numberline{#2}}#3}%
            {#4}
    \else
        \ifHy@linktocpage\relax\relax
            \@dottedtocline{-2}{0em}{5em}%
                {{\protect\numberline{#2}}#3}%
                {\hyper@linkstart{link}{#5}{#4}\hyper@linkend}%
        \else
            \@dottedtocline{-2}{0em}{5em}%
                {\hyper@linkstart{link}{#5}%
                  {{\protect\numberline{#2}}#3}\hyper@linkend}%
                    {#4}%
        \fi
    \fi}
}

\theoremlisttype{allname}

\AtBeginDocument{\renewcommand\contentsname{Table of Contents}}

% Heading formattings
% chapter format
\titleformat{\chapter}%
  {\huge\rmfamily\itshape\color{base16-eighties-magenta}}% format applied to label+text
  {\llap{\colorbox{base16-eighties-magenta}{\parbox{1.5cm}{\hfill\itshape\huge\textcolor{base16-eighties-dark}{\thechapter}}}}}% label
  {5pt}% horizontal separation between label and title body
  {}% before the title body
  []% after the title body

% section format
\titleformat{\section}%
  {\normalfont\Large\rmfamily\itshape\color{base16-eighties-blue}}% format applied to label+text
  {\llap{\colorbox{base16-eighties-blue}{\parbox{1.5cm}{\hfill\itshape\textcolor{base16-eighties-dark}{\thesection}}}}}% label
  {5pt}% horizontal separation between label and title body
  {}% before the title body
  []% after the title body

% subsection format
\titleformat{\subsection}%
  {\normalfont\large\itshape\color{base16-eighties-green}}% format applied to label+text
  {\llap{\colorbox{base16-eighties-green}{\parbox{1.5cm}{\hfill\textcolor{base16-eighties-dark}{\thesubsection}}}}}% label
  {1em}% horizontal separation between label and title body
  {}% before the title body
  []% after the title body

% Sidenote enhancements
\def\mathmarginnote#1{%
  \tag*{\rlap{\hspace\marginparsep\smash{\parbox[t]{\marginparwidth}{%
  \footnotesize#1}}}}
}

% Custom table columning
\newcolumntype{L}[1]{>{\raggedright\let\newline\\\arraybackslash\hspace{0pt}}m{#1}}
\newcolumntype{C}[1]{>{\centering\let\newline\\\arraybackslash\hspace{0pt}}m{#1}}
\newcolumntype{R}[1]{>{\raggedleft\let\newline\\\arraybackslash\hspace{0pt}}m{#1}}

% Custom math operator
% \DeclareMathOperator{\rem}{rem}
\DeclareMathOperator*{\argmax}{arg\,max}
\DeclareMathOperator*{\argmin}{arg\,min}
\DeclareMathOperator{\re}{Re}
\DeclareMathOperator{\im}{Im}
\DeclareMathOperator{\caparg}{Arg}
\DeclareMathOperator{\Ind}{Ind}
\DeclareMathOperator{\Res}{Res}

% Graph styles
\pgfplotsset{compat=1.15}
\usepgfplotslibrary{fillbetween}
\pgfplotsset{four quads/.append style={axis x line=middle, axis y line=
middle, xlabel={$x$}, ylabel={$y$}, axis equal }}
\pgfplotsset{four quad complex/.append style={axis x line=middle, axis y line=
middle, xlabel={$\re$}, ylabel={$\im$}, axis equal }}

% Shortcuts
\newcommand{\floor}[1]{\lfloor #1 \rfloor}      % simplifying the writing of a floor function
\newcommand{\ceiling}[1]{\lceil #1 \rceil}      % simplifying the writing of a ceiling function
\newcommand{\dotp}{\, \cdotp}			        % dot product to distinguish from \cdot
\newcommand{\qed}{\hfill\ensuremath{\square}}   % Q.E.D sign
\newcommand{\abs}[1]{\left|#1\right|}						% absolute value
\newcommand{\lra}[1]{\langle \; #1 \; \rangle}
\newcommand{\at}[2]{\Big|_{#1}^{#2}}
\newcommand{\Arg}[1]{\caparg #1}
\renewcommand{\bar}[1]{\mkern 1.5mu \overline{\mkern -1.5mu #1 \mkern -1.5mu} \mkern 1.5mu}
\newcommand{\quotient}[2]{\faktor{#1}{#2}}
\newcommand{\cyclic}[1]{\left\langle #1 \right\rangle}
	% highlighting shortcuts
\newcommand{\hlimpo}[1]{\textcolor{base16-eighties-red}{\textbf{#1}}}
\newcommand{\hlwarn}[1]{\textcolor{base16-eighties-yellow}{\textbf{#1}}}
\newcommand{\hldefn}[1]{\textcolor{base16-eighties-blue}{\index{#1}\textbf{#1}}}
\newcommand{\hlnotea}[1]{\textcolor{base16-eighties-green}{\textbf{#1}}}
\newcommand{\hlnoteb}[1]{\textcolor{base16-eighties-lightblue}{\textbf{#1}}}
\newcommand{\hlnotec}[1]{\textcolor{base16-eighties-brown}{\textbf{#1}}}
\newcommand{\WTP}{\textcolor{base16-eighties-brown}{WTP} }
\newcommand{\WTS}{\textcolor{base16-eighties-brown}{WTS} }
\newcommand{\ind}[2]{\Ind_{#2}\left( #1 \right)}
\newcommand{\notimply}{\centernot\implies}
\newcommand{\res}[2]{\underset{#2}{\Res} #1 }
\newcommand{\tworow}[3]{\begin{tabular}{@{}#1@{}} #2 \\ #3 \end{tabular}}
\renewcommand{\epsilon}{\varepsilon}
\newcommand{\lrarrow}{\leftrightarrow}
\newcommand{\larrow}{\leftarrow}
\newcommand{\rarrow}{\rightarrow}
\renewcommand{\atop}[2]{\genfrac{}{}{0pt}{}{#1}{#2}}
\newcommand*\dif{\mathop{}\!d}

  % inspiration from: https://tex.stackexchange.com/questions/8720/overbrace-underbrace-but-with-an-arrow-instead#37758
\newcommand{\overarrow}[2]{
  \overset{\makebox[0pt]{\begin{tabular}{@{}c@{}}#2\\[0pt]\ensuremath{\uparrow}\end{tabular}}}{#1}
}
\newcommand{\underarrow}[2]{
  \underset{\makebox[0pt]{\begin{tabular}{@{}c@{}}\downarrow\\[0pt]\ensuremath{#2}\end{tabular}}}{#1}
}

% Document header formatting
\renewcommand{\chaptermark}[1]{\markboth{#1}{}}
\renewcommand{\sectionmark}[1]{\markright{#1}}
\makeatletter
\pagestyle{fancy}
\fancyhead{}
\fancyhead[RO]{\textsl{\@title} \enspace \thepage}
\fancyhead[LE]{\thepage \enspace \textsl{\leftmark \enspace - \enspace \rightmark}}
\makeatother

% Comment the two lines below if you want to print the document
\pagecolor{base16-eighties-dark}
\color{base16-eighties-light}


\newenvironment{lrcases}
  {\left\lbrace\quad\begin{aligned}}
  {\end{aligned}\quad\right\rbrace}
\newenvironment{rcases}
  {\left.\quad\begin{aligned}}
  {\end{aligned}\quad\right\rbrace}
\renewcommand\rubysize{.6}

\theoremprework{\textcolor{base16-eighties-blue}{\hrule height 2pt}}
\theoremheaderfont{\color{base16-eighties-blue}\normalfont\bfseries}
\theorempostwork{\textcolor{base16-eighties-blue}{\hrule height 2pt}}
\theoremindent10pt
\newtheorem{grammar}{文法}[section]

\title{JAPAN201RS18}
\author{Johnson Ng}

% Header formatting
\renewcommand{\chaptermark}[1]{\markboth{#1}{}}
\renewcommand{\sectionmark}[1]{\markright{#1}}
\makeatletter
\pagestyle{fancy}
\fancyhead{}
\fancyhead[RO]{\textsl{\@title} \enspace \thepage}
\fancyhead[LE]{\thepage \enspace \textsl{\leftmark \enspace - \enspace \rightmark}}
\makeatother

\begin{document}

\hypersetup{pageanchor=false}
\maketitle
\hypersetup{pageanchor=true}
\tableofcontents

\chapter{Lecture 1 五月三日(水曜日)}
  \label{chapter:lecture_1_may_03rd_2018}

\section{Potential Forms} % (fold)
\label{sec:potential_forms}

We have the following rule for changing verbs to their potential forms:
\begin{itemize}
  \item る-verb $\to$ られる 
  \item う-verb $\to$ 〜える 
  \item Irregular verbs: くる $\to$ こられる $\quad$ する $\to$ できる
\end{itemize}

\begin{eg}
\hlnotea{る-verbs}
  \begin{itemize}
    \item 食べる $\to$ 食べられる $\to$ 食べられます
    \item 見る $\to$ 見られる $\to$ 見られます
  \end{itemize}
  
\noindent\hlnotea{う-verbs}
  \begin{itemize}
    \item 行く $\to$ 行ける $\to$ 行けます
    \item 飲む $\to$ 飲める $\to$ 飲められます
    \item 話す $\to$ 話せる $\to$ 話せます
  \end{itemize}
\end{eg}

\begin{ex}[げんきII - 37ページ、パートA]
Convert the base forms to their potential forms: \marginnote{We can drop the ら in 〜られる. This was originally colloquial, but has now become the norm that it is acceptable to do this even in written text.} \\
  \begin{tabular}{l l}
    1. はなす $\to$ はなせる   & 2. する $\to$ できる \\
    3. いく $\to$ いける       & 4. ねる $\to$ ねられる \\
    5. くる $\to$ こられる     & 6. みる $\to$ みられる \\
    7. やめる $\to$ やめられる & 8. かりる $\to$ かりられる \\
    9. のむ $\to$ のめる       & 10. まつ $\to$ まてる \\
    11. およぐ $\to$ およげる  & 12. はたらく $\to$ はたらける \\
    13. あむ $\to$ あめる
  \end{tabular}
\end{ex}

\begin{ex}[げんきII - 37ページ、パートB]
  Describe the things that Mary can do.
  \begin{enumerate}
    \item メアリーさんは日本語曲がう歌えます。
    \item メアリーさんはヴァイオリンが弾けます。
    \item メアリーさんは空手道ができます。
    \item メアリーさんは寿司が食べられます。
    \item メアリーさんは料理ができます。
    \item メアリーさんは日本語で電話がかけられます。
    \item メアリーさんは車を運転ができます。
    \item メアリーさんはセーターが編めます。
    \item メアリーさんは日本語で手紙がかけます。
    \item メアリーさんは朝早くで起きられる。
    \item メアリーさんは温かいお風呂に入られる。
  \end{enumerate}
\end{ex}

% section potential_forms (end)

% chapter lecture_1_may_03rd_2018 (end)

\chapter{Lecture 2 五月七日(月曜日)}
  \label{chapter:lecture_2_may_07th_2018}

\section{Vocabulary 1}
\label{sect:vocabulary_1}

\hlnotea{名詞 Nouns}
\marginnote{* are words that appear in dialogues.}

\begin{tabular}{r l l l}
  *  & ウェイター &        & waiter \\
     & おたく     & お宅   & (someone's) house/home \\
     & おとな     & 大人   & adult \\
     & がいこくご & 外国語 & foreign language \\
     & がっき     & 楽曲   & musical instruments \\
     & からて     & 空手   & karate \\
  *  & カレー     &        & curry \\
     & きもの     & 着物   & kimono; Japanese traditional dress \\
  *  & こうこく   & 広告   & advertisement \\
     & こうちゃ   & 紅茶   & black tea \\
     & ことば     & 言葉   & language \\
     & ゴルフ     &        & golf \\
     & セーター   &        & sweater \\
     & ぞう       & 象     & elephant \\
     & バイオリン &        & violin \\
     & バイク     &        & motorcycle \\
     & ぶっか     & 物価   & (consumer) prices \\
     & ぶんぽう   & 文法   & grammar \\
     & べんごし   & 弁護士 & lawyer \\
  *  & ぼしゅう   & 募集   & recruitment \\
  *  & みせ       & 店     & shop; store \\
     & やくざ     &        & \textit{yakuza}; gangster \\
     & やくそく   & 約束   & promise; appointment \\
     & レポート   &        & (term) paper; report \\
  *  & わたくし   & 私     & I (formal)
\end{tabular}

\hlnotea{い終わるの形容詞 | い - adjectives} \\
\begin{tabular}{r l l l}
   & うれしい & 嬉しい & glad \\
   & かなしい & 悲しい & sad \\
   & からい   & 辛い   & hot and spicy; salty \\
   & きびしい & 厳しい & strict \\
   & すごい   &        & incredible; awesome \\
   & ちかい   & 近い   & close; near
\end{tabular}

\hlnotea{な終わるの形容詞 | な - adjectives} \\
\begin{tabular}{r l l l}
  *  & いろいろ(な) &      & various; different kinds of \\
     & しあわせ(な) & 幸せ & happy (lasting happiness) \\
  *  & だめ(な)     &      & no good
\end{tabular}

\hlnotea{う終わるの動詞 | う - verbs} \\
\begin{tabular}{r l l l}
      & あむ             & 編む       & to knit (〜を) \\
      & かす             & 貸す       & to lend; to rent \\
      &                  &            & (\textit{person} に \textit{thing} を) \\
  *   & がんばる         & 頑張る     & to do one's best; to try hard \\
      & なく             & 泣く       & to cry \\
      & みがく           & 磨く       & to brush (teeth); to polish (〜を) \\
      & やくそくをまもる & 約束を守る & to keep a promise
\end{tabular}

\hlnotea{Irregular Verb} \\
\begin{tabular}{r l l l}
      & かんどうする & 感動する & to be moved/touched (by...) \\
      &              &          & (〜に)
\end{tabular}

\section{〜し}

\begin{grammar}[〜し]
\label{grammar:_shi}
  We can use 〜し to conjugate reasons. Te form for using 〜し is:
  \begin{equation*}
    \text{ short form } + \text{ し }
  \end{equation*}
\end{grammar}

\begin{ex}
  Write the present affirmative and present negative tenses for the following, using 〜し and ending. After that, write down the past tense form of the statements.

  \begin{tabular}{l l l}
                   & present affirmative & present negative \\
    おもしろいです & おいしいし          & おいしくないし \\
    きれいです     & きれいだし          & きれいじゃないし \\
    がくせいです   & がくせいだし        & がくせいじゃないし \\
    かんどうする   & かんどうだし        & かんどうじゃないし \\
    たべたいです   & たべたいし          & たべたくないし \\
    すんでいます   & すんでいるし        & すんでいないし \\
    いけます       & いけるし            & いけないし \\
                   & past affirmative    & past negative \\
    おもしろいです & おもしろかったし    & おもしろじゃなかったし \\
    きれいです     & きれいだったし      & きれいじゃなかったし \\
    がくせいです   & がくせいだったし    & がくせいじゃなかったし \\
    かんどうする   & かんどうだったし    & かんどうじゃなかったし \\
    たべたいです   & たべたかったし      & たべたくなかったし \\
    すんでいます   & すんでいたし        & すんでいなかったし \\
    いけます       & いけたし            & いけなかったし
  \end{tabular}
\end{ex}

\begin{ex}[げんき II - 39 ページ、パートA]
\hlnotea{物価が高いし、人がたくさんいるし}

  Answer the questions using 〜し〜し. Examine the ideas in the cues and decide whether you want to answer in the affirmative or negative form.

  Example:
  \begin{align*}
    Q &: \text{日本に住みたいですか。} \\
    A &: \text{(物価が高いです。人がたくさんいます。} \\
      &\to \text{物価が高いし、人がたくさんいるし、住みたくないです。}
  \end{align*}

  \begin{enumerate}
    \item 今週は忙しですか。\\
      (試験があります。宿題がたくさんあります。)\\
      \textbf{答え:} ええ、忙しです、試験があるし、宿題もたくさんあるし。
    \item 新しいアルバイトはいいですか。\\
      (会社に近いです。静かです。)\\
      \textbf{答え:} 会社に近いし、静かだし、新しいアルバイトはいいですよ。
    \item 経済の授業をとりますか。\\
      (先生は厳しいです。長いレポートを書かなきゃいけませ。)\\
      \textbf{答え:} 経済の授業を取りたくないです、先生は厳しいし、長いレポートを書かなきゃいけないし。
    \item 旅行は楽しかったですか。\\
      (たべものはおいしくなかったです。言葉がわかりませんでした。)\\
      \textbf{答え:} 楽しくなかったです、食べ物が美味しくなかったし、言葉がわかりませんだし。
    \item 今晩、パーテイーにいきますか。\\
      (かぜをひいています。昨日もパーティーに行きました。)\\
      \textbf{答え:} 今晩のパーティーに行きません、かぜをひいているし、昨日もパーティーにいっただし。
    \item 日本語の新聞が読めますか。\\
      (漢字が読めません。文法がわかりません。)\\
      \textbf{答え:} 読めないです、漢字が読めないし、文法もわからないし。
    \item 一人で旅行ができますか。\\
      (日本語が話せます。もう大人です。)\\
      \textbf{答え:} できますよ、日本語が話せるし、もう大人だし。
    \item 田中さんが好きですか。\\
      (うそをつきます。約束を守りません。)\\
      \textbf{答え:} あんまり好きじゃないです、うそをつくし、約束を守れないし。
  \end{enumerate}
\end{ex}

\section{〜そうです}%
\label{sec:_soudesu}
% section _soudesu

\begin{grammar}[〜そうです]
\label{grammar:_soudesu}
  「〜そうです」has the ``looks like'' meaning in English. For example,
  \begin{center}
    美味し\hlnoteb{そうです}。 \\
    \hlnoteb{Looks} delicious.
  \end{center}
  The negative form of 〜そうです is 〜なさそうです.

  Another usage of 〜そうです is as follows:
  \begin{center}
    (形容詞) $+$ そう $+$ \hlnoteb{な} $+$ (名詞)
  \end{center}
  For example,
  \begin{center}
    美味しそう\hlnoteb{な}寿司です。
  \end{center}
\end{grammar}

\begin{ex}
  げんき II、42ページ、(III) C の宿題を練習してください。
\end{ex}

% section _soudesu (end)

\section{漢字}%
\label{sec:kanji}
% section kanji

漢字の歴史と書き方を簡単的に紹介しました。

\begin{note}[Brief History and Information]
  \begin{itemize}
    \item Kanji are Chinese characters.
    \item Kanji were introduced to Japan 1500 years ago, when Japan has yet to have its own writing system.
    \item Both \textit{hiragana} and \textit{katakana} are evolutions of simplified Chinese characters later on.
    \item Kanji represents both meanings and sounds.
    \item (Just as in Chinese,) most Kanji have multiple readings, which can be divided into 2 types:
      \begin{itemize}
        \item \textit{On-yomi} (音読み)(Chinese readings)
          \begin{itemize}
            \item derived from pronunciations in China
            \item some Kanji has more than one \textit{on-yomi} due to temporal and regional variacnes in Chinese pronunciation
          \end{itemize}
        \item \textit{kun-yomi} (訓読み)(Japanese readings)
      \end{itemize}
  \end{itemize}
\end{note}

\begin{note}[Forms of Kanji]
  There are roughly 4 types of Kanji based on their formation:
  \begin{itemize}
    \item \hlnotea{Pictograms} - Kanji created from pictures (e.g. 山)
    \item \hlnotea{Simple ideograms} - Kanji made from dots and lines to represent numbers of abstract concepts (e.g. 三、上)
    \item \hlnotea{Compund ideograms} - Kanji made from two or more kanji characters (e.g. 曜)
    \item \hlnotea{Phonetic-ideographic characters} - Kanji made of two parts: a meaning element and a sound element
  \end{itemize}
\end{note}

% section kanji (end)

\chapter{Lecture 3 五月九日(水曜日)}%
\label{chp:lecture_3_wu_yue_jiu_ri_shui_yao_ri}
% chapter lecture_3_wu_yue_jiu_ri_shui_yao_ri

\section{〜てみます}%
\label{sec:_temimasu}
% section _temimasu

\begin{grammar}[〜てみます]
\label{grammar:_temimasu}
  「〜てみます」has the meaning of ``shall try''. For example,
  \begin{center}
    食べ\hlnotea{てみます}。\\
    (I) \hlnotea{shall try} to eat.
  \end{center}
\end{grammar}

\begin{ex}[げんき II - 43ページ、IV パート A]
  Respond to the following sentences using 〜てみる.

  Example:
  \begin{align*}
    A &: \text{この服はすてきですよ。} \\
    B &: \text{じゃあ、着てみます。}
  \end{align*}

  \begin{enumerate}
    \item 経済の授業はおもしろかったですよ。 \\
      \textbf{答え:} じゃあ、取ってみます。
    \item あの映画を見て泣きました。 \\
      \textbf{答え:} じゃあ、見てみます。
    \item このほんはかんどうしました。 \\
      \textbf{答え:} じゃあ、読んでみます。
    \item このケーキはおいしいですよ。 \\
      \textbf{答え:} じゃあ、食べてみます。
    \item 東京はおもしろかったですよ。 \\
      \textbf{答え:} じゃあ、行ってみます。
    \item このCDはよかったですよ。 \\
      \textbf{答え:} じゃあ、聞いてみます。
    \item この辞書は便利でしたよ。 \\
      \textbf{答え:} じゃあ、使ってみます。/ じゃあ、買ってみます。
  \end{enumerate}
\end{ex}

\begin{ex}[げんき II - 43-44 ページ、IV パート B]
  You are at a shopping center. Ask store attendants whether you can try out the following, using appropriate verbs.

  Example:
  \begin{align*}
    \text{Customer} &: \text{すみません。使ってもいいですか。} \\
    \text{Store attendant} &: \text{どうぞ、使ってみてください。}
  \end{align*}

  \begin{enumerate}
    \item (服)\\
      お客様:すみません。ワンピースを着てみてもいいですか。\\
      店員:どうぞ、着てみてください。
    \item (椅子)\\
      お客様:すみません。椅子に座ってもいいですか。\\
      店員:どうぞ、座ってください。
    \item (ギター)\\
      お客様:すみません。ギターを弾いてみてもいいですか。\\
      店員:どうぞ、弾いてみてください。
    \item (自転車)\\
      お客様:すみません。自転車に乗ってみてもいいですか。\\
      店員:どうぞ、乗ってみてください。
    \item (靴)\\
      お客様:すみません。靴を履いてみてもいいですか。\\
      店員:どうぞ、履いてみてください。
  \end{enumerate}
\end{ex}

\begin{ex}[げんき II - 44ページ、IV パート C]
  Talk about what you want to try in the following places.

  Example: インド (India)
  \begin{align*}
    A &: \text{インドに行ったことがありますか。} \\
    B &: \text{いいえ。ありません。でも、行ってみたいです。} \\
    A &: \text{そうですか。インドで何がしたいですか。} \\
    B &: \text{インドで象を見たり、ヨガ(Yoga)を習ったりしてみたいです。}
  \end{align*}

  \begin{enumerate}
    \item ケンア (Kenya)
      \begin{align*}
        A &: \text{ケニヤに行ったことがありますか。} \\
        B &: \text{いいえ。ありません。でも、行ってみたいです。} \\
        A &: \text{そうですか。ケニャで何がしたいですか。} \\
        B &: \text{ケニヤでサバンナの\ruby[j]{日暮}{ひ|ぐ}れと\ruby[j]{野生動物}{や|せい|どう|ぶつ}を見たり、お\ruby[j]{寺}{てら}を行ったりしてみたいです。}
      \end{align*}
    \item 東京
      \begin{align*}
        A &: \text{東京に行ったことがありますか。} \\
        B &: \text{はい、去年五月に行きました。} \\
        A &: \text{そうですか。何がしたんですか。} \\
        B &: \text{ラーメンとつけ\ruby[j]{麺}{めん}を食べたり、\ruby[j]{周}{まわ}りの観光地に行ったりしていた。}
      \end{align*}
    \item タイ (Thailand)
      \begin{align*}
        A &: \text{タイに行ったことがありますか。} \\
        B &: \text{いいえ、行ったことないです。でも、行ってみたいです。} \\
        A &: \text{そうですか。タイで何がしたいですか。} \\
        B &: \text{タイでメコン川に遊んたり、トムヤムを食べたりしてみたいです。}
      \end{align*}
    \item ブラジル (Brazil)
      \begin{align*}
        A &: \text{ブラジルに行ったことがありますか。} \\
        B &: \text{いいえ、ないです。でも、行ってみたいです。} \\
        A &: \text{そうですか。ブラジルで何がしたいですか。} \\
        B &: \text{ブラジルで\ruby[j]{滝}{たき}を見たり、山を\ruby[j]{乗}{の}ったりしてみたいです。}
      \end{align*}
    \item チベット (Tibet)
      \begin{align*}
        A &: \text{チベットに行ったことがありますか。} \\
        B &: \text{いいえ、ないです。でも、行ってみたいです。} \\
        A &: \text{そうですか。チベットで何がしたいですか。} \\
        B &: \text{チベットでお寺を見たり、\ruby[j]{自然}{しぜん}を近づくたりしてみたいです。}
      \end{align*}
    \item Your own
      \begin{align*}
        A &: \text{マレシアに行ったことがありますか。} \\
        B &: \text{はい、ありますよ。} \\
        A &: \text{そうですか。マレシアで何がしたんですか。} \\
        B &: \text{マレシアで美味しい食べ物を食べたり、海で泳いでたりしていました。}
      \end{align*}
  \end{enumerate}
\end{ex}

% section _temimasu (end)

% chapter lecture_3_wu_yue_jiu_ri_shui_yao_ri (end)

\chapter{Lecture 4 五月十四日(月曜日)}%
\label{chp:lecture_4_wu_yue_shi_si_ri_yue_yao_ri}
% chapter lecture_4_wu_yue_shi_si_ri_yue_yao_ri

\section{〜なら}%
\label{sec:_nara}
% section _nara

\begin{grammar}[なら]
\label{grammar:nara}
  A statement of the form "noun A なら predicate X" says that the predicate X \textit{applies only to} A and is not more generally valid. The main ideas of a なら sentence, in other words, are contrast (as in Situation 1) and limitation (as in Situation 2).
\end{grammar}

\begin{eg}[Situation 1]\label{eg:nara_situation_1}
\noindent    Q : ブラジルに行ったことがありますか。 \\
\noindent \quad\enspace Have you ever been to Brazil? \\
\noindent    A : チリ\underline{なら}行ったことがありますが、ブラジルはいったことがありません。\sidenote{
  \begin{note}
    Optionally, we can keep the particle に before なら in this example. Particles such as に, で, and から may, but do not have to, intervene between the noun and なら, while は, が, and を never go with なら.
  \end{note}
  } \\
\noindent \quad\enspace I've been to Chile, but never to Brazil. 
\end{eg}

\begin{eg}[Situation 2]\label{eg:nara_situation_2}
\noindent Q : 日本語がわかりますか。\\
\noindent \quad\enspace Do you understand Japanese? \\
\noindent A : ひたがな\underline{なら}わかります。\\
\noindent \quad\enspace If it is (written) in hiragana, yes.
\end{eg}

\begin{note}
  \marginnote{The examples illustrate a good way to keep the conversation ball rolling by adding related information to a question that can simply be answered with a no. Q can then use the ``positive'' information received from A to continue the conversation, instead of just ending the conversation or have a difficult time continuing it.}
  ならintroduces a sentence that says something ``positive'' about the item that is contrasted. In the first situation above, なら puts Chile in a positive light, and in constrast with Brazil, which the question was originally about. In the second situation, a smaller part, namely \textit{hiragana}, is brought up and contrasted with a larger area, namely, the language as  whole. 
\end{note}

\begin{ex}[げんき II - 44 ページ、V パート A]
  Answer the questions as in the example.

  \textbf{例:} \\
  \noindent Q: メアリーさんはけさ、コーヒーを飲みますか。\\
  \noindent A: ($\circ$ tea $\times$ coffee) \\
  \noindent \quad\enspace $\to$ \ruby[j]{紅茶}{こう|ちゃ}なら飲みましたが、コーヒーは\sidenote{
    \begin{note}
      It is entirely okay for us to use は here instead of を as we would normally do, since we are trying to make a comparison between コーヒー and 紅茶.
    \end{note}
  }飲みませんでした。

  \begin{enumerate}
    \item メアリーさんはバイクに乗れますか。($\circ$ bicycle $\times$ motorbike) \\
      \textbf{答え:}自転車なら乗れますが、バイクには乗れません。 
    \item メアリーさんはニュージーランドに行ったことありますか。($\circ$ Australia $\times$ New Zealand) \\
      \textbf{答え:}オーストラリアなら行ったことありますが、ニュージーランドはまだ行ったことありません。 
    \item メアリーさんはゴルフをしますか。($\circ$ tennis $\times$ golf) \\
      \textbf{答え:}テニスならしますが、ゴルフはしたことないです。
    \item けんさんは日本の\ruby[j]{経済}{けい|ざい}に\ruby[j]{興味}{きょう|み}がありますか。($\circ$ history $\times$ economics) \\
      \textbf{答え:}歴史なら興味がありますが、経済はあまり興味がありません。 
    \item けんさんは\ruby[j]{彼女}{かのじょ}がいますか。($\circ$ friend $\times$ girlfriend) \\
      \textbf{答え:}友達ならいますが、彼女はいません。 
    \item けんさんは土曜日に出かけられますか。($\circ$ Sunday $\times$ Saturday) \\
      \textbf{答え:}日曜日なら出かけられますが、土曜日はちょっといけません。 
  \end{enumerate}
\end{ex}

\begin{ex}[げんき II - 45 ページ、V パート B]
  Answer the following questions. Use 〜なら whenever possible.

  \textbf{例:}\\
  \noindent Q: スポーツをよく見ますか。\\
  \noindent A: ええ、\ruby[j]{野球}{や|きゅう}なら見ます。/いいえ、見ません。

  \begin{enumerate}
    \item 外国語ができますか。\\
      \textbf{答え:}はい、中国語、英語、マレー語、広東語と台湾語ならできます。 
    \item アルバイトをしたことがありますか。\\
      \textbf{答え:}はい、\ruby[j]{官公庁}{かん|こう|ちょう} にデータ\ruby[j]{分析}{ぶん|せき}の仕事がしたことがあります。
    \item 日本の料理が作れますか。\\
      \textbf{答え:}いいえ、 日本の料理が好きですが、作れません。
    \item 有名人に会ったことがありますか。\\
      \textbf{答え:}いいえ、そういう\ruby[j]{機会}{き|かい}がないです。 
    \item \ruby[j]{楽器}{がっ|き}できますか。\\
      \textbf{答え:}いいえ。でも、ギータなら前に\ruby[j]{学}{まな}びました。 
    \item お金を\ruby[j]{貸}{か}せますか。\\
      \textbf{答え:}お金を貸せませんが、何のお\ruby[j]{手伝}{て|つだ}いをほしいですか。 
  \end{enumerate}
\end{ex}

% section _nara (end)

\section{一週間に三回}%
\label{sec:yi_zhou_jian_nisan_hui_}
% section yi_zhou_jian_nisan_hui_

\begin{grammar}[(period) に (frequency)]
\label{grammar:_period_ni_frequency_}
  We can describe the frequency of events over a period of time using the following framework:
  \begin{center}
    (period) に (frequency) \qquad (frequency) per (period)
  \end{center}
\end{grammar}

\begin{eg}
\noindent 私は\underline{一週間に三回}髪を\ruby[j]{洗}{あら}います。\\
\noindent 私は\underline{一ヶ月に一回}家族に電話をかけます。\\
\noindent 父は\underline{一年に二回}旅行します。
\end{eg}

\begin{ex}[げんき II - 45 ページ、VI パートA]
  Express the following activities and their frequencies as in the given example.

  \textbf{例:} \\
  eat twice a day \\
  一日に二回食べます。

  \begin{enumerate}
    \item brush (teeth) twice a day \\
      $\to$ 一日に三回\ruby[j]{歯}{は}を\ruby[j]{磨}{み}きます。
    \item sleep seven hours a day \\
      $\to$ 一日に七時寝ます。
    \item study three hours a day \\
      $\to$ 一日に三時勉強します。
    \item do house chores once a week \\
      $\to$ 一週間に一回\ruby[j]{家事}{か|じ}をします。
    \item do the laundry twice a week
      $\to$ 一週間に二回洗濯します。
    \item do part-time job three days a week \\
      $\to$ 一週間に三日アルバイトをします。
    \item go to school five days a week \\
      $\to$ 一週間に五日学校に行きます。
    \item go to the movies once a month \\
      $\to$ 一ヶ月に一回\ruby[j]{映画館}{えい|が|かん}に行って、映画を見ます。
  \end{enumerate}
\end{ex}

\begin{ex}[げんき II - 46 ページ、 VI パート B]
  From the last exercise, create questions above the given activities using the following method:
  \begin{equation*}
    \begin{lrcases} 一日 \\ 一週間 \\ 一ヶ月 \end{lrcases} に \begin{lrcases} 何回 \\ 何時間 \\ 何日 \end{lrcases} 〜ますか
  \end{equation*}

  \textbf{例:} \\
\noindent A: Bさんは一日に何回食べますか。 \\
\noindent B: そうですね。たいてい一日に二回食べます。朝食は食べません。
\end{ex}

% section yi_zhou_jian_nisan_hui_ (end)

% chapter lecture_4_wu_yue_shi_si_ri_yue_yao_ri (end)

\chapter{Lecture 5 五月十六日(水曜日)}%
\label{chp:lecture_5_wu_yue_shi_liu_ri_shui_yao_ri_}
% chapter lecture_5_wu_yue_shi_liu_ri_shui_yao_ri_

\begin{ex}[げんき II - 47 ページ、VII パート A]
  Answer the following questions.

  \begin{enumerate}
    \item 子供の時に何ができましたが。何ができませんでしたか。
    \item 百円で何が買えますか。
    \item どこに行ってみたいですか。どうしてですか。
    \item 子供の時、何がしてみたかったですか。
    \item 今、何がしてみたいですか。
    \item 一日に何時間ぐらい勉強しますか。
    \item 一週間に何回レストランに行きますか。
    \item 一ヶ月にいくらぐらい使いますか。
  \end{enumerate}
\end{ex}

% chapter lecture_5_wu_yue_shi_liu_ri_shui_yao_ri_ (end)

\chapter{Lecture 6 五月二十二日(火曜日)}%
\label{chp:lecture_6_wu_yue_er_shi_er_ri_huo_yao_ri_}
% chapter lecture_6_wu_yue_er_shi_er_ri_huo_yao_ri_

\section{ほしい}
\label{sec:hoshii}
% section hoshii

\begin{grammar}[ほしい]
\label{grammar:hoshii}
  ほしい means ``(I) want (something)''. It is an \hlnotea{い - adjective} and conjugates as such. The object of desire is usually follwoed by the particle が. \hlnoteb{In negative sentences}, the particle は is also used.

  いい漢字の辞書が\hlnotec{ほしいです}。\\
  I want a good kanji dictionary. \\
  子供の時、ゴジラのおもちゃが\hlnotec{\ruby[j]{欲}{ほ}しかったです}。\\
  When I was young, I wanted a toy Godzilla. \\
  お金はあまり\hlnotec{欲しくないです}。\\
  I don't have much desire for money.

  \begin{center}
    (私は) Xが ほしい \qquad I want X.
  \end{center}

  \noindent ほしい is similar to たい (I want to do...), in that its use is primarily limited to the first person, the speaker. These words are called ``\hlnoteb{private predicates},'' and they refer to the inner sensations which are known only to the person feeling them. Everyone else needs to rely on observations and guesses when they want to claim that ``person X wnats such and such.'' Japanese grammar, ever demanding that everything be stated in explicit terms, therefore calls for an extra device for sentences with private predicates as applied to the second or third person.\sidenote{Among the words we have learned so far, 悲しい (sad), いれしい (glad), and 痛い (painful) are private predicates. The observations we make about ほしい below apply to these words as well.}

  \noindent You can quote the people who say that they are feeling these sensations.

  ロバートさんはパソコンが\hlnotec{ほしい}\hlnotea{と言っています}。 \\
  Robert says he wants a computer.

  \noindent You can make clear that you are only making a guess.

  京子さんはクラシックの\ruby[j]{CD}{シーディー}が\hlnotec{ほしくない}\hlnotea{でしょう}。\\
  It is likely that Kyoko does not want a CD of classical music.

  \noindent Or you can use the special construction which says that you are making an observation of a person feeling a private-predicate sensation. In an earlier lesson (in JAPAN102R), we were introduced to the verb \hlnotea{たがる}. which replaces たい.

  智子さんは英語を習いたがっています。\\
  (I understand that) Tomoko wants to study English.

  \noindent ほしい too has a special verb counterpart, ほしがる. It conjugates as an \textit{u}-verb and is usually used in the form 欲しがっている, to describe an observation that the speaker currently thinks holds true. Unlike ほしい, the particle after the object of desire is を.

  トムさんは友達を\hlnotec{欲しがっています}。\\
  (I understand that) Tom wants a friend.
\end{grammar}

\begin{eg}
  \begin{enumerate}
    \item みずさ先生はミニファミコンがほしいと言っていました。
    \item みずさ先生はミニファミコンがほしいと思います。
  \end{enumerate}
\end{eg}

\begin{ex}[げんき II - 61 ページ、I パート A]
  Items marked with $\circ$ are what you want, and items marked with $\times$ are what you do not want. Make sentences using ほしい.

  \begin{enumerate}
    \item $\circ$ money \\
      $\to$ お金がほしいです。
    \item $\times$ sweater \\
      $\to$ セーターが欲しくないです。
    \item $\times$ personal computer \\
      $\to$ パソコンは欲しくないです。
    \item $\circ$ motorbike \\
      $\to$ バイクがほしいです。
    \item $\times$ plush toy \\
      $\to$ ぬいぐるみは欲しくないです。
  \end{enumerate}
\end{ex}

\begin{ex}[げんき II - 61 ページ、I パート B]
  Items marked with $\circ$ are what you wanted when you were a child, and items marked with $\times$ are what you did not want. Make sentences using ほしい.

  \begin{enumerate}
    \item $\circ$ プレーステーション \\
      $\to$ 子供の時、プレーステーションが欲しかったです。
    \item $\times$ 指輪 \\
      $\to$ 子供の時、指輪はほしくなかったです。
    \item $\times$ \ruby[j]{腕}{うで}時計 \\
      $\to$ 子供の時、腕時計は欲しくないです。
    \item $\circ$ おもちゃ \\
      $\to$ 子供の時、おもちゃが欲しかったです。
    \item $\times$ \ruby[j]{花束}{はな|たば} \\
      $\to$ 子供の時、花束がほしくなかったです。
  \end{enumerate}
\end{ex}

% section hoshii (end)

\section{〜かもしれません}%
\label{sec:_kamoshiremasen}
% section _kamoshiremasen

\begin{grammar}[〜かもしれません]
\label{grammar:_kamoshiremasen}
  The usage of かもしれません is similar to that of でしょう that was introduced in Lesson 12 (JAPAN102R), which we use for stating possibilities.

  \noindent かもしれません is placed after the short forms of predicates, in the affirmative and in the negative, in the present as well as the past tense.

  あしたは雨が降るかもしれません。\\
  It may rain tomorrow. \\
  田中さんより、鈴木さんのほうが\ruby[j]{背}{せ}が高いかもしれません。\\
  Suzuki is perhaps taller than Tanaka. \\
  あしたは天気が良くないかもしれません。\\
  The weather may not be good tomorrow. \\
  トムさんは、子供の時、いじわるだったかもしれません。\\
  Tom may have been a bully when he was a kid.

  \noindent Just like でしょう, かもしれません goes directly after a nount or a な - adjective in the present tense affirmative sentences. In other words, だ is dropped in these sentences.

  トムさんはカナダ人だ。\\
  $\to$ トムさんはカナダ人かもしれません。 \\
  Tom is a Canadian. \\
  $\to$ Tom might be a Canadian. \\
  山下先生は犬が嫌いだ。 \\
  $\to$ 山下先生は犬が嫌いかもしれません。\\
  Professor Yamashita is not fond of dogs. \\
  $\to$ It is possible that Professor yamashita is not fond of dogs.

  \textbf{Prssent tense, affirmative} \\
  \begin{equation*}
    \begin{rcases}
      \text{verbs:} &\qquad 行く \\
      \text{い - adjectives:} &\qquad 寒い \\
      \text{な - adjectives:} &\qquad 元気 \\
      \text{noun + です} &\qquad 学生
    \end{rcases}\enspace かもしれません
  \end{equation*}
\end{grammar}

\begin{ex}[げんき II - 63 ページ、II パート A]
  Use the hints to change the following sentences using 〜かもしれません.

  \begin{enumerate}
    \item 女の人は会社員です。(maybe) \\
      $\to$ 女の人は会社員かもしれません。
    \item 男の人は先生です。(maybe not) \\
      $\to$ 男の人は先生じゃないかもしれません。
    \item 女の人はテニスが上手です。(maybe) \\
      $\to$ 女の人はテニスが上手かもしれません。
    \item 男の人は背が\ruby[j]{低}{ひく}いです。(maybe not) \\
      $\to$ 男の人は背が低くないかもしれません。
    \item 今、寒いです。(maybe not) \\
      $\to$ 今、寒くないかもしれません。
    \item 女の人は今日テニスをします。(maybe) \\
      $\to$ 女の人は今日テニスをするかもしれません。
    \item 男の人と女の人は、今、駅にいます。(maybe not) \\
      $\to$ 男の人と女の人は、今、駅にいないかもしれません。
    \item 男の人は結婚しています。(maybe) \\
      $\to$ 男の人は結婚しているかもしれません。
    \item 男の人と女の人は夫婦です。(maybe not) \\
      $\to$ 男の人と女の人は夫婦じゃないかもしれません。
    \item 女の人は男の人に興味があります。(maybe) \\
      $\to$ 女の人は男の人に興味があるかもしれません。
    \item 女の人は昨日テニスをしました。(maybe)
      $\to$ 女の人は昨日テニスをしたかもしれません。
  \end{enumerate}
\end{ex}

% section _kamoshiremasen (end)

% chapter lecture_6_wu_yue_er_shi_er_ri_huo_yao_ri_ (end)

\chapter*{Appendix}%
\label{chp:appendix}
% chapter appendix

\hlnotec{日の読み方}

\begin{fullwidth}
\begin{tabular}{c c c c c c c}
日曜日                          & 月曜日                        & 火曜日                          & 水曜日                            & 木曜日                        & 金曜日                          & 土曜日 \\
                                & 1                             & 2                               & 3                                 & 4                             & 5                               & 6 \\
                                & ついたち                      & ふつか                          & みっか                            & よっか                        & いつか                          & むいか \\
7                               & 8                             & 9                               & 10                                & 11                            & 12                              & 13 \\
なのか                          & ようか                        & ここのか                        & とうか                            & \scriptsize{ じゅういちにち } & \scriptsize{ じゅうににち }     & \scriptsize{ じゅうさんにち } \\
14                              & 15                            & 16                              & 17                                & 18                            & 19                              & 20 \\
\scriptsize{ じゅうよっか }     & \scriptsize{ じゅうごにち }   & \scriptsize{ じゅうろくにち }   & \scriptsize{ じゅうしちにち }     & \scriptsize{ じゅうはちにち } & \scriptsize{ じゅうくにち }     & はつか \\
21                              & 22                            & 23                              & 24                                & 25                            & 26                              & 27 \\
\scriptsize{ にじゅういちにち } & \scriptsize{ にじゅうににち } & \scriptsize{ にじゅうさんいち } & \scriptsize{ にじゅうよっか }     & \scriptsize{ にじゅうごにち } & \scriptsize{ にじゅうろくにち } & \scriptsize{ にじゅうしちにち } \\
28                              & 29                            & 30                              & 31 \\
\scriptsize{ にじゅうはちにち } & \scriptsize{ にじゅうくにち } & \scriptsize{ さんじゅうにち }   & \scriptsize{ さんじゅういちにち }
\end{tabular}
\end{fullwidth}

% chapter appendix (end)

\nobibliography*
\bibliography{bibliography}

\printindex
\end{document}
