% !TEX TS-program = lualatex
\documentclass[notoc,notitlepage]{tufte-book}
\usepackage{luatexja}
\usepackage{pxrubrica}
\nonstopmode
\setcounter{secnumdepth}{3}
\setcounter{tocdepth}{5}

\renewcommand{\baselinestretch}{1.1}

\usepackage{CJKutf8}
\setcounter{secnumdepth}{3}
\setcounter{tocdepth}{3}

\renewcommand{\baselinestretch}{1.2}
\usepackage{geometry}
\geometry{letterpaper}
\usepackage[parfill]{parskip}
\usepackage{graphicx}

% Essential Packages
\usepackage{makeidx}
\makeindex
\usepackage{enumitem}
\usepackage[T1]{fontenc}
\usepackage{natbib}
\bibliographystyle{apalike}
\usepackage{ragged2e}
\usepackage{etoolbox}
\usepackage{amssymb}
\usepackage{eso-pic}
\usepackage[fixed]{fontawesome5}
\usepackage{todonotes}
\usepackage{apptools, chngcntr}
\usepackage{amsmath}
\usepackage{mathrsfs}
\usepackage{stmaryrd}
\usepackage{mathtools}
\usepackage{tocloft}
\usepackage{tocbibind}
\usepackage{xparse}
\usepackage{tkz-euclide}
\usetkzobj{all}
\usepackage[utf8]{inputenc}
\usepackage{csquotes}
\usepackage[english]{babel}
\usepackage{marvosym}
\usepackage{pgf,tikz}
\usepackage{tikz-cd}
\usepackage{ifthen}
\usepackage{pgfplots}
\usepackage{fancyhdr}
\usepackage{array}
\usepackage{float}
\usepackage{xcolor}
\usepackage{soul}
\usepackage{centernot}
\usepackage{silence}
  \WarningFilter*{latex}{Marginpar on page \thepage\space moved}
\usepackage{tcolorbox}
\tcbuselibrary{skins,breakable}
\usepackage{longtable,booktabs}
\usepackage[amsmath,hyperref,thmmarks]{ntheorem}
\usepackage{thmtools}
\usepackage{hyperref}
\usepackage[noabbrev,capitalize,nameinlink]{cleveref}

\newcommand{\personalcolor}{false}
\ifthenelse{\equal{\personalcolor}{true}}{
  \usepackage{colorscheme-chaos}
}{
  \usepackage{colorscheme-student}
}

% hyperref Package Settings
\hypersetup{
    unicode=true,          % non-Latin characters in Acrobat’s bookmarks
    pdftoolbar=false,        % show Acrobat’s toolbar?
    pdfmenubar=false,        % show Acrobat’s menu?
    pdffitwindow=true,     % window fit to page when opened
    colorlinks=true,
    allcolors=magenta,
}

% tikz
\usepgfplotslibrary{polar}
\usepgflibrary{shapes.geometric}
\usetikzlibrary{angles,patterns,calc,decorations.markings,arrows.meta,tikzmark,bending}
\tikzset{midarrow/.style 2 args={
        decoration={markings,
            mark= at position #2 with {\arrow{#1}} ,
        },
        postaction={decorate}
    },
    midarrow/.default={latex}{0.5}
}
\def\centerarc[#1](#2)(#3:#4:#5)% Syntax: [draw options] (center) (initial angle:final angle:radius)
    { \draw[#1] ($(#2)+({#5*cos(#3)},{#5*sin(#3)})$) arc (#3:#4:#5); }
% from https://tex.stackexchange.com/questions/67573/tikz-shift-and-rotate-in-3d
\newcommand{\rotateRPY}[4][0/0/0]% point to be saved to \savedxyz, roll, pitch, yaw
{   \pgfmathsetmacro{\rollangle}{#2}
    \pgfmathsetmacro{\pitchangle}{#3}
    \pgfmathsetmacro{\yawangle}{#4}

    % to what vector is the x unit vector transformed, and which 2D vector is this?
    \pgfmathsetmacro{\newxx}{cos(\yawangle)*cos(\pitchangle)}% a
    \pgfmathsetmacro{\newxy}{sin(\yawangle)*cos(\pitchangle)}% d
    \pgfmathsetmacro{\newxz}{-sin(\pitchangle)}% g
    \path (\newxx,\newxy,\newxz);
    \pgfgetlastxy{\nxx}{\nxy};

    % to what vector is the y unit vector transformed, and which 2D vector is this?
    \pgfmathsetmacro{\newyx}{cos(\yawangle)*sin(\pitchangle)*sin(\rollangle)-sin(\yawangle)*cos(\rollangle)}% b
    \pgfmathsetmacro{\newyy}{sin(\yawangle)*sin(\pitchangle)*sin(\rollangle)+ cos(\yawangle)*cos(\rollangle)}% e
    \pgfmathsetmacro{\newyz}{cos(\pitchangle)*sin(\rollangle)}% h
    \path (\newyx,\newyy,\newyz);
    \pgfgetlastxy{\nyx}{\nyy};

    % to what vector is the z unit vector transformed, and which 2D vector is this?
    \pgfmathsetmacro{\newzx}{cos(\yawangle)*sin(\pitchangle)*cos(\rollangle)+ sin(\yawangle)*sin(\rollangle)}
    \pgfmathsetmacro{\newzy}{sin(\yawangle)*sin(\pitchangle)*cos(\rollangle)-cos(\yawangle)*sin(\rollangle)}
    \pgfmathsetmacro{\newzz}{cos(\pitchangle)*cos(\rollangle)}
    \path (\newzx,\newzy,\newzz);
    \pgfgetlastxy{\nzx}{\nzy};

    % transform the point given by #1
    \foreach \x/\y/\z in {#1}
    {   \pgfmathsetmacro{\transformedx}{\x*\newxx+\y*\newyx+\z*\newzx}
        \pgfmathsetmacro{\transformedy}{\x*\newxy+\y*\newyy+\z*\newzy}
        \pgfmathsetmacro{\transformedz}{\x*\newxz+\y*\newyz+\z*\newzz}
        \xdef\savedx{\transformedx}
        \xdef\savedy{\transformedy}
        \xdef\savedz{\transformedz}     
    }
}
\tikzset{RPY/.style={x={(\nxx,\nxy)},y={(\nyx,\nyy)},z={(\nzx,\nzy)}}}
\newcommand{\AxisRotator}[1][rotate=0]{%
    \tikz [x=0.25cm,y=0.60cm,line width=.2ex,-stealth,#1] \draw (0,0) arc (-150:150:1 and 1);%
  }

% enumitems
\newlist{inlinelist}{enumerate*}{1}
\setlist*[inlinelist,1]{%
  label=(\roman*),
}

% Theorem Style Customization
\setlength\theorempreskipamount{2ex}
\setlength\theorempostskipamount{3ex}

\makeatletter
\let\nobreakitem\item
\let\@nobreakitem\@item
\patchcmd{\nobreakitem}{\@item}{\@nobreakitem}{}{}
\patchcmd{\nobreakitem}{\@item}{\@nobreakitem}{}{}
\patchcmd{\@nobreakitem}{\@itempenalty}{\@M}{}{}
\patchcmd{\@xthm}{\ignorespaces}{\nobreak\ignorespaces}{}{}
\patchcmd{\@ythm}{\ignorespaces}{\nobreak\ignorespaces}{}{}

\renewtheoremstyle{break}%
  {\item{\theorem@headerfont
          ##1\ ##2\theorem@separator}\hskip\labelsep\relax\nobreakitem}%
  {\item{\theorem@headerfont
          ##1\ ##2\ (##3)\theorem@separator}\hskip\labelsep\relax\nobreakitem}
\makeatother

% ntheorem Declarations
\theorempreskip{10pt}
\theorempostskip{5pt}
\theoremstyle{break}

\theoremsymbol{\faComment}
\newtheorem{remark}{Remark}[section]
\theoremsymbol{}
\newtheorem*{strategy}{\faPaperPlane Strategy}
\newtheorem*{procedure}{\faCodeBranch\ }
\newtheorem{ex}{Exercise}[section]
\theorembodyfont{\normalfont}
\newtheorem*{solution}{\faPencil* Solution}
\theoremsymbol{\faGavel}
\newtheorem{eg}{Example}[section]
\theoremsymbol{}
\theorembodyfont{\it}

    % definition env
\theoremprework{\textcolor{blue}{\hrule height 2pt width \textwidth}}
\theoremheaderfont{\color{blue}\normalfont\bfseries}
\theorempostwork{\textcolor{blue}{\hrule height 2pt width \textwidth}}
\theoremindent10pt
\newtheorem{defn}{\faBook Definition}

    % definition env no num
\theoremprework{\textcolor{blue}{\hrule height 2pt width \textwidth}}
\theoremheaderfont{\color{blue}\normalfont\bfseries}
\theorempostwork{\textcolor{blue}{\hrule height 2pt width \textwidth}}
\theoremindent10pt
\newtheorem*{defnnonum}{\faBook Definition}

\theoremprework{\textcolor{blue}{\hrule height 2pt width \marginparwidth}}
\theoremheaderfont{\color{blue}\normalfont\bfseries}
\theorempostwork{\textcolor{blue}{\hrule height 2pt width \marginparwidth}}
\theoremindent10pt
\newtheorem{margindefn}[defn]{\faBook Definition}

\theoremprework{\textcolor{blue}{\hrule height 2pt width \textwidth}}
\theoremheaderfont{\color{blue}\normalfont\bfseries}
\theorempostwork{\textcolor{blue}{\hrule height 2pt width \textwidth}}
\theoremindent10pt
\newtheorem*{margindefnnonum}{\faBook Definition}

    % theorem envs
\theoremprework{\textcolor{magenta}{\hrule height 2pt width \textwidth}}
\theoremheaderfont{\color{magenta}\normalfont\bfseries}
\theorempostwork{\textcolor{magenta}{\hrule height 2pt width \textwidth}}
\theoremindent10pt
\newtheorem{thm}{\faCoffee Theorem}

\theoremprework{\textcolor{magenta}{\hrule height 2pt width \textwidth}}
\theorempostwork{\textcolor{magenta}{\hrule height 2pt width \textwidth}}
\theoremindent10pt
\newtheorem{propo}[thm]{\faTint Proposition}

\theoremprework{\textcolor{magenta}{\hrule height 2pt width \textwidth}}
\theorempostwork{\textcolor{magenta}{\hrule height 2pt width \textwidth}}
\theoremindent10pt
\newtheorem{crly}[thm]{\faSpaceShuttle Corollary}

\theoremprework{\textcolor{magenta}{\hrule height 2pt width \textwidth}}
\theorempostwork{\textcolor{magenta}{\hrule height 2pt width \textwidth}}
\theoremindent10pt
\newtheorem{lemma}[thm]{\faTree Lemma}

\theoremprework{\textcolor{magenta}{\hrule height 2pt width \textwidth}}
\theorempostwork{\textcolor{magenta}{\hrule height 2pt width \textwidth}}
\theoremindent10pt
\newtheorem{axiom}[thm]{\faShield Axiom}

    % theorem envs without counter
\theoremprework{\textcolor{magenta}{\hrule height 2pt width \textwidth}}
\theoremheaderfont{\color{magenta}\normalfont\bfseries}
\theorempostwork{\textcolor{magenta}{\hrule height 2pt width \textwidth}}
\theoremindent10pt
\newtheorem*{thmnonum}{\faCoffee Theorem}

\theoremprework{\textcolor{magenta}{\hrule height 2pt width \textwidth}}
\theorempostwork{\textcolor{magenta}{\hrule height 2pt width \textwidth}}
\theoremindent10pt
\newtheorem*{propononum}{\faTint Proposition}

\theoremprework{\textcolor{magenta}{\hrule height 2pt width \textwidth}}
\theorempostwork{\textcolor{magenta}{\hrule height 2pt width \textwidth}}
\theoremindent10pt
\newtheorem*{crlynonum}{\faSpaceShuttle Corollary}

\theoremprework{\textcolor{magenta}{\hrule height 2pt width \textwidth}}
\theorempostwork{\textcolor{magenta}{\hrule height 2pt width \textwidth}}
\theoremindent10pt
\newtheorem*{lemmanonum}{\faTree Lemma}

\theoremprework{\textcolor{magenta}{\hrule height 2pt width \textwidth}}
\theorempostwork{\textcolor{magenta}{\hrule height 2pt width \textwidth}}
\theoremindent10pt
\newtheorem*{axiomnonum}{\faShield Axiom}

    % envs on margins
\theoremprework{\textcolor{magenta}{\hrule height 2pt width \marginparwidth}}
\theoremheaderfont{\color{magenta}\normalfont\bfseries}
\theorempostwork{\textcolor{magenta}{\hrule height 2pt width \marginparwidth}}
\theoremindent10pt
\newtheorem{marginthm}[thm]{\faCoffee Theorem}

\theoremprework{\textcolor{magenta}{\hrule height 2pt width \marginparwidth}}
\theorempostwork{\textcolor{magenta}{\hrule height 2pt width \marginparwidth}}
\theoremindent10pt
\newtheorem{marginpropo}[thm]{\faTint Proposition}

\theoremprework{\textcolor{magenta}{\hrule height 2pt width \marginparwidth}}
\theorempostwork{\textcolor{magenta}{\hrule height 2pt width \marginparwidth}}
\theoremindent10pt
\newtheorem{margincrly}[thm]{\faSpaceShuttle Corollary}

\theoremprework{\textcolor{magenta}{\hrule height 2pt width \marginparwidth}}
\theorempostwork{\textcolor{magenta}{\hrule height 2pt width \marginparwidth}}
\theoremindent10pt
\newtheorem{marginlemma}[thm]{\faTree Lemma}

\theoremprework{\textcolor{magenta}{\hrule height 2pt width \marginparwidth}}
\theorempostwork{\textcolor{magenta}{\hrule height 2pt width \marginparwidth}}
\theoremindent10pt
\newtheorem{marginaxiom}[thm]{\faShield Axiom}

    % envs on margins without counter
\theoremprework{\textcolor{magenta}{\hrule height 2pt width \marginparwidth}}
\theoremheaderfont{\color{magenta}\normalfont\bfseries}
\theorempostwork{\textcolor{magenta}{\hrule height 2pt width \marginparwidth}}
\theoremindent10pt
\newtheorem*{marginthmnonum}{\faCoffee Theorem}

\theoremprework{\textcolor{magenta}{\hrule height 2pt width \marginparwidth}}
\theorempostwork{\textcolor{magenta}{\hrule height 2pt width \marginparwidth}}
\theoremindent10pt
\newtheorem*{marginpropononum}{\faTint Proposition}

\theoremprework{\textcolor{magenta}{\hrule height 2pt width \marginparwidth}}
\theorempostwork{\textcolor{magenta}{\hrule height 2pt width \marginparwidth}}
\theoremindent10pt
\newtheorem*{margincrlynonum}{\faSpaceShuttle Corollary}

\theoremprework{\textcolor{magenta}{\hrule height 2pt width \marginparwidth}}
\theorempostwork{\textcolor{magenta}{\hrule height 2pt width \marginparwidth}}
\theoremindent10pt
\newtheorem*{marginlemmanonum}{\faTree Lemma}

\theoremprework{\textcolor{magenta}{\hrule height 2pt width \marginparwidth}}
\theorempostwork{\textcolor{magenta}{\hrule height 2pt width \marginparwidth}}
\theoremindent10pt
\newtheorem*{marginaxiomnonum}{\faShield Axiom}

    % proof env
\theoremprework{\textcolor{green}{\hrule height 2pt width \textwidth}}
\theorembodyfont{\normalfont}
\theoremheaderfont{\color{green}\normalfont\bfseries}
\theorempostwork{\textcolor{green}{\hrule height 2pt width \textwidth}}
\theoremsymbol{\ensuremath{_\square}}
\newtheorem*{proof}{\faPencil* Proof}
\theoremsymbol{}

\theoremprework{\textcolor{green}{\hrule height 2pt width \marginparwidth}}
\theorembodyfont{\normalfont}
\theoremheaderfont{\color{green}\normalfont\bfseries}
\theorempostwork{\textcolor{green}{\hrule height 2pt width \marginparwidth}}
\theoremsymbol{\ensuremath{_\square}}
\newtheorem*{mproof}{\faPencil* Proof}
\theoremsymbol{}

    % note and notation env
\theorembodyfont{\it}

\theoremprework{\textcolor{yellow}{\hrule height 2pt width \textwidth}}
\theoremheaderfont{\color{yellow}\normalfont\bfseries}
\theorempostwork{\textcolor{yellow}{\hrule height 2pt width \textwidth}}
\newtheorem{note}{\faQuoteLeft Note}[section]

\theoremprework{\textcolor{yellow}{\hrule height 2pt width \marginparwidth}}
\theoremheaderfont{\color{yellow}\normalfont\bfseries}
\theorempostwork{\textcolor{yellow}{\hrule height 2pt width \marginparwidth}}
\newtheorem{mnote}[note]{\faQuoteLeft Note}

\theoremprework{\textcolor{yellow}{\hrule height 2pt width \textwidth}}
\theorempostwork{\textcolor{yellow}{\hrule height 2pt width \textwidth}}
\newtheorem*{notation}{\faPaw Notation}

    % warning env
\theoremprework{\textcolor{red}{\hrule height 2pt width \textwidth}}
\theoremheaderfont{\color{red}\normalfont\bfseries}
\theorempostwork{\textcolor{red}{\hrule height 2pt width \textwidth}}
\theoremindent10pt
\newtheorem*{warning}{\faBug Warning}

\theoremprework{\textcolor{red}{\hrule height 2pt width \marginparwidth}}
\theoremheaderfont{\color{red}\normalfont\bfseries}
\theorempostwork{\textcolor{red}{\hrule height 2pt width \marginparwidth}}
\theoremindent10pt
\newtheorem*{marginwarning}{\faBug Warning}

% rule for appendices
\AtAppendix{\counterwithin{defn}{chapter}}
\AtAppendix{\counterwithin{thm}{chapter}}
\AtAppendix{\counterwithin{propo}{chapter}}
\AtAppendix{\counterwithin{lemma}{chapter}}
\AtAppendix{\counterwithin{crly}{chapter}}
\AtAppendix{\counterwithin{axiom}{chapter}}

% more environments
\newtcolorbox{quotebox}[2]{
  blanker,enhanced,breakable,standard jigsaw,
  opacityback=0,
  coltext=\ifblank{#2}{black}{#2},
  left=5mm,right=5mm,top=2mm,bottom=2mm,
  colframe=\ifblank{#1}{bblack}{#1},
  boxrule=0pt,leftrule=3pt,
  fontupper=\itshape
}

\providecommand{\parthook}{}
\patchcmd{\part}{\thispagestyle}{\parthook\thispagestyle}{}{}
\newcommand{\partimage}[2][]{% \parthook[<options>]{<image>}
  \renewcommand{\parthook}{% Update \parthook
    \AddToShipoutPictureBG*{% Add picture to background of THIS page only
      \AtPageLowerLeft{\includegraphics[width=\paperwidth,height=\paperheight,#1]{#2}}}% Insert image
    \renewcommand{\parthook}{}}}% Restore \parthook

\AtBeginDocument{\renewcommand\contentsname{\slshape Table of Contents\normalfont}}
\cftpagenumbersoff{part}

\newcommand{\tuftepart}[1]{\newgeometry{}\part{#1}\restoregeometry}

% Heading formattings
% chapter format
\titleformat{\chapter}%
  {\huge\rmfamily\itshape\color{magenta}}% format applied to label+text
  {\llap{\colorbox{magenta}{\parbox[c][1cm]{3cm}{\hfill\itshape\Huge\textcolor{background}{\thechapter}}}}}% label
  {5pt}% horizontal separation between label and title body
  {\faLeaf}% before the title body
  []% after the title body

% section format
\titleformat{\section}%
  {\normalfont\Large\rmfamily\itshape\color{blue}}% format applied to label+text
  {\llap{\colorbox{blue}{\parbox{3cm}{\hfill\itshape\textcolor{background}{\thesection}}}}}% label
  {5pt}% horizontal separation between label and title body
  {}% before the title body
  []% after the title body

% subsection format
\titleformat{\subsection}%
  {\normalfont\large\itshape\color{green}}% format applied to label+text
  {\llap{\colorbox{green}{\parbox{3cm}{\hfill\textcolor{background}{\thesubsection}}}}}% label
  {1em}% horizontal separation between label and title body
  {}% before the title body
  []% after the title body

% subsubsection format
\titleclass{\subsubsection}{straight}
\titleformat{\subsubsection}%
  {\normalfont\large\itshape\color{yellow}}% format applied to label+text
  {\llap{\colorbox{yellow}{\parbox{3cm}{\hfill\textcolor{background}{\thesubsubsection}}}}}% label
  {1em}% horizontal separation between label and title body
  {}% before the title body
  []% after the title body

% Sidenote enhancements
\def\mathmarginnote#1{%
  \tag*{\rlap{\hspace\marginparsep\smash{\parbox[t]{\marginparwidth}{%
  \footnotesize#1}}}}
}

% Custom table columning
\newcolumntype{L}[1]{>{\raggedright\let\newline\\\arraybackslash\hspace{0pt}}m{#1}}
\newcolumntype{C}[1]{>{\centering\let\newline\\\arraybackslash\hspace{0pt}}m{#1}}
\newcolumntype{R}[1]{>{\raggedleft\let\newline\\\arraybackslash\hspace{0pt}}m{#1}}

% Graph styles
\pgfplotsset{compat=1.15}
\usepgfplotslibrary{fillbetween}
\pgfplotsset{four quads/.append style={axis x line=middle, axis y line=
middle, xlabel={$x$}, ylabel={$y$}, axis equal }}
\pgfplotsset{four quad complex/.append style={axis x line=middle, axis y line=
middle, xlabel={$\re$}, ylabel={$\im$}, axis equal }}
\def\axisdefaultwidth{360pt}
\pgfplotsset{
  tufteaxis/.append style = {thick},tick style = {thick,black},
  %
  % #1 = x, y, or z
  % #2 = the shift value
  /tikz/normal shift/.code 2 args = {%
    \pgftransformshift{%
        \pgfpointscale{#2}{\pgfplotspointouternormalvectorofticklabelaxis{#1}}%
    }%
  },%
  %
  range3frame/.style = {
    tick align        = outside,
    scaled ticks      = false,
    enlargelimits     = false,
    ticklabel shift   = {10pt},
    axis lines*       = left,
    line cap          = round,
    clip              = false,
    xtick style       = {normal shift={x}{10pt}},
    ytick style       = {normal shift={y}{10pt}},
    ztick style       = {normal shift={z}{10pt}},
    x axis line style = {normal shift={x}{10pt}},
    y axis line style = {normal shift={y}{10pt}},
    z axis line style = {normal shift={z}{10pt}},
  }
}

% Shortcuts
\DeclareMathOperator{\id}{id}
\DeclareMathOperator{\Img}{Img}
\DeclareMathOperator{\Res}{Res}
\DeclareMathOperator*{\argmax}{arg\,max}
\DeclareMathOperator*{\argmin}{arg\,min}
\DeclareMathOperator{\re}{Re}
\DeclareMathOperator{\im}{Im}
\DeclareMathOperator{\caparg}{Arg}
\DeclareMathOperator{\Char}{Char}
\DeclareMathOperator{\sgn}{sgn}
\DeclareMathOperator{\Range}{range}

\newcommand{\floor}[1]{\lfloor #1 \rfloor}      % simplifying the writing of a floor function
\newcommand{\ceiling}[1]{\lceil #1 \rceil}      % simplifying the writing of a ceiling function
\newcommand{\dotp}{\, \cdotp}			        % dot product to distinguish from \cdot
\newcommand{\abs}[1]{\left|#1\right|}						% absolute value
\newcommand{\lra}[1]{\left\langle \; #1 \; \right\rangle}
\newcommand{\at}[2]{\Big|_{#1}^{#2}}
\newcommand{\Arg}[1]{\caparg #1}
\renewcommand{\bar}[1]{\mkern 1.5mu \overline{\mkern -1.5mu #1 \mkern -1.5mu} \mkern 1.5mu}
\newcommand{\faktor}[2]{{\raisebox{.2em}{$#1$}\left/\raisebox{-.2em}{$#2$}\right.}}
\newcommand{\quotient}[2]{\faktor{#1}{#2}}
\newcommand{\cyclic}[1]{\left\langle #1 \right\rangle}
\newcommand{\ind}[2]{\Ind_{#2}\left( #1 \right)}
\newcommand{\notimply}{\centernot\implies}
\newcommand{\res}[2]{\underset{#2}{\Res} #1 }
\newcommand{\tworow}[3]{\begin{tabular}{@{}#1@{}} #2 \\ #3 \end{tabular}}
\renewcommand{\epsilon}{\varepsilon}
\renewcommand{\phi}{\varphi}
\newcommand{\lrarrow}{\leftrightarrow}
\newcommand{\larrow}{\leftarrow}
\newcommand{\rarrow}{\rightarrow}
\renewcommand{\atop}[2]{\genfrac{}{}{0pt}{}{#1}{#2}}
\newcommand*\dif{\mathop{}\!d}
\newcommand{\mmid}{\; \middle| \;}
\newcommand{\coprime}{\; \bot \;}
\newcommand{\norm}[1]{\left\| #1 \right\|}
\newenvironment{spmatrix}
  {\left(\begin{smallmatrix}}
  {\end{smallmatrix}\right)}

  % inspiration from: https://tex.stackexchange.com/questions/8720/overbrace-underbrace-but-with-an-arrow-instead#37758
\newcommand{\overarrow}[2]{
  \overset{\makebox[0pt]{\begin{tabular}{@{}c@{}}#2\\[0pt]\ensuremath{\uparrow}\end{tabular}}}{\ensuremath{#1}}
}
\newcommand{\underarrow}[2]{
  \underset{\makebox[0pt]{\begin{tabular}{@{}c@{}}\downarrow\\[0pt]\ensuremath{#2}\end{tabular}}}{\ensuremath{#1}}
}


	% highlighting shortcuts
\newcommand{\hlimpo}[1]{\textcolor{red}{\textbf{#1}}}
\newcommand{\hlwarn}[1]{\textcolor{yellow}{\textbf{#1}}}
\newcommand{\hldefn}[1]{\textcolor{blue}{\index{#1}\textbf{#1}}}
\newcommand{\hlnotea}[1]{\textcolor{green}{\textbf{#1}}}
\newcommand{\hlnoteb}[1]{\textcolor{cyan}{\textbf{#1}}}
\newcommand{\hlb}[2]{\colorbox{#1!30!background}{#2}}
\newcommand{\hlbnotea}[1]{\hlb{green}{#1}}
\newcommand{\hlbnoteb}[1]{\hlb{cyan}{#1}}
\newcommand{\hlbnotec}[1]{\hlb{yellow}{#1}}
\newcommand{\hlbnoted}[1]{\hlb{magenta}{#1}}
\newcommand{\hlbnotee}[1]{\hlb{red}{#1}}
\newcommand{\WTP}{\textcolor{bwhite}{WTP} }
\newcommand{\WTS}{\textcolor{bwhite}{WTS} }

  % stars on important stuff
\newcommand{\imponote}{\faStar}
\newcommand{\vimponote}{\faStar\faStar}
\newcommand{\vvimponote}{\faStar\faStar\faStar}

% Document header formatting
\makeatletter
\pagestyle{fancy}
\fancyhead{}
\fancyhead[RO]{\textsl{\@title} \enspace \thepage}
\fancyhead[LE]{\thepage \enspace \textsl{\leftmark \enspace \rightmark}}
\makeatother
\renewcommand{\chaptermark}[1]{\markboth{#1}{}}
\renewcommand{\sectionmark}[1]{\markright{#1}}

% Comment the two lines below if you want to print the document
\pagecolor{background}
\color{foreground}


\newenvironment{lrcases}
  {\left\lbrace\quad\begin{aligned}}
  {\end{aligned}\quad\right\rbrace}
\renewcommand\rubysize{.6}

\theoremprework{\textcolor{base16-eighties-blue}{\hrule height 2pt}}
\theoremheaderfont{\color{base16-eighties-blue}\normalfont\bfseries}
\theorempostwork{\textcolor{base16-eighties-blue}{\hrule height 2pt}}
\theoremindent10pt
\newtheorem{grammar}{文法}[section]

\title{JAPAN201RS18}
\author{Johnson Ng}

% Header formatting
\renewcommand{\chaptermark}[1]{\markboth{#1}{}}
\renewcommand{\sectionmark}[1]{\markright{#1}}
\makeatletter
\pagestyle{fancy}
\fancyhead{}
\fancyhead[RO]{\textsl{\@title} \enspace \thepage}
\fancyhead[LE]{\thepage \enspace \textsl{\leftmark \enspace - \enspace \rightmark}}
\makeatother

\begin{document}

\hypersetup{pageanchor=false}
\maketitle
\hypersetup{pageanchor=true}
\tableofcontents

\chapter{Lecture 1 五月三日(水曜日)}
  \label{chapter:lecture_1_may_03rd_2018}

\section{Potential Forms} % (fold)
\label{sec:potential_forms}

We have the following rule for changing verbs to their potential forms:
\begin{itemize}
  \item る-verb $\to$ られる 
  \item う-verb $\to$ 〜える 
  \item Irregular verbs: くる $\to$ こられる $\quad$ する $\to$ できる
\end{itemize}

\begin{eg}
\hlnotea{る-verbs}
  \begin{itemize}
    \item 食べる $\to$ 食べられる $\to$ 食べられます
    \item 見る $\to$ 見られる $\to$ 見られます
  \end{itemize}
  
\noindent\hlnotea{う-verbs}
  \begin{itemize}
    \item 行く $\to$ 行ける $\to$ 行けます
    \item 飲む $\to$ 飲める $\to$ 飲められます
    \item 話す $\to$ 話せる $\to$ 話せます
  \end{itemize}
\end{eg}

\begin{ex}[げんきII - 37ページ、パートA]
Convert the base forms to their potential forms: \marginnote{We can drop the ら in 〜られる. This was originally colloquial, but has now become the norm that it is acceptable to do this even in written text.} \\
  \begin{tabular}{l l}
    1. はなす $\to$ はなせる   & 2. する $\to$ できる \\
    3. いく $\to$ いける       & 4. ねる $\to$ ねられる \\
    5. くる $\to$ こられる     & 6. みる $\to$ みられる \\
    7. やめる $\to$ やめられる & 8. かりる $\to$ かりられる \\
    9. のむ $\to$ のめる       & 10. まつ $\to$ まてる \\
    11. およぐ $\to$ およげる  & 12. はたらく $\to$ はたらける \\
    13. あむ $\to$ あめる
  \end{tabular}
\end{ex}

\begin{ex}[げんきII - 37ページ、パートB]
  Describe the things that Mary can do.
  \begin{enumerate}
    \item メアリーさんは日本語曲がう歌えます。
    \item メアリーさんはヴァイオリンが弾けます。
    \item メアリーさんは空手道ができます。
    \item メアリーさんは寿司が食べられます。
    \item メアリーさんは料理ができます。
    \item メアリーさんは日本語で電話がかけられます。
    \item メアリーさんは車を運転ができます。
    \item メアリーさんはセーターが編めます。
    \item メアリーさんは日本語で手紙がかけます。
    \item メアリーさんは朝早くで起きられる。
    \item メアリーさんは温かいお風呂に入られる。
  \end{enumerate}
\end{ex}

% section potential_forms (end)

% chapter lecture_1_may_03rd_2018 (end)

\chapter{Lecture 2 五月七日(月曜日)}
  \label{chapter:lecture_2_may_07th_2018}

\section{Vocabulary 1}
\label{sect:vocabulary_1}

\hlnotea{名詞 Nouns}
\marginnote{* are words that appear in dialogues.}

\begin{tabular}{r l l l}
  *  & ウェイター &        & waiter \\
     & おたく     & お宅   & (someone's) house/home \\
     & おとな     & 大人   & adult \\
     & がいこくご & 外国語 & foreign language \\
     & がっき     & 楽曲   & musical instruments \\
     & からて     & 空手   & karate \\
  *  & カレー     &        & curry \\
     & きもの     & 着物   & kimono; Japanese traditional dress \\
  *  & こうこく   & 広告   & advertisement \\
     & こうちゃ   & 紅茶   & black tea \\
     & ことば     & 言葉   & language \\
     & ゴルフ     &        & golf \\
     & セーター   &        & sweater \\
     & ぞう       & 象     & elephant \\
     & バイオリン &        & violin \\
     & バイク     &        & motorcycle \\
     & ぶっか     & 物価   & (consumer) prices \\
     & ぶんぽう   & 文法   & grammar \\
     & べんごし   & 弁護士 & lawyer \\
  *  & ぼしゅう   & 募集   & recruitment \\
  *  & みせ       & 店     & shop; store \\
     & やくざ     &        & \textit{yakuza}; gangster \\
     & やくそく   & 約束   & promise; appointment \\
     & レポート   &        & (term) paper; report \\
  *  & わたくし   & 私     & I (formal)
\end{tabular}

\hlnotea{い終わるの形容詞 | い - adjectives} \\
\begin{tabular}{r l l l}
   & うれしい & 嬉しい & glad \\
   & かなしい & 悲しい & sad \\
   & からい   & 辛い   & hot and spicy; salty \\
   & きびしい & 厳しい & strict \\
   & すごい   &        & incredible; awesome \\
   & ちかい   & 近い   & close; near
\end{tabular}

\hlnotea{な終わるの形容詞 | な - adjectives} \\
\begin{tabular}{r l l l}
  *  & いろいろ(な) &      & various; different kinds of \\
     & しあわせ(な) & 幸せ & happy (lasting happiness) \\
  *  & だめ(な)     &      & no good
\end{tabular}

\hlnotea{う終わるの動詞 | う - verbs} \\
\begin{tabular}{r l l l}
      & あむ             & 編む       & to knit (〜を) \\
      & かす             & 貸す       & to lend; to rent \\
      &                  &            & (\textit{person} に \textit{thing} を) \\
  *   & がんばる         & 頑張る     & to do one's best; to try hard \\
      & なく             & 泣く       & to cry \\
      & みがく           & 磨く       & to brush (teeth); to polish (〜を) \\
      & やくそくをまもる & 約束を守る & to keep a promise
\end{tabular}

\hlnotea{Irregular Verb} \\
\begin{tabular}{r l l l}
      & かんどうする & 感動する & to be moved/touched (by...) \\
      &              &          & (〜に)
\end{tabular}

\section{〜し}

\begin{grammar}[〜し]
\label{grammar:_shi}
  We can use 〜し to conjugate reasons. Te form for using 〜し is:
  \begin{equation*}
    \text{ short form } + \text{ し }
  \end{equation*}
\end{grammar}

\begin{ex}
  Write the present affirmative and present negative tenses for the following, using 〜し and ending. After that, write down the past tense form of the statements.

  \begin{tabular}{l l l}
                   & present affirmative & present negative \\
    おもしろいです & おいしいし          & おいしくないし \\
    きれいです     & きれいだし          & きれいじゃないし \\
    がくせいです   & がくせいだし        & がくせいじゃないし \\
    かんどうする   & かんどうだし        & かんどうじゃないし \\
    たべたいです   & たべたいし          & たべたくないし \\
    すんでいます   & すんでいるし        & すんでいないし \\
    いけます       & いけるし            & いけないし \\
                   & past affirmative    & past negative \\
    おもしろいです & おもしろかったし    & おもしろじゃなかったし \\
    きれいです     & きれいだったし      & きれいじゃなかったし \\
    がくせいです   & がくせいだったし    & がくせいじゃなかったし \\
    かんどうする   & かんどうだったし    & かんどうじゃなかったし \\
    たべたいです   & たべたかったし      & たべたくなかったし \\
    すんでいます   & すんでいたし        & すんでいなかったし \\
    いけます       & いけたし            & いけなかったし
  \end{tabular}
\end{ex}

\begin{ex}[げんき II - 39 ページ、パートA]
\hlnotea{物価が高いし、人がたくさんいるし}

  Answer the questions using 〜し〜し. Examine the ideas in the cues and decide whether you want to answer in the affirmative or negative form.

  Example:
  \begin{align*}
    Q &: \text{日本に住みたいですか。} \\
    A &: \text{(物価が高いです。人がたくさんいます。} \\
      &\to \text{物価が高いし、人がたくさんいるし、住みたくないです。}
  \end{align*}

  \begin{enumerate}
    \item 今週は忙しですか。\\
      (試験があります。宿題がたくさんあります。)\\
      \textbf{答え:} ええ、忙しです、試験があるし、宿題もたくさんあるし。
    \item 新しいアルバイトはいいですか。\\
      (会社に近いです。静かです。)\\
      \textbf{答え:} 会社に近いし、静かだし、新しいアルバイトはいいですよ。
    \item 経済の授業をとりますか。\\
      (先生は厳しいです。長いレポートを書かなきゃいけませ。)\\
      \textbf{答え:} 経済の授業を取りたくないです、先生は厳しいし、長いレポートを書かなきゃいけないし。
    \item 旅行は楽しかったですか。\\
      (たべものはおいしくなかったです。言葉がわかりませんでした。)\\
      \textbf{答え:} 楽しくなかったです、食べ物が美味しくなかったし、言葉がわかりませんだし。
    \item 今晩、パーテイーにいきますか。\\
      (かぜをひいています。昨日もパーティーに行きました。)\\
      \textbf{答え:} 今晩のパーティーに行きません、かぜをひいているし、昨日もパーティーにいっただし。
    \item 日本語の新聞が読めますか。\\
      (漢字が読めません。文法がわかりません。)\\
      \textbf{答え:} 読めないです、漢字が読めないし、文法もわからないし。
    \item 一人で旅行ができますか。\\
      (日本語が話せます。もう大人です。)\\
      \textbf{答え:} できますよ、日本語が話せるし、もう大人だし。
    \item 田中さんが好きですか。\\
      (うそをつきます。約束を守りません。)\\
      \textbf{答え:} あんまり好きじゃないです、うそをつくし、約束を守れないし。
  \end{enumerate}
\end{ex}

\section{〜そうです}%
\label{sec:_soudesu}
% section _soudesu

\begin{grammar}[〜そうです]
\label{grammar:_soudesu}
  「〜そうです」has the ``looks like'' meaning in English. For example,
  \begin{center}
    美味し\hlnoteb{そうです}。 \\
    \hlnoteb{Looks} delicious.
  \end{center}
  The negative form of 〜そうです is 〜なさそうです.

  Another usage of 〜そうです is as follows:
  \begin{center}
    (形容詞) $+$ そう $+$ \hlnoteb{な} $+$ (名詞)
  \end{center}
  For example,
  \begin{center}
    美味しそう\hlnoteb{な}寿司です。
  \end{center}
\end{grammar}

\begin{ex}
  げんき II、42ページ、(III) C の宿題を練習してください。
\end{ex}

% section _soudesu (end)

\section{漢字}%
\label{sec:kanji}
% section kanji

漢字の歴史と書き方を簡単的に紹介しました。

\begin{note}[Brief History and Information]
  \begin{itemize}
    \item Kanji are Chinese characters.
    \item Kanji were introduced to Japan 1500 years ago, when Japan has yet to have its own writing system.
    \item Both \textit{hiragana} and \textit{katakana} are evolutions of simplified Chinese characters later on.
    \item Kanji represents both meanings and sounds.
    \item (Just as in Chinese,) most Kanji have multiple readings, which can be divided into 2 types:
      \begin{itemize}
        \item \textit{On-yomi} (音読み)(Chinese readings)
          \begin{itemize}
            \item derived from pronunciations in China
            \item some Kanji has more than one \textit{on-yomi} due to temporal and regional variacnes in Chinese pronunciation
          \end{itemize}
        \item \textit{kun-yomi} (訓読み)(Japanese readings)
      \end{itemize}
  \end{itemize}
\end{note}

\begin{note}[Forms of Kanji]
  There are roughly 4 types of Kanji based on their formation:
  \begin{itemize}
    \item \hlnotea{Pictograms} - Kanji created from pictures (e.g. 山)
    \item \hlnotea{Simple ideograms} - Kanji made from dots and lines to represent numbers of abstract concepts (e.g. 三、上)
    \item \hlnotea{Compund ideograms} - Kanji made from two or more kanji characters (e.g. 曜)
    \item \hlnotea{Phonetic-ideographic characters} - Kanji made of two parts: a meaning element and a sound element
  \end{itemize}
\end{note}

% section kanji (end)

\chapter{Lecture 3 五月九日(水曜日)}%
\label{chp:lecture_3_wu_yue_jiu_ri_shui_yao_ri}
% chapter lecture_3_wu_yue_jiu_ri_shui_yao_ri

\section{〜てみます}%
\label{sec:_temimasu}
% section _temimasu

\begin{grammar}[〜てみます]
\label{grammar:_temimasu}
  「〜てみます」has the meaning of ``shall try''. For example,
  \begin{center}
    食べ\hlnotea{てみます}。\\
    (I) \hlnotea{shall try} to eat.
  \end{center}
\end{grammar}

\begin{ex}[げんき II - 43ページ、IV パート A]
  Respond to the following sentences using 〜てみる.

  Example:
  \begin{align*}
    A &: \text{この服はすてきですよ。} \\
    B &: \text{じゃあ、着てみます。}
  \end{align*}

  \begin{enumerate}
    \item 経済の授業はおもしろかったですよ。 \\
      \textbf{答え:} じゃあ、取ってみます。
    \item あの映画を見て泣きました。 \\
      \textbf{答え:} じゃあ、見てみます。
    \item このほんはかんどうしました。 \\
      \textbf{答え:} じゃあ、読んでみます。
    \item このケーキはおいしいですよ。 \\
      \textbf{答え:} じゃあ、食べてみます。
    \item 東京はおもしろかったですよ。 \\
      \textbf{答え:} じゃあ、行ってみます。
    \item このCDはよかったですよ。 \\
      \textbf{答え:} じゃあ、聞いてみます。
    \item この辞書は便利でしたよ。 \\
      \textbf{答え:} じゃあ、使ってみます。/ じゃあ、買ってみます。
  \end{enumerate}
\end{ex}

\begin{ex}[げんき II - 43-44 ページ、IV パート B]
  You are at a shopping center. Ask store attendants whether you can try out the following, using appropriate verbs.

  Example:
  \begin{align*}
    \text{Customer} &: \text{すみません。使ってもいいですか。} \\
    \text{Store attendant} &: \text{どうぞ、使ってみてください。}
  \end{align*}

  \begin{enumerate}
    \item (服)\\
      お客様:すみません。ワンピースを着てみてもいいですか。\\
      店員:どうぞ、着てみてください。
    \item (椅子)\\
      お客様:すみません。椅子に座ってもいいですか。\\
      店員:どうぞ、座ってください。
    \item (ギター)\\
      お客様:すみません。ギターを弾いてみてもいいですか。\\
      店員:どうぞ、弾いてみてください。
    \item (自転車)\\
      お客様:すみません。自転車に乗ってみてもいいですか。\\
      店員:どうぞ、乗ってみてください。
    \item (靴)\\
      お客様:すみません。靴を履いてみてもいいですか。\\
      店員:どうぞ、履いてみてください。
  \end{enumerate}
\end{ex}

\begin{ex}[げんき II - 44ページ、IV パート C]
  Talk about what you want to try in the following places.

  Example: インド (India)
  \begin{align*}
    A &: \text{インドに行ったことがありますか。} \\
    B &: \text{いいえ。ありません。でも、行ってみたいです。} \\
    A &: \text{そうですか。インドで何がしたいですか。} \\
    B &: \text{インドで象を見たり、ヨガ(Yoga)を習ったりしてみたいです。}
  \end{align*}

  \begin{enumerate}
    \item ケンア (Kenya)
      \begin{align*}
        A &: \text{ケニヤに行ったことがありますか。} \\
        B &: \text{いいえ。ありません。でも、行ってみたいです。} \\
        A &: \text{そうですか。ケニャで何がしたいですか。} \\
        B &: \text{ケニヤでサバンナの\ruby[j]{日暮}{ひ|ぐ}れと\ruby[j]{野生動物}{や|せい|どう|ぶつ}を見たり、お\ruby[j]{寺}{てら}を行ったりしてみたいです。}
      \end{align*}
    \item 東京
      \begin{align*}
        A &: \text{東京に行ったことがありますか。} \\
        B &: \text{はい、去年五月に行きました。} \\
        A &: \text{そうですか。何がしたんですか。} \\
        B &: \text{ラーメンとつけ\ruby[j]{麺}{めん}を食べたり、\ruby[j]{周}{まわ}りの観光地に行ったりしていた。}
      \end{align*}
    \item タイ (Thailand)
      \begin{align*}
        A &: \text{タイに行ったことがありますか。} \\
        B &: \text{いいえ、行ったことないです。でも、行ってみたいです。} \\
        A &: \text{そうですか。タイで何がしたいですか。} \\
        B &: \text{タイでメコン川に遊んたり、トムヤムを食べたりしてみたいです。}
      \end{align*}
    \item ブラジル (Brazil)
      \begin{align*}
        A &: \text{ブラジルに行ったことがありますか。} \\
        B &: \text{いいえ、ないです。でも、行ってみたいです。} \\
        A &: \text{そうですか。ブラジルで何がしたいですか。} \\
        B &: \text{ブラジルで\ruby[j]{滝}{たき}を見たり、山を\ruby[j]{乗}{の}ったりしてみたいです。}
      \end{align*}
    \item チベット (Tibet)
      \begin{align*}
        A &: \text{チベットに行ったことがありますか。} \\
        B &: \text{いいえ、ないです。でも、行ってみたいです。} \\
        A &: \text{そうですか。チベットで何がしたいですか。} \\
        B &: \text{チベットでお寺を見たり、\ruby[j]{自然}{しぜん}を近づくたりしてみたいです。}
      \end{align*}
    \item Your own
      \begin{align*}
        A &: \text{マレシアに行ったことがありますか。} \\
        B &: \text{はい、ありますよ。} \\
        A &: \text{そうですか。マレシアで何がしたんですか。} \\
        B &: \text{マレシアで美味しい食べ物を食べたり、海で泳いでたりしていました。}
      \end{align*}
  \end{enumerate}
\end{ex}

% section _temimasu (end)

% chapter lecture_3_wu_yue_jiu_ri_shui_yao_ri (end)

\chapter{Lecture 4 五月十四日(月曜日)}%
\label{chp:lecture_4_wu_yue_shi_si_ri_yue_yao_ri}
% chapter lecture_4_wu_yue_shi_si_ri_yue_yao_ri

\section{〜なら}%
\label{sec:_nara}
% section _nara

\begin{grammar}[なら]
\label{grammar:nara}
  A statement of the form "noun A なら predicate X" says that the predicate X \textit{applies only to} A and is not more generally valid. The main ideas of a なら sentence, in other words, are contrast (as in \cref{eg:nara_situation_1}) and limitation (as in \cref{eg:nara_situation_2}).
\end{grammar}

\begin{eg}[Situation 1]\label{eg:nara_situation_1}
\noindent    Q : ブラジルに行ったことがありますか。 \\
\noindent \quad\enspace Have you ever been to Brazil? \\
\noindent    A : チリ\underline{なら}行ったことがありますが、ブラジルはいったことがありません。\sidenote{
  \begin{note}
    Optionally, we can keep the particle に before なら in this example. Particles such as に, で, and から may, but do not have to, intervene between the noun and なら, while は, が, and を never go with なら.
  \end{note}
  } \\
\noindent \quad\enspace I've been to Chile, but never to Brazil. 
\end{eg}

\begin{eg}[Situation 2]\label{eg:nara_situation_2}
\noindent Q : 日本語がわかりますか。\\
\noindent \quad\enspace Do you understand Japanese? \\
\noindent A : ひたがな\underline{なら}わかります。\\
\noindent \quad\enspace If it is (written) in hiragana, yes.
\end{eg}

\begin{note}
  \marginnote{The examples illustrate a good way to keep the conversation ball rolling by adding related information to a question that can simply be answered with a no. Q can then use the ``positive'' information received from A to continue the conversation, instead of just ending the conversation or have a difficult time continuing it.}
  ならintroduces a sentence that says something ``positive'' about the item that is contrasted. In the first situation above, なら puts Chile in a positive light, and in constrast with Brazil, which the question was originally about. In the second situation, a smaller part, namely \textit{hiragana}, is brought up and contrasted with a larger area, namely, the language as  whole. 
\end{note}

\begin{ex}[げんき II - 44 ページ、V パート A]
  Answer the questions as in the example.

  \textbf{例:} \\
  \noindent Q: メアリーさんはけさ、コーヒーを飲みますか。\\
  \noindent A: ($\circ$ tea $\times$ coffee) \\
  \noindent \quad\enspace $\to$ \ruby[j]{紅茶}{こう|ちゃ}なら飲みましたが、コーヒーは\sidenote{
    \begin{note}
      It is entirely okay for us to use は here instead of を as we would normally do, since we are trying to make a comparison between コーヒー and 紅茶.
    \end{note}
  }飲みませんでした。

  \begin{enumerate}
    \item メアリーさんはバイクに乗れますか。($\circ$ bicycle $\times$ motorbike) \\
      \textbf{答え:}自転車なら乗れますが、バイクには乗れません。 
    \item メアリーさんはニュージーランドに行ったことありますか。($\circ$ Australia $\times$ New Zealand) \\
      \textbf{答え:}オーストラリアなら行ったことありますが、ニュージーランドはまだ行ったことありません。 
    \item メアリーさんはゴルフをしますか。($\circ$ tennis $\times$ golf) \\
      \textbf{答え:}テニスならしますが、ゴルフはしたことないです。
    \item けんさんは日本の\ruby[j]{経済}{けい|ざい}に\ruby[j]{興味}{きょう|み}がありますか。($\circ$ history $\times$ economics) \\
      \textbf{答え:}歴史なら興味がありますが、経済はあまり興味がありません。 
    \item けんさんは\ruby[j]{彼女}{かのじょ}がいますか。($\circ$ friend $\times$ girlfriend) \\
      \textbf{答え:}友達ならいますが、彼女はいません。 
    \item けんさんは土曜日に出かけられますか。($\circ$ Sunday $\times$ Saturday) \\
      \textbf{答え:}日曜日なら出かけられますが、土曜日はちょっといけません。 
  \end{enumerate}
\end{ex}

\begin{ex}[げんき II - 45 ページ、V パート B]
  Answer the following questions. Use 〜なら whenever possible.

  \textbf{例:}\\
  \noindent Q: スポーツをよく見ますか。\\
  \noindent A: ええ、\ruby[j]{野球}{や|きゅう}なら見ます。/いいえ、見ません。

  \begin{enumerate}
    \item 外国語ができますか。\\
      \textbf{答え:}はい、中国語、英語、マレー語、広東語と台湾語ならできます。 
    \item アルバイトをしたことがありますか。\\
      \textbf{答え:}はい、\ruby[j]{官公庁}{かん|こう|ちょう} にデータ\ruby[j]{分析}{ぶん|せき}の仕事がしたことがあります。
    \item 日本の料理が作れますか。\\
      \textbf{答え:}いいえ、 日本の料理が好きですが、作れません。
    \item 有名人に会ったことがありますか。\\
      \textbf{答え:}いいえ、そういう\ruby[j]{機会}{き|かい}がないです。 
    \item \ruby[j]{楽器}{がっ|き}できますか。\\
      \textbf{答え:}いいえ。でも、ギータなら前に\ruby[j]{学}{まな}びました。 
    \item お金を\ruby[j]{貸}{か}せますか。\\
      \textbf{答え:}お金を貸せませんが、何のお\ruby[j]{手伝}{て|つだ}いをほしいですか。 
  \end{enumerate}
\end{ex}

% section _nara (end)

\section{一週間に三回}%
\label{sec:yi_zhou_jian_nisan_hui_}
% section yi_zhou_jian_nisan_hui_

\begin{grammar}[(period) に (frequency)]
\label{grammar:_period_ni_frequency_}
  We can describe the frequency of events over a period of time using the following framework:
  \begin{center}
    (period) に (frequency) \qquad (frequency) per (period)
  \end{center}
\end{grammar}

\begin{eg}
\noindent 私は\underline{一週間に三回}髪を\ruby[j]{洗}{あら}います。\\
\noindent 私は\underline{一ヶ月に一回}家族に電話をかけます。\\
\noindent 父は\underline{一年に二回}旅行します。
\end{eg}

\begin{ex}[げんき II - 45 ページ、VI パートA]
  Express the following activities and their frequencies as in the given example.

  \textbf{例:} \\
  eat twice a day \\
  一日に二回食べます。

  \begin{enumerate}
    \item brush (teeth) twice a day \\
      $\to$ 一日に三回\ruby[j]{歯}{は}を\ruby[j]{磨}{み}きます。
    \item sleep seven hours a day \\
      $\to$ 一日に七時寝ます。
    \item study three hours a day \\
      $\to$ 一日に三時勉強します。
    \item do house chores once a week \\
      $\to$ 一週間に一回\ruby[j]{家事}{か|じ}をします。
    \item do the laundry twice a week
      $\to$ 一週間に二回洗濯します。
    \item do part-time job three days a week \\
      $\to$ 一週間に三日アルバイトをします。
    \item go to school five days a week \\
      $\to$ 一週間に五日学校に行きます。
    \item go to the movies once a month \\
      $\to$ 一ヶ月に一回\ruby[j]{映画館}{えい|が|かん}に行って、映画を見ます。
  \end{enumerate}
\end{ex}

\begin{ex}[げんき II - 46 ページ、 VI パート B]
  From the last exercise, create questions above the given activities using the following method:
  \begin{equation*}
    \begin{lrcases} 一日 \\ 一週間 \\ 一ヶ月 \end{lrcases} に \begin{lrcases} 何回 \\ 何時間 \\ 何日 \end{lrcases} 〜ますか
  \end{equation*}

  \textbf{例:} \\
\noindent A: Bさんは一日に何回食べますか。 \\
\noindent B: そうですね。たいてい一日に二回食べます。朝食は食べません。
\end{ex}

% section yi_zhou_jian_nisan_hui_ (end)

% chapter lecture_4_wu_yue_shi_si_ri_yue_yao_ri (end)

\chapter{Lecture 5 五月十六日(水曜日)}%
\label{chp:lecture_5_wu_yue_shi_liu_ri_shui_yao_ri_}
% chapter lecture_5_wu_yue_shi_liu_ri_shui_yao_ri_

\begin{ex}[げんき II - 47 ページ、VII パート A]
  Answer the following questions.

  \begin{enumerate}
    \item 子供の時に何ができましたが。何ができませんでしたか。
    \item 百円で何が買えますか。
    \item どこに行ってみたいですか。どうしてですか。
    \item 子供の時、何がしてみたかったですか。
    \item 今、何がしてみたいですか。
    \item 一日に何時間ぐらい勉強しますか。
    \item 一週間に何回レストランに行きますか。
    \item 一ヶ月にいくらぐらい使いますか。
  \end{enumerate}
\end{ex}

% chapter lecture_5_wu_yue_shi_liu_ri_shui_yao_ri_ (end)

\chapter*{Appendix}%
\label{chp:appendix}
% chapter appendix

\hlnotec{日の読み方}

\begin{fullwidth}
\begin{tabular}{c c c c c c c}
日曜日                          & 月曜日                        & 火曜日                          & 水曜日                            & 木曜日                        & 金曜日                          & 土曜日 \\
                                & 1                             & 2                               & 3                                 & 4                             & 5                               & 6 \\
                                & ついたち                      & ふつか                          & みっか                            & よっか                        & いつか                          & むいか \\
7                               & 8                             & 9                               & 10                                & 11                            & 12                              & 13 \\
なのか                          & ようか                        & ここのか                        & とうか                            & \scriptsize{ じゅういちにち } & \scriptsize{ じゅうににち }     & \scriptsize{ じゅうさんにち } \\
14                              & 15                            & 16                              & 17                                & 18                            & 19                              & 20 \\
\scriptsize{ じゅうよっか }     & \scriptsize{ じゅうごにち }   & \scriptsize{ じゅうろくにち }   & \scriptsize{ じゅうしちにち }     & \scriptsize{ じゅうはちにち } & \scriptsize{ じゅうくにち }     & はつか \\
21                              & 22                            & 23                              & 24                                & 25                            & 26                              & 27 \\
\scriptsize{ にじゅういちにち } & \scriptsize{ にじゅうににち } & \scriptsize{ にじゅうさんいち } & \scriptsize{ にじゅうよっか }     & \scriptsize{ にじゅうごにち } & \scriptsize{ にじゅうろくにち } & \scriptsize{ にじゅうしちにち } \\
28                              & 29                            & 30                              & 31 \\
\scriptsize{ にじゅうはちにち } & \scriptsize{ にじゅうくにち } & \scriptsize{ さんじゅうにち }   & \scriptsize{ さんじゅういちにち }
\end{tabular}
\end{fullwidth}

% chapter appendix (end)

\nobibliography*
\bibliography{bibliography}

\printindex
% \end{jp}
\end{document}
