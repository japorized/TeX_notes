\documentclass[notoc,notitlepage]{tufte-book}
% \nonstopmode % uncomment to enable nonstopmode

\usepackage{classnotetitle}

\title{PMATH351 - Real Analysis}
\author{Johnson Ng}
\subtitle{Classnotes for Fall 2018}
\credentials{BMath (Hons), Pure Mathematics major, Actuarial Science Minor}
\institution{University of Waterloo}

\setcounter{secnumdepth}{3}
\setcounter{tocdepth}{3}

\renewcommand{\baselinestretch}{1.1}
\usepackage{geometry}
\geometry{letterpaper}
\usepackage[parfill]{parskip}
\usepackage{graphicx}

% Essential Packages
\usepackage{makeidx}
\makeindex
\usepackage{enumitem}
\usepackage[T1]{fontenc}
\usepackage{natbib}
\bibliographystyle{apalike}
\usepackage{ragged2e}
\usepackage{etoolbox}
\usepackage{amssymb}
\usepackage{fontawesome}
\usepackage{amsmath}
\usepackage{mathrsfs}
\usepackage{mathtools}
\usepackage{xparse}
\usepackage{tkz-euclide}
\usetkzobj{all}
\usepackage[utf8]{inputenc}
\usepackage{csquotes}
\usepackage[english]{babel}
\usepackage{marvosym}
\usepackage{pgf,tikz}
\usepackage{pgfplots}
\usepackage{fancyhdr}
\usepackage{array}
\usepackage{faktor}
\usepackage{float}
\usepackage{xcolor}
\usepackage{centernot}
\usepackage{silence}
  \WarningFilter*{latex}{Marginpar on page \thepage\space moved}
\usepackage{tcolorbox}
\tcbuselibrary{skins,breakable}
\usepackage{longtable}
\usepackage[amsmath,hyperref]{ntheorem}
\usepackage{hyperref}
\usepackage[noabbrev,capitalize,nameinlink]{cleveref}

% xcolor (scheme: base16 eighties)
\definecolor{base16-eighties-dark}{HTML}{2D2D2D}
\definecolor{base16-eighties-light}{HTML}{D3D0C8}
\definecolor{base16-eighties-magenta}{HTML}{CD98CD}
\definecolor{base16-eighties-red}{HTML}{F47678}
\definecolor{base16-eighties-yellow}{HTML}{E2B552}
\definecolor{base16-eighties-green}{HTML}{98CD97}
\definecolor{base16-eighties-lightblue}{HTML}{61CCCD}
\definecolor{base16-eighties-blue}{HTML}{6498CE}
\definecolor{base16-eighties-brown}{HTML}{D47B4E}
\definecolor{base16-eighties-gray}{HTML}{747369}

% hyperref Package Settings
\hypersetup{
    bookmarks=true,         % show bookmarks bar?
    unicode=true,          % non-Latin characters in Acrobat’s bookmarks
    pdftoolbar=false,        % show Acrobat’s toolbar?
    pdfmenubar=false,        % show Acrobat’s menu?
    pdffitwindow=true,     % window fit to page when opened
    colorlinks=true,
    allcolors=base16-eighties-magenta,
}

% tikz
\usepgfplotslibrary{polar}
\usepgflibrary{shapes.geometric}
\usetikzlibrary{angles,patterns,calc,decorations.markings}
\tikzset{midarrow/.style 2 args={
        decoration={markings,
            mark= at position #2 with {\arrow{#1}} ,
        },
        postaction={decorate}
    },
    midarrow/.default={latex}{0.5}
}
\def\centerarc[#1](#2)(#3:#4:#5)% Syntax: [draw options] (center) (initial angle:final angle:radius)
    { \draw[#1] ($(#2)+({#5*cos(#3)},{#5*sin(#3)})$) arc (#3:#4:#5); }

% enumitems
\newlist{inlinelist}{enumerate*}{1}
\setlist*[inlinelist,1]{%
  label=(\roman*),
}

% Theorem Style Customization
\setlength\theorempreskipamount{2ex}
\setlength\theorempostskipamount{3ex}

\makeatletter
\let\nobreakitem\item
\let\@nobreakitem\@item
\patchcmd{\nobreakitem}{\@item}{\@nobreakitem}{}{}
\patchcmd{\nobreakitem}{\@item}{\@nobreakitem}{}{}
\patchcmd{\@nobreakitem}{\@itempenalty}{\@M}{}{}
\patchcmd{\@xthm}{\ignorespaces}{\nobreak\ignorespaces}{}{}
\patchcmd{\@ythm}{\ignorespaces}{\nobreak\ignorespaces}{}{}

\renewtheoremstyle{break}%
  {\item{\theorem@headerfont
          ##1\ ##2\theorem@separator}\hskip\labelsep\relax\nobreakitem}%
  {\item{\theorem@headerfont
          ##1\ ##2\ (##3)\theorem@separator}\hskip\labelsep\relax\nobreakitem}
\makeatother

% ntheorem + framed
\makeatletter

% ntheorem Declarations
\theorempreskip{10pt}
\theorempostskip{5pt}
\theoremstyle{break}

\newtheorem*{solution}{\faPencil $\enspace$ Solution}
\newtheorem*{remark}{Remark}
\newtheorem{eg}{Example}[section]
\newtheorem{ex}{Exercise}[section]

    % definition env
\theoremprework{\textcolor{base16-eighties-blue}{\hrule height 2pt}}
\theoremheaderfont{\color{base16-eighties-blue}\normalfont\bfseries}
\theorempostwork{\textcolor{base16-eighties-blue}{\hrule height 2pt}}
\theoremindent10pt
\newtheorem{defn}{\faBook \enspace Definition}

    % definition env no num
\theoremprework{\textcolor{base16-eighties-blue}{\hrule height 2pt}}
\theoremheaderfont{\color{base16-eighties-blue}\normalfont\bfseries}
\theorempostwork{\textcolor{base16-eighties-blue}{\hrule height 2pt}}
\theoremindent10pt
\newtheorem*{defnnonum}{\faBook \enspace Definition}

    % theorem envs
\theoremprework{\textcolor{base16-eighties-magenta}{\hrule height 2pt}}
\theoremheaderfont{\color{base16-eighties-magenta}\normalfont\bfseries}
\theorempostwork{\textcolor{base16-eighties-magenta}{\hrule height 2pt}}
\theoremindent10pt
\newtheorem{thm}{\faCoffee \enspace Theorem}

\theoremprework{\textcolor{base16-eighties-magenta}{\hrule height 2pt}}
\theorempostwork{\textcolor{base16-eighties-magenta}{\hrule height 2pt}}
\theoremindent10pt
\newtheorem{propo}[thm]{\faTint \enspace Proposition}

\theoremprework{\textcolor{base16-eighties-magenta}{\hrule height 2pt}}
\theorempostwork{\textcolor{base16-eighties-magenta}{\hrule height 2pt}}
\theoremindent10pt
\newtheorem{crly}[thm]{\faSpaceShuttle \enspace Corollary}

\theoremprework{\textcolor{base16-eighties-magenta}{\hrule height 2pt}}
\theorempostwork{\textcolor{base16-eighties-magenta}{\hrule height 2pt}}
\theoremindent10pt
\newtheorem{lemma}[thm]{\faTree \enspace Lemma}

\theoremprework{\textcolor{base16-eighties-magenta}{\hrule height 2pt}}
\theorempostwork{\textcolor{base16-eighties-magenta}{\hrule height 2pt}}
\theoremindent10pt
\newtheorem{axiom}[thm]{\faShield \enspace Axiom}

    % theorem envs without counter
\theoremprework{\textcolor{base16-eighties-magenta}{\hrule height 2pt}}
\theoremheaderfont{\color{base16-eighties-magenta}\normalfont\bfseries}
\theorempostwork{\textcolor{base16-eighties-magenta}{\hrule height 2pt}}
\theoremindent10pt
\newtheorem*{thmnonum}{\faCoffee \enspace Theorem}

\theoremprework{\textcolor{base16-eighties-magenta}{\hrule height 2pt}}
\theorempostwork{\textcolor{base16-eighties-magenta}{\hrule height 2pt}}
\theoremindent10pt
\newtheorem*{propononum}{\faTint \enspace Proposition}

\theoremprework{\textcolor{base16-eighties-magenta}{\hrule height 2pt}}
\theorempostwork{\textcolor{base16-eighties-magenta}{\hrule height 2pt}}
\theoremindent10pt
\newtheorem*{crlynonum}{\faSpaceShuttle \enspace Corollary}

\theoremprework{\textcolor{base16-eighties-magenta}{\hrule height 2pt}}
\theorempostwork{\textcolor{base16-eighties-magenta}{\hrule height 2pt}}
\theoremindent10pt
\newtheorem*{lemmanonum}{\faTree \enspace Lemma}

\theoremprework{\textcolor{base16-eighties-magenta}{\hrule height 2pt}}
\theorempostwork{\textcolor{base16-eighties-magenta}{\hrule height 2pt}}
\theoremindent10pt
\newtheorem*{axiomnonum}{\faShield \enspace Axiom}

    % proof env
\theoremprework{\textcolor{base16-eighties-brown}{\hrule height 2pt}}
\theoremheaderfont{\color{base16-eighties-brown}\normalfont\bfseries}
\theorempostwork{\textcolor{base16-eighties-brown}{\hrule height 2pt}}
\newtheorem*{proof}{\faPencil \enspace Proof}

    % note and notation env
\theoremprework{\textcolor{base16-eighties-yellow}{\hrule height 2pt}}
\theoremheaderfont{\color{base16-eighties-yellow}\normalfont\bfseries}
\theorempostwork{\textcolor{base16-eighties-yellow}{\hrule height 2pt}}
\newtheorem*{note}{\faQuoteLeft \enspace Note}

\theoremprework{\textcolor{base16-eighties-yellow}{\hrule height 2pt}}
\theorempostwork{\textcolor{base16-eighties-yellow}{\hrule height 2pt}}
\newtheorem*{notation}{\faPaw \enspace Notation}

    % warning env
\theoremprework{\textcolor{base16-eighties-red}{\hrule height 2pt}}
\theoremheaderfont{\color{base16-eighties-red}\normalfont\bfseries}
\theorempostwork{\textcolor{base16-eighties-red}{\hrule height 2pt}}
\theoremindent10pt
\newtheorem*{warning}{\faBug \enspace Warning}

% more environments
\newtcolorbox{redquote}{
  blanker,enhanced,breakable,standard jigsaw,
  opacityback=0,
  coltext=base16-eighties-light,
  left=5mm,right=5mm,top=2mm,bottom=2mm,
  colframe=base16-eighties-red,
  boxrule=0pt,leftrule=3pt,
  fontupper=\itshape
}
\newtcolorbox{bluequote}{
  blanker,enhanced,breakable,standard jigsaw,
  opacityback=0,
  coltext=base16-eighties-light,
  left=5mm,right=5mm,top=2mm,bottom=2mm,
  colframe=base16-eighties-blue,
  boxrule=0pt,leftrule=3pt,
  fontupper=\itshape
}
\newtcolorbox{greenquote}{
  blanker,enhanced,breakable,standard jigsaw,
  opacityback=0,
  coltext=base16-eighties-light,
  left=5mm,right=5mm,top=2mm,bottom=2mm,
  colframe=base16-eighties-green,
  boxrule=0pt,leftrule=3pt,
  fontupper=\itshape
}
\newtcolorbox{yellowquote}{
  blanker,enhanced,breakable,standard jigsaw,
  opacityback=0,
  coltext=base16-eighties-light,
  left=5mm,right=5mm,top=2mm,bottom=2mm,
  colframe=base16-eighties-yellow,
  boxrule=0pt,leftrule=3pt,
  fontupper=\itshape
}
\newtcolorbox{magentaquote}{
  blanker,enhanced,breakable,standard jigsaw,
  opacityback=0,
  coltext=base16-eighties-light,
  left=5mm,right=5mm,top=2mm,bottom=2mm,
  colframe=base16-eighties-magenta,
  boxrule=0pt,leftrule=3pt,
  fontupper=\itshape
}

% ntheorem listtheorem style
\makeatother
\newlength\widesttheorem
\AtBeginDocument{
  \settowidth{\widesttheorem}{Proposition A.1.1.1\quad}
}

\makeatletter
\def\thm@@thmline@name#1#2#3#4{%
        \@dottedtocline{-2}{0em}{2.3em}%
                   {\makebox[\widesttheorem][l]{#1 \protect\numberline{#2}}#3}%
                   {#4}}
\@ifpackageloaded{hyperref}{
\def\thm@@thmline@name#1#2#3#4#5{%
    \ifx\#5\%
        \@dottedtocline{-2}{0em}{2.3em}%
            {\makebox[\widesttheorem][l]{#1 \protect\numberline{#2}}#3}%
            {#4}
    \else
        \ifHy@linktocpage\relax\relax
            \@dottedtocline{-2}{0em}{2.3em}%
                {\makebox[\widesttheorem][l]{#1 \protect\numberline{#2}}#3}%
                {\hyper@linkstart{link}{#5}{#4}\hyper@linkend}%
        \else
            \@dottedtocline{-2}{0em}{2.3em}%
                {\hyper@linkstart{link}{#5}%
                  {\makebox[\widesttheorem][l]{#1 \protect\numberline{#2}}#3}\hyper@linkend}%
                    {#4}%
        \fi
    \fi}
}

\makeatletter
\def\thm@@thmline@noname#1#2#3#4{%
        \@dottedtocline{-2}{0em}{5em}%
                   {{\protect\numberline{#2}}#3}%
                   {#4}}
\@ifpackageloaded{hyperref}{
\def\thm@@thmline@noname#1#2#3#4#5{%
    \ifx\#5\%
        \@dottedtocline{-2}{0em}{5em}%
            {{\protect\numberline{#2}}#3}%
            {#4}
    \else
        \ifHy@linktocpage\relax\relax
            \@dottedtocline{-2}{0em}{5em}%
                {{\protect\numberline{#2}}#3}%
                {\hyper@linkstart{link}{#5}{#4}\hyper@linkend}%
        \else
            \@dottedtocline{-2}{0em}{5em}%
                {\hyper@linkstart{link}{#5}%
                  {{\protect\numberline{#2}}#3}\hyper@linkend}%
                    {#4}%
        \fi
    \fi}
}

\theoremlisttype{allname}

\AtBeginDocument{\renewcommand\contentsname{Table of Contents}}

% Heading formattings
% chapter format
\titleformat{\chapter}%
  {\huge\rmfamily\itshape\color{base16-eighties-magenta}}% format applied to label+text
  {\llap{\colorbox{base16-eighties-magenta}{\parbox{1.5cm}{\hfill\itshape\huge\textcolor{base16-eighties-dark}{\thechapter}}}}}% label
  {5pt}% horizontal separation between label and title body
  {}% before the title body
  []% after the title body

% section format
\titleformat{\section}%
  {\normalfont\Large\rmfamily\itshape\color{base16-eighties-blue}}% format applied to label+text
  {\llap{\colorbox{base16-eighties-blue}{\parbox{1.5cm}{\hfill\itshape\textcolor{base16-eighties-dark}{\thesection}}}}}% label
  {5pt}% horizontal separation between label and title body
  {}% before the title body
  []% after the title body

% subsection format
\titleformat{\subsection}%
  {\normalfont\large\itshape\color{base16-eighties-green}}% format applied to label+text
  {\llap{\colorbox{base16-eighties-green}{\parbox{1.5cm}{\hfill\textcolor{base16-eighties-dark}{\thesubsection}}}}}% label
  {1em}% horizontal separation between label and title body
  {}% before the title body
  []% after the title body

% Sidenote enhancements
\def\mathmarginnote#1{%
  \tag*{\rlap{\hspace\marginparsep\smash{\parbox[t]{\marginparwidth}{%
  \footnotesize#1}}}}
}

% Custom table columning
\newcolumntype{L}[1]{>{\raggedright\let\newline\\\arraybackslash\hspace{0pt}}m{#1}}
\newcolumntype{C}[1]{>{\centering\let\newline\\\arraybackslash\hspace{0pt}}m{#1}}
\newcolumntype{R}[1]{>{\raggedleft\let\newline\\\arraybackslash\hspace{0pt}}m{#1}}

% Custom math operator
% \DeclareMathOperator{\rem}{rem}
\DeclareMathOperator*{\argmax}{arg\,max}
\DeclareMathOperator*{\argmin}{arg\,min}
\DeclareMathOperator{\re}{Re}
\DeclareMathOperator{\im}{Im}
\DeclareMathOperator{\caparg}{Arg}
\DeclareMathOperator{\Ind}{Ind}
\DeclareMathOperator{\Res}{Res}

% Graph styles
\pgfplotsset{compat=1.15}
\usepgfplotslibrary{fillbetween}
\pgfplotsset{four quads/.append style={axis x line=middle, axis y line=
middle, xlabel={$x$}, ylabel={$y$}, axis equal }}
\pgfplotsset{four quad complex/.append style={axis x line=middle, axis y line=
middle, xlabel={$\re$}, ylabel={$\im$}, axis equal }}

% Shortcuts
\newcommand{\floor}[1]{\lfloor #1 \rfloor}      % simplifying the writing of a floor function
\newcommand{\ceiling}[1]{\lceil #1 \rceil}      % simplifying the writing of a ceiling function
\newcommand{\dotp}{\, \cdotp}			        % dot product to distinguish from \cdot
\newcommand{\qed}{\hfill\ensuremath{\square}}   % Q.E.D sign
\newcommand{\abs}[1]{\left|#1\right|}						% absolute value
\newcommand{\lra}[1]{\langle \; #1 \; \rangle}
\newcommand{\at}[2]{\Big|_{#1}^{#2}}
\newcommand{\Arg}[1]{\caparg #1}
\renewcommand{\bar}[1]{\mkern 1.5mu \overline{\mkern -1.5mu #1 \mkern -1.5mu} \mkern 1.5mu}
\newcommand{\quotient}[2]{\faktor{#1}{#2}}
\newcommand{\cyclic}[1]{\left\langle #1 \right\rangle}
	% highlighting shortcuts
\newcommand{\hlimpo}[1]{\textcolor{base16-eighties-red}{\textbf{#1}}}
\newcommand{\hlwarn}[1]{\textcolor{base16-eighties-yellow}{\textbf{#1}}}
\newcommand{\hldefn}[1]{\textcolor{base16-eighties-blue}{\index{#1}\textbf{#1}}}
\newcommand{\hlnotea}[1]{\textcolor{base16-eighties-green}{\textbf{#1}}}
\newcommand{\hlnoteb}[1]{\textcolor{base16-eighties-lightblue}{\textbf{#1}}}
\newcommand{\hlnotec}[1]{\textcolor{base16-eighties-brown}{\textbf{#1}}}
\newcommand{\WTP}{\textcolor{base16-eighties-brown}{WTP} }
\newcommand{\WTS}{\textcolor{base16-eighties-brown}{WTS} }
\newcommand{\ind}[2]{\Ind_{#2}\left( #1 \right)}
\newcommand{\notimply}{\centernot\implies}
\newcommand{\res}[2]{\underset{#2}{\Res} #1 }
\newcommand{\tworow}[3]{\begin{tabular}{@{}#1@{}} #2 \\ #3 \end{tabular}}
\renewcommand{\epsilon}{\varepsilon}
\newcommand{\lrarrow}{\leftrightarrow}
\newcommand{\larrow}{\leftarrow}
\newcommand{\rarrow}{\rightarrow}
\renewcommand{\atop}[2]{\genfrac{}{}{0pt}{}{#1}{#2}}
\newcommand*\dif{\mathop{}\!d}

  % inspiration from: https://tex.stackexchange.com/questions/8720/overbrace-underbrace-but-with-an-arrow-instead#37758
\newcommand{\overarrow}[2]{
  \overset{\makebox[0pt]{\begin{tabular}{@{}c@{}}#2\\[0pt]\ensuremath{\uparrow}\end{tabular}}}{#1}
}
\newcommand{\underarrow}[2]{
  \underset{\makebox[0pt]{\begin{tabular}{@{}c@{}}\downarrow\\[0pt]\ensuremath{#2}\end{tabular}}}{#1}
}

% Document header formatting
\renewcommand{\chaptermark}[1]{\markboth{#1}{}}
\renewcommand{\sectionmark}[1]{\markright{#1}}
\makeatletter
\pagestyle{fancy}
\fancyhead{}
\fancyhead[RO]{\textsl{\@title} \enspace \thepage}
\fancyhead[LE]{\thepage \enspace \textsl{\leftmark \enspace - \enspace \rightmark}}
\makeatother

% Comment the two lines below if you want to print the document
\pagecolor{base16-eighties-dark}
\color{base16-eighties-light}


\DeclareMathOperator{\lub}{lub }
\DeclareMathOperator{\glb}{glb }
\DeclareMathOperator{\Range}{Range }
\DeclareMathOperator{\Domain}{Domain }

\newcommand{\norm}[1]{\left\| #1 \right\|}

\begin{document}
\hypersetup{pageanchor=false}
\maketitle
\hypersetup{pageanchor=true}
\tableofcontents

\chapter*{\faBook \enspace List of Definitions}
\addcontentsline{toc}{chapter}{List of Definitions}
\theoremlisttype{all}
\listtheorems{defn}

\chapter*{\faCoffee \enspace List of Theorems}
\addcontentsline{toc}{chapter}{List of Theorems}
\theoremlisttype{allname}
\listtheorems{axiom,lemma,thm,crly,propo}

\nocite{bforres2018}

\chapter{Lecture 1 Sep 06th}%
\label{chp:lecture_1_sep_06th}
% chapter lecture_1_sep_06th

\section{Course Logistics}%
\label{sec:course_logistics}
% section course_logistics

No content is covered in today's lecture so this chapter will cover some of the important logistical highlights that were mentioned in class.

\begin{itemize}
  \item Assignments are designed to help students understand the content.
  \item Due to shortage of manpower, not all assignment questions will be graded; however, students are encouraged to attempt all of the questions.
  \item To further motivate students to work on ungraded questions, the midterm and final exam will likely recycle some of the assignment questions.
  \item There are no required text, but the professor has prepared course notes for reading. The course note are self-contained.
  \item The approach of the class will be more interactive than most math courses.
  \item Due to the size of the class, students are encouraged to utilize Waterloo Learn for questions, so that similar questions by multiple students can be addressed at the same time.
\end{itemize}

% section course_logistics (end)

\section{Preview into the Introduction}%
\label{sec:preview_into_the_introduction}
% section preview_into_the_introduction

How do we compare the size of two sets?

\begin{itemize}
  \item If the sets are finite, this is a relatively easy task.
  \item If the sets are infinite, we will have to rely on functions.
    \begin{itemize}
      \item Injective functions tell us that the \hlnoteb{domain is of size that is lesser than or equal to the codomain}.
      \item Surjective functions tell us that the \hlnoteb{codomain is of size that is lesser than or equal to the domain}.
      \item So does a bijective function tell us that the domain and codomain have the same size? Yes, although this is not as intuitive as it looks, as it relies on \hlnotea{Cantor-Schr\"oder-Bernstein Theorem}.
    \end{itemize}
\end{itemize}

Now, given two arbitrary sets, are we guaranteed to always be able to compare their sizes? It would be very tempting to immediately say yes, but to do that, one would have to agree on the \hlnotea{Axiom of Choice}. Fortunately, within the realm of this course, the Axiom of Choice is taken for granted.

% section preview_into_the_introduction (end)

% chapter lecture_1_sep_06th (end)

\chapter{Lecture 2 Sep 10th}%
\label{chp:lecture_2_sep_10th}
% chapter lecture_2_sep_10th

\section{Basic Set Theory}%
\label{sec:basic_set_theory}
% section basic_set_theory

We shall use the following notations for some of the common set of numbers that we are already familiar with:
\begin{itemize}
  \item $\mathbb{N}$ denotes the set of natural numbers $\{1, 2, 3, ...\}$;
  \item $\mathbb{Z}$ denotes the set of integers $\{..., -2, -1, 0, 1, 2, ...\}$;
  \item $\mathbb{Q}$ denotes the set of rational numbers $\left\{ \frac{a}{b} \mid a \in \mathbb{Z}, b \in \mathbb{N} \right\}$; and
  \item $\mathbb{R}$ denotes the set of real numbers.
\end{itemize}

We shall start with having certain basic properties of $\mathbb{N}$, $\mathbb{Z}$, and $\mathbb{Q}$.

\newthought{We will use} the notation $A \subset B$ and $A \subseteq B$ interchangably to mean that $A$ is a subset of $B$ with the possibility that $A = B$. When we wish to explicitly emphasize this possibility, we shall use $A \subseteq B$. When we wish to explicitly state that $A$ is a \hlnotea{proper subset} of $B$, we will either specify that $A \neq B$ or simply $A \subsetneq B$.

\begin{defn}[Universal Set]\index{Universal Set}
\label{defn:universal_set}\marginnote{This is a hand-wavy definition, but it is not in the interest of this course to further explore on this topic.}
  A universal set, which we shall generally give the label $X$, is a set that contains all the mathematical objects that we are interested in.
\end{defn}

With a universal set in place, we can have the following definitions:

\begin{defn}[Union]\index{Union}
\label{defn:union}
  Let $X$ be a set. If $\{A_\alpha\}_{\alpha \in I}$ such that $A_{\alpha} \subset X$, then the \hlnoteb{union} for all $A_{\alpha}$ is defined as
  \begin{equation*}
    \bigcup_{\alpha \in I} A_{\alpha} := \{ x \in X \mid \exists \alpha \in I, x \in A_{\alpha} \}.
  \end{equation*}
\end{defn}

\begin{defn}[Intersection]\index{Intersection}
\label{defn:intersection}
  Let $X$ be a set. If $\{A_\alpha\}_{\alpha \in I}$ such that $A_\alpha \subset X$, then the \hlnoteb{intersection} for all $A_\alpha$ is defined as
  \begin{equation*}
    \bigcap_{\alpha \in I} A_\alpha := \{ x \in X \mid \forall \alpha \in I, x \in A_\alpha \}.
  \end{equation*}
\end{defn}

\begin{defn}[Set Difference]\index{Set Difference}
\label{defn:set_difference}
  Let $X$ be a set and $A, B \subseteq X$. The \hlnoteb{set difference} of $A$ from $B$ is defined as
  \begin{equation*}
    A \setminus B := \{ x \in X \mid x \in A, x \notin B \}.
  \end{equation*}
\end{defn}

On a similar notion:

\begin{defn}[Symmetric Difference]\index{Symmetric Difference}
\label{defn:symmetric_difference}
Let $X$ be a set and $A, B \subseteq X$. The \hlnoteb{symmetric difference} of $A$ and $B$ is defined as\marginnote{In words, for an element in the symmetric difference of two sets, the element is either in $A$ or $B$ but not both. We can also think of the symmetric difference as
\begin{equation*}
  ( A \cup B ) \setminus (A \cap B)
\end{equation*}
or
\begin{equation*}
  ( A \setminus B ) \cup (B \setminus A).
\end{equation*}
}
  \begin{equation*}
    A \Delta B := \{ x \in X \mid ( x \in A \land x \notin B ) \lor ( x \notin A \land x \in B ) \}.
  \end{equation*}
\end{defn}

We can also talk about the non-members of a set:

\begin{defn}[Set Complement]\index{Set Complement}
\label{defn:set_complement}
  Let $X$ be a set and $A \subset X$. The set of all non-members of $A$ is called the \hlnoteb{complement} of $A$, which we denote as
  \begin{equation*}
    A^c := \{ x \in X \mid x \notin A \}.
  \end{equation*}
\end{defn}

\begin{note}
  Note that
  \begin{equation*}
    \left( A^c \right)^c = \{ x \in X \mid x \notin A^c \} = \{ x \in X \mid x \in A \} = A.
  \end{equation*}
\end{note}

Now taking a step away from that, we define the following:

\begin{defn}[Empty Set]\index{Empty Set}
\label{defn:empty_set}
  An \hlnoteb{empty set}, denoted by $\emptyset$, is a set that contains nothing.
\end{defn}

\begin{note}
  The empty set is set to be a subset of all sets.
\end{note}

\begin{defn}[Power Set]\index{Power Set}
\label{defn:power_set}
  Let $X$ be a set. The power set of $X$ is the set that contains all subsets of $X$, i.e.
  \begin{equation*}
    \mathcal{P}(X) := \{ A \mid A \subset X \}.
  \end{equation*}
\end{defn}

\begin{note}
  A power set is always non-empty, since $\emptyset \in \mathcal{P}(\emptyset)$, and since $\emptyset \subset X$ for any set $X$, we have $\emptyset \in \mathcal{P}(X)$.
\end{note}

\begin{eg}
  Let $X = \{1, 2, ..., n\}$. There are several ways we can show that the size of $\mathcal{P}(X)$ is $2^n$. One of the methods is by using a characteristic function that maps from $A$ to $\{0, 1\}$, defined by
  \begin{gather*}
    X_A: A \to \{0, 1\} \\
    X_A(x) = \begin{cases}
      1 & x \in A \\
      0 & x \notin A
    \end{cases}.
  \end{gather*}
  Using this function, each element in $X$ have 2 states: one being in the subset, and the other being not in the subset, which are represented by $1$ and $0$ respectively. It is then clear that there are $2^n$ of such configurations.
\end{eg}

\begin{thm}[De Morgan's Laws]
\index{De Morgan's Laws}
\label{thm:de_morgan_s_laws}
  Let $X$ be a set. Given $\{A_\alpha\}_{\alpha \in I} \subset \mathcal{P}(X)$, we have
  \begin{enumerate}
    \item $\left( \bigcup\limits_{\alpha \in I} A_\alpha \right)^c = \bigcap\limits_{\alpha \in I} A_\alpha^c$; and
    \item $\left( \bigcap\limits_{\alpha \in I} A_\alpha \right)^c = \bigcup\limits_{\alpha \in I} A_\alpha^c$.
  \end{enumerate}
\end{thm}

\begin{proof}
  \begin{enumerate}
    \item Note that
      \begin{align*}
        x \in \left( \bigcup_{\alpha \in I} A_\alpha \right)^c &\iff \nexists \alpha \in I \enspace x \in A_\alpha \\
        &\iff \forall \alpha \in I \enspace x \notin A_\alpha \\
        &\iff \forall \alpha \in I \enspace x \in A_\alpha^c \text{ by set complementation } \\
        &\iff x \in \bigcap_{\alpha \in I} A_\alpha^c.
      \end{align*}

    \item Observe that, by part 1,
      \begin{align*}
        \left( \bigcap_{\alpha \in I} A_\alpha \right)^c = \left( \left( \bigcup_{\alpha \in I} A_{\alpha}^c \right)^c \right)^c = \bigcup_{\alpha \in I} A_\alpha^c.
      \end{align*}
  \end{enumerate}\qed
\end{proof}

\begin{eg}
  Suppose $I = \emptyset$. Then what is $\bigcup\limits_{\alpha \in \emptyset} A_\alpha$? It is sensible to think that all we are left with is simply a union of empty sets, and so
  \begin{equation}\label{eq:union_of_empty_sets}
    \bigcup_{\alpha \in \emptyset} A_\alpha = \emptyset.
  \end{equation}
  And what about $\bigcap\limits_{\alpha \in \emptyset} A_\alpha$? By \cref{thm:de_morgan_s_laws}, it is quite clear from \cref{eq:union_of_empty_sets} that
  \begin{equation*}
    \bigcap_{\alpha \in \emptyset} A_\alpha = X.
  \end{equation*}
\end{eg}

% section basic_set_theory (end)

\section{Products of Sets}%
\label{sec:products_of_sets}
% section products_of_sets

\begin{defn}[Product of Sets]\index{Product of Sets}
\label{defn:product_of_sets}
  Given $2$ sets $X$ and $Y$, the \hlnoteb{product} of $X$ and $Y$ is given by
  \begin{equation*}
    X \times Y := \{ (x, y) \mid x \in X , y \in Y \}.
  \end{equation*}
  We often refer to elements of $X \times Y$ as \hldefn{tuples}.
\end{defn}

\begin{note}
  Now if
  \begin{align*}
    X &= \{ x_1, x_2, ..., x_n \}, \\
    Y &= \{ y_1, y_2, ..., y_m \},
  \end{align*}
  then
  \begin{equation*}
    X \times Y = \{ (x_i, y_j) \mid i = 1, 2, ..., n, \; j = 1, 2, ..., m \}
  \end{equation*}
  and so the size of $X \times Y$ is $mn$.
\end{note}

Consequently, we can think of tuples as two elements being in some ``relation''.

\begin{defn}[Relation]\index{Relation}
\label{defn:relation}
  A \hlnoteb{relation} on sets $X$ and $Y$ is a subset $R$ of the product $X \times Y$. We write
  \begin{equation*}
    xRy \enspace \text{ if } (x, y) \in R \subset X \times Y.
  \end{equation*}
  We call
  \begin{itemize}
    \item $\{ x \in X \mid \exists y \in Y, (x, y) \in R \}$ as the \hldefn{domain} of $R$; and 
    \item $\{ y \in Y \mid \exists x \in X, (x, y) \in R \}$ as the \hldefn{range} of $R$.
  \end{itemize}
\end{defn}

In relation to that, functions are, essentially, relations.

\begin{defn}[Function]\index{Function}
\label{defn:function}
  A \hlnoteb{function} from $X$ to $Y$ is a relation $R$ such that
  \begin{equation*}
    \forall x \in X \; \exists! y \in Y \; (x, y) \in R.
  \end{equation*}
\end{defn}

\newthought{Suppose} $X_1, X_2, ..., X_n$ are non-empty\sidenote{We are typically only interested in non-empty sets, since empty sets usually lead us to vacuous truths, which are not interesting.} sets. We can define
\begin{equation*}
  X_1 \times X_2 \times \hdots \times X_n = \prod_{i=1}^{n} X_i := \{ (x_1, x_2, ..., x_n) \mid x_i \in X_i \}.
\end{equation*}
Now if $X_i = X_j = X$ for all $i, j = 1, 2, ..., n$, we write
\begin{equation*}
  \prod_{i=1}^{n} X_i = \prod_{i=1}^{n} X = X^n.
\end{equation*}

\newthought{And now comes the problem}: given a collection $\{X_\alpha\}_{\alpha \in I}$ of non-empty sets\sidenote{i.e. we now talk about arbitrary $\alpha \in I$.}, what do we mean by
\begin{equation*}
  \prod_{\alpha \in I} X_\alpha ?
\end{equation*}

To motivate for what comes next, consider
\begin{equation*}
  \prod_{i=1}^{n} X_i = X_1 \times \hdots \times X_n = \{ (x_1, ..., x_n) \mid x_i \in X_i \}.
\end{equation*}
Choose $(x_1, ..., x_n) \in \prod\limits_{i=1}^{n} X_i$. This induces a function
\begin{equation*}
  f_{(x_1, ..., x_n)} : \{1, ..., n\} \to \bigcup_{i=1}^{n} X_i
\end{equation*}
with
\begin{align*}
  f(1) &= x_1 \in X_1 \\
  f(2) &= x_2 \in X_2 \\
       &\vdots \\
  f(n) &= x_n \in X_n
\end{align*}

Now assume for a more general $f$ such that
\begin{equation*}
  f: \{1, ..., n\} \to \bigcup_{i=1}^{n} X_i
\end{equation*}
is defined by
\begin{equation*}
  f(i) \in X_i.
\end{equation*}
Then, we have
\begin{equation*}
  ( f(1), f(2), ..., f(n) ) \in \prod_{i=1}^{n} X_i,
\end{equation*}
which leads us to the following notion:

\begin{defn}[Choice Function]\index{Choice Function}
\label{defn:choice_function}
  Given a collection $\{X_\alpha\}_{\alpha \in I}$ of non-empty sets, let
  \begin{equation*}
    \prod_{\alpha \in I} X_\alpha = \left\{ f: I \to \bigcup_{\alpha \in I} X_\alpha \right\}
  \end{equation*}
  such that $f(\alpha) \in X_\alpha$. Such an $f$ is called a \hlnoteb{choice function}.
\end{defn}

And so we may ask a similar question as before: if each $X_\alpha$ is non-empty, is $\prod\limits_{\alpha \in I} X_\alpha$ non-empty? Turns out this is not as easy to show. In fact, it is essentially impossible to show, because this is exactly the \hlnotea{Axiom of Choice}.

% section products_of_sets (end)

% chapter lecture_2_sep_10th (end)

\chapter{Lecture 3 Sep 12th}%
\label{chp:lecture_3_sep_12th}
% chapter lecture_3_sep_12th

\section{Axiom of Choice}%
\label{sec:axiom_of_choice}
% section axiom_of_choice

\newthought{Recall} our final question of last lecture: If $\{ X_\alpha \}_{\alpha \in I}$ is a non-empty collection of non-empty sets, is
\begin{equation*}
  \prod_{\alpha \in I} X_\alpha \neq \emptyset \enspace ?
\end{equation*}

Turns out this is widely known (in the world of mathematics) as the \hlnotea{Axiom of Choice}.

\begin{axiom}[Zermelo's Axiom of Choice]
\index{Zermelo's Axiom of Choice}\index{Axiom of Choice}
\label{axiom:zermelo_s_axiom_of_choice}
  If $\{ X_\alpha \}_{\alpha \in I}$ is a non-empty collection of non-empty sets, then
  \begin{equation*}
    \prod_{\alpha \in I} X_\alpha \neq \emptyset.
  \end{equation*}
\end{axiom}

An equivalent statement of the above axiom is:

\begin{axiom}[Zermelo's Axiom of Choice v2]
\index{Zermelo's Axiom of Choice}\index{Axiom of Choice}
\label{axiom:zermelo_s_axiom_of_choice_v2}
  $X \neq \emptyset \implies$
  \begin{equation*}
    \exists f : \mathcal{P}(X) \setminus \{ \emptyset \} \to X \; \forall A \in \mathcal{P}(X) \setminus \{ \emptyset \} \; f(A) \in A
  \end{equation*}
  where $f$ is the \hyperref[defn:choice_function]{choice function}.
\end{axiom}

\begin{ex}
  Prove that \cref{axiom:zermelo_s_axiom_of_choice} and \cref{axiom:zermelo_s_axiom_of_choice_v2} are equivalent.
\end{ex}

\begin{proof}
  \textbf{From \cref{axiom:zermelo_s_axiom_of_choice} to \cref{axiom:zermelo_s_axiom_of_choice_v2}}:

  Since $X \neq \emptyset$, we have that $\mathcal{P}(X) \setminus \{ \emptyset \}$ is a non-empty collection of non-empty sets. Therefore,
  \begin{equation*}
    \prod_{A \in \mathcal{P}(X) \setminus \{ \emptyset \} } A \neq \emptyset.
  \end{equation*}
  So we know that
  \begin{equation*}
    \exists ( x_A )_{A \in \mathcal{P}(X) \setminus \{ \emptyset \} } \in \prod_{A \in \mathcal{P}(X) \setminus \{ \emptyset \} } A.
  \end{equation*}
  We then simply need to choose the choice function $f : \mathcal{P}(X) \setminus \{ \emptyset \} \to X$ such that
  \begin{equation*}
    f(A) = x_A \in A.
  \end{equation*}

  \noindent\textbf{From \cref{axiom:zermelo_s_axiom_of_choice_v2} to \cref{axiom:zermelo_s_axiom_of_choice}}: 

  Let $X_\alpha \in \mathcal{P}(X)$ for $\alpha \in I$, where $I$ is some index set. We know that not all $X_\alpha = \emptyset$ since $X \neq \emptyset$. Choose $J \subseteq I$ such that $\{ X_\alpha \}_{\alpha \in J}$ is a non-empty collection of non-empty sets. Let $f: \mathcal{P}(X) \setminus \{ \emptyset \}$ be any choice function. By \cref{axiom:zermelo_s_axiom_of_choice_v2}, 
  \begin{equation*}
    \forall X_\alpha \in \mathcal{P}(X) \setminus \{ \emptyset \} \quad f(X_\alpha) \in X_\alpha.
  \end{equation*}
  Therefore,
  \begin{equation*}
    ( f(X_\alpha) )_{\alpha \in J} \in \prod_{\alpha \in J} X_\alpha.
  \end{equation*}\qed
\end{proof}

% section axiom_of_choice (end)

\section{Relations}%
\label{sec:relations}
% section relations

Now, it is in our interest to start talking about comparisons or relations between the mathematical objects that we have defined.

\begin{defn}[Relations]\index{Relations}
\label{defn:relations}
  A relation $R$ on a set $X$ is \sidenote{We can look at this definition as $R \subseteq X \times X$. Under such a definition, we would have
  \begin{itemize}
    \item (\textbf{Reflexive}) $\forall x \in X \enspace (x, x) \in R$;
    \item (\textbf{Symmetric}) $\forall x, y \in X \enspace (x, y) \in R \iff (y, x) \in R$;
    \item (\textbf{Anti-symmetric}) $\forall x, y \in X \enspace (x, y), (y, x) \in R \implies x = y$;
    \item (\textbf{Transitive}) $\forall x, y, z \in X \enspace (x, y), (y, z) \in R \implies (x, z) \in R$.
  \end{itemize}}
  \begin{itemize}
    \item (\hldefn{Reflexive}) $\forall x \in X \enspace xRx$;
    \item (\hldefn{Symmetric}) $\forall x, y \in X \enspace xRy \iff yRx$;
    \item (\hldefn{Anti-symmetric}) $\forall x, y \in X \enspace xRy \land yRx \implies x = y$;
    \item (\hldefn{Transitive}) $\forall x, y , z \in X \enspace xRy \land yRz \implies xRz$.
  \end{itemize}
\end{defn}

\begin{eg}
  Let $X = \mathbb{R}$, and let $xRy \iff x \leq y$, where $\leq$ is the notion of ``less than or equal to'', which we shall assume that it has the meaning that we know. Observe that $\leq$ is:
  \begin{itemize}
    \item reflexive: $\forall x \in \mathbb{R} \enspace x \leq x$ is true;
    \item anti-symmetric: $\forall x, y \in \mathbb{R} \enspace x \leq y \land y \leq x \implies x = y$; and
    \item transitive: $\forall x, y , z \in \mathbb{R} \enspace x \leq y \land y \leq z \implies x \leq z$.
  \end{itemize}
\end{eg}

\begin{eg}
  Let $Y \neq \emptyset, \, X = \mathcal{P}(Y)$, with $ARB \iff A \subseteq B$. Observe that $\subseteq$ is:
  \begin{itemize}
    \item reflexive: $\forall A \in \mathcal{P}(Y) \enspace ARA \iff A \subseteq A$ is true;
    \item anti-symmetric: $\forall A, B \in \mathcal{P}(Y) \enspace ARB \land BRA \iff A \subseteq B \land B \subseteq A \implies A = B$;
    \item transitive: $\forall A, B, C \in \mathcal{P}(Y) \enspace ARB \land BRC \iff A \subseteq B \land B \subseteq C \implies A \subseteq C$.
  \end{itemize}
\end{eg}

\begin{eg}
  Let $Y \neq \emptyset, \, X = \mathcal{P}(Y)$, with $ARB \iff A \supseteq B$. Observe that $\supseteq$ is:
  \begin{itemize}
    \item reflexive: $\forall A \in \mathcal{P}(Y) \enspace ARA \iff A \subseteq A$;
    \item anti-symmetric: $\forall A, B \in \mathcal{P}(Y) \enspace ARB \land BRA \iff A \supseteq B \land B \supseteq A \implies A = B$;
    \item transitive: $\forall A, B, C \in \mathcal{P}(Y) \enspace ARB \land BRC \iff A \supseteq B \land B \supseteq C \implies A \supseteq C$.
  \end{itemize}
\end{eg}

All the above examples are also known as \textit{partially ordered sets}.

\begin{defn}[Partially Ordered Sets]\index{Partially Ordered Sets}\index{Poset}
\label{defn:partially_ordered_sets}
The set $X$ with the relation $R$ on $X$ is called a \hlnoteb{partially ordered set} (or a \hlnoteb{poset}) if $R$ is\marginnote{The ``partial'' in `partially ordered'' indicates that not every pair of elements need to be comparable, i.e. there may be pairs for which neither precedes the other (anti-symmetry).}
  \begin{itemize}
    \item reflexive;
    \item anti-symmetric; and
    \item transitive.
  \end{itemize}
  We denote a poset by $(X, R)$.
\end{defn}

\begin{note}
  If $(X, R)$ is a poset, then if $A \subseteq X$, and $R_1 = R \restriction_{A \times A}$, then $(A, R_1)$ is also a poset.
\end{note}

\begin{eg}\label{eg:number_of_posets}
  How many possible relations can we define on these sets to make them into posets?
  \begin{enumerate}
    \item $X = \emptyset$
      \begin{solution}
        We have that $R = \emptyset \times \emptyset$, and so the only relation we have is an empty relation. Then it is vacuously true that $(X, R)$ a poset.
      \end{solution}
    \item $X = \{ x \}$
      \begin{solution}
        We have that $R = X \times X = \{ ( x, x ) \}$. It it clear that $(X, R)$ is a poset.
      \end{solution}
    \item $X = \{ x, y \}$
      \begin{solution}
        There are 3 possible relations:\marginnote{
          3 possibilities represented as graphs (known as \href{https://en.wikipedia.org/wiki/Hasse_diagram}{Hasse diagram}\index{Hasse diagram}), separated by lines:

        \begin{tikzpicture}
          \node[circle,fill,inner sep=1pt,label={right:{ $x$ }}] at (1, 3) {};
          \node[circle,fill,inner sep=1pt,label={right:{ $y$ }}] at (3, 3) {};

          \draw[-] (0,2.5) -- (4,2.5);

          \node[circle,fill,inner sep=1pt,label={right:{ $x$ }}] at (1, 2) {};
          \draw[-] (1,2) -- (1,1);
          \node[circle,fill,inner sep=1pt,label={right:{ $y$ }}] at (1, 1) {};

          \draw[-] (2,2.5) -- (2,0.5);

          \node[circle,fill,inner sep=1pt,label={0:{ $y$ }}] at (3, 2) {};
          \draw[-] (3,2) -- (3,1);
          \node[circle,fill,inner sep=1pt,label={right:{ $x$ }}] at (3, 1) {};
        \end{tikzpicture}
        }
        \begin{itemize}
          \item a relation where $xRx$ and $yRy$;
          \item a relation where $xRy$; or
          \item a relation where $yRx$.
        \end{itemize}
      \end{solution}
    \item $X = \{ x, y, z \}$
      \begin{solution}
        The following are all the possibilities represented by graphs, where the underlined numbers represent the number of ways we can rearrange the elements for unique relations:

        \begin{center}
          \begin{tikzpicture}
            \node[circle,fill,inner sep=1pt,label={right:{ $x$ }}] at (1, 3) {};
            \node[circle,fill,inner sep=1pt,label={right:{ $y$ }}] at (2, 3) {};
            \node[circle,fill,inner sep=1pt,label={right:{ $z$ }}] at (3, 3) {};
            \node at (-2,3) {\underline{1}};

            \node[circle,fill,inner sep=1pt,label={right:{ $x$ }}] at (-1, 2) {};
            \node[circle,fill,inner sep=1pt,label={right:{ $y$ }}] at (1, 2) {};
            \node[circle,fill,inner sep=1pt,label={right:{ $z$ }}] at (0, 1) {};
            \draw[-] (-1,2) -- (0,1);
            \draw[-] (1,2) -- (0,1);
            \node at (-2,1.5) {\underline{3}};

            \node[circle,fill,inner sep=1pt,label={right:{ $x$ }}] at (3, 1) {};
            \node[circle,fill,inner sep=1pt,label={right:{ $y$ }}] at (5, 1) {};
            \node[circle,fill,inner sep=1pt,label={right:{ $z$ }}] at (4, 2) {};
            \draw[-] (4,2) -- (3,1);
            \draw[-] (4,2) -- (5,1);
            \node at (6,1.5) {\underline{3}};

            \node[circle,fill,inner sep=1pt,label={right:{ $x$ }}] at (-1, 0) {};
            \node[circle,fill,inner sep=1pt,label={right:{ $y$ }}] at (1, 0) {};
            \node[circle,fill,inner sep=1pt,label={right:{ $z$ }}] at (1, -1) {};
            \draw[-] (1,0) -- (1,-1);
            \node at (-2,-0.5) {\underline{6}};

            \node[circle,fill,inner sep=1pt,label={right:{ $x$ }}] at (4, 0) {};
            \node[circle,fill,inner sep=1pt,label={right:{ $y$ }}] at (4, -0.5) {};
            \node[circle,fill,inner sep=1pt,label={right:{ $z$ }}] at (4, -1) {};
            \draw[-] (4,0) -- (4,-1);
            \node at (6,-0.5) {\underline{6}};
          \end{tikzpicture}
        \end{center}
        Therefore, we see that there are a total of
        \begin{equation*}
          1 + 3 + 3 + 6 + 6 = 19 \text{ relations}.
        \end{equation*}
      \end{solution}
  \end{enumerate}
\end{eg}

\begin{ex}
  How many possible relations can we define on a set of $6$ elements to the set into a poset?

  \begin{solution}
    \hlwarn{to be added}
  \end{solution}
\end{ex}

\begin{defn}[Totally Ordered Sets / Chains]\index{Totally Ordered Sets}\index{Chains}
\label{defn:totally_ordered_sets_chains}
  The set $X$ with the relation $R$ on $X$ is called a \hlnoteb{totally ordered set} (or a \hlnoteb{chain}) if $(X, R)$ is a poset with the exception that, for any $x, y \in X$, either $xRy$ or $yRx$ but not both.
\end{defn}

\begin{defn}[Bounds]
\label{defn:bounds}
  Let $(X, \leq)$ be a poset. Let $A \subset X$. We say $x_0 \in X$ is an \hldefn{upper bound} for $A$ if
  \begin{equation*}
    \forall a \in A \quad a \leq x_0.
  \end{equation*}
  If $A$ has an upper bound, we say that $A$ is \hldefn{bounded above}. If $A$ is bounded above, then $x_0$ is the \hldefn{least upper bound} (or \hldefn{supremum}) of $A$ is for any $x_1 \in X$ that is an upper bound of $A$, we have
  \begin{equation*}
    x_0 \leq x_1.
  \end{equation*}
  We write $x_0 = \lub (A) = \sup (A)$. If $\sup (A) \in A$, then $\sup (A) = \max (A)$ is the \hldefn{maximum} of $A$.

  We can analogously define for:
  \begin{align*}
    \text{upper bound } &\to \text{ lower bound } \\
    \text{bounded above } &\to \text{ bounded below } \\
    \text{least upper bound, } \lub &\to \text{ greatest lower bound, } \glb \\
    \text{supremum, } \sup &\to \text{ infimum, } \inf \\
    \text{maximum, } \max &\to \text{ minimum, } \min
  \end{align*}
\end{defn}

\begin{note}
  By \hlnotea{anti-symmetry} of posets, we have that $\max, \sup, \min, \inf$ are all unique if they exists.
\end{note}

\begin{eg}[Least Upper Bound Property of $\mathbb{R}$]\index{Least Upper Bound Property of $\mathbb{R}$}
  Let $X = \mathbb{R}$, and $\leq$ be the order that we have defined. Every bounded non-empty subset of $X$ has a supremum.
\end{eg}

\begin{eg}
  Let $Y \neq \emptyset$, and $X = \mathcal{P}(Y)$, and $\subseteq$ the ordering by inclusion. We know that $Y$ is the maximum element of $(X, \subseteq)$. Then the collection $\{ A_\alpha \}_{\alpha \in I} \subset \mathcal{P}(Y)$ is bounded above by $Y$, and we have that\marginnote{Now if $Y = \emptyset$, we would end up having
  \begin{gather*}
    \sup \left( \left\{ A_\alpha \right\}_{\alpha \in I} \right) = \emptyset \\
    \inf \left( \left\{ A_\alpha \right\}_{\alpha \in I} \right) = X
  \end{gather*}
  But this makes sense, since the empty set would be the least of upper bounds, and since $X = \mathcal{P}(Y)$ would have to be the greatest of lower bounds.
  }
  \begin{gather*}
    \sup \left( \left\{ A_\alpha \right\}_{\alpha \in I} \right) = \bigcup_{\alpha \in I} A_\alpha \\
    \inf \left( \left\{ A_\alpha \right\}_{\alpha \in I} \right) = \bigcap_{\alpha \in I} A_\alpha
  \end{gather*}
\end{eg}

% section relations (end)

% chapter lecture_3_sep_12th (end)

\chapter{Lecture 4 Sep 14th}%
\label{chp:lecture_4_sep_14th}
% chapter lecture_4_sep_14th

\section{Zorn's Lemma}%
\label{sec:zorn_s_lemma}
% section zorn_s_lemma

\begin{defn}[Maximal Element]\index{Maximal Element}
\label{defn:maximal_element}
  Let $(X, \leq)$ be a poset. An element $x \in X$ is \hlnoteb{maximal} if whenever $y \in X$ is such that $x \leq y$, we must have $y = x$.
\end{defn}

\begin{eg}
  Looking back at \cref{eg:number_of_posets}, on the set $X = \{ x, y, z \}$, we have that the maximal element in each possible poset is/are:\marginnote{This shows to us that the maximal element does not have to be unique.}

  \begin{center}
  \begin{tikzpicture}
    \node[circle,fill,inner sep=1pt,label={right:{ $x$ }}] at (-1, 3) {};
    \node[circle,fill,inner sep=1pt,label={right:{ $y$ }}] at (0, 3) {};
    \node[circle,fill,inner sep=1pt,label={right:{ $z$ }}] at (1, 3) {};
    \node[label={right:{$x, y, z$ are all maximal}}] at (2, 3) {};

    \node[circle,fill,inner sep=1pt,label={right:{ $x$ }}] at (-1, 2) {};
    \node[circle,fill,inner sep=1pt,label={right:{ $y$ }}] at (1, 2) {};
    \node[circle,fill,inner sep=1pt,label={right:{ $z$ }}] at (0, 1) {};
    \draw[-] (-1,2) -- (0,1);
    \draw[-] (1,2) -- (0,1);
    \node[label={right:{$z$ is maximal}}] at (2, 1.5) {};

    \node[circle,fill,inner sep=1pt,label={right:{ $x$ }}] at (-1, -1) {};
    \node[circle,fill,inner sep=1pt,label={right:{ $y$ }}] at (1, -1) {};
    \node[circle,fill,inner sep=1pt,label={right:{ $z$ }}] at (0, 0) {};
    \draw[-] (0,0) -- (1,-1);
    \draw[-] (0,0) -- (-1,-1);
    \node[label={right:{$x, y$ are both maximal}}] at (2, -0.5) {};

    \node[circle,fill,inner sep=1pt,label={right:{ $x$ }}] at (-1, -2) {};
    \node[circle,fill,inner sep=1pt,label={right:{ $y$ }}] at (1, -2) {};
    \node[circle,fill,inner sep=1pt,label={right:{ $z$ }}] at (1, -3) {};
    \draw[-] (1,-2) -- (1,-3);
    \node[label={right:{$x, z$ are both maximal}}] at (2, -2.5) {};

    \node[circle,fill,inner sep=1pt,label={right:{ $x$ }}] at (0, -4) {};
    \node[circle,fill,inner sep=1pt,label={right:{ $y$ }}] at (0, -4.5) {};
    \node[circle,fill,inner sep=1pt,label={right:{ $z$ }}] at (0, -5) {};
    \draw[-] (0,-4) -- (0,-5);
    \node[label={right:{$z$ is maximal}}] at (2, -4.5) {};
  \end{tikzpicture}
  \end{center}
\end{eg}

\begin{eg}
  \begin{itemize}
    \item Given $X \neq \emptyset$, the maximal element of the poset $(\mathcal{P}(X), \subseteq)$ is $X$.
    \item Given $X \neq \emptyset$, the maximal element of the poset $(\mathcal{P}(X), \supseteq)$ is $\emptyset$.
    \item The poset $(\mathbb{R}, \leq)$ has no maximal element.
  \end{itemize}
\end{eg}

\begin{axiom}[Zorn's Lemma]
\index{Zorn's Lemma}
\label{axiom:zorn_s_lemma}
  If $(X, \leq)$ is a non-empty poset such that every chain $S \subset X$ has an upper bound, then $(X, \leq)$ has a maximal element.
\end{axiom}

\begin{thm}[\imponote Non-Zero Vector Spaces has a Basis]
\label{thm:non_zero_vector_spaces_has_a_basis}
  Every non-zero vector space, $V$, has a basis.
\end{thm}
\marginnote{The flow of this proof is a typical approach when Zorn's Lemma is involved.}
\begin{proof}[\faStar]
  Let
  \begin{equation*}
    \mathcal{L} := \{ A \subset V \mid A \text{ is linearly independent } \}.
  \end{equation*}
  Note that $\mathcal{L} \neq \emptyset$ since $V \neq \{ 0 \}$. Now order elements of $\mathcal{L}$ with $\subseteq$. It suffices to show that $(\mathcal{L}, \subseteq)$ has a maximal element, since this maximal element must be a basis. Otherwise, we would contradict the maximality of such an element.\sidenote{This is the key to this proof.}

  Now let $S = \{ A_\alpha \}_{\alpha \in I}$ be a chain in $\mathcal{L}$. Let
  \begin{equation*}
    A_0 = \bigcup_{\alpha \in I} A_\alpha.
  \end{equation*}
  \hlwarn{Require clarification before proceeding...}
\end{proof}

\begin{defn}[Well-Ordered]\index{Well-Ordered}
\label{defn:well_ordered}\marginnote{
\begin{ex}
  Prove that well-ordered sets are chains.
\end{ex}
}
\noindent We say that a poset $(X, \leq)$ is \hlnoteb{well-ordered} if every non-empty subset $A \subset X$ has a least/minimal element in $A$.
\end{defn}

\begin{eg}
  $(\mathbb{N}, \leq)$ is well-ordered.
\end{eg}

\begin{axiom}[Well-Ordering Principle]
\index{Well-Ordering Principle}
\label{axiom:well_ordering_principle}
  Every non-empty set can be well-ordered.
\end{axiom}

\begin{thm}[Axioms of Choice and Its Equivalents]
\label{thm:axioms_of_choice_and_its_equivalents}
  TFAE:\marginnote{\begin{ex}Prove \cref{thm:axioms_of_choice_and_its_equivalents}\end{ex}}
  \begin{enumerate}
    \item Axiom of Choice, \cref{axiom:zermelo_s_axiom_of_choice}
    \item Zorn's Lemma, \cref{axiom:zorn_s_lemma}
    \item Well-Ordering Principle, \cref{axiom:well_ordering_principle}.
  \end{enumerate}
\end{thm}

\begin{proof}
  (3) $\implies$ (1) is simple; let the choice function be such that we pick the minimal element from each set among a non-empty collection of non-empty sets. It is clear that the product of these sets will always have an element, in particular the tuple where each component is the minimal element of each set.

  \hlwarn{The rest will be added once I've worked it out}
\end{proof}

\begin{eg}
  Let $X = \mathbb{Q}$. Let $\phi: \mathbb{Q} \to \mathbb{N}$ be defined such that
  \begin{equation*}
    \phi\left(\frac{m}{n}\right) = \begin{cases}
      2^m 5^n    & m > 0 \\
      1          & m = 1 \\
      3^{-m} 7^n & m < 0
    \end{cases}
  \end{equation*}
  By the \hlnotea{unique prime factorization of natural numbers} (or \hlnotea{Fundamental Theorem of Arithmetic}), we have that $\phi$ is injective. In fact,
  \begin{equation*}
    r \leq s \iff \phi(r) \leq \phi(s),
  \end{equation*}
  showing to us that we have a well-ordering on $\mathbb{Q}$.
\end{eg}

% section zorn_s_lemma (end)

\section{Cardinality}%
\label{sec:cardinality}
% section cardinality

\subsection{Equivalence Relation}%
\label{sub:equivalence_relation}
% subsection equivalence_relation

\begin{defn}[Equivalence Relation]\index{Equivalence Relation}
\label{defn:equivalence_relation}
  Let $X$ be non-empty set. A relation $\sim$ on $X$ is an \hlnoteb{equivalence relation} if it is
  \begin{itemize}
    \item reflexive; 
    \item symmetric; and
    \item transitive.
  \end{itemize}
\end{defn}

\begin{defn}[Equivalence Class]\index{Equivalence Class}
\label{defn:equivalence_class}
  Let $X$ be a non-empty set, and $x \in X$. An \hlnoteb{equivalence class} of $x$ under the equivalence relation $\sim$ is defined as
  \begin{equation*}
    [x] := \{ y \in X \mid x \sim y \}.
  \end{equation*}
\end{defn}

\begin{note}
  Note that we either have $[x] = [y]$ or $[x] \cap [y] = \emptyset$. This is sensible, since if $w \in [x]$, then $w \sim x$. If $w \in [y]$, then we are done. If $w \notin [y]$, suppose $\exists v \in [y]$ such that $w \sim v$, which then implies $w \in [y]$ which is a contradiction.

  This results shows to us that
  \begin{equation*}
    X = \bigcup_{x \in X} [x],
  \end{equation*}
  or in words, equivalence classes \textit{partition} the set.
\end{note}

\begin{defn}[Partition]\index{Partition}
\label{defn:partition}
  Let $X \neq \emptyset$. A \hlnoteb{partition} of $X$ is a collection $\{A_{\alpha} \}_{\alpha \in I} \subset \mathcal{P}(X)$ such that
  \begin{enumerate}
    \item $A_\alpha \neq \emptyset$;
    \item $A_{\alpha} \cap A_{\beta} = \emptyset$ if $\alpha \neq \beta$ in $I$; and
    \item $X = \bigcup_{\alpha \in I} A_\alpha$.
  \end{enumerate}
\end{defn}

With this, we have ourselves another method to show that $\sim$ is an equivalence relation.

\begin{propo}[Characterization of An Equivalence Relation]
\label{propo:characterization_of_an_equivalence_relation}
  If $\{ A_\alpha \}_{\alpha \in I}$ is a partition of $X$ and $x \sim y \iff x, y \in A_{\alpha}$, then $\sim$ is an equivalence relation.\marginnote{The proof of this statement has been stated above.}
\end{propo}

Similar to when we defined partial orders, we can ask ourselves the following question:

\begin{eg}
  How many equivalence relations are there on the set $X = \{1, 2, 3\}$?\sidenote{By \cref{propo:characterization_of_an_equivalence_relation}, this question is equivalent to asking for the number of partitions we can create from the set $X$. The study of counting partitions is what is covered by \href{https://en.wikipedia.org/wiki/Bell_number}{Bell's Number}.}

\end{eg}

\begin{solution}
  Note that we can partition $X$ as
  \begin{equation*}
    \{ \{1\}, \{2\}, \{3\} \}, \{ \{1, 2, 3\} \},
  \end{equation*}
  and
  \begin{equation*}
    \{ \{1, 2\}, \{3\} \},
  \end{equation*}
  which we can rearrange in 3 different ways. Therefore, there are 5 different equivalence relations that we can define on $X$.
\end{solution}

\begin{eg}\label{eg:same_cardinaliy}
  Let $X$ be any set. Consider $\mathcal{P}(X)$. Define $\sim$ on $\mathcal{P}(X)$ by
  \begin{equation*}
    A \sim B \iff \exists f: A \to B
  \end{equation*}
  such that $f$ is surjective\sidenote{$\sim$ partitions $X$ into sets that have the same number of elements.}. It is easy to verify that $\sim$ is an equivalence relation.
\end{eg}

% subsection equivalence_relation (end)

% section cardinality (end)

% chapter lecture_4_sep_14th (end)

\chapter{Lecture 5 Sep 17th}%
\label{chp:lecture_5_sep_17th}
% chapter lecture_5_sep_17th

\section{Cardinality (Continued)}%
\label{sec:cardinality_continued}
% section cardinality_continued

\begin{defn}[Finite Sets]\index{Finite Sets}
\label{defn:finite_sets}
  A set $X$ is \hlnoteb{finite} if $X = \emptyset$ or if $X \sim \{ 1, 2, ..., n \}$ for some $n \in \mathbb{N}$, where $\sim$ is the equivalence relation defined in \cref{eg:same_cardinaliy}.
\end{defn}

\begin{defn}[Cardinality]\index{Cardinality}
\label{defn:cardinality}
  If $X \sim n$, we say $X$ has \hlnoteb{cardinality} $n$ and write $\abs{X} = n$. We also let $\abs{\emptyset} = 0$.
\end{defn}

\newthought{Now a good question} here is: if $n \neq m$, is $\{1, 2, ..., n \} \sim \{ 1, 2, ..., m \}$?

\begin{thm}[Pigeonhole Principle]
\index{Pigeonhole Principle}
\label{thm:pigeonhole_principle}
  The set $\{1, 2, ..., n\}$ is not equivalent to any of its proper subset.
\end{thm}

\begin{proof}
  We shall prove this by induction on $n$.\marginnote{This is a \textbf{proof by contradiction}, using the fact that we cannot find an injective function from a ``larger'' set to a ``smaller'' set.
  
  We can assume that the function $f$ is not surjective, since if the larger set is indeed equivalent to the smaller set, then it should not matter if $f$ is surjective or not. In particular, we only require that there be an injective function.}

  \noindent\textbf{Base case}: $\{1\} \nsim \emptyset$.\marginnote{\hlwarn{Requires clarification and confirmation of proof.}}

  \noindent Assume that the statement holds for $\{1, ..., k\}$. Suppose we have an injective function
  \begin{equation*}
    f : \{ 1, 2, ..., k, k + 1 \} \to \{ 1, 2, ..., k, k + 1 \}
  \end{equation*}
  that is not surjective. \\
  \noindent\textbf{Case 1}: $k + 1 \notin \Range(f)$, where $\Range(f)$ is the range of $f$. Then we have\marginnote{Note: $\restriction$ is the restriction sign.}
  \begin{equation*}
    f \restriction_{\{1, ..., k\}} : \{1, ..., k \} \to \{1, ..., k\} \setminus \{ f(k + 1) \}.
  \end{equation*}
  But $f$ is an injective function and clearly
  \begin{equation*}
    \{1, ..., k \} \setminus \{ f(k + 1) \} \subseteq \{1, ..., k\},
  \end{equation*}
  a contradition.

  \noindent\textbf{Case 2}: $k + 1 \in \Range(f)$. Then $\exists j_0 \in \{1, ..., k, k + 1 \}$ such that $f(j_0) = k + 1$, and since $f$ is not surjective, $\exists m \in \{1, ..., k\}$ such that $m \notin \Range(f)$. Then consider a new function $g: \{1, ..., k, k + 1\} \to \{1, ..., k\}$ such that
  \begin{equation*}
    g(a) = \begin{cases}
      m        & a = k + 1 \\
      f(k + 1) & a = j_0 \\
      f(a)     & a \neq j_0, k + 1
    \end{cases}
  \end{equation*}
\end{proof}

\begin{crly}[Pigeonhole Principle (Finite Case)]
\label{crly:pigeonhole_principle_finite_case}
  If the set $X$ is finite, then $X$ is not equivalent to any proper subset.
\end{crly}

\marginnote{
  Sketch of proof:

  \begin{tikzpicture}
    \node at (0, 2) {$\{1, ..., n\}$};
    \node at (2, 2) {$\{1, ..., n\}$};
    \node at (0, 0) {$X$};
    \node at (2, 0) {$A \subsetneq X$};

    \draw[<->] (0, 1.5) -- (0, 0.5) node[midway,right] {$f$};
    \node[left] at (0,1) {$f^{-1}$};
    \draw[->] (0.5,0) -- (1.5,0) node[midway,above] {$1 - 1$};
    \node[below] at (1,0) {onto};
    \draw[->] (2, 0.5) -- (2, 1.5) node[midway,right] {$f$};
    \draw[->] (0.75, 2) -- (1.25, 2);
  \end{tikzpicture}
}
\begin{ex}
  Prove \cref{crly:pigeonhole_principle_finite_case}.
\end{ex}

\begin{defn}[Infinite Sets]\index{Infinite Sets}
\label{defn:infinite_sets}
$X$ is \hlnoteb{infinite} if it is not finite.
\end{defn}

\begin{eg}
  Observe that we can construct a function $f : N \to \{ 2, 3, ... \}$ by $f(n) = n + 1$. It is clear that $f$ is a bijective funciton, and so $\mathbb{N} \sim \{2, 3, ... \}$.
\end{eg}

\begin{propo}[$\mathbb{N}$ is the Smallest Infinite Set]
\label{propo:_n_is_the_smallest_infinite_set}
  Every infinite set contains a subset $A \sim \mathbb{N}$.
\end{propo}

\begin{proof}
  Suppose $X$ is infinite. Let
  \begin{equation*}
    f : \mathcal{P}(X) \setminus \{ \emptyset \} \to X
  \end{equation*}
  such that for $S \subset X$ where $S \neq \emptyset$, $f(S) \in S$ \sidenote{\cref{axiom:zermelo_s_axiom_of_choice_v2} ahoy!}. Let $x_1 = f(X)$. Let $x_2 = f(X \setminus \{ x_1 \})$. Recursively, define
  \begin{equation*}
    x_n = f(X \setminus \{x_1, ..., x_{n - 1}\}).
  \end{equation*}
  This gives us a sequence
  \begin{equation*}
    X \supset S = \{x_1, ..., x_n, ... \}
  \end{equation*}
  which is equivalent to $\mathbb{N}$ via the map $n \mapsto x_n$.\qed
\end{proof}

\begin{crly}[Infinite Sets are Equivalent to Its Proper\- Subsets]
\label{crly:infinite_sets_are_equivalent_to_its_proper_subsets}
  Every infinite set $X$ is equivalent to a proper subset of $X$.
\end{crly}

\begin{proof}
  Given such an $X$, we construct a sequence $\{ x_n \}$ as in the previous proof. Define $f : X \to X \setminus \{ x_n \}$ by
  \begin{equation*}
    f(x) = \begin{cases}
      x_{n + 1} & x \in \{x_n\} \\
      x         & x \notin \{ x_n \}.
    \end{cases}
  \end{equation*}
  Clearly so, $f$ is injective.\qed
\end{proof}

\begin{defn}[Countable]\index{Countable}
\label{defn:countable}
  We say that a set is \hlnoteb{countable} (or \hldefn{denumerable}) is either $X$ is finite or if $X \sim \mathbb{N}$. If $X \sim \mathbb{N}$, we can say that $X$ is \hlnoteb{countably infinite} and write $\abs{X} = \abs{\mathbb{N}} = \aleph_0$.
\end{defn}

\begin{defn}[Smaller Cardinality]
\label{defn:smaller_cardinality}
  Given $2$ sets $X, Y$, we write
  \begin{equation*}
    \abs{X} \leq \abs{Y}
  \end{equation*}
  if $\exists f : X \to Y$ injective.
\end{defn}

\begin{propo}[Injectivity is Surjectivity Reversed]
\label{propo:injectivity_is_surjectivity_reversed}
  TFAE:
  \begin{enumerate}
    \item $\exists f : X \to Y$ injective
    \item $\exists g : Y \to X$ surjective
  \end{enumerate}
\end{propo}

\begin{proof}
  $(1) \implies (2)$: Define
  \begin{equation*}
    g(y) = \begin{cases}
      x   & \exists x \in X \enspace f(x) = y \\
      x_0 & \text{ any } x_0 \in X
    \end{cases}
  \end{equation*}
  Clearly $g$ is surjective.

  \noindent $(2) \implies (1)$: Since $g$ is surjective, for each $x \in X$, we have that\sidenote{The idea here is to collect the preimages into a set, and use the choice function as an injective map.}
  \begin{equation*}
    g^{-1}(\abs{x}) = \{ y \in Y : g(y) = x \} \neq \emptyset.
  \end{equation*}
  By the \hyperref[axiom:zermelo_s_axiom_of_choice_v2]{Axiom of Choice}, there exists a choice function $h : \mathcal{P}(Y) \setminus \{ \emptyset \} \to Y$ such that for each $A \subset Y$, $h(A) \in A$. Then, let $f : X \to Y$ such that
  \begin{equation*}
    f(x) = h( g^{-1}(\{ x \}) ).
  \end{equation*}
  Clearly so, $f$ is injective.\qed
\end{proof}

\begin{note}
  Note that we have $\abs{\mathbb{N}} \leq \abs{\mathbb{Q}}$, since we can define an injective function $f: \mathbb{N} \to \mathbb{Q}$ such that $f(n) = \frac{n}{1}$.

  We have also shown that $\abs{\mathbb{Q}} \leq \abs{\mathbb{N}}$ using our injective function $g : \mathbb{Q} \to \mathbb{N}$, given by
  \begin{equation*}
    g\left( \frac{m}{n} \right) = \begin{cases}
      2^m 3^n    & m > 0 \\
      1          & m = 0 \\
      5^{-m} 7^n & m < 0
    \end{cases}
  \end{equation*}
\end{note}

\newthought{Question}: Is $\abs{\mathbb{N} = \abs{\mathbb{Q}}}$? In other words, given $\abs{X} \leq \abs{Y} \land \abs{Y} \leq \abs{X}$, is $\abs{X} = \abs{Y}$?

% section cardinality_continued (end)

% chapter lecture_5_sep_17th (end)

\chapter{Lecture 6 Sep 19th}%
\label{chp:lecture_6_sep_19th}
% chapter lecture_6_sep_19th

\section{Cardinality (Continued 2)}%
\label{sec:cardinality_continued_2}
% section cardinality_continued_2

Before delving into resolving our last question in the previous lecture, note the following:

\begin{note}
  Suppose $f : X \to Y$ is bijective. Let $A \subseteq B$, then\marginnote{Prove this observation as an exercise:
  
  \begin{ex}
    Prove the note on the left.
  \end{ex}}
  \begin{equation*}
    f( B \setminus A ) = f(B) \setminus f(A).
  \end{equation*}
\end{note}

\begin{thm}[\vvimponote Cantor-Schr\"{o}der-Bernstein Theorem \-(CSB)]
\index{Cantor-Schr\"{o}der-Bernstein Theorem}
\label{thm:cantor_schroder_bernstein_theorem}
  Let $A_2 \subset A_1 \subset A_0 = A$. Assume that $A_2 \sim A_0$. Then $A_0 \sim A_1$.
\end{thm}

\begin{proof}
  \begin{marginfigure}
    \begin{tikzpicture}
      \draw[fill,color=base16-eighties-red] (0, 0) circle(2.2);
      \node at (0, 2.4) {$A_0$};
      \draw[fill,color=base16-eighties-yellow] (0, 0) circle(1.8);
      \node[color=base16-eighties-dark] at (0, 2.0) {$\boldsymbol{A_1}$};
      \draw[fill,color=base16-eighties-green] (0, 0) circle(1.4);
      \node[color=base16-eighties-dark] at (0, 1.6) {$\boldsymbol{A_2}$};
      \draw[fill,color=base16-eighties-lightblue] (0, 0) circle(1.0);
      \node[color=base16-eighties-dark] at (0, 1.2) {$\boldsymbol{A_3}$};
      \draw[fill,color=base16-eighties-blue] (0, 0) circle(0.6);
      \node[color=base16-eighties-dark] at (0, 0.8) {$\boldsymbol{A_4}$};
      \draw[fill,color=base16-eighties-light] (0, 0) circle(0.2);
      \node[color=base16-eighties-dark] at (0, 0.4) {$\boldsymbol{A_5}$};

      \draw[thick,->,color=base16-eighties-dark] (1.2,1.59705627484771405856) -- (0.9,0.8797958971132712) node[midway,left] {$\boldsymbol{f}$};
      \draw[thick,->,color=base16-eighties-dark] (1.2, 0) -- (0.4, 0) node[midway,below] {$\boldsymbol{f}$};
    \end{tikzpicture}
    \caption{The core idea of the proof for Cantor-Schr\"{o}der-Bernstein Theorem}
    \label{fig:csb_pf}
  \end{marginfigure}
  Let $\phi : A_0 \to A_2$ be bijective, by assumption. Since $A_1 \subset A_1$, let $A_3 = \phi(A_1) \subset A_2$, and since $A_2 \subset A_0$, let $A_4 = \phi(A_2) \subset A_3$. Recursively so, let
  \begin{equation*}
    A_{n + 2} = \phi( A_n )
  \end{equation*}
  Notice that
  \begin{align*}
    A_0 &= (A_0 \setminus A_1) \cup ( A_1 \setminus A_2 ) \cup ( A_2 \setminus A_3 ) \cup ( A_3 \setminus A_4 ) \cup \hdots \bigcap_{n = 0}^{\infty} A_n \\
    A_1 &= (A_1 \setminus A_2) \cup ( A_2 \setminus A_3 ) \cup ( A_3 \setminus A_4 ) \cup ( A_4 \setminus A_5 ) \cup \hdots \bigcap_{n = 1}^{\infty} A_n
  \end{align*}
  Observe that
  \begin{equation*}
    \bigcap_{n = 0}^{\infty} A_n = \bigcap_{n = 1}^{\infty} A_n.
  \end{equation*}
  \sidenote{Here, we employ the idea from \cref{fig:csb_pf}.}Define $f : A \to A_1$ by
  \begin{equation*}
    f(x) = \begin{cases}
      x       & x \in \bigcap_{n = 0}^{\infty} A_n \\
      x       & x \in A_{2k + 1} \setminus A_{2k + 2}, \; k = 0, 1, 2, ... \\
      \phi(x) & x \in A_{2k} \setminus A_{2k + 1}, \; k = 0, 1, 2, ...
    \end{cases}
  \end{equation*}
\end{proof}

\begin{crly}[Cantor-Schr\"{o}der-Bernstein Theorem - Resta\-ted]
\label{crly:cantor_schroder_bernstein_theorem_restated}
  If $A_1 \subset A \land B_1 \subset B \land A \sim B_1 \land B \sim A_1$, then $A \sim B$.\sidenote{This is equivalent to the statement
  \begin{equation*}
    \abs{A} \leq \abs{B} \land \abs{B} \leq \abs{A} \implies \abs{A} = \abs{B}.
  \end{equation*}}
\end{crly}

\begin{proof}
  By assumption, let $f : A \to B_1$ be bijective, and let $g : B \to A_1$ be bijective. Let $A_2 = g(B_1) \subseteq A_1 \subset A$ Let $A_2 = g(B_1) \subseteq A_1 \subset A$. Then the composite function $g \circ f : A \to A_2$ is bijective, and so $A \sim A_2$. By \cref{thm:cantor_schroder_bernstein_theorem}, we have
  \begin{equation*}
    A \sim A_2 \sim A_1 \sim B.
  \end{equation*}\qed
\end{proof}

\begin{eg}
  Our question from last lecture now has an answer: by \cref{thm:cantor_schroder_bernstein_theorem}, we have that $\abs{\mathbb{Q}} = \abs{\mathbb{N}}$.\sidenote{ Now that we know that they have the ssme cardinality:
 \begin{ex}
    Find a bijection between $\mathbb{Q}$ and $\mathbb{N}$.
 \end{ex} }
\end{eg}

\begin{propo}[Denumerability Check]
\label{propo:denumerability_check}
  If $X$ is infinite, then
  \begin{equation*}
    \abs{X} = \abs{\mathbb{N}} = \aleph_0 \iff \exists f : X \to \mathbb{N} \text{ bijective. }
  \end{equation*}
\end{propo}

\begin{proof}
  $(\implies)$ is immediate. For $(\impliedby)$, suppose $f : X \to \mathbb{N}$, which implies that $\abs{X} \leq \abs{\mathbb{N}}$. By \cref{propo:_n_is_the_smallest_infinite_set}, $\abs{\mathbb{N}} \leq \abs{X}$. Therefore, $\abs{X} = \abs{bbN} = \aleph_0$.\qed
\end{proof}

\begin{eg}
  $\mathbb{N} \times \mathbb{N}$ is countable. The function
  \begin{equation*}
    f : \mathbb{N} \times \mathbb{N} \times \mathbb{N} \text{ given by } f(m, n) = 2^n 3^m
  \end{equation*}
  is injective.
\end{eg}

\begin{defn}[Uncountable]\index{Uncountable}
\label{defn:uncountable}
  A set $X$ is \hlnoteb{uncountable} if it is not countable.
\end{defn}

\begin{thm}[Cantor's Diagonal Argument]
\index{Cantor's Diagonal Argument}
\label{thm:cantor_s_diagonal_argument}
  $(0, 1)$ is uncountable.
\end{thm}

\begin{proof}
  Suppose, for contradiction, that $(0, 1)$ is countable. Then we can write
  \begin{align*}
    a_1 &= .a_{11} a_{12} a_{13} ... \\
    a_2 &= .a_{21} a_{22} a_{23} ... \\
        &\vdots \\
    a_n &= .a_{n1} a_{n2} a_{n3} ...
  \end{align*}
  in $(0, 1)$. This representation is unique if we do not allow the representation to end in a string of $9$'s. Let $b \in (0, 1)$, expressed as $b = . b_1 b_2 b_3 ...$ such that
  \begin{equation*}
    b_i = \begin{cases}
      5 & a_i \neq 5 \\
      2 & a_i = 5
    \end{cases}
  \end{equation*}
  Then $b \notin (0, 1)$, a contradiction\sidenote{\hlwarn{Require more explanation.}}.\qed
\end{proof}

\begin{crly}[Uncountability of $\mathbb{R}$]
\label{crly:uncountability_of_r}
  $\mathbb{R}$ is uncountable.
\end{crly}

\begin{proof}
  Let $f : (0, 1) \to \mathbb{R}$ be given by
  \begin{equation*}
    f(x) = \tan \left( \pi x - \frac{\pi}{2} \right).
  \end{equation*}
  Clearly so, $(0, 1)$ is bijective.\qed
\end{proof}

\begin{note}
  We denote $\abs{\mathbb{R}} = c$.
\end{note}

\newthought{Question}: Given sets $X, Y$, is it always true that either\sidenote{As compare to $\leq$, $<$ implies that there is no surjection from the set on the LHS to the RHS.}
\begin{enumerate}
  \item $\abs{X} = \abs{Y}$;
  \item $\abs{X} < \abs{Y}$; or
  \item $\abs{Y} < \abs{X}$.
\end{enumerate}

% section cardinality_continued_2 (end)

% chapter lecture_6_sep_19th (end)

\chapter{Lecture 7 Sep 21st}%
\label{chp:lecture_7_sep_21st}
% chapter lecture_7_sep_21st

\section{Cardinality (Continued 3)}%
\label{sec:cardinality_continued_3}
% section cardinality_continued_3

\begin{thm}[\imponote Comparability of Cardinals]
\label{thm:comparability_of_cardinals}
  If $X$ and $Y$ are non-empty, then either
  \begin{equation*}
    \abs{X} \leq \abs{Y} \lor \abs{Y} \leq \abs{X}.
  \end{equation*}
\end{thm}

\begin{proof}
  Let
  \begin{equation*}
    S = \{ (A, B, f) \mid A \subseteq X, B \subseteq Y, f : A \to B \text{ bijective } \}.
  \end{equation*}
  Note that $S \neq \emptyset$, since $X$ and $Y$ are non-empty, and so we can have $f(a) = b$ for $A = \{a\} \subset X$ and $B = \abs{b} \subset Y$.\sidenote{We want to use the maximal element to obtain our result. To that end, we need \hyperref[axiom:zorn_s_lemma]{Zorn's Lemma}. So we need $S$ to build this up.} We order $S$ as follows: we say
  \begin{equation*}
    (A_1, B_1, f_1) \leq (A_2, B_2, f_2)
  \end{equation*}
  if
  \begin{equation*}
    A_1 \subseteq A_2, \enspace B_1 \subseteq B_2, \enspace f_1 = f_2 \restriction_{A_1}.
  \end{equation*}
  Let $C = \{ (A_\alpha, B_\alpha, f_\alpha) \}_{\alpha \in I}$ be a chain in $(S, \leq)$. Let $A_0 = \bigcup_{\alpha \in I} A_\alpha$, $B_0 = \bigcup_{\alpha \in I} B_\alpha$, and define $f_0 : A_0 \to B_0$ by
  \begin{equation*}
    f_0(x) = f_{\alpha_0} (x) \text{ if } x \in A_{\alpha_0}.
  \end{equation*}
  Now if $x \in A_{\alpha_1}$, $x \in A_{\alpha_2}$ and
  \begin{equation*}
    (A_{\alpha_1}, B_{\alpha_1}, f_{\alpha_1}) \leq (A_{\alpha_2}, B_{\alpha_2}, f_{\alpha_2}),
  \end{equation*}
  we have that
  \begin{equation*}
    f_{\alpha_1}(x) = f_{\alpha_2} \restriction_{A_{\alpha_1}} (x) = f_{\alpha_2} (x),
  \end{equation*}
  i.e. $f_0$ is well-defined.

  \noindent\underline{Claim 1: $f_0 : A_0 \to B_0$ is injective.}

  Let $x_1, x_2 \in A_0$ such that $x_1 \neq x_2$. \\
  \noindent $\implies \exists \alpha_1, \alpha_2 \in I \enspace x_1 \in A_{\alpha_1} \land x_2 \in A_{\alpha_2} \land A_{\alpha_1} \subseteq A_{\alpha_2}$ (wlog) \\
  \noindent $\implies x_1. x_2 \in A_{\alpha_2}$ \\
  \noindent $\implies ( \because f_{\alpha_2} \text{ injective } \implies f_{\alpha_2}(x_1) \neq f_{\alpha_2}(x) )$ \\
  \noindent $\implies f_0(x_1) \neq f_0(x_2) \implies f_0$ injective.

  \noindent\underline{Claim 2: $f_0 : A_0 \to B_0$ is surjective.}

  Let $y_0 \in B_0$ \\
  \noindent $\implies \exists \alpha_0 \in I \enspace y_0 \in B_{\alpha_0}$ \\
  \noindent $\implies \exists x_0 \in A_{\alpha_0} \enspace f_{\alpha_0}(x_0) = y_0$ ($\because f_{\alpha_0}$ surjective) \\
  \noindent $\implies f_0(x_0) = y_0$

  $\therefore (A_0, B_0, f_0)$ is an upper bound for $C$. Then by \hyperref[axiom:zorn_s_lemma]{Zorn's Lemma}, $(S, \leq)$ has a maximal element $(A, B, f)$.

  \noindent\underline{Case 1:} If $A = X$, then injectivity of $f$ implies $\abs{X} \leq \abs{Y}$.
  
  \noindent\underline{Case 2:} If $B = Y$, then surjectivity of $f$ implies $\abs{Y} \leq \abs{A} \leq \abs{X}$.

  \noindent\underline{Case 3:} If $A \neq X \land B \neq Y$, then $X \setminus A \neq \emptyset \land Y \setminus B \neq \emptyset$. Let $x_0 \in X \setminus A, \, y_0 \in Y \setminus B$. Let $A^* = A \cup \{x_0\}$, $B^* = B \cup \{ y_0 \}$, and $f^* : A^* \to B^*$ such that
  \begin{equation*}
    f^*(x) = \begin{cases}
      f(x) & x \in A \\
      y_0  & x = x_0
    \end{cases}
  \end{equation*}
  Then $(A, B, f) \leq (A^*, B^*, f^*)$, contradicting maximality.\qed
\end{proof}

\subsection{Cardinal Arithmetic}%
\label{sub:cardinal_arithmetic}
% subsection cardinal_arithmetic

\paragraph{Sum of Cardinals} Observe that if $X = \{ x_1, ..., x_n\}$, $Y = \{ y_1, ..., y_m\}$, and $X \cap Y = \emptyset$, then $\abs{X} = n$, $\abs{Y} = m$ and $\abs{X \cup Y} = n + m$. This motivates us to provide the following definition:

\begin{defn}[Sum of Cardinals]\index{Sum of Cardinals}
\label{defn:sum_of_cardinals}
  Assume that $X$ and $Y$ are such that $X \cap Y = \emptyset$. We define
  \begin{equation*}
    \abs{X} + \abs{Y} = \abs{X \cup Y}.
  \end{equation*}
\end{defn}

\newthought{Question}: So what about $\aleph_0 + \aleph_0$?

A thought that motivates us to give the following answer lies in the observation that: if $X$ is the set of even natural numbers and $Y$ the odd natural numbers, then $X \cup Y$ is the set of all natural numbers. All three sets are countably infinite, i.e. they have cardinality $\aleph_0$.

\newthought{Question}: What about $c + c$?

A similar motivation comes from the observation that: given $X = (0, 1)$ and $Y = (1, 2)$, we have
\begin{equation*}
  c = \abs{X} \leq \abs{X} + \abs{Y} \leq \abs{R} = c,
\end{equation*}
and so $\abs{X} = \abs{Y} = c \implies \abs{X \cup Y} = c$.

\begin{thm}[Sums of Cardinals]
\label{thm:sums_of_cardinals}
  Given sets $X$ and $Y$, if $X$ is infinite, then
  \begin{enumerate}
    \item $\abs{X} + \abs{X} = \abs{X}$
    \item $\abs{X} + \abs{Y} = \max( \abs{X}, \abs{Y} )$
  \end{enumerate}
\end{thm} \marginnote{
\begin{ex}
  Prove \cref{thm:sums_of_cardinals} as an exercise.
\end{ex}
}

\paragraph{Multiplication of Cardinals} Given
\begin{align*}
  X &= \{ x_1, x_2, ..., x_n \} \\
  Y &= \{ y_1, y_2, ..., y_m \}
\end{align*}
we have that
\begin{equation*}
  X \times Y = \{ (x_i, y_j) \mid i = 1, 2, ..., n, \, j = 1, 2, ..., m \}
\end{equation*}
and so
\begin{equation*}
  \abs{X \times Y} = nm.
\end{equation*}

\begin{defn}[Multiplication of Cardinals]\index{Multiplication of Cardinals}
\label{defn:multiplication_of_cardinals}
  Given sets $X$ and $Y$ , define
  \begin{equation*}
    \abs{X} \abs{Y} = \abs{X \times Y}.
  \end{equation*}
\end{defn}

\begin{eg}
  We have $\abs{\mathbb{N} \times \mathbb{N}} = \aleph_0$ since the function $f : \mathbb{N} \times \mathbb{N} \to \mathbb{N}$ given by
  \begin{equation*}
    f(n, m) = 2^n 3^m
  \end{equation*}
  is injective.
\end{eg}

\newthought{Question}: What about $c \cdot c$?

\begin{thm}[Multiplication of Cardinals]
\label{thm:multiplication_of_cardinals}
  If $X$ is infinite and $Y \neq \emptyset$, then
  \begin{itemize}
    \item $\abs{X \times X} = \abs{X} \implies \abs{X} \abs{X} = \abs{X}$;
    \item $\abs{X \times Y} = \max(\abs{X}, \abs{Y})$.
  \end{itemize}
\end{thm}\marginnote{
\begin{ex}
  Prove \cref{thm:multiplication_of_cardinals} as an exercise.
\end{ex}}

% subsection cardinal_arithmetic (end)

% section cardinality_continued_3 (end)

% chapter lecture_7_sep_21st (end)

\chapter{Lecture 8 Sep 24th}%
\label{chp:lecture_8_sep_24th}
% chapter lecture_8_sep_24th

\section{Cardinality (Continued 4)}%
\label{sec:cardinality_continued_4}
% section cardinality_continued_4

\subsection{Cardinal Arithmetic (Continued)}%
\label{sub:cardinal_arithmetic_continued}
% subsection cardinal_arithmetic_continued

\paragraph{Exponentiation of Cardinals} Recall if $\{Y_x\}_{x \in X}$ is a collection of non-empty sets, we have\sidenote{This should remind you of \cref{axiom:zermelo_s_axiom_of_choice_v2}}
\begin{equation*}
  \prod_{x \in X} Y_x = \{ f : X \to \bigcup_{x \in X} Y_x \mid f(x) \in Y_x \}.
\end{equation*}
Now if $Y = Y_x$ for all $x \in X$, we have
\begin{equation*}
  Y^X = \prod_{x \in X} = \{ f : X \to Y \}.
\end{equation*}

\begin{eg}
  Given
  \begin{equation*}
    Y = \{ 1, ..., m \} \quad X = \{ 1, ..., n \}
  \end{equation*}
  we have
  \begin{equation*}
    Y^X = \{ f : \{ 1, ..., n \} \to \{ 1, ..., m \} \}.
  \end{equation*}
  Observe that $Y^X$ is similar to $Y^n$. So $\abs{Y^X} = m^n$.\sidenote{\hlwarn{Need better explanation.}}
\end{eg}

\begin{defn}[Exponentiation of Cardinals]\index{Exponentiation of Cardinals}
\label{defn:exponentiation_of_cardinals}
  Given sets $X$ and $Y$, define
  \begin{equation*}
    \abs{Y}^{\abs{X}} := \abs{Y^X}.
  \end{equation*}
\end{defn}

\begin{thm}[Exponentiation of Cardinals]
\label{thm:exponentiation_of_cardinals}
  If $X, Y, Z$ are non-empty sets, then
  \begin{itemize}
    \item $\abs{Y}^{\abs{X}} \cdot \abs{Y}^{\abs{Z}} = \abs{Y}^{\abs{X} + \abs{Z}}$;
    \item $\left( \abs{Y}^{\abs{X}} \right)^{\abs{Z}} = \abs{Y}^{\abs{X} \cdot \abs{Z}}$.
  \end{itemize}
\end{thm}\marginnote{
\begin{ex}
  Prove \cref{thm:exponentiation_of_cardinals}.
\end{ex}}

\begin{thm}[$2^{\aleph_0} = c$]
\label{thm:2_aleph_0_c}
We have that $2^{\aleph_0} = c$.
\end{thm}

\begin{proof}\marginnote{This requires closer studying.}
  Note that $2^{\aleph_0} = \abs{\{ 0, 1\}^{\mathbb{N}} }$, where\sidenote{Explain 2nd equality.}
  \begin{equation*}
    \abs{ \{0, 1 \}^{\mathbb{N}} } = \abs{ \{ f : \mathbb{N} \to \{0, 1\} } = \abs{ \{ \{ a_n \}_{n = 1}^{\infty} \mid a_i = 0, 1 \} }
  \end{equation*}

  Given a sequence $\{ a_n \} \in \{ 0, 1 \}^{\mathbb{N}}$, define $\phi : \{ 0, 1 \}^{\mathbb{N}} \to (0, 1)$ such that\sidenote{This is a base 3 representation (\hlwarn{of what?})}
  \begin{equation*}
    \phi\left( \{ a_n \} \right) := \sum_{i=1}^{\infty} \frac{a_n}{3^n}.
  \end{equation*}
  which is injective since there are no trailing $2$'s. Therefore $2^{\aleph_0} \leq c$.

  Given $\alpha \in (0, 1)$, let\sidenote{This is a base 2 representation.}
  \begin{equation*}
    \alpha = \sum_{n=1}^{\infty} \frac{b_n}{2^n},
  \end{equation*}
  where $b_n = 0, 1$. Let $\psi : (0, 1) \to \{0, 1\}^{\mathbb{N}}$ such that
  \begin{equation*}
    \psi(\alpha) = \psi\left( \sum_{i=1}^{\infty} \frac{b_n}{2^n} \right) = \{ b_n \}
  \end{equation*}
  Then $\psi$ is injective, and so $c \leq 2^{\aleph_0}$. Thus $2^{\aleph_0} = c$ as required.\qed
\end{proof}

\begin{eg}
  Find $\abs{\aleph_0^{\aleph_0}}$ and $c^{\aleph_0}$.
\end{eg}

\begin{solution}
  We have that
  \begin{equation*}
    c = 2^{\aleph_0} \leq \aleph_0^{\aleph_0} \leq c^{\aleph_0} = \left( 2^{\aleph_0} \right)^{\aleph_0} = 2^{\aleph_0 \cdot \aleph_0} = 2^{\aleph_0} = c
  \end{equation*}
\end{solution}

\begin{eg}
  Show $\abs{\mathcal{P}(X)} = 2^{\abs{X}} = \abs{2^X}$.
\end{eg}

\begin{solution}
  Given $f : X \to \{ 0, 1 \}$, let\sidenote{$A$ is a collection of all $x$'s that gets mapped to $f$.}
  \begin{equation*}
    A = \{ x \in X \mid f(x) = 1 \} \subset X.
  \end{equation*}
  Define $\Gamma : 2^X \to \mathcal{P}(X)$ by
  \begin{equation*}
    \Gamma(f) = f^{-1}(\abs{1})
  \end{equation*}
  $\Gamma$ is injective\sidenote{\hlwarn{Why?}}.

  Conversely, given $A \subset X$, define the characteristic function
  \begin{equation*}
    \chi_A(x) = \begin{cases}
      1 & x \in A \\
      0 & x \notin A
    \end{cases} \in 2^X.
  \end{equation*}
  Then define $\Phi : P(A) \to 2^X$ such that
  \begin{equation*}
    \Phi(A) = \chi_A.
  \end{equation*}
  Clearly so, $\Phi$ is injective.\qed
\end{solution}

\begin{thm}[Russell's Paradox]
\index{Russell's Paradox}
\label{thm:russell_s_paradox}
  For any $X$, we have $\abs{X} < \abs{\mathcal{P}(X)} = 2^{\abs{X}}$.
\end{thm}

\begin{proof}
  Let $f : X \to \mathcal{P}(X)$ be $f(X) = \{ x \}$. Clearly, $f$ is injective, and so $\abs{X} < \abs{\mathcal{P}(X)}$.

  \noindent\underline{Claim: $\nexists g : X \to \mathcal{P}(X)$ surjective}.

  Suppose not. Let\sidenote{By the \hlnotea{Bounded Separation Axiom} (see ZF Set Theory), this is a set, and since it is a subset of $X$, it is a valid element of $\mathcal{P}(X)$. Thus, we can consider such a set.}
  \begin{equation*}
     A = \{ x \in X \mid x \notin g(x) \}
  \end{equation*}
  Pick $x_0 \in X$ with $g(x_0) = A$. Now if $x_0 \in A$, then $x_0 \in g(x_0)$, but this implies that $x \notin A$, a contradiction.

  So $x_0 \notin A$, i.e. $x \notin g(x_0)$, which in turn implies that $x \in A$, yet another contradiction. Therefore such a function $g$ cannot exist, as claimed.

  Therefore, we have $\abs{X} < \abs{\mathcal{P}(X)}$ as required.\qed
\end{proof}

\newthought{Question}: Is there anything between $\aleph_0$ and $c$?

\begin{axiom}[Continuum Hypothesis]
\index{Continuum Hypothesis}
\label{axiom:continuum_hypothesis}
  If $\aleph_0 \leq \gamma \leq c$, then either $\gamma = \aleph_0$ or $\gamma = c$.
\end{axiom}

\begin{axiom}[Generalized Continuum Hypothesis]
\index{Generalized Continuum Hypothesis}
\label{axiom:generalized_continuum_hypothesis}
  If $\abs{X} \leq \gamma \leq 2^{\abs{X}}$, then either $\gamma = \abs{X}$ or $\gamma = 2^{\abs{X}}$.
\end{axiom}

In this course, we shall assume that the Continuum Hypothesis is true.

% subsection cardinal_arithmetic_continued (end)

% section cardinality_continued_4 (end)

% chapter lecture_8_sep_24th (end)

\chapter{Lecture 9 Sep 26th}%
\label{chp:lecture_9_sep_26th}
% chapter lecture_9_sep_26th

\section{Introduction to Metric Spaces}%
\label{sec:introduction_to_metric_spaces}
% section introduction_to_metric_spaces

\begin{defn}[Metric \& Metric Space]\index{Metric}\index{Metric Space}
\label{defn:metric_n_metric_space}
  Given a set $X$, a \hlnoteb{metric} on $X$ is a function $d : X \times X \to \mathbb{R}$ such that\marginnote{
  \begin{remark}
    A metric is an abstract notion of distance.
  \end{remark}
  }
  \begin{enumerate}
    \item (\hlnotea{positive definiteness}) $d(x, y) \geq 0$ and $d(x, y) = 0 \iff x = y$;
    \item (\hlnotea{symmetry}) $d(x, y) = d(y, x)$; and
    \item (\hlnotea{triangle inequality}) $d(x, y) \leq d(x, z) + d(z, y)$.
  \end{enumerate}
  The pair $(X, d)$ is called a \hlnoteb{metric space}.
\end{defn}

\begin{eg}[Standard Metric on $\mathbb{R}$]\label{eg:standard_metric_on_r}\index{Standard Metric}
  Let $X = \mathbb{R}$, and let $d(x, y) = \abs{ x - y }$.

  Clearly so, the first 2 criterias are satisfied:
  \begin{itemize}
    \item $\abs{\cdot} \geq 0$ and $\abs{x - y} = 0 \iff x = y$; and
      \item $\abs{x - y} = \abs{ y - x }$.
  \end{itemize}
  The triangle inequality property is the usual triangle inequality of the absolute value function, i.e.
  \begin{equation*}
    \abs{ x - y } \leq \abs{ x } + \abs{ y }.
  \end{equation*}
\end{eg}

\newthought{Question}: For an arbitrary set $X$, can we define a metric on $X$? The following example shows that we can,

\begin{eg}[Discrete Metric]\label{eg:discrete_metric}\index{Discrete Metric}
  Let $X$ be any set. We can simply define
  \begin{equation*}
    d(x, y) = \begin{cases}
      1 & x \neq y \\
      0 & x = y
    \end{cases}
  \end{equation*}
  This metric clearly satisfies all 3 criterias of being a metric:
  \begin{itemize}
    \item $d : X \times X \to \{ 0, 1 \}$ and so $d(x, y) \geq 0$, and by definition, we have $d(x, y) = 0 \iff x = y$;
    \item By definition, $d(x, y) = d(y, x)$ as it does not matter how the pair is ordered; and
    \item Since $d(x, y) \geq 0$, we have that $d(x, y) \leq d(x, z) + d(y, z)$.
  \end{itemize}
\end{eg}

\begin{eg}[Euclidean Metric / 2-metric on $\mathbb{R}^n$]\label{eg:euclidean_metric}\index{Euclidean Metric}\index{2-metric}
  Let $X = \mathbb{R}^n$. Let $\vec{x} = \{ x_1, x_2, ..., x_n \}$ and $\vec{y} = \{ y_1, y_2, ..., y_n \}$. Define
  \begin{equation*}
    d_2(\vec{x}, \vec{y}) = \sqrt{ \sum_{i=1}^{n} ( x_i - y_i )^2 }
  \end{equation*}
  Note that in $\mathbb{R}^2$, this is our regular (Euclidean) distance between two points.

  It is not difficult to see that $d_2$ satisfies the first 2 criterion to being a metric:
  \begin{marginfigure}
    \begin{tikzpicture}
      \draw[->] (-.5,0) -- (2.5,0) node[right] {$x$};
      \draw[->] (0,-.5) -- (0,2.5) node[above] {$y$};
      \draw[-latex'] (0, 0) -- (2, 2) node[midway,left] {$\vec{x} + \vec{y}$};
      \draw[-latex'] (0, 0) -- (1.5, 0.7) node[midway,below] {$\vec{x}$};
      \draw[-latex'] (1.5, 0.7) -- (2, 2) node[midway,right] {$\vec{y}$};
    \end{tikzpicture}
    \caption{A visualization of the triangle inequality in $\mathbb{R}^2$.}
    \label{fig:visualize_triangle_inequality_r2}
  \end{marginfigure}
  \begin{itemize}
    \item $d_2$ is the square root of the sum of squares, and so $d_2(\vec{x}, \vec{y}) \geq 0$ for any $\vec{x}, \vec{y} \in \mathbb{R}^n$, and $d_2(\vec{x}, \vec{y}) = 0 \iff \forall i \in \{1, ..., n\} \; x_i = y_i \iff \vec{x} = \vec{y}$;
      \item Since $(x_i - y_i)^2 = (y_i - x_i)^2$ for any $x_i, y_i \in \mathbb{R}$, we have that $d_2(\vec{x}, \vec{y}) = d_2(\vec{y}, \vec{x})$.
  \end{itemize}
  However, it is not immediately clear that $d_2$ satisfies the triangle inequality criterion, especially if $n \geq 3$. If $n = 2$, heuristically, the triangle inequality simply tells that the length of any one side of a triangle is less than or equal to the sum of the other two, e.g. \cref{fig:visualize_triangle_inequality_r2}.\marginnote{
  \begin{remark}
    Many of the important examples of metric spaces are vector spaces with an abstract length function, or \hlnotea{norm}.
  \end{remark}
  }
\end{eg}

\begin{defn}[Norm \& Normed Linear Space]\index{Norm}\index{Normed Linear Space}
\label{defn:norm_n_normed_linear_space}
  Given a vector space $V$ (usually over $\mathbb{R}$), a \hlnoteb{norm} on $V$ is a function
  \begin{equation*}
    \norm{\cdot} : V \to \mathbb{R}
  \end{equation*}
  such that
  \begin{enumerate}
    \item (\hlnotea{positive definiteness}) $\norm{v} \geq 0$ and $\norm{v} = 0 \iff v = 0$;
    \item (\hlnotea{scalar multiplication}) $\norm{\alpha \cdot v} = \abs{\alpha} \norm{v}$; and
    \item (\hlnotea{triangle inequality}) $\norm{v + w} \leq \norm{v} + \norm{w}$.
  \end{enumerate}
  The pair $(V, \norm{\cdot})$ is called a \hlnoteb{normed linear space}.
\end{defn}

\begin{remark}
  Given a normed linear space $(V, \norm{\cdot})$, a natural metric, $d_{\norm{\cdot}}$, on $V$ induced by $\norm{\cdot}$ can be defined as\marginnote{
  \begin{ex}\label{ex:natural_metric_on_a_normed_linear_space}
    Prove that $d_{\norm{\cdot}}$ is indeed a metric.
  \end{ex}}
  \begin{equation*}
    d_{\norm{\cdot}} (x, y) = \norm{x - y}.
  \end{equation*}
\end{remark}

\begin{proof}[\cref{ex:natural_metric_on_a_normed_linear_space}]
  \begin{enumerate}
    \item (positive definiteness) It is clear from the definition of a norm that $d_{\norm{\cdot}}(x, y) = \norm{x - y} \geq 0$, and $d_{\norm{\cdot}}(x, y) = 0 \iff x - y = 0 \iff x = y$.
    \item (symmetry) Symmetry follows simply from definition, as $\norm{x - y} = \norm{y - x}$.
    \item (triangle inequality) For $x, y, z \in V$, we have
      \begin{align*}
        d_{\norm{\cdot}} (x, y) &= \norm{x - y} = \norm{ x - z + z - y } \\
                                &\leq \norm{x - z} + \norm{ z - y } \enspace \because \text{ triangle inequality of norms } \\
                                &= \norm{x - z} + \norm{ y - z } \enspace \because \text{ symmetry } \\
                                &= d_{\norm{\cdot}}(x, z) + d_{\norm{\cdot}}(y, z)
      \end{align*}
  \end{enumerate}\qed
\end{proof}

\begin{eg}[Euclidean Norm]\label{eg:euclidean_norm}
  Let $X = \mathbb{R}^n$, and $\vec{x} = ( x_1, ..., x_n ) \in \mathbb{R}^2$. Define $\norm{\cdot}_2$ such that
  \begin{equation*}
    \norm{(x_1, ..., x_n)}_2 = \sqrt{ \sum_{i=1}^{n} x_i^2 }
  \end{equation*}
  From \cref{eg:euclidean_metric}, we are given the triangle inequality property, in which we have yet to prove. Positive definiteness is clear. For scalar multiplication, let $\vec{x} = { x_1, ..., x_n }$, and notice that
  \begin{equation*}
    \norm{\alpha \cdot \vec{x}}_2 = \sqrt{ \sum_{i=1}^{n} (\alpha x_i)^2 } = \sqrt{ \alpha^2 \sum_{i=1}^{n} x_i^2 } = \abs{\alpha} \sqrt{ \sum_{i=1}^{n} x_i^2 } = \abs{\alpha} \norm{\vec{x}}_2.
  \end{equation*}
  Thus $\norm{\cdot}_2$ is indeed a norm. We call $\norm\cdot_2$ the \hldefn{2-norm} or the \hldefn{Euclidean norm}.

  We observe that, in comparison with \cref{eg:euclidean_metric}, we have that
  \begin{equation*}
    d_2(\vec{x}, \vec{y}) = \norm{\vec{x} - \vec{y}}_2.
  \end{equation*}
\end{eg}

\begin{eg}[1-norm]\label{eg:1_norm}
  Let $X = \mathbb{R}^n$, and $\vec{x} = ( x_1, ..., x_n )$. Define
  \begin{equation*}
    \norm{\vec{x}}_1 := \sum_{i = 1}^{n} \abs{x_i}.
  \end{equation*}
  Clearly so, $\norm\cdot_1$ is a norm:
  \begin{itemize}
    \item (\textbf{positive definiteness}) This is true by the absolute value function, i.e. every $\abs{x_i} \geq 0$, and so the sum over these $x_i$'s is also non-negative, and $\sum_{i=1}^{n} \abs{x_i} = 0 \iff \forall i \in \{ 1, ..., n\} \; x_i = 0 \iff \vec{x} = 0$.
    \item (\textbf{scalar multiplication}) This uses a similar argument as in the previous example.
    \item (\textbf{triangle inequality}) This is true by, again, the triangle inequality on absolute values.
  \end{itemize}
  We call $\norm\cdot_1$ the \hldefn{1-norm}.

  Thus, we can define
  \begin{equation*}
    d_1(\vec{x}, \vec{y}) = \norm{\vec{x} - \vec{y}}_1
  \end{equation*}
  and it can easily be verified that $d_1$ is indeed a metric.
\end{eg}

\begin{eg}
  Let $X = \mathbb{R}^n$ and $\vec{x} = ( x_1, ..., x_n )$. Define
  \begin{equation*}
    \norm{\vec{x}}_\infty = \max \{ \abs{x_i} \}
  \end{equation*}
  Again, it is easy to see that $\norm\cdot_\infty$ is a norm;
  \begin{itemize}
    \item (\textbf{positive definiteness}) $\because \forall i \in \{ 1, ..., n \} \; \abs{x_i} \geq 0 \implies \max \{ \abs{x_i \} \geq 0}$ and $\max \{ \abs{x_i} \} = 0 \iff x_i = 0 \iff \vec{x} = 0$.
    \item (\textbf{scalar multiplication}) Notice that
      \begin{equation*}
        \norm{\alpha \cdot \vec{x}}_\infty = \max \{ \abs{\alpha x_i} \} = \abs{\alpha} \max \{ \abs{x_i} \} = \abs{\alpha} \norm{\vec{x}}_\infty.
      \end{equation*}
    \item (\textbf{triangle inequality}) This is once again true by the triangle inequality on the absolute value function, i.e.
      \begin{gather*}
        \because \forall i \in \{ 1, ..., n \} \enspace \abs{x_i + y_i} \leq \abs{x_i} + \abs{y_i} \\
        \max \{ \abs{x_i + y_i} \} \leq \max \{ \abs{x_i} \} + \max \{ \abs{y_i} \}.
      \end{gather*}
  \end{itemize}
  We can then define
  \begin{equation*}
    d_\infty(\vec{x}, \vec{y}) = \max \{ \abs{x_i - y_i} \},
  \end{equation*}
  which we can easily verify that it is indeed a metric\sidenote{Symmetry holds by the property of the absolute value function.}.
\end{eg}

\newthought{Here's} an interesting notion: let
\begin{equation*}
  S_i = \{ \vec{x} \in \mathbb{R}^2 \mid \norm{\vec{x}}_i = 1 \}, \quad i = 1, 2, \infty
\end{equation*}
Notice that we would then have the following graph:\begin{figure}[h]
  \begin{center}
  \begin{tikzpicture}
    \draw[-latex'] (-3, 0) -- (3, 0);
    \draw[-latex'] (0, -3) -- (0, 3);
    \draw[-] (2, 2) -- (-2, 2) -- (-2, -2) -- (2, -2) -- (2, 2);
    \draw (0, 0) circle(2);
    \draw[-] (2, 0) -- (0, 2) -- (-2, 0) -- (0, -2) -- (2,0);
    \node[above,right] at (2,2) {$S_\infty$};
    \node[above,right] at (1.4142135623730950488,1.4142135623730950488) {$S_2$};
    \node[above,right] at (1, 1) {$S_1$};
  \end{tikzpicture}
  \end{center}
  \caption{Unit ball depending on $\norm{\vec{x}}_i$}
  \label{fig:unit_ball_depending_on_norm}
\end{figure}
In fact, it is true that if we let $i \in \mathbb{N} \setminus \{ 0 \}$, as suggested by \cref{fig:unit_ball_depending_on_norm}, we would see that the ``diamond'' would grow into a ``circle'' as in $S_2$, and as $i \geq 3$, the unit ball will expand and approach the ``square'', which is $S_\infty$.

Another observation that we can make is if we can show that a set is open for a ``smaller'' $S_i$, then the same set is open for any $S_j$ for $j \geq i$.\marginnote{
Note that if we allow for $0 < i < 1$, then we would have a graph that looks like the following, which is a convex graph, i.e. we cannot create well-defined norms.}
\begin{marginfigure}
\begin{center}
\begin{tikzpicture}
  \draw[-latex'] (-2,0) -- (2, 0);
  \draw[-latex'] (0,-2) -- (0, 2);
  \draw (0, 1) arc (180:270:1);
  \draw (1, 0) arc (90:180:1);
  \draw (0, -1) arc (0:90:1);
  \draw (-1, 0) arc (270:360:1);
\end{tikzpicture}
\end{center}
\caption{$\norm\cdot_p$ for $0 < p < 1$}
\end{marginfigure}

If we apply these norms into metrics, we have
\begin{equation*}
  d_\infty \leq d_2 \leq d_1
\end{equation*}
where we say that $d_\infty$ is the \hlnotea{least sensitive}, and $d_1$ being the \hlnotea{most sensitive}\sidenote{For sufficently close points, we see that $d_\infty$ would reflect the least change, while we can see change in $d_1$ for every two points that we take.}.

\begin{eg}
  For $1 < p < \infty$, define on $\mathbb{R}^n$
  \begin{equation*}
    \norm{\vec{x}}_p = \left( \sum_{i=1}^{n} \abs{x_i}^p \right)^\frac{1}{p}
  \end{equation*}
  Continuing with the same idea as in previous examples, we can let
  \begin{equation*}
    d_p(\vec{x}, \vec{y}) = \left( \sum_{i=1}^{n} \abs{x_i - y_i}^p \right)^\frac{1}{p}
  \end{equation*}
  In the next lecture, we will go into proving that this is indeed a norm, and so we can define a metric using this norm.
\end{eg}

% section introduction_to_metric_spaces (end)

% chapter lecture_9_sep_26th (end)

\chapter{Lecture 10 Sep 28th}%
\label{chp:lecture_10_sep_28th}
% chapter lecture_10_sep_28th

\section{Introduction to Metric Spaces (Continued)}%
\label{sec:introduction_to_metric_spaces_continued}
% section introduction_to_metric_spaces_continued

\begin{defn}[$\norm\cdot_p$-norm]\index{$\norm\cdot_p$-norm}
\label{defn:p_norm}
  Given $\vec{x} = ( x_1, ..., x_n ) \in \mathbb{R}^n$, we define, for $1 < p < \infty$, the $\norm\cdot_p$-norm to be
  \begin{equation*}
    \norm{\vec{x}}_p = \left( \sum_{i=1}^{n} \abs{x_i}^p \right)^{\frac{1}{p}}
  \end{equation*}
\end{defn}

We asked the question: why is $\norm\cdot_p$ a norm?

\begin{lemma}
\label{lemma:lemma_for_holders}
  If $1 < p < \infty$,
  \begin{equation*}
    \frac{1}{p} + \frac{1}{q} = 1,
  \end{equation*}
  and if $\alpha, beta > 0$, then
  \begin{equation*}
    \alpha \cdot \beta \leq \frac{\alpha^p}{p} + \frac{\beta^q}{q}.
  \end{equation*}
\end{lemma}

\begin{proof}
  Motivated by \cref{fig:lemma_for_holders_idea}, using notions from calculus, we have from calculus,
  \begin{marginfigure}
    \begin{tikzpicture}
    \draw[->] (-0.5,0) -- (3,0) node[right] {$t$};
    \draw[->] (0,-0.5) -- (0,3) node[above] {$u$};
    \draw[fill=base16-eighties-blue!20!base16-eighties-dark] (1.5,2.25) -- (0, 2.25) -- (0, 0) -- (1.5, 0) -- cycle;
    \draw[domain=0:1.73205080756887729352] plot ( {\x},{\x^2} ) node[above] {$u = t^{ p - 1 }$};
    \draw[dotted] (1.5, 2.25) -- (0, 2.25) node[left] {$\beta$};
    \draw[dotted] (1.5, 2.25) -- (1.5, 0) node[below] {$\alpha$};
    \end{tikzpicture}
    \caption{Motivation for \cref{lemma:lemma_for_holders}.}
    \label{fig:lemma_for_holders_idea}
  \end{marginfigure}
  \begin{align*}
    \alpha \beta &\leq \int_{0}^{\alpha} t^{p - 1} \dif{t} + \int_{0}^{\beta} u^{q - 1} \dif{u} \\
                 &= \frac{t^p}{p} \at{0}{\alpha} + \frac{u^q}{q} \at{0}{\beta} \\
                 &= \frac{\alpha^p}{p} + \frac{\beta^q}{q},
  \end{align*}
  where we note that
  \begin{gather*}
    \frac{1}{p} + \frac{1}{q} = 1 \\
    \frac{q}{p} = q - 1 \\
    \frac{p}{q} = p - 1 \\
    1 = (p - 1)(q - 1)
  \end{gather*}\qed
\end{proof}

\begin{thm}[H\"{o}lder's Inequality]
\index{H\"{o}lder's Inequality}
\label{thm:holder_s_inequality}
  For $1 < p < \infty$, let $\frac{1}{p} + \frac{1}{q} = 1$ \sidenote{We also call $q$ the conjugate of $p$.}. Let
  \begin{equation*}
    \vec{x} = ( x_1, ..., x_n ), \quad \vec{y} = ( y_1, ..., y_n ).
  \end{equation*}
  Then
  \begin{equation*}
    \sum_{i=1}^{n}  \abs{ x_i y_i } \leq \left( \sum_{i=1}^{n} \abs{x_i}^p \right)^{\frac{1}{p}} \cdot \left( \sum_{i=1}^{n} \abs{y_i}^q \right)^{\frac{1}{q}}.
  \end{equation*}
\end{thm}

\begin{note}
  Note that $p = 2$ is just the \hldefn{Cauchy-Schwarz Inequality}:
  \begin{gather*}
    \sum_{i=1}^{n} \abs{x_i y_i} \leq \left( \sum_{i=1}^{n} \abs{x_i}^2 \right)^{ \frac{1}{2} } \cdot \left( \sum_{i=1}^{n} \abs{y_i}^2 \right)^{\frac{1}{2}} \implies \\
    \left( \sum_{i=1}^{n} \abs{x_i y_i} \right)^2 \leq \left( \sum_{i=1}^{n} \abs{x_i}^2 \right) \cdot \left( \sum_{i=1}^{n} \abs{y_i}^2 \right)
  \end{gather*}
\end{note}

\begin{proof}
  Since if either $\vec{x}$ or $\vec{y}$ is zero, then we have that the inequality is trivially true, we can suppose that $\vec{x} \neq 0 \neq \vec{y}$. Now, note that for $\alpha, \beta \neq 0$, we have that\sidenote{In the second inequality, notice that we can easily get back to the first equation by dividing both sides by $\alpha \beta$.}
  \begin{gather*}
    \sum_{i=1}^{n}  \abs{ x_i y_i } \leq \left( \sum_{i=1}^{n} \abs{x_i}^p \right)^{\frac{1}{p}} \cdot \left( \sum_{i=1}^{n} \abs{y_i}^q \right)^{\frac{1}{q}} \\
    \Updownarrow \\
    \sum_{i=1}^{n}  \abs{ \alpha x_i \cdot \beta y_i } \leq \left( \sum_{i=1}^{n} \abs{\alpha x_i}^p \right)^{\frac{1}{p}} \cdot \left( \sum_{i=1}^{n} \abs{\beta y_i}^q \right)^{\frac{1}{q}}.
  \end{gather*}
  So we can assume that
  \begin{equation}\label{eq:holders_inequality_core}
    \left( \sum_{i=1}^{n} \abs{x_i}^p \right)^\frac{1}{p} = 1 = \left( \sum_{i=1}^{n} \abs{y_i}^q \right)^\frac{1}{q},
  \end{equation}
  and if not, we can simply choose $\alpha, \beta \neq 0$ to scale these values to become one. By \cref{lemma:lemma_for_holders}, we have
  \begin{equation*}
    \abs{x_i y_i} \leq \frac{\abs{x_i}^p}{p} + \frac{\abs{y_i}^q}{q}.
  \end{equation*}
  Hence
  \begin{align*}
    \sum_{i=1}^{n} \abs{x_i y_i} &\leq \sum_{i=1}^{n} \frac{\abs{x_i}^p}{p} + \sum_{i=1}^{n} \frac{\abs{y_i}^q}{q} = \frac{1}{p} + \frac{1}{q} \quad \because \cref{eq:holders_inequality_core} \\
                                 &= 1 = \left( \sum_{i=1}^{n} \abs{x_i}^p \right)^\frac{1}{p} \cdot \left( \sum_{i=1}^{n} \abs{y_i}^q \right)^\frac{1}{q}
  \end{align*}
  as required.\qed
\end{proof}

We are now ready to prove our long-awaited result.

\begin{thm}[Minkowski's Inequality]
\index{Minkowski's Inequality}
\label{thm:minkowshi_s_inequality}
  Let $1 < p < \infty$. If
  \begin{equation*}
    \vec{x} = ( x_1, ..., x_n ), \quad \vec{y} = ( y_1, ..., y_n )
  \end{equation*}
  in $\mathbb{R}^n$, then
  \begin{equation*}
    \left( \sum_{i=1}^{n} \abs{x_i + y_i}^p \right)^\frac{1}{p} \leq \left( \sum_{i=1}^{n} \abs{x_i}^p \right)^\frac{1}{p} + \left( \sum_{i=1}^{n} \abs{y_i}^p \right)^\frac{1}{p},
  \end{equation*}
  i.e.
  \begin{equation*}
    \norm{\vec{x} + \vec{y}}_p \leq \norm{\vec{x}}_p + \norm{\vec{y}}_p.
  \end{equation*}
\end{thm}

\begin{proof}
  Let
  \begin{equation*}
    \frac{1}{p} + \frac{1}{q} = 1.
  \end{equation*}
  Once again, we may assume that $\vec{x} \neq 0 \neq \vec{y}$, as otherwise the inequality is true trivially so. Now, notice that
  \begin{align*}
    \sum_{i=1}^{n} \abs{x_i + y_i}^p &= \sum_{i=1}^{n} \abs{x_i + y_i} \abs{x_i + y_i}^{p - 1} \\
                                     &\leq \sum_{i=1}^{n} \abs{x_i} \abs{x_i + y_i}^{p - 1} + \sum_{i=1}^{n} \abs{y_i}\abs{x_i + y_i} \enspace \because \overset{\text{triangle}}{\text{inequality}} \\
                                     &\leq \left( \sum_{i=1}^{n} \abs{x_i}^p \right)^\frac{1}{p} \left( \sum_{i=1}^{n} \abs{x_i + y_i}^{(p - 1)q} \right)^\frac{1}{q} \\
                                     & \qquad\quad + \left( \sum_{i=1}^{n} \abs{y_i}^p \right)^\frac{1}{p} \left( \sum_{i=1}^{n} \abs{x_i + y_i}^{(p - 1)q} \right)^\frac{1}{q}
  \end{align*}
  where the last step is by \hyperref[thm:holder_s_inequality]{H\"{o}lder's Inequality}. Note that $\frac{1}{p} + \frac{1}{q} = 1 \implies p = q(p - 1)$. Thus
  \begin{gather*}
    \sum_{i=1}^{n} \abs{x_i + y_i}^p \leq \left[ \left( \sum_{i=1}^{n} \abs{x_i}^p \right)^\frac{1}{p} + \left( \sum_{i=1}^{n} \abs{y_i}^p \right)^{\frac{1}{p}} \right] \cdot \left( \sum_{i=1}^{n} \abs{x_i + y_i}^p \right)^\frac{1}{q} \\
    \implies \left( \sum_{i=1}^{n} \abs{x_i + y_i}^p \right)^{1 - \frac{1}{q}} \leq \left( \sum_{i=1}^{n} \abs{x_i}^p \right)^\frac{1}{p} + \left( \sum_{i=1}^{n} \abs{y_i}^p \right)^\frac{1}{p}
  \end{gather*}\qed
\end{proof}

\begin{note}
  With this we have that $\norm\cdot_p$ satisfies the triangle inequaltiy condition, and so $\norm\cdot_p$ is a norm on $\mathbb{R}^n$.
\end{note}

\begin{note}
  Given $1 \leq p \leq q < \infty$, we have\sidenote{For a visual representation of this result, see \cref{fig:unit_ball_depending_on_norm}.}
  \begin{equation*}
    \norm\cdot_\infty \leq \norm\cdot_q \leq \norm\cdot_p \leq \norm\cdot_1.
  \end{equation*}

  \begin{proof}
    It is quite clear that $\forall p \geq 1$,
    \begin{equation*}
      \norm\cdot_\infty = \max \{ \abs{\cdot} \} \leq \left( \sum \abs{\cdot}^p \right)^{\frac{1}{p}} =\norm\cdot_p.
    \end{equation*}
    For $1 \leq p \leq q < \infty$, consider \hyperref[thm:holder_s_inequality]{Holder's Inequality}, where we have
    \begin{equation*}
      \sum_{i=1}^{n} \abs{a_i} \abs{b_i} \leq \left( \sum_{i=1}^{n} \abs{a_i}^r \right)^{\frac{1}{r}} \cdot \left( \sum_{i=1}^{n} \abs{b_i}^{\frac{r}{r - 1}} \right)^{1 - \frac{1}{r}}.
    \end{equation*}
    Let $\abs{a_i} = \abs{x_i}^p$, $\abs{b_i} = 1$ and $r = \frac{q}{p} \geq 1$ \sidenote{Note that this is true by $p \leq q$.}. Then we have
    \begin{align*}
      \sum_{i=1}^{n} \abs{x_i}^p &\leq \left( \sum_{i=1}^{n} \abs{x_i}^q \right)^{\frac{p}{q}} \cdot \left( \sum_{i=1}^{n} 1^{\frac{q}{q - p}} \right)^{1 - \frac{p}{q}} \\
                                 &= n^{1 - \frac{p}{q}} \cdot \left( \sum_{i=1}^{n} \abs{x_i}^q \right)^\frac{p}{q}
    \end{align*}
    Therefore, for $\vec{x} = (x_1..., x_n) \in \mathbb{R}$,
    \begin{align*}
      \norm{\vec{x}}_p &= \left( \sum_{i=1}^{n} \abs{x_i}^p \right)^{\frac{1}{p}} \leq \left( n^{1 - \frac{p}{q}} \cdot \left( \sum_{i=1}^{n} \abs{x_i}^q \right)^\frac{p}{q} \right)^{\frac{1}{p}} \\
                       &= n^{\frac{1}{p} - \frac{1}{q}} \cdot \left( \sum_{i=1}^{n} \abs{x_i}^q \right)^{\frac{1}{q}} = n^{\frac{1}{p} - \frac{1}{q}} \cdot \norm{\vec{x}}_q.
    \end{align*}
    Thus, we have
    \begin{equation*}
      \norm\cdot_q \leq \norm\cdot_p.
    \end{equation*}
    The chain of inequality follows.\qed
  \end{proof}
\end{note}

\begin{eg}[Sequence Spaces]\label{eg:sequence_spaces}\index{Sequence Spaces}
  \begin{enumerate}
    \item Let $l_1 = \left\{ \{ x_i \} \mid \sum\limits_{i=1}^{\infty} \abs{x_i} < \infty \right\}$. Define
      \begin{equation*}
        \norm{ \{ x_i \} }_1 = \sum_{i=1}^{\infty} \abs{x_i}
      \end{equation*}
      Let $\{ x_i \}, \{ y_i \} \in l_1$. Observe that $\forall n \in \mathbb{N}$, we have
      \begin{equation*}
        \sum_{i=1}^{n} \abs{x_i + y_i} \leq \sum_{i=1}^{n} \abs{x_i} + \sum_{i=1}^{n} \abs{y_i} \leq \norm{ \{ x_i \} }_1 + \norm{ \{ y_i \} }_1.
      \end{equation*}
      Then by the \hyperref[thm:monotone_convergence_theorem]{Monotone Convergence Theorem}, we have that
      \begin{equation*}
        \sum_{i=1}^{\infty} \abs{x_i + y_i} \leq \norm{ \{ x_i \} }_1 + \norm{ \{ y_i \} }_1.
      \end{equation*}
      Thus $\{ x_i + y_i \} \in l_1$ and
      \begin{equation*}
        \norm{ \{ x_i + y_i \} }_1 \leq \norm{ \{ x_i \} }_1 + \norm{ \{ y_i \} }_1.
      \end{equation*}

      Let $\{ x_n \} \in l_1$ and $\alpha \in \mathbb{R}$. Then
      \begin{equation*}
        \sum_{i=1}^{\infty} \abs{ \alpha x_i } = \abs{\alpha} \sum_{i=1}^{\infty} \abs{x_i}.
      \end{equation*}
      Therefore $\{ \alpha x_n \} \in l_1$ and $\norm{ \{ \alpha x_n \} }_1 = \abs{\alpha} \norm{ \{ x_n \} }_1$.

      Thus, we have that $l_1$ is a vector space, and $(l_1, \norm\cdot_1)$ is a normed linear space.

    \item Let $l_\infty(\mathbb{N}) = l_\infty = \{ \{ x_i \} \mid \{ x_i \} \text{ is bounded } \}$. Define
      \begin{equation*}
        \norm{ \{ x_i \} }_\infty = \sup \{ \abs{ x_1 } \mid i \in \mathbb{N} \}.
      \end{equation*}
      Observe that $\forall \{ x_i \}, \{ y_i \} \in l_\infty$, then $\forall i \in \mathbb{N}$, we have
      \begin{equation*}
        \abs{x_i + y_i} \leq \abs{ x_i } + \abs{ y_i } \leq \norm{\{x_i\}}_\infty + \norm{\{y_i\}}_\infty.
      \end{equation*}
      So $\{ x_i + y_i \} \in l_\infty$, and
      \begin{equation*}
        \norm{ \{ x_i + y_i \} }_\infty \leq \norm{ \{ x_i\} }_\infty + \norm{ \{ y_i \} }_\infty.
      \end{equation*}
      Consequently so, $\{ \alpha x_i \} \in l_\infty$ and
      \begin{equation*}
        \norm{ \{ \alpha x_i \} }_\infty = \abs{ \alpha } \norm{ \{ x_i \} }_\infty.
      \end{equation*}
  \end{enumerate}
\end{eg}

\newthought{Question}: What about $l_p(\mathbb{R})$?

% section introduction_to_metric_spaces_continued (end)

% chapter lecture_10_sep_28th (end)

\chapter{Lecture 11 Oct 01st}%
\label{chp:lecture_11_oct_01st}
% chapter lecture_11_oct_01st

\section{Introduction to Metric Spaces (Continued 2)}%
\label{sec:introduction_to_metric_spaces_continued_2}
% section introduction_to_metric_spaces_continued_2

We wondered about $l_p(\mathbb{R})$ in the last lecture but let us consider a case that is even more general.

\newthought{Question}: Can we define $l_p(\Gamma)$ for any set $\Gamma$?

\begin{eg}\label{eg:l_infty_on_gamma}
  Let $l_\infty(\Gamma) = \{ f : \Gamma \to \mathbb{R} \mid f(\Gamma) \text{ is bounded } \}$. If $f \in l_\infty(\Gamma)$, define
  \begin{equation*}
    \norm{f}_\infty = \sup \{ \abs{f(x)} \mid x \in \Gamma \}.
  \end{equation*}
  Notice that for $f, g \in l_\infty(\Gamma)$, and $\alpha \in \mathbb{R}$, then we have, by the Triangle Inequality,
  \begin{align*}
    \norm{f + g}_\infty &= \sup \{ \abs{ (f + g)(x) } \mid x \in \Gamma \} \\
                        &= \sup \{ \abs{ f(x) + g(x) } \mid x \in \Gamma \} \\
                        &\leq \sup \{ \abs{ f(x) } \mid x \in \Gamma \} + \sup \{ \abs{ g(x) } \mid x \in \Gamma \} \\
                        &= \norm{f}_\infty \norm{g}_\infty.
  \end{align*}
  So $f + g \in l_\infty(\Gamma)$, and
  \begin{equation*}
    \norm{f + g}_\infty \leq \norm{f}_\infty + \norm{f}_\infty.
  \end{equation*}
  Also, we have
  \begin{align*}
    \norm{\alpha f}_\infty &= \sup \{ \abs{ (\alpha f)(x) } \mid x \in \Gamma \} \\
                           &= \sup \{ \abs{\alpha} \abs{ f(x) } \mid x \in \Gamma \} \\
                           &= \abs{\alpha} \sup \{ \abs{f(x)} \mid x \in \Gamma \} \\
                           &= \abs{\alpha} \norm{f}_\infty.
  \end{align*}
  So $\alpha f \in l_\infty(\Gamma)$, and $\norm{\alpha f}_\infty = \abs{\alpha} \norm{f}$.

  Therefore, $(l_\infty(\Gamma), \norm\cdot_\infty)$ is a normed linear space.
\end{eg}

\begin{eg}
  Let $l_1(\Gamma) = \{ f : \Gamma \to \mathbb{R} \mid P(f) \}$, where $P(f)$ is the statement\marginnote{\hlwarn{Require clarification} Notice that $\forall f \in l_1(\Gamma)$, for each $n \in \mathbb{N}$,
  \begin{equation*}
    A_n = \{ x \in \Gamma \mid \abs{f(x)} \geq \frac{1}{n} \} \text{ is finite. }
  \end{equation*}
  So
  \begin{equation*}
    A_0 = \bigcup_{n = 1}^{\infty} A_n \text{ is countable. }
  \end{equation*}
  and
  \begin{equation*}
    A_0 = \{ x \in \Gamma \mid \abs{ f(x) } \neq \emptyset \}
  \end{equation*}
  }
  \begin{equation*}
    \norm{f}_1 = \sup \left\{ \sum_{i=1}^{n} \abs{f(x_i)} \mid x_1, ..., x_n \in \Gamma, n \in \mathbb{N} \setminus \{ 0 \} \right\} < \infty.
  \end{equation*}
  It is clear that $l_1(\Gamma) \subseteq l_\infty(\Gamma)$, where $l_\infty(\Gamma)$ is from \cref{eg:l_infty_on_gamma}. Consequently, $(l_1(\Gamma), \norm\cdot_1)$ is a normed linear space.
\end{eg}

We can extend the same idea onto $l_p$ spaces.

\begin{eg}
  Let $X = C[a, b] = \{ f : [a, b] \to \mathbb{R} \mid f \text{ is continuous on } [a, b] \}$, and define\sidenote{Note in this case $\sup$ is also $\max$, since we are on a closed interval.}
  \begin{align*}
    \norm{f}_\infty &= \sup \{ \abs{f(x)} \mid x \in [a, b] \} \\
                    &= \max \{ \abs{f(x)} \mid x \in [a, b] \}
  \end{align*}
  By (regular) Triangle Inequality, for any $f, g \in C[a, b]$, we have
  \begin{align*}
    \norm{f + g}_\infty &= \max \{ \abs{ f(x) + g(x) } \mid x \in [a, b] \} \\
                        &\leq \max \{ \abs{f(x)} \mid x \in [a, b] \} + \max \{ \abs{g(x)} \mid x \in [a, b] \} \\
                        &= \norm{f}_\infty + \norm{g}_\infty,
  \end{align*}
  and, for $\alpha \in \mathbb{R}$,
  \begin{align*}
    \norm{\alpha f}_\infty &= \max \{ \abs{ \alpha f(x) } \mid x \in [a, b] \} \\
                           &= \abs{\alpha} \max \{ \abs{f(x)} \mid x \in [a, b] \\
                           &= \abs{\alpha} \norm{f}_\infty.
  \end{align*}
  Thus $\norm\cdot_\infty$ is a norm on $C[a, b]$, and $( C[a, b], \norm\cdot_\infty )$ is a normed linear space\sidenote{This space is complete.}\sidenote{This space is important for us for the purpose of this course.}.

  Also, observe that
  \begin{equation*}
    C[a, b] \subset l_\infty( [a, b] ).
  \end{equation*}
\end{eg}

\begin{eg}
  Let $X = C[a, b]$ \sidenote{Some authors also write this as $L'[a, b]$.} have the same definition as the previous example, but this time define
  \begin{equation*}
    \norm{f}_1 = \int_{a}^{b} \abs{ f(t) } \dif{t}.
  \end{equation*}
  By \hlnotea{linearity of integration}, both the triangle equality and scalar multiplication hold, and so $(C[a, b], \norm\cdot_1)$ is a normed linear space\sidenote{Compared to the last example, this is not a complete space.}.
\end{eg}

\begin{eg}
  Let $X = C[a, b]$, and $1 < p < \infty$. Define
  \begin{equation*}
    \norm{f}_p = \left( \int_{a}^{b} \abs{f(x)}^p \dif{x} \right)^{\frac{1}{p}}
  \end{equation*}
  Again, by linearity of integration, scalar multiplication holds. However, it is not as easy to show for the triangle inequality; we are now asking the same question as we did before for $l_p$, which we solved using \hyperref[thm:holder_s_inequality]{H\"{o}lder's Inequality} and \hyperref[thm:minkowshi_s_inequality]{Minkowski's Inequality}. But now, instead of summations, we have integrations.
\end{eg}

\begin{thm}[H\"{o}lder's Inequality v2]
\index{H\"{o}lder's Inequality v2}
\label{thm:holder_s_inequality_v2}
  Let $1 < p < \infty$ and $\frac{1}{p} + \frac{1}{q} = 1$. For each $f, g \in C[a, b]$, we have
  \begin{equation*}
    \int_{a}^{b} \abs{f(t)g(t)} \dif{t} \leq \left( \int_{a}^{b} \abs{f(t)}^p \dif{t} \right)^\frac{1}{p} \left( \int_{a}^{b} \abs{g(t)}^q \dif{t} \right)^\frac{1}{q}.
  \end{equation*}
\end{thm}

\begin{proof}
  If either $f(x) = 0$ or $g(x) = 0$ for all $x \in [a, b]$, then the inequality holds trivially so. Thus, we may assume that $\forall x \in [a, b]$, $f(x) \neq 0 \neq g(x)$. By the linearity of integration, we can apply the same reasoning as we did in \cref{thm:holder_s_inequality}, and assume that
  \begin{equation*}
    \int_{a}^{b} \abs{f(t)}^p \dif{t} = 1 = \int_{a}^{b} \abs{g(t)}^q \dif{t} 
  \end{equation*}
  By \cref{lemma:lemma_for_holders}, we have
  \begin{equation*}
    \abs{f(t) g(t)} \leq \frac{\abs{f(t)}^p}{p} + \frac{\abs{g(t)}^q}{q}.
  \end{equation*}
  Thus
  \begin{align*}
    \int_{a}^{b} \abs{f(t) g(t)} \dif{t} &\leq \int_{a}^{b} \left( \frac{\abs{f(t)}^p}{p} + \frac{\abs{g(t)}^q}{q} \right) \dif{t} \\
                                         &= \frac{1}{p} + \frac{1}{q} = 1 \\
                                         &= \left( \int_{a}^{b} \abs{f(t)}^p \dif{t} \right)^\frac{1}{p} \left( \int_{a}^{b} \abs{g(t)}^q \dif{t} \right)^\frac{1}{q}
  \end{align*}
  as required.\qed
\end{proof}

\begin{thm}[Minkowski's Inequality v2]
\index{Minkowski's Inequality v2}
\label{thm:minkowski_s_inequality_v2}
  Let $1 < p < \infty$. If $f, g \in C[a, b]$, then
  \begin{equation*}
    \left( \int_{a}^{b} \abs{(f + g)(t)}^p \dif{t} \right)^\frac{1}{p} \leq \left( \abs{f(t)}^p \dif{t} \right)^\frac{1}{p} \cdot \left( \int_{a}^{b} \abs{g(t)}^p \dif{t} \right)^\frac{1}{p}.
  \end{equation*}
\end{thm}

\begin{proof}
  The proof is similar to the one we had in \cref{thm:minkowshi_s_inequality}; if $\forall x \in [a, b]$, either $f(x) = 0$ or $g(x) = 0$, then the inequality holds trivially so. Thus we may assyme that $\forall x \in [a, b]$, $f(x) \neq 0 \neq g(x)$. Now, notice that by (regular) Triangle Inequality and, later on, \cref{thm:holder_s_inequality_v2},
  \begin{align*}
    &\int_{a}^{b} \abs{(f + g)(t)}^p \dif{t} \\
    &= \int_{a}^{b} \abs{(f + g)(t)} \abs{(f + g)(t)}^{p - 1} \dif{t} \\
    &\leq \int_{a}^{b} \abs{f(t)} \abs{(f + g)(t)}^{p - 1} \dif{t} + \int_{a}^{b} \abs{g(t)} \abs{(f + g)(t)} \dif{t} \\
    &\leq \left( \int_{a}^{b} \abs{f(t)}^p \dif{t} \right)^\frac{1}{p} \left( \int_{a}^{b} \abs{(f + g)(t)}^{q (p - 1)} \dif{t} \right)^\frac{1}{q} \\
    &\qquad + \left( \int_{a}^{b} \abs{g(t)}^p \dif{t} \right)^\frac{1}{p} \left( \int_{a}^{b} \abs{(f + g)(t)}^{q (p - 1)} \dif{t} \right)^\frac{1}{q} \\
    &= \left[ \left( \int_{a}^{b} \abs{f(t)}^p \dif{t} \right)^\frac{1}{p} + \left( \int_{a}^{b} \abs{g(t)}^p \dif{t} \right)^\frac{1}{p} \right] \\
    &\qquad \cdot \left( \int_{a}^{b} \abs{(f + g)(t)}^{p} \right)^\frac{1}{q}
  \end{align*}
  where we note that $\frac{1}{p} + \frac{1}{q} = 1 \implies p = q(p - 1)$. Consequently, since $\frac{1}{p} = 1 - \frac{1}{q}$,
  \begin{equation*}
    \left( \int_{a}^{b} \abs{(f + g)(t)}^p \dif{t} \right)^\frac{1}{p} \leq \left( \abs{f(t)}^p \dif{t} \right)^\frac{1}{p} \cdot \left( \int_{a}^{b} \abs{g(t)}^p \dif{t} \right)^\frac{1}{p},
  \end{equation*}
  as required.\qed
\end{proof}

This shows that our definition of $\norm\cdot_p$ on $C[a, b]$ is indeed a norm, and so $( C[a, b], \norm\cdot_p )$ is a normed linear space.\marginnote{
\begin{ex}
  Show that there exists an injection from $(C[a, b], \norm\cdot_2)$ to $l_2(\mathbb{N})$. Note that this does not work for $p \geq 3$.
\end{ex}}

\begin{eg}[Bounded Operator]
  Let $(X, \norm\cdot_X)$ and $(Y, \norm\cdot_Y)$ be normed linear spaces. Let $T : X \to Y$ be linear. Define
  \begin{equation*}
    \norm{T} = \sup \{ \norm{T_X}_Y \mid \norm{x}_X < 1 \}.
  \end{equation*}
  We say that $T$ is bounded if $\norm{T} < \infty$. Let
  \begin{equation*}
    B(X, Y) = \{ T : X \to Y \mid T \text{ is bounded } \}.
  \end{equation*}
  In the next lecture, we shall show that $(B(X, Y), \norm\cdot)$ is a normed linear space.
\end{eg}

\newthought{Question}: Consider the transformation $\begin{bmatrix} 1 & 2 \\ 1 & 1 \end{bmatrix} : \mathbb{R}^2 \to \mathbb{R}^2$. What is a norm $\norm{ \begin{bmatrix} 1 & 2 \\ 1 & 1 \end{bmatrix} }$ that works?

% section introduction_to_metric_spaces_continued_2 (end)

% chapter lecture_11_oct_01st (end)

\appendix

\chapter{Useful Theorems from Earlier Calculus}%
\label{chp:useful_theorems_from_earlier_calculus}
% chapter useful_theorems_from_earlier_calculus

\begin{thm}[Monotone Convergence Theorem]
\index{Monotone Convergence Theorem}
\label{thm:monotone_convergence_theorem}
  Let $\{x_k\}$ be a sequence in $\mathbb{R}$.
  \begin{enumerate}
    \item Suppose $\{x_k\}$ is increasing.
      \begin{itemize}
        \item If $\{ x_k \}$ is bounded above, then $x_k \to \sup \{ x_k \}$ as $k \to \infty$.
        \item If $\{ x_k \}$ is not bounded above, then $x_k \to \infty$ as $k \to \infty$.
      \end{itemize}
    \item Suppose $\{ x_k \}$ is decreasing.
      \begin{itemize}
        \item If $\{ x_k \}$ is bounded below, then $x_k \to \inf \{ x_k \}$ as $k \to \infty$.
        \item If $\{ x_k \}$ is not bounded below, then $x_k \to -\infty$.
      \end{itemize}
  \end{enumerate}
\end{thm}

% chapter useful_theorems_from_earlier_calculus (end)

\backmatter

\pagestyle{plain}

% \nobibliography*
\bibliography{references}

\printindex

\end{document}

