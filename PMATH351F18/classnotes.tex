\documentclass[notoc,notitlepage]{tufte-book}
% \nonstopmode % uncomment to enable nonstopmode

\usepackage{classnotetitle}

\title{PMATH351 - Real Analysis}
\author{Johnson Ng}
\subtitle{Classnotes for Fall 2018}
\credentials{BMath (Hons), Pure Mathematics major, Actuarial Science Minor}
\institution{University of Waterloo}

\setcounter{secnumdepth}{3}
\setcounter{tocdepth}{3}

\renewcommand{\baselinestretch}{1.2}
\usepackage{geometry}
\geometry{letterpaper}
\usepackage[parfill]{parskip}
\usepackage{graphicx}

% Essential Packages
\usepackage{makeidx}
\makeindex
\usepackage{enumitem}
\usepackage[T1]{fontenc}
\usepackage{natbib}
\bibliographystyle{apalike}
\usepackage{ragged2e}
\usepackage{etoolbox}
\usepackage{amssymb}
\usepackage{eso-pic}
\usepackage[fixed]{fontawesome5}
\usepackage{todonotes}
\usepackage{apptools, chngcntr}
\usepackage{amsmath}
\usepackage{mathrsfs}
\usepackage{stmaryrd}
\usepackage{mathtools}
\usepackage{tocloft}
\usepackage{tocbibind}
\usepackage{xparse}
\usepackage{tkz-euclide}
\usetkzobj{all}
\usepackage[utf8]{inputenc}
\usepackage{csquotes}
\usepackage[english]{babel}
\usepackage{marvosym}
\usepackage{pgf,tikz}
\usepackage{tikz-cd}
\usepackage{ifthen}
\usepackage{pgfplots}
\usepackage{fancyhdr}
\usepackage{array}
\usepackage{float}
\usepackage{xcolor}
\usepackage{soul}
\usepackage{centernot}
\usepackage{silence}
  \WarningFilter*{latex}{Marginpar on page \thepage\space moved}
\usepackage{tcolorbox}
\tcbuselibrary{skins,breakable}
\usepackage{longtable,booktabs}
\usepackage[amsmath,hyperref,thmmarks]{ntheorem}
\usepackage{thmtools}
\usepackage{hyperref}
\usepackage[noabbrev,capitalize,nameinlink]{cleveref}

\newcommand{\personalcolor}{false}
\ifthenelse{\equal{\personalcolor}{true}}{
  \usepackage{colorscheme-chaos}
}{
  \usepackage{colorscheme-student}
}

% hyperref Package Settings
\hypersetup{
    unicode=true,          % non-Latin characters in Acrobat’s bookmarks
    pdftoolbar=false,        % show Acrobat’s toolbar?
    pdfmenubar=false,        % show Acrobat’s menu?
    pdffitwindow=true,     % window fit to page when opened
    colorlinks=true,
    allcolors=magenta,
}

% tikz
\usepgfplotslibrary{polar}
\usepgflibrary{shapes.geometric}
\usetikzlibrary{angles,patterns,calc,decorations.markings,arrows.meta,tikzmark,bending}
\tikzset{midarrow/.style 2 args={
        decoration={markings,
            mark= at position #2 with {\arrow{#1}} ,
        },
        postaction={decorate}
    },
    midarrow/.default={latex}{0.5}
}
\def\centerarc[#1](#2)(#3:#4:#5)% Syntax: [draw options] (center) (initial angle:final angle:radius)
    { \draw[#1] ($(#2)+({#5*cos(#3)},{#5*sin(#3)})$) arc (#3:#4:#5); }
% from https://tex.stackexchange.com/questions/67573/tikz-shift-and-rotate-in-3d
\newcommand{\rotateRPY}[4][0/0/0]% point to be saved to \savedxyz, roll, pitch, yaw
{   \pgfmathsetmacro{\rollangle}{#2}
    \pgfmathsetmacro{\pitchangle}{#3}
    \pgfmathsetmacro{\yawangle}{#4}

    % to what vector is the x unit vector transformed, and which 2D vector is this?
    \pgfmathsetmacro{\newxx}{cos(\yawangle)*cos(\pitchangle)}% a
    \pgfmathsetmacro{\newxy}{sin(\yawangle)*cos(\pitchangle)}% d
    \pgfmathsetmacro{\newxz}{-sin(\pitchangle)}% g
    \path (\newxx,\newxy,\newxz);
    \pgfgetlastxy{\nxx}{\nxy};

    % to what vector is the y unit vector transformed, and which 2D vector is this?
    \pgfmathsetmacro{\newyx}{cos(\yawangle)*sin(\pitchangle)*sin(\rollangle)-sin(\yawangle)*cos(\rollangle)}% b
    \pgfmathsetmacro{\newyy}{sin(\yawangle)*sin(\pitchangle)*sin(\rollangle)+ cos(\yawangle)*cos(\rollangle)}% e
    \pgfmathsetmacro{\newyz}{cos(\pitchangle)*sin(\rollangle)}% h
    \path (\newyx,\newyy,\newyz);
    \pgfgetlastxy{\nyx}{\nyy};

    % to what vector is the z unit vector transformed, and which 2D vector is this?
    \pgfmathsetmacro{\newzx}{cos(\yawangle)*sin(\pitchangle)*cos(\rollangle)+ sin(\yawangle)*sin(\rollangle)}
    \pgfmathsetmacro{\newzy}{sin(\yawangle)*sin(\pitchangle)*cos(\rollangle)-cos(\yawangle)*sin(\rollangle)}
    \pgfmathsetmacro{\newzz}{cos(\pitchangle)*cos(\rollangle)}
    \path (\newzx,\newzy,\newzz);
    \pgfgetlastxy{\nzx}{\nzy};

    % transform the point given by #1
    \foreach \x/\y/\z in {#1}
    {   \pgfmathsetmacro{\transformedx}{\x*\newxx+\y*\newyx+\z*\newzx}
        \pgfmathsetmacro{\transformedy}{\x*\newxy+\y*\newyy+\z*\newzy}
        \pgfmathsetmacro{\transformedz}{\x*\newxz+\y*\newyz+\z*\newzz}
        \xdef\savedx{\transformedx}
        \xdef\savedy{\transformedy}
        \xdef\savedz{\transformedz}     
    }
}
\tikzset{RPY/.style={x={(\nxx,\nxy)},y={(\nyx,\nyy)},z={(\nzx,\nzy)}}}
\newcommand{\AxisRotator}[1][rotate=0]{%
    \tikz [x=0.25cm,y=0.60cm,line width=.2ex,-stealth,#1] \draw (0,0) arc (-150:150:1 and 1);%
  }

% enumitems
\newlist{inlinelist}{enumerate*}{1}
\setlist*[inlinelist,1]{%
  label=(\roman*),
}

% Theorem Style Customization
\setlength\theorempreskipamount{2ex}
\setlength\theorempostskipamount{3ex}

\makeatletter
\let\nobreakitem\item
\let\@nobreakitem\@item
\patchcmd{\nobreakitem}{\@item}{\@nobreakitem}{}{}
\patchcmd{\nobreakitem}{\@item}{\@nobreakitem}{}{}
\patchcmd{\@nobreakitem}{\@itempenalty}{\@M}{}{}
\patchcmd{\@xthm}{\ignorespaces}{\nobreak\ignorespaces}{}{}
\patchcmd{\@ythm}{\ignorespaces}{\nobreak\ignorespaces}{}{}

\renewtheoremstyle{break}%
  {\item{\theorem@headerfont
          ##1\ ##2\theorem@separator}\hskip\labelsep\relax\nobreakitem}%
  {\item{\theorem@headerfont
          ##1\ ##2\ (##3)\theorem@separator}\hskip\labelsep\relax\nobreakitem}
\makeatother

% ntheorem Declarations
\theorempreskip{10pt}
\theorempostskip{5pt}
\theoremstyle{break}

\theoremsymbol{\faComment}
\newtheorem{remark}{Remark}[section]
\theoremsymbol{}
\newtheorem*{strategy}{\faPaperPlane Strategy}
\newtheorem*{procedure}{\faCodeBranch\ }
\newtheorem{ex}{Exercise}[section]
\theorembodyfont{\normalfont}
\newtheorem*{solution}{\faPencil* Solution}
\theoremsymbol{\faGavel}
\newtheorem{eg}{Example}[section]
\theoremsymbol{}
\theorembodyfont{\it}

    % definition env
\theoremprework{\textcolor{blue}{\hrule height 2pt width \textwidth}}
\theoremheaderfont{\color{blue}\normalfont\bfseries}
\theorempostwork{\textcolor{blue}{\hrule height 2pt width \textwidth}}
\theoremindent10pt
\newtheorem{defn}{\faBook Definition}

    % definition env no num
\theoremprework{\textcolor{blue}{\hrule height 2pt width \textwidth}}
\theoremheaderfont{\color{blue}\normalfont\bfseries}
\theorempostwork{\textcolor{blue}{\hrule height 2pt width \textwidth}}
\theoremindent10pt
\newtheorem*{defnnonum}{\faBook Definition}

\theoremprework{\textcolor{blue}{\hrule height 2pt width \marginparwidth}}
\theoremheaderfont{\color{blue}\normalfont\bfseries}
\theorempostwork{\textcolor{blue}{\hrule height 2pt width \marginparwidth}}
\theoremindent10pt
\newtheorem{margindefn}[defn]{\faBook Definition}

\theoremprework{\textcolor{blue}{\hrule height 2pt width \textwidth}}
\theoremheaderfont{\color{blue}\normalfont\bfseries}
\theorempostwork{\textcolor{blue}{\hrule height 2pt width \textwidth}}
\theoremindent10pt
\newtheorem*{margindefnnonum}{\faBook Definition}

    % theorem envs
\theoremprework{\textcolor{magenta}{\hrule height 2pt width \textwidth}}
\theoremheaderfont{\color{magenta}\normalfont\bfseries}
\theorempostwork{\textcolor{magenta}{\hrule height 2pt width \textwidth}}
\theoremindent10pt
\newtheorem{thm}{\faCoffee Theorem}

\theoremprework{\textcolor{magenta}{\hrule height 2pt width \textwidth}}
\theorempostwork{\textcolor{magenta}{\hrule height 2pt width \textwidth}}
\theoremindent10pt
\newtheorem{propo}[thm]{\faTint Proposition}

\theoremprework{\textcolor{magenta}{\hrule height 2pt width \textwidth}}
\theorempostwork{\textcolor{magenta}{\hrule height 2pt width \textwidth}}
\theoremindent10pt
\newtheorem{crly}[thm]{\faSpaceShuttle Corollary}

\theoremprework{\textcolor{magenta}{\hrule height 2pt width \textwidth}}
\theorempostwork{\textcolor{magenta}{\hrule height 2pt width \textwidth}}
\theoremindent10pt
\newtheorem{lemma}[thm]{\faTree Lemma}

\theoremprework{\textcolor{magenta}{\hrule height 2pt width \textwidth}}
\theorempostwork{\textcolor{magenta}{\hrule height 2pt width \textwidth}}
\theoremindent10pt
\newtheorem{axiom}[thm]{\faShield Axiom}

    % theorem envs without counter
\theoremprework{\textcolor{magenta}{\hrule height 2pt width \textwidth}}
\theoremheaderfont{\color{magenta}\normalfont\bfseries}
\theorempostwork{\textcolor{magenta}{\hrule height 2pt width \textwidth}}
\theoremindent10pt
\newtheorem*{thmnonum}{\faCoffee Theorem}

\theoremprework{\textcolor{magenta}{\hrule height 2pt width \textwidth}}
\theorempostwork{\textcolor{magenta}{\hrule height 2pt width \textwidth}}
\theoremindent10pt
\newtheorem*{propononum}{\faTint Proposition}

\theoremprework{\textcolor{magenta}{\hrule height 2pt width \textwidth}}
\theorempostwork{\textcolor{magenta}{\hrule height 2pt width \textwidth}}
\theoremindent10pt
\newtheorem*{crlynonum}{\faSpaceShuttle Corollary}

\theoremprework{\textcolor{magenta}{\hrule height 2pt width \textwidth}}
\theorempostwork{\textcolor{magenta}{\hrule height 2pt width \textwidth}}
\theoremindent10pt
\newtheorem*{lemmanonum}{\faTree Lemma}

\theoremprework{\textcolor{magenta}{\hrule height 2pt width \textwidth}}
\theorempostwork{\textcolor{magenta}{\hrule height 2pt width \textwidth}}
\theoremindent10pt
\newtheorem*{axiomnonum}{\faShield Axiom}

    % envs on margins
\theoremprework{\textcolor{magenta}{\hrule height 2pt width \marginparwidth}}
\theoremheaderfont{\color{magenta}\normalfont\bfseries}
\theorempostwork{\textcolor{magenta}{\hrule height 2pt width \marginparwidth}}
\theoremindent10pt
\newtheorem{marginthm}[thm]{\faCoffee Theorem}

\theoremprework{\textcolor{magenta}{\hrule height 2pt width \marginparwidth}}
\theorempostwork{\textcolor{magenta}{\hrule height 2pt width \marginparwidth}}
\theoremindent10pt
\newtheorem{marginpropo}[thm]{\faTint Proposition}

\theoremprework{\textcolor{magenta}{\hrule height 2pt width \marginparwidth}}
\theorempostwork{\textcolor{magenta}{\hrule height 2pt width \marginparwidth}}
\theoremindent10pt
\newtheorem{margincrly}[thm]{\faSpaceShuttle Corollary}

\theoremprework{\textcolor{magenta}{\hrule height 2pt width \marginparwidth}}
\theorempostwork{\textcolor{magenta}{\hrule height 2pt width \marginparwidth}}
\theoremindent10pt
\newtheorem{marginlemma}[thm]{\faTree Lemma}

\theoremprework{\textcolor{magenta}{\hrule height 2pt width \marginparwidth}}
\theorempostwork{\textcolor{magenta}{\hrule height 2pt width \marginparwidth}}
\theoremindent10pt
\newtheorem{marginaxiom}[thm]{\faShield Axiom}

    % envs on margins without counter
\theoremprework{\textcolor{magenta}{\hrule height 2pt width \marginparwidth}}
\theoremheaderfont{\color{magenta}\normalfont\bfseries}
\theorempostwork{\textcolor{magenta}{\hrule height 2pt width \marginparwidth}}
\theoremindent10pt
\newtheorem*{marginthmnonum}{\faCoffee Theorem}

\theoremprework{\textcolor{magenta}{\hrule height 2pt width \marginparwidth}}
\theorempostwork{\textcolor{magenta}{\hrule height 2pt width \marginparwidth}}
\theoremindent10pt
\newtheorem*{marginpropononum}{\faTint Proposition}

\theoremprework{\textcolor{magenta}{\hrule height 2pt width \marginparwidth}}
\theorempostwork{\textcolor{magenta}{\hrule height 2pt width \marginparwidth}}
\theoremindent10pt
\newtheorem*{margincrlynonum}{\faSpaceShuttle Corollary}

\theoremprework{\textcolor{magenta}{\hrule height 2pt width \marginparwidth}}
\theorempostwork{\textcolor{magenta}{\hrule height 2pt width \marginparwidth}}
\theoremindent10pt
\newtheorem*{marginlemmanonum}{\faTree Lemma}

\theoremprework{\textcolor{magenta}{\hrule height 2pt width \marginparwidth}}
\theorempostwork{\textcolor{magenta}{\hrule height 2pt width \marginparwidth}}
\theoremindent10pt
\newtheorem*{marginaxiomnonum}{\faShield Axiom}

    % proof env
\theoremprework{\textcolor{green}{\hrule height 2pt width \textwidth}}
\theorembodyfont{\normalfont}
\theoremheaderfont{\color{green}\normalfont\bfseries}
\theorempostwork{\textcolor{green}{\hrule height 2pt width \textwidth}}
\theoremsymbol{\ensuremath{_\square}}
\newtheorem*{proof}{\faPencil* Proof}
\theoremsymbol{}

\theoremprework{\textcolor{green}{\hrule height 2pt width \marginparwidth}}
\theorembodyfont{\normalfont}
\theoremheaderfont{\color{green}\normalfont\bfseries}
\theorempostwork{\textcolor{green}{\hrule height 2pt width \marginparwidth}}
\theoremsymbol{\ensuremath{_\square}}
\newtheorem*{mproof}{\faPencil* Proof}
\theoremsymbol{}

    % note and notation env
\theorembodyfont{\it}

\theoremprework{\textcolor{yellow}{\hrule height 2pt width \textwidth}}
\theoremheaderfont{\color{yellow}\normalfont\bfseries}
\theorempostwork{\textcolor{yellow}{\hrule height 2pt width \textwidth}}
\newtheorem{note}{\faQuoteLeft Note}[section]

\theoremprework{\textcolor{yellow}{\hrule height 2pt width \marginparwidth}}
\theoremheaderfont{\color{yellow}\normalfont\bfseries}
\theorempostwork{\textcolor{yellow}{\hrule height 2pt width \marginparwidth}}
\newtheorem{mnote}[note]{\faQuoteLeft Note}

\theoremprework{\textcolor{yellow}{\hrule height 2pt width \textwidth}}
\theorempostwork{\textcolor{yellow}{\hrule height 2pt width \textwidth}}
\newtheorem*{notation}{\faPaw Notation}

    % warning env
\theoremprework{\textcolor{red}{\hrule height 2pt width \textwidth}}
\theoremheaderfont{\color{red}\normalfont\bfseries}
\theorempostwork{\textcolor{red}{\hrule height 2pt width \textwidth}}
\theoremindent10pt
\newtheorem*{warning}{\faBug Warning}

\theoremprework{\textcolor{red}{\hrule height 2pt width \marginparwidth}}
\theoremheaderfont{\color{red}\normalfont\bfseries}
\theorempostwork{\textcolor{red}{\hrule height 2pt width \marginparwidth}}
\theoremindent10pt
\newtheorem*{marginwarning}{\faBug Warning}

% rule for appendices
\AtAppendix{\counterwithin{defn}{chapter}}
\AtAppendix{\counterwithin{thm}{chapter}}
\AtAppendix{\counterwithin{propo}{chapter}}
\AtAppendix{\counterwithin{lemma}{chapter}}
\AtAppendix{\counterwithin{crly}{chapter}}
\AtAppendix{\counterwithin{axiom}{chapter}}

% more environments
\newtcolorbox{quotebox}[2]{
  blanker,enhanced,breakable,standard jigsaw,
  opacityback=0,
  coltext=\ifblank{#2}{black}{#2},
  left=5mm,right=5mm,top=2mm,bottom=2mm,
  colframe=\ifblank{#1}{bblack}{#1},
  boxrule=0pt,leftrule=3pt,
  fontupper=\itshape
}

\providecommand{\parthook}{}
\patchcmd{\part}{\thispagestyle}{\parthook\thispagestyle}{}{}
\newcommand{\partimage}[2][]{% \parthook[<options>]{<image>}
  \renewcommand{\parthook}{% Update \parthook
    \AddToShipoutPictureBG*{% Add picture to background of THIS page only
      \AtPageLowerLeft{\includegraphics[width=\paperwidth,height=\paperheight,#1]{#2}}}% Insert image
    \renewcommand{\parthook}{}}}% Restore \parthook

\AtBeginDocument{\renewcommand\contentsname{\slshape Table of Contents\normalfont}}
\cftpagenumbersoff{part}

\newcommand{\tuftepart}[1]{\newgeometry{}\part{#1}\restoregeometry}

% Heading formattings
% chapter format
\titleformat{\chapter}%
  {\huge\rmfamily\itshape\color{magenta}}% format applied to label+text
  {\llap{\colorbox{magenta}{\parbox[c][1cm]{3cm}{\hfill\itshape\Huge\textcolor{background}{\thechapter}}}}}% label
  {5pt}% horizontal separation between label and title body
  {\faLeaf}% before the title body
  []% after the title body

% section format
\titleformat{\section}%
  {\normalfont\Large\rmfamily\itshape\color{blue}}% format applied to label+text
  {\llap{\colorbox{blue}{\parbox{3cm}{\hfill\itshape\textcolor{background}{\thesection}}}}}% label
  {5pt}% horizontal separation between label and title body
  {}% before the title body
  []% after the title body

% subsection format
\titleformat{\subsection}%
  {\normalfont\large\itshape\color{green}}% format applied to label+text
  {\llap{\colorbox{green}{\parbox{3cm}{\hfill\textcolor{background}{\thesubsection}}}}}% label
  {1em}% horizontal separation between label and title body
  {}% before the title body
  []% after the title body

% subsubsection format
\titleclass{\subsubsection}{straight}
\titleformat{\subsubsection}%
  {\normalfont\large\itshape\color{yellow}}% format applied to label+text
  {\llap{\colorbox{yellow}{\parbox{3cm}{\hfill\textcolor{background}{\thesubsubsection}}}}}% label
  {1em}% horizontal separation between label and title body
  {}% before the title body
  []% after the title body

% Sidenote enhancements
\def\mathmarginnote#1{%
  \tag*{\rlap{\hspace\marginparsep\smash{\parbox[t]{\marginparwidth}{%
  \footnotesize#1}}}}
}

% Custom table columning
\newcolumntype{L}[1]{>{\raggedright\let\newline\\\arraybackslash\hspace{0pt}}m{#1}}
\newcolumntype{C}[1]{>{\centering\let\newline\\\arraybackslash\hspace{0pt}}m{#1}}
\newcolumntype{R}[1]{>{\raggedleft\let\newline\\\arraybackslash\hspace{0pt}}m{#1}}

% Graph styles
\pgfplotsset{compat=1.15}
\usepgfplotslibrary{fillbetween}
\pgfplotsset{four quads/.append style={axis x line=middle, axis y line=
middle, xlabel={$x$}, ylabel={$y$}, axis equal }}
\pgfplotsset{four quad complex/.append style={axis x line=middle, axis y line=
middle, xlabel={$\re$}, ylabel={$\im$}, axis equal }}
\def\axisdefaultwidth{360pt}
\pgfplotsset{
  tufteaxis/.append style = {thick},tick style = {thick,black},
  %
  % #1 = x, y, or z
  % #2 = the shift value
  /tikz/normal shift/.code 2 args = {%
    \pgftransformshift{%
        \pgfpointscale{#2}{\pgfplotspointouternormalvectorofticklabelaxis{#1}}%
    }%
  },%
  %
  range3frame/.style = {
    tick align        = outside,
    scaled ticks      = false,
    enlargelimits     = false,
    ticklabel shift   = {10pt},
    axis lines*       = left,
    line cap          = round,
    clip              = false,
    xtick style       = {normal shift={x}{10pt}},
    ytick style       = {normal shift={y}{10pt}},
    ztick style       = {normal shift={z}{10pt}},
    x axis line style = {normal shift={x}{10pt}},
    y axis line style = {normal shift={y}{10pt}},
    z axis line style = {normal shift={z}{10pt}},
  }
}

% Shortcuts
\DeclareMathOperator{\id}{id}
\DeclareMathOperator{\Img}{Img}
\DeclareMathOperator{\Res}{Res}
\DeclareMathOperator*{\argmax}{arg\,max}
\DeclareMathOperator*{\argmin}{arg\,min}
\DeclareMathOperator{\re}{Re}
\DeclareMathOperator{\im}{Im}
\DeclareMathOperator{\caparg}{Arg}
\DeclareMathOperator{\Char}{Char}
\DeclareMathOperator{\sgn}{sgn}
\DeclareMathOperator{\Range}{range}

\newcommand{\floor}[1]{\lfloor #1 \rfloor}      % simplifying the writing of a floor function
\newcommand{\ceiling}[1]{\lceil #1 \rceil}      % simplifying the writing of a ceiling function
\newcommand{\dotp}{\, \cdotp}			        % dot product to distinguish from \cdot
\newcommand{\abs}[1]{\left|#1\right|}						% absolute value
\newcommand{\lra}[1]{\left\langle \; #1 \; \right\rangle}
\newcommand{\at}[2]{\Big|_{#1}^{#2}}
\newcommand{\Arg}[1]{\caparg #1}
\renewcommand{\bar}[1]{\mkern 1.5mu \overline{\mkern -1.5mu #1 \mkern -1.5mu} \mkern 1.5mu}
\newcommand{\faktor}[2]{{\raisebox{.2em}{$#1$}\left/\raisebox{-.2em}{$#2$}\right.}}
\newcommand{\quotient}[2]{\faktor{#1}{#2}}
\newcommand{\cyclic}[1]{\left\langle #1 \right\rangle}
\newcommand{\ind}[2]{\Ind_{#2}\left( #1 \right)}
\newcommand{\notimply}{\centernot\implies}
\newcommand{\res}[2]{\underset{#2}{\Res} #1 }
\newcommand{\tworow}[3]{\begin{tabular}{@{}#1@{}} #2 \\ #3 \end{tabular}}
\renewcommand{\epsilon}{\varepsilon}
\renewcommand{\phi}{\varphi}
\newcommand{\lrarrow}{\leftrightarrow}
\newcommand{\larrow}{\leftarrow}
\newcommand{\rarrow}{\rightarrow}
\renewcommand{\atop}[2]{\genfrac{}{}{0pt}{}{#1}{#2}}
\newcommand*\dif{\mathop{}\!d}
\newcommand{\mmid}{\; \middle| \;}
\newcommand{\coprime}{\; \bot \;}
\newcommand{\norm}[1]{\left\| #1 \right\|}
\newenvironment{spmatrix}
  {\left(\begin{smallmatrix}}
  {\end{smallmatrix}\right)}

  % inspiration from: https://tex.stackexchange.com/questions/8720/overbrace-underbrace-but-with-an-arrow-instead#37758
\newcommand{\overarrow}[2]{
  \overset{\makebox[0pt]{\begin{tabular}{@{}c@{}}#2\\[0pt]\ensuremath{\uparrow}\end{tabular}}}{\ensuremath{#1}}
}
\newcommand{\underarrow}[2]{
  \underset{\makebox[0pt]{\begin{tabular}{@{}c@{}}\downarrow\\[0pt]\ensuremath{#2}\end{tabular}}}{\ensuremath{#1}}
}


	% highlighting shortcuts
\newcommand{\hlimpo}[1]{\textcolor{red}{\textbf{#1}}}
\newcommand{\hlwarn}[1]{\textcolor{yellow}{\textbf{#1}}}
\newcommand{\hldefn}[1]{\textcolor{blue}{\index{#1}\textbf{#1}}}
\newcommand{\hlnotea}[1]{\textcolor{green}{\textbf{#1}}}
\newcommand{\hlnoteb}[1]{\textcolor{cyan}{\textbf{#1}}}
\newcommand{\hlb}[2]{\colorbox{#1!30!background}{#2}}
\newcommand{\hlbnotea}[1]{\hlb{green}{#1}}
\newcommand{\hlbnoteb}[1]{\hlb{cyan}{#1}}
\newcommand{\hlbnotec}[1]{\hlb{yellow}{#1}}
\newcommand{\hlbnoted}[1]{\hlb{magenta}{#1}}
\newcommand{\hlbnotee}[1]{\hlb{red}{#1}}
\newcommand{\WTP}{\textcolor{bwhite}{WTP} }
\newcommand{\WTS}{\textcolor{bwhite}{WTS} }

  % stars on important stuff
\newcommand{\imponote}{\faStar}
\newcommand{\vimponote}{\faStar\faStar}
\newcommand{\vvimponote}{\faStar\faStar\faStar}

% Document header formatting
\makeatletter
\pagestyle{fancy}
\fancyhead{}
\fancyhead[RO]{\textsl{\@title} \enspace \thepage}
\fancyhead[LE]{\thepage \enspace \textsl{\leftmark \enspace \rightmark}}
\makeatother
\renewcommand{\chaptermark}[1]{\markboth{#1}{}}
\renewcommand{\sectionmark}[1]{\markright{#1}}

% Comment the two lines below if you want to print the document
\pagecolor{background}
\color{foreground}


\begin{document}
\hypersetup{pageanchor=false}
\maketitle
\hypersetup{pageanchor=true}
\tableofcontents

\chapter*{\faBook \enspace List of Definitions}
\addcontentsline{toc}{chapter}{List of Definitions}
\theoremlisttype{all}
\listtheorems{defn}

\chapter*{\faCoffee \enspace List of Theorems}
\addcontentsline{toc}{chapter}{List of Theorems}
\theoremlisttype{allname}
\listtheorems{axiom,lemma,thm,crly,propo}

\chapter{Lecture 1 Sep 06th}%
\label{chp:lecture_1_sep_06th}
% chapter lecture_1_sep_06th

\section{Course Logistics}%
\label{sec:course_logistics}
% section course_logistics

No content is covered in today's lecture so this chapter will cover some of the important logistical highlights that were mentioned in class.

\begin{itemize}
  \item Assignments are designed to help students understand the content.
  \item Due to shortage of manpower, not all assignment questions will be graded; however, students are encouraged to attempt all of the questions.
  \item To further motivate students to work on ungraded questions, the midterm and final exam will likely recycle some of the assignment questions.
  \item There are no required text, but the professor has prepared course notes for reading. The course note are self-contained.
  \item The approach of the class will be more interactive than most math courses.
  \item Due to the size of the class, students are encouraged to utilize Waterloo Learn for questions, so that similar questions by multiple students can be addressed at the same time.
\end{itemize}

% section course_logistics (end)

\section{Preview into the Introduction}%
\label{sec:preview_into_the_introduction}
% section preview_into_the_introduction

How do we compare the size of two sets?

\begin{itemize}
  \item If the sets are finite, this is a relatively easy task.
  \item If the sets are infinite, we will have to rely on functions.
    \begin{itemize}
      \item Injective functions tell us that the \hlnoteb{domain is of size that is lesser than or equal to the codomain}.
      \item Surjective functions tell us that the \hlnoteb{codomain is of size that is lesser than or equal to the domain}.
      \item So does a bijective function tell us that the domain and codomain have the same size? Yes, although this is not as intuitive as it looks, as it relies on \hlnotea{Cantor-Schr\"oder-Bernstein Theorem}.
    \end{itemize}
\end{itemize}

Now, given two arbitrary sets, are we guaranteed to always be able to compare their sizes? It would be very tempting to immediately say yes, but to do that, one would have to agree on the \hlnotea{Axiom of Choice}. Fortunately, within the realm of this course, the Axiom of Choice is taken for granted.

% section preview_into_the_introduction (end)

% chapter lecture_1_sep_06th (end)

\chapter{Lecture 2 Sep 10th}%
\label{chp:lecture_2_sep_10th}
% chapter lecture_2_sep_10th

\section{Basic Set Theory}%
\label{sec:basic_set_theory}
% section basic_set_theory

We shall use the following notations for some of the common set of numbers that we are already familiar with:
\begin{itemize}
  \item $\mathbb{N}$ denotes the set of natural numbers $\{1, 2, 3, ...\}$;
  \item $\mathbb{Z}$ denotes the set of integers $\{..., -2, -1, 0, 1, 2, ...\}$;
  \item $\mathbb{Q}$ denotes the set of rational numbers $\left\{ \frac{a}{b} \mid a \in \mathbb{Z}, b \in \mathbb{N} \right\}$; and
  \item $\mathbb{R}$ denotes the set of real numbers.
\end{itemize}

We shall start with having certain basic properties of $\mathbb{N}$, $\mathbb{Z}$, and $\mathbb{Q}$.

\newthought{We will use} the notation $A \subset B$ and $A \subseteq B$ interchangably to mean that $A$ is a subset of $B$ with the possibility that $A = B$. When we wish to explicitly emphasize this possibility, we shall use $A \subseteq B$. When we wish to explicitly state that $A$ is a \hlnotea{proper subset} of $B$, we will either specify that $A \neq B$ or simply $A \subsetneq B$.

\begin{defn}[Universal Set]\index{Universal Set}
\label{defn:universal_set}\marginnote{This is a hand-wavy definition, but it is not in the interest of this course to further explore on this topic.}
  A universal set, which we shall generally give the label $X$, is a set that contains all the mathematical objects that we are interested in.
\end{defn}

With a universal set in place, we can have the following definitions:

\begin{defn}[Union]\index{Union}
\label{defn:union}
  Let $X$ be a set. If $\{A_\alpha\}_{\alpha \in I}$ such that $A_{\alpha} \subset X$, then the \hlnoteb{union} for all $A_{\alpha}$ is defined as
  \begin{equation*}
    \bigcup_{\alpha \in I} A_{\alpha} := \{ x \in X \mid \exists \alpha \in I, x \in A_{\alpha} \}.
  \end{equation*}
\end{defn}

\begin{defn}[Intersection]\index{Intersection}
\label{defn:intersection}
  Let $X$ be a set. If $\{A_\alpha\}_{\alpha \in I}$ such that $A_\alpha \subset X$, then the \hlnoteb{intersection} for all $A_\alpha$ is defined as
  \begin{equation*}
    \bigcap_{\alpha \in I} A_\alpha := \{ x \in X \mid \forall \alpha \in I, x \in A_\alpha \}.
  \end{equation*}
\end{defn}

\begin{defn}[Set Difference]\index{Set Difference}
\label{defn:set_difference}
  Let $X$ be a set and $A, B \subseteq X$. The \hlnoteb{set difference} of $A$ from $B$ is defined as
  \begin{equation*}
    A \setminus B := \{ x \in X \mid x \in A, x \notin B \}.
  \end{equation*}
\end{defn}

On a similar notion:

\begin{defn}[Symmetric Difference]\index{Symmetric Difference}
\label{defn:symmetric_difference}
Let $X$ be a set and $A, B \subseteq X$. The \hlnoteb{symmetric difference} of $A$ and $B$ is defined as\marginnote{In words, for an element in the symmetric difference of two sets, the element is either in $A$ or $B$ but not both. We can also think of the symmetric difference as
\begin{equation*}
  ( A \cup B ) \setminus (A \cap B)
\end{equation*}
or
\begin{equation*}
  ( A \setminus B ) \cup (B \setminus A).
\end{equation*}
}
  \begin{equation*}
    A \Delta B := \{ x \in X \mid ( x \in A \land x \notin B ) \lor ( x \notin A \land x \in B ) \}.
  \end{equation*}
\end{defn}

We can also talk about the non-members of a set:

\begin{defn}[Set Complement]\index{Set Complement}
\label{defn:set_complement}
  Let $X$ be a set and $A \subset X$. The set of all non-members of $A$ is called the \hlnoteb{complement} of $A$, which we denote as
  \begin{equation*}
    A^c := \{ x \in X \mid x \notin A \}.
  \end{equation*}
\end{defn}

\begin{note}
  Note that
  \begin{equation*}
    \left( A^c \right)^c = \{ x \in X \mid x \notin A^c \} = \{ x \in X \mid x \in A \} = A.
  \end{equation*}
\end{note}

Now taking a step away from that, we define the following:

\begin{defn}[Empty Set]\index{Empty Set}
\label{defn:empty_set}
  An \hlnoteb{empty set}, denoted by $\emptyset$, is a set that contains nothing.
\end{defn}

\begin{note}
  The empty set is set to be a subset of all sets.
\end{note}

\begin{defn}[Power Set]\index{Power Set}
\label{defn:power_set}
  Let $X$ be a set. The power set of $X$ is the set that contains all subsets of $X$, i.e.
  \begin{equation*}
    \mathcal{P}(X) := \{ A \mid A \subset X \}.
  \end{equation*}
\end{defn}

\begin{note}
  A power set is always non-empty, since $\emptyset \in \mathcal{P}(\emptyset)$, and since $\emptyset \subset X$ for any set $X$, we have $\emptyset \in \mathcal{P}(X)$.
\end{note}

\begin{eg}
  Let $X = \{1, 2, ..., n\}$. There are several ways we can show that the size of $\mathcal{P}(X)$ is $2^n$. One of the methods is by using a characteristic function that maps from $A$ to $\{0, 1\}$, defined by
  \begin{gather*}
    X_A: A \to \{0, 1\} \\
    X_A(x) = \begin{cases}
      1 & x \in A \\
      0 & x \notin A
    \end{cases}.
  \end{gather*}
  Using this function, each element in $X$ have 2 states: one being in the subset, and the other being not in the subset, which are represented by $1$ and $0$ respectively. It is then clear that there are $2^n$ of such configurations.
\end{eg}

\begin{thm}[De Morgan's Laws]
\index{De Morgan's Laws}
\label{thm:de_morgan_s_laws}
  Let $X$ be a set. Given $\{A_\alpha\}_{\alpha \in I} \subset \mathcal{P}(X)$, we have
  \begin{enumerate}
    \item $\left( \bigcup\limits_{\alpha \in I} A_\alpha \right)^c = \bigcap\limits_{\alpha \in I} A_\alpha^c$; and
    \item $\left( \bigcap\limits_{\alpha \in I} A_\alpha \right)^c = \bigcup\limits_{\alpha \in I} A_\alpha^c$.
  \end{enumerate}
\end{thm}

\begin{proof}
  \begin{enumerate}
    \item Note that
      \begin{align*}
        x \in \left( \bigcup_{\alpha \in I} A_\alpha \right)^c &\iff \nexists \alpha \in I \enspace x \in A_\alpha \\
        &\iff \forall \alpha \in I \enspace x \notin A_\alpha \\
        &\iff \forall \alpha \in I \enspace x \in A_\alpha^c \text{ by set complementation } \\
        &\iff x \in \bigcap_{\alpha \in I} A_\alpha^c.
      \end{align*}

    \item Observe that, by part 1,
      \begin{align*}
        \left( \bigcap_{\alpha \in I} A_\alpha \right)^c = \left( \left( \bigcup_{\alpha \in I} A_{\alpha}^c \right)^c \right)^c = \bigcup_{\alpha \in I} A_\alpha^c.
      \end{align*}
  \end{enumerate}\qed
\end{proof}

\begin{eg}
  Suppose $I = \emptyset$. Then what is $\bigcup\limits_{\alpha \in \emptyset} A_\alpha$? It is sensible to think that all we are left with is simply a union of empty sets, and so
  \begin{equation}\label{eq:union_of_empty_sets}
    \bigcup_{\alpha \in \emptyset} A_\alpha = \emptyset.
  \end{equation}
  And what about $\bigcap\limits_{\alpha \in \emptyset} A_\alpha$? By \cref{thm:de_morgan_s_laws}, it is quite clear from \cref{eq:union_of_empty_sets} that
  \begin{equation*}
    \bigcap_{\alpha \in \emptyset} A_\alpha = X.
  \end{equation*}
\end{eg}

% section basic_set_theory (end)

\section{Products of Sets}%
\label{sec:products_of_sets}
% section products_of_sets

\begin{defn}[Product of Sets]\index{Product of Sets}
\label{defn:product_of_sets}
  Given $2$ sets $X$ and $Y$, the \hlnoteb{product} of $X$ and $Y$ is given by
  \begin{equation*}
    X \times Y := \{ (x, y) \mid x \in X , y \in Y \}.
  \end{equation*}
  We often refer to elements of $X \times Y$ as \hldefn{tuples}.
\end{defn}

\begin{note}
  Now if
  \begin{align*}
    X &= \{ x_1, x_2, ..., x_n \}, \\
    Y &= \{ y_1, y_2, ..., y_m \},
  \end{align*}
  then
  \begin{equation*}
    X \times Y = \{ (x_i, y_j) \mid i = 1, 2, ..., n, \; j = 1, 2, ..., m \}
  \end{equation*}
  and so the size of $X \times Y$ is $mn$.
\end{note}

Consequently, we can think of tuples as two elements being in some ``relation''.

\begin{defn}[Relation]\index{Relation}
\label{defn:relation}
  A \hlnoteb{relation} on sets $X$ and $Y$ is a subset $R$ of the product $X \times Y$. We write
  \begin{equation*}
    xRy \enspace \text{ if } (x, y) \in R \subset X \times Y.
  \end{equation*}
  We call
  \begin{itemize}
    \item $\{ x \in X \mid \exists y \in Y, (x, y) \in R \}$ as the \hldefn{domain} of $R$; and 
    \item $\{ y \in Y \mid \exists x \in X, (x, y) \in R \}$ as the \hldefn{range} of $R$.
  \end{itemize}
\end{defn}

In relation to that, functions are, essentially, relations.

\begin{defn}[Function]\index{Function}
\label{defn:function}
  A \hlnoteb{function} from $X$ to $Y$ is a relation $R$ such that
  \begin{equation*}
    \forall x \in X \; \exists! y \in Y \; (x, y) \in R.
  \end{equation*}
\end{defn}

\newthought{Suppose} $X_1, X_2, ..., X_n$ are non-empty\sidenote{We are typically only interested in non-empty sets, since empty sets usually lead us to vacuous truths, which are not interesting.} sets. We can define
\begin{equation*}
  X_1 \times X_2 \times \hdots \times X_n = \prod_{i=1}^{n} X_i := \{ (x_1, x_2, ..., x_n) \mid x_i \in X_i \}.
\end{equation*}
Now if $X_i = X_j = X$ for all $i, j = 1, 2, ..., n$, we write
\begin{equation*}
  \prod_{i=1}^{n} X_i = \prod_{i=1}^{n} X = X^n.
\end{equation*}

\newthought{And now comes the problem}: given a collection $\{X_\alpha\}_{\alpha \in I}$ of non-empty sets\sidenote{i.e. we now talk about arbitrary $\alpha \in I$.}, what do we mean by
\begin{equation*}
  \prod_{\alpha \in I} X_\alpha ?
\end{equation*}

To motivate for what comes next, consider
\begin{equation*}
  \prod_{i=1}^{n} X_i = X_1 \times \hdots \times X_n = \{ (x_1, ..., x_n) \mid x_i \in X_i \}.
\end{equation*}
Choose $(x_1, ..., x_n) \in \prod\limits_{i=1}^{n} X_i$. This induces a function
\begin{equation*}
  f_{(x_1, ..., x_n)} : \{1, ..., n\} \to \bigcup_{i=1}^{n} X_i
\end{equation*}
with
\begin{align*}
  f(1) &= x_1 \in X_1 \\
  f(2) &= x_2 \in X_2 \\
       &\vdots
  f(n) &= x_n \in X_n
\end{align*}

Now assume for a more general $f$ such that
\begin{equation*}
  f: \{1, ..., n\} \to \bigcup_{i=1}^{n} X_i
\end{equation*}
is defined by
\begin{equation*}
  f(i) \in X_i.
\end{equation*}
Then, we have
\begin{equation*}
  ( f(1), f(2), ..., f(n) ) \in \prod_{i=1}^{n} X_i,
\end{equation*}
which leads us to the following notion:

\begin{defn}[Choice Function]\index{Choice Function}
\label{defn:choice_function}
  Given a collection $\{X_\alpha\}_{\alpha \in I}$ of non-empty sets, let
  \begin{equation*}
    \prod_{\alpha \in I} X_\alpha = \left\{ f: I \to \bigcup_{\alpha \in I} X_\alpha \right\}
  \end{equation*}
  such that $f(\alpha) \in X_\alpha$. Such an $f$ is called a \hlnoteb{choice function}.
\end{defn}

And so we may ask a similar question as before: if each $X_\alpha$ is non-empty, is $\prod\limits_{\alpha \in I} X_\alpha$ non-empty? Turns out this is not as easy to show. In fact, it is essentially impossible to show, because this is exactly the \hlnotea{Axiom of Choice}.

% section products_of_sets (end)

% chapter lecture_2_sep_10th (end)

\appendix

\backmatter

\pagestyle{plain}

\nobibliography*
\bibliography{references}

\printindex

\end{document}

